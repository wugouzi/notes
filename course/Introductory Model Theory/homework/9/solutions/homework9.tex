% Created 2021-12-01 Wed 19:17
% Intended LaTeX compiler: pdflatex
\documentclass[11pt]{article}
\usepackage[utf8]{inputenc}
\usepackage[T1]{fontenc}
\usepackage{graphicx}
\usepackage{longtable}
\usepackage{wrapfig}
\usepackage{rotating}
\usepackage[normalem]{ulem}
\usepackage{amsmath}
\usepackage{amssymb}
\usepackage{capt-of}
\usepackage{hyperref}
\input{../../../../preamble-lite.tex}
\author{Qi'ao Chen\\21210160025}
\date{\today}
\title{Homework9}
\hypersetup{
 pdfauthor={Qi'ao Chen\\21210160025},
 pdftitle={Homework9},
 pdfkeywords={},
 pdfsubject={},
 pdfcreator={Emacs 27.2 (Org mode 9.6)}, 
 pdflang={English}}
\begin{document}

\maketitle
\begin{exercise}
Show that the following is not a field
\end{exercise}

\begin{proof}
As \(a=1+1\), \(b=a+1=1+1+1\), we have \(a\cdot a=(1+1)\cdot(1+1)=1+1+1+1=0=b\). Then \(a\cdot b=1=0\). Hence
this is not a field.
\end{proof}

\begin{proof}
Let \(\varphi(x)\) be the formula \(\exists y(x\cdot y+y=1)\). Find a quantifier-free formula \(\psi(x)\) equivalent
to \(\varphi(x)\) in all algebraically closed fields
\end{proof}

\begin{proof}
\(x+1\neq 0\)
\end{proof}

\begin{exercise}
Let \(\varphi(x,y,z)\) be the formula \(\exists w(x\cdot w^2+y\cdot w+z=0)\). Find a quantifier-free formula \(\psi(x,y,z)\)
equivalent to \(\varphi(x,y,z)\) in all algebraically closed fields
\end{exercise}

\begin{proof}
\(x\neq 0\)
\end{proof}

\begin{exercise}
If \(M\) is a structure and \(\varphi(x)\) is a formula in one variable, then \(\varphi(M)\) denotes the
set \(\{a\in M:M\vDash\varphi(a)\}\). Show that if \(M\preceq N\) and \(\varphi(M)\) is finite, then \(\varphi(M)=\varphi(N)\)
\end{exercise}

\begin{proof}
Suppose \(\abs{\varphi(M)}=n\), then let \(\psi_n\) be
\begin{equation*}
\exists x_1\dots x_n(\bigwedge_{\substack{i\neq j\\1\le i\le n\\ 1\le j\le n}}x_i\neq x_j\wedge\bigwedge_{i=1}^n\varphi(x_i))
\end{equation*}
and let \(\psi:=\psi_n\wedge\neg\psi_{n+1}\). Apparently \(\neg\psi_{n+1}\vDash\neg\psi_{n+m}\) for all \(m\ge 1\). Thus \(\psi\) states
that there is exactly \(n\) solutions for \(\varphi(x)\) and we have \(M\vDash\psi\). As \(M\preceq N\), we have \(N\vDash\psi\)
and \(N\) has exactly \(n\) solutions for \(\varphi(x)\). But for any \(m\in M\), \(M\vDash\varphi(m)\Leftrightarrow N\vDash\varphi(m)\). Hence \(\varphi(M)=\varphi(N)\)
\end{proof}

\begin{exercise}
Let \(T\) be a theory with quantifier elimination. Let \(M\) be a structure and \(N\) be an
extension. Suppose that \(M\) and \(N\) are both models of \(T\). Let \(\varphi(\barx)\) be a
quantifier-free \(L(M)\)-formula in several variables. Suppose that \(N\vDash\exists\barx\varphi(\barx)\). Show that \(M\vDash\exists\barx\varphi(\barx)\)
\end{exercise}

\begin{proof}
Given any formula \(\psi(x,\bara)\) where \(\bara\in M^n\) and let \(\chi(\bary):=\exists\barx\;\psi(\barx,\bary)\),
which is equivalent to a quantifier-free formula \(\theta(\bary)\). As \(N\vDash\chi(\bara)\) we
have \(N\vDash\theta(\bara)\). As \(M\) is a submodel of \(N\), \(N\vDash\theta(\bara)\Leftrightarrow M\vDash\theta(\bara)\). Hence we
have \(M\vDash\exists\barx\;\psi(\barx,\bary)\), that is \(M\vDash\exists\barx\varphi(\barx)\)
\end{proof}

\begin{exercise}
Let \(K\) be an algebraically closed field. Let \(L\supseteq K\) be an extension field.
Let \(P(x,y,z)\), \(Q(x,y,z)\) and \(R(x,y,z)\) be polynomials over \(K\). Suppose that the system
of equations
\begin{align*}
&P(x,y,z)=0\\
&Q(x,y,z)=0\\
&R(x,y,z)=0
\end{align*}
has a solution in \(L\). Show that it has a solution in \(K\)
\end{exercise}

\begin{proof}
By the fact, there is a model \(M\supseteq L\supseteq K\) s.t. \(M\) and \(K\) are both algebraically closed field.
By Theorem 36, \(K\preceq M\). Let \(\psi(x,y,z)\), an \(\call_K\)-formula, be
\begin{equation*}
P(x,y,z)=0\wedge Q(x,y,z)=0\wedge R(x,y,z)=0
\end{equation*}
As \(L\vDash\exists xyz\;\psi(x,y,z)\), \(M\vDash\exists xyz\;\psi(x,y,z)\) and hence \(K\vDash\exists xyz\;\psi(x,y,z)\)
\end{proof}
\end{document}
