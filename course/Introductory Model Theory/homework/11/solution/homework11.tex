% Created 2021-12-13 Mon 21:47
% Intended LaTeX compiler: pdflatex
\documentclass[11pt]{article}
\usepackage[utf8]{inputenc}
\usepackage[T1]{fontenc}
\usepackage{graphicx}
\usepackage{longtable}
\usepackage{wrapfig}
\usepackage{rotating}
\usepackage[normalem]{ulem}
\usepackage{amsmath}
\usepackage{amssymb}
\usepackage{capt-of}
\usepackage{hyperref}
\input{../../../../preamble-lite.tex}
\author{Qi'ao Chen\\21210160025}
\date{\today}
\title{Homework11}
\hypersetup{
 pdfauthor={Qi'ao Chen\\21210160025},
 pdftitle={Homework11},
 pdfkeywords={},
 pdfsubject={},
 pdfcreator={Emacs 27.2 (Org mode 9.6)}, 
 pdflang={English}}
\begin{document}

\maketitle
\begin{exercise}
Show that the field \(\C\) is strongly \(\abs{\C}\)-homogeneous
\end{exercise}

\begin{proof}
For any partial elementary map \(f:A\to B\) where \(A,B\subseteq\C\) and \(\abs{A}<\fc\), we can enumerate \(\C\)
as \((c_\alpha:\alpha<\fc)\). We build a sequence of \((f_\alpha:\alpha<\fc)\)
\begin{equation*}
f=f_0\subset f_1\subset f_2\subset\cdots
\end{equation*}
s.t. \(\abs{\dom(f_\alpha)}<\fc\) and each \(f_\alpha\) is partial elementary for each \(\alpha<\fc\).
If \(\alpha=\beta\cdot\omega+2n+1\) where \(n\in\omega\), since \(\dom(f_{\beta+2n})<\fc\), there is \(b\in\C\)
s.t. \(f_{\beta+2n}\cup\{(c_{\beta+n},b)\}\) is partial elementary. Let \(f_\alpha=f_{\beta+2n}\cup\{(c_{\beta+1},b)\}\).
If \(\alpha=\beta\cdot\omega+2n+2\), there is \(b'\in\C\) s.t. \(f_{\beta+2n+1}^{-1}\cup\{(b',c_{\beta+1})\}\) is partial elementary.
Let \(f_\alpha=f_{\beta+2n+1}\cup\{(c_{\beta+1},b')\}\). If \(\alpha\) is a limit ordinal, let \(f_\alpha=\bigcup_{\beta<\alpha}f_\beta\).
Then \(\abs{\dom(f_\alpha)}\le\abs{A}+\abs{\alpha}<\fc\).

Let \(g=\bigcup_{\alpha<\fc}f_\alpha\). Then \(g\) is partial elementary with \(\dom(g)=\im(g)=\fc\). Thus \(g\in\Aut(\C)\)
\end{proof}

\begin{exercise}
Show that the field \(\R\) is strongly \(\kappa\)-homogeneous for any cardinal \(\kappa\)
\end{exercise}

\begin{proof}
Suppose a partial elementary map \(f:A\to B\) where \(A,B\subseteq\R\). Since we are working in a
field, \(\R=(\R,+,\cdot,0,1)\).

Given \(a\in A\), first we show that \(f(a)=a\).

First note that for any integers \(n\in\N\),
\begin{equation*}
n:=\underbrace{1+\cdots+1}_{n\text{ times}}
\end{equation*}
Then integers are definable via 1 since \(\R\) is an abelian group and has a unique additive inverse.
Then since \(\R\) has unique multiplicative inverse, each \(q\in\Q\) is expressble via 1.

Also, we can define \(a\le b\) by \(\exists z(a+z\cdot z=b)\). Thus for all \(q\in\Q\), if \(q<a\),
then \(q<x\in\tp(a)\) and if \(q>a\), then \(q>x\in\tp(a)\). Thus \(f(a)=a\) as \(\tp(a)=\tp(f(a))\).

Thus we can extend \(f\) to the identity function of \(\R\).
\end{proof}

\begin{exercise}
Let \(S=\{0,1\}\times\Z\) and let \(\le\) be the lexicographic order on \(S\):
\begin{gather*}
(0,x)<(1,y)\\
(0,x)\le(0,y)\Leftrightarrow x\le y\\
(1,x)\le(1,y)\Leftrightarrow x\le y
\end{gather*}
Show that \((S,\le)\) is not strongly \(\omega\)-homogeneous, but some expansion of \((S,\le)\) is strongly \(\omega\)-homogeneous
\end{exercise}

\begin{proof}
Consider the map \(f=\{((0,0),(1,0))\}\). \(f\) is a partial elementary if and only
if \((S,(0,0))\equiv(S,(1,0))\), which is true by Theorem 1.8 on book. If there is \(f\subset\sigma\in\Aut(S)\), then
\(d((0,0),(1,0))=\infty\) but \(d(\sigma(0,0),\sigma(1,0))\) is finite, a contradiction. Thus \((S,\le)\) is not
strongly \(\omega\)-homogeneous.

We claim that for any \(a,b\in\Z\), \((S,\le,(0,a),(1,b))\) is strongly \(\omega\)-homogeneous. Note that identity
function is the only automorphism since automorphism needs to respect the distance function
and \(\le\) relation, which will uniquely determine an element given \((0,a)\) and \((1,b)\). Thus any
finite elementary map is a subset of the identity function and we can extend it.
\end{proof}

\(T=\Th(\R,+,\cdot,0)\)

\begin{exercise}
Suppose \(M\vDash T\). Show that there is at most one linear order \(\le\) on \(M\) s.t. the following hold
\begin{itemize}
\item If \(x\le y\), then \(x+z\le y+z\)
\item If \(x\le y\) and \(0\le z\), then \(xz\le yz\)
\end{itemize}
\end{exercise}

\begin{proof}
Suppose there is a linear order \(R\) on \(M\) s.t.
\begin{itemize}
\item \(\forall z(xRy\to(x+z)R(y+z))\)
\item \(xRy\wedge 0Rz\to xzRyz\)
\end{itemize}

Fist note that \(0Ra\leftrightarrow(-a)R0\). If \(0Ra\), then \((-a)R0\) and so \((-a^2)R0\). If \(aR0\),
then \(0R(-a)\) and so \((-a^2)R0\). Therefore for all \(a\), \((-a^2)R0\) and so \(0Ra^2\). Hence
for all \(a\), \(0\le a\Leftrightarrow a=b^2\Leftrightarrow 0Ra\). Then \(a\le b\Leftrightarrow (a-b)\le 0\Leftrightarrow(a-b)R0\Leftrightarrow aRb\).
\end{proof}

\begin{proof}
Write down an explicit definition of \(\le\) in \((\R,+,\cdot)\)
\end{proof}

\begin{proof}
Let \(\phi(x,y):=\exists z(x+z\cdot z=y)\).
\begin{itemize}
\item If \(\phi(x,y)\), then let \(z\in\R\) s.t. \(x+z\cdot z=y\), for any \(c\in\R\), \(x+c+z\cdot z=y+c\),
hence \(\phi(x+c,y+c)\)
\item If \(\phi(x,y)\) and \(\phi(0,z)\), then there is \(a,b\in\R\) s.t. \(x+a\cdot a=y\) and \(0+b\cdot b=z\), thus
\(yz=(x+a\cdot a)z=xz+a\cdot a\cdot b\cdot b=xz+(a\cdot b)\cdot(a\cdot b)\), hence \(\phi(xz,yz)\)
\item If \(x\le y\), \(\phi(x,y)\) is clear
\end{itemize}
\end{proof}
\end{document}
