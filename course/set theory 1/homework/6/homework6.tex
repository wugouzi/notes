% Created 2021-11-07 Sun 18:54
% Intended LaTeX compiler: pdflatex
\documentclass[11pt]{article}
\usepackage[utf8]{inputenc}
\usepackage[T1]{fontenc}
\usepackage{graphicx}
\usepackage{longtable}
\usepackage{wrapfig}
\usepackage{rotating}
\usepackage[normalem]{ulem}
\usepackage{amsmath}
\usepackage{amssymb}
\usepackage{capt-of}
\usepackage{hyperref}
\input{../../../preamble-lite.tex}
\usepackage[UTF8]{ctex}
\author{陈淇奥\\21210160025}
\date{\today}
\title{Homework6}
\hypersetup{
 pdfauthor={陈淇奥\\21210160025},
 pdftitle={Homework6},
 pdfkeywords={},
 pdfsubject={},
 pdfcreator={Emacs 27.2 (Org mode 9.5)}, 
 pdflang={English}}
\begin{document}

\maketitle
\begin{exercise}[2.2.6]
证明以上定义的\(\lhd\)是\(\kappa\times\kappa\)上的良序
\end{exercise}

\begin{proof}
\begin{enumerate}
\item 传递性:若\((\alpha_1,\beta_1)\lhd(\alpha_2,\beta_2)\lhd(\alpha_3,\beta_3)\),则
\begin{align*}
 &\alpha_1+\beta_1<\alpha_2+\beta_2,\text{或}\\
 &\alpha_1+\beta_1=\alpha_2+\beta_2\wedge\alpha_1<\alpha_2,\text{或}\\
 &\alpha_1+\beta_1=\alpha_2+\beta_2\wedge\alpha_1=\alpha_2\wedge\beta_1<\beta_2
\end{align*}
且
 \begin{align*}
 &\alpha_2+\beta_2<\alpha_3+\beta_3,\text{或}\\
 &\alpha_2+\beta_2=\alpha_3+\beta_3\wedge\alpha_2<\alpha_3,\text{或}\\
 &\alpha_2+\beta_2=\alpha_3+\beta_3\wedge\alpha_2=\alpha_3\wedge\beta_2<\beta_3
\end{align*}
若\(\alpha_1+\beta_1<\alpha_2+\beta_2\),则\(\alpha_1+\beta_1<\alpha_2+\beta_2\le\alpha_3+\beta_3\),因此\((\alpha_1,\beta_1)\lhd(\alpha_3,\beta_3)\)

若\(\alpha_1+\beta_1=\alpha_2+\beta_2\wedge\alpha_1<\alpha_2\),则\(\alpha_1+\beta_1=\alpha_2+\beta_2<\alpha_3+\beta_3\)或\(\alpha_1+\beta_1=\alpha_3+\beta_3\wedge\alpha_1<\alpha_2\le\alpha_3\),因
此\((\alpha_1,\beta_1)\lhd(\alpha_3,\beta_3)\)

若\(\alpha_1+\beta_1=\alpha_2+\beta_2\wedge\alpha_1=\alpha_2\wedge\beta_1<\beta_2\),则\(\alpha_1+\beta_1<\alpha_3+\beta_3\)或\(\alpha_1+\beta_1=\alpha_3+\beta_3\wedge\alpha_1<\alpha_3\)
或\(\alpha_1+\beta_1=\alpha_3+\beta_3\wedge\alpha_1=\alpha_3\wedge\beta_1<\beta_2<\beta_3\),因此\((\alpha_1,\beta_1)\lhd(\alpha_3,\beta_3)\)

\item 非自反性:若\((\alpha_1,\beta_1)\lhd(\alpha_2,\beta_2)\)且\((\alpha_2,\beta_2)\lhd(\alpha_1,\beta_1)\),则\(\alpha_1+\beta_1><\alpha_2+\beta_2\)或\(\alpha_1><\alpha_2\)
或\(\beta_1><\beta_2\),这与\(<\)的非自反性矛盾

\item 对任意的\((\alpha_1,\beta_1)\)与\((\alpha_2,\beta_2)\),因为\(<\)有三岐性,于是\(\alpha_1\)与\(\alpha_2\),\(\beta_1\)与\(\beta_2\)只见有关系,
因此\((\alpha_1,\beta_1)\)与\((\alpha_2,\beta_2)\)之间有关系

\item 对任意\(\kappa\times\kappa\)的子集\(A\times B\)。因为\(A,B\)是序数的集合,有\(<\)的最小值,
令\(a=\min A\),\(b=\min B\),于是对于任意\(\alpha\in A\)与\(\beta\in B\),\(a+b<a+\beta<\alpha+\beta\),于是\((a,b)\)
是\(A\times B\)在\(\lhd\)的最小值。
\end{enumerate}
\end{proof}

\begin{exercise}[2.2.9]
令\(A,B\)为集合
\begin{enumerate}
\item \(\tensor[^A]{B}{}\)为集合
\item 当\(B=\emptyset\)时,\(\tensor[^A]{B}{}\)是什么?
\item 如果\(A=\emptyset\)呢?
\end{enumerate}
\end{exercise}

\begin{proof}
\begin{enumerate}
\item 因为\(A,B\)是集合,于是\(A\times B\)是集合,于是\(\tensor[^A]{B}{}=\{R\in A\times B\mid R\text{是函数}\}\)是集合
\item 是\(\emptyset\)
\item 是\(\{\emptyset\}\)
\end{enumerate}
\end{proof}

\begin{exercise}[2.2.16]
证明\(\abs{\R}=2^{\aleph_0}\)
\end{exercise}

\begin{proof}
\(\abs{\R}\ge 2^{\aleph_0}\): 对于每个0,1序列\(f\in 2^{\aleph_0}\),都可以唯一映射到康托集的一个元素,因
此\(2^{\aleph_0}\le\abs{\R}\)。

\(\abs{\R}\le 2^{\aleph_0}\):定义\(f:\R\to\calp(\Q)\)为\(f(r)=\{q\in\Q\mid q<r\}\)。如果\(r\neq r'\),不妨令\(r<r'\),则存在有
理数\(q\)使得\(r<q<r'\),因此\(f(r)\neq f(r')\),从而\(f\)是1-1的
\end{proof}

\begin{lemma}[]
对于集合\(A,C\),\(\abs{\tensor[^{\abs{A}}]{C}{}}=\abs{\tensor[^A]{C}{}}\)
\end{lemma}

\begin{proof}
给定双射\(f:\abs{A}\to A\),定义\(h:\tensor[^A]{C}{}\to\tensor[^{\abs{A}}]{C}{}\)为\(h(g)=g\circ f\)

若\(h(g_1)=h(g_2)\),则\(g_1\circ f=g_2\circ f\),因为\(f\)是双射,于是\(g_1=g_2\)

对于任意\(k\in\tensor[^{\abs{A}}]{C}{}\),则\(k\circ f^{-1}\in\tensor[^A]{C}{}\)且\(h(k\circ f^{-1})=k\)

因此\(h\)是双射
\end{proof}

\begin{lemma}[]
对于集合\(A,B\),\(\abs{A\times B}=\abs{\abs{A}\times\abs{B}}\)
\end{lemma}

\begin{proof}
给定双射\(f:\abs{A}\to A\),\(g:\abs{B}\to B\),定义映射\(h:\abs{A}\times\abs{B}\to A\times B\)
为\(h((a,b))=(f(a),g(b))\)。

若\(h((a,b))=h((a',b'))\),则\(f(a)=f(a')\)且\(g(b)=g(b')\),因此\((a,b)=(a',b')\)

同时对任意\((a,b)\in A\times B\),\(h(f^{-1}(a),g^{-1}(b))=(a,b)\),因此\(h\)是双射
\end{proof}

\begin{exercise}[2.2.12]
假设\(\kappa,\lambda\)是无穷基数,则
\begin{enumerate}
\item \(\kappa^{\lambda\oplus\mu}=\kappa^\lambda\oplus\kappa^\mu\)
\item \((\kappa^\lambda)^\mu=\kappa^{\lambda\otimes \mu}\)
\item \((\kappa\otimes\lambda)^\mu=\kappa^\mu\otimes\lambda^\mu\)
\item \(2^\kappa>\kappa\)
\end{enumerate}
\end{exercise}

\begin{proof}
令\(\abs{A}=\lambda,\abs{B}=\mu,\abs{C}=\kappa\)且\(A,B,C\)相互不交
\begin{enumerate}
\item 由引理
\begin{align*}
&\kappa^{\lambda\oplus\mu}=\abs{\tensor[^{\lambda\oplus\mu}]{C}{}}=\abs{\tensor[^{\abs{A\cup B}}]{C}{}}=\abs{\tensor[^{A\cup B}]{C}{}}\\
&\kappa^\lambda\otimes\kappa^\mu=\abs{\abs{\tensor[^{\abs{A}}]{C}{}}\times\abs{\tensor[^{\abs{B}}]{C}{}}}=
 \abs{\tensor[^A]{C}{}\times\tensor[^B]{C}{}}
\end{align*}
定义映射\(h:\tensor[^{A\cup B}]{C}{}\to\tensor[^A]{C}{}\times\tensor[^B]{C}{}\)为\(h(f)=(f|A,f|B)\)。
若\(h(f)=h(f')\),则\((f|A,f|B)=(f'|A,f'|B)\),因为\(A\cap B=\emptyset\),\(f=f'\)。

对任意\(f\in\tensor[^A]{C}{},g\in\tensor[^B]{C}{}\),则\(h(f\cup g)=(f,g)\)。因此\(h\)是双射

\item \((\kappa^\lambda)^\mu=\abs{\tensor[^B]{(^AC)}{}}\),\(\kappa^{\lambda\otimes\mu}=\abs{\tensor[^{A\times B}]{C}{}}\)

对于任意函数\(f:A\times B\to C\),定义\(g_b:A\to C\)为\(g_b(a)=f(a,b)\),定义\(h:B\to(A\to C)\)为\(h(b)=g_b\)。
定义映射\(k:f\to h\). 如果 \(k(f)=k(f')\),则对于任意\((a,b)\in A\times B\),
\(f(a,b)=k(f)(b)(a)=k(f')(b)(a)=f'(a,b)\),于是\(f=f'\)。

同时对于任意\(h\in\tensor[^B]{(^AC)}{}\),令\(f(a,b)=h(b)(a)\),则\(k(f)=h\)。因此\(k\)是双射

\item \((\kappa\otimes\lambda)^\mu=\abs{\tensor[^B]{(C\times A)}{}}\),\(\kappa^\mu\otimes\lambda^\mu=\abs{\tensor[^B]{C}{}\times\tensor[^B]{A}{}}\)

定义函数\(k:\tensor[^B]{C}{}\times\tensor[^B]{A}{}\to\tensor[^B]{(C\times A)}{}\)为\(k(f,g)=f\times g\),
对于任意\(x\in B\),\((f\times g)(x)=(f(x),g(x))\)。若\(k(f,g)=k(f',g')\),则对于任意\(x\in B\),
\((f(x),g(x))=(f'(x),g'(x))\),于是\(f=f'\),\(g=g'\),因此\((f,g)=(f',g')\)

对于任意\(g\in\tensor[^B]{(C\times A)}{}\),令\(\pi_1:C\times A\to C,\pi_2:C\times A\to A\)为对应的投影函数,则
\(k(\pi_1\circ g,\pi_2\circ g)=g\)。因此\(k\)是双射。

\item \(2^\kappa>\kappa\)等价于\(\abs{2^A}>A\)。对任意函数\(f:A\to\calp(A)\),令\(Y=\{x\in A\mid x\notin f(x)\}\),如果存在\(a\in A\)使
得\(f(a)=Y\),则\(a\in Y\)当且仅当\(a\notin f(a)\)当且仅当\(a\notin Y\)。因此\(f\)不是满射。因此\(2^\kappa\neq\kappa\)。同
时我们有单射\(x\mapsto\{x\}\),于是\(2^\kappa>\kappa\)
\end{enumerate}
\end{proof}

\begin{exercise}[2.3.2]
\begin{enumerate}
\item \(A\subset\alpha\)是无界的当且仅当\(\alpha=\bigcup\{\xi+1\mid\xi\in A\}\)。因此对任意序数,如果\(f:\cf(\alpha)\to\alpha\)是共尾映射,
则\(\bigcup_{\xi<\cf(\alpha)}[f(\xi)+1]=\alpha\)
\item 对任意\(\alpha\),\(\cf(\alpha)\le\alpha\)
\item 任意后继序数\(\alpha=\beta+1\)的共尾是1
\item 对任意极限序数\(\alpha>0\),\(\cf(\alpha)\ge\omega\)
\end{enumerate}
\end{exercise}

\begin{proof}
\begin{enumerate}
\item 若\(A\subset\alpha\)无界,因此对于任意\(\beta\in\alpha\),都存在\(\xi\in A\)使得\(\gamma\le\xi<\xi+1\),因此\(\alpha\subseteq\bigcup\{\xi+1\mid\xi\in A\}\)。同时对
任意\(\xi\in A\),因为\(A\subset\alpha\),因此\(\xi\subset\alpha\),因此\(\xi+1=\xi\cup\{\xi\}\subset\alpha\),所以\(\bigcup\{\xi+1\mid\xi\in A\}\subseteq\alpha\)。于是
\(\alpha=\bigcup\{\xi+1\mid\xi\in A\}\)

若\(\alpha=\bigcup\{\xi+1\mid\xi\in A\}\),则对于任意\(\beta<\alpha\),都存在\(\xi\in A\)使得\(\beta<\xi+1\),因此\(\beta\le\xi\),因此\(A\)无界

\item 因为恒等映射\(id:\alpha\to\alpha\)是一个共尾映射,因此\(\cf(\alpha)\le\alpha\)

\item 令映射\(f:1\to\alpha\)为\(f(0)=\beta\),而\(\beta\)在\(\alpha\)中无界,因此\(f\)是共尾映射。因为\(\cf(\alpha)\neq 0\),因此\(\cf(\alpha)=1\)

\item 因为\(A\subset\alpha\)无界当且仅当\(\alpha=\cup\{\xi+1\mid\xi\in A\}\)。
若\(\abs{A}\in\omega\),则\(\bigcup\{\xi+1\mid\xi\in A\}\)为后继序数,矛盾
\end{enumerate}
\end{proof}

\begin{exercise}
\label{ex2.3.10}
对任意序数\(\alpha,\beta\),\(\cf(\aleph_\alpha^{\aleph_\beta})>\aleph_\beta\)
\end{exercise}

\begin{proof}
若\(\cf(\aleph_\alpha^{\aleph_\beta})\le\aleph_\beta\),则
\begin{equation*}
\left( \aleph_\alpha^{\aleph_\beta} \right)^{\cf(\aleph_\alpha^{\aleph_\beta})}\le
\left( \aleph_\alpha^{\aleph_\beta} \right)^{\aleph_\beta}=\aleph_\alpha^{\aleph_\beta}
\end{equation*}
与定理矛盾
\end{proof}
\end{document}
