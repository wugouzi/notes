% Created 2021-11-29 Mon 23:49
% Intended LaTeX compiler: pdflatex
\documentclass[11pt]{article}
\usepackage[utf8]{inputenc}
\usepackage[T1]{fontenc}
\usepackage{graphicx}
\usepackage{longtable}
\usepackage{wrapfig}
\usepackage{rotating}
\usepackage[normalem]{ulem}
\usepackage{amsmath}
\usepackage{amssymb}
\usepackage{capt-of}
\usepackage{hyperref}
\input{../../../preamble-lite.tex}
\usepackage[UTF8]{ctex}
\author{陈淇奥\\21210160025}
\date{\today}
\title{Homework}
\hypersetup{
 pdfauthor={陈淇奥\\21210160025},
 pdftitle={Homework},
 pdfkeywords={},
 pdfsubject={},
 pdfcreator={Emacs 27.2 (Org mode 9.6)}, 
 pdflang={English}}
\begin{document}

\maketitle
\begin{exercise}[3.2.13]
如果\(F\)是由\(G\)生成的滤,则\(F\)是包含\(G\)的最小的滤,即\(G\subseteq F\)并且如果\(F'\supseteq G\)也是滤,
则\(F\subseteq F'\)
\end{exercise}

\begin{proof}
对任意\(x\in F\),存在\(g\in G\)使得\(g\le b\)。因为\(F'\)是包含\(G\)的滤,\(g\in F'\),于是对于任意\(g\le b\)
都有\(b\in F'\),因此\(x\in F'\)
\end{proof}

\begin{exercise}[3.2.16]
假设\(G\subseteq B\)为非空子集,\(F\)是由\(G\)生成的滤,则以下命题等价
\begin{enumerate}
\item \(G\)是单点集
\item \(F\)是超滤
\item \(F\)是主超滤
\end{enumerate}
\end{exercise}

\begin{proof}
若\(G\)有有限交性质,\(F\)是由\(G\)生成的主滤(书上这里写了主滤),则以下命题等价
\begin{enumerate}
\item \(G\)包含原子的单点集
\item \(F\)是超滤
\item \(F\)是主超滤
\end{enumerate}



\(1\to 2\)。因为\(G\)有有限交性质,\(G\)只包含一个原子\(a\),因为对任意\(b\in B\),或者\(a\le b\)或者\(a\le -b\),因此\(G\)是超滤。

\(3\to 2\)。显然。

\(3\to 1\)。若\(F\)是主超滤,假设它由\(\{a\}\)生成,则对于任意\(b\in B\),\(a\le b\)或者\(a\le -b\)。因此\(a\)
是原子且\(a\in G\)。

\(2\to 3\)。显然。
\end{proof}

\begin{exercise}[3.2.18]
如果\(a\neq b\),则存在超滤\(U\),\(a\in U\)但\(b\notin U\)
\end{exercise}

\begin{proof}
如果\(a<b\),则任何包含\(a\)的滤都包含\(b\)
\end{proof}

\begin{exercise}[3.2.19]
令\(F\)是\(\calb\)上的滤,令\((\{0,1\},+,\cdot,-,0,1)\)为两个元素的布尔代数。定义\(f:B\to\{0,1\}\)为
\begin{align*}
f(b)=
\begin{cases}
1&b\in F\\
0&b\notin F
\end{cases}
\end{align*}
证明:\(F\)是超滤当且仅当\(f\)是布尔代数\(\calb\)到\(\{0,1\}\)的同态映射
\end{exercise}

\begin{proof}
若\(F\)是超滤,则\(f(0)=0\)且\(f(1)=1\)。因为\(a\cdot b\in F\)当且仅当\(a\in F\wedge b\in F\),因
此\(f(a\cdot b)=f(a)\cdot f(b)\)。而\(a+b\in F\)当且仅当\(a\in F\vee b\in F\),因此\(f(a+b)=f(a)+f(b)\)。因此\(f\)是
同态映射

若\(f\)是同态映射,于是有\(a\cdot b\in F\Leftrightarrow a\in F\wedge b\in F\)且\(a+b\in F\Leftrightarrow a\in F\vee b\in F\)。若\(a,-a\notin F\),
则\(a+(-a)=1\notin F\),矛盾。因此\(F\)是超滤。
\end{proof}

\begin{exercise}[3.2.33]
令\(X\)为任意集合。
\begin{enumerate}
\item 如果\(X\)是可数集合,则\(\calp(X)\)上的所有\(\aleph_1\)-完全滤都是主滤
\item 如果\(X\)不可数,则\(\{G\subset X\mid\abs{G}\le\aleph_0\}\)是\(X\)上\(\aleph_1\)-完全的理想
\item 如果\(\kappa>\aleph_1\)正则,而\(\abs{X}\ge\kappa\),则\(I=\{G\subset S\mid\abs{G}<\kappa\}\)是\(\kappa\)-完全的理想
\end{enumerate}
\end{exercise}

\begin{proof}
\begin{enumerate}
\item 

\item 令\(I=\{G\subset X\mid\abs{G}\le\aleph_0\}\),因为可数集合的可数并依然可数,因此\(I\)是\(\aleph_1\)-完全的。下证\(I\)是理
想。首先\(\emptyset\in I\),对于任意\(X,Y\in I\),\(\abs{X\cap Y}\le\abs{X}\le\aleph_0\),因此\(X\cap Y\in I\)。若\(X\in I\)
且\(Y\subset X\),则\(Y\in I\)
\item 对\(I\)的任意子集\(I'\)且\(\abs{I'}=\gamma<\kappa\),有一个枚举\(I'=\{I_\alpha\mid\alpha<\gamma\}\),定义\(Y_\alpha\)为
\begin{align*}
 &Y_0=\{I_0\}\\
 &Y_\alpha=Y_\beta\cup \{I_\alpha\}\hspace{1cm}\alpha=\beta+1\\
 &Y_\gamma=\bigcup_{\alpha<\gamma}Y_\alpha\hspace{1cm}\alpha\text{是极限序数}
\end{align*}
于是有
\begin{equation*}
Y_0\subseteq Y_1\subseteq Y_2\subseteq\cdots
\end{equation*}
且对于任意\(\alpha<\gamma\),\(\abs{Y_\alpha}<\gamma\)。因为\(I'=\bigcup_{\alpha<\gamma} Y_\alpha\),又因为\(\kappa\)正则,于是\(\abs{I'}<\kappa\),
因此\(\bigcap I'\in I\),\(I\)是\(\kappa\)-完全的理想
\end{enumerate}
\end{proof}
\end{document}
