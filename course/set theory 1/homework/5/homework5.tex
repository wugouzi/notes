% Created 2021-11-01 Mon 10:17
% Intended LaTeX compiler: pdflatex
\documentclass[11pt]{article}
\usepackage[utf8]{inputenc}
\usepackage[T1]{fontenc}
\usepackage{graphicx}
\usepackage{longtable}
\usepackage{wrapfig}
\usepackage{rotating}
\usepackage[normalem]{ulem}
\usepackage{amsmath}
\usepackage{amssymb}
\usepackage{capt-of}
\usepackage{hyperref}
\input{../../../preamble-lite.tex}
\usepackage[UTF8]{ctex}
\declareslashed{}{/}{0.05}{0}{A}
\author{陈淇奥\\21210160025}
\date{\today}
\title{Homework5}
\hypersetup{
 pdfauthor={陈淇奥\\21210160025},
 pdftitle={Homework5},
 pdfkeywords={},
 pdfsubject={},
 pdfcreator={Emacs 27.2 (Org mode 9.5)}, 
 pdflang={English}}
\begin{document}

\maketitle
\begin{exercise}[2.1.31]
如果\(X\)是冯\(\cdot\)诺伊曼基数的集合,则\(\bigcup X\)也是冯\(\cdot\)诺伊曼基数
\end{exercise}

\begin{proof}
若\(X\)是所有有穷冯诺依曼序数的集合,则\(X\)中有一个最大的元素\(n\),且\(\bigcup X=n\),于是\(\bigcup X\)也是冯
诺依曼序数。

否则,假设\(\alpha=Card(\bigcup X)\)且\(\alpha<\bigcup X\)。则存在一个双射\(f:\bigcup X\to\alpha\)。
因为\(\alpha\in\bigcup X\),于是存在一个\(X\)中的冯诺依曼序数\(\kappa\)使得\(\alpha\in\kappa\)。因为\(\kappa\subseteq\bigcup X\),于
是\(f\upharpoonright\kappa\)是一个从\(\kappa\)到\(f(\kappa)\subseteq\alpha\)的双射,于是\(Card(\kappa)<\kappa\),矛盾。因此\(\alpha=\bigcup X\),
\(\bigcup X\)是冯诺依曼序数。
\end{proof}

\begin{exercise}[2.1.39]
令\(X\)是一个不可良序化的集合,令\(\lambda=H(X)\)。\(\lambda\)是冯诺依曼基数。证明:\(\lambda\slashed{\precsim} X\)并
且\(X\slashed{\precsim}\lambda\)
\end{exercise}

\begin{proof}
若\(X\precsim\lambda\),则存在单射\(f:X\to\lambda\),于是有双射\(g:X\to f(X)\),而\(f(X)\subseteq\lambda\)是良序集,于是\(X\)可良
序,矛盾。

若\(\lambda\precsim X\),而\(\lambda\)是最小的不与\(X\)的子集等势的序数,矛盾。
\end{proof}

\begin{exercise}[2.1.37]
如果\(F:\O\to\O\)是严格递增的,并且是连续的,则对任意序数\(\alpha\),存在\(\epsilon>\alpha\),\(F(\epsilon)=\epsilon\)。即,\(F\)有任意
大的不动点
\end{exercise}

\begin{proof}
首先证明对任意序数\(\alpha\)都有\(F(\alpha)\ge\alpha\)。

若\(\alpha=0\),则\(F(0)\ge 0\)。

若\(\alpha=\beta+1\),则\(F(\alpha)=F(\beta+1)>F(\beta)\ge\beta+1\)。

若\(\alpha=\bigcup_{\beta<\alpha}\beta\),则\(F(\alpha)=\bigcup\{F(\beta)\mid\beta<\alpha\}\ge\bigcup\{\beta<\alpha\}=\alpha\)。


注意到\(F(\alpha)\le F(\alpha)^\alpha\),\(F(\alpha)^{F(\alpha)^\alpha}\ge F(\alpha)^\alpha\),令\(\epsilon_0=\alpha\),对于任意\(i\in\omega\),构造\(\epsilon_{i+1}\)为
\begin{align*}
&\epsilon_{i+1,0}=F(\epsilon_i)\\
&\epsilon_{i+1,n+1}=F(\epsilon_i)^{\epsilon_{i+1,n}}\quad n\in\omega\\
&\epsilon_{i+1}=\bigcup_{n\in\omega}\epsilon_{i,n}
\end{align*}
于是\(F(\epsilon_i)^{\epsilon_{i+1}}=\bigcup\{F(\epsilon_i)^{\epsilon_{i+1,n}}\mid n\in\omega\}=\bigcup\{\epsilon_{i+1,n+1}\mid n\in\omega\}=\epsilon_{i+1}\)。令\(\epsilon=\bigcup_{i\in\omega}\epsilon_i\),
则\(\epsilon=F(\epsilon)^\epsilon\)。由于\(F(\epsilon)\ge\epsilon\)且\(F(\epsilon)\le F(\epsilon)^\epsilon=\epsilon\),我们有\(F(\epsilon)=\epsilon\)。
\end{proof}
\end{document}
