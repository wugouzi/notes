% Created 2021-10-03 Sun 20:56
% Intended LaTeX compiler: pdflatex
\documentclass[11pt]{article}
\usepackage[utf8]{inputenc}
\usepackage[T1]{fontenc}
\usepackage{graphicx}
\usepackage{grffile}
\usepackage{longtable}
\usepackage{wrapfig}
\usepackage{rotating}
\usepackage[normalem]{ulem}
\usepackage{amsmath}
\usepackage{textcomp}
\usepackage{amssymb}
\usepackage{capt-of}
\usepackage{hyperref}
\author{wu}
\date{\today}
\title{}
\hypersetup{
 pdfauthor={wu},
 pdftitle={},
 pdfkeywords={},
 pdfsubject={},
 pdfcreator={Emacs 27.2 (Org mode 9.5)}, 
 pdflang={English}}
\begin{document}

\tableofcontents

\% Created 2021-10-03 Sun 20:56
\% Intended \LaTeX{} compiler: pdflatex
\documentclass[11pt]{article}
\usepackage[utf8]{inputenc}
\usepackage[T1]{fontenc}
\usepackage{graphicx}
\usepackage{grffile}
\usepackage{longtable}
\usepackage{wrapfig}
\usepackage{rotating}
\usepackage[normalem]{ulem}
\usepackage{amsmath}
\usepackage{textcomp}
\usepackage{amssymb}
\usepackage{capt-of}
\usepackage{hyperref}
\usepackage[UTF8]{ctex}
\input{../../../preamble-lite.tex}
\author{陈淇奥\\21210160025}
\date{\today}
\title{Homework}
\hypersetup\{
 pdfauthor=\{陈淇奥$\backslash$\21210160025\},
 pdftitle=\{Homework\},
 pdfkeywords=\{\},
 pdfsubject=\{\},
 pdfcreator=\{Emacs 27.2 (Org mode 9.5)\}, 
 pdflang=\{English\}\}
\begin{document}

\maketitle
\tableofcontents

\begin{exercise}[1.3.22]
假设\(X,Y\)是传递集
\begin{enumerate}
\item \(X\cap Y\),\(X\cup Y\) 与\(\calp(X)\)是传递集
\item \(X\cup\{X\}\)是传递集
\item 如果\(\calt\)的元素都是传递集,则\(\bigcup\calt\)是传递集
\end{enumerate}
\end{exercise}

\begin{proof}
\begin{enumerate}
\item 对于任意\(x\in X\cap Y\),因为\(X,Y\)是传递集,因此\(x\subseteq X\)且\(x\subseteq Y\),于是\(x\subseteq X\cap Y\)

对于任意\(x\in X\cap Y\),因为\(X\)是传递集,因此\(x\subseteq X\subseteq X\cup Y\),于是\(x\subseteq X\cup Y\)

对于任意\(x\in\calp(X)\),\(x\subseteq X\)。对于任意\(y\in x\),\(y\in X\),于是\(y\subseteq X\),因此\(x\subseteq\calp(X)\)

\item 对于任意\(x\in X\cup\{X\}\),若\(x\neq X\),则\(x\in X\),于是\(x\subseteq X\subset X\cup\{X\}\)。若\(x=X\),则\(x\subset X\cup\{X\}\)

\item 对于任意\(x\in\bigcup\calt\),则存在\(Y\in\calt\)使得\(x\in Y\)。由于\(Y\)传递,于是\(x\subseteq Y\subseteq\bigcup\calt\)
\end{enumerate}
\end{proof}

\begin{exercise}[1.3.27]
令 \(\alpha\),\(\beta\) 是序数
\begin{enumerate}
\item \(\alpha\cap\beta\)也是序数;(\(\alpha\cup\beta\)是序数吗)
\item \(\alpha\cup\{\alpha\}\)也是序数
\item 如果\(\beta\subseteq\alpha\)并且\(\beta\neq\alpha\),则\(\beta\in\alpha\)
\end{enumerate}
\end{exercise}

\begin{proof}
\begin{enumerate}
\item 因为\(\alpha\cap\beta\subseteq\alpha\),因此\(\in\)在\(\alpha\cap\beta\)上是良序。对于任意\(x\in\alpha\cap\beta\),有\(x\subseteq\alpha\)且\(x\subseteq\beta\),因此\(x\subseteq\alpha\cap\beta\)。

\(\alpha\cup\beta\)是序数。由引理1.3.28我们知道\(\alpha<\beta\)或\(\alpha=\beta\)或\(\beta<\alpha\),于是\(\alpha\cup\beta=\beta\)或\(\alpha\),因此\(\in\)
在\(\alpha\cup\beta\)上是良序,而且它也是良序

\item 因为\(\in\)在\(\alpha\) 上是良序,而对于所有\(x\in\alpha\),自然有\(x\in\alpha\),因此\(\in\)在\(\alpha\cup\{\alpha\}\)上也是良序。而由前面
的练习得到\(\alpha\cup\{\alpha\}\)是传递的,因此是序数。

\item 考虑集合\(S=\{\gamma\in\alpha:\forall \eta\in\beta(\eta<\gamma)\}\),令\(\gamma_0\)是\(S\)中最小的元素,那么显然\(\beta\subseteq\gamma_0\)。若\(\gamma_0\neq\beta\),则
存在\(x\in\gamma_0\setminus\beta\),且\(\beta\subseteq x\),于是\(\gamma_0\)不是\(S\)中的最小元素,矛盾,因此\(\beta=\gamma_0\in\alpha\)
\end{enumerate}
\end{proof}

\begin{exercise}[1.3.33]
如果\(X\)是序数的集合,则\(\bigcap X\)和\(\bigcup X\)都是序数
\end{exercise}

\begin{proof}
由于\(\in\)在\(X\)的每个元素上都是良序,因此在\(\bigcap X\)上是良序的。而对于任意\(x\in\bigcap X\),对于任
意\(\alpha\in\bigcap X\)都有\(x\in\alpha\),于是\(x\subseteq\alpha\),因此\(x\subseteq\bigcap X\),因此\(\bigcap X\)是序数

由引理1.3.28,\(\in\)在\(\bigcup X\)良基。对任意\(x,y,z\in\bigcup X\),则存在\(\alpha,\beta,\gamma\in X\)使得\(x\in\alpha,y\in\beta,z\in\gamma\),由引
理1.3.28,\(x,y,z\in\sup\{\alpha,\beta,\gamma\}\in X\),而\(\in\)在\(\sup\{\alpha,\beta,\gamma\}\)上是线序,因此\(\in\)在\(\bigcup X\)上是线序。

对任意\(x\in\bigcup X\),存在\(\alpha\in X\)使得\(x\in\alpha\),于是\(x\subseteq\alpha\),因此\(x\subseteq\bigcup X\)。因此\(\bigcup X\)是序数
\end{proof}

\begin{exercise}[1.3.34]
对任意序数\(\alpha\),\(\alpha\) 是极限序数当且仅当\(\bigcup\alpha\notin\alpha\)
\end{exercise}

\begin{proof}
\(\Rightarrow\)。若\(\alpha\) 是极限序数且\(\bigcup\alpha\in\alpha\),则存在\(\beta\in\alpha\)使得\(\beta=\bigcup\alpha\),即对于任意\(\gamma\in\alpha\)都有\(\gamma\le\beta\),因
此\(\beta+1\notin\alpha\),因此\alpha不是极限序数,矛盾。

\(\Leftarrow\)。若\alpha不是极限序数,则存在一个序数\beta使得\(\alpha=\beta\cup\{\beta\}\),于是\(\bigcup\alpha=\beta\in\alpha\),矛盾。
\end{proof}

\begin{exercise}[1.3.43]
利用超穷归纳法证明一下关于\(V_\alpha\)的性质
\begin{enumerate}
\item 对所有\(\alpha\),\(V_\alpha\)是传递集
\item 如果\(\alpha<\beta\),则\(V_\alpha\subseteq V_\beta\)
\end{enumerate}
\end{exercise}

\begin{proof}
\begin{enumerate}
\item \(V_0=V_\emptyset\)是传递集

若\alpha是后继序数且\(V_\alpha\)是传递集,对于任意\(x\in\calp(V_\alpha)\),\(x\subseteq V_\alpha\),于是对于任意\(y\in x\),
有\(y\in V_\alpha\),因为\(V_\alpha\)传递,\(y\subseteq V_\alpha\),因此\(x\subseteq\calp(V_\alpha)=V_{\alpha+1}\)

若\alpha是极限序数且对所有\(\lambda<\alpha\),\(V_\lambda\)都是传递集。对于任意\(x\in V_\alpha\),存在\(\beta<\alpha\)使得\(x\in V_\beta\)
且\(V_\beta\)传递,于是\(x\subseteq V_\beta\subset\bigcup_{\lambda<\alpha}V_\lambda\),因此\(V_\alpha\)传递

\item 对\beta做归纳

若\(\beta=0\),则命题恒成立

若\(\beta=\gamma+1\),因为\(V_\beta\)是传递集,因此由于\(V_\gamma\in \calp(V_\gamma)=V_\beta\),有\(V_\gamma\subseteq V_\beta\)。对任意\(\alpha<\gamma\),由归纳假
设,\(V_\alpha\subseteq V_\gamma\subseteq V_\beta\)。因此对于任意\(\alpha<\beta\)都有\(V_\alpha\subseteq V_\beta\)

若\beta是极限序数,则\(V_\beta=\bigcup_{\lambda<\beta}V_\lambda\)。因此对于任意\(\alpha<\beta\),都有\(V_\alpha\subseteq \bigcup_{\lambda<\beta}V_\lambda=V_\beta\)
\end{enumerate}
\end{proof}

\begin{exercise}[1.3.45]
证明以下命题
\begin{enumerate}
\item \(V_\alpha=\{x\in V\mid\rank(x)<\alpha\}\)
\item \(V\)是传递的
\item 对任意\(x,y\in V\),如果\(x\in y\),则\(\rank(x)<\rank(y)\)
\item 对任意\(x\in V\),\(\rank(x)=\bigcup\{\rank(y)+1\mid y\in x\}\)
\end{enumerate}
\end{exercise}

\begin{proof}
\begin{enumerate}
\item 令\(S=\{x\in V\mid\rank(x)<\alpha\}\)。根据定义\(V_\alpha\subseteq S\)。对于任意\(x\in S\),因为\(\rank(x)<\alpha\),因此存在
\(\beta<\alpha\)使得\(x\in V_{\beta+1}\subseteq V_\alpha\)。因此\(V_\alpha=S\)
\item 对任意\(y\in V\),则存在\(\alpha\in Ord\)使得\(y\in V_\alpha\),于是\(y\subseteq V_\alpha\subseteq V\)
\item 对任意\(x,y\in V\),令\(\rank(y)=\alpha\),于是\(y\in V_{\alpha+1}\),因此\(x\in y\subseteq\bigcup V_{\alpha+1}=V_\alpha\),所以\(\rank(x)<\rank(y)\)
\item 由3得,\(\rank(x)\ge\bigcup\{\rank(y)+1\mid y\in x\}\)。令\(\rank(x)=\alpha\),假设对于所有\(y\in x\)都
有\(\rank(x)>\rank(y)+1\),\(\rank(y)<\alpha-1\),于是\(\rank(x)<\alpha\),矛盾。因
此\(\rank(x)=\bigcup\{\rank(y)+1\mid y\in x\}\)
\end{enumerate}
\end{proof}
\end{document}
\end{document}
