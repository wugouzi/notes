% Created 2021-11-22 Mon 10:20
% Intended LaTeX compiler: pdflatex
\documentclass[11pt]{article}
\usepackage[utf8]{inputenc}
\usepackage[T1]{fontenc}
\usepackage{graphicx}
\usepackage{longtable}
\usepackage{wrapfig}
\usepackage{rotating}
\usepackage[normalem]{ulem}
\usepackage{amsmath}
\usepackage{amssymb}
\usepackage{capt-of}
\usepackage{hyperref}
\input{../../../preamble-lite.tex}
\usepackage[UTF8]{ctex}
\author{陈淇奥\\21210160025}
\date{\today}
\title{Homework7}
\hypersetup{
 pdfauthor={陈淇奥\\21210160025},
 pdftitle={Homework7},
 pdfkeywords={},
 pdfsubject={},
 pdfcreator={Emacs 27.2 (Org mode 9.6)}, 
 pdflang={English}}
\begin{document}

\maketitle
\begin{lemma}[]
\begin{enumerate}
\item \(-0=1\)
\item \(-1=0\)
\item \(a\cdot 1=a\)
\item \(a+0=a\)
\item \(a+a=a\)
\item \(a\cdot a=a\)
\item \(1+a=1\)
\item \(0\cdot a=0\)
\item \(a+b=1\wedge a\cdot b=0\Rightarrow b=-a\)
\item \(-(a\cdot b)=(-a)+(-b)\)
\end{enumerate}
\end{lemma}

\begin{proof}
\begin{enumerate}
\item \(1=0+(-0)=(0\cdot(-0))+(-0)=-0\)
\item \(0=1\cdot(-1)=(1+(-1))\cdot(-1)=-1\)
\item \(a\cdot 1=a\cdot(a+(-a))=a\)
\setcounter{enumi}{4}
\item \(a+a=a+(a\cdot 1)=a\)
\setcounter{enumi}{6}
\item \(1+a=(a+1)\cdot 1=(a+1)\cdot(a+-a)=a\cdot a+0+a+-a=a+-a=1\)
\item \(0\cdot a=(a\cdot (-a))\cdot a=a\cdot a\cdot (-a)=a\cdot (-a)=0\)
\setcounter{enumi}{8}
\item \(-a=(-a)\cdot 1=(-a)\cdot(a+b)=(-a)\cdot a+(-a)\cdot b=(-a)\cdot b\).

\(ab+(-a)b=-a\). \(b(a+(-a))=b=-a\)
\item \(ab+(-a)+(-b)=ab+(-a)+(-b)\cdot 1=ab+(-a)+(-b)a+(-b)(-a)=a(b+(-b))+(-a)+(-b)(-a)=1+(-b)(-a)=1\)
\end{enumerate}
\end{proof}

\begin{exercise}[]
证明不存在基数为3的布尔代数
\end{exercise}

\begin{proof}
若存在基数为3的布尔代数\(\calb\),则令\(B=\{0,1,a\}\)。

如果 \(-a=0\),那么 \(a+(-a)=a+0=a+(a\cdot(-a))=a\neq 1\),矛盾。
如果\(-a=1\),那么\(a\cdot(-a)=a\cdot 1=a\cdot(a+(-a))=a\neq 0\),矛盾。
如果\(-a=a\),那么\(a=a\cdot 1=a\cdot(a+a)=a\cdot a+a\cdot a=0+0=0\),矛盾。

因此不存在基数为3的布尔代数。
\end{proof}

\begin{exercise}[3.1.10]
令\(\calb\)为任意布尔代数
\begin{enumerate}
\item 证明任意布尔代数\(\calb\)在关系\(\le\)下是偏序
\setcounter{enumi}{3}
\item 对任意\(a,b\in\calb\),\(a\cdot(-b)=0\)当且仅当\(a\le b\)
\end{enumerate}
\end{exercise}

\begin{proof}
\begin{enumerate}
\item 因为\(x=x\),因此\(x=x\)

若\(x\le y\wedge y\le x\)则存在\(c,d\)使得\(c\neq 0\wedge d\neq 0\wedge x+c=y\wedge y+d=x\),于
是\(x=y+d=y+y+d=x+y=x+x+c=x+c=y\)。
\end{enumerate}


若\(x\le y\wedge y\le z\),则存在\(c,d\)使得\(c\neq 0\wedge d\neq 0\wedge x+c=y\wedge y+d=z\),因此\(x+c+d=z\)。

\begin{enumerate}
\setcounter{enumi}{3}
\item \(a\cdot(-b)=0\Rightarrow a=ab\Rightarrow b=(a+1)b=a+b\)

若存在\(c\neq 0\wedge a+c=b\),则\(a+b=a+a+c=a+c=b\),
因此\(a(-b)=a(-b)+0=b(-b)=0\)。若\(a=b\),则\(a\cdot(-b)=0\)。
\end{enumerate}
\end{proof}

\begin{exercise}[3.1.13]
任意有穷的布尔代数都是原子化的
\end{exercise}

\begin{proof}
给定一个有限布尔代数\(\calb\),对任意\(b\in\calb\),任选一条\(b\)的最长下降链\(C=\{c_0,c_1,\dots,c_n\}\)使
得\(0=c_0<c_1<\dots<c_n=b\),若\(c_1\)不是原子,则存在\(0<c'<c_1\),于是\(C\)不是最长的,矛盾。因
此\(c_1\le b\)是原子
\end{proof}

\begin{exercise}[3.1.16]
证明\(f\)是同态映射
\end{exercise}

\begin{proof}
因为0不是原子,于是\(f(0)=\emptyset\)。有因为所有原子都小于等于1,因此\(f(1)=A\)。

对于任意\(b_1,b_2\in B\),因为\(b_1\le b_1+b_2\)且\(b_2\le b_1+b_2\),因此\(f(b_1+b_2)\supseteq f(b_1)\)
且\(f(b_1+b_2)\supseteq f(b_2)\),于是\(f(b_1+b_2)\supseteq f(b_1)\cup f(b_2)\)。对于任意原子\(a\le b_1+b_2\),由引理3.1.14,
\(a\le b_1\)或\(a\le b_2\),于是\(a\in f(b_1)\cup f(b_2)\)。因此\(f(b_1+b_2)=f(b_1)\cup f(b_2)\)。

对于任意\(b_1,b_2\in B\),有\(b_1b_2\le b_1\wedge b_1b_2\le b_2\),因此\(f(b_1b_2)\subseteq f(b_1)\cap f(b_2)\)。对于任意原
子\(a\in f(b_1)\cap f(b_2)\),则\(a\le b_1\wedge a\le b_2\),若\(a\le-(b_1b_2)=(-b_1)+(-b_2)\),则\(a\le -b_1\)或\(a\le -b_2\),矛
盾。因此\(a\le b_1b_2\),于是\(a\in f(b_1b_2)\)

对于任意\(b\in B\),\(x\in f(-b)\Leftrightarrow x\le -b\Leftrightarrow x\not\le b\Leftrightarrow x\notin f(b)\Leftrightarrow x\in \calp(A)-f(b)\)。
\end{proof}

\begin{exercise}[3.1.22]
若\(\calb\)完全且是原子化的,\(A\)是\(\calb\)中所有原子的集合,则\(f:B\to\calp(A)\)是一个同构
\end{exercise}

\begin{proof}
对于任意\(Y\subseteq A\)与原子\(a\),若\(a\in Y\),则\(a\le\sum Y\);若\(a\le\sum Y\),假设\(a\notin Y\),那么对于任意\(b\in Y\)都
有\(a\le -b\)等价于\(b\le -a\),于是\(-a\ge\sum Y\ge a\),因此\(-a=a+(-a)=1\),而\(a=0\),与\(a\)是原子矛盾。因
此存在\(b\in Y\)使得\(a\le b\),因为\(a,b\)都是原子,因此\(a=b\in Y\)。

因此对于任意原子\(a\),\(a\in Y\)当且仅当\(a\le\sum Y\),所以\(f(\sum Y)=Y\),于是\(f\)是满射,于是\(f\)是双射
\end{proof}
\end{document}
