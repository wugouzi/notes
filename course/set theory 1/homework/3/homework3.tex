% Created 2021-10-19 Tue 16:16
% Intended LaTeX compiler: pdflatex
\documentclass[11pt]{article}
\usepackage[utf8]{inputenc}
\usepackage[T1]{fontenc}
\usepackage{graphicx}
\usepackage{grffile}
\usepackage{longtable}
\usepackage{wrapfig}
\usepackage{rotating}
\usepackage[normalem]{ulem}
\usepackage{amsmath}
\usepackage{textcomp}
\usepackage{amssymb}
\usepackage{capt-of}
\usepackage{hyperref}
\input{../../../preamble-lite.tex}
\usepackage[UTF8]{ctex}
\author{陈淇奥\\21210160025}
\date{\today}
\title{Homework3}
\hypersetup{
 pdfauthor={陈淇奥\\21210160025},
 pdftitle={Homework3},
 pdfkeywords={},
 pdfsubject={},
 pdfcreator={Emacs 27.2 (Org mode 9.5)}, 
 pdflang={English}}
\begin{document}

\maketitle
\begin{exercise}[1.5.26]
证明任何序数都可表示为\(\alpha+n\),其中\(\alpha\) 是0或极限序数,而\(n\in\omega\)。并且这种表示唯一。
\end{exercise}

\begin{proof}
由定理1.4.12,任何非0序数\(\beta\) 都可唯一表示为
\begin{equation*}
\beta=\omega^{\gamma_0}\cdot\delta_0+\dots+\omega^{\gamma_{k-1}}\cdot\delta_{k-1}
\end{equation*}
其中\(k\in\omega\),\(\delta_i\)和\(\gamma_i\)都是序数,\(\gamma_i\in\omega\),并且\(\gamma_0>\cdots>\gamma_{k-1}\)。若\(\gamma_{k-1}\neq 0\),由练习1.5.31,
\(\beta\) 是极限序数且\(\beta=\beta+0\)。若\(\gamma_{k-1}=0\),令\(\alpha=\omega^{\gamma_0}\cdot\delta_0+\dots+\omega^{\gamma_{k-2}}\cdot\delta_{k-2}\),由练习1.5.31,
\(\alpha\) 是极限序数,\(\beta=\alpha+\delta_{k-1}\)
\end{proof}

\begin{exercise}[1.5.30]
如果\(\alpha<\beta\),则
\begin{enumerate}
\item \(\alpha+\gamma\le\beta+\gamma\)
\item \(\alpha\cdot\gamma\le\beta\cdot\gamma\)
\end{enumerate}


而\(\le\)不能替换为<
\end{exercise}

\begin{proof}
\begin{enumerate}
\item 对\(\gamma\) 应用超穷归纳证明:若\(\gamma=0\),由条件可知\(\alpha\le\beta\);若\(\gamma=\delta+1\),由归纳假设,\(\alpha+\delta\le\beta+\delta\)。若\(\alpha+\delta=\beta+\delta\),
则\((\alpha+\delta)+1=(\beta+\delta)+1=\alpha+(\delta+1)=\beta+(\delta+1)\);若\(\alpha+\delta<\beta+\delta\),
于是\((\alpha+\delta)+1\le\beta+\delta<(\beta+\delta)+1\),而\((\alpha+\delta)+1=\alpha+(\delta+1)\),\((\beta+\delta)+1=\beta+(\delta+1)\),
因此\(\alpha+(\delta+1)<\beta+(\delta+1)\)。综上,\(\alpha+\gamma\le\beta+\gamma\)。

若\gamma是极限序数,对于任意\(\alpha+\theta\in\bigcup\{\alpha+\delta\mid\delta<\gamma\}\),由归纳假设,有\(\alpha+\theta\le\beta+\theta\)。于
是\(\bigcup\{\alpha+\delta\mid\delta<\gamma\}\subseteq\bigcup\{\beta+\delta\mid\delta<\gamma\}\) ,于是\(\alpha+\gamma\le\beta+\gamma\)。

若\(\alpha,\beta\in\omega\)且\(\gamma=\omega\),则\(\alpha+\omega=\beta+\omega\)。

\item 对\(\gamma\) 应用超穷归纳证明:若\(\gamma=0\),则\(\alpha\cdot\gamma=0=\beta\cdot\gamma\);若\(\gamma=\delta+1\),
 \(\alpha\cdot\gamma=\alpha\cdot(\delta+1)=\alpha\cdot\delta+\delta\le\beta\cdot\delta+\delta=\beta\cdot(\delta+1)=\beta\cdot\gamma\);若\(\gamma\)
 是极限序数,对于任意\(\alpha\cdot\theta\in\bigcup\{\alpha\cdot\theta\mid\theta<\gamma\}\),都
有\(\alpha\cdot\theta\le\beta\cdot\theta\),于是\(\alpha\cdot\gamma\subseteq\beta\cdot\gamma\),因此\(\alpha\cdot\gamma\le\beta\cdot\gamma\)

若\(\alpha,\beta\in\omega\),则\(\alpha\cdot\omega=\beta\cdot\omega\)
\end{enumerate}
\end{proof}

\begin{exercise}[1.5.31]
一个序数\(\alpha\) 是极限序数当且仅当存在\(\beta\) ,\(\alpha=\omega\cdot\beta\)
\end{exercise}

\begin{proof}
若\(\alpha=\omega\cdot\beta\),对于任意\(\omega\le\gamma<\alpha\),由定理1.4.12,\(\gamma=\omega\cdot\delta_0+\delta_1\),其中\(\delta_0<\beta\)。于是
\begin{equation*}
\gamma+1=(\omega\cdot\delta_0+\delta_1)+1=\omega\cdot\delta_0+(\delta_1+1)<\omega\cdot\delta_0+\omega=\omega\cdot(\delta_0+1)\le\omega\cdot\beta
\end{equation*}
对于任意\(\gamma<\omega\),\(\gamma+1<\omega<\alpha\)。因此\(\alpha\) 不是后继序数,于是它是极限序数

若\(\alpha\) 是极限序数,由定理1.4.12可知
\begin{equation*}
\alpha=\omega^{\gamma_0}\cdot\delta_0+\dots+\omega^{\gamma_{k-1}}\cdot\delta_{k-1}
\end{equation*}
其中\(k\in\omega\),\(\delta_i\)和\(\gamma_i\)都是序数且\(\gamma_0>\dots>\gamma_{k-1}\)。若\(\gamma_{k-1}=0\),则因
为\(\delta_{k-1}\in\omega\)是后继序数,所以\(\delta_{k-1}=\delta_{k-1}'+1'\),于是
\begin{equation*}
\alpha=\left( \omega^{\gamma_0}\cdot\delta_0+\dots+\omega^{\gamma_{k-2}}\cdot\delta_{k-2}+\delta_{k-1}' \right)+1
\end{equation*}
是后继序数,矛盾。因此\(\gamma_{k-1}\neq 0\),于是
\begin{equation*}
\alpha=\omega\cdot(\omega^{\gamma_0-1}\cdot\delta_0+\dots+\omega^{\gamma_{k-1}-1}\cdot\delta_{k-1})
\end{equation*}
\end{proof}

\begin{exercise}[1.5.33]
找到函数\(f:\omega\to\omega+\omega\)和\(g:\omega+\omega\to\omega+\omega+\omega\)满足
\begin{enumerate}
\item \(\sup(f[\omega])=\omega+\omega\)
\item \(\sup(g[\omega+\omega])=\omega+\omega+\omega\)
\item 如果\(h=g\circ f\),有\(\sup(h[\omega])<\omega+\omega+\omega\)
\end{enumerate}
\end{exercise}

\begin{proof}
对于任意\(x\in\omega\)
\begin{equation*}
f(x)=
\begin{cases}
\omega+x&x\text{是偶数}\\
0&x\text{是奇数}
\end{cases}
\end{equation*}
\begin{equation*}
g(x)=
\begin{cases}
0&x\text{是偶数}\\
\omega+x&x\text{是奇数}
\end{cases}\hspace{1cm}
g(\omega+x)=
\begin{cases}
0&x\text{是偶数}\\
\omega+\omega+x&x\text{是奇数}
\end{cases}
\end{equation*}
\end{proof}

\begin{exercise}[1.5.38]
对任意集合\(X\),存在一个序数\(H(X)\),\(H(X)\)不与\(X\)的任意子集等势,并且是具有如此性质的最小序数。
令\(W=\{w\subseteq X\mid w\text{上存在良序}\}\),
\begin{equation*}
H(X)=\{\alpha\mid\text{存在}w\in W,\alpha\text{是与$w$同构的唯一序数}\}
\end{equation*}
证明\(W\)是集合,\(H(X)\)是序数。
\end{exercise}

\begin{proof}
令
\begin{align*}
\varphi(w)&=\exists R(R\subseteq X\times X\wedge\forall x\forall y((x,x)\notin R\wedge((x,y)\in R\to\neg(y,x)\notin R))\\&\wedge\forall Y(Y\subseteq X\wedge Y\neq\emptyset\wedge\exists y_0(y_0\in Y\wedge\forall y(y\in Y\to y_0=y\vee y_0<y))))
\end{align*}
于是\(\varphi(w)\)表达了\(w\)上存在良序,于是\(W=\{w\in\calp(X)\mid\varphi(w)\}\)是集合。

由替换公理,\(H(X)\)是集合。由引理1.3.28,\(\in\)在\(H(X)\)是良序。对于任意非空\(y\in H(X)\)与\(x\in y\),
\(y\)与某个\(w\in W\)同构,记为\(f:y\to w\),则\(f|x\)依然是同构。因为\(x\)是序数,\(f\)保序,于
是\(f(x)\)是良序集,因此\(f(x)\in W\),所以\(x\in H(X)\)。从而\(H(X)\)是传递的,于是\(H(X)\)是序数。
\end{proof}
\end{document}
