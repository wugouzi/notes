% Created 2021-12-13 Mon 21:45
% Intended LaTeX compiler: pdflatex
\documentclass[11pt]{article}
\usepackage[utf8]{inputenc}
\usepackage[T1]{fontenc}
\usepackage{graphicx}
\usepackage{longtable}
\usepackage{wrapfig}
\usepackage{rotating}
\usepackage[normalem]{ulem}
\usepackage{amsmath}
\usepackage{amssymb}
\usepackage{capt-of}
\usepackage{hyperref}
\input{../../../preamble-lite.tex}
\usepackage[UTF8]{ctex}
\author{陈淇奥\\21210160025}
\date{\today}
\title{Homework 10}
\hypersetup{
 pdfauthor={陈淇奥\\21210160025},
 pdftitle={Homework 10},
 pdfkeywords={},
 pdfsubject={},
 pdfcreator={Emacs 27.2 (Org mode 9.6)}, 
 pdflang={English}}
\begin{document}

\maketitle
\begin{exercise}
对任意\(\alpha\)上的滤\(F\),如果\(X\notin F\),则\(\alpha-X\)有正测度
\end{exercise}

\begin{proof}
如果存在\(Y\in F\)使得\((\alpha-X)\cap Y=\emptyset\),于是\(Y\subseteq X\),因此\(X\in F\),矛盾
\end{proof}

\begin{exercise}
令\(\kappa\)为不可数正则基数,\(U\)为\(\kappa\)上\(\kappa\)完全的正则非主超滤,则\(\kappa\)的所有无界闭集都属于\(U\)
\end{exercise}


\begin{exercise}
对任意不可数正则基数\(\kappa\),任意连续共尾函数\(f:\kappa\to\kappa\),它的不动点集\(\{\epsilon\mid f(\epsilon)=\epsilon\}\)是\(\kappa\)的无界闭集
\end{exercise}

\begin{proof}
令\(S=\{\epsilon\mid f(\epsilon)=\epsilon\}\)

无界:对任意\(\alpha\in\kappa\),令\(\epsilon_0=f(\alpha)\),\(\epsilon_{n+1}=f(\epsilon_n)\),\(\epsilon=\bigcup_{n\in\omega}\epsilon_n\)。于是\(\epsilon\)是不动点且\(\epsilon>\alpha\)

闭:对任意\(\eta\)使得\(\sup(S\cap\eta)=\eta\),于是\(f(\eta)=\bigcup\{f(\beta):\beta\in S\cap\eta\}=\bigcup\{\beta:\beta\in S\cap\eta\}=\eta\),因此\(\eta\in S\)
\end{proof}

\begin{exercise}
\(\kappa\)是马洛基数当且仅当集合\(S=\{\lambda<\kappa\mid\lambda\text{是不可达基数}\}\)是\(\kappa\)上的平稳集
\end{exercise}

\begin{proof}
\(\Rightarrow\)。对任意无界闭集\(C\),都存在一个严格递增且连续的函数\(f:\kappa\to\kappa\)使得\(C=\im(f)\)。于是\(f\)有一个不动点\(\epsilon\)且\(\epsilon\)是不可达基数,因此\(C\cap S\neq\emptyset\)

\(\Leftarrow\)。对于任意\(\kappa\)上的连续共尾函数\(f:\kappa\to\kappa\),它的不动点集\(A=\{\epsilon\mid f(\epsilon)=\epsilon\}\)是无界闭集,因此存
在\(\lambda\in A\cap S\)
\end{proof}

\begin{exercise}
令\(T\subseteq\N^{<\omega}\),满足对任意\(s\in T\),任意\(n\in\dom(s)\),\(s(n+1)<s(n)\)。\((T,\subset)\)是一棵树,证
明\(T\)没有无穷枝
\end{exercise}

\begin{proof}
若\(T\)有无穷枝\((s_i:i\in\omega)\),令\(s=\bigcup_{i\in\omega} s_i\),于是\(\dom(s)=\omega\),而对于任意\(n\in\omega\),
\(s(n+1)<s(n)\),于是得到一条无穷下降链,矛盾
\end{proof}

\begin{exercise}
3.4.9中\(F\)是\(\aleph_1\)完全的非主超滤
\end{exercise}

\begin{proof}
\(X_0=\emptyset\),因为\(U\)是超滤,\(\emptyset\in F\)。

\(X_\gamma=\bigcup_{\beta\in\gamma}X_\gamma\in U\),因此\(\gamma\in F\)。

若\(M\subseteq N\)且\(M\in F\),于是\(X_M\subseteq X_N\),而\(X_M\in U\),于是\(X_N\in U\),\(N\in F\)。

若\(M,N\in F\),\(X_{M\cap N}=X_M\cap X_N\in U\),于是\(M\cap N\in F\)。

对任意\(M\subseteq\gamma\),\(X_M\in U\)或\(X_M^c\in U\),而\(X_M^c=X_{\gamma\setminus M}\),因此\(M\in F\)或\(\gamma\setminus M\in F\)。于是\(F\)是超
滤。

由于每个\(X_\beta\notin U\),所以\(\{\beta\}\notin F\),\(F\)是非主超滤。

对于任意\(\{Y_i:i\in\omega\}\),因为\(U\)是\(\aleph_1\)完全的滤,\(\bigcap_{i\in\omega}X_{Y_i}\in U\),于是\(\bigcap_{i\in\omega}Y_i\in U\)。
\end{proof}
\end{document}
