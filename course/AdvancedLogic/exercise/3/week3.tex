% Created 2022-03-26 Sat 15:19
% Intended LaTeX compiler: pdflatex
\documentclass[11pt]{article}
\usepackage[utf8]{inputenc}
\usepackage[T1]{fontenc}
\usepackage{graphicx}
\usepackage{longtable}
\usepackage{wrapfig}
\usepackage{rotating}
\usepackage[normalem]{ulem}
\usepackage{amsmath}
\usepackage{amssymb}
\usepackage{capt-of}
\usepackage{hyperref}
\input{../../../preamble-lite.tex}
\usepackage[UTF8]{ctex}
\author{陈淇奥\\21210160025}
\date{\today}
\title{Week3}
\hypersetup{
 pdfauthor={陈淇奥\\21210160025},
 pdftitle={Week3},
 pdfkeywords={},
 pdfsubject={},
 pdfcreator={Emacs 28.0.90 (Org mode 9.6)}, 
 pdflang={English}}
\begin{document}

\maketitle
\begin{exercise}[1.1.36]
如果\(\calb\)是一个完全的集合代数,则存在\(X\),\(\calb\cong\calp(X)\)
\end{exercise}

\begin{proof}
如果\(\calb\)是\(\calp(A)\)的完全子代数,定义\(A\)上的等价关系\(\sim\)为
\begin{equation*}
x\sim y\quad\text{ 当且仅当}\quad\forall b\in \calb, x\in b\Leftrightarrow y\in b
\end{equation*}
令\(f:\calb\to\calp(A/\sim)\)为\(f(b)=\{[a]\in A/\sim:a\in b\}\),于是我们可以验证这是一个同构,其中满射由完全性得到:
对于任何\(a\in A\),\(\{[a]\}=f(\bigcap\{b\in \calb:a\in b\})\),因为若\([a]\neq[c]\),则存在\(b\in\calb\)使得\(a\in b\)且\(c\notin b\)。
\end{proof}

\begin{exercise}[1.1.38]
若\(\calb\)是一个完全的原子化的布尔代数,则存在集合\(X\),\(\calb\cong\calp(A)\)
\end{exercise}

\begin{proof}
由定理1.1.26,
\(A\)是全体原子的集合,对于任何\(Y\subseteq A\),\(f(\Sigma Y)=\{a\in A\mid a\le\Sigma Y\}\),因此\(Y\subseteq f(\Sigma Y)\)。若存
在\(a\in A\setminus Y\)且\(a\le\sum Y\),于是对于任何\(b\in Y\),\(a\neq b\),于是\(a\cdot b=0\),因
此\(a\cdot\sum Y=\sum\{a\cdot b\mid b\in Y\}=0=a\)矛盾。因此\(f(\sum Y)=Y\),于是\(f\)是满射,因此\(\calb\cong\calp(A)\)
\end{proof}

\begin{exercise}[1.1.36]
如果\(\calb\)是一个完全集合代数,则存在\(X\),\(\calb\cong\calp(X)\)
\end{exercise}

\begin{proof}

\end{proof}

\begin{exercise}[1.2.3]
令\(\calb\)是布尔代数,\(F\subseteq B\),以下命题等价
\begin{enumerate}
\item \(F\)是滤
\item \(0\notin F\), \(1\in F\)并且对任意\(a,b\in B\),\(a\cdot b\in F\)当且仅当\(a\in F\)且\(b\in F\)
\end{enumerate}
\end{exercise}

\begin{proof}
\(1\to 2\); 因为\(F\neq\emptyset\),对于任何\(a\in F\), \(a\le 1\),因此\(1\in F\)。若\(ab\in F\),则\(ab\le a\)
且\(ab\le b\),因此\(a,b\in F\).若\(a,b\in F\),则\(ab\in F\)

\(2\to 1\): 因为\(1\in F\),因此\(F\neq\emptyset\),其它显然
\end{proof}

\begin{exercise}[1.2.5]
如果\(G\subseteq B\)有有穷交性质,\(a\in B\),则\(G\cup\{a\}\)或\(G\cup\{-a\}\)有有穷交性质
\end{exercise}

\begin{proof}
若\(G\cup\{a\}\)与\(G\cup\{-a\}\)都没有有穷交性质,则对于任意\(n\in\omega\),任意\(g_1,\dots,g_n\in G\),
\(g_1\cdot \dots g_n\cdot a=0=g_1\cdot\dots\cdot g_n\cdot(-a)\),于是\(g_1\cdot\dots\cdot g_n\cdot(a+-a)=0\),于是\(G\)没有有穷交性质
\end{proof}

\begin{exercise}[1.2.7]
如果\(F\)是由\(G\)生成的滤,则\(F\)是包含\(G\)的最小的滤,即\(G\subseteq F\)且如果\(F'\supseteq G\)也是滤,
则\(F\subseteq F'\)
\end{exercise}

\begin{proof}
对于任何包含\(G\)的滤\(F'\),若\(g_1\cdot\dots\cdot g_n\in F'\),对于任何\(b\in B\)且\(g_1\cdot\dots\cdot g_n\le b\),有\(b\in F'\),
因此\(F\subseteq F'\)
\end{proof}
\end{document}
