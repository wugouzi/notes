% Created 2022-02-23 Wed 18:16
% Intended LaTeX compiler: pdflatex
\documentclass[11pt]{article}
\usepackage[utf8]{inputenc}
\usepackage[T1]{fontenc}
\usepackage{graphicx}
\usepackage{longtable}
\usepackage{wrapfig}
\usepackage{rotating}
\usepackage[normalem]{ulem}
\usepackage{amsmath}
\usepackage{amssymb}
\usepackage{capt-of}
\usepackage{hyperref}
\input{../../../preamble-lite.tex}
\usepackage[UTF8]{ctex}
\author{陈淇奥\\21210160025}
\date{\today}
\title{week 1}
\hypersetup{
 pdfauthor={陈淇奥\\21210160025},
 pdftitle={week 1},
 pdfkeywords={},
 pdfsubject={},
 pdfcreator={Emacs 28.0.90 (Org mode 9.6)}, 
 pdflang={English}}
\begin{document}

\maketitle
\begin{exercise}[1.1.5]
令\(T\)为命题逻辑中的理论
\begin{enumerate}
\item 请验证\(\calb(T)=(B,+,\cdot,-,0,1)\)是一个布尔代数
\item \(T\)是一致的当且仅当\(\calb(T)\)是非平反的
\end{enumerate}
\end{exercise}

\begin{proof}
\begin{enumerate}
\item 下面验证\(\calb(T)\)满足布尔代数公理。对于任意公式\([\alpha],[\beta],[\gamma]\in B\)
\begin{enumerate}
\item ​
\begin{align*}
[\alpha]+([\beta]+[\gamma])&=[\alpha]+[\beta\vee\gamma]=[\alpha\vee(\beta\vee\gamma)]\\
&=[(\alpha\vee\beta)\vee\gamma]=[\alpha\vee\beta]+[\gamma]=([\alpha]+[\beta])+[\gamma]\\
[\alpha]\cdot([\beta]\cdot[\gamma])&=[\alpha]\cdot[\beta\wedge\gamma]=[\alpha\wedge(\beta\wedge\gamma)]\\
&=[(\alpha\wedge\beta)\wedge\gamma]=[\alpha\wedge\beta]\cdot[\gamma]=([\alpha]\cdot[\beta])\cdot[\gamma]
\end{align*}
\item ​
\begin{align*}
[\alpha]+[\beta]&=[\alpha\vee\beta]=[\beta\vee\alpha]=[\beta]+[\alpha]\\
[\alpha]\cdot[\beta]&=[\alpha\wedge\beta]=[\beta\wedge\alpha]=[\beta]\cdot[\alpha]
\end{align*}
\item ​
\begin{align*}
[\alpha]+([\alpha]\cdot[\beta])&=[\alpha]+[\alpha\wedge\beta]=[\alpha\vee(\alpha\wedge\beta)]=[\alpha]\\
[\alpha]\cdot([\alpha]+[\beta])&=[\alpha]\cdot[\alpha\vee\beta]=[\alpha\wedge(\alpha\vee\beta)]=[\alpha]
\end{align*}
\item ​
\begin{align*}
 [\alpha]\cdot([\beta]+[\gamma])&=[\alpha]\cdot[\beta\vee\gamma]=[\alpha\wedge(\beta\vee\gamma)]\\
 &=[(\alpha\wedge\beta)\vee(\alpha\wedge\gamma)]=([\alpha]\cdot[\beta])+([\alpha]\cdot[\gamma])\\
 [\alpha]+([\beta]\cdot[\gamma])&=[\alpha\vee(\beta\wedge\gamma)]=[(\alpha\vee\beta)\wedge(\alpha\vee\gamma)]\\
 &=([\alpha]+[\beta])\cdot([\alpha]+[\gamma])
\end{align*}
\item ​
\begin{align*}
[\alpha]+(-[\alpha])&=[\alpha]+[\neg\alpha]=[\alpha\vee\neg\alpha]=1\\
[\alpha]\cdot(-[\alpha])&=[\alpha]\cdot[\neg\alpha]=[\alpha\wedge\neg\alpha]=0
\end{align*}
\end{enumerate}
\item 若\(T\)不一致,则对于任意公式\(\alpha,\beta\),总有\(T\vdash\alpha\leftrightarrow\beta\),因此\(\abs{B}=1\),\(\calb(T)\)平凡

若\(\calb(T)\)是平凡的,于是对于任意公式\(\alpha\)与公式\(\beta\in T\),有\(T\vdash\beta\leftrightarrow\alpha\),于是\(T\vdash\alpha\)。因此\(T\)不一
致
\end{enumerate}
\end{proof}

\begin{exercise}[1.1.8]
令\(B=\{Y\subseteq X\mid Y\text{是有穷的或余有穷的}\}\),则\(X,\emptyset\in B\)。证明\(B\)对\(\cap,\cup,-\)封闭,所以\(\calb\)是一个
布尔代数,是一个集合代数
\end{exercise}

\begin{proof}
对于任意\(X,Y\in B\),
\begin{enumerate}
\item 若\(X,Y\)有穷,则\(X\cap Y\)有穷,\(X\cup Y\)有穷,\(-X\)余有穷
\item 若\(X\)有穷,\(Y\)余有穷,则\(X\cap Y\)余有穷,\(X\cup Y\)余有穷
\item 若\(X\)余有穷,\(Y\)有穷,则\(X\cap Y,X\cup Y\)余有穷,\(-X\)有穷
\item 若\(X,Y\)余有穷,则\(X\cap Y,X\cup Y\)余有穷
\end{enumerate}
\end{proof}

\begin{exercise}[1.1.9]
证明不存在基数为3的布尔代数。一个有穷的布尔代数,其基数需要满足什么条件?
\end{exercise}

\begin{proof}
若存在基数为3的布尔代数\(\calb\),则令\(B=\{0,1,a\}\)。

如果 \(-a=0\),那么 \(a+(-a)=a+0=a+(a\cdot(-a))=a\neq 1\),矛盾。
如果\(-a=1\),那么\(a\cdot(-a)=a\cdot 1=a\cdot(a+(-a))=a\neq 0\),矛盾。
如果\(-a=a\),那么\(a=a\cdot 1=a\cdot(a+a)=a\cdot a+a\cdot a=0+0=0\),矛盾。

因此不存在基数为3的布尔代数。

\(\abs{\calb}=2^n\) for some \(n\)
\end{proof}

\begin{lemma}[]
给定布尔代数\(\calb\),对于任意\(a,b,c\in B\)
\begin{enumerate}
\item \(a+0=a\)
\item \(a\cdot 1=a\)
\item 若\(a+b=1\)且\(a\cdot b=0\),则\(b=-a\)
\item \(-(a\cdot b)=(-a)+(-b)\)。\(-(a+b)=(-a)\cdot(-b)\)
\item \(--a=a\)
\end{enumerate}
\end{lemma}

\begin{proof}
\begin{enumerate}
\item \(a+0=a+(a\cdot(-a))=a\)
\item \(a\cdot 1=a\cdot(a+(-a))=a\)
\item \(b=b+0=b+a\cdot(-a)=(a+b)\cdot(b+(-a))=1\cdot(b+(-a))=(a+(-a))\cdot(b+(-a))=(-a)+(a\cdot b)=-a\)
\item \((-a)+(-b)+(a\cdot b)=ab+(-a)+(-b)\cdot(a+-a)=a(b+-b)+(-a)(1+-b)=a+(-a)=1\).

\((a\cdot b)\cdot(-a+-b)=ab(-a)+ab(-b)=0+0=0\).

同理可得\(-(a+b)=(-a)\cdot(-b)\)
\item 由3可知
\end{enumerate}
\end{proof}

\begin{exercise}[1.1.10]
\(1\to 2\): 定义

\(2\to 3\): 对任意\(a,b\in A\),  \(f(a\cdot b)=f(--(a\cdot b))=-f((-a)+(-b))=-(-f(a)+-f(b))=f(a)\cdot f(b)\)

\(3\to 2\): 对任意\(a,b\in A\),\(f(a+b)=f(--(a+b))=-f((-a)\cdot(-b))=f(a)+f(b)\)

\(2\to 4\): 对任意\(a,b\in A\), \(f(1)=f(a+(-a))=f(a)+-f(a)=1\), \(f(0)=f(-1)=-f(1)=1\)。若\(a\cdot b=0\),
则\(f(a\cdot b)=f(0)=0\)

\(4\to 2\): 对任意\(a\in A\) \(f(-a)\cdot f(a)=f(-a\cdot a)=0\), \(f(-a)+f(a)=f(-a+a)=f(1)=1\),因
此\(f(-a)=-f(a)\)

因此\(4\leftrightarrow 3\leftrightarrow 2\),而\(4\wedge 3\wedge 2\to 1\),因此\(2\to 1\)
\end{exercise}
\end{document}
