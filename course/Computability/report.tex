% Created 2022-09-21 Wed 23:31
% Intended LaTeX compiler: pdflatex
\documentclass[11pt]{article}
\usepackage[utf8]{inputenc}
\usepackage[T1]{fontenc}
\usepackage{graphicx}
\usepackage{longtable}
\usepackage{wrapfig}
\usepackage{rotating}
\usepackage[normalem]{ulem}
\usepackage{amsmath}
\usepackage{amssymb}
\usepackage{capt-of}
\usepackage{hyperref}
% wrong resolution of image
% https://tex.stackexchange.com/questions/21627/image-from-includegraphics-showing-in-wrong-image-size?rq=1

%%%%%%%%%%%%%%%%%%%%%%%%%%%%%%%%%%%%%%
%% TIPS                                 %%
%%%%%%%%%%%%%%%%%%%%%%%%%%%%%%%%%%%%%%
% \substack{a\\b} for multiple lines text
% \usepackage{expl3}
% \expandafter\def\csname ver@l3regex.sty\endcsname{}
% \usepackage{pkgloader}
\usepackage[utf8]{inputenc}

% nfss error
% \usepackage[B1,T1]{fontenc}
\usepackage{fontspec}

% \usepackage[Emoticons]{ucharclasses}
\newfontfamily\DejaSans{DejaVu Sans}
% \setDefaultTransitions{\DejaSans}{}

% pdfplots will load xolor automatically without option
\usepackage[dvipsnames]{xcolor}

%                                                             ┳┳┓   ┓
%                                                             ┃┃┃┏┓╋┣┓
%                                                             ┛ ┗┗┻┗┛┗
% \usepackage{amsmath} mathtools loads the amsmath
\usepackage{amsmath}
\usepackage{mathtools}

\usepackage{amsthm}
\usepackage{amsbsy}

%\usepackage{commath}

\usepackage{amssymb}

\usepackage{mathrsfs}
%\usepackage{mathabx}
\usepackage{stmaryrd}
\usepackage{empheq}

\usepackage{scalerel}
\usepackage{stackengine}
\usepackage{stackrel}



\usepackage{nicematrix}
\usepackage{tensor}
\usepackage{blkarray}
\usepackage{siunitx}
\usepackage[f]{esvect}

% centering \not on a letter
\usepackage{slashed}
\usepackage[makeroom]{cancel}

%\usepackage{merriweather}
\usepackage{unicode-math}
\setmainfont{TeX Gyre Pagella}
% \setmathfont{STIX}
%\setmathfont{texgyrepagella-math.otf}
%\setmathfont{Libertinus Math}
\setmathfont{Latin Modern Math}

 % \setmathfont[range={\smwhtdiamond,\enclosediamond,\varlrtriangle}]{Latin Modern Math}
\setmathfont[range={\rightrightarrows,\twoheadrightarrow,\leftrightsquigarrow,\triangledown,\vartriangle,\precneq,\succneq,\prec,\succ,\preceq,\succeq,\tieconcat}]{XITS Math}
 \setmathfont[range={\int,\setminus}]{Libertinus Math}
 % \setmathfont[range={\mathalpha}]{TeX Gyre Pagella Math}
%\setmathfont[range={\mitA,\mitB,\mitC,\mitD,\mitE,\mitF,\mitG,\mitH,\mitI,\mitJ,\mitK,\mitL,\mitM,\mitN,\mitO,\mitP,\mitQ,\mitR,\mitS,\mitT,\mitU,\mitV,\mitW,\mitX,\mitY,\mitZ,\mita,\mitb,\mitc,\mitd,\mite,\mitf,\mitg,\miti,\mitj,\mitk,\mitl,\mitm,\mitn,\mito,\mitp,\mitq,\mitr,\mits,\mitt,\mitu,\mitv,\mitw,\mitx,\mity,\mitz}]{TeX Gyre Pagella Math}
% unicode is not good at this!
%\let\nmodels\nvDash

 \usepackage{wasysym}

 % for wide hat
 \DeclareSymbolFont{yhlargesymbols}{OMX}{yhex}{m}{n} \DeclareMathAccent{\what}{\mathord}{yhlargesymbols}{"62}

%                                                               ┏┳┓•┓
%                                                                ┃ ┓┃┏┓
%                                                                ┻ ┗┛┗┗

\usepackage{pgfplots}
\pgfplotsset{compat=1.18}
\usepackage{tikz}
\usepackage{tikz-cd}
\tikzcdset{scale cd/.style={every label/.append style={scale=#1},
    cells={nodes={scale=#1}}}}
% TODO: discard qtree and use forest
% \usepackage{tikz-qtree}
\usepackage{forest}

\usetikzlibrary{arrows,positioning,calc,fadings,decorations,matrix,decorations,shapes.misc}
%setting from geogebra
\definecolor{ccqqqq}{rgb}{0.8,0,0}

%                                                          ┳┳┓•    ┓┓
%                                                          ┃┃┃┓┏┏┏┓┃┃┏┓┏┓┏┓┏┓┓┏┏
%                                                          ┛ ┗┗┛┗┗ ┗┗┗┻┛┗┗ ┗┛┗┻┛
%\usepackage{twemojis}
\usepackage[most]{tcolorbox}
\usepackage{threeparttable}
\usepackage{tabularx}

\usepackage{enumitem}
\usepackage[indLines=false]{algpseudocodex}
\usepackage[]{algorithm2e}
% \SetKwComment{Comment}{/* }{ */}
% \algrenewcommand\algorithmicrequire{\textbf{Input:}}
% \algrenewcommand\algorithmicensure{\textbf{Output:}}
% wrong with preview
\usepackage{subcaption}
\usepackage{caption}
% {\aunclfamily\Huge}
\usepackage{auncial}

\usepackage{float}

\usepackage{fancyhdr}

\usepackage{ifthen}
\usepackage{xargs}

\definecolor{mintedbg}{rgb}{0.99,0.99,0.99}
\usepackage[cachedir=\detokenize{~/miscellaneous/trash}]{minted}
\setminted{breaklines,
  mathescape,
  bgcolor=mintedbg,
  fontsize=\footnotesize,
  frame=single,
  linenos}
\usemintedstyle{xcode}
\usepackage{tcolorbox}
\usepackage{etoolbox}



\usepackage{imakeidx}
\usepackage{hyperref}
\usepackage{soul}
\usepackage{framed}

% don't use this for preview
%\usepackage[margin=1.5in]{geometry}
% \usepackage{geometry}
% \geometry{legalpaper, landscape, margin=1in}
\usepackage[font=itshape]{quoting}

%\LoadPackagesNow
%\usepackage[xetex]{preview}
%%%%%%%%%%%%%%%%%%%%%%%%%%%%%%%%%%%%%%%
%% USEPACKAGES end                       %%
%%%%%%%%%%%%%%%%%%%%%%%%%%%%%%%%%%%%%%%

%%%%%%%%%%%%%%%%%%%%%%%%%%%%%%%%%%%%%%%
%% Algorithm environment
%%%%%%%%%%%%%%%%%%%%%%%%%%%%%%%%%%%%%%%
\SetKwIF{Recv}{}{}{upon receiving}{do}{}{}{}
\SetKwBlock{Init}{initially do}{}
\SetKwProg{Function}{Function}{:}{}

% https://github.com/chrmatt/algpseudocodex/issues/3
\algnewcommand\algorithmicswitch{\textbf{switch}}%
\algnewcommand\algorithmiccase{\textbf{case}}
\algnewcommand\algorithmicof{\textbf{of}}
\algnewcommand\algorithmicotherwise{\texttt{otherwise} $\Rightarrow$}

\makeatletter
\algdef{SE}[SWITCH]{Switch}{EndSwitch}[1]{\algpx@startIndent\algpx@startCodeCommand\algorithmicswitch\ #1\ \algorithmicdo}{\algpx@endIndent\algpx@startCodeCommand\algorithmicend\ \algorithmicswitch}%
\algdef{SE}[CASE]{Case}{EndCase}[1]{\algpx@startIndent\algpx@startCodeCommand\algorithmiccase\ #1}{\algpx@endIndent\algpx@startCodeCommand\algorithmicend\ \algorithmiccase}%
\algdef{SE}[CASEOF]{CaseOf}{EndCaseOf}[1]{\algpx@startIndent\algpx@startCodeCommand\algorithmiccase\ #1 \algorithmicof}{\algpx@endIndent\algpx@startCodeCommand\algorithmicend\ \algorithmiccase}
\algdef{SE}[OTHERWISE]{Otherwise}{EndOtherwise}[0]{\algpx@startIndent\algpx@startCodeCommand\algorithmicotherwise}{\algpx@endIndent\algpx@startCodeCommand\algorithmicend\ \algorithmicotherwise}
\ifbool{algpx@noEnd}{%
  \algtext*{EndSwitch}%
  \algtext*{EndCase}%
  \algtext*{EndCaseOf}
  \algtext*{EndOtherwise}
  %
  % end indent line after (not before), to get correct y position for multiline text in last command
  \apptocmd{\EndSwitch}{\algpx@endIndent}{}{}%
  \apptocmd{\EndCase}{\algpx@endIndent}{}{}%
  \apptocmd{\EndCaseOf}{\algpx@endIndent}{}{}
  \apptocmd{\EndOtherwise}{\algpx@endIndent}{}{}
}{}%

\pretocmd{\Switch}{\algpx@endCodeCommand}{}{}
\pretocmd{\Case}{\algpx@endCodeCommand}{}{}
\pretocmd{\CaseOf}{\algpx@endCodeCommand}{}{}
\pretocmd{\Otherwise}{\algpx@endCodeCommand}{}{}

% for end commands that may not be printed, tell endCodeCommand whether we are using noEnd
\ifbool{algpx@noEnd}{%
  \pretocmd{\EndSwitch}{\algpx@endCodeCommand[1]}{}{}%
  \pretocmd{\EndCase}{\algpx@endCodeCommand[1]}{}{}
  \pretocmd{\EndCaseOf}{\algpx@endCodeCommand[1]}{}{}%
  \pretocmd{\EndOtherwise}{\algpx@endCodeCommand[1]}{}{}
}{%
  \pretocmd{\EndSwitch}{\algpx@endCodeCommand[0]}{}{}%
  \pretocmd{\EndCase}{\algpx@endCodeCommand[0]}{}{}%
  \pretocmd{\EndCaseOf}{\algpx@endCodeCommand[0]}{}{}
  \pretocmd{\EndOtherwise}{\algpx@endCodeCommand[0]}{}{}
}%
\makeatother
% % For algpseudocode
% \algnewcommand\algorithmicswitch{\textbf{switch}}
% \algnewcommand\algorithmiccase{\textbf{case}}
% \algnewcommand\algorithmiccaseof{\textbf{case}}
% \algnewcommand\algorithmicof{\textbf{of}}
% % New "environments"
% \algdef{SE}[SWITCH]{Switch}{EndSwitch}[1]{\algorithmicswitch\ #1\ \algorithmicdo}{\algorithmicend\ \algorithmicswitch}%
% \algdef{SE}[CASE]{Case}{EndCase}[1]{\algorithmiccase\ #1}{\algorithmicend\ \algorithmiccase}%
% \algtext*{EndSwitch}%
% \algtext*{EndCase}
% \algdef{SE}[CASEOF]{CaseOf}{EndCaseOf}[1]{\algorithmiccaseof\ #1 \algorithmicof}{\algorithmicend\ \algorithmiccaseof}
% \algtext*{EndCaseOf}



%\pdfcompresslevel0

% quoting from
% https://tex.stackexchange.com/questions/391726/the-quotation-environment
\NewDocumentCommand{\bywhom}{m}{% the Bourbaki trick
  {\nobreak\hfill\penalty50\hskip1em\null\nobreak
   \hfill\mbox{\normalfont(#1)}%
   \parfillskip=0pt \finalhyphendemerits=0 \par}%
}

\NewDocumentEnvironment{pquotation}{m}
  {\begin{quoting}[
     indentfirst=true,
     leftmargin=\parindent,
     rightmargin=\parindent]\itshape}
  {\bywhom{#1}\end{quoting}}

\indexsetup{othercode=\small}
\makeindex[columns=2,options={-s /media/wu/file/stuuudy/notes/index_style.ist},intoc]
\makeatletter
\def\@idxitem{\par\hangindent 0pt}
\makeatother


% \newcounter{dummy} \numberwithin{dummy}{section}
\newtheorem{dummy}{dummy}[section]
\theoremstyle{definition}
\newtheorem{definition}[dummy]{Definition}
\theoremstyle{plain}
\newtheorem{corollary}[dummy]{Corollary}
\newtheorem{lemma}[dummy]{Lemma}
\newtheorem{proposition}[dummy]{Proposition}
\newtheorem{theorem}[dummy]{Theorem}
\newtheorem{notation}[dummy]{Notation}
\newtheorem{conjecture}[dummy]{Conjecture}
\newtheorem{fact}[dummy]{Fact}
\newtheorem{warning}[dummy]{Warning}
\theoremstyle{definition}
\newtheorem{examplle}{Example}[section]
\theoremstyle{remark}
\newtheorem*{remark}{Remark}
\newtheorem{exercise}{Exercise}[subsection]
\newtheorem{problem}{Problem}[subsection]
\newtheorem{observation}{Observation}[section]
\newenvironment{claim}[1]{\par\noindent\textbf{Claim:}\space#1}{}

\makeatletter
\DeclareFontFamily{U}{tipa}{}
\DeclareFontShape{U}{tipa}{m}{n}{<->tipa10}{}
\newcommand{\arc@char}{{\usefont{U}{tipa}{m}{n}\symbol{62}}}%

\newcommand{\arc}[1]{\mathpalette\arc@arc{#1}}

\newcommand{\arc@arc}[2]{%
  \sbox0{$\m@th#1#2$}%
  \vbox{
    \hbox{\resizebox{\wd0}{\height}{\arc@char}}
    \nointerlineskip
    \box0
  }%
}
\makeatother

\setcounter{MaxMatrixCols}{20}
%%%%%%% ABS
\DeclarePairedDelimiter\abss{\lvert}{\rvert}%
\DeclarePairedDelimiter\normm{\lVert}{\rVert}%

% Swap the definition of \abs* and \norm*, so that \abs
% and \norm resizes the size of the brackets, and the
% starred version does not.
\makeatletter
\let\oldabs\abss
%\def\abs{\@ifstar{\oldabs}{\oldabs*}}
\newcommand{\abs}{\@ifstar{\oldabs}{\oldabs*}}
\newcommand{\norm}[1]{\left\lVert#1\right\rVert}
%\let\oldnorm\normm
%\def\norm{\@ifstar{\oldnorm}{\oldnorm*}}
%\renewcommand{norm}{\@ifstar{\oldnorm}{\oldnorm*}}
\makeatother

% \stackMath
% \newcommand\what[1]{%
% \savestack{\tmpbox}{\stretchto{%
%   \scaleto{%
%     \scalerel*[\widthof{\ensuremath{#1}}]{\kern-.6pt\bigwedge\kern-.6pt}%
%     {\rule[-\textheight/2]{1ex}{\textheight}}%WIDTH-LIMITED BIG WEDGE
%   }{\textheight}%
% }{0.5ex}}%
% \stackon[1pt]{#1}{\tmpbox}%
% }

% \newcommand\what[1]{\ThisStyle{%
%     \setbox0=\hbox{$\SavedStyle#1$}%
%     \stackengine{-1.0\ht0+.5pt}{$\SavedStyle#1$}{%
%       \stretchto{\scaleto{\SavedStyle\mkern.15mu\char'136}{2.6\wd0}}{1.4\ht0}%
%     }{O}{c}{F}{T}{S}%
%   }
% }

% \newcommand\wtilde[1]{\ThisStyle{%
%     \setbox0=\hbox{$\SavedStyle#1$}%
%     \stackengine{-.1\LMpt}{$\SavedStyle#1$}{%
%       \stretchto{\scaleto{\SavedStyle\mkern.2mu\AC}{.5150\wd0}}{.6\ht0}%
%     }{O}{c}{F}{T}{S}%
%   }
% }

% \newcommand\wbar[1]{\ThisStyle{%
%     \setbox0=\hbox{$\SavedStyle#1$}%
%     \stackengine{.5pt+\LMpt}{$\SavedStyle#1$}{%
%       \rule{\wd0}{\dimexpr.3\LMpt+.3pt}%
%     }{O}{c}{F}{T}{S}%
%   }
% }

\newcommand{\bl}[1] {\boldsymbol{#1}}
\newcommand{\Wt}[1] {\stackrel{\sim}{\smash{#1}\rule{0pt}{1.1ex}}}
\newcommand{\wt}[1] {\widetilde{#1}}
\newcommand{\tf}[1] {\textbf{#1}}

\newcommand{\wu}[1]{{\color{red} #1}}

%For boxed texts in align, use Aboxed{}
%otherwise use boxed{}

\DeclareMathSymbol{\widehatsym}{\mathord}{largesymbols}{"62}
\newcommand\lowerwidehatsym{%
  \text{\smash{\raisebox{-1.3ex}{%
    $\widehatsym$}}}}
\newcommand\fixwidehat[1]{%
  \mathchoice
    {\accentset{\displaystyle\lowerwidehatsym}{#1}}
    {\accentset{\textstyle\lowerwidehatsym}{#1}}
    {\accentset{\scriptstyle\lowerwidehatsym}{#1}}
    {\accentset{\scriptscriptstyle\lowerwidehatsym}{#1}}
  }


\newcommand{\cupdot}{\mathbin{\dot{\cup}}}
\newcommand{\bigcupdot}{\mathop{\dot{\bigcup}}}

\usepackage{graphicx}

\usepackage[toc,page]{appendix}

% text on arrow for xRightarrow
\makeatletter
%\newcommand{\xRightarrow}[2][]{\ext@arrow 0359\Rightarrowfill@{#1}{#2}}
\makeatother

% Arbitrary long arrow
\newcommand{\Rarrow}[1]{%
\parbox{#1}{\tikz{\draw[->](0,0)--(#1,0);}}
}

\newcommand{\LRarrow}[1]{%
\parbox{#1}{\tikz{\draw[<->](0,0)--(#1,0);}}
}


\makeatletter
\providecommand*{\rmodels}{%
  \mathrel{%
    \mathpalette\@rmodels\models
  }%
}
\newcommand*{\@rmodels}[2]{%
  \reflectbox{$\m@th#1#2$}%
}
\makeatother

% Roman numerals
\makeatletter
\newcommand*{\rom}[1]{\expandafter\@slowromancap\romannumeral #1@}
\makeatother
% \\def \\b\([a-zA-Z]\) {\\boldsymbol{[a-zA-z]}}
% \\DeclareMathOperator{\\b\1}{\\textbf{\1}}

\DeclareMathOperator*{\argmin}{arg\,min}
\DeclareMathOperator*{\argmax}{arg\,max}

\DeclareMathOperator{\bone}{\textbf{1}}
\DeclareMathOperator{\bx}{\textbf{x}}
\DeclareMathOperator{\bz}{\textbf{z}}
\DeclareMathOperator{\bff}{\textbf{f}}
\DeclareMathOperator{\ba}{\textbf{a}}
\DeclareMathOperator{\bk}{\textbf{k}}
\DeclareMathOperator{\bs}{\textbf{s}}
\DeclareMathOperator{\bh}{\textbf{h}}
\DeclareMathOperator{\bc}{\textbf{c}}
\DeclareMathOperator{\br}{\textbf{r}}
\DeclareMathOperator{\bi}{\textbf{i}}
\DeclareMathOperator{\bj}{\textbf{j}}
\DeclareMathOperator{\bn}{\textbf{n}}
\DeclareMathOperator{\be}{\textbf{e}}
\DeclareMathOperator{\bo}{\textbf{o}}
\DeclareMathOperator{\bU}{\textbf{U}}
\DeclareMathOperator{\bL}{\textbf{L}}
\DeclareMathOperator{\bV}{\textbf{V}}
\def \bzero {\mathbf{0}}
\def \bbone {\mathbb{1}}
\def \btwo {\mathbf{2}}
\DeclareMathOperator{\bv}{\textbf{v}}
\DeclareMathOperator{\bp}{\textbf{p}}
\DeclareMathOperator{\bI}{\textbf{I}}
\def \dbI {\dot{\bI}}
\DeclareMathOperator{\bM}{\textbf{M}}
\DeclareMathOperator{\bN}{\textbf{N}}
\DeclareMathOperator{\bK}{\textbf{K}}
\DeclareMathOperator{\bt}{\textbf{t}}
\DeclareMathOperator{\bb}{\textbf{b}}
\DeclareMathOperator{\bA}{\textbf{A}}
\DeclareMathOperator{\bX}{\textbf{X}}
\DeclareMathOperator{\bu}{\textbf{u}}
\DeclareMathOperator{\bS}{\textbf{S}}
\DeclareMathOperator{\bZ}{\textbf{Z}}
\DeclareMathOperator{\bJ}{\textbf{J}}
\DeclareMathOperator{\by}{\textbf{y}}
\DeclareMathOperator{\bw}{\textbf{w}}
\DeclareMathOperator{\bT}{\textbf{T}}
\DeclareMathOperator{\bF}{\textbf{F}}
\DeclareMathOperator{\bmm}{\textbf{m}}
\DeclareMathOperator{\bW}{\textbf{W}}
\DeclareMathOperator{\bR}{\textbf{R}}
\DeclareMathOperator{\bC}{\textbf{C}}
\DeclareMathOperator{\bD}{\textbf{D}}
\DeclareMathOperator{\bE}{\textbf{E}}
\DeclareMathOperator{\bQ}{\textbf{Q}}
\DeclareMathOperator{\bP}{\textbf{P}}
\DeclareMathOperator{\bY}{\textbf{Y}}
\DeclareMathOperator{\bH}{\textbf{H}}
\DeclareMathOperator{\bB}{\textbf{B}}
\DeclareMathOperator{\bG}{\textbf{G}}
\def \blambda {\symbf{\lambda}}
\def \boldeta {\symbf{\eta}}
\def \balpha {\symbf{\alpha}}
\def \btau {\symbf{\tau}}
\def \bbeta {\symbf{\beta}}
\def \bgamma {\symbf{\gamma}}
\def \bxi {\symbf{\xi}}
\def \bLambda {\symbf{\Lambda}}
\def \bGamma {\symbf{\Gamma}}

\newcommand{\bto}{{\boldsymbol{\to}}}
\newcommand{\Ra}{\Rightarrow}
\newcommand{\xrsa}[1]{\overset{#1}{\rightsquigarrow}}
\newcommand{\xlsa}[1]{\overset{#1}{\leftsquigarrow}}
\newcommand\und[1]{\underline{#1}}
\newcommand\ove[1]{\overline{#1}}
%\def \concat {\verb|^|}
\def \bPhi {\mbfPhi}
\def \btheta {\mbftheta}
\def \bTheta {\mbfTheta}
\def \bmu {\mbfmu}
\def \bphi {\mbfphi}
\def \bSigma {\mbfSigma}
\def \la {\langle}
\def \ra {\rangle}

\def \caln {\mathcal{N}}
\def \dissum {\displaystyle\Sigma}
\def \dispro {\displaystyle\prod}

\def \caret {\verb!^!}

\def \A {\mathbb{A}}
\def \B {\mathbb{B}}
\def \C {\mathbb{C}}
\def \D {\mathbb{D}}
\def \E {\mathbb{E}}
\def \F {\mathbb{F}}
\def \G {\mathbb{G}}
\def \H {\mathbb{H}}
\def \I {\mathbb{I}}
\def \J {\mathbb{J}}
\def \K {\mathbb{K}}
\def \L {\mathbb{L}}
\def \M {\mathbb{M}}
\def \N {\mathbb{N}}
\def \O {\mathbb{O}}
\def \P {\mathbb{P}}
\def \Q {\mathbb{Q}}
\def \R {\mathbb{R}}
\def \S {\mathbb{S}}
\def \T {\mathbb{T}}
\def \U {\mathbb{U}}
\def \V {\mathbb{V}}
\def \W {\mathbb{W}}
\def \X {\mathbb{X}}
\def \Y {\mathbb{Y}}
\def \Z {\mathbb{Z}}

\def \cala {\mathcal{A}}
\def \cale {\mathcal{E}}
\def \calb {\mathcal{B}}
\def \calq {\mathcal{Q}}
\def \calp {\mathcal{P}}
\def \cals {\mathcal{S}}
\def \calx {\mathcal{X}}
\def \caly {\mathcal{Y}}
\def \calg {\mathcal{G}}
\def \cald {\mathcal{D}}
\def \caln {\mathcal{N}}
\def \calr {\mathcal{R}}
\def \calt {\mathcal{T}}
\def \calm {\mathcal{M}}
\def \calw {\mathcal{W}}
\def \calc {\mathcal{C}}
\def \calv {\mathcal{V}}
\def \calf {\mathcal{F}}
\def \calk {\mathcal{K}}
\def \call {\mathcal{L}}
\def \calu {\mathcal{U}}
\def \calo {\mathcal{O}}
\def \calh {\mathcal{H}}
\def \cali {\mathcal{I}}
\def \calj {\mathcal{J}}

\def \bcup {\bigcup}

% set theory

\def \zfcc {\textbf{ZFC}^-}
\def \BGC {\textbf{BGC}}
\def \BG {\textbf{BG}}
\def \ac  {\textbf{AC}}
\def \gl  {\textbf{L }}
\def \gll {\textbf{L}}
\newcommand{\zfm}{$\textbf{ZF}^-$}

\def \ZFm {\text{ZF}^-}
\def \ZFCm {\text{ZFC}^-}
\DeclareMathOperator{\WF}{WF}
\DeclareMathOperator{\On}{On}
\def \on {\textbf{On }}
\def \cm {\textbf{M }}
\def \cn {\textbf{N }}
\def \cv {\textbf{V }}
\def \zc {\textbf{ZC }}
\def \zcm {\textbf{ZC}}
\def \zff {\textbf{ZF}}
\def \wfm {\textbf{WF}}
\def \onm {\textbf{On}}
\def \cmm {\textbf{M}}
\def \cnm {\textbf{N}}
\def \cvm {\textbf{V}}

\renewcommand{\restriction}{\mathord{\upharpoonright}}
%% another restriction
\newcommand\restr[2]{{% we make the whole thing an ordinary symbol
  \left.\kern-\nulldelimiterspace % automatically resize the bar with \right
  #1 % the function
  \vphantom{\big|} % pretend it's a little taller at normal size
  \right|_{#2} % this is the delimiter
  }}

\def \pred {\text{pred}}

\def \rank {\text{rank}}
\def \Con {\text{Con}}
\def \deff {\text{Def}}


\def \uin {\underline{\in}}
\def \oin {\overline{\in}}
\def \uR {\underline{R}}
\def \oR {\overline{R}}
\def \uP {\underline{P}}
\def \oP {\overline{P}}

\def \dsum {\displaystyle\sum}

\def \Ra {\Rightarrow}

\def \e {\enspace}

\def \sgn {\operatorname{sgn}}
\def \gen {\operatorname{gen}}
\def \Hom {\operatorname{Hom}}
\def \hom {\operatorname{hom}}
\def \Sub {\operatorname{Sub}}

\def \supp {\operatorname{supp}}

\def \epiarrow {\twoheadarrow}
\def \monoarrow {\rightarrowtail}
\def \rrarrow {\rightrightarrows}

% \def \minus {\text{-}}
% \newcommand{\minus}{\scalebox{0.75}[1.0]{$-$}}
% \DeclareUnicodeCharacter{002D}{\minus}


\def \tril {\triangleleft}

\def \ISigma {\text{I}\Sigma}
\def \IDelta {\text{I}\Delta}
\def \IPi {\text{I}\Pi}
\def \ACF {\textsf{ACF}}
\def \pCF {\textit{p}\text{CF}}
\def \ACVF {\textsf{ACVF}}
\def \HLR {\textsf{HLR}}
\def \OAG {\textsf{OAG}}
\def \RCF {\textsf{RCF}}
\DeclareMathOperator{\GL}{GL}
\DeclareMathOperator{\PGL}{PGL}
\DeclareMathOperator{\SL}{SL}
\DeclareMathOperator{\Inv}{Inv}
\DeclareMathOperator{\res}{res}
\DeclareMathOperator{\Sym}{Sym}
%\DeclareMathOperator{\char}{char}
\def \equal {=}

\def \degree {\text{degree}}
\def \app {\text{App}}
\def \FV {\text{FV}}
\def \conv {\text{conv}}
\def \cont {\text{cont}}
\DeclareMathOperator{\cl}{\text{cl}}
\DeclareMathOperator{\trcl}{\text{trcl}}
\DeclareMathOperator{\sg}{sg}
\DeclareMathOperator{\trdeg}{trdeg}
\def \Ord {\text{Ord}}

\DeclareMathOperator{\cf}{cf}
\DeclareMathOperator{\zfc}{ZFC}

%\DeclareMathOperator{\Th}{Th}
%\def \th {\text{Th}}
% \newcommand{\th}{\text{Th}}
\DeclareMathOperator{\type}{type}
\DeclareMathOperator{\zf}{\textbf{ZF}}
\def \fa {\mathfrak{a}}
\def \fb {\mathfrak{b}}
\def \fc {\mathfrak{c}}
\def \fd {\mathfrak{d}}
\def \fe {\mathfrak{e}}
\def \ff {\mathfrak{f}}
\def \fg {\mathfrak{g}}
\def \fh {\mathfrak{h}}
%\def \fi {\mathfrak{i}}
\def \fj {\mathfrak{j}}
\def \fk {\mathfrak{k}}
\def \fl {\mathfrak{l}}
\def \fm {\mathfrak{m}}
\def \fn {\mathfrak{n}}
\def \fo {\mathfrak{o}}
\def \fp {\mathfrak{p}}
\def \fq {\mathfrak{q}}
\def \fr {\mathfrak{r}}
\def \fs {\mathfrak{s}}
\def \ft {\mathfrak{t}}
\def \fu {\mathfrak{u}}
\def \fv {\mathfrak{v}}
\def \fw {\mathfrak{w}}
\def \fx {\mathfrak{x}}
\def \fy {\mathfrak{y}}
\def \fz {\mathfrak{z}}
\def \fA {\mathfrak{A}}
\def \fB {\mathfrak{B}}
\def \fC {\mathfrak{C}}
\def \fD {\mathfrak{D}}
\def \fE {\mathfrak{E}}
\def \fF {\mathfrak{F}}
\def \fG {\mathfrak{G}}
\def \fH {\mathfrak{H}}
\def \fI {\mathfrak{I}}
\def \fJ {\mathfrak{J}}
\def \fK {\mathfrak{K}}
\def \fL {\mathfrak{L}}
\def \fM {\mathfrak{M}}
\def \fN {\mathfrak{N}}
\def \fO {\mathfrak{O}}
\def \fP {\mathfrak{P}}
\def \fQ {\mathfrak{Q}}
\def \fR {\mathfrak{R}}
\def \fS {\mathfrak{S}}
\def \fT {\mathfrak{T}}
\def \fU {\mathfrak{U}}
\def \fV {\mathfrak{V}}
\def \fW {\mathfrak{W}}
\def \fX {\mathfrak{X}}
\def \fY {\mathfrak{Y}}
\def \fZ {\mathfrak{Z}}

\def \sfA {\textsf{A}}
\def \sfB {\textsf{B}}
\def \sfC {\textsf{C}}
\def \sfD {\textsf{D}}
\def \sfE {\textsf{E}}
\def \sfF {\textsf{F}}
\def \sfG {\textsf{G}}
\def \sfH {\textsf{H}}
\def \sfI {\textsf{I}}
\def \sfJ {\textsf{J}}
\def \sfK {\textsf{K}}
\def \sfL {\textsf{L}}
\def \sfM {\textsf{M}}
\def \sfN {\textsf{N}}
\def \sfO {\textsf{O}}
\def \sfP {\textsf{P}}
\def \sfQ {\textsf{Q}}
\def \sfR {\textsf{R}}
\def \sfS {\textsf{S}}
\def \sfT {\textsf{T}}
\def \sfU {\textsf{U}}
\def \sfV {\textsf{V}}
\def \sfW {\textsf{W}}
\def \sfX {\textsf{X}}
\def \sfY {\textsf{Y}}
\def \sfZ {\textsf{Z}}
\def \sfa {\textsf{a}}
\def \sfb {\textsf{b}}
\def \sfc {\textsf{c}}
\def \sfd {\textsf{d}}
\def \sfe {\textsf{e}}
\def \sff {\textsf{f}}
\def \sfg {\textsf{g}}
\def \sfh {\textsf{h}}
\def \sfi {\textsf{i}}
\def \sfj {\textsf{j}}
\def \sfk {\textsf{k}}
\def \sfl {\textsf{l}}
\def \sfm {\textsf{m}}
\def \sfn {\textsf{n}}
\def \sfo {\textsf{o}}
\def \sfp {\textsf{p}}
\def \sfq {\textsf{q}}
\def \sfr {\textsf{r}}
\def \sfs {\textsf{s}}
\def \sft {\textsf{t}}
\def \sfu {\textsf{u}}
\def \sfv {\textsf{v}}
\def \sfw {\textsf{w}}
\def \sfx {\textsf{x}}
\def \sfy {\textsf{y}}
\def \sfz {\textsf{z}}

\def \ttA {\texttt{A}}
\def \ttB {\texttt{B}}
\def \ttC {\texttt{C}}
\def \ttD {\texttt{D}}
\def \ttE {\texttt{E}}
\def \ttF {\texttt{F}}
\def \ttG {\texttt{G}}
\def \ttH {\texttt{H}}
\def \ttI {\texttt{I}}
\def \ttJ {\texttt{J}}
\def \ttK {\texttt{K}}
\def \ttL {\texttt{L}}
\def \ttM {\texttt{M}}
\def \ttN {\texttt{N}}
\def \ttO {\texttt{O}}
\def \ttP {\texttt{P}}
\def \ttQ {\texttt{Q}}
\def \ttR {\texttt{R}}
\def \ttS {\texttt{S}}
\def \ttT {\texttt{T}}
\def \ttU {\texttt{U}}
\def \ttV {\texttt{V}}
\def \ttW {\texttt{W}}
\def \ttX {\texttt{X}}
\def \ttY {\texttt{Y}}
\def \ttZ {\texttt{Z}}
\def \tta {\texttt{a}}
\def \ttb {\texttt{b}}
\def \ttc {\texttt{c}}
\def \ttd {\texttt{d}}
\def \tte {\texttt{e}}
\def \ttf {\texttt{f}}
\def \ttg {\texttt{g}}
\def \tth {\texttt{h}}
\def \tti {\texttt{i}}
\def \ttj {\texttt{j}}
\def \ttk {\texttt{k}}
\def \ttl {\texttt{l}}
\def \ttm {\texttt{m}}
\def \ttn {\texttt{n}}
\def \tto {\texttt{o}}
\def \ttp {\texttt{p}}
\def \ttq {\texttt{q}}
\def \ttr {\texttt{r}}
\def \tts {\texttt{s}}
\def \ttt {\texttt{t}}
\def \ttu {\texttt{u}}
\def \ttv {\texttt{v}}
\def \ttw {\texttt{w}}
\def \ttx {\texttt{x}}
\def \tty {\texttt{y}}
\def \ttz {\texttt{z}}

\def \bara {\bbar{a}}
\def \barb {\bbar{b}}
\def \barc {\bbar{c}}
\def \bard {\bbar{d}}
\def \bare {\bbar{e}}
\def \barf {\bbar{f}}
\def \barg {\bbar{g}}
\def \barh {\bbar{h}}
\def \bari {\bbar{i}}
\def \barj {\bbar{j}}
\def \bark {\bbar{k}}
\def \barl {\bbar{l}}
\def \barm {\bbar{m}}
\def \barn {\bbar{n}}
\def \baro {\bbar{o}}
\def \barp {\bbar{p}}
\def \barq {\bbar{q}}
\def \barr {\bbar{r}}
\def \bars {\bbar{s}}
\def \bart {\bbar{t}}
\def \baru {\bbar{u}}
\def \barv {\bbar{v}}
\def \barw {\bbar{w}}
\def \barx {\bbar{x}}
\def \bary {\bbar{y}}
\def \barz {\bbar{z}}
\def \barA {\bbar{A}}
\def \barB {\bbar{B}}
\def \barC {\bbar{C}}
\def \barD {\bbar{D}}
\def \barE {\bbar{E}}
\def \barF {\bbar{F}}
\def \barG {\bbar{G}}
\def \barH {\bbar{H}}
\def \barI {\bbar{I}}
\def \barJ {\bbar{J}}
\def \barK {\bbar{K}}
\def \barL {\bbar{L}}
\def \barM {\bbar{M}}
\def \barN {\bbar{N}}
\def \barO {\bbar{O}}
\def \barP {\bbar{P}}
\def \barQ {\bbar{Q}}
\def \barR {\bbar{R}}
\def \barS {\bbar{S}}
\def \barT {\bbar{T}}
\def \barU {\bbar{U}}
\def \barVV {\bbar{V}}
\def \barW {\bbar{W}}
\def \barX {\bbar{X}}
\def \barY {\bbar{Y}}
\def \barZ {\bbar{Z}}

\def \baralpha {\bbar{\alpha}}
\def \bartau {\bbar{\tau}}
\def \barsigma {\bbar{\sigma}}
\def \barzeta {\bbar{\zeta}}

\def \hata {\hat{a}}
\def \hatb {\hat{b}}
\def \hatc {\hat{c}}
\def \hatd {\hat{d}}
\def \hate {\hat{e}}
\def \hatf {\hat{f}}
\def \hatg {\hat{g}}
\def \hath {\hat{h}}
\def \hati {\hat{i}}
\def \hatj {\hat{j}}
\def \hatk {\hat{k}}
\def \hatl {\hat{l}}
\def \hatm {\hat{m}}
\def \hatn {\hat{n}}
\def \hato {\hat{o}}
\def \hatp {\hat{p}}
\def \hatq {\hat{q}}
\def \hatr {\hat{r}}
\def \hats {\hat{s}}
\def \hatt {\hat{t}}
\def \hatu {\hat{u}}
\def \hatv {\hat{v}}
\def \hatw {\hat{w}}
\def \hatx {\hat{x}}
\def \haty {\hat{y}}
\def \hatz {\hat{z}}
\def \hatA {\hat{A}}
\def \hatB {\hat{B}}
\def \hatC {\hat{C}}
\def \hatD {\hat{D}}
\def \hatE {\hat{E}}
\def \hatF {\hat{F}}
\def \hatG {\hat{G}}
\def \hatH {\hat{H}}
\def \hatI {\hat{I}}
\def \hatJ {\hat{J}}
\def \hatK {\hat{K}}
\def \hatL {\hat{L}}
\def \hatM {\hat{M}}
\def \hatN {\hat{N}}
\def \hatO {\hat{O}}
\def \hatP {\hat{P}}
\def \hatQ {\hat{Q}}
\def \hatR {\hat{R}}
\def \hatS {\hat{S}}
\def \hatT {\hat{T}}
\def \hatU {\hat{U}}
\def \hatVV {\hat{V}}
\def \hatW {\hat{W}}
\def \hatX {\hat{X}}
\def \hatY {\hat{Y}}
\def \hatZ {\hat{Z}}

\def \hatphi {\hat{\phi}}

\def \barfM {\bbar{\fM}}
\def \barfN {\bbar{\fN}}

\def \tila {\tilde{a}}
\def \tilb {\tilde{b}}
\def \tilc {\tilde{c}}
\def \tild {\tilde{d}}
\def \tile {\tilde{e}}
\def \tilf {\tilde{f}}
\def \tilg {\tilde{g}}
\def \tilh {\tilde{h}}
\def \tili {\tilde{i}}
\def \tilj {\tilde{j}}
\def \tilk {\tilde{k}}
\def \till {\tilde{l}}
\def \tilm {\tilde{m}}
\def \tiln {\tilde{n}}
\def \tilo {\tilde{o}}
\def \tilp {\tilde{p}}
\def \tilq {\tilde{q}}
\def \tilr {\tilde{r}}
\def \tils {\tilde{s}}
\def \tilt {\tilde{t}}
\def \tilu {\tilde{u}}
\def \tilv {\tilde{v}}
\def \tilw {\tilde{w}}
\def \tilx {\tilde{x}}
\def \tily {\tilde{y}}
\def \tilz {\tilde{z}}
\def \tilA {\tilde{A}}
\def \tilB {\tilde{B}}
\def \tilC {\tilde{C}}
\def \tilD {\tilde{D}}
\def \tilE {\tilde{E}}
\def \tilF {\tilde{F}}
\def \tilG {\tilde{G}}
\def \tilH {\tilde{H}}
\def \tilI {\tilde{I}}
\def \tilJ {\tilde{J}}
\def \tilK {\tilde{K}}
\def \tilL {\tilde{L}}
\def \tilM {\tilde{M}}
\def \tilN {\tilde{N}}
\def \tilO {\tilde{O}}
\def \tilP {\tilde{P}}
\def \tilQ {\tilde{Q}}
\def \tilR {\tilde{R}}
\def \tilS {\tilde{S}}
\def \tilT {\tilde{T}}
\def \tilU {\tilde{U}}
\def \tilVV {\tilde{V}}
\def \tilW {\tilde{W}}
\def \tilX {\tilde{X}}
\def \tilY {\tilde{Y}}
\def \tilZ {\tilde{Z}}

\def \tilalpha {\tilde{\alpha}}
\def \tilPhi {\tilde{\Phi}}

\def \barnu {\bar{\nu}}
\def \barrho {\bar{\rho}}
%\DeclareMathOperator{\ker}{ker}
\DeclareMathOperator{\im}{im}

\DeclareMathOperator{\Inn}{Inn}
\DeclareMathOperator{\rel}{rel}
\def \dote {\stackrel{\cdot}=}
%\DeclareMathOperator{\AC}{\textbf{AC}}
\DeclareMathOperator{\cod}{cod}
\DeclareMathOperator{\dom}{dom}
\DeclareMathOperator{\card}{card}
\DeclareMathOperator{\ran}{ran}
\DeclareMathOperator{\textd}{d}
\DeclareMathOperator{\td}{d}
\DeclareMathOperator{\id}{id}
\DeclareMathOperator{\LT}{LT}
\DeclareMathOperator{\Mat}{Mat}
\DeclareMathOperator{\Eq}{Eq}
\DeclareMathOperator{\irr}{irr}
\DeclareMathOperator{\Fr}{Fr}
\DeclareMathOperator{\Gal}{Gal}
\DeclareMathOperator{\lcm}{lcm}
\DeclareMathOperator{\alg}{\text{alg}}
\DeclareMathOperator{\Th}{Th}
%\DeclareMathOperator{\deg}{deg}


% \varprod
\DeclareSymbolFont{largesymbolsA}{U}{txexa}{m}{n}
\DeclareMathSymbol{\varprod}{\mathop}{largesymbolsA}{16}
% \DeclareMathSymbol{\tonm}{\boldsymbol{\to}\textbf{Nm}}
\def \tonm {\bto\textbf{Nm}}
\def \tohm {\bto\textbf{Hm}}

% Category theory
\DeclareMathOperator{\ob}{ob}
\DeclareMathOperator{\Ab}{\textbf{Ab}}
\DeclareMathOperator{\Alg}{\textbf{Alg}}
\DeclareMathOperator{\Rng}{\textbf{Rng}}
\DeclareMathOperator{\Sets}{\textbf{Sets}}
\DeclareMathOperator{\Set}{\textbf{Set}}
\DeclareMathOperator{\Grp}{\textbf{Grp}}
\DeclareMathOperator{\Met}{\textbf{Met}}
\DeclareMathOperator{\BA}{\textbf{BA}}
\DeclareMathOperator{\Mon}{\textbf{Mon}}
\DeclareMathOperator{\Top}{\textbf{Top}}
\DeclareMathOperator{\hTop}{\textbf{hTop}}
\DeclareMathOperator{\HTop}{\textbf{HTop}}
\DeclareMathOperator{\Aut}{\text{Aut}}
\DeclareMathOperator{\RMod}{R-\textbf{Mod}}
\DeclareMathOperator{\RAlg}{R-\textbf{Alg}}
\DeclareMathOperator{\LF}{LF}
\DeclareMathOperator{\op}{op}
\DeclareMathOperator{\Rings}{\textbf{Rings}}
\DeclareMathOperator{\Ring}{\textbf{Ring}}
\DeclareMathOperator{\Groups}{\textbf{Groups}}
\DeclareMathOperator{\Group}{\textbf{Group}}
\DeclareMathOperator{\ev}{ev}
% Algebraic Topology
\DeclareMathOperator{\obj}{obj}
\DeclareMathOperator{\Spec}{Spec}
\DeclareMathOperator{\spec}{spec}
% Model theory
\DeclareMathOperator*{\ind}{\raise0.2ex\hbox{\ooalign{\hidewidth$\vert$\hidewidth\cr\raise-0.9ex\hbox{$\smile$}}}}
\def\nind{\cancel{\ind}}
\DeclareMathOperator{\acl}{acl}
\DeclareMathOperator{\tspan}{span}
\DeclareMathOperator{\acleq}{acl^{\eq}}
\DeclareMathOperator{\Av}{Av}
\DeclareMathOperator{\ded}{ded}
\DeclareMathOperator{\EM}{EM}
\DeclareMathOperator{\dcl}{dcl}
\DeclareMathOperator{\Ext}{Ext}
\DeclareMathOperator{\eq}{eq}
\DeclareMathOperator{\ER}{ER}
\DeclareMathOperator{\tp}{tp}
\DeclareMathOperator{\stp}{stp}
\DeclareMathOperator{\qftp}{qftp}
\DeclareMathOperator{\Diag}{Diag}
\DeclareMathOperator{\MD}{MD}
\DeclareMathOperator{\MR}{MR}
\DeclareMathOperator{\RM}{RM}
\DeclareMathOperator{\el}{el}
\DeclareMathOperator{\depth}{depth}
\DeclareMathOperator{\ZFC}{ZFC}
\DeclareMathOperator{\GCH}{GCH}
\DeclareMathOperator{\Inf}{Inf}
\DeclareMathOperator{\Pow}{Pow}
\DeclareMathOperator{\ZF}{ZF}
\DeclareMathOperator{\CH}{CH}
\def \FO {\text{FO}}
\DeclareMathOperator{\fin}{fin}
\DeclareMathOperator{\qr}{qr}
\DeclareMathOperator{\Mod}{Mod}
\DeclareMathOperator{\Def}{Def}
\DeclareMathOperator{\TC}{TC}
\DeclareMathOperator{\KH}{KH}
\DeclareMathOperator{\Part}{Part}
\DeclareMathOperator{\Infset}{\textsf{Infset}}
\DeclareMathOperator{\DLO}{\textsf{DLO}}
\DeclareMathOperator{\PA}{\textsf{PA}}
\DeclareMathOperator{\DAG}{\textsf{DAG}}
\DeclareMathOperator{\ODAG}{\textsf{ODAG}}
\DeclareMathOperator{\sfMod}{\textsf{Mod}}
\DeclareMathOperator{\AbG}{\textsf{AbG}}
\DeclareMathOperator{\sfACF}{\textsf{ACF}}
\DeclareMathOperator{\DCF}{\textsf{DCF}}
% Computability Theorem
\DeclareMathOperator{\Tot}{Tot}
\DeclareMathOperator{\graph}{graph}
\DeclareMathOperator{\Fin}{Fin}
\DeclareMathOperator{\Cof}{Cof}
\DeclareMathOperator{\lh}{lh}
% Commutative Algebra
\DeclareMathOperator{\ord}{ord}
\DeclareMathOperator{\Idem}{Idem}
\DeclareMathOperator{\zdiv}{z.div}
\DeclareMathOperator{\Frac}{Frac}
\DeclareMathOperator{\rad}{rad}
\DeclareMathOperator{\nil}{nil}
\DeclareMathOperator{\Ann}{Ann}
\DeclareMathOperator{\End}{End}
\DeclareMathOperator{\coim}{coim}
\DeclareMathOperator{\coker}{coker}
\DeclareMathOperator{\Bil}{Bil}
\DeclareMathOperator{\Tril}{Tril}
\DeclareMathOperator{\tchar}{char}
\DeclareMathOperator{\tbd}{bd}

% Topology
\DeclareMathOperator{\diam}{diam}
\newcommand{\interior}[1]{%
  {\kern0pt#1}^{\mathrm{o}}%
}

\DeclareMathOperator*{\bigdoublewedge}{\bigwedge\mkern-15mu\bigwedge}
\DeclareMathOperator*{\bigdoublevee}{\bigvee\mkern-15mu\bigvee}

% \makeatletter
% \newcommand{\vect}[1]{%
%   \vbox{\m@th \ialign {##\crcr
%   \vectfill\crcr\noalign{\kern-\p@ \nointerlineskip}
%   $\hfil\displaystyle{#1}\hfil$\crcr}}}
% \def\vectfill{%
%   $\m@th\smash-\mkern-7mu%
%   \cleaders\hbox{$\mkern-2mu\smash-\mkern-2mu$}\hfill
%   \mkern-7mu\raisebox{-3.81pt}[\p@][\p@]{$\mathord\mathchar"017E$}$}

% \newcommand{\amsvect}{%
%   \mathpalette {\overarrow@\vectfill@}}
% \def\vectfill@{\arrowfill@\relbar\relbar{\raisebox{-3.81pt}[\p@][\p@]{$\mathord\mathchar"017E$}}}

% \newcommand{\amsvectb}{%
% \newcommand{\vect}{%
%   \mathpalette {\overarrow@\vectfillb@}}
% \newcommand{\vecbar}{%
%   \scalebox{0.8}{$\relbar$}}
% \def\vectfillb@{\arrowfill@\vecbar\vecbar{\raisebox{-4.35pt}[\p@][\p@]{$\mathord\mathchar"017E$}}}
% \makeatother
% \bigtimes

\DeclareFontFamily{U}{mathx}{\hyphenchar\font45}
\DeclareFontShape{U}{mathx}{m}{n}{
      <5> <6> <7> <8> <9> <10>
      <10.95> <12> <14.4> <17.28> <20.74> <24.88>
      mathx10
      }{}
\DeclareSymbolFont{mathx}{U}{mathx}{m}{n}
\DeclareMathSymbol{\bigtimes}{1}{mathx}{"91}
% \odiv
\DeclareFontFamily{U}{matha}{\hyphenchar\font45}
\DeclareFontShape{U}{matha}{m}{n}{
      <5> <6> <7> <8> <9> <10> gen * matha
      <10.95> matha10 <12> <14.4> <17.28> <20.74> <24.88> matha12
      }{}
\DeclareSymbolFont{matha}{U}{matha}{m}{n}
\DeclareMathSymbol{\odiv}         {2}{matha}{"63}


\newcommand\subsetsim{\mathrel{%
  \ooalign{\raise0.2ex\hbox{\scalebox{0.9}{$\subset$}}\cr\hidewidth\raise-0.85ex\hbox{\scalebox{0.9}{$\sim$}}\hidewidth\cr}}}
\newcommand\simsubset{\mathrel{%
  \ooalign{\raise-0.2ex\hbox{\scalebox{0.9}{$\subset$}}\cr\hidewidth\raise0.75ex\hbox{\scalebox{0.9}{$\sim$}}\hidewidth\cr}}}

\newcommand\simsubsetsim{\mathrel{%
  \ooalign{\raise0ex\hbox{\scalebox{0.8}{$\subset$}}\cr\hidewidth\raise1ex\hbox{\scalebox{0.75}{$\sim$}}\hidewidth\cr\raise-0.95ex\hbox{\scalebox{0.8}{$\sim$}}\cr\hidewidth}}}
\newcommand{\stcomp}[1]{{#1}^{\mathsf{c}}}

\setlength{\baselineskip}{0.5in}

\stackMath
\newcommand\yrightarrow[2][]{\mathrel{%
  \setbox2=\hbox{\stackon{\scriptstyle#1}{\scriptstyle#2}}%
  \stackunder[0pt]{%
    \xrightarrow{\makebox[\dimexpr\wd2\relax]{$\scriptstyle#2$}}%
  }{%
   \scriptstyle#1\,%
  }%
}}
\newcommand\yleftarrow[2][]{\mathrel{%
  \setbox2=\hbox{\stackon{\scriptstyle#1}{\scriptstyle#2}}%
  \stackunder[0pt]{%
    \xleftarrow{\makebox[\dimexpr\wd2\relax]{$\scriptstyle#2$}}%
  }{%
   \scriptstyle#1\,%
  }%
}}
\newcommand\yRightarrow[2][]{\mathrel{%
  \setbox2=\hbox{\stackon{\scriptstyle#1}{\scriptstyle#2}}%
  \stackunder[0pt]{%
    \xRightarrow{\makebox[\dimexpr\wd2\relax]{$\scriptstyle#2$}}%
  }{%
   \scriptstyle#1\,%
  }%
}}
\newcommand\yLeftarrow[2][]{\mathrel{%
  \setbox2=\hbox{\stackon{\scriptstyle#1}{\scriptstyle#2}}%
  \stackunder[0pt]{%
    \xLeftarrow{\makebox[\dimexpr\wd2\relax]{$\scriptstyle#2$}}%
  }{%
   \scriptstyle#1\,%
  }%
}}

\newcommand\altxrightarrow[2][0pt]{\mathrel{\ensurestackMath{\stackengine%
  {\dimexpr#1-7.5pt}{\xrightarrow{\phantom{#2}}}{\scriptstyle\!#2\,}%
  {O}{c}{F}{F}{S}}}}
\newcommand\altxleftarrow[2][0pt]{\mathrel{\ensurestackMath{\stackengine%
  {\dimexpr#1-7.5pt}{\xleftarrow{\phantom{#2}}}{\scriptstyle\!#2\,}%
  {O}{c}{F}{F}{S}}}}

\newenvironment{bsm}{% % short for 'bracketed small matrix'
  \left[ \begin{smallmatrix} }{%
  \end{smallmatrix} \right]}

\newenvironment{psm}{% % short for ' small matrix'
  \left( \begin{smallmatrix} }{%
  \end{smallmatrix} \right)}

\newcommand{\bbar}[1]{\mkern 1.5mu\overline{\mkern-1.5mu#1\mkern-1.5mu}\mkern 1.5mu}

\newcommand{\bigzero}{\mbox{\normalfont\Large\bfseries 0}}
\newcommand{\rvline}{\hspace*{-\arraycolsep}\vline\hspace*{-\arraycolsep}}

\font\zallman=Zallman at 40pt
\font\elzevier=Elzevier at 40pt

\newcommand\isoto{\stackrel{\textstyle\sim}{\smash{\longrightarrow}\rule{0pt}{0.4ex}}}
\newcommand\embto{\stackrel{\textstyle\prec}{\smash{\longrightarrow}\rule{0pt}{0.4ex}}}

% from http://www.actual.world/resources/tex/doc/TikZ.pdf

\tikzset{
modal/.style={>=stealth’,shorten >=1pt,shorten <=1pt,auto,node distance=1.5cm,
semithick},
world/.style={circle,draw,minimum size=0.5cm,fill=gray!15},
point/.style={circle,draw,inner sep=0.5mm,fill=black},
reflexive above/.style={->,loop,looseness=7,in=120,out=60},
reflexive below/.style={->,loop,looseness=7,in=240,out=300},
reflexive left/.style={->,loop,looseness=7,in=150,out=210},
reflexive right/.style={->,loop,looseness=7,in=30,out=330}
}


\makeatletter
\newcommand*{\doublerightarrow}[2]{\mathrel{
  \settowidth{\@tempdima}{$\scriptstyle#1$}
  \settowidth{\@tempdimb}{$\scriptstyle#2$}
  \ifdim\@tempdimb>\@tempdima \@tempdima=\@tempdimb\fi
  \mathop{\vcenter{
    \offinterlineskip\ialign{\hbox to\dimexpr\@tempdima+1em{##}\cr
    \rightarrowfill\cr\noalign{\kern.5ex}
    \rightarrowfill\cr}}}\limits^{\!#1}_{\!#2}}}
\newcommand*{\triplerightarrow}[1]{\mathrel{
  \settowidth{\@tempdima}{$\scriptstyle#1$}
  \mathop{\vcenter{
    \offinterlineskip\ialign{\hbox to\dimexpr\@tempdima+1em{##}\cr
    \rightarrowfill\cr\noalign{\kern.5ex}
    \rightarrowfill\cr\noalign{\kern.5ex}
    \rightarrowfill\cr}}}\limits^{\!#1}}}
\makeatother

% $A\doublerightarrow{a}{bcdefgh}B$

% $A\triplerightarrow{d_0,d_1,d_2}B$

\def \uhr {\upharpoonright}
\def \rhu {\rightharpoonup}
\def \uhl {\upharpoonleft}


\newcommand{\floor}[1]{\lfloor #1 \rfloor}
\newcommand{\ceil}[1]{\lceil #1 \rceil}
\newcommand{\lcorner}[1]{\llcorner #1 \lrcorner}
\newcommand{\llb}[1]{\llbracket #1 \rrbracket}
\newcommand{\ucorner}[1]{\ulcorner #1 \urcorner}
\newcommand{\emoji}[1]{{\DejaSans #1}}
\newcommand{\vprec}{\rotatebox[origin=c]{-90}{$\prec$}}

\newcommand{\nat}[6][large]{%
  \begin{tikzcd}[ampersand replacement = \&, column sep=#1]
    #2\ar[bend left=40,""{name=U}]{r}{#4}\ar[bend right=40,',""{name=D}]{r}{#5}\& #3
          \ar[shorten <=10pt,shorten >=10pt,Rightarrow,from=U,to=D]{d}{~#6}
    \end{tikzcd}
}


\providecommand\rightarrowRHD{\relbar\joinrel\mathrel\RHD}
\providecommand\rightarrowrhd{\relbar\joinrel\mathrel\rhd}
\providecommand\longrightarrowRHD{\relbar\joinrel\relbar\joinrel\mathrel\RHD}
\providecommand\longrightarrowrhd{\relbar\joinrel\relbar\joinrel\mathrel\rhd}
\def \lrarhd {\longrightarrowrhd}


\makeatletter
\providecommand*\xrightarrowRHD[2][]{\ext@arrow 0055{\arrowfill@\relbar\relbar\longrightarrowRHD}{#1}{#2}}
\providecommand*\xrightarrowrhd[2][]{\ext@arrow 0055{\arrowfill@\relbar\relbar\longrightarrowrhd}{#1}{#2}}
\makeatother

\newcommand{\metalambda}{%
  \mathop{%
    \rlap{$\lambda$}%
    \mkern3mu
    \raisebox{0ex}{$\lambda$}%
  }%
}

%% https://tex.stackexchange.com/questions/15119/draw-horizontal-line-left-and-right-of-some-text-a-single-line
\newcommand*\ruleline[1]{\par\noindent\raisebox{.8ex}{\makebox[\linewidth]{\hrulefill\hspace{1ex}\raisebox{-.8ex}{#1}\hspace{1ex}\hrulefill}}}

% https://www.dickimaw-books.com/latex/novices/html/newenv.html
\newenvironment{Block}[1]% environment name
{% begin code
  % https://tex.stackexchange.com/questions/19579/horizontal-line-spanning-the-entire-document-in-latex
  \noindent\textcolor[RGB]{128,128,128}{\rule{\linewidth}{1pt}}
  \par\noindent
  {\Large\textbf{#1}}%
  \bigskip\par\noindent\ignorespaces
}%
{% end code
  \par\noindent
  \textcolor[RGB]{128,128,128}{\rule{\linewidth}{1pt}}
  \ignorespacesafterend
}

\mathchardef\mhyphen="2D % Define a "math hyphen"

\def \QQ {\quad}
\def \QW {​\quad}

\makeindex
\usepackage[UTF8]{ctex}
\DeclareMathOperator{\CPO}{\textbf{CPO}}
\newtheorem{theorem}{定理}[section]
\newtheorem{assumption}[theorem]{假设}
\newtheorem{corollary}[theorem]{推论}
\newtheorem{proposition}[theorem]{命题}
\newtheorem{lemma}[theorem]{引理}
\newtheorem{definition}[theorem]{定义}
\newtheorem{assum}[theorem]{假设}
\newtheorem{note}[theorem]{注}
\newtheorem{fact}[theorem]{性质}
\newtheorem*{claim}{断言}
\newtheorem{Theorem}{定理}
\newtheorem{example}[theorem]{例}
\newtheorem{property}[theorem]{性质}
\newtheorem{annotation}[theorem]{注}
\author{陈淇奥}
\date{2022年6月}
\title{Scott拓扑与\(D_\infty\)}
\hypersetup{
 pdfauthor={陈淇奥},
 pdftitle={Scott拓扑与\(D_\infty\)},
 pdfkeywords={},
 pdfsubject={},
 pdfcreator={Emacs 28.0.92 (Org mode 9.6)}, 
 pdflang={English}}
\begin{document}

\maketitle
\begin{abstract}
本篇文章介绍了Dana Scott \cite{10.1007/BFb0073967} 构造的lambda演算的一种模型\(D_\infty\)。
\end{abstract}

\section{Scott拓扑}
\label{sec:orgc7ec157}

\begin{definition}[]
给定偏序集\(\la D,\sqsubseteq\ra\)以及集合\(X\subseteq D\),
\begin{enumerate}
\item 用\(\bot\)表示\(D\)的 \textbf{最小元} ;
\item 用\(\bigsqcup X\)表示\(X\)的最小上界;
\item 若\(X\)非空且对任意\(a,b\in X\)都存在\(c\in X\)使得\(a\sqsubseteq c\)且\(b\sqsubseteq c\),则称\(X\)是 \textbf{有向集} ;
\item 若\(D\)满足
\begin{enumerate}
\item \(D\)有最小元;
\item 每一个\(D\)的有向子集\(X\)都有最小上界。
\end{enumerate}
则称\(D\)是 \textbf{完全偏序} (complete partial order) ,记作c.p.o.。
\end{enumerate}
\end{definition}

\begin{definition}[]
给定任意\(\bot\notin\N\),定义\(\N^+=\N\cup\{\bot\}\),并且对任意\(a,b\in\N^+\),定义
\begin{equation*}
a\sqsubseteq b\Leftrightarrow(a=\bot\wedge b\in\N)\vee a=b
\end{equation*}
我们用\(\N^+\)表示\(\la\N^+,\sqsubseteq\ra\)。
\end{definition}

\begin{lemma}[]
\(\N^+\)是完全偏序。
\end{lemma}

\begin{proof}
注意到\(\N^+\)的有向子集只包括单点集与\(\{\bot,n\}\),其中\(n\in\N\)
\end{proof}

\begin{definition}[]
给定完全偏序\(D,D'\),令\(f\)是从\(D\)到\(D'\)的函数,定义\(f\)是 \textbf{单调的} 当且仅当
\begin{equation*}
a\sqsubseteq b\Rightarrow f(a)\sqsubseteq'f(b)
\end{equation*}
\end{definition}

\begin{definition}[]
给定完全偏序\(\la D,\sqsubseteq\ra\),定义\(D\)上的 \textbf{Scott拓扑} :\(O\subseteq D\)是开集当且仅当
\begin{enumerate}
\item \(x\in O\wedge x\sqsubseteq y\Rightarrow y\in O\);
\item 若\(X\subseteq D\)有向且\(\bigsqcup X\in O\),则\(X\cap O\neq\emptyset\)。
\end{enumerate}
\end{definition}

\begin{lemma}[]
令\(U_x=\{z\in D\mid z\not\sqsubseteq x\}\),则\(U_x\)是开集
\end{lemma}

\begin{proof}
\begin{enumerate}
\item 若\(y\in U_x\)且\(y\sqsubseteq z\),若\(z\sqsubseteq x\),则\(y\sqsubseteq x\)矛盾。
\item 若\(X\subseteq D\)有向且\(\bigsqcup X\in U_x\),若\(X\cap U_x=\emptyset\),则\(\bigsqcup X\sqsubseteq x\),矛盾。
\end{enumerate}
\end{proof}

\begin{corollary}[]
\(D\)是\(T_0\)空间
\end{corollary}

\begin{proof}
令\(x,y\in D\)且\(x\neq y\),则\(x\in U_y\)且\(y\notin U_y\)。
\end{proof}

\begin{proposition}[]
考虑函数\(f:D\to D'\),则
\begin{center}
\(f\)连续当且仅当对任意有向集\(X\subseteq D\),\(f(\bigsqcup X)=\bigsqcup f(X)\)
\end{center}
其中\(f(X)=\{f(x)\mid x\in X\}\)。
\end{proposition}

\begin{proof}
\(\Rightarrow\):若\(f\)连续,假设\(x\sqsubseteq y\)且\(f(x)\not\sqsubseteq' f(y)\),则\(f(x)\in U_{f(y)}\),
\(x\in f^{-1}(U_{f(y)})\),由于\(f^{-1}(U_{f(y)})\)是开集,\(y\in f^{-1}(U_{f(y)})\),
\(f(y)\in U_{f(y)}\),矛盾。因此对于任意\(x\in X\),\(f(\bigsqcup X)\sqsupseteq f(x)\),\(f(\bigsqcup X)\sqsupseteq\bigsqcup f(X)\)。
若\(f(\bigsqcup X)\not\sqsubseteq\bigsqcup f(X)\),则\(f(\bigsqcup X)\in U_{\bigsqcup f(X)}\),\(\bigsqcup X\in f^{-1}(U_{\bigsqcup f(X)})\),由定义,存
在\(a\in X\)使得\(a\in X\cap f^{-1}(U_{\bigsqcup f(X)})\),因此\(f(a)\in U_{\bigsqcup f(X)}\),\(f(a)\not\sqsubseteq\bigcup f(X)\),矛盾。

\(\Leftarrow\):若\(x\sqsubseteq y\),则\(y=x\sqcup y\),\(f(y)=f(x)\sqcup f(y)\),因此\(f(x)\sqsubseteq f(y)\)。因此若\(O\subseteq D'\)是开集,
对于任意有向\(X\subseteq D\)且\(\bigsqcup X\in f^{-1}(O)\),有\(f(\bigsqcup X)=\bigsqcup f(X)\in O\),而\(f(X)\)是有向,于
是\(f(X)\cap O\neq\emptyset\),因此\(X\cap f^{-1}(O)\neq\emptyset\)。
\end{proof}

\begin{proposition}[]
给定完全偏序\(D,D'\),定义\(D\times D'\)上的偏序为
\begin{equation*}
(x,x')\sqsubseteq(y,y')\Leftrightarrow x\sqsubseteq y\wedge x'\sqsubseteq y'
\end{equation*}
则\(D\times D'\)是完全偏序,给定任意有向集\(X\subseteq D\times D'\),它的最小上界是
\begin{equation*}
\bigsqcup X=(\bigsqcup X_0,\bigsqcup X_1)
\end{equation*}
其中
\begin{align*}
X_0&=\{x\in D\mid\exists x'\in  D'(x,x')\in X\}\\
X_1&=\{x'\in D'\mid\exists x\in D(x,x')\in X\}
\end{align*}
\end{proposition}

\begin{proof}
首先\((\bot,\bot')\)是\(D\times D'\)的最小元。对于任意有向集合\(X\subseteq D\times D'\),\(X_0,X_1\)也是有向集合,因
此\(\bigsqcup X_0,\bigsqcup X_1\)存在,于是对于任意\(X\)的上界\((A,B)\),\(A\)是\(X_0\)的上界,\(B\)是\(X_1\)的上
界,因此\((\bigsqcup X_0,\bigsqcup X_1)\sqsubseteq(A,B)\),因此\(\bigsqcup X=(\bigsqcup X_0,\bigsqcup X_1)\)。
\end{proof}


\begin{definition}[]
给定完全偏序\(D,D'\),定义
\begin{equation*}
[D\to D']=\{f:D\to D'\mid f\text{连续}\}
\end{equation*}
并且定义\([D\to D']\)上的偏序为
\begin{equation*}
f\sqsubseteq g\Leftrightarrow\forall x\in D(f(x)\sqsubseteq'g(x))
\end{equation*}
\end{definition}

\begin{lemma}[]
\label{1.2.10}
令\(\{f_i\}_i\subseteq[D\to D']\)为有向的函数集合,定义
\begin{equation*}
f(x)=\bigsqcup_if_i(x)
\end{equation*}
则\(f\)是良定义的并且是连续的。
\end{lemma}

\begin{proof}
应为\(\{f_i\}_i\)有向,因此对于任意\(x\in D\),\(\{f_i(x)\}_i\)有向,因此\(f\)存在且\(f(x)\)唯一。对于任意
有向集合\(X\subseteq D\),
\begin{equation*}
f(\bigsqcup X)=\bigsqcup_i\bigsqcup_{x\in X}f_i(x)=\bigsqcup_{x\in X}\bigsqcup_if_i(x)=\bigsqcup f(X)
\end{equation*}
\end{proof}

下面使用\(\metalambda d\in D.\phi(a_1,\dots,a_n,d)\)来表示函数\(f(d)=\phi(a_1,\dots,a_n,d)\),其中\(d\in D\)。

\begin{proposition}[]
\label{1.2.11}
\([D\to D']\)是完全偏序,并且对于任意有向\(F\subseteq[D\to D']\),它的最小上界为
\begin{equation*}
(\bigsqcup F)(x)=\bigsqcup\{f(x)\mid f\in F\}
\end{equation*}
\end{proposition}

\begin{proof}
\(\metalambda x.\bot'\)是\([D\to D']\)的最小元,由引理\ref{1.2.10} ,\(\metalambda x.\bigsqcup\{f(x)\mid f\in F\}\)是
连续的,因此属于\([D\to D']\),显然它是最小上界。
\end{proof}

\begin{proposition}[]
给定完全偏序\(D,D',D''\),若\(f\in[D\to D']\),\(g\in[D'\to D'']\),定义\(g\circ f\)为对任意\(d\in D\),
\((g\circ f)(d)=g(f(d))\),则\(g\circ f\in[D\to D'']\)。
\end{proposition}

\begin{proof}
任给有向集合\(X\subseteq D\),\(f\in[D\to D']\),\(g\in[D'\to D'']\),则
\begin{align*}
g\circ f(\bigsqcup X)&=g(f(\bigsqcup X))=g(\bigsqcup_{x\in X} f(x))=\bigsqcup_{x\in X}g(f(x))=\bigsqcup_{x\in X} g\circ f(x)
\end{align*}
\end{proof}


\begin{lemma}[]
\label{1.2.12}
令\(f:D\times D'\to D''\),则\(f\)连续当且仅当它在\(D\)跟\(D'\)上连续,即对于任意\(x_0\in D,x_0'\in D'\),
\(\metalambda x.f(x,x_0')\)和\(\metalambda x.f(x_0,x)\)连续。
\end{lemma}

\begin{proof}
\(\Rightarrow\):令\(g=\metalambda x.f(x,x_0')\),则对于有向集合\(X\subseteq D\)
\begin{align*}
g(\bigsqcup X)&=f(\bigsqcup X,x_0')=f(\bigsqcup\{(x,x_0')\mid x\in X\})\\
&=\bigsqcup\{f(x,x_0')\mid x\in X\}\\
&=\bigsqcup g(X)
\end{align*}
同理,\(\metalambda x.f(x_0,x)\)连续。

\(\Leftarrow\):给定有向集合\(X\subseteq D\times D'\),
\begin{align*}
f(\bigsqcup X)&=f(\bigsqcup X_0,\bigsqcup X_1)\\
&=\bigsqcup_{x\in X_0}f(x,\bigsqcup X_1)=\bigsqcup_{x\in X_0}\bigsqcup_{x'\in X_0'}f(x,x')\\
&=\bigsqcup_{(x,x')\in X}f(x,x')\\
&=\bigsqcup f(X)
\end{align*}
因此\(f\)连续。
\end{proof}

\begin{proposition}[]
\label{1.2.13}
给定完全偏序\(D,D'\),令
\begin{equation*}
app:[D\to D']\times D\to D'
\end{equation*}
为\(app(f,x)=f(x)\),则\(app\)连续。
\end{proposition}

\begin{proof}
给定有向集合\(F\subseteq[D\to D']\),令\(h=\metalambda f.f(x)\),则
\begin{align*}
h(\bigsqcup F)&=(\bigsqcup F)(x)=\bigsqcup\{f(x)\mid f\in F\}\\
&=\bigsqcup\{h(f)\mid f\in F\}=\bigsqcup h(F)
\end{align*}
因此\(h\)连续,同时因为\(\metalambda x.f(x)=f\)连续,由命题\ref{1.2.11} \(app\)连续
\end{proof}

\begin{proposition}[]
\label{1.2.14}
给定\(f\in[D\times D'\to D'']\),定义\(\hatf(x)=\metalambda y\in D'(f(x,y))\),则
\begin{enumerate}
\item \(\hatf\)连续;
\item \(\metalambda f.\hatf:[D\times D'\to D'']\to[D\to[D'\to D'']]\)连续。
\end{enumerate}
\end{proposition}

\begin{proof}
\begin{enumerate}
\item 对于任意有向集\(X\subseteq D\),
\begin{align*}
\hatf(\bigsqcup X)&=\metalambda y.f(\bigsqcup X,y)=\metalambda y.\bigsqcup_{x\in X}f(x,y)\\
&=\bigsqcup_{x\in X}(\metalambda y.f(x,y))\\
&=\bigsqcup\hatf(X)
\end{align*}
\item 令\(L=\metalambda f.\hatf\),对于任意有向集\(F\subseteq[D\times D'\to D'']\),
\begin{align*}
L(\bigsqcup F)&=\metalambda x.\metalambda y.(\bigsqcup F)(x,y)=\metalambda x\metalambda y.\bigsqcup_{f\in F}f(x,y)\\
&=\bigsqcup_{f\in F}\metalambda x.\metalambda y.f(x,y)=\bigsqcup L(F)
\end{align*}
\end{enumerate}
\end{proof}

\begin{definition}[]
定义\(\CPO\)是以完全偏序为元素连续映射为态射的范畴。
\end{definition}

\begin{theorem}[]
\(\CPO\)是笛卡儿闭范畴。
\end{theorem}

\begin{proof}
\(D\times D'\)是\(\CPO\)中的乘积,同时单元素完全偏序是终对象,而
对于任意\(f:D\times D'\to D''\),由命题\ref{1.2.13} 和\ref{1.2.14} ,都存在
唯一的\(\hatf:D\to[D'\to D'']\)使得
\begin{center}\begin{tikzcd}
D\times D'\ar[r,"f"]\ar[d,dashed,"\hatf\times\id_{D'}"']&D\\
\left[D'\to D''\right]\times D'\ar[ur,"app"]
\end{tikzcd}\end{center}
交换。
\end{proof}

\begin{definition}[]
令\(D_0,D_1,\dots\)是可数的完全偏序序列,令\(f_i\in[D_{i+1}\to D_i]\),
\begin{enumerate}
\item 序列\((D_i,f_i)\)称为完全偏序的 \textbf{逆向系统} (inverse system)。
\item 系统\((D_i,f_i)\)的 \textbf{逆向极限} (inverse limit) \(\varprojlim(D_i,f_i)\)(或记作\(\varprojlim D_i\))是偏序集\((D_\infty,\sqsubseteq_\infty)\),其中
\begin{align*}
D_\infty=\{(x_0,x_1,\dots)\mid\forall i\in\N(x_i\in D_i\wedge\psi_i(x_{i+1})=x_i)\}
\end{align*}
并且
\begin{equation*}
(x_0,x_1,\dots)\sqsubseteq_\infty(y_0,y_1,\dots)\Leftrightarrow\forall i\in\N(x_i\sqsubseteq y_i)
\end{equation*}
\end{enumerate}
\end{definition}

\begin{proposition}[]
给定逆向系统\((D_i,f_i)\),则\(\varprojlim(D_i,f_j)\)是完全偏序且对任意有向\(X\subseteq\varprojlim D_i\),
\begin{equation*}
\bigsqcup X=\metalambda i.\bigsqcup\{x(i)\mid x\in X\}
\end{equation*}
\end{proposition}

\begin{proof}
对于任意有向\(X\subseteq D_\infty\),则对任意\(i\in\N\),\(\{x(i)\mid x\in X\}\)有向,令
\begin{equation*}
y_i=\bigsqcup\{x(i)\mid x\in X\}
\end{equation*}
则由\(\psi_i\)的连续性,
\begin{equation*}
\psi_i(y_{i+1})=\bigsqcup f_i(\{x(i+1)\mid x\in X\})=\bigsqcup\{x(i)\mid x\in X\}=y_i
\end{equation*}
因此\((y_0,y_1,\dots)\in \varprojlim D_i\)。
\end{proof}

因此在\(\CPO\)中,逆向极限存在。
\section{\texorpdfstring{\(D_\infty\)}{D}}
\label{sec:org1af03b9}

\begin{definition}[]
给定完全偏序\(D\)和\(D'\),\(D\)与\(D'\) \textbf{同构} 当且仅当存在\(\phi\in[D\to D']\)与\(\psi\in[D'\to D]\)使得
\begin{equation*}
\psi\circ\phi=\id_D,\quad\phi\circ\psi=\id_{D'}
\end{equation*}
\end{definition}


\begin{definition}[]
给定完全偏序\(D\)和\(D'\)。函数的二元组\(\la\varphi,\psi\ra\)是从\(D'\)到\(D\)的 \textbf{投射} 如果
\begin{enumerate}
\item \(\varphi\in[D\to D']\), \(\psi\in[D'\to D]\)
\item \(\psi\circ\varphi=\id_D\)
\item \(\varphi\circ\psi\sqsubseteq\id_{D'}\)
\end{enumerate}
\end{definition}

注意到\(D\)与\(\varphi\psi(D)\)同构,因此在同构的意义下\(D\subseteq D'\)。

\begin{definition}[]
定义\(D_0=\N^+\),\(D_{n+1}=[D_n\to D_n]\),记\(D_n\)的最小元为\(\bot_n\)
\end{definition}

由\ref{1.2.11} ,对任意\(n\in\N\),\(D_n\)是完全偏序。

\begin{lemma}[]
给定\(D'\)到\(D\)的投射\((\varphi,\psi)\),存在从\([D'\to D']\)到\([D\to D]\)的投射\((\varphi^*,\psi^*)\)满足:对于任
意\(f\in[D\to D]\),\(g\in[D'\to D']\)有
\begin{equation*}
\varphi^*(f)=\varphi\circ f\circ\psi,\quad\psi^*(g)=\psi\circ g\circ\varphi
\end{equation*}

\begin{center}\begin{tikzcd}
D\ar[d,"f"']&D'\ar[l,"\psi"']\ar[d,dashed,"\varphi^*(f)"]\\
D\ar[r,"\varphi"']&D'
\end{tikzcd}\quad\begin{tikzcd}
D\ar[r,"\varphi"]\ar[d,dashed,"\psi^*(g)"']&D'\ar[d,"g"]\\
D&D'\ar[l,"\psi"]
\end{tikzcd}\end{center}
\end{lemma}

\begin{proof}
注意到
\begin{align*}
\varphi^*(f)&=\metalambda x'\in D'.\varphi(f(\psi(x)))\\
&=\metalambda x'\in D'.\varphi(app(f,\psi(x)))
\end{align*}
于是\(\varphi^*\)是连续的,类似的\(\psi^*\)是连续的。同时
\begin{gather*}
\psi^*(\varphi^*(f))=\psi\circ\varphi\circ f\circ\psi\circ\varphi=f\\
\varphi^*(\psi^*(f))=\varphi\circ\psi\circ f\circ\varphi\circ\psi\sqsubseteq f
\end{gather*}
\end{proof}

\begin{lemma}[]
给定完全偏序\(D\),定义\(\varphi_0:D\to[D\to D]\),\(\psi_0:[D\to D]\to D\)为
\begin{align*}
&\varphi_0(x)=\metalambda y\in D.x\\
&\psi_0(f)=f(\bot)
\end{align*}
则\((\varphi_0,\psi_0)\)是从\([D\to D]\)到\(D\)的投射。
\end{lemma}

\begin{proof}
首先证明\(\varphi_0\)连续,给定有向集\(X\subseteq D\),
\begin{align*}
\varphi_0(\bigsqcup X)&=\metalambda y\in D.\bigsqcup X=\bigsqcup_{x\in X}\metalambda y\in D.x\\
&=\bigsqcup\varphi_0(X)
\end{align*}
同理,\(\psi_0\)连续。同时
\begin{align*}
\varphi_0(\psi_0(f))&=\varphi_0(f(\bot))=\metalambda x.f(\bot)\\
&\sqsubseteq\metalambda x.f(x)=f\\
\psi_0\circ\varphi_0(f)&=\varphi_0(f)(\bot)=f
\end{align*}
\end{proof}

\begin{definition}[构造\(D_\infty\)]
给定完全偏序\(D\)与\((\varphi_0,\psi_0)\)如上,定义
\begin{align*}
&D_0=D\\
&D_{n+1}=[D_n\to D_n]\\
&(\varphi_{n+1},\psi_{n+1})=(\varphi_n^*,\psi_n^*)
\end{align*}
令\(D_\infty=\varprojlim(D_n,\psi_n)\),记\(x\in D_\infty\)为\((x_0,x_1,\dots)\)。
\end{definition}


\begin{definition}[]
\begin{enumerate}
\item 对于\(n,m\in\N\),定义\(\Phi_{nm}:D_n\to D_m\)为:

若\(n\le m\), \(m=n+k\),则递归定义\(\Phi_{nm}\)为
\begin{align*}
&\Phi_{nn}=\lambda x\in D_n.x\\
&\Phi_{n(m+1)}=\varphi_m\circ\Phi_{nm}
\end{align*}
若\(m\le n\),\(n=m+k\),递归定义\(\Phi_{nm}\)为
\begin{equation*}
\Phi_{(n+1)m}=\Phi_{nm}\circ\psi_n
\end{equation*}
\item 定义\(\Phi_{\infty n}:D_\infty\to D_n\)为\(\Phi_{\infty n}(x)=x_n\)。
\item 定义\(\Phi_{n\infty}:D_n\to D_\infty\)为\(\Phi_{n\infty}(x)=(\Phi_{ni}(x))_{i\in\N}\)
\end{enumerate}
\end{definition}

\begin{lemma}[]
\begin{enumerate}
\item 对于\(0\le n\le m\le\infty\),\((\Phi_{nm},\Phi_{mn})\)是从\(D_m\)到\(D_n\)的投射
\item 对于\(0\le n\le m\le l\le\infty\),\(\Phi_{ml}\circ\Phi_{nm}=\Phi_{nl}\)
\end{enumerate}
\end{lemma}

\begin{proof}
\begin{enumerate}
\item 若\(n<m<\infty\),对于任意\(x\in D_m\),
\begin{align*}
\Phi_{nm}\circ\Phi_{mn}&=(\varphi_{m-1}\circ\dots\circ\varphi_n\circ\id_{D_n})\circ(\id_{D_n}\circ\psi_n\circ\dots\circ\psi_{m-1})\\
&\sqsubseteq\id_{D_m}\\
\Phi_{mn}\circ\Phi_{nm}&=(\id_{D_n}\circ\psi_1\circ\dots\circ\psi_{m-1})\circ(\varphi_{m-1}\circ\dots\circ\varphi_1\circ\id_{D_n})\\
&=\id_{D_n}
\end{align*}
\(n<m=\infty\)和\(n=m=\infty\)的情况类似。
\item 根据定义类似可得。
\end{enumerate}
\end{proof}

注意到在同构的意义下,
\begin{equation*}
D_0\subseteq D_1\subseteq\dots\subseteq D_\infty
\end{equation*}
又有一个事实是在\(\CPO\)中,\(D_\infty\)不仅是逆向极限,也是正向极限
\begin{equation*}
D_\infty\cong\varinjlim(D_n,\varphi_n)
\end{equation*}
因此每个元素\(x\in D_n\)也可被\(\Phi_{n\infty}(x)\in D_\infty\)刻画。

\begin{lemma}[]
\label{18.2.7}
\begin{enumerate}
\item 如果\(x\in D_n\),则\((\Phi_{n\infty}(x))n=x\)。
\item 如果\(x\in D_n\),则\(\Phi_{(n+1)\infty}\varphi_n(x)=\Phi_{n\infty}x\)。
\item 如果\(x\in D_{n+1}\),则\(\Phi_{n\infty}\psi_n(x)\sqsubseteq \Phi_{(n+1)\infty}x\)。
\end{enumerate}
\end{lemma}

\begin{proof}
\begin{enumerate}
\item 在\(D_\infty\)中,\(x\)为\(\Phi_{n\infty}(x)\),因此\(x_n=x\)。
\item \(\varphi_n(x)\)在\(D_\infty\)中为\((\dots,\psi_n(\varphi_n(x)),\varphi_n(x),\varphi_{n+1}\varphi_n(x),\dots)\),因
为\(\psi_n(\varphi_n(x))=x\),因此\(\varphi_n(x)=x\)。
\item \(\varphi_n\psi_n(x)\sqsubseteq x\)。
\end{enumerate}
\end{proof}

\begin{lemma}[]
\label{18.2.8}
在\(D_\infty\)中,若\(x\in D_\infty\),则
\begin{enumerate}
\item \((\Phi_{n\infty}x_n)_m=x_{\min(n,m)}\)
\item \(n\le m\Rightarrow \Phi_{n\infty}(x_n)\sqsubseteq\Phi_{m\infty}(x_m)\sqsubseteq x\)
\item \(x=\bigsqcup_{n\in\N}\Phi_{n\infty}x_n\)
\item \(\Phi_{n\infty}(\bot_n)=\bot\)
\end{enumerate}
\end{lemma}

\begin{proof}
\begin{enumerate}
\item 由 \ref{18.2.7} (2).
\item 由\ref{18.2.7} (3),\(\Phi_{m\infty}(x_m)=\Phi_{m\infty}(\psi_m(x_{m+1}))\sqsubseteq\Phi_{(m+1)\infty}(x_{m+1})\),因此
\(\Phi_{0\infty}(x_0)\sqsubseteq\Phi_{1\infty}(x_1)\sqsubseteq\cdots\)。并且,由于对于任意\(i\in\N\),
\((\Phi_{n\infty}x_n)_i=x_{\min(i,n)}\sqsubseteq x_i\),有\(x_n\sqsubseteq x\)。
\item 由(2),集合\(X=\{\Phi_{n\infty}(x_n)\mid n\in\N\}\)有向,因此
\begin{align*}
\bigsqcup X&=(\bigsqcup_n(\Phi_{n\infty}(x_n))_i)_{i\in\N}\\
&=(\bigsqcup_n\Phi_{\min(n,i)\infty}(x_{\min(n,i)}))_{i\in\N}\\
&=(x_i)_{i\in\N}=x
\end{align*}
\item 由(2),\(\Phi_{n\infty}(\bot_n)\sqsubseteq\bot\sqsubseteq\Phi_{n\infty}\bot_n\)。
\end{enumerate}
\end{proof}

\begin{lemma}[]
\label{18.2.9}
若\(x,y\in D_\infty\),则对所有\(n,k\in\N\),\(n\le k\),有
\begin{enumerate}
\item \(\Phi_{n\infty}(x_{n+1}(y_n))\sqsubseteq \Phi_{(n+1)\infty}(x_{k+1}(y_k))\)
\item \(\Phi_{(k+1)\infty}((\Phi_{(n+1)\infty}(x_{n+1}))_{k+1}(y_k))=\Phi_{n\infty}(x_{n+1}(y_n))\)
\end{enumerate}
\end{lemma}

\begin{proof}
\begin{enumerate}
\item 只需证明\(k=n+1\)的情况:
\begin{align*}
\Phi_{n\infty}(x_{n+1}(y_n))&=\Phi_{n\infty}((\psi_{n+1}(x_{n+2}))(\psi_n(y_{n+1})))\\
&=\Phi_{n\infty}(\psi_n\circ x_{n+2}\circ\varphi_n(\psi_n(y_{n+1})))\\
&\sqsubseteq\Phi_{n\infty}(\psi_n(x_{n+2}(y_{n+1})))\\
&\sqsubseteq\Phi_{(n+1)\infty}(x_{n+2}(y_{n+1}))
\end{align*}
\item 对\(k\ge n\)归纳,考虑\(k+1\)的情况:
\begin{align*}
\Phi_{(k+1)\infty}((\Phi_{(n+1)\infty}(x_{n+1}))_{k+2}(y_{k+1}))
&=\Phi_{(k+1)\infty}(\varphi_{k+1}(\Phi_{(n+1)\infty}(x_{n+1}))_{k+1}(y_{k+1}))\\
&=\Phi_{(k+1)\infty}(\varphi_k\circ(\Phi_{(n+1)\infty}(x_{n+1}))_{k+1}\circ\psi_k(y_{k+1}))\\
&=\Phi_{(k+1)\infty}(\varphi_k\circ(\Phi_{(n+1)\infty}(x_{n+1}))_{k+1}(y_k))\\
&=\Phi_{k\infty}(\Phi_{(n+1)\infty}(x_{n+1})_{k+1}(y_k))\\
&=\Phi_{n\infty}(x_{n+1}(y_n))
\end{align*}
\end{enumerate}
\end{proof}

\begin{lemma}[]
\label{16.42}
对于任意\(x,y\in D_\infty\),
\begin{equation*}
\Phi_{n\infty}(x_{n+1}(y_n))\sqsubseteq\Phi_{(n+1)\infty}(x_{n+2}(y_{n+1}))
\end{equation*}
\end{lemma}

\begin{proof}
首先
\begin{align*}
\phi_n(x_{n+1}(y_n))&=\phi_n(\psi_{n+1}(x_{n+2})(\psi_{n}(y_{n+1})))\\
&=\phi_n(\psi_n(x_{n+2}(\phi_n(\psi_n(y_{n+1})))))\\
&\sqsubseteq\phi_n(\psi_n(x_{n+2}(y_{n+1})))\\
&\sqsubseteq x_{n+2}(y_{n+1})
\end{align*}
于是
\begin{align*}
\Phi_{(n+1)\infty}(\phi_n(x_{n+1}(y_n)))\sqsubseteq\Phi_{(n+1)\infty}(x_{n+2}(y_{n+1}))
\end{align*}
注意到\(\Phi_{(n+1)\infty}\phi_n=\Phi_{(n+1)\infty}\Phi_{n(n+1)}=\Phi_{n\infty}\),因此
\begin{equation*}
\Phi_{n\infty}(x_{n+1}(y_n))\sqsubseteq\Phi_{(n+1)\infty}(x_{n+2}(y_{n+1}))
\end{equation*}
\end{proof}

\begin{definition}[]
给定\(x,y\in D_\infty\),于是由引理\ref{16.42} ,\(\{\Phi_{n\infty}(x_{n+1}(y_n)):n\ge 0\}\)是一个递增序列,因此有最
小上界,定义
\begin{equation*}
x\cdot y=\bigsqcup_{n\ge 0}\Phi_{n\infty}(x_{n+1}(y_n))
\end{equation*}
即
\begin{equation*}
x\cdot y=\bigsqcup_n\Phi_{n\infty}(app_n(\Phi_{\infty(n+1)}(x),\Phi_{\infty n}(y)))
\end{equation*}
其中\(app_n:[D_{n+1}\times D_n]\to D_n\)。
\end{definition}


\begin{proposition}[]
\label{18.2.11}
\(D_\infty\)上的\(\cdot\)连续。
\end{proposition}

\begin{proposition}[]
\label{18.2.12}
若\(x\in D_{n+1},y\in D_n\),则
\begin{equation*}
\Phi_{(n+1)\infty}(x)\cdot\Phi_{n\infty}(y)=\Phi_{n\infty}(x(y))
\end{equation*}
\end{proposition}

\begin{proof}
\begin{align*}
\Phi_{(n+1)\infty}(x)\cdot\Phi_{n\infty}(y)&=
\bigsqcup_{k=0}^\infty\Phi_{k\infty}(\Phi_{(n+1)(k+1)}(x)(\Phi_{nk}(y)))\\
&=\bigsqcup_{k=0}^n\Phi_{k\infty}x_{i+1}(y_i)\tag{\ref{18.2.8}(1)}\\
&=\Phi_{n\infty}(x_{n+1}(y_n))\tag{\ref{18.2.9}}
\end{align*}
\end{proof}

\begin{proposition}[]
\label{18.2.13}
对于任意\(x,y\in D_\infty\)以及\(n\in\N\)
\begin{enumerate}
\item \((\Phi_{(n+1)\infty}x_n)\cdot y=\Phi_{(n+1)\infty}(x)_{n+1}\cdot\Phi_{n\infty}(y)=\Phi_{n\infty}((x\cdot\Phi_{n\infty}(y))_n)\)
\item \(\Phi_{0\infty}(x_0)\cdot y=\Phi_{0\infty}(x_0)=\Phi_{0\infty}((x\cdot\bot)_0)\)
\end{enumerate}
\end{proposition}

\begin{proof}
\begin{enumerate}
\item \begin{align*}
\Phi_{(n+1)\infty}(x_{n+1})\cdot y&=\bigsqcup_{i=0}^\infty\Phi_{i\infty}((\Phi_{(n+1)\infty}x_{n+1})_{i+1}(y_i))\\
&=\bigsqcup_{i=n}^\infty\Phi_{i\infty}((\Phi_{(n+1)\infty}x_{n+1})_{i+1}(y_i))\tag{\ref{18.2.9}(1)}\\
&=\bigsqcup_{i=n}^\infty\Phi_{n\infty}(x_{n+1}(y_n))\tag{\ref{18.2.9}(2)}\\
&=\Phi_{n\infty}(x_{n+1}(y_n))\tag{\ref{18.2.12}}
\end{align*}
另一方面,
\begin{align*}
\Phi_{n\infty}((x\cdot\Phi_{n\infty}(y))_n)&=\Phi_{n\infty}\left(\left(\bigsqcup_{i=0}^\infty\Phi_{i\infty}((x_{i+1}(\Phi_{n\infty}(y_n))_i))\right)_n\right)\\
&=\Phi_{n\infty}\left(\bigsqcup_{i=0}^\infty\Big(\Phi_{i\infty}((x_{i+1}(\Phi_{n\infty}(y_n))_i))\Big)_n\right)\\
&=\Phi_{n\infty}\left(\bigsqcup_{i=n}^\infty\Big(\Phi_{i\infty}((x_{i+1}(\Phi_{n\infty}(y_n))_i))\Big)_n\right)\\
&=\Phi_{n\infty}\left( \bigsqcup_{i=n}^\infty\Phi_{n\infty}(x_{n+1}(y_n)) \right)\\
&=\Phi_{(n+1)\infty}(x_{n+1})\cdot\Phi_{n\infty}(y_n)
\end{align*}
\item \begin{align*}
 \Phi_{0\infty}(x_0)\cdot y&=\Phi_{1\infty}((\Phi_{0\infty}(x_0))_1)\cdot y\\
 &=\Phi_{0\infty}((\Phi_{0\infty}(x_0))_1((\Phi_{1\infty})(y_0)))\tag{\ref{18.2.12}}\\
 &=\Phi_{0\infty}(\varphi_0(x_0)(y_0))=\Phi_{0\infty}(x_0)
\end{align*}
\end{enumerate}
\end{proof}

\begin{theorem}[外延性]
\label{18.2.14}
对于\(x,y\in D_\infty\)
\begin{enumerate}
\item \(x\sqsubseteq y\Leftrightarrow\forall z\in D_\infty(x\cdot z\sqsubseteq y\cdot z)\)
\item \(x=y\Leftrightarrow\forall z\in D_\infty(x\cdot z=y\cdot z)\)
\end{enumerate}
\end{theorem}

\begin{proof}
\begin{enumerate}
\item \(\Rightarrow\):因为\(\cdot\)是连续的,因此\(\metalambda x.x\cdot z\)是单调的。

\(\Leftarrow\):假设\(\forall z\in D_\infty(x\cdot z\sqsubseteq y\cdot z)\),于是\(x\cdot\bot\sqsubseteq y\cdot\bot\),由命题\ref{18.2.13} (2)得
\begin{equation*}
\Phi_{0\infty}(x_0)=\Phi_{0\infty}((x\cdot\bot)_0)\sqsubseteq\Phi_{0\infty}((y\cdot\bot)_0)=\Phi_{0\infty}(y_0)
\end{equation*}
由于\(x\cdot\Phi_{n\infty}(z_n)\sqsubseteq y\cdot\Phi_{n\infty}(z_n)\),由命题\ref{18.2.12} 和\ref{18.2.13} 得
\begin{equation*}
\Phi_{n\infty}(x_{n+1}(z_n))=\Phi_{n\infty}((x\cdot\Phi_{n\infty}(z_n))_n)
\sqsubseteq\Phi_{n\infty}((y\cdot\Phi_{n\infty}(z_n))_n)=\Phi_{n\infty}(y_{n+1}(z_n))
\end{equation*}
因此
\begin{equation*}
\forall n\in\N\forall z\in D_n(\Phi_{n\infty}(x_{n+1}(z))\sqsubseteq\Phi_{n\infty}(y_{n+1}(z)))
\end{equation*}
即\(\Phi_{n+1}(x_{n+1})\sqsubseteq\Phi_{n+1}(y_{n+1})\),即\(x\sqsubseteq y\)。
\item 由(1)。
\end{enumerate}
\end{proof}

\begin{definition}[\(\lambda\)-模型]
\textbf{\(\lambda\)-模型} 是一个三元组 \(\D=\la D,\cdot,\llb{}\ra\),其中\(\cdot:D^2\to D\),\(\llb{}\)是
 一个从 \(\lambda\)-项\(M\)和它的赋值\(\rho\)到\(D\)的函数,满足
\begin{enumerate}
\item \(\llb{x}_\rho=\rho(x)\)
\item \(\llb{PQ}_{\rho}=\llb{P}_\rho\cdot\llb{Q}_\rho\)
\item 对于所有\(d\in D\),\(\llb{\lambda x.P}_\rho\cdot d=\llb{P}_{[d/x]\rho}\)
\item 如果对所有\(M\)的自由变元\(x\),若\(\rho(x)=\sigma(x)\),则\(\llb{M}_\rho=\llb{M}_\sigma\)
\item 若\(y\)不是\(M\)的自由变元,则\(\llb{\lambda x.M}_\rho=\llb{\lambda y.[y/x]M}_\rho\)
\item 若对所有\(d\in D\)都有\(\llb{P}_{[d/x]\rho}=\llb{Q}_{[d/x]\rho}\),则\(\llb{\lambda x.P}_\rho=\llb{\lambda x.Q}_\rho\)
\end{enumerate}
\end{definition}

\begin{fact}[]
\begin{enumerate}
\item 每一个外延的组合代数\(\la D,\cdot\ra\)可以被唯一地映射到一个 \(\lambda\)-模型\(\la D,\cdot,\Lambda\ra\),其中对所
有\(a\in D\), \(\Lambda(a)=a\)。
\item \(D_\infty\)是组合代数。
\end{enumerate}
\end{fact}

\begin{corollary}[]
\(D_\infty\)是外延的\(\lambda\)-模型。
\end{corollary}

\begin{theorem}[完全性]
\label{18.2.15}
对于\(f\in[D_\infty\to D_\infty]\),定义
\begin{equation*}
\Box f=\bigsqcup_n\Phi_{(n+1)\infty}(\metalambda y\in D_n.(f(y))_n)
\end{equation*}
则
\begin{equation*}
\forall y\in D_\infty(f(y))=\Box f\cdot y
\end{equation*}
\end{theorem}

\begin{proof}
\begin{align*}
\Box f\cdot y&=\bigsqcup_m\Phi_{m\infty}((\Box f)_{m+1}(y_m))=\bigsqcup_m\Phi_{m\infty}((\Box f\cdot\Phi_{m\infty}(y_m))_m)\\
&=\bigsqcup_m\Phi_{m\infty}\left( \left(
\Big(\bigsqcup_n\Phi_{(n+1)\infty}(
\metalambda y\in D_n.(f(y))_n)\Big)\cdot\Phi_{m\infty}(y_m)  \right)_m \right)\\
&=\bigsqcup_{m,n}\Phi_{m\infty}\left(
\left(
\Phi_{(n+1)\infty}(\metalambda y\in D_n.(f(y))_n)\cdot\Phi_{m\infty}(y_m)  \right)_m \right)\\
&=\bigsqcup_m\Phi_{m\infty}\left( \left(
(\metalambda y\in D_m.(f(y))_m)(y_m)  \right)_m \right)\\
&=\bigsqcup_m\Phi_{m\infty}(f(\Phi_{m\infty}(y_m))_m)=\bigsqcup_{k,l}\Phi_{l\infty}((f(\Phi_{k\infty}(y_k)))_l)\\
&=\bigsqcup_k f(\Phi_{k\infty}(y_k))=f(y)
\end{align*}
\end{proof}

\begin{theorem}[]
\(D_\infty\cong[D_\infty\to D_\infty]\)
\end{theorem}

\begin{proof}
对于\(x\in D_\infty\),令\(F(x)=\metalambda y\in D_\infty.x\cdot y\),由定理 \ref{18.2.15} ,\(F\)是满射,由定理
\ref{18.2.14} (2),\(F\)是单射,由命题 \ref{18.2.11} \(F\)连续,\(F\)的逆是
\begin{equation*}
G=\metalambda f.\bigsqcup_n\Phi_{(n+1)\infty}(\metalambda y\in D_n.\Phi_{\infty n}(f(\Phi_{n\infty}(y))))
\end{equation*}
\end{proof}

\nocite{hindley2008lambda}
\nocite{zbMATH03877147}

\label{bibliographystyle link}
\bibliographystyle{acm}

\label{bibliography link}
\bibliography{../../references}
\end{document}
