% Created 2021-10-25 Mon 18:25
% Intended LaTeX compiler: pdflatex
\documentclass[11pt]{article}
\usepackage[utf8]{inputenc}
\usepackage[T1]{fontenc}
\usepackage{graphicx}
\usepackage{longtable}
\usepackage{wrapfig}
\usepackage{rotating}
\usepackage[normalem]{ulem}
\usepackage{amsmath}
\usepackage{amssymb}
\usepackage{capt-of}
\usepackage{hyperref}
\input{../../../preamble-lite.tex}
\author{Qi'ao Chen\\21210160025}
\date{\today}
\title{Exercise}
\hypersetup{
 pdfauthor={Qi'ao Chen\\21210160025},
 pdftitle={Exercise},
 pdfkeywords={},
 pdfsubject={},
 pdfcreator={Emacs 27.2 (Org mode 9.5)}, 
 pdflang={English}}
\begin{document}

\maketitle
\begin{exercise}
Suppose \(G\) is infinite planar
\end{exercise}

\begin{proof}
Let \(\call=\{E,R,W,B,Y\}\),
\begin{align*}
\sigma=\forall x(&(R(x)\wedge\neg W(x)\wedge\neg B(x)\wedge\neg Y(x))\vee\\
&(\neg R(x)\wedge W(x)\wedge\neg B(x)\wedge\neg Y(x))\vee\\
&(\neg R(x)\wedge\neg W(x)\wedge B(x)\wedge\neg Y(x))\vee\\
&\neg (R(x)\wedge\neg W(x)\wedge\neg B(x)\wedge Y(x)))
\end{align*}
\(\sigma_R:\forall x,y(E(x,y)\to\neg (R(x)\wedge R(y)))\) and \(\sigma_W,\sigma_B,\sigma_Y\) similarly.

\(\Diag_{\el}(G)=\{\phi(a_1,\dots,a_n)\mid G\vDash\phi(a_1,\dots,a_n),a_i\in V,\phi\in L\}\)

Let \(L_V=L\cup V\)

Let \(\Sigma=\Diag(G)\cup\{\sigma,\sigma_R,\sigma_W,\sigma_B,\sigma_Y\}\). \(\Sigma\) is finitely satisfiable. For any finite \(\Delta\subset\Diag(G)\),
assume \(a_1,\dots,a_m\) occurs in \(\Delta\), then the subgraph \(T\) of \(G\) with vertices \(a_1,\dots,a_m\)
s.t. \(Ea_ia_j\) in \(T\) iff \(Ea_ia_j\) in \(G\)
is a
model of \(\Delta\). As we can color \(T\) in 4 colors, \(\Delta\) is satisfiable and thus \(\Sigma\) is satisfiable.

Thus \(\Sigma\) has a model \(G'\) with \(f:G\xrightarrow{\prec}G'\) an elementary map. Prove

Let \(f(a)=a^{G'}\). For any \(a_1,a_2\in G\)
\begin{enumerate}
\item If \(a_1,a_2\) are distinct elements of \(G\), then \(a_1\neq a_2\in\Diag_{\el}(G)\). Hence \(f(a_1)\neq f(a_2)\)
\item For any relation \(R\), if \(\bara\in R^G\), then \(R(\bara)\in\Diag_{\el}(G)\),
hence \(f(\bara)\in R^{G'}\)

If \(\bara\notin R^G\), then \(\neg R(\bara)\in\Diag_{\el}(G)\), hence \(f(\bara)\notin R^{G'}\)
\end{enumerate}

As \(G'\) has 4 color, so does \(G\).
\end{proof}
\end{document}
