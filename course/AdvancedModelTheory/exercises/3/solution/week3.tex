% Created 2022-03-15 Tue 14:26
% Intended LaTeX compiler: pdflatex
\documentclass[11pt]{article}
\usepackage[utf8]{inputenc}
\usepackage[T1]{fontenc}
\usepackage{graphicx}
\usepackage{longtable}
\usepackage{wrapfig}
\usepackage{rotating}
\usepackage[normalem]{ulem}
\usepackage{amsmath}
\usepackage{amssymb}
\usepackage{capt-of}
\usepackage{hyperref}
\input{../../../../preamble-lite.tex}
\author{Qi'ao Chen\\21210160025}
\date{\today}
\title{Week3}
\hypersetup{
 pdfauthor={Qi'ao Chen\\21210160025},
 pdftitle={Week3},
 pdfkeywords={},
 pdfsubject={},
 pdfcreator={Emacs 28.0.90 (Org mode 9.6)}, 
 pdflang={English}}
\begin{document}

\maketitle
\begin{exercise}
Show that the collection of formulas \(x>a\) for \(a\in M\) generates a complete
type \(\tau_M(x)\in S_1(M)\). In other words, show that the partial type \(\{(x>a):a\in M\}\) has a unique
completion
\end{exercise}

\begin{proof}
Let \(\Sigma(x)=\{x>a:a\in M\}\), \(p,q\in S_1(M)\), \(p,q\supseteq \Sigma(x)\), \(p\neq q\). Since \(\DLO\) has quantifier
elimination, there is a quantifier free formula \(\varphi(x)\in p\setminus q\). \(\varphi\) has the form
\begin{equation*}
\bigwedge_{a\in A}x>a\wedge\bigwedge_{b\in B}x\le b\wedge\bigwedge_{c\in C}x\neq c\wedge\bigwedge_{d\in D}x=d
\end{equation*}
where \(A,B,C,D\) are finite. But since \(p,q\supseteq\Sigma(x)\), \(B=D=\emptyset\). Also \(x\neq c\) is implied
by \(\Sigma(x)\). Thus if we choose \(a'=\max\{a:a\in A\}\), then
\begin{equation*}
M\vdash\varphi\leftrightarrow x>a'
\end{equation*}
Thus \(\varphi\in q\), a contradiction. Hence \(p=q\)
\end{proof}

\begin{exercise}
Show that \(\tau_M\) is definable
\end{exercise}

\begin{proof}
By exercise 3 for each \(N\succeq M\), \(\tau_M\) has a unique heir and thus \(\tau_M\) is definable
\end{proof}

\begin{exercise}
Suppose \(N\succeq M\). Show that \(\tau_N\) is an heir of \(\tau_M\)
\end{exercise}

\begin{proof}
Let \(q\in S_1(N)\) be an heir of \(\tau_M\) and suppose \(x\le a\in q(x)\)  for some \(a\in N\). Then there
is \(a'\in M\) s.t. \(x\le a'\in\tau_M\), which is impossible. Thus \(\{x>a:a\in N\}\subseteq q\) and \(q=\tau_N\).
Hence \(\tau_N\) is the unique heir of \(\tau_M\) by Exercise 1
\end{proof}

\begin{exercise}
Suppose \(N\succeq M\) and \(N\) is \(\abs{M}^+\)-saturated. Show that \(\tau_N\) is not a coheir of \(\tau_M\)
\end{exercise}

\begin{proof}
Since \(N\) is \(\abs{M}^+\)-saturated, there is \(c\in N\) s.t. \(N\vDash\tau_M(c)\). Then there is
no \(a\in M\) satisfying \(x>c\).
\end{proof}

\begin{exercise}
If \(N\succeq M\), show that \(\tau_M\) has a unique coheirs over \(N\)
\end{exercise}

\begin{proof}
Suppose \(\tau_M\) has two different coheirs \(p,q\in S_1(N)\). Because \(p\) and \(q\) are the same thing
as cuts, we may assume that there is \(c\in N\) s.t. \(x<c\in p\)
and \(x>c\in q\). But for any \(x\) realizing, \(x>m\) for all \(m\in M\). Thus there is
no \(m\in M\) satisfying \(x>c\). Hence \(p\) is not a coheir of \(\tau_M\), a contradiction
\end{proof}

\begin{exercise}
Give an example of models \(M\preceq N\) of \(\DLO\) where \(\tau_N\) is a coheir of \(\tau_M\)
\end{exercise}

\begin{proof}
\(M=\Q\), \(N=\R\)
\end{proof}
\end{document}
