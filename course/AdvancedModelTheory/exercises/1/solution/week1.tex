% Created 2022-03-02 Wed 19:23
% Intended LaTeX compiler: pdflatex
\documentclass[11pt]{article}
\usepackage[utf8]{inputenc}
\usepackage[T1]{fontenc}
\usepackage{graphicx}
\usepackage{longtable}
\usepackage{wrapfig}
\usepackage{rotating}
\usepackage[normalem]{ulem}
\usepackage{amsmath}
\usepackage{amssymb}
\usepackage{capt-of}
\usepackage{hyperref}
\input{../../../../preamble-lite.tex}
\author{Qi'ao Chen\\21210160025}
\date{\today}
\title{Homework 1}
\hypersetup{
 pdfauthor={Qi'ao Chen\\21210160025},
 pdftitle={Homework 1},
 pdfkeywords={},
 pdfsubject={},
 pdfcreator={Emacs 28.0.90 (Org mode 9.6)}, 
 pdflang={English}}
\begin{document}

\maketitle
\begin{exercise}
Consider the structure \((\Z,+,\cdot,<)\). Show that there is a complete type \(p\in S_1(\Z)\) containing the
formula \(n<x\) for each \(n\in\Z\)
\end{exercise}

\begin{proof}
Let \(\Gamma=\{n<x:n\in\Z\}\). Then \(\Gamma\) is finitely satisfiable and hence there is a complete type \(q\in S_1(\Z)\)
s.t. \(q(x)\supset\Gamma\)
\end{proof}

\begin{exercise}
Let \(p\in S_1(\Z)\) be as in the previous problem, meaning that the formula \(n<x\) is in \(p(x)\) for
all \(n\in\Z\). Suppose \(M\succeq\Z\) and \(q\in S_1(M)\) is an heir of \(p\). Show that \(q(x)\) contains the
formula \(n<x\) for each \(n\in M\)
\end{exercise}

\begin{proof}
If for some \(n\in M\), \(\psi(x,n):=n<x\notin q(x)\). Then \(\neg\psi(x,n)\in q(x)\) and hence there is \(n'\in\Z\)
s.t. \(\neg\psi(x,n')\in p\), which is impossible
\end{proof}

\begin{exercise}
Find a first-order formula \(\varphi(x,y,z)\) equivalent to \(\exists^\infty w(xw^2+yw+z=0)\) in the structure \(\C\)
\end{exercise}

\begin{proof}
Let \(\psi(x):=\forall y(y\cdot x=x)\) and let \(\varphi(x,y,z):=\psi(x)\wedge\psi(y)\wedge\psi(z)\)
\end{proof}

\begin{exercise}
Let \(M=\R\setminus[0,2]\) and \(N=\R\setminus[0,1)\). From quantifier elimination in \(\DLO\), one can show
that \((M,\le)\preceq(N,\le)\preceq(\R,\le)\). It turns out that \(\tp(0/N)\) is an heir of \(\tp(0/M)\). Show
that \(\tp(0/N)\) is not a strong heir of \(\tp(0/M)\)
\end{exercise}

\begin{proof}
Let \(p=\tp(0/M)\) and \(q=\tp(0/N)\). Let \(\varphi(x,y):=x>y\), then \((N,dq)\vDash\forall y(d\varphi(y)\leftrightarrow y<1)\).
But in \(M\), for any \(c\in M\), \(M\vDash x<0\not\leftrightarrow x<c\). Thus \((M,dp)\vDash\neg\exists c(\forall y(d\varphi)\leftrightarrow y<c)\).
Hence \(\tp(0/N)\) is not a strong heir of \(\tp(0/M)\)
\end{proof}
\end{document}
