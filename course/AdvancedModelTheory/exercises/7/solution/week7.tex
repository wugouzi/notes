% Created 2022-04-20 Wed 21:50
% Intended LaTeX compiler: pdflatex
\documentclass[11pt]{article}
\usepackage[utf8]{inputenc}
\usepackage[T1]{fontenc}
\usepackage{graphicx}
\usepackage{longtable}
\usepackage{wrapfig}
\usepackage{rotating}
\usepackage[normalem]{ulem}
\usepackage{amsmath}
\usepackage{amssymb}
\usepackage{capt-of}
\usepackage{hyperref}
\input{../../../../preamble-lite.tex}
\author{Qi'ao Chen\\21210160025}
\date{\today}
\title{Week 7}
\hypersetup{
 pdfauthor={Qi'ao Chen\\21210160025},
 pdftitle={Week 7},
 pdfkeywords={},
 pdfsubject={},
 pdfcreator={Emacs 28.0.92 (Org mode 9.6)}, 
 pdflang={English}}
\begin{document}

\maketitle
\begin{exercise}
Suppose \(\sqrt{a}\) and \(\sqrt{b}\) both exist (in \(\M\)). Show that \(\sqrt{a}+\sqrt{b}\in\acl(\{a,b\})\)
\end{exercise}

\begin{proof}
Let \(\varphi(x)\) be \(\exists y,z(y\cdot y=a\wedge z\cdot z=b\wedge x=y+z)\). Then since \(\sqrt{a}\) and \(\sqrt{b}\) are
unique, \(\abs{\varphi(\M)}=1\) and \(\M\vDash\varphi(\sqrt{a}+\sqrt{b})\). Therefore \(\sqrt{a}+\sqrt{b}\in\acl(\{a,b\})\)
\end{proof}

\begin{exercise}
Suppose \(c\neq 0\). Let \(D=\{(x,y,z)\in\M^3:ax+by+cz=0\}\). Show that \((a/c,b/c)\) is a ``code'' for \(D\)
\end{exercise}

\begin{proof}
If \(\sigma\in\Aut(\M)\) fixes \(a/c\) and \(b/c\), then for any \((x,y,z)\in D\), then
\begin{align*}
(a/c)\sigma(x)+(b/c)\sigma(x)+\sigma(z)&=\sigma(ax/c)+\sigma(by/c)+\sigma(z)\\
&=\sigma(ax/c+by/c+z)=\sigma(0)=0
\end{align*}
Therefore \(a\sigma(x)+b\sigma(y)+c\sigma(z)=0\cdot c=0\) and \((\sigma(x),\sigma(y),\sigma(z))\in D\)

If there is \(\sigma\in\Aut(\M)\) that fixes \(D\) and not fixes \(b/c\), then since \(b/c-\sigma(b)/\sigma(c)\neq 0\),  \(a\sigma(x)+b\sigma(y)+c\sigma(z)=0\)
and \(\sigma(a)\sigma(x)+\sigma(b)\sigma(y)+\sigma(c)\sigma(z)=0\), we have
\begin{align*}
&\sigma(y)=\frac{a\sigma(c)-c\sigma(a)}{b\sigma(c)-c\sigma(b)}\sigma(x)\\
&y=\frac{c\sigma^{-1}(a)-a\sigma^{-1}(c)}{c\sigma^{-1}(b)-b\sigma^{-1}(c)}x
\end{align*}
let \(k=\frac{c\sigma^{-1}(a)-a\sigma^{-1}(c)}{c\sigma^{-1}(b)-b\sigma^{-1}(c)}\), then
\begin{equation*}
z=-\frac{a+bk}{c}x
\end{equation*}
But \(D=\{(x,kx,-\frac{a+bk}{c}x):x\in\M\}\subsetneq\{(x,y,z)\in\M^3:ax+by+c=0\}\), therefore we have a
contradiction. Thus for any \(\sigma\in\Aut(\M)\) fixing \(D\), \(\sigma\) fixes \(a/c\) and \(b/c\)
\end{proof}

\begin{exercise}
Let \(D=\M^3\setminus\{(0,0,0)\}\). Let \(E\) be the equivalence relation on \(D\)
where \((a_1,a_2,a_3)E(b_1,b_2,b_3)\) iff the two vectors are parallel. Find a definable
function \(f:D\to\M^n\) s.t.
\begin{equation*}
f(a_1,a_2,a_3)=f(b_1,b_2,b_3)\Leftrightarrow(a_1,a_2,a_3)E(b_1,b_2,b_3)
\end{equation*}
\end{exercise}

\begin{proof}
Let \(f:D\to\M^3\) be
\begin{equation*}
f(a_1,a_2,a_3)=
\begin{cases}
(1,a_2/a_1,a_3/a_1)&a_1\neq 0\\
(0,1,a_3/a_2)&a_1=0\wedge a_2\neq 0\\
(0,0,1)&a_1=0\wedge a_2=0\wedge a_3\neq 0\\
(0,0,0)&\text{otherwise}
\end{cases}
\end{equation*}
Then if \(a_1\neq 0\), \(b_1\neq 0\) for otherwise \(\lambda=0\) and \(b_1=b_2=b_3=0\).
\begin{align*}
(a_1,a_2,a_3)E(b_1,b_2,b_3)&\Leftrightarrow(1,a_2/a_1,a_3/a_1)E(1,b_2/b_1,b_3/b_1)\\
&\Leftrightarrow(1,a_2/a_1,a_3/a_1)=(1,b_2/b_1,b_3/b_1)\\
&\Leftrightarrow f(a_1,a_2,a_3)=f(b_1,b_2,b_3)
\end{align*}
Other cases are similar
\end{proof}

\begin{exercise}
Suppose \(\le\) is a definable linear order on \(\M\). Show that \(\dcl(A)=\acl(A)\) for any \(A\subseteq\M\)
\end{exercise}

\begin{proof}
If \(a\in\acl(A)\), then there is a \(L(A)\)-formula \(\varphi(x)\) s.t. \(\M\vDash\varphi(a)\) and \(\abs{\varphi(\M)}\) is
finite. Suppose \(\varphi(\M)=\{a_1,\dots,a_i,a,a_{i+1},\dots,a_m\}\) with \(a_1<a_2<\dots<a_i<a<a_{i+1}<\dots<a_m\) for
some \(i,m\in\N\). Then if there is \(b\in A\) with \(b<a\) and \(a\) is the \(n\)th elements
greater than \(b\) in \(\varphi(\M)\), then we can take \(\psi(x)\) as
\begin{align*}
\exists x_1,\dots,x_{n-1}\Big(&\varphi(x)\wedge\bigwedge_{i=1}^{n-1}\varphi(x_i)
\\&\wedge(a<x_1<\dots<x_{n-1}<x)
\\&\wedge\bigwedge_{1\le i<j\le n-1}x_i\neq x_j
\\&\wedge\neg\exists y(a< y<x\wedge\varphi(y)\wedge\bigwedge_{i=1}^{n-1}x\neq x_i)\Big)
\end{align*}
and \(\psi(\M)=\{a\}\).

If there is no such \(b\in A\), then there is some \(c\in A\) and \(a\) is the \(n\)th elements lesser
than \(b\) in \(\varphi(\M)\),  then we can take \(\psi(x)\) as
\begin{align*}
\exists x_1,\dots,x_{n-1}\Big(&\varphi(x)\wedge\bigwedge_{i=1}^{n-1}\varphi(x_i)
\\&\wedge(x<x_1<\dots<x_{n-1}<c)
\\&\wedge\bigwedge_{1\le i<j\le n-1}x_i\neq x_j
\\&\wedge\neg\exists y(x< y<c\wedge\varphi(y)\wedge\bigwedge_{i=1}^{n-1}x\neq x_i)\Big)
\end{align*}
and \(\psi(\M)=\{a\}\)
\end{proof}
\end{document}
