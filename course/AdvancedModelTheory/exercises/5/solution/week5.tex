% Created 2022-03-30 Wed 23:01
% Intended LaTeX compiler: pdflatex
\documentclass[11pt]{article}
\usepackage[utf8]{inputenc}
\usepackage[T1]{fontenc}
\usepackage{graphicx}
\usepackage{longtable}
\usepackage{wrapfig}
\usepackage{rotating}
\usepackage[normalem]{ulem}
\usepackage{amsmath}
\usepackage{amssymb}
\usepackage{capt-of}
\usepackage{hyperref}
\input{../../../../preamble-lite.tex}
\author{Qi'ao Chen\\21210160025}
\date{\today}
\title{Week5}
\hypersetup{
 pdfauthor={Qi'ao Chen\\21210160025},
 pdftitle={Week5},
 pdfkeywords={},
 pdfsubject={},
 pdfcreator={Emacs 28.0.92 (Org mode 9.6)}, 
 pdflang={English}}
\begin{document}

\maketitle
\begin{exercise}
Show that \(a_1,a_2,\dots\) is not totally indiscenible
\end{exercise}

\begin{proof}
If \(a_i=a_j\) and \(i<j\), then since \(a_ia_j\equiv a_ma_n\) for any \(m<n\), \(a_1,a_2,\dots\) is a constant sequence.
Because \(a_1,a_2,\dots\) is  a non-constant
indiscenible sequence, either \(a_1<a_2\) or \(a_1>a_2\). We may assume \(a_1<a_2\).
Then \(a_1a_2\not\equiv a_2a_1\) since \(x<y\in\tp(a_1a_2)\) but \(x>y\in\tp(a_2a_1)\)
\end{proof}

\begin{exercise}
Show that \(a_1a_2>0\)
\end{exercise}

\begin{proof}
If \(a_i=0\), then \(x=0\in\tp(a_i)\). But since \(a_i\equiv a_j\) for any \(j\), \(a_1,a_2,\dots\) is a constant
sequence, a contradiction.

If \(a_1a_2<0\), then \(a_2a_3<0\) and so \(a_1a^2_2a_3>0\) which implies \(a_1a_3>0\). But \(a_1a_2\equiv a_1a_3\), we
get a contradiction. Hence \(a_1a_2>0\)
\end{proof}

\begin{exercise}
Suppose \(a_2-a_1\ge 1\). Show that \(a_2-a_1\ge 7\)
\end{exercise}

\begin{proof}
we have \(a_8-a_7\ge 1,a_7-a_6\ge 1,\dots,a_2-a_1\ge 1\), and so \(a_8-a_1\ge 7\). Hence \(a_2-a_1\ge 7\)
\end{proof}

\begin{exercise}
Show that at least one of the following is true: \(a_2<(1.01)\cdot a_1\) or \(a_2>200\cdot a_1\)
\end{exercise}

\begin{proof}
Assume \(a_2\ge (1.01)\cdot a_1\) and\(a_2\le 200\cdot a_1\).

Claim: \(a_{2n}\ge(1.01)^{2n-1}a_1\)

If \(a_{2n}\ge(1.01)^{2n-1}a_1\), then \(a_{2n+2}\ge(1.01)\cdot a_{2n+1}\), \(a_{2n+1}\ge(1.01)\cdot a_{2n}\), and
so \(a_{2n+2}a_{2n+1}a_{2n}\ge(1.01)^{2n+1}a_1a_{2n}a_{2n+1}\).
Since \(a_{2n+1}a_{2n}>0\), \(a_{2n+2}\ge(1.01)^{2n+1}a_1\).

Hence if we take \(N\) large enough, then \(a_{2N}\ge(1.01)^{2N-1}a_1>200\cdot a_1\). Then by
indiscernibility, \(a_2>200\cdot a_1\), a contradiction
\end{proof}


\begin{exercise}
Show that \(a_i+a_j\neq a_k\) for any \(i,j,k\)
\end{exercise}

\begin{proof}
Without loss of generality, we may assume that \(\{a_i,a_j,a_k\}=\{a_1,a_2,a_3\}\) and \(a_1>0\)
\begin{enumerate}
\item If \(a_1+a_2=a_3\). Then \(a_3-a_2=a_1\) and there is \(q\in\Q\) s.t. \(q\le a_1< 1+q\) where \(q>0\).
Then \(a_3-a_2\ge q\). Take \(N=\ucorner{\frac{1+q}{q}}\),
since \(a_{N+2}-a_{N+1}\ge q,a_{N+1}-a_N\ge q,a\dots,a_3-a_2\ge q\), we have \(a_{N+2}-a_2\ge Nq\ge 1+q> a_1\).
Hence \(a_3-a_2> a_1\), a contradiction
\item If \(a_1+a_3=a_2\), take \(a_2-a_1=a_3\) and we can prove similarly
\item If \(a_2+a_3=a_1\). Then \(a_2-a_1=-a_3\) and there \(q\in\Q\) s.t. \(-1-q<-a_3\le -q\) where \(q>0\).
Similarly we can prove that \(a_2-a_1<-a_3\).
\end{enumerate}


Therefore \(a_i+a_j\neq a_k\)
\end{proof}

\begin{exercise}
Show that there is an indiscenible sequence \(b_1,b_2,b_3,\dots\) s.t. \(b_2>200\cdot b_1\)
\end{exercise}

\begin{proof}
Let \((a_i:i\in\N\setminus\{0\})\) be an infinite sequence s.t. \(a_i=201^i\), \(i\in\N\). Then by Theorem 10 in the notes,
there is an indiscenible sequence \((b_j:j\in\N\setminus\{0\})\) with \(\tp^{EM}(\barb)=\tp^{EM}(\bara)\). Since
in \(\bara\), for any \(j>i\), \(a_j>200\cdot a_i\), therefore \(b_m>200\cdot b_n\) for any \(m>n\).
Particularly, \(b_2>200\cdot b_1\).
\end{proof}
\end{document}
