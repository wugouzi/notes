% Created 2022-03-08 Tue 17:12
% Intended LaTeX compiler: pdflatex
\documentclass[presentation]{beamer}
\usepackage[utf8]{inputenc}
\usepackage[T1]{fontenc}
\usepackage{graphicx}
\usepackage{longtable}
\usepackage{wrapfig}
\usepackage{rotating}
\usepackage[normalem]{ulem}
\usepackage{amsmath}
\usepackage{amssymb}
\usepackage{capt-of}
\usepackage{hyperref}
\mode<beamer>{\usetheme{Madrid}}
% TIPS
% \substack{a\\b} for multiple lines text





% pdfplots will load xolor automatically without option
\usepackage[dvipsnames]{xcolor}

\usepackage{forest}
% two-line text in node by [two \\ lines]
% \begin{forest} qtree, [..] \end{forest}
\forestset{
  qtree/.style={
    baseline,
    for tree={
      parent anchor=south,
      child anchor=north,
      align=center,
      inner sep=1pt,
    }}}
%\usepackage{flexisym}
% load order of mathtools and mathabx, otherwise conflict overbrace

\usepackage{mathtools}
%\usepackage{fourier}
\usepackage{pgfplots}
\usepackage{amsthm, mathabx,  amsmath, commath}
\usepackage{amsfonts}

\usepackage{empheq}
\usepackage{tikz}
\usetikzlibrary{arrows.meta}
\usepackage[most]{tcolorbox}

\newtheorem{theorem}{Theorem}[section]
\newtheorem{definition}{Definition}[section]
\newtheorem{corollary}{Corollary}[section]
\newtheorem{example}{Example}[section]
\newtheorem{lemma}{Lemma}[section]
\newtheorem{proposition}{Proposition}[section]

\newcommand{\bl}[1] {\boldsymbol{#1}}
\newcommand{\Wt}[1] {\stackrel{\sim}{\smash{#1}\rule{0pt}{1.1ex}}}
\newcommand{\wt}[1] {\widetilde{#1}}


%For boxed texts in align, use Aboxed{}
%otherwise use boxed{}

\DeclareMathSymbol{\widehatsym}{\mathord}{largesymbols}{"62}
\newcommand\lowerwidehatsym{%
  \text{\smash{\raisebox{-1.3ex}{%
    $\widehatsym$}}}}
\newcommand\fixwidehat[1]{%
  \mathchoice
    {\accentset{\displaystyle\lowerwidehatsym}{#1}}
    {\accentset{\textstyle\lowerwidehatsym}{#1}}
    {\accentset{\scriptstyle\lowerwidehatsym}{#1}}
    {\accentset{\scriptscriptstyle\lowerwidehatsym}{#1}}
}

\usepackage{graphicx}
    
% text on arrow for xRightarrow
\makeatletter
%\newcommand{\xRightarrow}[2][]{\ext@arrow 0359\Rightarrowfill@{#1}{#2}}
\makeatother


\def \bx {\boldsymbol{x}}
\def \ba {\boldsymbol{a}}
\def \bI {\boldsymbol{I}}
\def \bt {\boldsymbol{t}}
\def \bb {\boldsymbol{b}}
\def \bA {\boldsymbol{A}}
\def \bX {\boldsymbol{X}}
\def \bu {\boldsymbol{u}}
\def \bS {\boldsymbol{S}}
\def \bZ {\boldsymbol{Z}}
\def \bz {\boldsymbol{z}}
\def \by {\boldsymbol{y}}
\def \bw {\boldsymbol{w}}
\def \bT {\boldsymbol{T}}
\def \bS {\boldsymbol{S}}
\def \bm {\boldsymbol{m}}
\def \bW {\boldsymbol{W}}
\def \bY {\boldsymbol{Y}}
\def \bH {\boldsymbol{H}}
\def \blambda {\boldsymbol{\lambda}}
\def \bPhi {\boldsymbol{\Phi}}
\def \btheta {\boldsymbol{\theta}}
\def \bmu {\boldsymbol{\mu}}
\def \bphi {\boldsymbol{\phi}}
\def \bSigma {\boldsymbol{\Sigma}}
\def \lb {\left\{}
\def \rb {\right\}}
\def \caln {\mathcal{N}}
\def \dissum {\displaystyle\Sigma}
\def \dispro {\displaystyle\prod}
\def \E {\mathbb{E}}
\def \Q {\mathbb{Q}}
\def \V {\mathbb{V}}
\def \R {\mathbb{R}}
\def \calq {\mathcal{Q}}
\def \calg {\mathcal{G}}
\def \caln {\mathcal{N}}
\def \calr {\mathcal{R}}
\def \calm {\mathcal{M}}
\def \calc {\mathcal{C}}
\def \bcup {\bigcup}

\def \TIME {\text{TIME}}
\def \EXP {\textbf{EXP}}
\def \SPACE {\textbf{SPACE}}
\def \PSPACE {\textbf{PSPACE}}
\def \NPSPACE {\textbf{NPSPACE}}
\def \NSPACE {\textbf{NSPACE}}
\def \coNSPACE {\textbf{coNSPACE}}
\def \NTIME {\textbf{NTIME}}
\def \NP {\textbf{NP}}
\def \coNP {\textbf{coNP}}
\def \NEXP {\textbf{NEXP}}
\def \NE {\textbf{NE}}
\def \NL {\textbf{NL}}
\def \coNL {\textbf{coNL}}
\def \Pspoly {\textbf{P}/poly}
\def \AC {\text{AC}}
\def \BPP {\textbf{BPP}}
\def \start {\text{start}}
\def \tend {\text{end}}
\def \halt {\text{halt}}
\def \pad {\text{pad}}
\def \HALT {\text{HALT}}
\def \DTIME {\textbf{DTIME}}
\def \NP {\textbf{NP}}
\def \INDSET {\texttt{INDSET}}
\def \accept {\text{accept}}
\def \TMSAT {\texttt{TMSAT}}
\def \SAT {\texttt{SAT}}
\def \TSAT {\texttt{3SAT}}
\def \ZOIPROG {\texttt{1/0 IPROG}}
\def \dHAMPATH {\texttt{dHAMPATH}}
\def \TAUTOLOGY {\texttt{TAUTOLOGY}}
\def \PATH {\texttt{PATH}}
\def \TQBF {\texttt{TQBF}}
\usetheme{default}
\author{Qi'ao Chen\\21210160025@m.fudan.edu.cn}
\date{\today}
\title{Cook-Levin Theorem}
\hypersetup{
 pdfauthor={Qi'ao Chen\\21210160025@m.fudan.edu.cn},
 pdftitle={Cook-Levin Theorem},
 pdfkeywords={},
 pdfsubject={},
 pdfcreator={Emacs 28.0.90 (Org mode 9.6)}, 
 pdflang={English}}
\begin{document}

\maketitle
\begin{frame}{Outline}
\tableofcontents
\end{frame}

\section{Goal}
\label{sec:orgb52ab98}
\begin{frame}[label={sec:org7f66a30}]{​​}
\begin{theorem}[Cook-Levin Theorem]
\begin{enumerate}
\item \(\SAT\) is \(\NP\)-complete
\item \(\TSAT\) is \(\NP\)-complete
\end{enumerate}
\end{theorem}
\end{frame}


\begin{frame}[label={sec:org14b6eb5}]{The web of reductions}
\begin{figure}[htbp]
\centering
\includegraphics[width=.8\textwidth]{./3.pdf}
\label{}
\end{figure}
\end{frame}


\section{Intro}
\label{sec:orge010cbc}
\begin{frame}[label={sec:org2c13950}]{Turing machine}
\begin{definition}[]
A TM \(M\) is described by a tuple \((\Gamma,Q,\delta)\) containing
\begin{itemize}
\item A finite set \(\Gamma\) of the symbols that \(M\)'s tapes can contain. We assume that \(\Gamma\) contains a
designated ``blank'' symbol, denoted \(\Box\); a designated ``start'' symbol, denoted \(\rhd\);
and the numbers 0 and 1. We call \(\Gamma\) the \alert{alphabet} of \(M\)
\item A finite set \(Q\) of possible states \(M\)' register can be in. We assume that \(Q\) contains
a designated start state, denoted \(q_{\start}\), and a designated halting state, denoted \(q_{\halt}\)
\item A function \(\delta:Q\times\Gamma^k\to Q\times\Gamma^{k-1}\times\{\text{L,S,R}\}^k\),
where \(k\ge2\), describing the rules \(M\) use in performing each step. This function is
called the \alert{transition function} of \(M\)
\end{itemize}
\end{definition}
\end{frame}
\begin{frame}[label={sec:org3f2060f}]{Turing machine}
\begin{figure}[htbp]
\centering
\includegraphics[width=.7\textwidth]{./6.png}
\label{}
\end{figure}
\end{frame}
\begin{frame}[label={sec:orgf190eac}]{Efficiency and running time}
\begin{definition}[Computing a function and running time]
Let \(f:\{0,1\}^*\to\{0,1\}^*\) and let \(T:\N\to\N\) be some functions, and let \(M\) be a Turing
machine. We say that \(M\) \alert{computes} \(f\) if for every \(x\in\{0,1\}^*\) whenever \(M\) is
initialized to the start configuration on input \(x\), then it halts with \(f(x)\) written on
its output tape. We say \(M\) \alert{computes} \(f\) in \alert{\(T(\abs{x})\)-time} if its computation on every
input \(x\) requires at most \(T(\abs{x})\) steps
\end{definition}
\end{frame}
\begin{frame}[label={sec:org75a3af5}]{The class \texorpdfstring{\(\bP\)}{P}}
A \alert{complexity class} is a set of functions that can be computed within given resource bounds. We
say that a machine \alert{decides} a language \(L\subseteq\{0,1\}^*\) if it computes the
function \(f_L:\{0,1\}^*\to\{0,1\}\) where \(f_L(x)=1\Leftrightarrow x\in L\)

\begin{definition}[]
Let \(T:\N\to\N\) be some function. A language \(L\) is in \(\DTIME(T(n))\) iff there is a
deterministic Turing machine that runs in time \(c\cdot T(n)\) for some constant \(c>0\) and decides \(L\)
\end{definition}

\begin{definition}[]
\(\bP=\bigcup_{c\ge 1}\DTIME(n^c)\)
\end{definition}
\end{frame}
\begin{frame}[label={sec:org5c50345}]{The class \texorpdfstring{\(\NP\)}{NP}}
\begin{definition}[]
A language \(L\subseteq\{0,1\}^*\) is in \(\NP\) if there exists a polynomial function \(p:\N\to\N\) and a
polynomial-time TM \(M\) (called the \alert{verifier} for \(L\)) such that for every \(x\in\{0,1\}^*\),
\begin{equation*}
x\in L\Leftrightarrow\exists u\in\{0,1\}^{p(\abs{x})} \text{ s.t. } M(x,u)=1
\end{equation*}
If \(x\in L\) and \(u\in\{0,1\}^{p(\abs{x})}\) satisfy \(M(x,u)=1\), then we call \(u\) a \alert{certificate}
for \(x\) w.r.t. \(L\) and \(M\)
\end{definition}
\end{frame}
\begin{frame}[label={sec:org65a8a6d}]{Non-deterministic Turing machine}
\begin{definition}[]
\alert{Non-deterministic Turing machine} has \alert{two} transition function \(\delta_0\) and \(\delta_1\), and a special state denoted
by \(q_{\accept}\). When an NDTM \(M\) computes a function, we envision that at each
computational step \(M\) makes an arbitrary choice at to which of its two transition functions
to apply. For every input \(x\), we say that \(M(x)=1\) if there \alert{exists} some sequence of this
choices that would make \(M\) reach \(q_{\accept}\) on input \(x\). We say that \(M\) runs
in \(T(n)\) time if for every input \(x\in\{0,1\}^*\) and every sequence of nondeterministic
choices, \(M\) reaches the halting state or \(q_{\accept}\) within \(T(\abs{x})\) steps
\end{definition}
\end{frame}

\begin{frame}[label={sec:orgbb9d0bb}]{The class \texorpdfstring{\(\NP\)}{NP}}
\begin{definition}[]
For every function \(T:\N\to\N\) and \(L\subseteq\{0,1\}^*\), we say that \(L\in\NTIME(T(n))\) if there is a
constant \(c>0\) and a \(c\cdot T(n)\)-time NDTM \(M\) s.t. for
every \(x\in\{0,1\}^*\), \(x\in L\Leftrightarrow M(x)=1\)
\end{definition}

\begin{theorem}[]
\(\NP=\bigcup_{c\in\N}\NTIME(n^c)\)
\end{theorem}

\begin{proof}
The main idea is that the sequence of nondeterministic choices made by an accepting computation
of an NDTM  can be viewed as a certificate that the input is in the language, and vice versa
\end{proof}
\end{frame}

\begin{frame}[label={sec:org9508fec}]{Reducibility}
\begin{definition}[]
A language \(L\subseteq\{0,1\}^*\) is \alert{polynomial-time Karp reducible to a
language} \(L'\subseteq\{0,1\}^*\) (sometimes shortened to just ``polynomial-time reducible''), denoted
by \(L\le_p L'\) if there is a polynomial-time
computable function \(f:\{0,1\}^*\to\{0,1\}^*\) s.t. for every \(x\in\{0,1\}^*\),
\(x\in L\) iff \(f(x)\in L'\)

We say that \(L'\) is \alert{\(\NP\)-hard} if \(L\le_pL'\) for every \(L\in\NP\). We say that \(L'\)
is \alert{\(\NP\)-complete} if \(L'\) is \(\NP\)-hard and \(L'\in\NP\)
\end{definition}
\end{frame}
\section{Cook-Levin Theorem}
\label{sec:org48127eb}
\begin{frame}[label={sec:orge03247f}]{Goal}
We denote by \(\SAT\) the language of all satisfiable CNF (conjunction normal form) formulae and by \(\TSAT\) the
language of all satisfiable 3CNF formulae

\begin{theorem}[Cook-Levin Theorem]
\begin{enumerate}
\item \(\SAT\) is \(\NP\)-complete
\item \(\TSAT\) is \(\NP\)-complete
\end{enumerate}
\end{theorem}
\end{frame}

\begin{frame}[label={sec:org7595a6b}]{Oblivious Turing machine}
\begin{definition}[]
Define a TM \(M\) to be \alert{oblivious} if its head movements do not depend on the input but only on
the input length. That is, \(M\) is oblivious if for every input \(x\in\{0,1\}^*\) and \(i\in\N\), the
location of each of \(M\)'s heads at the \(i\)th step of execution on input \(x\) is only a
function of \(\abs{x}\) and \(i\).
\end{definition}

\begin{theorem}[]
For any Turing machine \(M\) that decides a language in time \(T(n)\), there exists an oblivious
Turing machine that decides the same language in \(T(n)^2\)
\end{theorem}
\end{frame}
\begin{frame}[label={sec:org57495ff}]{A lemma}
\begin{lemma}[]
For every Boolean function \(f:\{0,1\}^l\to\{0,1\}\), there is an \(l\)-variable CNF formula \(\varphi\)
of size \(l2^l\) s.t. \(\varphi(u)=f(u)\) for every \(u\in\{0,1\}^l\), where the size of a CNF
formula is defined to be the number of \(\wedge/\vee\) symbols it contains
\end{lemma}

\begin{proof}
For every \(v\in\{0,1\}^l\), there exists a clause \(C_v(z_1,\dots,z_l)\) s.t. \(C_v(v)=0\)
and \(C_v(u)=1\) for every \(u\neq v\).

We let \(\varphi\) be the AND of all the clauses \(C_v\) for \(v\) s.t. \(f(v)=0\)
     \begin{equation*}
\varphi=\bigwedge_{v:f(v)=0}C_v(z_1,\dots,z_l)
     \end{equation*}
Note that \(\varphi\) has size at most \(l2^l\).
\end{proof}
\end{frame}

\begin{frame}[label={sec:org47a9003}]{Main lemma}
\begin{lemma}[]
\(\SAT\) is \(\NP\)-hard
\end{lemma}

\begin{proof}
Let \(L\) be an \(\NP\) language. By definition, there is a polynomial time TM \(M\) s.t. for
every \(x\in\{0,1\}^*\), \(x\in L\Leftrightarrow M(x,u)=1\) for
some \(u\in\{0,1\}^{p(\abs{x})}\), where \(p:\N\to\N\) is some polynomial. We show \(L\) is
polynomial-time Karp reducible to \(\SAT\) by describing a \alert{polynomial-time
transformation} \(x\to\varphi_x\) from strings to CNF formulae s.t. \(x\in L\) iff \(\varphi_x\)
is satisfiable. Equivalently
     \begin{equation*}
\varphi_x\in\SAT \quad\text{ iff }\quad\exists u\in\{0,1\}^{p(\abs{x})}
\text{ s.t. }M(x\circ u)=1
     \end{equation*}
where \(\circ\) denotes concatenation
\end{proof}
\end{frame}
\begin{frame}[label={sec:org6486699}]{Assumption}
Assume
\begin{enumerate}
\item \(M\) only has two tapes - an input tape and a work/output tape
\item \(M\) is an oblivious TM in the sense that its head movement does not depend on the contents
of its tapes. That is, \(M\)'s computation takes the same time for all inputs of size \(n\),
and for every \(i\) the location of \(M\)'s head at the \(i\)th step depends only on \(i\)
and the length of the input
\end{enumerate}
\end{frame}
\begin{frame}[label={sec:orgcd92e9f}]{Proof}
Denote by \(Q\) the set of \(M\)'s possible states and by \(\Gamma\) its alphabet. The \alert{snapshot}
of \(M\)'s execution on some input \(y\) at a particular step \(i\) is the triple
\(\la a,b,q\ra\in\Gamma\times\Gamma\times Q\) s.t. \(a,b\) are the symbols read by \(M\)'s
heads from the two tapes and \(q\) is the state \(M\) is in at the \(i\)th step. Clearly the
snapshot can be encoded as a binary string. Let \(c\) denote the length of this string, which
is some constant depending upon \(\abs{Q}\) and \(\abs{\Gamma}\)
\end{frame}

\begin{frame}[label={sec:org1741654}]{Proof}
For every \(y\in\{0,1\}^*\), the snapshot of \(M\)'s execution on input \(y\) at the \(i\)th
step depends on its state in the \((i-1)\)st step and the contents of the current cells of its
input and work tapes.

And it suffices to check that for each \(i\le T(n)\), the snapshot \(z_i\) is correct
given the snapshot for the previous \(i-1\) steps.
\end{frame}
\begin{frame}[label={sec:orgcc11729}]{Proof}
However, since the TM can only read/modify
 one bit at a time, to check the correctness of \(z_i\) it suffices to look at only \emph{two} of the
 previous snapshots. Specifically, to check \(z_i\) we need to only look at the following:
 \(z_{i-1}\), \(y_{\text{inputpos}(i)}\), \(z_{\text{prev}(i)}\).

Here \(y\) is a shorthand
for \(x\circ u\). \(\text{inputpos}(i)\) denotes the location of \(M\)'s input tape head at
the \(i\)th step. \(\text{prev}(i)\) is the last step before \(i\) when \(M\)'s head was in the
same cell on its work tape that it is during step \(i\).
\end{frame}
\begin{frame}[label={sec:orgbfac464}]{Proof}
Since \(M\) is a deterministic TM, for every triple of values
to \(z_{i-1},y_{\text{inputpos}(i)}\), \(z_{\text{prev}(i)}\), there is at most one value
of \(z_i\) that is correct. Thus there is some function \(F\) that maps \(\{0,1\}^{2c+1}\)
to \(\{0,1\}^c\) s.t. a correct \(z_i\) satisfies
     \begin{equation*}
z_i=F(z_{i-1},z_{\text{prev}(i)},y_{\text{inputpos}(i)})
     \end{equation*}

Because \(M\) is oblivious, the values \(\text{inputpos}(i)\) and \(\text{prev}(i)\) do not
depend on the particular input \(i\). These indices can be computed in polynomial-time by
simulating \(M\) on a trivial input.
\end{frame}
\begin{frame}[label={sec:orgc1aa9b0}]{Proof}
\begin{figure}[htbp]
\centering
\includegraphics[width=.8\textwidth]{./2.png}
\label{}
\end{figure}
\end{frame}
\begin{frame}[label={sec:org42502b6}]{Proof}
Now \(M(x\circ u)=1\) for some \(u\in\{0,1\}^{p(n)}\) iff
 there exists a string \(y\in\{0,1\}^{n+p(n)}\) and a sequence of strings
 \(z_1,\dots,z_{T(n)}\in\{0,1\}^c\) (where \(T(n)\) is the number of steps \(M\) takes on inputs
 of length \(n+p(n)\)) satisfying the following conditions
\begin{enumerate}
\item The first \(n\) bits of \(y\) are equal to \(x\)
\item The string \(z_1\) encodes the initial snapshot of \(M\). That is, \(z_1\) encodes the
triple \(\la\rhd,\Box,q_{\start}\ra\).
\item For every \(i\in\{2,\dots,T(n)\}\), \(z_i=F(z_{i-1},z_{\text{prev}(i)},y_{\text{inputpos}(i)})\).
\item The last string \(z_{T(n)}\) encodes a snapshot where the machine halts and outputs 1
\end{enumerate}
\end{frame}
\begin{frame}[label={sec:orga15d7da}]{Analysis}
\begin{itemize}
\item The formula \(\varphi_x\) will take variables \(y\in\{0,1\}^{n+p(n)}\)
and \(z\in\{0,1\}^{cT(n)}\).

\item Condition 1 can be expressed as a CNF formula of size \(4n\)
\item Conditions 2 and 4 each depend on \(c\) variables and hence can be expressed by CNF formulae of
size \(c2^c\)
\item Condition 3, which is an AND of \(T(n)\) conditions each  depending on at most \(3c+1\)
variables, can be expressed as a CNF formula of size at most \(T(n)(3c+1)2^{3c+1}\).
\item ALL these conditions can be expressed as a CNF formula of size \(d(n+T(n))\) where d is some constant
\item this CNF formula can be computed in time polynomial in the running time of \(M\).
\end{itemize}
\end{frame}
\begin{frame}[label={sec:org5d7e086}]{3SAT}
\begin{lemma}[]
\(\SAT\le_p\TSAT\)
\end{lemma}

\begin{proof}
Suppose \(\varphi\) is a 4CNF. Let \(C\) be a clause of \(\varphi\), say \(C=u_1\vee\baru_2\vee\baru_3\vee u_4\).
We add a new variable \(z\) to the \(\varphi\) and replace \(C\) with the pair
\(C_1=u_1\vee\baru_2\vee z\) and \(C_2=\baru_3\vee u_4\vee\barz\). If \(C\) is true, then there
is an assignment to \(z\) that satisfies both \(C_1\) and \(C_2\). If \(C\) is false, then no
matter what value we assign to \(z\) either \(C_1\) or \(C_2\) will be false.


For every clause \(C\) of size \(k>3\), we change it into an equivalent pair of clauses \(C_1\)
of size \(k-1\) and \(C_2\) of size 3.
\end{proof}
\end{frame}
\begin{frame}[label={sec:org5a3a71b}]{The web of reductions}
\begin{figure}[htbp]
\centering
\includegraphics[width=.8\textwidth]{./3.pdf}
\label{}
\end{figure}
\end{frame}
\end{document}
