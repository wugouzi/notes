% Created 2022-03-26 Sat 20:44
% Intended LaTeX compiler: pdflatex
\documentclass[presentation]{beamer}
\usepackage[utf8]{inputenc}
\usepackage[T1]{fontenc}
\usepackage{graphicx}
\usepackage{longtable}
\usepackage{wrapfig}
\usepackage{rotating}
\usepackage[normalem]{ulem}
\usepackage{amsmath}
\usepackage{amssymb}
\usepackage{capt-of}
\usepackage{hyperref}
\mode<beamer>{\usetheme{Madrid}}
% TIPS
% \substack{a\\b} for multiple lines text





% pdfplots will load xolor automatically without option
\usepackage[dvipsnames]{xcolor}

\usepackage{forest}
% two-line text in node by [two \\ lines]
% \begin{forest} qtree, [..] \end{forest}
\forestset{
  qtree/.style={
    baseline,
    for tree={
      parent anchor=south,
      child anchor=north,
      align=center,
      inner sep=1pt,
    }}}
%\usepackage{flexisym}
% load order of mathtools and mathabx, otherwise conflict overbrace

\usepackage{mathtools}
%\usepackage{fourier}
\usepackage{pgfplots}
\usepackage{amsthm, mathabx,  amsmath, commath}
\usepackage{amsfonts}

\usepackage{empheq}
\usepackage{tikz}
\usetikzlibrary{arrows.meta}
\usepackage[most]{tcolorbox}

\newtheorem{theorem}{Theorem}[section]
\newtheorem{definition}{Definition}[section]
\newtheorem{corollary}{Corollary}[section]
\newtheorem{example}{Example}[section]
\newtheorem{lemma}{Lemma}[section]
\newtheorem{proposition}{Proposition}[section]

\newcommand{\bl}[1] {\boldsymbol{#1}}
\newcommand{\Wt}[1] {\stackrel{\sim}{\smash{#1}\rule{0pt}{1.1ex}}}
\newcommand{\wt}[1] {\widetilde{#1}}


%For boxed texts in align, use Aboxed{}
%otherwise use boxed{}

\DeclareMathSymbol{\widehatsym}{\mathord}{largesymbols}{"62}
\newcommand\lowerwidehatsym{%
  \text{\smash{\raisebox{-1.3ex}{%
    $\widehatsym$}}}}
\newcommand\fixwidehat[1]{%
  \mathchoice
    {\accentset{\displaystyle\lowerwidehatsym}{#1}}
    {\accentset{\textstyle\lowerwidehatsym}{#1}}
    {\accentset{\scriptstyle\lowerwidehatsym}{#1}}
    {\accentset{\scriptscriptstyle\lowerwidehatsym}{#1}}
}

\usepackage{graphicx}
    
% text on arrow for xRightarrow
\makeatletter
%\newcommand{\xRightarrow}[2][]{\ext@arrow 0359\Rightarrowfill@{#1}{#2}}
\makeatother


\def \bx {\boldsymbol{x}}
\def \ba {\boldsymbol{a}}
\def \bI {\boldsymbol{I}}
\def \bt {\boldsymbol{t}}
\def \bb {\boldsymbol{b}}
\def \bA {\boldsymbol{A}}
\def \bX {\boldsymbol{X}}
\def \bu {\boldsymbol{u}}
\def \bS {\boldsymbol{S}}
\def \bZ {\boldsymbol{Z}}
\def \bz {\boldsymbol{z}}
\def \by {\boldsymbol{y}}
\def \bw {\boldsymbol{w}}
\def \bT {\boldsymbol{T}}
\def \bS {\boldsymbol{S}}
\def \bm {\boldsymbol{m}}
\def \bW {\boldsymbol{W}}
\def \bY {\boldsymbol{Y}}
\def \bH {\boldsymbol{H}}
\def \blambda {\boldsymbol{\lambda}}
\def \bPhi {\boldsymbol{\Phi}}
\def \btheta {\boldsymbol{\theta}}
\def \bmu {\boldsymbol{\mu}}
\def \bphi {\boldsymbol{\phi}}
\def \bSigma {\boldsymbol{\Sigma}}
\def \lb {\left\{}
\def \rb {\right\}}
\def \caln {\mathcal{N}}
\def \dissum {\displaystyle\Sigma}
\def \dispro {\displaystyle\prod}
\def \E {\mathbb{E}}
\def \Q {\mathbb{Q}}
\def \V {\mathbb{V}}
\def \R {\mathbb{R}}
\def \calq {\mathcal{Q}}
\def \calg {\mathcal{G}}
\def \caln {\mathcal{N}}
\def \calr {\mathcal{R}}
\def \calm {\mathcal{M}}
\def \calc {\mathcal{C}}
\def \bcup {\bigcup}

\usepackage{ebproof}
\usepackage{syntax}
\makeindex
\usetheme{default}
\author{Qi'ao Chen\\21210160025@m.fudan.edu.cn}
\date{\today}
\title{Hoare Logic and Program Verification}
\hypersetup{
 pdfauthor={Qi'ao Chen\\21210160025@m.fudan.edu.cn},
 pdftitle={Hoare Logic and Program Verification},
 pdfkeywords={},
 pdfsubject={},
 pdfcreator={Emacs 28.0.90 (Org mode 9.6)}, 
 pdflang={English}}
\begin{document}

\maketitle
\begin{frame}{Outline}
\tableofcontents
\end{frame}

\section{Introduction}
\label{sec:org675ebdb}
\begin{frame}[label={sec:orgbc8549e}]{Bird's Eye View}
\begin{itemize}
\item Also known as \alert{Floyd Hoare Logic} is a formal system for reasoning rigorously abotu the
correctness of computer programs
\item First proposed by C. A. R. Hoare (Turing Award, 1980)
\item Original Idea seeded by Robert Floyd (Turing Award, 1978)
\end{itemize}
\end{frame}
\begin{frame}[label={sec:org14cf818}]{Formally}
\begin{itemize}
\item A Proof System for reasoning about \alert{partial corretness} of certain kinds of programs
\begin{itemize}
\item set of axioms
\item rules of inference
\item underlying logic
\end{itemize}
\item \alert{Motivation}: Assertion checking in (sequential) programs
\end{itemize}
\end{frame}
\begin{frame}[label={sec:orgcae9782},fragile]{What does a program like}
\end{frame}
\section{Preliminaries}
\label{sec:orgcf696da}
\begin{frame}[label={sec:orgef21826}]{Backus–Naur form}
\begin{itemize}
\item \alert{Backus–Naur form} or \alert{Backus normal form} (BNF) is a metasyntax notation for Chomsky's context-free
grammars, often used to describe the syntax of languages used in computing
\item Context-free grammar has the same computability as pushdown automata \href{https://knightscholar.geneseo.edu/cgi/viewcontent.cgi?article=1001\&context=computability-oer}{(a proof)}
\end{itemize}
\end{frame}
\begin{frame}[label={sec:org5d673fa},fragile]{Example}
 \begin{grammar}
<S> ::= `-' <FN> | <FN>

<FN> ::= <DL> | <DL> `.' <DL>

<DL> ::= <D>|<D> <DL>

<D> ::=  `0' | `1'|`2'|`3'|`4'|`5'|`6'|`7'|`8'|`9'
\end{grammar}
\bigskip Here \texttt{S} is the start symbol, \texttt{FN} products a fractional number, \texttt{DL} is a digit list, while \texttt{D} is a digit
Then for \texttt{S}, we have
\begin{center}
\texttt{S} \(\Rightarrow\) \texttt{FN} \(\Rightarrow\)  \texttt{DL . DL} \(\Rightarrow\) \texttt{D . DL} \(\Rightarrow\) \texttt{3 . DL} \(\Rightarrow\)\par \texttt{3 . D DL} \(\Rightarrow\) \texttt{3 . D D}  \(\Rightarrow\)
\texttt{3 . 1 D} \(\Rightarrow\) \texttt{3 . 1 4}
\end{center}
\end{frame}
\begin{frame}[label={sec:orgb7cd74a}]{A simple imperative language}
\begin{itemize}
\item Expressions
\begin{equation*}
E\;::=\;n\mid x\mid -E\mid E+E\mid\dots
\end{equation*}
\item Boolean Conditions
\begin{equation*}
B\;::=\;\texttt{true}\mid E=E\mid E>= E\mid\neg B\mid B\wedge B
\end{equation*}
\item Program Statements
\begin{equation*}
P\;::=\; x:= E\mid P;P\mid\texttt{ if }B\texttt{ then }P\texttt{ else }P\mid\texttt{ while }B\; P
\end{equation*}
\end{itemize}
\end{frame}
\begin{frame}[label={sec:orga1d6c44}]{A simple assertion language}
\alert{Assertion} : A logical formula describing a set of valuations on program variables with some
\emph{interesting} property.

Expressed in the underlying logic (FO here)

\begin{itemize}
\item Expressions
\begin{equation*}
E\;::=\;n\mid x\mid -E\mid E+E\mid\dots
\end{equation*}
Here the set of variables is not restricted to the set of program variables
\item Basic Propositions
\begin{equation*}
E\;::=\; E=E\mid E>= E
\end{equation*}
\item Assertions
\begin{equation*}
A\;::=\;\texttt{true}\mid B\mid\neg A\mid A\wedge A\mid\forall v\;A
\end{equation*}
\end{itemize}
\end{frame}
\begin{frame}[label={sec:org1fb7fcc}]{Assertion Semantics}
\begin{itemize}
\item As program executes, the valuation of variables (read \alert{state}) changes
\item An execution of a program statement, transforms one state to another state
\item At some point during execution, let the state be \(s\)
\item Program satisfies assertion \(A\) at this point iff \(s\vDash A\)
\begin{align*}
s\vDash B&\quad\text{ iff }\quad\llb{B}_s=\texttt{true}\\
s\vDash\neg A&\quad\text{ iff }\quad s\not\vDash A\\
s\vDash A_1\wedge A_2&\quad\text{ iff }\quad s\vDash A_1\text{ and }s\vDash A_2\\
s\vDash\forall v.A&\quad\text{ iff }\quad \forall x\in\Z.s[x\mapsto v]\vDash A
\end{align*}
Here, the free variables in assertions are assumed to be included in the set of program variables
\end{itemize}
\end{frame}
\begin{frame}[label={sec:org724fdbe},fragile]{Example program}
 Consider the following program written in our imperative language, annotated with assertions
from our assertions language:
\begin{verbatim}
_(ensures n>= 0)
k := 0;
j := 1;
while (k != n) {
k := k+1;
j := 2*j;
}
_(assert j = 2^n)
\end{verbatim}

We wish to check if starting from a positive value for \(n\), is the value of \(j\) equal
to \(2^n\) after having executed all the statements?
\end{frame}
\section{Hoare Logic}
\label{sec:orgf078f3c}
\begin{frame}[label={sec:orgc31ea3b}]{Hoare Triple: Syntax}
A \alert{Hoare triple} \(\{\phi_1\}P\{\phi_2\}\) is a formula:
\begin{itemize}
\item \(\phi_1\) and \(\phi_2\) are formulae in a base logic (FO logic for us)
\item \(P\) is a program in our imperative language
\item \(\phi_1\): \alert{Precondition}, \(\phi_2\): \alert{Postcondition}
\end{itemize}


Examples of syntactically correct Hoare triples
\begin{itemize}
\item \(\{(n\ge 0)\wedge(n^2>28)\}\;m:=n+1; m:=m*m\;\{\neg(m=36)\}\)
\item \(\{\exists x,y.(y>0)\wedge(n=x^y)\}\;n:=n*(n+1)\;\{\exists x,y.(n=x^y)\}\)
\end{itemize}
\end{frame}
\begin{frame}[label={sec:org147f379}]{Hoare Triple: Semantics}
\begin{itemize}
\item The \alert{partial correctness} specification \(\{\phi_1\}P\{\phi_2\}\) is valid iff starting from a state \(s\)
satisfying \(\phi_1\)
\begin{itemize}
\item Whenever an execution of \(P\) terminates in state \(s'\), then \(s'\vDash\phi_2\)
\end{itemize}
\item The \alert{total corretness} specification \(\{\phi_1\}P\{\phi_2\}\) is valid iff starting from a state \(s\)
satisfying \(\phi_1\)
\begin{itemize}
\item Every execution of \(P\) terminates, and
\item Whenever an execution of \(P\) terminates in state \(s'\), then \(s'\vDash\phi_2\)
\end{itemize}
\end{itemize}


\begin{block}{Partial/Total Correctness}
For programs without loops, both semantics coincide
\end{block}
\end{frame}
\begin{frame}[label={sec:org5589228}]{Assignment Rule}
\begin{block}{Program Construct}
\begin{align*}
&E::= x\mid n\mid E+E\mid E\mid \dots\\
&P::=x:=E
\end{align*}
\end{block}

\begin{block}{Inference Rule}
\begin{equation*}
\begin{prooftree}%[center=false]
\hypo{}
\infer1{\{\phi([x\mapsto E])\}x:=E\{\phi(x)\}}
\end{prooftree}
\end{equation*}
where \(\phi([x\mapsto E])\) replaces every free occurrence of \(x\) in \(\phi\) by \(E\)
\end{block}

Example:
\begin{equation*}
\{(z\cdot y>5)\wedge(\exists x.y=x^x)\}x:=z*y\{(x>5)\wedge(\exists x.y=x^x)\}
\end{equation*}
\end{frame}
\begin{frame}[label={sec:org4d3232e}]{Rule for Sequential Composition}
\begin{block}{Program Construct}
\(P::=P;P\)
\end{block}

\begin{block}{Inference Rule}
\begin{equation*}
\begin{prooftree}%[center=false]
\hypo{\{\phi\}P_1\{\eta\}}
\hypo{\{\eta\}P_2\{\psi\}}
\infer2{\{\phi\}P_1;P_2\{\psi\}}
\end{prooftree}
\end{equation*}
\end{block}

Example:
\begin{equation*}
\begin{prooftree}%[center=false]
\hypo{\{y+z>4\}y:=y+z\{y>4\}}
\hypo{\{y>4\}x:=y+2\{x>6\}}
\infer2{\{y+z>4\}y:=y+z;x:=y+2\{x>6\}}
\end{prooftree}
\end{equation*}
\end{frame}
\begin{frame}[label={sec:orgeea7ab3}]{Rule of Consequence}
\begin{block}{Inference Rule}
\begin{equation*}
\begin{prooftree}%[center=false]
\hypo{\phi\Rightarrow\phi_1}
\hypo{\{\phi_1\}P\{\psi_1\}}
\hypo{\psi_1\Rightarrow\psi}
\infer3{\{\phi\}P\{\psi\}}
\end{prooftree}
\end{equation*}
\(\phi\Rightarrow\phi_1\) and \(\psi_1\Rightarrow\psi\) are implications in underlying (FO) logic
\end{block}
\end{frame}
\begin{frame}[label={sec:orgd24dabd}]{Rules for Conditional Branch}
\begin{block}{Program Construct}
\begin{align*}
&E::=n\mid x\mid -E\mid E+E\mid\dots\\
&B::=\texttt{true}\mid E=E\mid E>= E\mid\neg B\mid B\wedge B\\
&P::=\texttt{if }P\texttt{ then }P\texttt{ else }P
\end{align*}
\end{block}

\begin{block}{Inference Rule}
\begin{equation*}
\begin{prooftree}%[center=false]
\hypo{\{\phi\wedge B\}P_1\{\psi\}}
\hypo{\{\phi\wedge\neg B\}P_2\{\psi\}}
\infer2{\{\phi\}\texttt{if $B$ then $P_1$ else $P_2$\{\psi\}}}
\end{prooftree}
\end{equation*}
\end{block}

Example:

\scalebox{0.8}{
\begin{prooftree}%[center=false]
\hypo{\{(y>4)\wedge(z>1)\}y:=y+z\{y>3\}}
\hypo{\{(y>4)\wedge\neg(z>1)\}y:=y-1\{y>3\}}
\infer2{\{y>4\}\texttt{ if }(z>1)\texttt{ then $y:=y+z$ else $y:=y-1$}\{y>3\}}
\end{prooftree}
}
\end{frame}
\begin{frame}[label={sec:orgf9a9e0d}]{Partial Corretness of Loops}
\begin{block}{Program Construct}
\begin{align*}
&E::=n\mid x\mid -E\mid E+E\mid\dots\\
&B::=\texttt{true}\mid E=E\mid E>= E\mid\neg B\mid B\wedge B\\
&P::=\texttt{while }B\;P
\end{align*}
\end{block}

\begin{block}{Inference Rule}
\begin{equation*}
\begin{prooftree}%[center=false]
\hypo{\{\phi\wedge B\}P\{\phi\}}
\infer1{\{\phi\}\texttt{ while }B\;P\{\phi\wedge\neg B\}}
\end{prooftree}
\end{equation*}
\begin{itemize}
\item \(\phi\) is \alert{loop invariant}
\item Partial Corretness Semantics:
\begin{itemize}
\item If loop does not terminate, Hoare triples is vacuously satisfied
\item If it terminates, \(\phi\wedge\neg B\) must be satisfied after termination
\end{itemize}
\end{itemize}
\end{block}
\end{frame}
\begin{frame}[label={sec:org11e363b}]{Partial Correctness of Loops}
\begin{block}{Inference Rule}
\begin{equation*}
\begin{prooftree}%[center=false]
\hypo{\{\phi\wedge B\}P\{\phi\}}
\infer1{\{\phi\}\texttt{ while }B\;P\{\phi\wedge\neg B\}}
\end{prooftree}
\end{equation*}
\end{block}

Example:
\begin{equation*}
\begin{prooftree}%[center=false]
\hypo{\{(y=x+z)\wedge (z\neq 0)\}x:=x+1;z:=z-1\{y=x+z\}}
\infer1{\{y=x+z\}}\texttt{while }(z!=0)x:=x+1;z:=z-1\{(y=x+z)\wedge(z=0)\}
\end{prooftree}
\end{equation*}
\end{frame}
\begin{frame}[label={sec:org7b84f27}]{Summary of Axioms}
\begin{itemize}
\item Assignment
\begin{equation*}
\begin{prooftree}%[center=false]
\hypo{}
\infer1{\{\phi([x\mapsto E])\}x:=E\{\phi(x)\}}
\end{prooftree}
\end{equation*}
\item Sequential Composition
\begin{equation*}
\begin{prooftree}%[center=false]
\hypo{\{\phi\}P_1\{\eta\}}
\hypo{\{\eta\}P_2\{\psi\}}
\infer2{\{\phi\}P_1;P_2\{\psi\}}
\end{prooftree}
\end{equation*}
\item Conditional Statement
\begin{equation*}
\begin{prooftree}%[center=false]
\hypo{\{\phi\wedge B\}P_1\{\psi\}}
\hypo{\{\phi\wedge\neg B\}P_2\{\psi\}}
\infer2{\{\phi\}\texttt{if $B$ then $P_1$ else $P_2$\{\psi\}}}
\end{prooftree}
\end{equation*}
\item Iteration
\begin{equation*}
\begin{prooftree}%[center=false]
\hypo{\{\phi\wedge B\}P\{\phi\}}
\infer1{\{\phi\}\texttt{ while }B\;P\{\phi\wedge\neg B\}}
\end{prooftree}
\end{equation*}
\item Weakening pre-condition, Strengthening post-condition
\begin{equation*}
\begin{prooftree}%[center=false]
\hypo{\phi\Rightarrow\phi_1}
\hypo{\{\phi_1\}P\{\psi_1\}}
\hypo{\psi_1\Rightarrow\psi}
\infer3{\{\phi\}P\{\psi\}}
\end{prooftree}
\end{equation*}
\end{itemize}
\end{frame}
\begin{frame}[label={sec:org588b58f}]{Structural Rules}
\begin{itemize}
\item Conjunction
\begin{equation*}
\begin{prooftree}%[center=false]
\hypo{\{\phi_1\}P\{\psi_1\}}
\hypo{\{\phi_2\}P\{\psi_2\}}
\infer2{\{\phi_1\wedge\phi_2\}P\{\psi_1\wedge\psi_2\}}
\end{prooftree}
\end{equation*}
\item Disjunction
\begin{equation*}
\begin{prooftree}%[center=false]
\hypo{\{\phi_1\}P\{\psi_1\}}
\hypo{\{\phi_2\}P\{\psi_2\}}
\infer2{\{\phi_1\vee\psi_2\}P\{\psi_1\vee\psi_2\}}
\end{prooftree}
\end{equation*}
\item Existential Quantification(\(v\) is not free in \(P\))
\begin{equation*}
\begin{prooftree}%[center=false]
\hypo{\{\phi\}P\{\psi\}}
\infer1{\{\exists v.\phi\}P\{\exists v.\psi\}}
\end{prooftree}
\end{equation*}
\item Universal Quantification(\(v\) is not free in \(P\))
\begin{equation*}
\begin{prooftree}%[center=false]
\hypo{\{\phi\}P\{\psi\}}
\infer1{\{\forall v.\phi\}P\{\forall v.\psi\}}
\end{prooftree}
\end{equation*}
\end{itemize}
\end{frame}
\begin{frame}[label={sec:org90d6e51},fragile]{A Hoare logic proof}
 Let \(P\) be
\begin{verbatim}
      k := 0
      j := 1
      while (k != n) {
        k := k + 1;
        j := 2 + j;
      }
\end{verbatim}

Our goal is to prove the validity of \(\{n>0\}P\{j=1+2*n\}\)
\end{frame}
\begin{frame}[label={sec:org1cdecf8}]{A Hoare logic proof}
Sequential composition rule will give us a proof if we can fill in the template
\begin{gather*}
\{n>0\}\\
\texttt{k := 0}\\
\{\varphi_1\}\\
\texttt{j := 1}\\
\{\varphi_2\}\\
\texttt{while (k != n) \{k := k+1; j := 2+j;\}}\\
\{j=1+2*n\}
\end{gather*}
\end{frame}
\begin{frame}[label={sec:org45e6eae}]{A Hoare logic proof}
To prove
\begin{equation*}
\{\varphi_2\}\texttt{while(k != n)\{k := k+1;j := 2+j;\}}\{j=1+2*n\}
\end{equation*}
using loop invariant \(j=1+2*k\)

We only need to show that
\begin{itemize}
\item \(\varphi_2\Rightarrow(j=1+2*k)\)
\item \(\{(j=1+2*k)\wedge(k\neq n)\}\texttt{k:=k+1;j:=2+j}\{j=1+2*k\}\)
\item \(((j=1+2*k)\wedge\neg(k\neq n))\Rightarrow(j=1+2*n)\)
\end{itemize}
\end{frame}
\begin{frame}[label={sec:org96c7d46}]{A Hoare logic proof}
\begin{itemize}
\item \(\varphi_2\Rightarrow(j=1+2*k)\) holds if \(\varphi_2\) is \(j=1+2*k\)
\item \((j=1+2*k)\wedge\neg(k\neq n)\Rightarrow(j=1+2*n)\) holds in integer arithmetic
\end{itemize}
\end{frame}
\begin{frame}[label={sec:orgd44ecf6}]{A Hoare logic proof}
To show
\begin{equation*}
\{(j=1+2*k)\wedge(k\neq n)\}\texttt{k:=k+1;j:=2+j}\{j=1+2*k\}
\end{equation*}
Applying assignment rule twice
\begin{gather*}
\{2+j=1+2*k\}\texttt{j:=2+j}\{j=1+2*k\}\\
\{2+j=1+2*(k+1)\}\texttt{k:=k+1}\{2+j=1+2*k\}
\end{gather*}
Simplifying and applying sequential compositon rule we we get
\begin{equation*}
\{j=1+2*k\}\texttt{k:=k+1;j:=2+j}\{j=1+2*k\}
\end{equation*}
Then apply rule for strengthening precedent
\begin{equation*}
\begin{prooftree}%[center=false]
\hypo{(j=1+2*k)\wedge(k\neq n)\Rightarrow(j=1+2*k)}
\infer[no rule]1{\{j=1+2*k\}\texttt{k:=k+1;j:=2+j}\{j=1+2*k\}}
\infer1{\{(j=1+2*k)\wedge(k\neq n)\}\texttt{k:=k+1;j:=2+j}\{j=1+2*k\}}
\end{prooftree}
\end{equation*}
\end{frame}
\begin{frame}[label={sec:orgf2c3c6b}]{A Hoare logic proof}
we have thus show that
\begin{gather*}
\{n>0\}\\
\texttt{k := 0}\\
\{\varphi_1\}\\
\texttt{j := 1}\\
\{\varphi_2:j=1+2*k\}\\
\texttt{while (k != n) \{k := k+1; j := 2+j;\}}\\
\{j=1+2*n\}
\end{gather*}
\end{frame}
\begin{frame}[label={sec:orgd330970}]{A Hoare logic proof}
Similarly, we choose \(\varphi_1\) as \(k=0\), hence we have
\begin{gather*}
\{n>0\}\\
\texttt{k := 0}\\
\{\varphi_1:k=0\}\\
\texttt{j := 1}\\
\{\varphi_2:j=1+2*k\}\\
\texttt{while (k != n) \{k := k+1; j := 2+j;\}}\\
\{j=1+2*n\}
\end{gather*}
\end{frame}
\section{Soundness and Completeness}
\label{sec:orgea99f67}
\begin{frame}[label={sec:orgc68dd73}]{Soundness}
Hoare Logic has a sound proof system
\end{frame}
\begin{frame}[label={sec:orgc5bad14}]{Relative Completeness of Hoare Logic}
\begin{theorem}[Cook, 1974]
If there is a complete proof system for proving assertions in the underlying logic, then all
valid Hoare triples have a proof
\end{theorem}
\end{frame}
\end{document}
