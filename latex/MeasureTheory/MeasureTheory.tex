% Created 2020-01-12 日 20:27
% Intended LaTeX compiler: pdflatex
\documentclass[11pt]{article}
\usepackage[utf8]{inputenc}
\usepackage[T1]{fontenc}
\usepackage{graphicx}
\usepackage{grffile}
\usepackage{longtable}
\usepackage{wrapfig}
\usepackage{rotating}
\usepackage[normalem]{ulem}
\usepackage{amsmath}
\usepackage{textcomp}
\usepackage{amssymb}
\usepackage{capt-of}
\usepackage{hyperref}
\usepackage{minted}
% TIPS
% \substack{a\\b} for multiple lines text





% pdfplots will load xolor automatically without option
\usepackage[dvipsnames]{xcolor}

\usepackage{forest}
% two-line text in node by [two \\ lines]
% \begin{forest} qtree, [..] \end{forest}
\forestset{
  qtree/.style={
    baseline,
    for tree={
      parent anchor=south,
      child anchor=north,
      align=center,
      inner sep=1pt,
    }}}
%\usepackage{flexisym}
% load order of mathtools and mathabx, otherwise conflict overbrace

\usepackage{mathtools}
%\usepackage{fourier}
\usepackage{pgfplots}
\usepackage{amsthm, mathabx,  amsmath, commath}
\usepackage{amsfonts}

\usepackage{empheq}
\usepackage{tikz}
\usetikzlibrary{arrows.meta}
\usepackage[most]{tcolorbox}

\newtheorem{theorem}{Theorem}[section]
\newtheorem{definition}{Definition}[section]
\newtheorem{corollary}{Corollary}[section]
\newtheorem{example}{Example}[section]
\newtheorem{lemma}{Lemma}[section]
\newtheorem{proposition}{Proposition}[section]

\newcommand{\bl}[1] {\boldsymbol{#1}}
\newcommand{\Wt}[1] {\stackrel{\sim}{\smash{#1}\rule{0pt}{1.1ex}}}
\newcommand{\wt}[1] {\widetilde{#1}}


%For boxed texts in align, use Aboxed{}
%otherwise use boxed{}

\DeclareMathSymbol{\widehatsym}{\mathord}{largesymbols}{"62}
\newcommand\lowerwidehatsym{%
  \text{\smash{\raisebox{-1.3ex}{%
    $\widehatsym$}}}}
\newcommand\fixwidehat[1]{%
  \mathchoice
    {\accentset{\displaystyle\lowerwidehatsym}{#1}}
    {\accentset{\textstyle\lowerwidehatsym}{#1}}
    {\accentset{\scriptstyle\lowerwidehatsym}{#1}}
    {\accentset{\scriptscriptstyle\lowerwidehatsym}{#1}}
}

\usepackage{graphicx}
    
% text on arrow for xRightarrow
\makeatletter
%\newcommand{\xRightarrow}[2][]{\ext@arrow 0359\Rightarrowfill@{#1}{#2}}
\makeatother


\def \bx {\boldsymbol{x}}
\def \ba {\boldsymbol{a}}
\def \bI {\boldsymbol{I}}
\def \bt {\boldsymbol{t}}
\def \bb {\boldsymbol{b}}
\def \bA {\boldsymbol{A}}
\def \bX {\boldsymbol{X}}
\def \bu {\boldsymbol{u}}
\def \bS {\boldsymbol{S}}
\def \bZ {\boldsymbol{Z}}
\def \bz {\boldsymbol{z}}
\def \by {\boldsymbol{y}}
\def \bw {\boldsymbol{w}}
\def \bT {\boldsymbol{T}}
\def \bS {\boldsymbol{S}}
\def \bm {\boldsymbol{m}}
\def \bW {\boldsymbol{W}}
\def \bY {\boldsymbol{Y}}
\def \bH {\boldsymbol{H}}
\def \blambda {\boldsymbol{\lambda}}
\def \bPhi {\boldsymbol{\Phi}}
\def \btheta {\boldsymbol{\theta}}
\def \bmu {\boldsymbol{\mu}}
\def \bphi {\boldsymbol{\phi}}
\def \bSigma {\boldsymbol{\Sigma}}
\def \lb {\left\{}
\def \rb {\right\}}
\def \caln {\mathcal{N}}
\def \dissum {\displaystyle\Sigma}
\def \dispro {\displaystyle\prod}
\def \E {\mathbb{E}}
\def \Q {\mathbb{Q}}
\def \V {\mathbb{V}}
\def \R {\mathbb{R}}
\def \calq {\mathcal{Q}}
\def \calg {\mathcal{G}}
\def \caln {\mathcal{N}}
\def \calr {\mathcal{R}}
\def \calm {\mathcal{M}}
\def \calc {\mathcal{C}}
\def \bcup {\bigcup}

\author{Claudio Landim}
\date{\today}
\title{Measure Theory}
\hypersetup{
 pdfauthor={Claudio Landim},
 pdftitle={Measure Theory},
 pdfkeywords={},
 pdfsubject={},
 pdfcreator={Emacs 26.3 (Org mode 9.3)}, 
 pdflang={English}}
\begin{document}

\maketitle
\tableofcontents \clearpage
\section{Introduction: a non-measurable set}
\label{sec:orgaf94a3a}
Suppose we want a measure that satisfies:
\begin{enumerate}
\setcounter{enumi}{-1}
\item \(\lambda:\calp(\R)\to\R_+\cup\{+\infty\}\)
\item \(\lambda((a,b])=b-a\)
\item \(A\subseteq\R,A+x=\{x+y:y\in A\}\)
\begin{equation*}
\forall A\subseteq\R\forall x\in\R,\lambda(A+x)=\lambda(A)
\end{equation*}
\item \(A=\bigcup_{j\ge 1}A_j,A_j\cap A_k=\emptyset\)
\begin{equation*}
\lambda(A)=\displaystyle\sum_{k\ge1}\lambda(A_k)
\end{equation*}
\end{enumerate}



Define \(x\sim y\) for \(x,y\in\R\) if \(y-x\in\Q\). \(\Lambda=\R/\sim\) and
\(\alpha,\beta\in\Lambda\). \(\Gamma\) is uncountable since each equivalent class
is countable.

By the \textbf{Axiom of Choice}, we have a \(\Omega\subseteq\R\) s.t. for each
\([x]\in\R/\sim\), there is a \(x\in[x]\) s.t. \(x\in\Omega\). Hence we can assume
\(\Omega\subseteq(0,1)\). 

\begin{claim}
For \(p,q\in\Q\), either \(\Omega+p=\Omega+q\) or
\(\Omega+p\cap\Omega+q=\emptyset\).
\end{claim}

\begin{proof}
Assume \((\Omega+p)\cap(\Omega+q)\neq\emptyset\), \(x=\alpha+p=\beta+q\). Hence
\(\alpha-\beta=q-p\in\Q\), which implies \(\alpha=\beta\).
\end{proof}

\begin{claim}
\(\Omega+q\subseteq(-1,2)\) since \(-1<q<1\).
\end{claim}

In particular,
\begin{equation*}
\displaystyle\bigcup_{\substack{q\in\Q\\-1<q<1}}(\Omega+q)\subseteq(-1,2)
\end{equation*}

\begin{claim}
If \(E\subseteq F\), then \(\lambda(E)\le\lambda(F)\)
\end{claim}

\begin{proof}
\(\lambda(F)=\lambda(E\cup(F-E))=\lambda(E)+\lambda(F-E)\)
\end{proof}

If \(q\neq p\),
\begin{equation*}
\lambda(\displaystyle\bigcup_{\substack{q\in\Q\\-1<q<1}}(\Omega+q))
=\displaystyle\sum_{\substack{q\in\Q\\-1<q<1}}\lambda(\Omega+q)
=\displaystyle\sum_{\substack{q\in\Q\\-1<q<1}}\lambda(\Omega)
\le\lambda((-1,2))
=3
\end{equation*}

Hence \(\lambda(\Omega)=0\)

\begin{claim}
\((0,1)\subseteq\sum_{q\in\Q,-1<q<1}(\Omega+q)\)
\end{claim}

\begin{proof}
Fix \(x\in[0,1]\), \(\exists\alpha\in[x]\cap\Omega\) and \(\alpha\in(0,1)\). Hence
\(\alpha-x=q\in\Q\). Then \(x\in\Omega+q\)
\end{proof}

Hence we have a contradiction and there is no such \(\lambda\) function.

\section{Classes of subsets}
\label{sec:orgcdbe658}
\begin{definition}[]
For \(\cals\subseteq\calp(\Omega)\), \(\cals\) is a \textbf{semi-algebra} if
\begin{enumerate}
\item \(\Omega\in\cals\)
\item If \(A,B\in\cals\), then \(A\cap B\in\cals\)
\item For all \(A\in\cals\), there are \(E_1,\dots,E_n\in\cals\) s.t. \(A^c=\sqcup E_j\)
\end{enumerate}
\end{definition}

\begin{examplle}[]
If \(\Omega=\R\) and 
\begin{align*}
\cals&=\R\cup\{(a,b]:a<b,a,b\in\R\}\\
&\cup\{(-\infty,b]:b\in\R\}\\
&\cup\{(a,\infty):a\in\R\}\\
&\cup\emptyset
\end{align*}
then \(\cals\) is a semi-algebra
\end{examplle}

\begin{definition}[]
Take \(\cala\subseteq\calp(\Omega)\), \(\cala\) is an \textbf{algebra} if
\begin{enumerate}
\item \(\Omega\in\cala\)
\item If \(A,B\in\cala\), then \(A\cap B\in\cala\)
\item If \(A\in\cala\), then \(A^c\in\cala\)
\end{enumerate}
\end{definition}

If \(\cala\) is an algebra, then it is also semi-algebra.

\begin{definition}[]
\(\calf\subseteq\calp(\Omega)\) is a \textbf{\(\sigma\)-algebra} if
\begin{enumerate}
\item \(\Omega\in\calf\)
\item If \(A_j\in\calf\) for \(j\ge 1\), then \(\bigcap_{j\ge1}A_j\in\calf\)
\item If \(A\in\calf\), then \(A^c\in\calf\)
\end{enumerate}
\end{definition}

\begin{proposition}[]
Suppose \(\cala_\alpha\subseteq\calp(\Omega)\), \(\cala_\alpha\) is an algebra, 
\(\alpha\in I\). Then \(\cala=\bigcap_{\alpha\in I}\cala_\alpha\) is an algebra
\end{proposition}
\end{document}