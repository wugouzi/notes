% Created 2020-08-10 一 20:43
% Intended LaTeX compiler: pdflatex
\documentclass[11pt]{article}
\usepackage[utf8]{inputenc}
\usepackage[T1]{fontenc}
\usepackage{graphicx}
\usepackage{grffile}
\usepackage{longtable}
\usepackage{wrapfig}
\usepackage{rotating}
\usepackage[normalem]{ulem}
\usepackage{amsmath}
\usepackage{textcomp}
\usepackage{amssymb}
\usepackage{capt-of}
\usepackage{hyperref}
\usepackage{minted}
% TIPS
% \substack{a\\b} for multiple lines text





% pdfplots will load xolor automatically without option
\usepackage[dvipsnames]{xcolor}

\usepackage{forest}
% two-line text in node by [two \\ lines]
% \begin{forest} qtree, [..] \end{forest}
\forestset{
  qtree/.style={
    baseline,
    for tree={
      parent anchor=south,
      child anchor=north,
      align=center,
      inner sep=1pt,
    }}}
%\usepackage{flexisym}
% load order of mathtools and mathabx, otherwise conflict overbrace

\usepackage{mathtools}
%\usepackage{fourier}
\usepackage{pgfplots}
\usepackage{amsthm, mathabx,  amsmath, commath}
\usepackage{amsfonts}

\usepackage{empheq}
\usepackage{tikz}
\usetikzlibrary{arrows.meta}
\usepackage[most]{tcolorbox}

\newtheorem{theorem}{Theorem}[section]
\newtheorem{definition}{Definition}[section]
\newtheorem{corollary}{Corollary}[section]
\newtheorem{example}{Example}[section]
\newtheorem{lemma}{Lemma}[section]
\newtheorem{proposition}{Proposition}[section]

\newcommand{\bl}[1] {\boldsymbol{#1}}
\newcommand{\Wt}[1] {\stackrel{\sim}{\smash{#1}\rule{0pt}{1.1ex}}}
\newcommand{\wt}[1] {\widetilde{#1}}


%For boxed texts in align, use Aboxed{}
%otherwise use boxed{}

\DeclareMathSymbol{\widehatsym}{\mathord}{largesymbols}{"62}
\newcommand\lowerwidehatsym{%
  \text{\smash{\raisebox{-1.3ex}{%
    $\widehatsym$}}}}
\newcommand\fixwidehat[1]{%
  \mathchoice
    {\accentset{\displaystyle\lowerwidehatsym}{#1}}
    {\accentset{\textstyle\lowerwidehatsym}{#1}}
    {\accentset{\scriptstyle\lowerwidehatsym}{#1}}
    {\accentset{\scriptscriptstyle\lowerwidehatsym}{#1}}
}

\usepackage{graphicx}
    
% text on arrow for xRightarrow
\makeatletter
%\newcommand{\xRightarrow}[2][]{\ext@arrow 0359\Rightarrowfill@{#1}{#2}}
\makeatother


\def \bx {\boldsymbol{x}}
\def \ba {\boldsymbol{a}}
\def \bI {\boldsymbol{I}}
\def \bt {\boldsymbol{t}}
\def \bb {\boldsymbol{b}}
\def \bA {\boldsymbol{A}}
\def \bX {\boldsymbol{X}}
\def \bu {\boldsymbol{u}}
\def \bS {\boldsymbol{S}}
\def \bZ {\boldsymbol{Z}}
\def \bz {\boldsymbol{z}}
\def \by {\boldsymbol{y}}
\def \bw {\boldsymbol{w}}
\def \bT {\boldsymbol{T}}
\def \bS {\boldsymbol{S}}
\def \bm {\boldsymbol{m}}
\def \bW {\boldsymbol{W}}
\def \bY {\boldsymbol{Y}}
\def \bH {\boldsymbol{H}}
\def \blambda {\boldsymbol{\lambda}}
\def \bPhi {\boldsymbol{\Phi}}
\def \btheta {\boldsymbol{\theta}}
\def \bmu {\boldsymbol{\mu}}
\def \bphi {\boldsymbol{\phi}}
\def \bSigma {\boldsymbol{\Sigma}}
\def \lb {\left\{}
\def \rb {\right\}}
\def \caln {\mathcal{N}}
\def \dissum {\displaystyle\Sigma}
\def \dispro {\displaystyle\prod}
\def \E {\mathbb{E}}
\def \Q {\mathbb{Q}}
\def \V {\mathbb{V}}
\def \R {\mathbb{R}}
\def \calq {\mathcal{Q}}
\def \calg {\mathcal{G}}
\def \caln {\mathcal{N}}
\def \calr {\mathcal{R}}
\def \calm {\mathcal{M}}
\def \calc {\mathcal{C}}
\def \bcup {\bigcup}

\usepackage[UTF8]{ctex}
\author{Renlin Jin}
\date{\today}
\title{Nonstandard Analysis}
\hypersetup{
 pdfauthor={Renlin Jin},
 pdftitle={Nonstandard Analysis},
 pdfkeywords={},
 pdfsubject={},
 pdfcreator={Emacs 26.3 (Org mode 9.4)}, 
 pdflang={English}}
\begin{document}

\maketitle
\tableofcontents \clearpage
\section{实数域和超结构的非标准扩张}
\label{sec:orgd7a642b}
设\(\R\)为所有(标准)实数的集合而
\begin{equation*}
\calr:=(\R;+,*,0,1,<)
\end{equation*}
表示(标准)实数域
\subsection{非主超滤子存在性}
\label{sec:org356a8a9}
\begin{definition}[]
\(X\) be an infinite set. \(\calu\subset\calp(X)\). if any finite
intersection is infinite, then \(\calu\) is a 非主滤子基 on \(X\). 非主滤子基
\(\calu\) is a 非主滤子 on \(X\) if
\begin{enumerate}
\item if \(A,B\in\calu\), then \(A\cap B\in\calu\)
\item if \(A\in\calu\) and \(A\subseteq B\subseteq X\),then \(B\in\calu\)
\end{enumerate}


let \(X\setminus A\) be \(A^C\). if 非主滤子再满足
\begin{enumerate}
\setcounter{enumi}{2}
\item for any \(A\subseteq X\) there is \(A\in\calu\) or \(A^C\in\calu\)
\end{enumerate}


then \(\calu\) is 非主超滤子 on \(X\)
\end{definition}

\begin{proposition}[Suppose Zorn's Lemma]
\label{prop1.2}
Any nonprincipal filter on infinite set \(X\) can be extended to a
nonprincipal ultrafilter
\end{proposition}
\subsection{实数域的非标准扩张}
\label{sec:orgbf6741b}
By \label{prop1.2} we could fix a nonprincipal ultrafilter \(\calu\) on \(\omega\).
\begin{definition}[]
For any \(\la x_n\ra,\la y_n\ra\in\R^\omega\), we define \(\la x_n\ra\sim\la
   y_n\ra\) iff
\begin{equation*}
\{n\in\omega:x_n=y_n\}\in\calu
\end{equation*}
In other words, \(\la x_n\ra\) and \(\la y_n\ra\) is equivalent iff two
sequences equal almost everywhere
\end{definition}

\begin{definition}[]
\([\la x_n\ra]:=\{\la y_n\ra\in\R^\omega:\la y_n\ra\sim\la x_n\ra\}\)
\begin{equation*}
^*\R:=\{[\la x_n\ra]:\la x_n\ra\in\R^\omega\}
\end{equation*}
\end{definition}

\begin{definition}[]
For any \(m\)-ary relation \(P\) on \(\R\) we define
\begin{equation*}
^*P:=\{([\la r_n^{(1)}\ra],\dots,[\la r_n^{(m)}\ra]):\{n\in\omega:(r_n^{(1)},\dots,r_n^{(m)})\in P\}\in\calu\}
\end{equation*}
\end{definition}

Hence
\begin{equation*}
[\la a_n\ra]+[\la b_n\ra]=[\la a_n+b_n\ra]\quad\text{ and }\quad
[\la a_n\ra]*[\la b_n\ra]=[\la a_n*b_n\ra]
\end{equation*}
since \(\{n\in\omega:(a_n,b_n,a_n+b_n)\in P_+\}=\omega\)

\(^*\R:=({}^*\R;+,*,0,1,<)\) is an ordered field

For any \(r\in\R\), we have \([\la r\ra]\in{}^*\R\) and hence we could regard
\(\R\) as a subset of \(^*\R\)

if \([\la r_n\ra]\in{^*\R}\) is larger than every element of \(\R\), then we
call \([\la r_n\ra]\) 无穷大, and 无穷小 vice versa

\begin{proposition}[]
In \(^*\R\), \([\la n\ra]\) 是无穷大, \([\la1/n\ra]\) 是无穷小
\end{proposition}
\subsection{超结构的非标准扩张}
\label{sec:org204d3a2}
\begin{definition}[]
For any set \(X\), let \(V_0:=\R\cup X\). For any \(m\in\omega\) define
\(V_{m+1}:=V_m\cup\calp(V_m)\). Choose a large enough natural number \(\fn\),
define
\begin{equation*}
V:=\bigcup_{m=0}^{\fn}V_m
\end{equation*}
let \(\in\) be a 从属关系 on \(V\). Then structure \(\calv=(V;\in\) is
called 超结构. If \(a\in V\), \(a\in V_m\) but \(a\not\in V_{m-1}\), then we
call \(a\) \(\calv\) 中的第 \(m\) 层元素, written \(l(a)=m\)
\end{definition}

\begin{definition}[]
Suppose \(\calu\) is a nonprincipal ultrafilter on \(\omega\), \(m\le\fn\)
\begin{enumerate}
\item For any element sequence \(\la a_n\ra,\la b_n\ra\in V_m\), define \(\la
      a_n\ra\sim\la b_n\ra\) iff
\begin{equation*}
\{n\in\omega:a_n=b_n\}\in\calu
\end{equation*}
\item For any element sequence \(\la a_n\ra\in V_m^\omega\), define
\([\la a_n\ra]:=\{\la b_n\ra\in V_m^\omega:\la a_n\ra\sim\la b_n\ra\}\)
\item \(^*V_m:=\{[\la a_n\ra]:\la a_n\ra\in V_m^\omega\}\)
\item \(^*V:=\displaystyle\bigcup_{m=0}^{\fn}{}^*V_m\)
\item for \([\la a_n\ra],[\la A_n\ra]\in{}^*V\), define \([\la
      a_n\ra]{}^*\in[\la A_n\ra]\) iff
\begin{equation*}
\{n\in\omega:a_n\in A_n\}\in\calu
\end{equation*}
\end{enumerate}
\end{definition}

\begin{lemma}[]
suppose \(\varphi(\bbar{[\la a_n\ra]})\) is a statement on \(\tensor[^*]{\calv}\)
\end{lemma}
\subsection{exercise}
\label{sec:org11cd399}
\begin{exercise}
\label{ex1.22}
\(A=\{[\la 1\ra],[\la2\ra],\dots\}\)
\end{exercise}

\section{非标准分析和微积分}
\label{sec:orgd6ad7cd}

\section{非标准分析和测度论}
\label{sec:org1b6c9eb}

\section{非标准分析和随机过程}
\label{sec:orgcfa59d2}

\section{非标准分析和组合数论}
\label{sec:org6695405}
\end{document}
