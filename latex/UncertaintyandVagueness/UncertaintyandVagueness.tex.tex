% Created 2019-08-02 五 21:50
% Intended LaTeX compiler: pdflatex
\documentclass[11pt]{article}
\usepackage[utf8]{inputenc}
\usepackage[T1]{fontenc}
\usepackage{graphicx}
\usepackage{grffile}
\usepackage{longtable}
\usepackage{wrapfig}
\usepackage{rotating}
\usepackage[normalem]{ulem}
\usepackage{amsmath}
\usepackage{textcomp}
\usepackage{amssymb}
\usepackage{capt-of}
\usepackage{hyperref}
\usepackage{minted}
\author{wu}
\date{\today}
\title{}
\hypersetup{
 pdfauthor={wu},
 pdftitle={},
 pdfkeywords={},
 pdfsubject={},
 pdfcreator={Emacs 26.2 (Org mode 9.2.4)}, 
 pdflang={English}}
\begin{document}

\tableofcontents \clearpage\% Created 2019-08-02 五 21:49
\% Intended \LaTeX{} compiler: pdflatex
\documentclass[11pt]{article}
\usepackage[utf8]{inputenc}
\usepackage[T1]{fontenc}
\usepackage{graphicx}
\usepackage{grffile}
\usepackage{longtable}
\usepackage{wrapfig}
\usepackage{rotating}
\usepackage[normalem]{ulem}
\usepackage{amsmath}
\usepackage{textcomp}
\usepackage{amssymb}
\usepackage{capt-of}
\usepackage{hyperref}
\usepackage{minted}
% TIPS
% \substack{a\\b} for multiple lines text





% pdfplots will load xolor automatically without option
\usepackage[dvipsnames]{xcolor}

\usepackage{forest}
% two-line text in node by [two \\ lines]
% \begin{forest} qtree, [..] \end{forest}
\forestset{
  qtree/.style={
    baseline,
    for tree={
      parent anchor=south,
      child anchor=north,
      align=center,
      inner sep=1pt,
    }}}
%\usepackage{flexisym}
% load order of mathtools and mathabx, otherwise conflict overbrace

\usepackage{mathtools}
%\usepackage{fourier}
\usepackage{pgfplots}
\usepackage{amsthm, mathabx,  amsmath, commath}
\usepackage{amsfonts}

\usepackage{empheq}
\usepackage{tikz}
\usetikzlibrary{arrows.meta}
\usepackage[most]{tcolorbox}

\newtheorem{theorem}{Theorem}[section]
\newtheorem{definition}{Definition}[section]
\newtheorem{corollary}{Corollary}[section]
\newtheorem{example}{Example}[section]
\newtheorem{lemma}{Lemma}[section]
\newtheorem{proposition}{Proposition}[section]

\newcommand{\bl}[1] {\boldsymbol{#1}}
\newcommand{\Wt}[1] {\stackrel{\sim}{\smash{#1}\rule{0pt}{1.1ex}}}
\newcommand{\wt}[1] {\widetilde{#1}}


%For boxed texts in align, use Aboxed{}
%otherwise use boxed{}

\DeclareMathSymbol{\widehatsym}{\mathord}{largesymbols}{"62}
\newcommand\lowerwidehatsym{%
  \text{\smash{\raisebox{-1.3ex}{%
    $\widehatsym$}}}}
\newcommand\fixwidehat[1]{%
  \mathchoice
    {\accentset{\displaystyle\lowerwidehatsym}{#1}}
    {\accentset{\textstyle\lowerwidehatsym}{#1}}
    {\accentset{\scriptstyle\lowerwidehatsym}{#1}}
    {\accentset{\scriptscriptstyle\lowerwidehatsym}{#1}}
}

\usepackage{graphicx}
    
% text on arrow for xRightarrow
\makeatletter
%\newcommand{\xRightarrow}[2][]{\ext@arrow 0359\Rightarrowfill@{#1}{#2}}
\makeatother


\def \bx {\boldsymbol{x}}
\def \ba {\boldsymbol{a}}
\def \bI {\boldsymbol{I}}
\def \bt {\boldsymbol{t}}
\def \bb {\boldsymbol{b}}
\def \bA {\boldsymbol{A}}
\def \bX {\boldsymbol{X}}
\def \bu {\boldsymbol{u}}
\def \bS {\boldsymbol{S}}
\def \bZ {\boldsymbol{Z}}
\def \bz {\boldsymbol{z}}
\def \by {\boldsymbol{y}}
\def \bw {\boldsymbol{w}}
\def \bT {\boldsymbol{T}}
\def \bS {\boldsymbol{S}}
\def \bm {\boldsymbol{m}}
\def \bW {\boldsymbol{W}}
\def \bY {\boldsymbol{Y}}
\def \bH {\boldsymbol{H}}
\def \blambda {\boldsymbol{\lambda}}
\def \bPhi {\boldsymbol{\Phi}}
\def \btheta {\boldsymbol{\theta}}
\def \bmu {\boldsymbol{\mu}}
\def \bphi {\boldsymbol{\phi}}
\def \bSigma {\boldsymbol{\Sigma}}
\def \lb {\left\{}
\def \rb {\right\}}
\def \caln {\mathcal{N}}
\def \dissum {\displaystyle\Sigma}
\def \dispro {\displaystyle\prod}
\def \E {\mathbb{E}}
\def \Q {\mathbb{Q}}
\def \V {\mathbb{V}}
\def \R {\mathbb{R}}
\def \calq {\mathcal{Q}}
\def \calg {\mathcal{G}}
\def \caln {\mathcal{N}}
\def \calr {\mathcal{R}}
\def \calm {\mathcal{M}}
\def \calc {\mathcal{C}}
\def \bcup {\bigcup}

\author{R. Kruse E.Schewecke J.Heinsohn}
\date{\today}
\title{Uncertainty and vagueness in knowledge based systems}
\hypersetup\{
 pdfauthor=\{R. Kruse E.Schewecke J.Heinsohn\},
 pdftitle=\{Uncertainty and vagueness in knowledge based systems\},
 pdfkeywords=\{\},
 pdfsubject=\{\},
 pdfcreator=\{Emacs 26.2 (Org mode 9.2.4)\}, 
 pdflang=\{English\}\}
\begin{document}

\maketitle
\tableofcontents \clearpage\section{General Considerations of uncertainty and vagueness}
\label{sec:orgf8aeedf}
\subsection{Modeling ignorance}
\label{sec:org60d05b2}
Ignorance arises from a restricted reliability of technical devices, from
partial knowledge, from insufficiencies of observations or from other causes. 

In the sequel we distinguish between two different types of ignorance:
\textbf{uncertainty} and \textbf{vagueness}.  

Vagueness arises whenever a datum, although its meaning is not in doubt,
lacks the desired precision. 

Uncertainty, on the other hand, corresponds to a
human being's valuation of some datum, reflecting his or her faith or doubt
in its source. This concept covers those cases in which the actual state of
affairs or process is not completely determined but where we have to rely on
some human expert's subjective preferences among the different possibilities. 

the basic intention of any model is to reflect properties of the real world,
i.e. to enable the prediction of a system's behavior in the real world. 

a model can never be verified, and the only reasonable argument for its
validity is that all efforts to falsify it have failed.
\section{Introduction}
\label{sec:org66e7f83}
\subsection{Basic notations}
\label{sec:org6d10ca8}
\(\Omega\) \emph{universe of discourse} or \emph{frame of discernment}.
\(\what{\emptyset}\)
\begin{definition}[]
A set \(\Omega\)' is called a \emph{refinement} of \(\Omega\) if there is a mapping 
\(\what{\Pi}:2^\Omega\to 2^{\Omega'}\) s.t.
\begin{enumerate}
\item \(\what{\Pi}(\lb\omega\rb)\neq\emptyset\) for all \(\omega\in\Omega\)
\item \(\what{\Pi}(\lb\omega\rb)\cap\what{\Pi}(\lb\omega'\rb)=\emptyset\), if
\(\omega\neq\omega'\)
\item \(\bigcup\lb\what{\Pi}(\lb\omega\rb)|\omega\in\Omega\rb=\Omega'\)
\item \(\what{\Pi}(A)=\bigcup\lb\what{\Pi}(\lb\omega\rb)|\omega\in A\rb\)
\end{enumerate}
\end{definition}
\(\what{\Pi}\) is called a \emph{refinement mapping}. If such a mapping exists, then
the sets \(\Omega\) and \(\Omega\)' are compatible, and the refined space \(\Omega\)' is
able to carry more information than \(\Omega\). \(\Omega\) is a \emph{coarsening} of \(\Omega\)'

\begin{definition}[]
Let \(\Omega\)' be a refinement of \(\Omega\) where \(\what{\Pi}:2^\Omega\to
   2^{\Omega'}\) is the corresponding refinement mapping. The mapping
\begin{equation*}
\Pi:2^{\Omega'}\to 2^\Omega,\quad\Pi(A')\stackrel{d}{=}\lb
\omega\in\Omega\mid\what{\Pi}(\lb\omega\rb)\cap A\neq\emptyset\rb
\end{equation*}
is called the \emph{outer reduction} induced by \(\what{\Pi}\)
\end{definition}

Consider a frame of discernment \(\Omega=\lb\text{not\_at\_sea,at\_sea}\rb\).
The granularity of this set is coarse and we might switch to a refined set
\(\Omega'=\lb\text{open\_sea, 12-mile-zone, 3-mile-zone, canal, refueling\_dock,
   loading\_dock}\rb\). We obtain the refinement mapping
\begin{align*}
\what{\Pi}(\lb\text{at\_sea}\rb)&=\lb\text{open\_sea, 12-mile-zone, 3-mile-zone}
\rb\\
\what{\Pi}(\lb\text{not\_at\_sea}\rb)&=\lb
\text{canal, refueling\_dock, loading\_dock}\rb
\end{align*}
and 
\begin{align*}
\Pi(\lb\text{open\_sea}\rb)&=\lb\text{at\_sea}\rb\\
\Pi(\lb\text{canal}\rb)&=\lb\text{not\_at\_sea}\rb\\
\end{align*}

A family \(\calu\) of set \(\Omega^{(i)}\) is a \emph{universe}.

\begin{definition}[]
Let \(\calu\) be a universe with index set \(M\). If \(S,T,C\) are index subsets of
\(M\) s.t. \(T=S\cup C,S\cap C=\emptyset\), then we define the pointwise
projection by setting
\begin{enumerate}
\item \(\pi_S^T:\Omega^T\to\Omega^S,\pi^T_S(\omega^T)\triangleq y^S\), where 
\(y^{(i)}=\omega^{(i)}\) for all \(i\in S,S\neq\emptyset\) and
\item \(\pi^T_\emptyset:\Omega^T\to\Omega^{\emptyset},\pi^T_\emptyset=\epsilon\)
\end{enumerate}


\(\omega^T=(\omega^S,\omega^C)\) if 
\begin{equation*}
\pi^T_S(\omega^T)=\omega^S\;\text{ and }\; \pi^T_C(\omega^T)=\omega^C
\end{equation*}
\end{definition}

\begin{definition}[]
Let \(\calu\) be a universe with index set \(M\). \(S,T,C\subseteq M\) s.t.
\(T=S\cup C,S\cap C=\emptyset\), then
\begin{enumerate}
\item the mapping
\begin{align*}
&\Pi^T_S:2^{(\Omega^T)}\to 2^{(\Omega^S)}\\
&\Pi^T_S(A)\triangleq\lb\omega^S\in\Omega^S\mid\exists\omega^T\in A:
\pi^T_S(\omega^T)=\omega^S\rb
\end{align*}
is called the \emph{projection} of \(\Omega^T\) onto \(\Omega^S\)
\item the mapping
\begin{align*}
&\what{\Pi}^T_S:2^{(\Omega^S)}\to 2^{(\Omega^T)}\\
&\what{\Pi}^T_S(A)\triangleq\lb\omega^T\in\Omega^T\mid\pi^T_S(\omega^T)\in B\rb\\
\end{align*}
is called the \emph{cylindrical extension} of \(\Omega^S\) onto \(\Omega^T\)
\end{enumerate}
\end{definition}
\subsection{vagueness and uncertainty}
\label{sec:org700537b}
\subsubsection{modeling vague data}
\label{sec:org585bce1}
Observations allow us to restrict the set \(\Omega\) of possible states of the world;
these observations may be precise but in general they will contain some inherent ambiguity.

Suppose
\begin{equation*}
\Omega=\lb z3,z2,z1,ca,rd,ld\rb
\end{equation*}
a radar device provides vague outputs consisting of the three grey levels
black, grey and white. Assume \(\lb ca,rd,ld\rb\) appear in black, \(\lb z1\rb\)
appears in grey and \(\lb z2,z3\rb\) appear in white

We have to distinguish between models for describing a \emph{vague datum} and
the \emph{uncertainty} about the location of the ship based on the vague datum. The
information contained in the vague radar device image should be encoded by a
function \(\mu:\Omega\to\lb black,grey,white\rb\)

We can consider a totally ordered finite set \((L,\le)\) of acceptability
degrees. The expert is allowed to specify for each acceptability degree
\(l\in L\) a corresponding region \(\mu_l\). 

Layered sets can be characterized by functions
\begin{equation*}
\mu:\Omega\to L
\end{equation*}
where \(\mu(\omega)\) is the biggest value \(l\in L\) s.t. \(\omega\in\mu_l\)
\subsubsection{Modeling partial belief}
\label{sec:org5162c3d}
\section{Vague data}
\label{sec:org7e2a490}
\subsection{Basic concepts}
\label{sec:orga69be10}
Throughout this chapter let us assume, for simplicity, that we perfectly
trust any source of information, so that we can focus on vagueness.

\begin{definition}[]
Each function \(\eta:\Omega\to L\) is called an \emph{L-set} of \(\Omega\).
\(\calf_L(\Omega)\) denotes the set of all L-sets of \(\Omega\)
\end{definition}
Given a vague datum in terms of an L-set \(\eta:\Omega\to L\) the question of
interest always concerns the actual location of the original entity
\(\omega_0\in\Omega\) 

\textbf{Example.} Let \(\Omega=\lb z3,z2,z1,ca,ld,rd\rb\) and let
\begin{equation*}
\eta_1:\Omega\to L_1=\lb black,white\rb
\end{equation*}
where
\begin{equation*}
\eta_1(\omega)=
\begin{cases}
white&\text{if } \omega\in\lb z3,z2,z1\rb\\
black&\text{otherwise}
\end{cases}
\end{equation*}
\(A=\lb z3,z2,z1\rb\) is a \emph{minimal} set that \emph{necessarily} covers \(\omega_0\).

Now suppose \(\eta_2:\Omega\to L_2=\lb black,grey,white\rb\)
\begin{equation*}
\eta_2(\omega)=
\begin{cases}
white &\text{if } \omega\in\lb z3,z2\rb\\
grey & \text{if } \omega\in\lb ld\rb\\
black&\text{otherwise}
\end{cases}
\end{equation*}
\subsection{On the origin of vague data}
\label{sec:org7237d97}
Grey levels often arise physically from the superposition of multiple layers
of shaded patterns, where the grey gradation corresponds to the number of levels.

In the sequel we consider a
vague datum to represent the superposition of \emph{retrictions}, each of which
reflects what is known about \(\omega_0\) in some well defined \emph{context}. 
In the presence of \(n\) different contexts we obtain \(n\) restrictions
\(A^1,\dots,A^n\) for the unknown original \(\omega_0\).
The
sets \(A^i,i=1,\dots,n\), we imagine to refer to n copies
\(\Omega_{(1)},\dots,\Omega_{(n)}\) of \(\Omega\). So on the formal level we have
to deal with the set 
\(\Omega_n\triangleq \displaystyle\bigcup_{\omega\in\Omega}\lb\omega^1,\dots,\omega^n\rb\)
which is a refinement of \(\Omega\). The corresponding refinement mapping is
\(\what{\sigma}_n:2^\Omega\to 2^{\Omega_n}\) where
\begin{equation*}
\what{\sigma}_n(\lb\omega\rb)=\lb\omega^1,\dots,\omega^n\rb
\end{equation*}
and the projection
\begin{equation*}
\sigma_n:2^{\Omega_n}\to 2^\Omega;\e\sigma_n(\lb\omega^i\rb)=\lb\omega\rb
\end{equation*}
Define \(\Omega_{(i)}\triangleq\lb\omega^i\mid\omega\in\Omega\rb\).

At the perception level we obtain an image consisting of up to \(n\) grey tones.
On the formal level the superposition of layered restrictions can be easily
described by a mapping
\begin{align*}
vag^{\sigma_n}_{A^1,\dots,A^n}:&\Omega\to\lb 0,\dots,n\rb\\
&\omega\mapsto card(\lb i\mid\omega^i\in A^i\rb)
\end{align*}
\begin{definition}[]
Each subset \(A\subseteq\Omage_n,n\in\N\) induces via the projection mapping
\(\sigma_n\) the grey level image
\begin{align*}
vag^{\sigma_n}_{A^1,\dots,A^n}:&\Omega\to\lb 0,\dots,n\rb\\
&\omega\mapsto card(\lb i\mid\omega^i\in A^i\rb)
\end{align*}
\end{definition}
\subsection{Uncertainty handling by means of layered contexts}
\label{sec:org2e8c290}
To each \(l\in L\) we can assign a number \(j(l)\) which is interpreted as the
number of superposed layers
\subsubsection{Possibility and necessity}
\label{sec:org7c462f1}
Based on some vague datum \(\eta:\Omega\to L\) we have to evaluate whether a
given set \(A\subseteq\Omega\) covers the element \(\omega_0\) or not.
\begin{definition}[]
Let \(\mu:\Omega\to\lb 0,\dots,n\rb\). \(\omega\in\Omega\) is called \emph{j-possible}
w.r.t. \(\mu\) if and only if \(\mu(\omega)=j\). A subset \(A\neq \emptyset\) of
\(\Omega\) is called \emph{j-possible} if
\(A\cap\lb\omega\in\Omega\mid\mu(\omega)\ge j\rb\neq\emptyset\)  but 
\(A\cap\lb\omega\in\Omega\mid\mu(\omega)\ge j+1\rb=\emptyset\)
\end{definition}

Let
\begin{align*}
Poss_\mu:&2^\Omega\to\lb 0,\dots,n\\
&A\mapsto \max\lb\mu(\omega)\mid\omega\in A\rb
\end{align*}
\begin{definition}[]
Let \(\mu:\Omega\to\lb 0,\dots,n\rb\). A subset \(A\subseteq\Omega\) is called
\emph{j-necessary} w.r.t. \(\mu\) if 
\begin{equation*}
\lb\omega\in\Omega\mid\mu(\omega)\ge j\rb\subseteq A\e\text{but}\e
\lb\omega\in\Omega\mid\mu(\omega)\ge j-1\rb\not\subseteq A
\end{equation*}
\end{definition}
\(\mu_j\triangleq\lb\omega\in\Omega\mid\mu(\omega)\ge j\rb\), the \emph{level sets}

\begin{equation*}
\mu(\omega)=\max\lb\min(j,)\rb
\end{equation*}
\(\mathbb{1}\)
\end{document}
\end{document}