% Created 2019-08-02 五 17:48
% Intended LaTeX compiler: pdflatex
\documentclass[11pt]{article}
\usepackage[utf8]{inputenc}
\usepackage[T1]{fontenc}
\usepackage{graphicx}
\usepackage{grffile}
\usepackage{longtable}
\usepackage{wrapfig}
\usepackage{rotating}
\usepackage[normalem]{ulem}
\usepackage{amsmath}
\usepackage{textcomp}
\usepackage{amssymb}
\usepackage{capt-of}
\usepackage{hyperref}
\usepackage{minted}
% TIPS
% \substack{a\\b} for multiple lines text





% pdfplots will load xolor automatically without option
\usepackage[dvipsnames]{xcolor}

\usepackage{forest}
% two-line text in node by [two \\ lines]
% \begin{forest} qtree, [..] \end{forest}
\forestset{
  qtree/.style={
    baseline,
    for tree={
      parent anchor=south,
      child anchor=north,
      align=center,
      inner sep=1pt,
    }}}
%\usepackage{flexisym}
% load order of mathtools and mathabx, otherwise conflict overbrace

\usepackage{mathtools}
%\usepackage{fourier}
\usepackage{pgfplots}
\usepackage{amsthm, mathabx,  amsmath, commath}
\usepackage{amsfonts}

\usepackage{empheq}
\usepackage{tikz}
\usetikzlibrary{arrows.meta}
\usepackage[most]{tcolorbox}

\newtheorem{theorem}{Theorem}[section]
\newtheorem{definition}{Definition}[section]
\newtheorem{corollary}{Corollary}[section]
\newtheorem{example}{Example}[section]
\newtheorem{lemma}{Lemma}[section]
\newtheorem{proposition}{Proposition}[section]

\newcommand{\bl}[1] {\boldsymbol{#1}}
\newcommand{\Wt}[1] {\stackrel{\sim}{\smash{#1}\rule{0pt}{1.1ex}}}
\newcommand{\wt}[1] {\widetilde{#1}}


%For boxed texts in align, use Aboxed{}
%otherwise use boxed{}

\DeclareMathSymbol{\widehatsym}{\mathord}{largesymbols}{"62}
\newcommand\lowerwidehatsym{%
  \text{\smash{\raisebox{-1.3ex}{%
    $\widehatsym$}}}}
\newcommand\fixwidehat[1]{%
  \mathchoice
    {\accentset{\displaystyle\lowerwidehatsym}{#1}}
    {\accentset{\textstyle\lowerwidehatsym}{#1}}
    {\accentset{\scriptstyle\lowerwidehatsym}{#1}}
    {\accentset{\scriptscriptstyle\lowerwidehatsym}{#1}}
}

\usepackage{graphicx}
    
% text on arrow for xRightarrow
\makeatletter
%\newcommand{\xRightarrow}[2][]{\ext@arrow 0359\Rightarrowfill@{#1}{#2}}
\makeatother


\def \bx {\boldsymbol{x}}
\def \ba {\boldsymbol{a}}
\def \bI {\boldsymbol{I}}
\def \bt {\boldsymbol{t}}
\def \bb {\boldsymbol{b}}
\def \bA {\boldsymbol{A}}
\def \bX {\boldsymbol{X}}
\def \bu {\boldsymbol{u}}
\def \bS {\boldsymbol{S}}
\def \bZ {\boldsymbol{Z}}
\def \bz {\boldsymbol{z}}
\def \by {\boldsymbol{y}}
\def \bw {\boldsymbol{w}}
\def \bT {\boldsymbol{T}}
\def \bS {\boldsymbol{S}}
\def \bm {\boldsymbol{m}}
\def \bW {\boldsymbol{W}}
\def \bY {\boldsymbol{Y}}
\def \bH {\boldsymbol{H}}
\def \blambda {\boldsymbol{\lambda}}
\def \bPhi {\boldsymbol{\Phi}}
\def \btheta {\boldsymbol{\theta}}
\def \bmu {\boldsymbol{\mu}}
\def \bphi {\boldsymbol{\phi}}
\def \bSigma {\boldsymbol{\Sigma}}
\def \lb {\left\{}
\def \rb {\right\}}
\def \caln {\mathcal{N}}
\def \dissum {\displaystyle\Sigma}
\def \dispro {\displaystyle\prod}
\def \E {\mathbb{E}}
\def \Q {\mathbb{Q}}
\def \V {\mathbb{V}}
\def \R {\mathbb{R}}
\def \calq {\mathcal{Q}}
\def \calg {\mathcal{G}}
\def \caln {\mathcal{N}}
\def \calr {\mathcal{R}}
\def \calm {\mathcal{M}}
\def \calc {\mathcal{C}}
\def \bcup {\bigcup}

\author{R. Kruse E.Schewecke J.Heinsohn}
\date{\today}
\title{Uncertainty and vagueness in knowledge based systems}
\hypersetup{
 pdfauthor={R. Kruse E.Schewecke J.Heinsohn},
 pdftitle={Uncertainty and vagueness in knowledge based systems},
 pdfkeywords={},
 pdfsubject={},
 pdfcreator={Emacs 26.2 (Org mode 9.2.4)}, 
 pdflang={English}}
\begin{document}

\maketitle
\tableofcontents \clearpage\section{General Considerations of uncertainty and vagueness}
\label{sec:orgabc4af6}
\subsection{Modeling ignorance}
\label{sec:orgb9a4f31}
Ignorance arises from a restricted reliability of technical devices, from
partial knowledge, from insufficiencies of observations or from other causes. 

In the sequel we distinguish between two different types of ignorance:
\textbf{uncertainty} and \textbf{vagueness}.  

Vagueness arises whenever a datum, although its meaning is not in doubt,
lacks the desired precision. 

Uncertainty, on the other hand, corresponds to a
human being's valuation of some datum, reflecting his or her faith or doubt
in its source. This concept covers those cases in which the actual state of
affairs or process is not completely determined but where we have to rely on
some human expert's subjective preferences among the different possibilities. 

the basic intention of any model is to reflect properties of the real world,
i.e. to enable the prediction of a system's behavior in the real world. 

a model can never be verified, and the only reasonable argument for its
validity is that all efforts to falsify it have failed.
\section{Introduction}
\label{sec:org8ed5bbe}
\subsection{Basic notations}
\label{sec:orga7c16de}
\(\Omega\) \emph{universe of discourse} or \emph{frame of discernment}.
\(\what{\emptyset}\)
\begin{definition}[]
A set \(\Omega\)' is called a \emph{refinement} of \(\Omega\) if there is a mapping 
\(\what{\Pi}:2^\Omega\to 2^{\Omega'}\) s.t.
\begin{enumerate}
\item \(\what{\Pi}(\lb\omega\rb)\neq\emptyset\) for all \(\omega\in\Omega\)
\item \(\what{\Pi}(\lb\omega\rb)\cap\what{\Pi}(\lb\omega'\rb)=\emptyset\), if
\(\omega\neq\omega'\)
\item \(\bigcup\lb\what{\Pi}(\lb\omega\rb)|\omega\in\Omega\rb=\Omega'\)
\item \(\what{\Pi}(A)=\bigcup\lb\what{\Pi}(\lb\omega\rb)|\omega\in A\rb\)
\end{enumerate}
\end{definition}
\(\what{\Pi}\) is called a \emph{refinement mapping}. If such a mapping exists, then
the sets \(\Omega\) and \(\Omega\)' are compatible, and the refined space \(\Omega\)' is
able to carry more information than \(\Omega\). \(\Omega\) is a \emph{coarsening} of \(\Omega\)'

\begin{definition}[]
Let 
\end{definition}
\end{document}