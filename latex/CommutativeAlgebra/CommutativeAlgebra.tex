% Created 2021-05-06 四 15:25
% Intended LaTeX compiler: pdflatex
\documentclass[11pt]{article}
\usepackage[utf8]{inputenc}
\usepackage[T1]{fontenc}
\usepackage{graphicx}
\usepackage{grffile}
\usepackage{longtable}
\usepackage{wrapfig}
\usepackage{rotating}
\usepackage[normalem]{ulem}
\usepackage{amsmath}
\usepackage{textcomp}
\usepackage{amssymb}
\usepackage{capt-of}
\usepackage{hyperref}
% TIPS
% \substack{a\\b} for multiple lines text





% pdfplots will load xolor automatically without option
\usepackage[dvipsnames]{xcolor}

\usepackage{forest}
% two-line text in node by [two \\ lines]
% \begin{forest} qtree, [..] \end{forest}
\forestset{
  qtree/.style={
    baseline,
    for tree={
      parent anchor=south,
      child anchor=north,
      align=center,
      inner sep=1pt,
    }}}
%\usepackage{flexisym}
% load order of mathtools and mathabx, otherwise conflict overbrace

\usepackage{mathtools}
%\usepackage{fourier}
\usepackage{pgfplots}
\usepackage{amsthm, mathabx,  amsmath, commath}
\usepackage{amsfonts}

\usepackage{empheq}
\usepackage{tikz}
\usetikzlibrary{arrows.meta}
\usepackage[most]{tcolorbox}

\newtheorem{theorem}{Theorem}[section]
\newtheorem{definition}{Definition}[section]
\newtheorem{corollary}{Corollary}[section]
\newtheorem{example}{Example}[section]
\newtheorem{lemma}{Lemma}[section]
\newtheorem{proposition}{Proposition}[section]

\newcommand{\bl}[1] {\boldsymbol{#1}}
\newcommand{\Wt}[1] {\stackrel{\sim}{\smash{#1}\rule{0pt}{1.1ex}}}
\newcommand{\wt}[1] {\widetilde{#1}}


%For boxed texts in align, use Aboxed{}
%otherwise use boxed{}

\DeclareMathSymbol{\widehatsym}{\mathord}{largesymbols}{"62}
\newcommand\lowerwidehatsym{%
  \text{\smash{\raisebox{-1.3ex}{%
    $\widehatsym$}}}}
\newcommand\fixwidehat[1]{%
  \mathchoice
    {\accentset{\displaystyle\lowerwidehatsym}{#1}}
    {\accentset{\textstyle\lowerwidehatsym}{#1}}
    {\accentset{\scriptstyle\lowerwidehatsym}{#1}}
    {\accentset{\scriptscriptstyle\lowerwidehatsym}{#1}}
}

\usepackage{graphicx}
    
% text on arrow for xRightarrow
\makeatletter
%\newcommand{\xRightarrow}[2][]{\ext@arrow 0359\Rightarrowfill@{#1}{#2}}
\makeatother


\def \bx {\boldsymbol{x}}
\def \ba {\boldsymbol{a}}
\def \bI {\boldsymbol{I}}
\def \bt {\boldsymbol{t}}
\def \bb {\boldsymbol{b}}
\def \bA {\boldsymbol{A}}
\def \bX {\boldsymbol{X}}
\def \bu {\boldsymbol{u}}
\def \bS {\boldsymbol{S}}
\def \bZ {\boldsymbol{Z}}
\def \bz {\boldsymbol{z}}
\def \by {\boldsymbol{y}}
\def \bw {\boldsymbol{w}}
\def \bT {\boldsymbol{T}}
\def \bS {\boldsymbol{S}}
\def \bm {\boldsymbol{m}}
\def \bW {\boldsymbol{W}}
\def \bY {\boldsymbol{Y}}
\def \bH {\boldsymbol{H}}
\def \blambda {\boldsymbol{\lambda}}
\def \bPhi {\boldsymbol{\Phi}}
\def \btheta {\boldsymbol{\theta}}
\def \bmu {\boldsymbol{\mu}}
\def \bphi {\boldsymbol{\phi}}
\def \bSigma {\boldsymbol{\Sigma}}
\def \lb {\left\{}
\def \rb {\right\}}
\def \caln {\mathcal{N}}
\def \dissum {\displaystyle\Sigma}
\def \dispro {\displaystyle\prod}
\def \E {\mathbb{E}}
\def \Q {\mathbb{Q}}
\def \V {\mathbb{V}}
\def \R {\mathbb{R}}
\def \calq {\mathcal{Q}}
\def \calg {\mathcal{G}}
\def \caln {\mathcal{N}}
\def \calr {\mathcal{R}}
\def \calm {\mathcal{M}}
\def \calc {\mathcal{C}}
\def \bcup {\bigcup}

\author{David Eisenbud}
\date{\today}
\title{Commutative Algebra with a View Toward Algebraic Geometry}
\hypersetup{
 pdfauthor={David Eisenbud},
 pdftitle={Commutative Algebra with a View Toward Algebraic Geometry},
 pdfkeywords={},
 pdfsubject={},
 pdfcreator={Emacs 27.1 (Org mode 9.3)}, 
 pdflang={English}}
\begin{document}

\maketitle
\tableofcontents

\section{Elementary Definitions}
\label{sec:orgfbc8f1d}
\subsection{Rings and Ideals}
\label{sec:org9f207fe}
If \(R\) is a commutative ring, then a \textbf{commutative algebra} over \(R\) (or
\textbf{commutative \(R\)-algebra}) is a commutative ring \(R\) together with a
homomorphism \(\alpha:R\to S\) of rings. We usually suppress the homomorphism \(\alpha\)
from the notation, and write \(rs\) in place of \(\alpha(r)s\). Any ring is a
\(\Z\)-ring in a unique way. Polynomial ring \(S=R[x_1,\dots,x_n]\) is an \(R\)-algebra
\subsection{Unique Factorization}
\label{sec:orge49c9ca}
\subsection{Modules}
\label{sec:org9461403}
If \(M\) and \(N\) are \(R\)-modules, then the \textbf{direct sum} of \(M\) and \(N\)
is the module \(M\oplus N=\{(m,n)\mid m\in M,n\in N\}\) with the module
structure \(r(m,n)=(rm,rn)\). There are natural inclusion and projections
maps \(M\subset M\oplus N\) and \(M\oplus N\to M\) given by \(m\mapsto(m,0)\)
and \((m,n)\mapsto m\). These maps are enough to identify a direct sum: That
is \(M\) is a \textbf{direct summand} of a module \(P\) iff there are homomorphisms
\(\alpha:M\to P\) and \(\sigma:P\to M\) whose composition \(\sigma\alpha\) is the
identity map of \(M\); then \(P\cong M\oplus(\ker\sigma)\)
\section{Roots of Commutative Algebra}
\label{sec:org57f8d70}
\subsection{The Basis Theorem}
\label{sec:orgfb3f124}
A ring \(R\) is \textbf{Noetherian} if every ideal of \(R\) is finitely generated,
which is equivalent to the \textbf{ascending chain condition on ideals of \(R\)},
which says that every strictly ascending chain of ideals must terminate

\begin{theorem}[Hilbert Basis Theorem]
\label{thm1.2}
If a ring \(R\) is Noetherian, then the polynomial ring \(R[x]\) is Noetherian
\end{theorem}

If \(f=a_nx^n+a_{n-1}x^{n-1}+\dots+a_n\in R[x]\) with \(a_n\neq0\), we define
the \textbf{initial term} of \(f\) to be \(a_nx^n\), and \textbf{initial coefficient} of \(f\)
to be \(a_n\)

\begin{proof}
Let \(I\subset R[x]\) be an ideal; we shall show that \(I\) is finitely
generated. Choose a sequence of elements \(f_1,f_2,\dots\in I\) as folows:
Let \(f_1\) be a nonzero element of least degree in \(I\). For \(i\ge1\), if
\((f_1,\dots,f_i)\neq I\), then choose \(f_{i+1}\) to be an element of least
degree among \(I\setminus(f_1,\dots,f_i)\). If \((f_1,\dots,f_i)=I\), stop
choosing.

Let \(a_j\) be the initial coefficient of \(f_j\). Since \(R\) is Noetherian,
the ideal \(J=(a_1,a_2,\dots)\) of all the \(a_i\) produced is finitely
generated. We may choose a set of generators from among the \(a_i\)
themselves. Let \(m\) be the first integer s.t. \(a_1,\dots,a_m\) generate
\(J\). We claim that \(I=(f_1,\dots,f_m)\)

In the contrary case, our process chose an element \(f_{m+1}\). We may write
\(a_{m+1}=\displaystyle\sum_{j=1}^mu_ja_j\), for some \(u_j\in R\). Since the
degree of \(f_{m+1}\) is at least as great as the degree of any of the
\(f_1,\dots,f_m\), we any define a polynomial \(g\in R\) having the same
degree and initial term as \(f_{m+1}\) by the formula
\begin{equation*}
g=\sum_{j=1}^mu_jf_jx^{\deg f_{m+1}-\deg f_j}\in (f_1,\dots,f_m)
\end{equation*}
The difference \(f_{m+1}-g\in I\) but not in \((f_1,\dots,f_m)\), and has
degree strictly less than the degree of \(f_{m+1}\). This contradicts the
choice of \(f_{m+1}\) as having minimal degree
\end{proof}

\begin{corollary}[]
Any homomorphic image of a Noetherian ring is Noetherian. Furthermore, if
\(R_0\) is a Noetherian ring, and \(R\) is a finitely generated algebra over
\(R_0\), then \(R\) is Noetherian
\end{corollary}

\begin{proof}
Given an ideal \(I\) in \(R/J\), with \(R\) Noetherian, the preimage of \(I\)
in \(R\) is finitely generated, and the images of its generators generate
\(I\)

Since \(R\) is a finitely generated algebra over \(R_0\), \(R\) is a
homomorphic image of \(S:=R_0[x_1,\dots,x_r]\) for some \(r\). Using Theorem
\ref{thm1.2} and induction on \(r\), we see that \(S\) is Noetherian. Since a
homomorphic image of a Noetherian ring is Noetherian, we are done.
\end{proof}

An \(R\)-module \(M\) is \textbf{Noetherian} if every submodule of \(N\) is finitely
generated

\begin{proposition}[]
\label{prop1.4}
If \(R\) is a Noetherian ring and \(M\) is a finitely generated \(R\)-module,
then \(M\) is Noetherian
\end{proposition}

\begin{proof}
Suppose that \(M\) is generated by \(f_1,\dots,f_t\), and let \(N\) be a
submodule. We shall show that \(N\) is finitely generated by induction on
\(t\).

If \(t=1\), then the map \(R\to M\) sending 1 to \(f_1\) is surjective. The
preimage of \(N\) is an ideal, which is finitely generated since \(R\) is
Noetherian. The images of its generators generate \(N\)

Now suppose \(\iffalse<\fi t>1\). The image \(\bbar{N}\) of \(N\) in \(M/Rf_1\) is
finitely generated by induction. Let \(g_1,\dots,g_s\) be elements of \(N\)
whose images generate \(\bbar{N}\). Since \(Rf_1\subset M\)  is generated by
one element, its submodule \(N\cap Rf_1\) is finitely generated, say by
\(h_1,\dots,h_r\) \footnote{\href{https://math.stackexchange.com/questions/561655/that-submodule-generated-by-one-element-leads-to-submodule-being-finitely-genera}{StackExchange} Let \(R_N=\{r\in R\mid rf_1\in N\}\). Then \(R_N\) is an
ideal of \(R\). Since \(R\) is Noetherian, the ideal \(R_N\) of \(R\) is
finitely generated.}.

We shall show that the elements \(h_1,\dots,h_r\) and \(g_1,\dots,g_s\)
together generate \(N\)
\end{proof}
\subsection{Graded Rings}
\label{sec:orgecc5eee}
A \textbf{graded ring} is a ring \(R\) together with a direct sum decomposition
\begin{equation*}
R=R_0\oplus R_1\oplus R_2\oplus\cdots\quad\text{ as abelian groups}
\end{equation*}
s.t.
\begin{equation*}
R_iR_j\subset R_{i+j}\quad\text{ for }i,j\ge0
\end{equation*}
A \textbf{homogeneous element} of \(R\) is an element of one of the groups \(R_i\),
and a \textbf{homogeneous ideal} of \(R\) is an ideal that is generated by homogeneous
elements. If \(f\in R\), there is a unique expression for \(f\) of the form
\begin{equation*}
f=f_0+f_1+\cdots\quad\text{ with } f_i\in R_i\text{ and }f_j=0\text{ for }j\gg0
\end{equation*}
the \(f_i\) are called the \textbf{homogeneous components} of \(f\).

The ideal consisting of all elements of degree greater than 0 is called the
\textbf{irrelevant ideal} written as \(R_+\)

The simplest example of a graded ring is the ring of polynomials
\(S=k[x_1,\dots,x_r]\) \textbf{graded by degree}: that is, with grading
\begin{equation*}
S=S_0\oplus S_1\oplus\cdots
\end{equation*}
where \(S_d\) is the vector space of homogeneous polynomials (also called
forms) of degree \(d\)

\begin{definition}[]
A polynomial is \textbf{homogeneous of degree} \(d\) if its a linear combination of
monomials of degree \(d\)

A monomial in \(n\) variables is \(x_1^{i_1}\dots x_n^{i_n}\), its \textbf{degree} is \(i_1+\dots+i_n\)
\end{definition}

The space of all homogeneous polynomials of a given degree \(d\) in \(n\)
variables is \emph{finite dimensional}

\begin{proposition}[]
The number of monomials of degree \(d\) in 3 variables is \(C_{d+2}^2\). And
for \(n\) variables, the number is \(C_{d+n-1}^{n-1}\)
\end{proposition}

Suppose that \(I\) is a homogeneous ideal of a graded ring \(R\), and \(I\)
is generated by homogeneous elements \(f_1,\dots,f_s\). If \(f\in I\) is any
homogeneous elements, then we can write \(f=\sum g_if_i\) with each \(g_i\)
homogeneous of degree \(\deg g_i=\deg f-\deg f_i\)
\subsection{Algebra and Geometry: The Nullstellensatz}
\label{sec:org3c408cf}
Gauss' \textbf{fundamental theorem of algebra} establishes the basic link between
algebra and geometry: It says that a polynomial in one variable over \(\C\),
an algebra object, is determined up to a scalar factor by the set of its
roots(with multiplicites), a geometric object

A polynomial \(f\in k[x_1,\dots,x_n]\) with coefficients in a field \(k\)
defines a function \(f:k^n\to k\); the value of \(f\) at a point
\((a_1,\dots,a_n)\in k^n\) is obtained by substituting the \(a_i\) for the
\(x_i\) in \(f\). The function defined by \(f\) is called a \textbf{polynomial
function} on the \(n\)-dimensional vector space \(k^n\) over \(k\), with
values in \(k\). If \(k\) is infinite, then no polynomial function other than
0 can vanish identically (always 0) on \(k^n\). (The case of one variable is
the statement that a polynomial in one variable can have only finitely many
roots, and follows from Euclid's algorithm for division. In the general case
we think of a nonzero polynomial \(f(x_1,\dots,x_n)\) in \(n\) variables as a
polynomial in \(n-1\) variables with coefficients that are polynomials in one
variable)

If follows that if \(k\) is infinite, then distinct polynomials define
distinct functions. Thus we may regard the polynomial ring
\(k[x_1,\dots,x_n]\) as the ring of polynomial functions on \(k^n\).  Viewed
with its ring of polynomial functions, \(k^n\) is usually called \textbf{affine
\(n\)-space} over \(k\), written \(\bA^n(k)\) or simply \(\bA^n\)

Given a subset \(I\subset k[x_1,\dots,x_n]\), we define a corresponding
\textbf{algebraic subset} of \(k^n\) to be
\begin{equation*}
Z(I)=\{(a_1,\dots,a_n)\in k^n\mid f(a_1,\dots,a_n)=0\text{ for all }f\in I\}
\end{equation*}
Such algebraic sets are sometimes called an \textbf{affine algebraic sets}

If \(X=Z(I)\) is an algebraic set, then an \textbf{algebraic subset} \(Y\subset X\) is
a set of the form \(Y=Z(J)\) that happens to be contained in \(X\). An
algebraic set is called \textbf{irreducible} if it not the union to two smaller
algebraic subsets. Irreducible algebraic sets are called \textbf{algebraic varieties}

If \(k=\R\) or \(k=\C\), then \(k^r\) is naturally a topological space, and
an algebraic subset \(X\subset\bA^r\) inherits the subspace topology, called
the \textbf{classical topology}. But there is another, coarser topology on \(X\) that
is defined over any filed. Polynomial functions on \(X\) will play the role
of continuous functions, even when the fields we are working over have no
topology, and by analogy with the continuous case it is natural to think of
an algebraic subset \(Y\) as a \textbf{closed} subset of \(X\). Since we  obviously
have \(\bigcap_iZ(J_i)=Z(\bigcup_iJ_i)\). Furthermore, if we define
\(\prod_{r=1}^nJ_i\) to be the set consisting of all products of one function
from each \(J_i\), then \(\bigcup_{i=1}^nZ(J_i)=Z(\prod_{i=1}^nJ_i)\). Thus
we may define a topology on \(X\) by taking the closed sets to be the
algebraic subsets of \(X\). This topology is called the \textbf{Zariski topology}.

Given any set \(X\subset k^n\), we define
\begin{equation*}
I(X)=\{f\in k[x_1,\dots,x_n]\mid f(a_1,\dots,a_n)=0\text{ for all }(a_1,\dots,a_n)\in X\}
\end{equation*}
It is clear that \(I(X)\) is an ideal. A \textbf{polynomial function} (or \textbf{regular
function} ) on \(X\) is the restriction of a polynomial function on \(k^n\) on
\(X\). Identifying two polynomial functions if they agree at all the points
of \(X\), we get the \textbf{coordinate ring} \(A(X)\) of \(X\). Clearly we have
\(A(X)=k[x_1,\dots,x_n]/I(X)\)

Not every homomorphic image \(A=k[x_1,\dot,x_n]/I\) could be the coordinate
ring of a set. For suppose an element \(f\in A\) satisfies \(f^n=0\). If
\(f\) were a function on some set \(X\), we would have \(0=f^d(p)=f(p)^d\);
that is, \(f(p)\) is \textbf{nilpotent} for all \(p\in X\). But the values of \(f\)
are elements of \(k\), a field; so they are all 0, and \(f\) itself is the
zero element of \(A(X)\). In general, a ring is said to be \textbf{reduced} if its
only nilpotent element is 0; we have just shown that \(A(X)\) is reduced

If \(R\) is a ring and \(I\subset R\) is an ideal, then the set
\begin{equation*}
\rad I:=\{f\in R\mid f^m\in I\text{ for some integer }m\}
\end{equation*}
is an ideal. It is called the \textbf{radical} of \(I\). An ideal \(I\) is called a
\textbf{radical ideal} if \(I=\rad I\). It follows that \(R/I\) is a reduced ring iff
\(I\) is a radical ideal. Thus, the ideals \(I(X)\) are all radical ideals

Not even every radical ideal in \(S\) can occur as \(I(X)\): For example, the
ideal \(I=(x^2+1)\subset \R[x]\) is radical because \(\R[x]/(x^2+1)\cong\C\)
is reduced. But \(Z(I)=\emptyset\), so \(I\) is not of the form \(I(X)\) for
any \(X\). If \(k\) is algebraically closed, the situation is better. For
example, every polynomial in one variable is a product of linear factors, and
a polynomial \(f\in k[x]\) generates a radical ideal iff it has no multiple
roots. In this case if \(X\) is the set of roots of \(f\), then \(I(X)=(f)\).
Hilbert's Nullstellensatz extends this to polynomial rings with many
variables
\begin{theorem}[Nullstellensatz]
Let \(k\) be an algebraically closed field. If \(I\subset k[x_1,\dots,x_n]\)
is an ideal, then
\begin{equation*}
I(Z(I))=\rad I
\end{equation*}
Thus, the correspondences \(I\mapsto Z(I)\) and \(X\mapsto I(X)\) induce a
bijection between the collection of algebraic subsets of \(\bA^n_k=k^n\) and
radical ideals of \(k[x_1,\dots,x_n]\)
\end{theorem}

\begin{corollary}[]
A system of polynomial equations
\begin{align*}
&f_1(x_1,\dots,x_n)=0\\
&\dots\\
&f_m(x_1,\dots,x_n)=0
\end{align*}
over an algebraically closed field \(k\) has no solution in \(k^n\) iff 1 can
be expressed as a linear combination
\begin{equation*}
1=\sum p_if_i
\end{equation*}
with polynomial coefficients \(p_i\)
\end{corollary}

\begin{proof}
By the Nullstellensatz, if \(Z(f_1,\dots,f_m)=\emptyset\), then 1 is in the
radical of \((f_1,\dots,f_m)\)
\end{proof}

\begin{corollary}[]
If \(k\) is an algebraically closed field and \(A\) is a \(k\)-algebra, then
\(A=A(X)\) for some algebraic set \(X\) iff \(A\) is reduced and finitely
generated as a \(k\)-algebra
\end{corollary}

\begin{proof}
If \(A=A(X)\) for some \(X\subset k^n\), then \(A=k[x_1,\dots,x_n]/I(X)\) is
generated as a \(k\)-algebra by \(x_1,\dots,x_n\). Since \(I(X)\) is radical,
\(A\) is reduced

Conversely, if \(A\) is a finitely generated \(k\)-algebra, then after
choosing generators we may write \(A=k[x_1,\dots,x_n]/I\) for some ideal
\(I\). Since \(A\) is reduced, \(I\) is radical. Thus \(I=I(Z(I))\) by the
Nullstellensatz, and we may take \(X=Z(I)\)
\end{proof}
\subsection{Hilbert Functions and Polynomials}
\label{sec:org64b3499}
\begin{definition}[]
If \(R=R_0\oplus R_1\oplus\cdots\) is a graded ring, then a \textbf{graded module}
over \(R\) is a module \(M\) with a docomposition
\begin{equation*}
M=\bigoplus_{-\infty}^{+\infty}M_i\quad\text{as abelian groups}
\end{equation*}
s.t. \(R_iM_j\subset M_{i+j}\) for all \(i,j\)
\end{definition}

\begin{definition}[]
Let \(M\) be a finitely generated graded module over \(k[x_1,\dots,x_r]\),
with grading by degree. The numerical function
\begin{equation*}
H_M(s):=\dim_kM_s
\end{equation*}
is called the \textbf{Hilbert function of \(M\)} (These dimensions are all finite; if
\(M_s\) were not finite dimensional, then the submodule
\(\oplus_s^{\infty}M_i\) would not be finitely generated, contradicating
Proposition \ref{prop1.4})
\end{definition}

\begin{theorem}[Hilbert]
\label{thm1.11}
If \(M\) is a finitely generated graded module over \(k[x_1,\dots,x_r]\),
then \(H_M(s)\) agrees, for large \(s\), with a polynomial of degree \(\le r-1\)
\end{theorem}

\begin{definition}[]
This polynomial, denoted \(P_M(s)\), is called the \textbf{Hilbert polynomial of \(M\)}
\end{definition}

We define \(M(d)\) to be this graded module; more formally, \(M(d)\) is
isomorphic to \(M\) as a module and has grading defined by
\begin{equation*}
M(d)_e=M_{d+e}
\end{equation*}
\(M(d)\) is sometimes referred to as the \textbf{\(d\)th twist of \(M\)}.

\begin{lemma}[]
Let \(H(s)\in\Z\) be defined for all for all natural numbers \(s\). If the
"first difference" \(H'(s)=H(s)-H(s-1)\) agrees with a polynomial of degree
\(\le n-1\) having rational coefficients for \(s\ge s_0\), then \(H(s)\)
agrees with a polynomial of degree \(\le n\) having rational coefficients for
all \(s\ge s_0\)
\end{lemma}

\begin{proof}
Suppose that \(Q(s)\) is a polynomial of degree \(\le n-1\) with rational
coefficients s.t. \(H'(s)=Q(s)\) for \(s\ge s_0\). For any integer \(s\) set
\(P(s)=H(s_0)+\sum_{t=s_0+1}^sQ(t)\), where the sum is taken over all
integers between \(s_0+1\) and \(s\) whether \(s\ge s_0+1\) or \(s\le
   s_0+1\).
For \(s\ge s_0\) we have \(P(s)=H(s)\). For all \(s\) we have
\(P(s)-P(s-1)=Q(s)\). It follows that \(P(s)\) is a polynomial of degree
\(\le n\) with rational coefficients
\end{proof}

\begin{proof}[Proof of Theorem \ref{thm1.11}]
We do induction on \(r\). If \(r=0\), then \(M\) is simply a
finite-dimensional graded vector space. In this case \(H_M(s)=0\)
\end{proof}
\subsection{Exerncise}
\label{sec:org8796bb0}
\begin{exercise}
\label{ex1.1}
Prove that the following conditions on a module \(M\) over a commutative ring
are equivalent. The case \(M=R\) is the case of ideals
\begin{enumerate}
\item \(M\) is Noetherian
\item Every ascending chain of submodules of \(M\) terminates
\item Every set of submodules of \(M\) contains elements maximal under inclusion
\item Given any sequence of elements \(f_1,f_2,\dots\in M\), there is a number
\(m\) s.t. for each \(n>m\) there is an expression
\(f_n=\sum_{i=1}^ma_if_i\) with \(a_i\in R\)
\end{enumerate}
\end{exercise}

\begin{proof}
\(1\to 2\) obvious.

\(2\to 3\) zorn's lemma

\(2\to 4\).
\end{proof}
\section{Localization}
\label{sec:orgf3b943d}
A \textbf{local ring} is a ring with just one maximal ideal.
\subsection{Fractions}
\label{sec:orgc443fe5}
Given a ring \(R\), an \(R\)-module \(M\), and a multiplicatively closed
subset \(U\subset R\), we define the \textbf{localization of \(M\) at \(U\)}, written
as \(M[U^{-1}]\) or \(U^{-1}M\), to be the set of equivalence classes of pair
\((m,u)\) with \(m\in M\) and \(u\in U\) with equivalence relation
\((m,u)\sim (m',u')\) if there is an element \(v\in U\) s.t. \(v(u'm-um')=0\)
in \(M\). The equivalence class of \((m,u)\) is denoted \(m/u\). We make
\(M[U^{-1}]\) into an \(R\)-module by defining
\begin{align*}
&m/u+m'/u'=(u'm+um')/uu'\\
&r(m/u)=(rm)/u
\end{align*}
for \(m,m'\in M,u,u'\in U,r\in R\). Note that \(u'm/u'u=m/u\). The
localization comes equipped with a natural map of \(R\)-modules \(M\to
   M[U^{-1}]\) carrying \(m\) to \(m/1\)

It is convenient to extend the notation a little further: If \(U\subset R\)
is an arbitrary set, and \(\bbar{U}\subset R\)  is the multiplicatively
closed set of all products of elements in \(U\), then we set
\(M[U^{-1}]:=M[\bbar{U}^{-1}]\)

If we apply the definition in the case \(M=R\), the resulting localization is
a ring, with multiplication defined by
\begin{equation*}
(r/u)(r'/u')=rr'/uu'
\end{equation*}
and in fact \(M[U^{-1}]\) is an \(R[U^{-1}]\)-module with action defined by
\begin{equation*}
(r/u)(m/u')=rm/uu'\quad\text{for }r\in R,m\in M,u,u'\in U
\end{equation*}
\begin{proposition}[]
\label{prop2.1}
Let \(U\) be a multiplicatively closed set of \(R\), and let \(M\) be an
\(R\)-module. An element \(m\in M\) goes to 0 in \(M[U^{-1}]\) iff \(m\) is
annihilated by an element \(u\in U\). In particular, if \(M\) is finitely
generated, then \(M[U^{-1}]=0\) iff \(M\) is annihilated by an element of \(U\)
\end{proposition}

\begin{proof}
If generators \(m_i\in M\) are annihilated by elements \(u_i\in U\), then
\(M\) is annihilated by the product of the \(u_i\)
\end{proof}

The quotient field of an integral domain \(R\), which we shall denote by
\(K(R)\), is the localization \(R[U^{-1}]\) where \(U=R-\{0\}\). For an
arbitrary ring \(R\), take \(U\) to be the set of nonzerodivisors of \(R\),
and define the \textbf{total quotient ring} \(K(R)\) of \(R\) by \(K(R):=R[U^{-1}]\).
By Proposition \ref{prop2.1} \(K(R)\) is the "biggest" localization of \(R\) s.t.
the natural map \(R\to R[U^{-1}]\) is an injection

An ideal \(P\subset R\) is prime iff \(R-P\) is a multiplicatively closed
set. If \(P\) is a prime ideal and \(U=R-P\), then we write \(R_P\)for
\(R[U^{-1}]\). Similarly, for any \(R\)-module \(M\), we write \(M_P\) for
\(M[U^{-1}]\). We write \(\kappa(P)\) for the ring \(R_P/P_P\), the
\textbf{residue class field  of \(R\) at \(P\)}.
\end{document}