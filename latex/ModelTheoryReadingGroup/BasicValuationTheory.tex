% Created 2023-03-02 Thu 13:03
% Intended LaTeX compiler: pdflatex
\documentclass[11pt]{article}
\usepackage[utf8]{inputenc}
\usepackage[T1]{fontenc}
\usepackage{graphicx}
\usepackage{longtable}
\usepackage{wrapfig}
\usepackage{rotating}
\usepackage[normalem]{ulem}
\usepackage{amsmath}
\usepackage{amssymb}
\usepackage{capt-of}
\usepackage{hyperref}
\graphicspath{{../../books/}}
% TIPS
% \substack{a\\b} for multiple lines text





% pdfplots will load xolor automatically without option
\usepackage[dvipsnames]{xcolor}

\usepackage{forest}
% two-line text in node by [two \\ lines]
% \begin{forest} qtree, [..] \end{forest}
\forestset{
  qtree/.style={
    baseline,
    for tree={
      parent anchor=south,
      child anchor=north,
      align=center,
      inner sep=1pt,
    }}}
%\usepackage{flexisym}
% load order of mathtools and mathabx, otherwise conflict overbrace

\usepackage{mathtools}
%\usepackage{fourier}
\usepackage{pgfplots}
\usepackage{amsthm, mathabx,  amsmath, commath}
\usepackage{amsfonts}

\usepackage{empheq}
\usepackage{tikz}
\usetikzlibrary{arrows.meta}
\usepackage[most]{tcolorbox}

\newtheorem{theorem}{Theorem}[section]
\newtheorem{definition}{Definition}[section]
\newtheorem{corollary}{Corollary}[section]
\newtheorem{example}{Example}[section]
\newtheorem{lemma}{Lemma}[section]
\newtheorem{proposition}{Proposition}[section]

\newcommand{\bl}[1] {\boldsymbol{#1}}
\newcommand{\Wt}[1] {\stackrel{\sim}{\smash{#1}\rule{0pt}{1.1ex}}}
\newcommand{\wt}[1] {\widetilde{#1}}


%For boxed texts in align, use Aboxed{}
%otherwise use boxed{}

\DeclareMathSymbol{\widehatsym}{\mathord}{largesymbols}{"62}
\newcommand\lowerwidehatsym{%
  \text{\smash{\raisebox{-1.3ex}{%
    $\widehatsym$}}}}
\newcommand\fixwidehat[1]{%
  \mathchoice
    {\accentset{\displaystyle\lowerwidehatsym}{#1}}
    {\accentset{\textstyle\lowerwidehatsym}{#1}}
    {\accentset{\scriptstyle\lowerwidehatsym}{#1}}
    {\accentset{\scriptscriptstyle\lowerwidehatsym}{#1}}
}

\usepackage{graphicx}
    
% text on arrow for xRightarrow
\makeatletter
%\newcommand{\xRightarrow}[2][]{\ext@arrow 0359\Rightarrowfill@{#1}{#2}}
\makeatother


\def \bx {\boldsymbol{x}}
\def \ba {\boldsymbol{a}}
\def \bI {\boldsymbol{I}}
\def \bt {\boldsymbol{t}}
\def \bb {\boldsymbol{b}}
\def \bA {\boldsymbol{A}}
\def \bX {\boldsymbol{X}}
\def \bu {\boldsymbol{u}}
\def \bS {\boldsymbol{S}}
\def \bZ {\boldsymbol{Z}}
\def \bz {\boldsymbol{z}}
\def \by {\boldsymbol{y}}
\def \bw {\boldsymbol{w}}
\def \bT {\boldsymbol{T}}
\def \bS {\boldsymbol{S}}
\def \bm {\boldsymbol{m}}
\def \bW {\boldsymbol{W}}
\def \bY {\boldsymbol{Y}}
\def \bH {\boldsymbol{H}}
\def \blambda {\boldsymbol{\lambda}}
\def \bPhi {\boldsymbol{\Phi}}
\def \btheta {\boldsymbol{\theta}}
\def \bmu {\boldsymbol{\mu}}
\def \bphi {\boldsymbol{\phi}}
\def \bSigma {\boldsymbol{\Sigma}}
\def \lb {\left\{}
\def \rb {\right\}}
\def \caln {\mathcal{N}}
\def \dissum {\displaystyle\Sigma}
\def \dispro {\displaystyle\prod}
\def \E {\mathbb{E}}
\def \Q {\mathbb{Q}}
\def \V {\mathbb{V}}
\def \R {\mathbb{R}}
\def \calq {\mathcal{Q}}
\def \calg {\mathcal{G}}
\def \caln {\mathcal{N}}
\def \calr {\mathcal{R}}
\def \calm {\mathcal{M}}
\def \calc {\mathcal{C}}
\def \bcup {\bigcup}

\makeindex
\author{Chen Qi'ao}
\date{\today}
\title{Basic Valuation Theory}
\hypersetup{
 pdfauthor={Chen Qi'ao},
 pdftitle={Basic Valuation Theory},
 pdfkeywords={},
 pdfsubject={},
 pdfcreator={Emacs 28.0.92 (Org mode 9.6)}, 
 pdflang={English}}
\begin{document}

\maketitle
\section{Absolute Values}
\label{sec:org51d1aca}
\subsection{Absolute Values - Completions}
\label{sec:org4189f9b}
Let \(K\) be a field. An \textbf{absolute value} on \(K\) is a map
\begin{equation*}
\abs{\;}:K\to\R
\end{equation*}
satisfying the following axioms for all \(x,y\in K\)
\begin{enumerate}
\item \(\abs{x}>0\) for all \(x\neq 0\), and \(\abs{0}=0\)
\item \(\abs{xy}=\abs{x}\abs{y}\)
\item \(\abs{x+y}\le\abs{x}+\abs{y}\)
\end{enumerate}

The absolute value sending all \(x\neq 0\) to 1 is called the \textbf{trivial} absolute value on \(K\).

Observation: \(\abs{1}^2=\abs{1^2}=\abs{1}\), \(\abs{1}=1=\abs{-1}\), \(\abs{x}=\abs{-x}\) for all
\(x\in K\), \(\abs{x^{-1}}=\abs{x}^{-1}\) for \(x\neq 0\).

\begin{proposition}[]
The set \(\{\abs{n\cdot 1}\mid n\in\Z\}\) is bounded iff \(\abs{\;}\) satisfies the ``ultrametric'' inequality
\begin{equation}
\label{1.2}
\abs{x+y}\le\max\{\abs{x},\abs{y}\}
\end{equation}
for all \(x,y\in K\)
\end{proposition}

\begin{proof}
\(\Leftarrow\): Easy, bounded by 1

\(\Rightarrow\): let \(\abs{n\cdot 1}\le C\), then
\begin{equation*}
\abs{x+y}^n=\abs{(x+y)^n}\le\sum_\nu\abs{\binom{n}{\nu}x^\nu y^{n-\nu}}\le(n+1)C\max(\abs{x},\abs{y})^n
\end{equation*}
\end{proof}

If an absolute value satisfies \eqref{1.2}, it is called \textbf{non-archimedean}; otherwise it is called
\textbf{archimedean}. Clearly, if \(\tchar K\neq 0\), \(K\) cannot carry any archimedean absolute value.

\begin{examplle}[]
Let
\begin{equation*}
\abs{x}_0=
\begin{cases}
x&x\ge 0\\
-x&x\le 0
\end{cases}
\end{equation*}
for all \(x\in\R\); we call \(\abs{\;}_0\) the \textbf{usual} absolute value on \(\R\). This is an
archimedean absolute value.
\end{examplle}

\begin{examplle}[]
For every prime \(p\), the \textbf{\(p\)-adic} absolute value \(\abs{\;}_p\) on \(\Q\) is defined by
\(\abs{0}_p=0\) and
\begin{equation*}
\abs{p^\nu\frac{m}{n}}_p=\frac{1}{e^\nu}
\end{equation*}
where \(\nu\in\Z\), and \(n,m\in\Z\setminus\{0\}\) are not divisible
by \(p\). In this case
\begin{equation*}
\{\abs{n\cdot 1}_p\mid n\in\Z\}=\{e^{-\nu}\mid\nu\in\N\}
\end{equation*}
is bounded in \(\R\).
\end{examplle}

\begin{examplle}[]
Let \(F\) be a field and let \(F[[T]]=\{\sum_{i=0}^\infty a_iT^i\mid a_i\in F\}\), which is called
the \textbf{formal power series over \(F\)}. We can define the absolute value \(\abs{\;}\) as
\begin{equation*}
\abs{f}=e^{-m}
\end{equation*}
when \(f=\sum_{i=m}^\infty a_iT^i\) where \(a_m\neq 0\) .
\end{examplle}

\begin{examplle}[]
\label{e1.1.4}
We define for every irreducible polynomial \(p\in k[X]\), \(k\) a field, the following absolute
value \(\abs{\;}_p\) on the rational function field \(K=k(X)\):
Let \(\abs{0}_p=0\) and
\begin{equation*}
\abs{p^\nu\frac{f}{g}}_p=\frac{1}{e^\nu}
\end{equation*}
where \(\nu\in\Z\) and \(f,g\in k[X]\setminus\{0\}\) are not divisible by \(p\). Hence the
set \(\{\abs{n\cdot 1}_p\mid n\in\Z\}\) is bounded in \(\R\).
\end{examplle}

\begin{proposition}[]
If \(A\) is a domain, \(K\) is the fraction field of \(A\) and \(\abs{\;}\) is an absolute
value, then we can uniquely extend \(\abs{\;}\) to \(K\)
\end{proposition}

\begin{proof}
For any \(a,b\in A\) and \(b\neq 0\),
\begin{equation*}
\abs{a}=\abs{b\cdot\frac{a}{b}}=\abs{b}\abs{\frac{a}{b}}
\end{equation*}
\end{proof}

An absolute value \(\abs{\;}\) on \(K\) defines a metric by taking \(\abs{x-y}\) as distance,
for \(x,y\in K\). In particular, \(\abs{\;}\) induces a topology on \(K\) by taking basic open
balls \(B_\epsilon(a)=\{x:\abs{x-a}<\epsilon\}\).

Since a non-trivial absolute value \(\abs{\;}\) defines a metric on \(K\), we may consider the
completion of \(K\) w.r.t. \(\abs{\;}\). Fix a \(\abs{\;}\).

A sequence \((x_n)_{n\in\N}\) of elements of \(k\) is called a \textbf{Cauchy sequence} if for every \(\epsilon>0\)
there exists \(N\in\N\) s.t. for all \(n,m>N\) we have
\begin{equation*}
\abs{x_n-x_m}<\epsilon
\end{equation*}
We say a sequence \((x_n)_{n\in\N}\) \textbf{converges} to \(x\in K\)  and write \(\lim_{n\to\infty}x_n=x\) if for
every \(\epsilon>0\) there is an \(N\in\N\) s.t. for all \(n>N\) we have
\begin{equation*}
\abs{x_n-x}<\epsilon
\end{equation*}
\(K\) is \textbf{complete} if every Cauchy sequence from \(K\) converges to some element of \(K\).

The next theorem will show that every field \(K\) with a non-trivial absolute value can be densely
embedded into a field complete with respect to an ab- solute value extending the given one
on \(K\).

\begin{theorem}[]
\label{1.1.4}
There exists a field \(\hatK\), complete under an absolute value \(\abs{\hat{\;}}\), and an
embedding \(\iota:K\to\hatK\), s.t. \(\abs{x}=\abs{\hat{\iota{x}}}\) for all \(x\in K\). The image \(\iota(K)\)
is dense in \(\hatK\). If \((\hatK',\iota')\) is another such pair, then there exists a unique
continuous isomorphism \(\varphi:\hatK\to\hatK'\) preserving the absolute value and making  the diagram
\begin{center}\begin{tikzcd}
\hatK\ar[rr,"\varphi"]&&\hatK'\\
&K\ar[ul,"\iota"]\ar[ur,"\iota'"']
\end{tikzcd}\end{center}
Such a pair is called a \textbf{completion} of the valued field \(K,\abs{\;}\)
\end{theorem}

\begin{proof}
\emph{Sketch of completion}:

Let \(\calc\) be the set of all Cauchy sequences \((x_n)_{n\in\N}\) of elements of \(K\). \(\calc\) is a
ring with componentwise addition and multiplication. \(\caln=\{(x_n)_{n\in\N}\mid\lim_{n\to\infty}x_n=0\}\) is
an ideal of \(\calc\).

Each \((a_n)_{n\in\N}\in\calc\setminus\caln\) has a positive lower bound, and therefore there is \(M\in\N\) and \(\eta>0\)
s.t. \(\abs{a_n}>\eta\) for every \(n>M\).

Setting \(c_n=1\) for every \(n=1,\dots,M\) and \(c_n=a_n^{-1}\) for every \(n>M\).
Then \((c_n)_{n\in\N}\) is a Cauchy sequence, and \((a_n)_{n\in\N}(c_n)_{n\in\N}-(1)_{n\in\N}\in\caln\). Thus the
ideal \(\caln\) is a maximal ideal of \(\calc\), and the quotient ring \(\hatK\) is a field.

The map \(\iota:K\to\hatK\) defined by \(\iota(x)=(x_n)_{n\in\N}+\caln\), where \(x_n=x\) for every \(n\),
embeds \(K\) in \(\hatK\).

For \((a_n)_{n\in\N}\in\calc\) the sequence \((\abs{a_n})_{n\in\N}\) is a Cauchy sequence of real numbers,
since \(\abs{\abs{a_p}-\abs{a_q}}_0\le\abs{a_p-a_q}\) for all \(p,q\). Moreover, for every
sequence \((a_n)_{n\in\N}\in\caln\) the sequence of real numbers \((\abs{a_n})_{n\in\N}\) has limit 0.
Consequently for \(\xi=(a_n)_{n\in\N}+\caln\) the value
\begin{equation*}
\abs{\hat{\xi}}=\lim_{n\to\infty}\abs{a_n}
\end{equation*}
does not depend on the representative \((b_n)_{n\in\N}\) of \(\xi\).And it's an absolute value
of \(\hatK\) that induces \(\abs{\;}\) on \(K\).
\end{proof}

\begin{definition}[]
Let \(\Q_p\) be the completion of \(\Q\) w.r.t. the \(p\)-adic absolute value \(\abs{\;}_p\),
called \textbf{\(p\)-adic numbers}. The ring of \textbf{\(p\)-adic integers} is \(\Z_p=\{x\in\Q_p\mid\abs{x}_p\le 1\}\)
\end{definition}

\begin{fact}[]
\begin{enumerate}
\item \(\Z_p\) is the completion of \(\Z\) w.r.t. the \(p\)-adic absolute value.
\item \(\Q_p=\Z_p[1/p]\).
\item Every \(x\in\Z_p\) can be written in the form
\begin{equation*}
x=b_0+b_1p+b_2p^2+\dots+b_np^n+\dots
\end{equation*}
where \(0\le b_i\le p-1\), and this representation is unique.
\item Every \(x\in\Q_p\) can be written in the form
\begin{equation*}
x=\sum_{n\ge-n_0}b_np^n
\end{equation*}
where \(0\le b_n\le p-1\) and \(\abs{x}_p=e^{n_0}\). This representation is unique.
\end{enumerate}
\end{fact}

\subsection{Archimedean Complete Fields}
\label{sec:org5d32c35}
Let \(K\) be a field complete w.r.t. an archimedean absolute value \(\abs{\;}\). Since the
set \(\{\abs{n\cdot 1}\mid n\in\Z\}\) is not bounded, \(\tchar K=0\). Thus \(K\) contains the field \(\Q\) of
rationals.

\(\abs{\;}\) restricted to \(\Q\) induces the same topology as the usual absolute value of \(\Q\). Thus the
complete field \(K\) contains the completion of \(\Q\) w.r.t. the ordinary absolute value,
i.e., \(K\) contains \(\R\) as a closed subfield.

Then \(K\) must be equal to \(\R\) or to \(\C\). Consequently, every field \(K\) admitting an
archimedean absolute value may be considered as a subfield of \(\C\) or even \(\R\) with the
absolute value dependent on the induced one from \(\C\) (or from \(\R\))

\subsection{Non-Archimedean Complete Fields}
\label{sec:org4a077eb}
Assume \(\abs{\;}\) is a non-trivial, non-archimedean absolute value on the field \(K\), we can
define an ``additive'' presentation of the absolute value \(\abs{\;}\):
\begin{equation*}
v(x):=-\ln\abs{x}
\end{equation*}
In the case of the \(p\)-adic absolute value \(\abs{\;}_q\) on \(\Q\), we obtain
\begin{equation*}
v_p(p^\nu\frac{m}{n})=\nu
\end{equation*}
\(v_p\) is called the \textbf{\(p\)-adic valuation} on \(\Q\).

Using the additive notion, the axioms of a non-archimedean absolute value
\begin{equation*}
v:K\to\R\cup\{\infty\}
\end{equation*}
now reads for all \(x,y\in K\)
\begin{enumerate}
\item \(v(x)\in\R\) for \(x\neq 0\), \(v(0)=\infty\)
\item \(v(xy)=v(x)+v(y)\)
\item \(v(x+y)\ge\min\{v(x),v(y)\}\)
\end{enumerate}

First we note that only the additive structure of \(\R\) together with the ordering on \(\R\) is
used, we will generalize this later. Secondly, \(\infty\) is a symbol that satisfies, for
all \(\gamma\in\R\), the following axiom:
\begin{equation*}
\infty=\infty+\infty=\gamma+\infty=\infty+\gamma
\end{equation*}

By an \textbf{ordered abelian group} we mean an abelian group \((\Gamma,+,0)\) together with a binary
relation \(\le\) on \(\Gamma\), where \(\le\) is a linear order on \(\Gamma\) and for any \(\gamma,\delta,\lambda\in\Gamma\),
\begin{equation*}
\gamma\le\delta\Rightarrow\gamma+\lambda\le\delta+\lambda
\end{equation*}

Let \(\Gamma\) be an ordered abelian group, and \(\infty\) a symbol satisfying for all \(\gamma\in\Gamma\),
\begin{equation*}
\infty=\infty+\infty=\gamma+\infty=\infty+\gamma.
\end{equation*}
We then define a \textbf{valuation} \(v\) on a field \(K\) to be a surjective map
\begin{equation*}
v:K\twoheadrightarrow\Gamma\cup\{\infty\}
\end{equation*}
satisfying the following axioms: for all \(x,y\in K\),
\begin{enumerate}
\item \(v(x)=\infty\Rightarrow x=0\)
\item \(v(xy)=v(x)+v(y)\)
\item \(v(x+y)\ge\min\{v(x),v(y)\}\)
\end{enumerate}

If \(\Gamma=\{0\}\), we call \(v\) the \textbf{trivial valuation}; for all \(x,y\in K\):
\begin{gather*}
v(1)=0,\hspace{1cm}v(x^{-1})=-v(x),\hspace{1cm}(-x)=v(x),\\
v(x)<v(y)\Rightarrow v(x+y)=v(x)
\end{gather*}

\begin{definition}[]
Let \(v:K^\times\to\Gamma\) be a valuation on a field. We set
\begin{enumerate}
\item \(\calo_v:=\{x\in K:v(x)\ge 0\}\)
\item \(\fm_v:=\{x\in K:v(x)>0\}\)
\item \(\bk_v:=\calo_v/\fm_v\).
\end{enumerate}
\end{definition}

For all \(x,y\in\calo_v\) we have
\begin{gather*}
v(x\pm y)\ge\min\{v(x),v(\pm y)\}\ge 0\\
v(xy)=v(x)+v(y)\ge 0
\end{gather*}
Hence \(x\pm y,xy\in\calo\). From \(v(x^{-1})=-v(x)\), we deduce that \(x\) is a unit in \(\calo_v\) iff \(v(x)=0\) and for
every \(x\in K\), either \(x\) or \(x^{-1}\) or both lie in \(\calo_v\). A subring \(\calo\) of \(K\)
satisfying
\begin{equation*}
x\in\calo \quad\text{ or }\quad x^{-1}\in\calo
\end{equation*}
for all \(x\in K^\times\) is called a \textbf{valuation ring} of \(K\). Thus \(\calo_v\)  is a valuation ring.
Moreover, \(\fm_v\) is an ideal of \(\calo_v\).
Since \(\fm_v\) consists exactly of the non-units of \(\calo_v\), \(\fm_v\) is a
maximal ideal, and in fact the only maximal ideal of \(\calo_v\). Thus \(\calo_v\) is a local ring(ring
with only one maximal ideal) and \(\bk_v\) is a field, called the \textbf{residue class field} of \(v\). The
residue class of \(a\in\calo_v\) is denoted
by \(\bara\). Note that \(v\) is trivial iff \(\calo_v=K\) iff \(\bk_v=K\). The group \(v(K^\times)\)
will be called the \textbf{value group} of \(v\).

\begin{proposition}[]
Let \(\calo\subseteq K\) be a valuation ring of \(K\). Then there exists a valuation \(v\) on \(K\) s.t. \(\calo=\calo_v\).
\end{proposition}

\begin{proof}
Denote by \(\calo^\times\) the group of units of \(\calo\). The group \(\Gamma=K^\times/\calo^\times\) is an abelian group and
we can define a binary relation on \(\Gamma\) by
\begin{equation*}
x\calo^\times\le y\calo^\times\Leftrightarrow\frac{y}{x}\in\calo
\end{equation*}
We can check that \(\Gamma\) is an ordered abelian group. The valuation is defined by
\begin{equation*}
v(x)=x\calo^\times\in\Gamma
\end{equation*}
for \(x\in K^\times\), and \(v(0)=\infty\). If \(v(x)\le v(y)\), then \(y/x\in\calo\). Therefore \((x+y)/x=1+y/x\in\calo\)
and \(v(x+y)\ge v(x)=\min\{v(x),v(y)\}\). Now
\begin{equation*}
\calo_v=\{x\in K\mid v(x)\ge 0\}=\{x\in K\mid x\in\calo\}=\calo
\end{equation*}
\end{proof}

\begin{examplle}[]
Consider \(K=\Q\), \(v=v_p\), then
\begin{align*}
\calo_{v_p}&=\{\frac{a}{b}\mid a,b\in\Z,b\text{ is not divisible by }p\}\\
\fm_{v_p}&=\{\frac{pa}{b}\mid a,b\in\Z,b\text{ is not divisible by }p\}
\end{align*}
\(\calo_{v_p}\) is the localization \(\Z_{(p)}=(\Z-(p))^{-1}\Z)\) of the ring \(\Z\) at the prime ideal \((p)=p\Z\),
and \(\fm_{v_p}\) is \(p\Z_{(p)}\). Thus the residue class field \(\bk_{v_p}\) is isomorphic to
the finite field \(\F_p\).
\end{examplle}

\begin{examplle}[]
Consider \(K=F((T))=\{\sum_{n=m}^\infty a_nT^n\mid m\in\Z,a_n\in F\}\), field of formal Laurent series with
valuation \(v(f)=m\) where \(f=\sum_{n=m}^\infty a_nT^n\) and \(a_m\neq 0\), then \(\calo_{v}=F[[T]]\), \(\fm_v\) is all
series \(\sum_{n=m}^\infty a_nT^n\) where \(m>0\) and the residue field \(\bk_v\) is \(F\).
\end{examplle}

\section{Hensel's Lemma}
\label{sec:org7402638}
\begin{definition}[]
A local domain \(A\) with maximal ideal \(\fm\) is \textbf{henselian} if whenever \(f(x)\in A[X]\) and there
is \(a\in A\) s.t. \(f(a)\in\fm\) and \(f'(a)\notin\fm\), then there is \(\alpha\in A\) s.t. \(f(\alpha)=0\)
and \(\alpha-a\in\fm\).

A \textbf{valued field} is a pair \((K,\calo)\) where \(K\) is a field and \(A\) is a valuation ring. A
valued field is \textbf{henselian} if its valuation ring is henselian.
\end{definition}

\begin{remark}
A ring is local iff all non-units form an ideal, therefore henselianity is a first-order property.
\end{remark}

\begin{theorem}[Hensel's Lemma]
Suppose \(K\) is a complete field with non-archimedean absolute value \(\abs{\;}\) and valuation
ring \(\calo=\{x\in K:\abs{x}\le 1\}\). Then \(\calo\) is henselian
\end{theorem}

\begin{proof}
Suppose \(a\in\calo_v\), \(\abs{f(a)}=\epsilon<1\) and \(\abs{f'(a)}=1\). We think of \(a\) as our first
approximation to a zero of \(f\) and use Newton's method to find a better approximation.

Let \(\delta=\frac{-f(a)}{f'(a)}\). Note that \(\abs{\delta}=\abs{\frac{f(a)}{f'(a)}}=\epsilon\). Consider the
Taylor expansion
\begin{equation*}
f(a+x)=f(a)+f'(a)x+\text{terms of degree at least 2 in \(x\)}
\end{equation*}
Thus
\begin{equation*}
f(a+\delta)=f(a)+f'(a)\frac{-f(a)}{f'(a)}+\text{terms of degree at least 2 in \(\delta\)}
\end{equation*}
Thus \(\abs{f(a+\delta)}\le\epsilon^2\). Similarly
\begin{equation*}
f'(a+\delta)=f'(a)+\text{terms of degree at least 2 in \(\delta\)}
\end{equation*}
and \(\abs{f'(a+\delta)}=\abs{f'(a)}=1\).

Thus starting with an approximation where \(\abs{f(a)}=\epsilon<1\) and \(\abs{f'(a)}=1\), we get a
better approximation \(b\) where \(\abs{f(b)}\le\epsilon^2\) and \(\abs{f'(b)}=1\). We now iterate this
procedure to build \(a=a_0,a_1,a_2,\dots\) where
\begin{equation*}
a_{n+1}=a_n-\frac{a_n}{f'(a_n)}
\end{equation*}
It follows, by induction, that for all \(n\):
\begin{enumerate}
\item \(\abs{a_{n+1}-a_n}\le\epsilon^{2^{n+1}}\)
\item \(\abs{f(a_n)}\le\epsilon^{2^n}\)
\item \(\abs{f'(a_n)}=1\)
\end{enumerate}
Thus \((a_n)_{n\in\N}\) is a Cauchy sequence and converges to \(\alpha\), \(\abs{\alpha-a}\le\epsilon\), and \(f(\alpha)=\lim_{n\to\infty}f(a_n)=0\)
\end{proof}

Therefore we have henselian field \((\Q_p,\Z_p)\) and \((F((T)),F[[T]])\).

\begin{fact}[Chevalley]
For a field \(K\), let \(A\subseteq K\) be a subring and let \(P\subseteq A\) be a prime ideal of \(A\). Then
there exists a valuation ring \(\calo\) of \(K\) s.t.
\begin{equation*}
R\subseteq\calo \quad\text{ and }\quad M\cap R=P
\end{equation*}
where \(M\) is the maximal ideal of \(\calo\).
\end{fact}

\begin{lemma}[]
Let \(K_2/K_1\) be a field extension and let \(\calo_1\subseteq K_1\) be a valuation ring. Then there is a
valuation ring \(\calo_2\subseteq K_2\) with \(\calo_2\cap K_1=\calo_1\).
\end{lemma}

\begin{proof}
Since \(\calo_1\) is a subring of \(K_2\), according to Chevalley's Theorem there exists a valuation
ring \(\calo_2\) of \(K_2\) with \(\calo_1\subseteq\calo_2\) and \(\fm_2\cap\calo_1=\fm_1\) for maximal ideals. Since \(\calo_2\cap K_1\)
and \(\calo_1\) are valuation rings with the same maximal ideal they must coincide.
\end{proof}


\begin{fact}[]
Let \((K,\calo)\) be a valued field. T.F.A.E.:
\begin{enumerate}
\item \((K,\calo)\) is henselian.
\item For any separable extension \(L/K\) there is a unique extension of \(\calo\) to a
valuation ring of \(L\).
\item For any algebraic extension \(L/K\) there is a unique extension of \(\calo\) to a valuation ring
of \(L\).
\end{enumerate}
\end{fact}

\section{Hahn Series}
\label{sec:orge8c8d13}
For each group \(\Gamma\) and field \(k\), there is a field \(K=k((t^\Gamma))\) with valuation \(v\) having
\(\Gamma\) as the value group and \(k\) as the residue field.
\begin{lemma}[]
Let \(A,B\subseteq\Gamma\) be well-ordered (by the ordering of \(\Gamma\)). Then \(A\cup B\) is well-ordered, the
set \(A+B:=\{\alpha+\beta:\alpha\in A,\beta\in B\}\) is well-ordered, and for each \(\gamma\in\Gamma\) there are only finitely
many \((\alpha,\beta)\in A\times B\)  s.t. \(\alpha+\beta=\gamma\).
\end{lemma}

\begin{proof}
Suppose \((a_0,b_0),(a_1,b_1),\dots\) are distinct s.t. \(a_i+b_i>a_j+b_j\) for \(i<j\). Then we can
find a strictly monotone subsequence of the \(a_i\). Since \(A\) is well-ordered, the sequence
cannot be decreasing. But then there is a strictly decreasing subsequence of \(b_i\).
\end{proof}

\begin{lemma}[Neumann's Lemma]
Let \(A\subseteq\Gamma^{>0}\) be well-ordered. Then
\begin{equation*}
[A]:=\{\alpha_1+\dots+\alpha_n:\alpha_1,\dots,\alpha_n\in A\}\hspace{2cm}(\text{allowing }n=0)
\end{equation*}
is also well-ordered, and for each \(\gamma\in[A]\) there are only finitely many
tuples \((n,\alpha_1,\dots,\alpha_n)\) with \(\alpha_1,\dots,\alpha_n\in A\) s.t. \(\gamma=\alpha_1+\dots+\alpha_n\)
\end{lemma}

Define \(K=k((t^\Gamma))\) to be the set of all formal series \(f(t)=\sum_{\gamma\in\Gamma}a_\gamma t^\gamma\) with
coefficients \(a_\gamma\in k\), s.t. the support of \(f\),
\begin{equation*}
\supp(f)    :=\{\gamma\in\Gamma:a_\gamma\neq 0\}
\end{equation*}
is a well-ordered subset of \(\Gamma\).By the first lemma, we can define binary operations of addition
and multiplication on \(k((t^\Gamma))\) as
\begin{gather*}
\sum a_\gamma t^\gamma+\sum b_\gamma t^\gamma=\sum(a_\gamma+b_\gamma)t^\gamma\\
\left( \sum a_\gamma t^\gamma \right)\left( \sum b_\gamma t^\gamma \right)=
\sum_\gamma\left( \sum_{\alpha+\beta=\gamma}a_\alpha b_\beta \right)t^\gamma
\end{gather*}
Define \(v:K\setminus\{0\}\to\Gamma\) by
\begin{equation*}
v(\sum a_\gamma t^\gamma):=\min\{\gamma:a_\gamma\neq 0\}
\end{equation*}
Then \(v\) is a valuation on \(K\). If \(v(f)>0\), then by the second lemma \(\sum_{n=0}^\infty f^n\)
makes sense as an element of \(K\): for any \(\gamma\in\Gamma\) there are only finitely many \(n\) s.t. the
coefficients of \(t^\gamma\) in \(f^n\) is not zero. Then
\begin{equation*}
(1-f)\sum_{n=0}^\infty f^n=1
\end{equation*}
Now for any \(g\in K\setminus\{0\}\), \(g=ct^\gamma(1-f)\), with \(c\in k^\times\) and \(v(f)>0\).
Then \(g^{-1}=c^{-1}t^{-\gamma}\sum_nf^n\).

For \(f=\sum a_\gamma t^\gamma\in K\), call \(a_0\) the constant term of \(f\). The map sending sending \(f\) to
its constant term sends \(\calo_v\) onto \(k\), and this is a ring homomorphism. Its kernel
is \(\fm_v\). Therefore \(\calo_v/\fm_v\cong k\).

We call \(K\) the \textbf{Hahn field}.

\begin{definition}[]
Let \(K\) be a valued field. We say that \(K\) is \textbf{spherically complete} if whenever \((I,<)\) is
a linear order and \((B_i:i\in I)\) is a family of open balls s.t. \(B_i\supset B_j\) for all \(i<j\),
then \(\bigcap_{i\in I}B_i\neq\emptyset\).
\end{definition}

\begin{definition}[]
If \((K,v)\) is a valuation field extending \(L\) as a subfield, then \(K\) is an \textbf{immediate
extension} if \(v(K)=v(L)\) and \(\bk_K=\bk_L\).
\end{definition}


\begin{fact}[]
\begin{enumerate}
\item Hahn field is henselian.
\item Hahn field is spherically complete.
\item Hahn field has no proper immediate extensions.
\end{enumerate}
\end{fact}
\end{document}