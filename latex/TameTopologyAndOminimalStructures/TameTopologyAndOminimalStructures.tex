% Created 2023-01-03 Tue 00:08
% Intended LaTeX compiler: pdflatex
\documentclass[11pt]{article}
\usepackage[utf8]{inputenc}
\usepackage[T1]{fontenc}
\usepackage{graphicx}
\usepackage{longtable}
\usepackage{wrapfig}
\usepackage{rotating}
\usepackage[normalem]{ulem}
\usepackage{amsmath}
\usepackage{amssymb}
\usepackage{capt-of}
\usepackage{hyperref}
\graphicspath{{../../books/}}
% TIPS
% \substack{a\\b} for multiple lines text





% pdfplots will load xolor automatically without option
\usepackage[dvipsnames]{xcolor}

\usepackage{forest}
% two-line text in node by [two \\ lines]
% \begin{forest} qtree, [..] \end{forest}
\forestset{
  qtree/.style={
    baseline,
    for tree={
      parent anchor=south,
      child anchor=north,
      align=center,
      inner sep=1pt,
    }}}
%\usepackage{flexisym}
% load order of mathtools and mathabx, otherwise conflict overbrace

\usepackage{mathtools}
%\usepackage{fourier}
\usepackage{pgfplots}
\usepackage{amsthm, mathabx,  amsmath, commath}
\usepackage{amsfonts}

\usepackage{empheq}
\usepackage{tikz}
\usetikzlibrary{arrows.meta}
\usepackage[most]{tcolorbox}

\newtheorem{theorem}{Theorem}[section]
\newtheorem{definition}{Definition}[section]
\newtheorem{corollary}{Corollary}[section]
\newtheorem{example}{Example}[section]
\newtheorem{lemma}{Lemma}[section]
\newtheorem{proposition}{Proposition}[section]

\newcommand{\bl}[1] {\boldsymbol{#1}}
\newcommand{\Wt}[1] {\stackrel{\sim}{\smash{#1}\rule{0pt}{1.1ex}}}
\newcommand{\wt}[1] {\widetilde{#1}}


%For boxed texts in align, use Aboxed{}
%otherwise use boxed{}

\DeclareMathSymbol{\widehatsym}{\mathord}{largesymbols}{"62}
\newcommand\lowerwidehatsym{%
  \text{\smash{\raisebox{-1.3ex}{%
    $\widehatsym$}}}}
\newcommand\fixwidehat[1]{%
  \mathchoice
    {\accentset{\displaystyle\lowerwidehatsym}{#1}}
    {\accentset{\textstyle\lowerwidehatsym}{#1}}
    {\accentset{\scriptstyle\lowerwidehatsym}{#1}}
    {\accentset{\scriptscriptstyle\lowerwidehatsym}{#1}}
}

\usepackage{graphicx}
    
% text on arrow for xRightarrow
\makeatletter
%\newcommand{\xRightarrow}[2][]{\ext@arrow 0359\Rightarrowfill@{#1}{#2}}
\makeatother


\def \bx {\boldsymbol{x}}
\def \ba {\boldsymbol{a}}
\def \bI {\boldsymbol{I}}
\def \bt {\boldsymbol{t}}
\def \bb {\boldsymbol{b}}
\def \bA {\boldsymbol{A}}
\def \bX {\boldsymbol{X}}
\def \bu {\boldsymbol{u}}
\def \bS {\boldsymbol{S}}
\def \bZ {\boldsymbol{Z}}
\def \bz {\boldsymbol{z}}
\def \by {\boldsymbol{y}}
\def \bw {\boldsymbol{w}}
\def \bT {\boldsymbol{T}}
\def \bS {\boldsymbol{S}}
\def \bm {\boldsymbol{m}}
\def \bW {\boldsymbol{W}}
\def \bY {\boldsymbol{Y}}
\def \bH {\boldsymbol{H}}
\def \blambda {\boldsymbol{\lambda}}
\def \bPhi {\boldsymbol{\Phi}}
\def \btheta {\boldsymbol{\theta}}
\def \bmu {\boldsymbol{\mu}}
\def \bphi {\boldsymbol{\phi}}
\def \bSigma {\boldsymbol{\Sigma}}
\def \lb {\left\{}
\def \rb {\right\}}
\def \caln {\mathcal{N}}
\def \dissum {\displaystyle\Sigma}
\def \dispro {\displaystyle\prod}
\def \E {\mathbb{E}}
\def \Q {\mathbb{Q}}
\def \V {\mathbb{V}}
\def \R {\mathbb{R}}
\def \calq {\mathcal{Q}}
\def \calg {\mathcal{G}}
\def \caln {\mathcal{N}}
\def \calr {\mathcal{R}}
\def \calm {\mathcal{M}}
\def \calc {\mathcal{C}}
\def \bcup {\bigcup}

\makeindex
\DeclareMathOperator{\tint}{\text{int}}
\author{Lou van den Dries}
\date{\today}
\title{Tame Topology And O-minimal Structures}
\hypersetup{
 pdfauthor={Lou van den Dries},
 pdftitle={Tame Topology And O-minimal Structures},
 pdfkeywords={},
 pdfsubject={},
 pdfcreator={Emacs 28.0.92 (Org mode 9.6)}, 
 pdflang={English}}
\begin{document}

\maketitle
\tableofcontents


\section{Some Elementary Results}
\label{sec:org226eacd}
\begin{definition}[]
A \textbf{structure} on a nonempty set \(R\) is a sequence \(\cals=(\cals_m)_{m\in\N}\) s.t. for each \(m\ge 0\)
\begin{enumerate}
\item \(\cals_m\) is a boolean algebra of subsets of \(R^m\)
\item if \(A\in\cals_m\), then \(R\times A\) and \(A\times R\) belong to \(\cals_{m+1}\) (\(\forall\))
\item \(\{(x_1,\dots,x_m)\in R^m:x_1=x_m\}\in\cals_m\)
\item if \(A\in S_{m+1}\), then \(\pi(A)\in\cals_m\) where \(\pi:R^{m+1}\to R^m\) is the projection map on the
first \(m\) coordinates \((\exists)\)
\item \(\{a\}\in\cals_1\) for \(a\in R\)
\end{enumerate}
\end{definition}

\begin{fact}[]
If \((R,\dots)\) is a model-theoretic structure and \(\cals_n=\{D\subseteq R^n:D\text{ is definable}\}\),
then \(\{\cals_n\}_{n\in\N}\) is a structure on \(R\)
\end{fact}

\begin{definition}[]
\(X\subseteq\C^n\) is \textbf{constructible} if \(X=\bigcup_{i=1}^mY_i\) where each \(Y_i\) has the form
\begin{equation*}
\{\barx\in\C^n:P_1(\barx)=0,\dots,P_n(\barx)=0, Q_1(\barx)\neq 0,\dots,Q_n(\barx)\neq 0\}
\end{equation*}
\end{definition}

\begin{fact}[]
If \(S_m=\{D\subseteq\C^m:D\text{ constructible}\}\), then \(\{\cals_n\}_{n\in\N}\) is a structure on \(\C\)
\end{fact}

\begin{theorem}[Chevalley's Theorem, Quantifier elimination in \(\C\)]
Projections works
\end{theorem}

\begin{definition}[]
\(X\subseteq\R^n\) is \textbf{semialgebraic} if \(X\) is a finite union of sets of the form
\begin{equation*}
\{\barx\in\R^n:P_1(\barx)=0,\dots,P_n(\barx)=0,Q_1(\barx)> 0,\dots,Q_m(\barx)>0\}
\end{equation*}
\end{definition}

Semialgebraic sets are closed under intersection, union, complement, cartesian product, projection

\begin{fact}[Tarski-Seidenberg]
Semialgebraic sets are a structure on \(\R\) (projection)
\end{fact}

\begin{fact}[]
If \(f:X\to Y\) is definable
\begin{enumerate}
\item if \(f^{-1}\) exists then \(f^{-1}\) is definable
\item if \(g:Y\to Z\) is definable, then \(g\circ f\) is definable
\item if \(A\subseteq X\) is definable, then \(f(A)\) is definable
\item if \(A\subseteq Y\) is definable, then \(f^{-1}(A)\) is definable
\item If \(A\subseteq X\) is definable, then so is \(f\uhr A\)
\end{enumerate}
\end{fact}

Given functions \(f,g:X\to R_\infty\) on a set \(X\subseteq R^m\) we put
\begin{align*}
(f,g)&:=\{(x,r)\in X\times R:f(x)<r<g(x)\}\\
[f,g]&:=\{(x,r)\in X\times R_\infty:f(x)\le r\le g(x)\}
\end{align*}
We consider \((f,g)\) as a subset of \(R^{m+1}\); also \([f,g]\subseteq R^{m+1}\) if \(f\) and \(g\) are \(R\)-valued

\begin{definition}[]
Let \((R,<)\) be a dense linearly ordered nonempty set without endpoints. An \textbf{o-minimal structure}
on \((R,<)\) is by definition a structure \(\cals\) on \(R\) s.t.
\begin{enumerate}
\item \(\{(x,y)\in R^2:x<y\}\in\cals_2\)
\item the sets in \(\cals_1\) are exactly the finite unions of intervals and points
\end{enumerate}
\end{definition}

In \(\R\), ``definable'' = ``semialgebraic'', in \(\Q\), ``definable'' = ``semilinear''

\begin{fact}[]
Semialgebraic sets are an o-minimal structure on \(\R\)
\end{fact}

context
\begin{itemize}
\item \((R,\le)\) dense linear order with no endpoints
\item for each \(n\), there's \(\cals_n\)
\end{itemize}

Fix an o-minimal structure \(\cals\) on \((R,<)\)

Why o-minimality?
\begin{enumerate}
\item results results for definable sets
\item a bunch of o-minimal structures exist
\end{enumerate}

\begin{fact}[Wilkie]
There is an o-minimal structure on \(\R\) where \(\exp(-)\), \(\log(-)\) are definable
\end{fact}

\(\sin(x)\) cannot be definable in o-minimal structure on \(\R\)

\begin{lemma}[]
Let \(A\subseteq R\) be definable. Then
\begin{enumerate}
\item \(\inf(A)\) and \(\sup(A)\) exist in \(R_\infty\) (dedekind completeness for definable sets)
\item the
boundary \(bd(A):=\{x\in R:\text{ each interval containing $x$ intersects both $A$ and $R-A$}\}\)
is finite, and if \(a_1<\dots<a_k\) are the points of \(bd(A)\) in order, then each
interval \((a_i,a_{i+1})\), where \(a_0=-\infty\) and \(a_{k+1}=+\infty\) is either part of \(A\) or
disjoint from \(A\)
\item If \(\abs{X}=\infty\) then \(X\supseteq I\) for some \(I\)
\item If \(X\) is dense in \(I\), then \(\abs{X}=\infty\), \(X\supseteq J\) (not true in \(\Q\)/\(\R\)) (\(X\subseteq I\) is
\textbf{dense} in \(I\)if \(\forall J\subseteq I(J\cap X\neq\emptyset)\))
\item If \(p\in R\), then \(\exists b>a\) s.t. \((a,b)\subseteq X\) or \((a,b)\cap X=\emptyset\). Locally,
\end{enumerate}
\end{lemma}

\begin{proof}
\begin{enumerate}
\setcounter{enumi}{1}
\item \(bd(X\cup Y)\subseteq bd(X)\cup bd(Y)\)
\item \(X\) is a union of interval and points
\end{enumerate}
\end{proof}

\begin{lemma}[]
\begin{enumerate}
\item If \(A\subseteq R^m\) is definable, so are \(\cl(A)\) and \(\tint(A)\)
\item If \(A\subseteq B\subseteq R^m\) are definable sets, and \(A\) is open in \(B\), then there is a definable
open \(U\subseteq R^m\) with \(U\cap B=A\)
\end{enumerate}
\end{lemma}

\begin{proof}
\begin{gather*}
(x_1,\dots,x_m)\in\cl(A)\\
\Leftrightarrow\\
(\forall y_1,\dots,y_m\forall z_1,\dots,z_m[y_1<x<z_1\wedge\dots\wedge y_m<x_m<z_m)\to\\
\exists a_1,\dots,a_m(y_1<a<z_1)\wedge\dots\wedge y_m<a_m<z_m\wedge(a_1,\dots,a_m)\in A]
\end{gather*}

take \(U\) is as the union of all boxes in \(R^m\)  whose intersection with \(B\) is contained in \(A\)
\end{proof}

\begin{definition}[]
A set \(X\subseteq R^m\) is \textbf{definably connected} if \(X\) is definable and \(X\) is not the union of two
disjoint nonempty definable open subsets of \(X\)
\end{definition}

\begin{lemma}[]
\label{1.3.6}
\begin{enumerate}
\item the definably connected subsets of \(R\) are the following: the empty set, the intervals, the
sets \([a,b)\) with \(-\infty<a<b\le+\infty\), the sets \((a,b]\) with \(-\infty\le a<b<+\infty\) and the sets
\([a,b)\) with \(-\infty<a\le b<+\infty\), and the sets \([a,b]\) with \(-\infty<a\le b<+\infty\)
\item the image of a definably connected set \(X\subseteq R^m\) under a definable continuous map \(f:X\to R^n\)
is definably connected
\item if \(X\) and \(Y\) are definable subsets of \(R^m\), \(X\subseteq Y\subseteq\cl(X)\), and \(X\) is definably
connected, then \(Y\) is definably connected
\item if \(X\) and \(Y\) are definably connected subsets of \(R^m\) and \(X\cap Y\neq\emptyset\) , then \(X\cup Y\)
is definably connected
\end{enumerate}
\end{lemma}

\begin{proof}
\begin{enumerate}
\setcounter{enumi}{2}
\item suppose \(Y=U_1\cup U_2\) where \(U_1,U_2\) are definably open, then \(X\subseteq U_1\) or \(X\subseteq U_2\)
\end{enumerate}
\end{proof}

note the following special case of (2):
\begin{center}
If the function \(f:[a,b]\to R\) is definable and continuous, then \(f\) assumes all values
between \(f(a)\) and \(f(b)\)
\end{center}

\begin{lemma}[]
If \(I,J\subseteq R\) intervals, \(X\subseteq R\) definable, \(I<J\), \(\abs{I\setminus X}=\infty=\abs{J\cap X}\), then there
is \(a\) s.t. \(I<a<J\), and there is \(c<a<b\) s.t. \((c,a)\cap X=\emptyset\), \((a,b)\subseteq X\)
\end{lemma}

\begin{proof}
take \(a=\inf X\setminus bd(X)\)
\end{proof}
\subsection{O-minimal ordered groups and rings}
\label{sec:org422bf33}
\textbf{Order group} is a group equipped with a linear order that is invariant under left and right
 multiplication:
 \begin{equation*}
x<y\Rightarrow zx<zy\wedge xz<yz
 \end{equation*}

\begin{lemma}[]
The only definable subsets of \(R\) that are also subgroups are \(\{1\}\) and \(R\)
\end{lemma}

\begin{proof}
Given a definable subgroup \(G\) we first show that \(G\) is convex: if not, then there
are \(g\in G\), \(r\in R-G\) with \(1<r<g\). This gives a sequence
 \begin{equation*}
1<r<g<rg<g^2<rg^2<g^3<\dots
 \end{equation*}
whose terms alternate in being in and out of the definable set \(G\).

So \(G\) is convex, hence assuming \(G\neq\{1\}\) we have \(s:=\sup(G)>1\) with \((1,s)\subseteq G\).
If \(G=+\infty\), then clearly \(R=G\). If \(s<+\infty\), then we take any \(g\in(1,s)\) and
obtain \(s=gg^{-1}s\in G\), since \(g^{-1}s\in(1,s)\) hence \(s<gs\in G\)
\end{proof}

\begin{proposition}[]
Suppose \((R,<,\cals)\) is an o-minimal structure and \(\cals\) contains a binary operation \(\cdot\)
on \(R\), s.t. \((R,<,\cdot)\) is an ordered group. Then the group \((R,\cdot)\) is abelian, divisible
and torsion-free
\end{proposition}

\begin{proof}
for each \(r\in R\) the centralizer \(C_r:=\{x\in R:rx=xr\}\) is a definable subgroup
containing \(r\), so \(C_r=R\)  by the lemma. Hence \(R\) is abelian. For each \(n>0\) the
subgroup \(\{x^n:x\in R\}\) is definable, hence equal to \(R\). Every ordered group is torsion free
\end{proof}

\begin{remark}
Let \((R,<,+)\) be an ordered abelian group, \(R\neq\{0\}\), so \((R,<)\) has no endpoints. Assume
also that the linearly ordered set \((R,<)\) is dense. Then the addition operation \(+:R^2\to R\)
and the additive inverse operation \(-:R\to R\) are continuous w.r.t. the interval topology, that
is, \((R,+)\) is a topological group w.r.t. the interval topology
\end{remark}

An \textbf{ordered ring} is a ring (associative with 1) equipped with a linear order < s.t.
\begin{enumerate}
\item \(0<1\)
\item < is translation invariant
\item < is invariant under multiplication by positive elements
\end{enumerate}

Note that then the additive group of the ring is an ordered group, that the ring has no zero
divisors, that \(x^2\ge 0\) for all \(x\), and that \(k\mapsto k\cdot 1:\Z\to\text{ring}\) is a strictly
increasing ring embedding

suppose our ordered ring is moreover a \textbf{division ring}: for each \(x\neq 0\) there is \(y\)
with \(x\cdot y=1\). It is easy to check that such a \(y\) is unique, and satisfies \(y\cdot x=1\) and
that \(x>0\) implies \(y>0\). It is easy to see that the additive group is divisible, the
underlying ordered set is dense without endpoints, and the maps \((x,y)\to xy\) and \(x\mapsto x^{-1}\)
are continuous w.r.t. the interval topology

An \textbf{ordered field} is an ordered division ring with commutative multiplication. Examples: field of
reals, field of rational numbers. Define \textbf{real closed field} to be an ordered field s.t.
if \(f(X)\) is a one-variable polynomial with coefficients in the field and \(a<b\) are elements
in the field with \(f(a)<0<f(b)\), then there is \(c\in(a,b)\) in the field with \(f(c)=0\)


\begin{proposition}[]
Suppose \((R,<,\cals)\) is an o-minimal structure and \(\cals\) contains binary operations \(+:R^2\to R\)
and \(\cdot:R^2\to R\) s.t. \((R,<,​+,\cdot)\). Then \((R,<, ​+,\cdot)\) is a real closed field
\end{proposition}

\begin{proof}
For each \(r\in R\) we have a definable additive subgroup \(rR\) of \((R,+)\), hence \(rR=R\)
if \(r\neq 0\). This shows that \((R,<,+,\cdot)\) is an ordered division ring.
Let \(Pos(R)=\{r\in R:r>0\}\). Clearly \(Pos(R)\) is an ordered multiplicative group. By
restricting \(\cals\) to \(Pos(R)\) it follows from the previous proposition that multiplication is
commutative on \(Pos(R)\), hence on all of \(R\). So \((R,<,+,\cdot)\) is an ordered field. Each
one-variable polynomial \(f(X)\in R[X]\) gives rise to a definable continuous
function \(x\mapsto f(x):R\to R\). Now apply \ref{1.3.6}
\end{proof}
\subsection{Model-theoretic structures}
\label{sec:org4d25b6c}
\begin{definition}[]
A model-theoretic structure \(\calr=(R,<,\dots)\) where < is a dense linear order without endpoints
on \(R\), is called \textbf{o-minimal} if \(\Def(\calr_R)\) is an o-minimal structure on \((R,<)\), in other
words, every set \(S\subseteq R\) that is definable in \(\calr\) using constants is a union of finitely many
intervals and points
\end{definition}

\((\R,+,\cdot,\le)\), \((\R,+,\le)\), \((\Q,+,\le)\), \((\R,\le)\), \((\Q,\le)\)

Wilkie's theorem: \((\R,+,\cdot,\le,\exp)\),
\subsection{The simplest o-minimal structures}
\label{sec:org4c1b96a}
Let \((R,<)\) be a dense linearly ordered nonempty set without endpoints

We prove below that the model theoretic structure \((R,<)\) is o-minimal

Let \(1\le i\le m\). The function \((x_1,\dots,x_m)\mapsto x_i:R^m\to R\) will be denoted by \(x_i\). The \textbf{simple}
functions on \(R^m\) are by definition these coordinate functions \(x_1,\dots,x_m\) and the constant
functions \(R^m\to R\)

Let \(f_1,\dots,f_N\) be simple functions on \(R^m\), and let \(\epsilon:\{1,\dots,N\}^2\to\{-1,0,1\}\) be given. Then
we put
\begin{align*}
\epsilon(f_1,\dots,f_N):=\{x\in R^m:&\forall(i,j)\in\{1,\dots,N\}^2\\
&f_i(x)<f_j(x)\text{ if }\epsilon(i,j)=-1\\
&f_i(x)=f_j(x)\text{ if }\epsilon(i,j)=0\\
&f_i(x)>f_j(x)\text{ if }\epsilon(i,j)=1\}
\end{align*}
If \(\xi\) and \(\eta\) are the restrictions of \(f_i\) and \(f_j\) to \(\epsilon(f_1,\dots,f_N)\), then either \(\xi<\eta\)
or \(\xi=\eta\) or \(\xi>\eta\). Let \(\xi_1<\dots<\xi_k\) be the restrictions of \(f_1,\dots,f_N\) to \(\epsilon(f_1,\dots,f_N)\)
arranged in increasing order. One checks easily that the sets \(\Gamma(\xi_j)\) \((1\le j\le k)\) and the
sets \((\xi_j,\xi_{j+1})\) (\(0\le j\le k\), where \(\xi_0=-\infty\) and \(\xi_{k+1}=+\infty\) by convention) are
exactly the nonempty subsets of \(R^{m+1}\) of the form \(\epsilon'(f_1,\dots,f_N,x_{m+1})\) where
\begin{equation*}
\epsilon':\{1,\dots,N,N+1\}^2\to\{-1,0,1\}
\end{equation*}
is an extension of \(\epsilon\).
\wu{
suppose \(x_{m+1}(x)=y\), we only need to know the relation among \(f_1(x),\dots,f_N(x),y\). And
\(\bigcup\Gamma(\xi_j)\cup\bigcup(\xi_j,\xi_{j+1})=\epsilon(f_1,\dots,f_N)\times R\)
}

Define a \textbf{simple set} in \(R^m\) to be the subset of \(R^m\) of the
form \(\epsilon(f_1,\dots,f_N)\) with \(f_1,\dots,f_N\) simple functions on \(R^m\) and \(\epsilon:\{1,\dots,N\}^2\to\{-1,0,1\}\).
We have just proved that if \(S\subseteq R^{m+1}\) is simple, then its image under the projection map
    \begin{equation*}
(x_1,\dots,x_m,x_{m+1})\mapsto(x_1,\dots,x_m):R^{m+1}\to R^m
    \end{equation*}
is simple in \(R^m\)

\begin{proposition}[]
The subsets of \(R^m\) that are definable in \((R,<)\) using constants are exactly the finite
unions of simple sets in \(R^m\)
\end{proposition}

\begin{proof}
Let \(\cals_m\) be the collection of finite unions of simple sets in \(R^m\). Clearly \(\cals_m\) is a
boolean algebra of subsets of \(R^m\), and each set in \(\cals_m\) is definable in \((R,<)\) using
constants. Texts above show that \(\cals:=(\cals_m)_{m\in\N}\) is a structure on the set \(R\), hence the
sets in \(\cals_m\) are exactly the subsets of \(R^m\) definable in \((R,<)\) using constants
\end{proof}

\begin{corollary}[]
The model-theoretic structure \((R,<)\) is o-minimal
\end{corollary}
\subsection{Semilinear sets}
\label{sec:org4423948}
In this section we show that the sets definable using constants in an ordered vector space over
an ordered field are exactly the semilinear sets.

definition
\section{Semialgebriac sets}
\label{sec:orgc84d90e}
\subsection{Thom's lemma and continuity of roots}
\label{sec:orgf53ac04}
\begin{lemma}[]
Let \(\alpha\in\C\) be a zero of the monic polynomial
\begin{equation*}
a_0+a_1T+\dots+a_{d-1}T^{d-1}+T^d\in\C[T],d\ge 1
\end{equation*}
Then \(\abs{\alpha}\le 1+\max\{\abs{a_i}:i=0,\dots,d-1\}\)
\end{lemma}

\begin{proof}
Put \(M:=\max\{\abs{a_i}:i=0,\dots,d-1\}\) and suppose \(\alpha>1+M\). Then
\(\abs{a_0+a_1\alpha+\dots+a^{d-1}\alpha^{d-1}}\le M(1+abs{\alpha}+\dots+\abs{\alpha}^{d-1})=M(\abs{\alpha}^d-1)/(\abs{\alpha}-1)<\abs{\alpha}^d\),
contradicting \(0=\abs{f(\alpha)}\)
\end{proof}

\begin{lemma}[Thom]
Let \(f_1,\dots,f_k\in\R[T]\) be nonzero polynomials s.t. if \(f_i'\neq 0\), then \(f_i'\in\{f_1,\dots,f_k\}\).
Let \(\epsilon:\{1,\dots,k\}\to\{-1,0,1\}\), and put
\begin{equation*}
A_\epsilon:=\{t\in\R:\sgn(f_i(t))=\epsilon(i),i=1,\dots,k\}\subseteq\R
\end{equation*}
Then \(A_\epsilon\) is empty, a point, or an interval. If \(A_\epsilon\neq\emptyset\), then its closure is given by
\begin{equation*}
\cl(A_\epsilon)=\{t\in\R:\sgn(f_i(t))\in\{\epsilon(i),0\},i=1,\dots,k\}
\end{equation*}
If \(A_\epsilon=\emptyset\), then \(\{t\in\R:\sgn(f_i(t))\in\{\epsilon(i),0\},i=1,\dots,k\}\) is empty or a point
\end{lemma}

We call \(\epsilon\) a \textbf{sign condition} for \(f_1,\dots,f_k\). The \(3^k\) possible sign conditions \(\epsilon\)
determine \(3^K\) disjoint sets \(A_\epsilon\), which together cover the real line \(\R\). The second
statement of the lemma says that for nonempty \(A_\epsilon\) its closure can be obtained by relaxing
all strict inequalities to weak inequalities

\begin{proof}
By induction on \(k\). The lemma holds trivially for \(k=0\). Let \(f_1,\dots,f_k,f_{k+1}\in\R[T]-\{0\}\)
be polynomials s.t. if \(f_i'\neq 0\), then \(f_i'\in\{f_1,\dots,f_{k+1}\}\). We may assume
that \(\deg(f_{k+1})=\max\{\deg(f_i):1\le i\le k+1\}\). Let \(\epsilon':\{1,\dots,k+1\}\to\{-1,0,1\}\), and let \(\epsilon\) be the
restriction of \(\epsilon'\) to \(\{1,\dots,k\}\). By the inductive hypothesis, \(A_\epsilon\) is empty, a point or
an interval. It \(A_\epsilon\) is empty or a point, so
is \(A_{\epsilon'}=A_\epsilon\cap\{t\in\R:\sgn(f_{k+1}(t))=\epsilon'(k+1)\}\), and the other properties to be checked in this
case follow easily from the inductive hypothesis on \(A_\epsilon\)

Suppose \(A_\epsilon\) is an interval. Since \(f_{k+1}'\) has a constant sign on \(A_\epsilon\), the
function \(f_{k+1}\) is either strictly monotone on \(A_\epsilon\), or constant. In both cases, it is
routine to check that \(A_{\epsilon'}=A_\epsilon\cap\{t\in\R:\sgn(f_{k+1}(t))=\epsilon'(k+1)\}\) has the required properties
\end{proof}

\begin{lemma}[Continuity of roots]
Let \(f(T)=a_0+a_1+\dots+a_dT^d\in\C[T]\) be a polynomial that has no zero on the boundary
circle \(\abs{z-c}=r\) of a given open disc \(\abs{z-c}<r\) in the complex plane \((c\in\C,r>0)\).
Then there is \(\epsilon>0\) s.t. if \(\abs{a_i-b_i}\le\epsilon\) for \(i=0,\dots,d\)
then \(g(T):=b_0+b_1T+\dots+b_dT_d\in\C[T]\) also has no zero on the circle, and \(f\) and \(g\) have
the same number of zeros in the disc
\end{lemma}
\section{Cell Decomposition}
\label{sec:org94b25fe}
Fix an arbitrary o-minimal structure \((R,<,\cals)\). Instead of saying that a set \(A\subseteq R^m\) belongs
to \(\cals\), we say that \(A\) is definable
\subsection{The monotonicity theorem and the finiteness lemma}
\label{sec:orgb46e75b}
\begin{theorem}[Monotonicity theorem]
Let \(f:(a,b)\to R\) be a definable function on the interval \((a,b)\). Then there are
points \(a_1<\dots<a_k\) in \((a,b)\) s.t. on each subinterval \((a_j,a_{j+1})\)
with \(a_0=a\), \(a_{k+1}=b\), the function is either constant, or strictly monotone and continuous
\end{theorem}


We derive this from the threes below. In these lemmas we consider a definable
function \(f:I\to R\) on an interval \(I\)
\begin{lemma}[]
\label{3.1.1}
There is a subinterval of \(I\) on which \(f\) is constant or injective
\end{lemma}

\begin{lemma}[]
\label{3.1.2}
If \(f\) is injective, then \(f\) is strictly monotone on a subinterval of \(I\)
\end{lemma}

\begin{lemma}[]
\label{3.1.3}
If \(f\) is strictly monotone, then \(f\) is continuous on a subinterval of \(I\)
\end{lemma}

These lemmas imply the monotonicity theorem as follows:

Let
\begin{align*}
X:=\{x\in(a,b):&\text{on some subinterval of $(a,b)$ containing $x$ the function $f$}\\
&\text{is either constant, or strictly monotonicity and continuous}\}
\end{align*}
Now \((a,b)-X\) must be finite, since otherwise it would contain an interval \(I\); applying
successively lemmas \ref{3.1.1}, \ref{3.1.2}, \ref{3.1.3} we can make \(I\) so small that \(f\) is
either constant, or strictly monotone and continuous on \(I\). But then \(I\subseteq X\), a
contradiction \label{Problem1}

Since \((a,b)-X\) is finite, we can reduce the proof of the theorem to the case
that \((a,b)=X\), by replacing \((a,b)\) by each of the finitely many intervals of which the
open set \(X\) consists. In particular, we may assume that \(f\) is continuous. By splitting
up \((a,b)\) further we can reduce to one of the following three cases

Case 1. For all \(x\in(a,b)\), \(f\) is constant on some neighborhood of \(x\)

Case 2. For all \(x\in(a,b)\), \(f\) is strictly increasing on some neighborhood of \(x\)

Case 3. For all \(x\in(a,b)\), \(f\) is strictly decreasing on some neighborhood of \(x\)

Case 1. Take \(x_0\in(a,b)\) and put
\begin{equation*}
s:=\sup\{x:x_0<x<b,f\text{ is constant on }[x_0,x)\}
\end{equation*}
Then \(s=b\), since \(s<b\) implies that \(f\) is constant on some neighborhood of \(s\),
contradiction. From \(s=b\) it follows that \(f\) is constant on \([x_0,b)\). Similarly we prove
that \(f\) is constant on \((a,x_0]\), therefore \(f\) is constant on \((a,b)\)

Case 2. Take \(x_0\in(a,b)\) and put
\begin{equation*}
s:=\sup\{x:x_0<x<b,f\text{ is strictly increasing on }[x_0,x)\}
\end{equation*}
Then \(s=b\), since \(s<b\) leads to a contradiction

We now prove the lemmas

\begin{proof}[Proof of Lemma \ref{3.1.1}]
If some \(y\in R\) had infinite preimage \(f^{-1}(y)\), then this preimage would contain a
subinterval of \(I\) and \(f\) would take the constant value \(y\) on that subinterval. So we
may assume that each \(y\in R\) has finite preimage. Then \(f(I)\) is infinite, and so contains an
interval \(J\). Define an ``inverse'' \(g:J\to I\) by
\begin{equation*}
g(y):=\min\{x\in I:f(x)=y\}
\end{equation*}
Since \(g\) is injective by definition, \(g(J)\) is infinite, and hence \(g(J)\) contains a
subinterval of \(I\), and \(f\) is necessarily injective on this subinterval

If \(x_1,x_2\in J'\subseteq g(J)\), \(x_i=g(y_i)\), \(f(x_1)=f(x_2)\Rightarrow y_1=y_2\Rightarrow x_1=x_2\) and \(f\) is injective
\end{proof}
Fix \(f:I\to R\), \(a\in I\),  \(\Phi_{-+}(a)\) means \(\exists\epsilon\) s.t. if \(x\in(a-\epsilon,a)\) then \(f(x)<f(a)\),
and if \(x\in(a,a+\epsilon)\) then \(f(x)>f(a)\). ``locally increasing''

\(\Phi_{+-}(a)\), \(\Phi_{++}(a)\), \(\Phi_{--}\) is similar

\(\Phi_{00}(a)\), \(\exists\epsilon\), \(x\in(a-\epsilon,a+\epsilon)\Rightarrow f\) is increasing

\begin{definition}[]
\(a\in slbd(D)\) if \((a-\epsilon,a)\cap D=\emptyset\), \((a,a+\epsilon)\subseteq D\), strong left boundary
\end{definition}

\begin{fact}[]
If \(X,Y\subseteq R\), \(\abs{X}=\abs{Y}=\infty\), \(X<Y\), if \(D\subseteq R\), \(X\cap D=\emptyset\), \(Y\subseteq D\)
then \(\exists a\in slbd(D)\), \(X\le a\le Y\)
\end{fact}


\begin{lemma}[]
If \(\Phi_{-+}(a)\), \(\forall a\in I\), then \(f\) is increasing
\end{lemma}

\begin{proof}
suppose \(a,b\in I\), \(a<b\), \(f(a)\ge f(b)\). there is \(\epsilon\) s.t.
if \(x\in(a,a+\epsilon)\) then \(f(x)>f(a)\), and if \(x\in(b-\epsilon,b)\), \(f(x)<f(b)\le f(a)\).

\(D=\{x:f(x)\le f(a)\}\), \((a,a+\epsilon)\cap D=\emptyset\), \((b-\epsilon,b)\subseteq D\), then \(\exists c\in slbd(D)\), \(c-\delta,c\cap D=\emptyset\)
and \((c,c+\delta)\subseteq D\), so \(\Phi_{-+}(c)\) is false
\end{proof}

\begin{lemma}[]
\begin{enumerate}
\item If \(\forall a\in I\), \(\Phi_{+-}(a)\), then \(f\) is decreasing
\item If \(\forall a\in I\),  \(\Phi_{00}(a)\), then \(f\) is constant
\end{enumerate}
\end{lemma}

\begin{lemma}[]
If \(f:I\to R\) injective, \(a\in I\), then \(\Phi_{++}(a)\) or \(\Phi_{+-}(a)\) or \(\Phi_{-+}(a)\) or \(\Phi_{--}(a)\)
\end{lemma}

if \(f\) is not injective, then there may be 9 cases

\begin{fact}[]
If \(D\subseteq R\) definable, \(a\in R\), then there is \(\epsilon\) s.t. \((a,a+\epsilon)\subseteq D\) or \((a,a+\epsilon)\cap D=\emptyset\) and
\((a-\epsilon,a)\subseteq D\) or \((a-\epsilon,a)\cap D=\emptyset\)
\end{fact}

\begin{proof}
Let \(D=\{x\in I:f(x)>f(a)\}\), then the fact gives 4 cases
\end{proof}

\begin{lemma}[]
\label{w3.4}
If \(f:I\to R\) is definable
\begin{enumerate}
\item It can't be that: \(\forall a\in I\), \(\Phi_{++}(a)\)
\item It can't be that: \(\forall a\in I\), \(\Phi_{--}(a)\)
\end{enumerate}
\end{lemma}

\begin{proof}
\begin{enumerate}
\item Assume \(\forall x\Phi_{++}(x)\)

\(\Psi_{+-}(a)\Leftrightarrow\exists y,\epsilon\), if \(x\in(a-\epsilon,a)\), then \(f(x)>y\), \(x\in (a,a+\epsilon), f(x)<y\)

Let \(I=(a,b)\), \(S=\{x\in I\mid \exists x'\in I, x'>x,f(x')<f(x)\}\)

\textbf{Case 1}: \((\exists \epsilon)(b-\epsilon,b)\cap S=\emptyset\). Then on the interval \((b-\epsilon,b)\), \(f\) is
increasing, \(\Phi_{++}(x)\) doesn't hold

\textbf{Case 2}: \((\exists\epsilon)(b-\epsilon,b)\subseteq S\)

Take \(x_0\in(b-\epsilon,b)\), \(x_0\in S\), and we could get a decreasing sequence

 Let \(D=\{x\in I:f(x)>f(x_0)\}\). So there are infinitely many points \(<x_0\) in \(D\), and
infinitely many points \(>x_0\) not in \(D\)

\(\exists c\) s.t. \((c-\epsilon,c)\subseteq D\), \((c,c+\epsilon)\cap D=\emptyset\). So \(\Psi_{+-}(c)\) is true
\end{enumerate}
\end{proof}

\begin{lemma}[]
\(\exists J\subseteq I\), \(\forall x\in J\), \(\Psi_{+-}(x)\),
\end{lemma}

\begin{proof}
\(S=\{x\in I:\Psi_{+-}(x)\}\). If \(S\) is finite, replace \(I\) with \(I'\subseteq I\setminus S\), replace \(f\)
with \(f|_{I'}\), apply previous lemma, get \(c\in I'\), \(\Psi_{+-}(c)\), a contradiction
\end{proof}

Similarly,     \(\exists J\subseteq I\), \(\forall x\in J\), \(\Psi_{-+}(x)\)

Combine these, get \(I\supseteq I'\supseteq I''\), \(\forall x\in I'\), \(\Psi_{+-}(x)\), and \(\forall x\in I''\), \(\Psi_{+-}(x)\), a contradiction

\begin{lemma}[]
\label{w3.5}
If \(f:I\to R\), \(\exists a\in I\), \(\Phi_{-+}(a)\) or \(\Phi_{+-}(a)\) or \(\Phi_{00}(a)\)
\end{lemma}

\begin{proof}
By Lemma \ref{3.1.1}, there is \(J\subseteq I\), if \(f|_J\) is constant, then we are done.

If \(f|_J\) is injective, let \(S_{+-}=\{a\in J,\Phi_{+-}(a)\}\) and other sets
similarly. \(J=S_{+-}\cup S_{++}\cup S_{-+}\cup S_{--}\). If \(\abs{S_{++}}=\infty\), there
is \(I'\subseteq S_{++}\), a contradiction. Therefore \(S_{--}\) and \(S_{++}\) are finite. But \(\abs{J}\)
is infinite, so \(S_{+-}\) or \(S_{-+}\) is nonempty
\end{proof}

\begin{lemma}[]
\label{w3.6}
\(f:I\to R\), \(\exists c_0<c_1<\dots<c_n\), \(I=(c_0,c_n)\), \(f|_{(c_i,c_{i+1})}\) is constant or decreasing
or increasing
\end{lemma}

\begin{proof}
Let \(E=I\setminus(S_{+-}\cup S_{-+}\cup S_{00})\). If \(\abs{E}=\infty\), then \(J\subseteq E\)
and \(f|_E\) contradicts \ref{w3.5}.
Take \(\{c_0,\dots,c_n\}\supseteq E\cup bd(I)\cup bd(S_{+-})\cup bd(S_{-+})\cup bd(S_{00})\).

So all the sets respect the partition
\begin{equation*}
(c_0,c_1),\{c_1\},(c_1,c_2),\dots,(c_{n-1},c_n)
\end{equation*}
\end{proof}

\begin{lemma}[]
\label{w3.7}
If \(f:I\to R\) definable and \(S=\{x\in I:f\text{ is not continuous at }x\}\), then \(S\) is finite
\end{lemma}

\begin{proof}
\(S\) is definable. If \(\abs{S}=\infty\), take \(J\subseteq S\), replace \(f\) with \(f|_J\), we may
assume \(f\) is nowhere continuous. By Lemma \ref{w3.6}, there is \(J\subseteq I\), \(f|_J\) is constant
or monotone. Replace \(f\) with \(f|_J\), now \(f\) is monotone (constant is continuous).
Assume \(f\) is increasing, then \(f\) is injective, \(\abs{f(I)}=\infty\),
take \(J\subseteq f(I)\), \([c,d]\subseteq f(I)\), \(c=f(a)\), \(d=f(b)\), \(x\in(a,b)\Rightarrow f(x)\in(c,d)\). \(f\) is
strictly increasing. if \(y\in(c,d)\subseteq f(I)\), so \(\exists x\in I\), \(y=f(x)\), therefore \(f\) is
surjective. Also \(f\) is order-preserving, thus \(f\) is continuous on \((a,b)\) (since we are
using order to define the topology). But \(f\) is continuous at nowhere, so a contradiction
\end{proof}

Then the monotonicity theorem follows from the proof of Lemma \ref{w3.6} (modify the boundary to
include the discontinuous points)

\begin{corollary}[]
If \(f:(a,b)\to R\) definable, \(\lim_{x\to a^+}f(x)\) exists in \(R_{\infty}\)
\end{corollary}

\begin{proof}
\begin{enumerate}
\item Take \(\epsilon\), \(f|_{a,a+\epsilon}\) is continuous and monotone. Then \(\lim_{x\to a^+}f(x)\)
is \(\sup\{f(x):x\in(a,a+\epsilon)\}\) or \(\inf\{f(x):x\in(a,a+\epsilon)\}\)
\end{enumerate}
\end{proof}

\begin{corollary}[]
If \(f:[a,b]\to R\) is definable and continuous, then \(\max_{x\in[a,b]f(x)}\)
and \(\min_{x\in[a,b]}f(x)\) exist
\end{corollary}

\begin{proof}
Take maximum for each piece and combine
\end{proof}

\textbf{Uniform Finiteness}

Suppose \(D\subseteq R^n\times R\), for \(\bara\in R^n\), \(D_{\bara}=\{y\in R:(a,y\}\in D\)

\begin{theorem}[Uniform Finitness]
Suppose \(\forall\bara\), \(\abs{D_{\bara}<\infty}\). Then \(\exists N<\infty\forall\bara\abs{D_{\bara}}<N\)
\end{theorem}

For now, consider \(n=1\).

Fix \(D\subseteq R^2\) definable, \(\abs{D_a}<\infty\) for all \(a\in R\)

\begin{definition}[]
\((a,b)\subseteq R\times R_{\infty}\) is \textbf{normal} if either
\begin{itemize}
\item \((a,b)\notin\cl(D)\), \((\exists \epsilon)(a-\epsilon,a+\epsilon)\times(b-\epsilon,b+\epsilon)\cap D=\emptyset\)
\item \((a,b)\in D\) and \((\exists\epsilon,\delta)D\cap(a-\epsilon,a+\epsilon)\times(b-\delta,b+\delta)\) is \(\Gamma(f)\) for some continuous
function \(f\)
\end{itemize}

Otherwise \((a,b)\) is abnormal
\end{definition}

\begin{remark}
\(\{(x,y)\text{ normal}\}\) is open, \(\{(x,y)\text{ abnormal}\}\) is closed.
\end{remark}

\begin{definition}[]
\(a\in R\) is \textbf{good} if \(\forall b\in R_\infty\), \((a,b)\) is normal, is \textbf{bad} if \(\exists b\in R_\infty\), \((a,b)\) is abnormal
\end{definition}

This is a definable definition

\begin{lemma}[]
\(\{x\in R:x\text{ is bad}\}\) is finite
\end{lemma}

\begin{proof}
Otherwise, take \(I\subseteq B\), \(\forall x\in I\), \(\{y\in R_\infty:(x,y)\text{ abnormal}\}\) is closed, nonempty.

Let \(f(x)=\min\{y\in R_\infty:(x,y)\text{ abnormal}\}\), \(f:I\to R_\infty\) definable.

\(\forall x\), break into cases based on these questions
\begin{itemize}
\item \(f(x)=-\infty\) vs \(f(x)\in R\) vs \(f(x)=+\infty\)
\item \((x,f(x))\in D\) vs not
\item whether \(\exists y>f(x)\), \((x,y)\in D\)
\item whether \(\exists y<f(x)\), \((x,y)\in D\)
\end{itemize}

So 24 pieces

Shrink \(I\) to make all the answers constant

Assume \(\forall x\in I\), \(f(x)\in R\), \((x,f(x))\in D\), \((\exists y<f(x))(x,y)\in D\), \((\exists z>f(x))(x,z)\in D\)

Let \(g(x)=\max\{y:y<f(x), (x,y)\in D\}\), \(h(x)=\min\{y:y>f(x),(x,y)\in D\}\)

\(D_x\) is finite and we can take the min and max

For each \(x\in I\), \((x,f(x))\in D\), \((x,f(x))\) is abnormal, if \(f(x)<y<h(x)\),
then \((x,y)\notin D\)

Idea: apply monotonicity theorem, get \(f,g,h\) continuous, then \((x,f(x))\) is normal

Use monotonicity theorem to get \(J\subseteq I\), \(f|_J,g|_J,h|_J\) are continuous

Take \(a\in J\), \((a,f(a))\in D\), \((a,f(a))\) is normal. Take \(\epsilon\)
s.t. \(g(a)+\epsilon<f(a)-\epsilon\), \(f(a)+\epsilon<h(a)-\epsilon\). Take \(\delta\) s.t. if \(\abs{x-a}<\delta\)
then \(\abs{f(x)-f(a)}<\epsilon\), same for \(g,h\)

If \(x\in(a-\epsilon,a+\epsilon)\), then \((x,f(x))\in D\), \((a-\delta,a+\delta)\times(f(a)-\epsilon,f(a)+\epsilon)\cap D\)
is \(\Gamma(f|_{(a-\delta,a+\delta)})\)

if \((x,y)\in D\) and \(x\in(a-\delta,a+\delta)\), \(y\in(f(a)-\epsilon,f(a)+\epsilon)\) then \(y=f(x)\) or else \(y\neq f(x)\),
then \(y\in D_x\), so either \(y\ge h(x)\) or \(y\le g(x)\). But \(\abs{x-a}<\delta\), \(\abs{h(x)-h(a)}<\epsilon\)
\end{proof}

\begin{lemma}[]
If \(I\subseteq R\) and \(\forall x\in I\), \(x\) is good, then \(\exists n<\infty\), \(\forall a\in I\), \(\abs{D_a}=n\)
\end{lemma}

\begin{proof}
Take \(a_0\in I\), let \(n=\abs{D_{a_0}}<\infty\).

Let \(S=\{a\in I:\abs{D_a}=n\}\).

Goal: \(S\) and \(I\setminus S\) are open.

This is sufficient because o-minimality shows intervals are definably connected. \(I\neq X\cup Y\)
where \(X,Y\) are disjoint, nonempty, definable and open. So if \(I=X\cup Y\), \(X\cap Y=\emptyset\), \(X,Y\)
definable open, then one of them is empty

Let \(S_n=\{a\in I:\abs{D_a}\}=n\), \(S=S_n\)

Goal: each \(S_n\) is open. Then \(I\setminus S_n\) and \(S_n\) are open

Idea: \(I\to\N\), \(a\mapsto\abs{D_a}\) is locally constant

Fix \(a\in I\), let \(n=\abs{D_a}\), \(D_a=\{b_1,\dots,b_n\}\), \(b_1<\dots<b_n\). Since \(a\) is good,
so \((a,b_i)\) is normal. Take \(\epsilon\), \(\delta\) small enough, then \((a-\delta,a+\delta)\times(b_i-\epsilon,b_i+\epsilon)\cap D\) is a graph of
a continuous function \(f_i:(a-\delta,a+\delta)\to R\). Take \(\epsilon\) small enough s.t. \(b_i+\epsilon<b_{i+1}-\epsilon\).
If \(x\in(a-\delta,a+\delta)\), then \((f_1(a),\dots,f_n(a))\subseteq D_x\), then \(\abs{D_x}\ge n\).

Suppose \(\abs{D_x}\ge n\), if \((a-\delta,a+\delta)\subseteq S_n\), then we are done. Otherwise, by
o-minimality, \((a,a+\delta')\cap S_n=\empty\) or \((a-\delta',a)\cap S_n=\emptyset\).

Shrinking \(\delta',\delta\) we can assume \(\delta=\delta'\). WLOG, \(a<x<a+\delta\), \(x\notin S_n\).

Let \(g(x)=\min D_x\setminus\{f_1(x),\dots,f_n(x)\}\). By monotonicity theorem, \(\lim_{x\to a^+}g(x)=:c\in R_\infty\)

\((a,c)\) is normal because \(a\) is good.

\((a,c)=\lim_{x\to a^+}(x,g(\abs{x}))\) , so \((a,c)\in\cl(D)\). Because \((a,c)\) is
normal, \((a,c)\in D\) and \(D\) looks like \(\Gamma(h)\) around \((a,c)\).
\end{proof}

\begin{proof}[Proof of uniform boundedness]
Let \(b_1,\dots,b_n\) be the bad points.
\begin{equation*}
(-\infty,b_1),(b_1,b_2),\dots,(b_{n-1},b_n),(b_n,+\infty)
\end{equation*}
and \(\abs{D_a}\) is constant on each intervals
\end{proof}





\subsection{The cell decomposition theorem}
\label{sec:org87b961d}
for each definable set \(X\) in \(R^m\) we put
\begin{align*}
C(X)&:=\{f:X\to R:f\text{ definable and continuous}\}\\
C_\infty(X)&:=C(X)\cup\{-\infty,+\infty\}
\end{align*}
where we regard \(-\infty\) and \(+\infty\) as constant functions on \(X\)

For \(f,g\in C_\infty(X)\) we write \(f<g\) if \(f(x)<g(x)\) for all \(x\in X\), and in this case we put
\begin{equation*}
(f,g)_X:=\{(x,r)\in X\times R:f(x)<r<g(x)\}
\end{equation*}
So \((f,g)_X\) is a definable subset of \(R^{m+1}\)

\begin{definition}[]
Let \((i_1,\dots,i_m)\) be a sequence of zeros and ones of length \(m\). An \textbf{\((i_1,\dots,i_m)\)-cell} is a
definable subset of \(R^m\) obtained by induction on \(m\) as follows:
\begin{enumerate}
\item a \((0)\)-cell is a one-element set \(\{r\}\subseteq R\), a \((1)\)-cell is an
interval \((a,b)\subseteq R\)
\item suppose \((i_1,\dots,i_m)\)-cells are already defined, then an \((i_1,\dots,i_m,0)\)-cell is the
graph \(\Gamma(f)\) of a function \(f\in C(X)\), where \(X\) is an \((i_1,\dots,i_m)\)-cell; further,
an \((i_1,\dots,i_m,1)\)-cell is a set \((f,g)_X\) where \(X\) is an \((i_1,\dots,i_m)\)-cell
and \(f,g\in C_\infty(X)\), \(f<g\)
\end{enumerate}
\end{definition}

So a \((0,0)\)-cell is a ``point'' \(\{(r,s)\}\subseteq R^2\), a \((0,1)\)-cell is an ``interval'' on a
vertical line \(\{a\}\times R\), and a \((1,0)\)-cell is the graph of a continuous definable function
defined on an interval.

\begin{definition}[]
A \textbf{cell in} \(R^m\) is an \((i_1,\dots,i_m)\)-cell for some (necessarily unique)
sequence \((i_1,\dots,i_m)\). Since the \((1,\dots,1)\)-cells are exactly the cells which are open in
their ambient space \(R^m\), we call these \textbf{open cells}
\end{definition}

The non-open cells are ``thin'':

The union of finitely many non-open cells in \(R^m\) has empty interior

\begin{proposition}[]
Each cell is locally closed, i.e., open in its closure
\end{proposition}

\begin{proof}
Let \(C\subseteq R^{m+1}\) be a cell. Put \(B:=\pi(C)\subseteq R^m\) and assume inductively that the cell \(B\) is
open in its closure \(\cl(B)\), so that \(\cl(B)-B\) is a closed set. If \(C=\Gamma(f)\)
with \(f:B\to R\) a definable continuous function, then \(\cl(C)-C\) is contained
in \((\cl(B)-B)\times R\), hence \(C\) is open in the closed set \(C\cup((\cl(B)- B)\times R)\)  \label{Problem1}

If \(C=(f,g)\) with \(f,g:B\to R\) definable continuous functions on \(B\), \(f<g\), then one
verifies that \(\cl(C)-C\subseteq\Gamma(f)\cup\Gamma(g)\cup((\cl(B)-B)\times R)\) and that \(C\) is open in the closed set
\(C\cup\Gamma(f)\cup\Gamma(g)\cup((\cl(B)-B)\times R)\)
\end{proof}

we consider the point-space \(R^0\) as a cell, or ()-cell, where () is the sequence of length 0

Each cell is homeomorphic under a coordinate projection to an open cell. We now make this
explicit. Let \(i=(i_1,\dots,i_m)\) be a sequence of zeros and ones

Define \(p_i:R^m\to R^k\) as follows: let \(\lambda(1)<\dots<\lambda(k)\) be the indices \(\lambda\in\{1,\dots,m\}\) for
which \(i_\lambda=1\), so that \(k=i_1+\dots+i_m\); then
\begin{equation*}
p_i(x_1,\dots,x_m):=(x_{\lambda(1),\dots,x_{\lambda(k)}})
\end{equation*}
It is easy to show by induction on \(m\) that \(p_i\) maps each \(i\)-cell \(A\) homeomorphically
onto an open cell \(p_i(A)\) in \(R^k\). We denote \(p_i(A)\) also by \(p(A)\) and the
homeomorphism \(p_i|A:A\to p(A)\) by \(p_A\). Clearly \(p_A=\id_A\) if \(A\) is an open cell

If \(A\) is a cell in \(R^{m+1}\) then \(\pi(A)\) is a cell in \(R^m\), where \(\pi:R^{m+1}\to R^m\) is
the projection on the first \(m\) coordinates. Here is a simple application of this fact

\begin{proposition}[]
Each cell is definably connected
\end{proposition}

\begin{proof}
For intervals and points this is stated in \ref{1.3.6}

If \(A\) is a cell in \(R^{m+1}\), then we assume inductively that the cell \(\pi(A)\) in \(R^m\)
is definably connected and use the fact that each fiber \(\pi^{-1}(x)\cap A\) is definably connected
\end{proof}

\begin{theorem}[Cell decomposition]
If \(X\subseteq R^m\) is definable, then \(\exists C_1,\dots,C_n\) cells, \(X=\bigcup_{i=1}^nC_i\), \(C_i\cap C_j=\emptyset\)
for \(i\neq j\).
\end{theorem}

\begin{examplle}[]
In \((\R,+,0,\le,0,1)\), \(X=\{(x,y):x^2+y^2\le 1\}\) is a \((1,1)\)-cell
\end{examplle}

\begin{theorem}[]
For any \(m\in\N\),
\begin{enumerate}
\item (\(Cell_m\)): any definable \(A\subseteq R^m\) has a cell decomposition. \(A=\bigcup_{i=1}^nC_i\)
\item (\(Con_m\)): if \(f:A\to R\) definable, then \(\exists\) cell decomposition \(A=\bigcup_{i=1}^nC_i\)
s.t. \(f|_{C_i}\) is continuous for all \(i\)
\item (\(Fin_m\)): if \(A\subseteq R^m\times R\) definable, and if \(A_{\barx}=\{y\in R:(\barx,y)\in A\}\) is finite
\(\forall\barx\in R^m\), then \(\exists N\in\N\), \(\forall\barx\in R^m\), \(\abs{A_{\barx}}\le N\).
\end{enumerate}
\end{theorem}

Proof stragegy: Cell\textsubscript{1}, Con\textsubscript{1}, Fin\textsubscript{1},
Cell\textsubscript{2}, Con\textsubscript{2}, Fin\textsubscript{2}, and so on.

(Cell\textsubscript{1}) is the definition of o-minimality.

(Con\textsubscript{1}) is the monotonicity theorem.

(Fin\textsubscript{1}) is the uniform finiteness part 1.

Suppose \(m>1\), take \(m=2\) for simplicity.

If \(D\subseteq R\) definable, and \(\tbd(D)=\{x_1,\dots,x_n\}\), \(x_1<\dots<x_n\), then \(D\) is the union of some
of
\begin{equation*}
c_0:=(-\infty,x_1),c_1:=\{x_1\},c_2:=(x_1,x_2),\dots,c_{2n-1}:=\{x_n\},c_{2n}:=(x_n,+\infty)
\end{equation*}
The ``shape'' of \(D\) is the string \((j_0,\dots,j_{2n})\) where \(j_i=1\) if \(C_i\subseteq D\) and 0
if \(C_i\cap D=\emptyset\)

(Cell\textsubscript{2}): Fix definable \(A\subseteq R^2\), \(A_x=\{y\in R:(x,y)\in A\}\), \(\tbd(A_x)\) is finite
for all \(x\in R\). By (Fin\textsubscript{1}), \(\exists N\in\N\) s.t. \(\forall x\), \(\abs{\tbd(A_x)}\le N\),
then \(\{shape(A_x):x\in R\}\) is finite.
\(\{x\in R:shape(A_x)=0011001\}\) is definable.

By (Cell\textsubscript{1}), can partition \(R\) into cells \(C_1,\dots,C_n\) s.t. \(shape(A_x)\) is a
constant for \(x\in C_i\).

Let \(f_i(x)\) be the \(i\)th smallest element of \(\tbd(A_x)\), then \(f\) is definable. We can
further partition and WMA \(f_1,\dots,f_N\) is continuous on \(C_1,\dots,C_n\) using
(Con\textsubscript{1}).

(Con\textsubscript{2}):
\begin{lemma}[]
If \(B\subseteq R^2\) is a box and \(f:B\to R\) is continuous in each variable separately and monotone in
each variable separately, then \(f\) is continuous.
\end{lemma}

\begin{proof}
WLOG, \(f\) is increasing in each variable.

Fix \((x,y)\in B\), \(\epsilon>0\), \(\exists\delta_1>0\) s.t. \(f(x+\delta_1,y)<f(x,y)+\epsilon\). \(\exists\delta_2>0\)
s.t. \(f(x,y+\delta_2)<f(x,y)+\epsilon\).

\(\exists\delta_3,\delta_4>0\), \(f(x-\delta_3,y-\delta_4)>f(x,y)-\epsilon\).

If \(x'\in(x-\delta_3,x+\delta_4)\) and \(y'\in(y-\delta_4,y+\delta_2)\), then
\begin{equation*}
f(x,y)-\epsilon<\dots<f(x',y')<f(x+\delta_1,y+\delta_2)<f(x,y)+\epsilon
\end{equation*}
\end{proof}

For \(x\in A\) ask questions:
\begin{itemize}
\item is \(f\) continuous at \(x\)
\item is \(f\) continuous in first variable at \(x\)
\item is \(f\) continuous in second variable at \(x\)
\item is \(f\) increasing in first variable
\item is \(f\) decreasing in first variable
\item increasing in second
\item decreasing in second
\end{itemize}
By (Cell\textsubscript{2}), \(A=\bigcup_{i=1}^nC_i\), answers are constant on each \(C_i\). Fix \(C=C_i\)
\begin{itemize}
\item if \(f\) is continuous \(\forall x\in C\), then \(f|_C\) is continuous
\item if \(C\) is not a \((1,1)\)-cell, then there is a coordinate projection \(\pi:R^2\to R\)
s.t. \(C\to\pi(C)\) is a bijection, \(\pi(C)\) is a cell. \(C\to\pi(C)\) is a homeomorphism, subcells
of \(C\) corresponds to subcells of \(\pi(C)\)

(Con\textsubscript{1}) on \(\pi(C)\) implies (Con\textsubscript{2}) on \(C\) by homeomorphism.
\item \(C\) is a \((1,1)\)-cell, \(f\) is continuous nowhere on \(C\). Take \((a,b)\in C\), look
at \(f(x,b)\) for \(x\in(a-\epsilon,a+\epsilon)\). Apply monotonicity theorem, get \(a'\) s.t. \(f\)
at \((a',b)\) is continuous and monotone in \(x\), then \(f\) is continuous and monotone in
1st coordinate everywhere in \(C\). Similary, \(f\) is cts \& monotone in 2nd coordinate. By
the lemma, \(f\) is continuous,
\end{itemize}

Fix \(A\subseteq R^2\times R\), \(\abs{A_{(x,y)}<\infty}\) for any \(x,y\in R\)

\begin{definition}[]
\((x,y,z)\in R^2\times R_{\infty}\) is \textbf{normal} if
\begin{itemize}
\item locally, \(A\) is \(\emptyset\), \((x,y,z)\notin\cl(A)\), or
\item locally, \(A\) is \(\Gamma(f)\), \(f\) is continuous
\end{itemize}

abnormal otherwise

\((x,y)\) is \textbf{good} if \(\forall z\in R_{\infty}\), \((x,y,z)\) is normal. Bad otherwise
\end{definition}

\begin{lemma}[]
If \(B\subseteq R^2\) is a box, then there is \((x,y)\in B\) s.t. \((x,y)\) is good.
\end{lemma}

\begin{proof}
Similar, use Con\textsubscript{2}
\end{proof}

\begin{lemma}[]
\label{ww2}
If \(B\subseteq R^2\) is a box and \(\forall(x,y)\in B\), \((x,y)\) is good, then \(\abs{A_{(x,y)}}\) is constant
for \((x,y)\in B\)
\end{lemma}

\begin{proof}
Suppose \(B=I\times J\), fix \(a_0\in I\),
let \(A'=\{(y,z):(a_0,y,z)\in A\}\), \(A'_y=A_{(a_0,y)}\). \(A'_y\) is finite for each \(y\).

Check \(A'\) is good on \(J\). Every \(y\in J\) is good w.r.t. \(A'\).

By the \(m=1\) version of lemma, \(\abs{A'_y}\) is constant for \(y\in J\)

\(f(x,y)=\abs{A_{(x,y)}}\) doesn't depend on 2nd coordinate.

Similary, \(f(x,y)\) is constant on 1st coordinate

So \(f\) is constant on \(B\).
\end{proof}

\begin{lemma}[]
If \(C\) is an open cell, \((1,1)\)-cell, and \(\forall(x,y)\in C\) is good, then \(\abs{A_{(x,y)}}\) is
constant for \((x,y)\in C\)
\end{lemma}

\begin{proof}
Take \((x_0,y_0)\in C\), let \(n=\abs{A_{(x_0,y_0)}}\), let \(D=\{(x,y)\in C:\abs{A_{(x,y)}=n}\}\).
Using \ref{ww2}, can show that \(D\) is open, \(C\setminus D\) open

But \(C\) is definably connected

\begin{equation*}
\chi_D(x)
\begin{cases}
1&x\in D\\
0&x\notin D
\end{cases}
\end{equation*}
want \(\chi_D\) constant in the cell.
\begin{enumerate}
\item \(\chi_D(x,y)\) doesn't depend on \(y\), so \(\chi_D(x,y)=f(x)\)
\item \(f(x)\) doesn't depend on \(x\)
\end{enumerate}
\end{proof}

Proof of Fin\textsubscript{2}: Use (Cell\textsubscript{2}) to split \(R^2\)
into \(R^2=\bigcup_{i=1}^nC_i\), where for \(i\), (normal is definable)
\begin{enumerate}
\item \(\forall(x,y)\in C_i\), \((x,y)\) is good
\item \(C_i\) is bad.
\end{enumerate}
We only need uniform finiteness above each \(C_i\)

Fix \(C_i\)

Case 1: \(C_i\) is a \((1,1)\)-cell, all the points in \(C_i\) is good.

Case 2: \(C_i\) is not a \((1,1)\)-cell. Take a projection \(\pi\) s.t. \(\pi(C_i)\) is an cell.
Uniform finiteness holds over \(\pi(C_i)\), then uniform finiteness holds over \(C_i\), transfering
the definable bijection \(\pi\)

\(R\) is o-minimal.
\begin{lemma}[]
If \(\varphi(\barx,y)\) is a formula and \(\abs{\varphi(\bara,M)}<\infty\) for all \(\bara\in R^m\), then \(\exists N<\infty\)
s.t. \(\forall\bara\in R^m\), \(\abs{\varphi(\bara,M)}<N\)
\end{lemma}

\begin{remark}
If \(\varphi(\barx,y)\) a formula and we let \(\varphi'(\barx,y)\) be
\begin{equation*}
\forall z,w[z<y<w\to(\exists s:z<s<w\wedge\varphi(\barx,s))\wedge(\exists s:z<s<w\wedge\neg\varphi(\barx,s))]
\end{equation*}
Then \(\varphi'(\bara,R)\) is \(\tbd(\varphi(\bara,R))\)
\end{remark}

\begin{theorem}[]
If \(R\equiv R'\) and \(R\) is o-minimal, then \(R'\) is o-minimal
\end{theorem}

\begin{proof}
If \(D\subseteq R'\) definable, (want \(D\) be the union of finite intervals)

\(D=\varphi(\bara,R')\) for some \(\bara\in R'\). Let \(\varphi'(\barx,y)\) be the formula from the remark.

\(\varphi'(\barb,R')=\tbd(\varphi(\barb,R'))\) for \(\barb\in(R')^m\), so
\(\varphi'(\barb,R)=\tbd(\varphi(\barb,R))\) for all \(\barb\in R^m\).

O-minimality \(\Rightarrow\) \(\abs{\varphi'(\barb,R)}<\infty\) for all \(\barb\in R^m\). Uniform finiteness gives \(N\)
s.t. \(\forall \barb\in R^m\), \(\abs{\varphi'(\barb,R)}<N\), therefore \(\varphi'(\barb,R')\) is finite for
all \(\barb\in(R')^m\). Take \(\barb=\bara\), therefore \(\tbd(D)\) is finite.

\textbf{Claim}: In \(R'\), If \(y<z\) and \([y,z]\cap\tbd(D)\), then \(y\in D\Leftrightarrow z\in D\)

\emph{Proof}: True in \(R\) by o-minimality.
\begin{equation*}
R\vDash\forall\barx,y,z(y<z\wedge\neg\exists w(y\le w\le z\wedge\varphi'(\barx,w))\to[\varphi(\barx,y)\leftrightarrow\varphi(\barx,z)])
\end{equation*}

If \(\tbd(D)=\{c_1,\dots,c_m\}\), \(c_1<\dots<c_m\), the claim shows that \(D\) respects the partition
of \(R'\).

So \(D\) is a union of some points and intervals
\end{proof}

\begin{definition}[]
\(T\) is o-minimal is every model of \(T\) is o-minimal
\end{definition}

\begin{examplle}[]
\(\DLO\), \(\RCF\), \(\ODAG\)
\end{examplle}

\begin{theorem}[]
If \(\calm\) is o-minimal, then \(\Th(\calm)\) is o-minimal
\end{theorem}

\begin{definition}[]
\(\calm\) is minimal if \(\forall\) definable \(D\subseteq M\), \(D\) is either finite or cofinite.
\end{definition}

\begin{definition}[]
\(\calm\) is strongly minimal if
\begin{enumerate}
\item \(\calm\vDash T\) where \(T\) is strongly minimal, or
\item \(\forall\caln\equiv\calm\), \(\caln\) is minimal.
\end{enumerate}
\end{definition}

\begin{fact}[]
\((\C,+,\cdot)\) is strongly minimal, \((\N,\le)\) is minimal but not strongly minimal
\end{fact}

Idea: strong o-minimality = o-minimality


\begin{definition}[]
A \textbf{decomposition} of \(R^m\) is a special kind of partition of \(R^m\) into finitely many cells. The
definition is by induction on \(m\)
\begin{enumerate}
\item a decomposition of \(R^1=R\) is a collection
\begin{equation*}
\{(-\infty,a_1),(a_1,a_2),\dots,(a_k,+\infty),\{a_1\},\dots,\{a_k\}\}
\end{equation*}
where \(a_1<\dots<a_k\) are points
\item a decomposition of \(R^{m+1}\) is a finite partition of \(R^{m+1}\) into cells \(A\) s.t. the
set of projections \(\pi(A)\) is a decomposition of \(R^m\)
\end{enumerate}
\end{definition}

Let \(\cald=\{A(1),\dots,A(k)\}\) be a decomposition of \(R^m\), \(A(i)\neq A(j)\) if \(i\neq j\), and let for
each \(i\in\{1,\dots,k\}\) functions \(f_{i1}<\dots<f_{in(i)}\) in \(C(A_i)\) be given

Then
\begin{equation*}
\cald_i:=\{(-\infty,f_{i1}),(f_{i1},f_{i2}),\dots,(f_{in(i)},+\infty),\Gamma(f_{i1}),\dots,\Gamma(f_{in(i)})\}
\end{equation*}
is a partition of \(A(i)\times R\) and one easily checks that \(\cald^*:=\cald_1\cup\dots\cup\cald_k\) is a decomposition
of \(R^{m+1}\), and that every decomposition of \(R^{m+1}\) arises in this way from a
decomposition \(\cald\) of \(R^m\). We write \(\cald=\pi(\cald^*)\)

A decomposition \(\cald\) of \(R^m\) is said to be \textbf{partition} a set \(S\subseteq R^m\) if each cell in \(\cald\) is
either part of \(S\) or disjoint from \(S\), in other words, if \(S\) is a union of cells
in \(\cald\).

\begin{theorem}[Cell Decomposition Theorem]
\begin{enumerate}
\item \((\text{\rom{1}}_m)\)Given any definable sets \(A_1,\dots,A_k\subseteq R^m\) there is a decomposition of \(R^m\) partitioning
each of \(A_1,\dots,A_k\)
\item (\(\text{\rom{2}}_m\))For each definable function \(f:A\to R\), \(A\subseteq R^m\), there is a decomposition \(\cald\) of \(R^m\)
partitioning \(A\) s.t. the restriction \(f|B:B\to R\) to each cell \(B\in\cald\) with \(B\subseteq A\) is continuous
\end{enumerate}
\end{theorem}

\((\text{\rom{1}}_1)\) holds by o-minimality, and that \((\text{\rom{2}}_1)\) follows from the monotonicity
theorem

We now assume that \(\text{\rom{1}}_1,\dots,\text{\rom{1}}_m\)
and \(\text{\rom{2}}_1,\dots,\text{\rom{2}}_m\) hold

The proof is lengthy. The first step is to generalize the finiteness lemma of the previous
section. Call a set \(Y\subseteq R^{m+1}\) \textbf{finite over} \(R^m\) if for each \(x\in R^m\) the
fiber \(Y_x:=\{r\in R:(x,r\}\in Y\) is finite; call \(Y\) \textbf{uniformly finite over} \(R^m\) if there
is \(N\in\N\) s.t. \(\abs{Y_x}\le N\) for all \(x\in R^m\)

\begin{lemma}[Uniform Finitness Property]
Suppose the definable subset \(Y\) of \(R^{m+1}\) is finite over \(R^m\), then \(Y\) is uniformly
finite over \(R^m\)
\end{lemma}

\begin{proof}

\end{proof}

\begin{lemma}[]
Let \(X\) be a topological space, \((R_1,<)\), \((R_2,<)\) dense linear orderings without
endpoints and \(f:X\times R_1\to R_2\) a function s.t. for each \((x,r)\in X\times R_1\)
\begin{enumerate}
\item \(f(x,\cdot):R_1\to R_2\) is continuous
\item \(f(\cdot,r):X\to R_2\) is continuous
\end{enumerate}
Then \(f\) is continuous
\end{lemma}

\begin{proof}
Let \((x,r)\in X\times R_1\) and \(f(x,r)\in J\), where \(J\) is an interval in \(R_2\). We shall find a
neighborhood \(U\) of \(x\) and an interval \(I\) around \(r\) s.t. \(f(U\times I)\subseteq J\). By (1) there
are \(r_-,r_+\) in \(R_1\) s.t. \(r_-<r<r_+\) and \(f(x,r_-),f(x,r_+)\in J\). Now use (2) to get a
neighborhood \(U\) of \(x\) s.t. \(f(U\times\{r_-\})\subseteq J\) and \(f(U\times\{r_+\})\subseteq J\). We claim that
then \(f(U\times I)\subseteq J\) for \(I=(r_-,r_+)\)

Let \(x'\in U\) and \(r_-<r'<r_+\). Assume \(f(x',\cdot)\) is increasing,
then \(f(x',r_-)\le f(x',r')\le f(x',r_+)\) and \(f(x',r_-)\), \(f(x',r_+)\) are both in \(J\),
hence \(f(x',r')\) is in \(J\)
\end{proof}

A \textbf{definably connected component} of a nonempty definable set \(X\subseteq R^m\) is by definition a maximal
definably connected subset of \(X\)

\begin{proposition}[]
Let \(X\subseteq R^m\) be a nonempty definable set. Then \(X\) has only finitely many definably connected
components. They are open and closed in \(X\) and form a finite partition of \(X\)
\end{proposition}

\begin{proof}
Let \(\{C_1,\dots,C_k\}\) be a partition of \(X\) into \(k\) disjoint cells. For each nonempty set of
indices \(I\subseteq\{1,\dots,k\}\), put \(C_I:=\bigcup_{i\in I}C_i\). Among the \(2^k-1\) sets \(C_I\), let \(C'\) be
maximal w.r.t. being definably connected.

Claim: If a set \(Y\subseteq X\) is definably connected and \(C'\cap Y\neq\emptyset\), then \(Y\subseteq C'\)

Put \(C_Y:=\bigcup\{C_i:C_i\cap Y\neq\emptyset\}\). Since the \(C_i\)'s cover \(X\) we have \(Y\subseteq C_Y\), so \(C_Y\)
is the union of \(Y\) with certain cells that intersect \(Y\). Hence \(C_Y\) is definably
connected \label{Problem2}. By maximality of \(C'\) it follows that \(C'\cup C_Y=C'\).
Hence \(Y\subseteq C_Y\subseteq C'\), which proves the claim.

It follows in particular that \(C'\) is a definably connected component of \(X\). Further the
claim shows that the sets \(C'\) are the only definable connected components of \(X\). Note that
because the closure in \(X\) of a definably connected subset of \(X\) is also definably
connected, the definably connected components of \(X\) are closed in \(X\). Hence they are open
in \(X\)
\end{proof}

\subsection{Definable families}
\label{sec:orge160a7d}
Let \(S\subseteq R^{m+n}=R^m\times R^n\) be definable. For each \(a\in R^m\) we put
\begin{equation*}
S_a:=\{x\in R^n:(a,x)\in S\}\subset R^n
\end{equation*}
We view \(S\) as describing the family of sets \((S_a)_{a\in R^m}\). Such a family is called a
\textbf{definable family} (of subsets of \(R^n\), with parameter space \(R^m\)). The sets \(S_a\) are also
called the \textbf{fibers} of the family

\begin{examplle}[]
Let \(\calr:=(\R,<,+,\cdot)\) and consider the formula
\begin{equation*}
ax^2+bxy+cy^2+dx+ey+f=0
\end{equation*}
This defines a relation \(S\subseteq\R^6\times\R^2\). For each point \((a,b,c,d,e,f)\in\R^6\) the
subset \(S_{(a,b,c,d,e,f)}\in\R^2\) consists of the points \((x,y)\) satisfying the equation
\end{examplle}

In the following \(\pi:R^{m+n}\to R^m\) denotes the projection on the first \(m\) coordinates

\begin{proposition}[]
\begin{enumerate}
\item Let \(C\) be a cell in \(R^{m+n}\) and \(a\in\pi(C)\). Then \(C_a\) is a cell in \(R^n\)
\item Let \(\cald\) be a decomposition of \(R^{m+n}\) and \(a\in R^m\). Then the collection
\begin{equation*}
\cald_a:=\{C_a:C\in\cald,a\in\pi(C)\}
\end{equation*}
is a decomposition of \(R^n\)
\end{enumerate}
\end{proposition}

\begin{proof}
For \(n=1\) this is immediate from the definitions

Suppose the proposition holds for a certain \(n\), and let \(C\) be a cell in \(R^{m+(n+1)}\).
Let \(\pi_1:R^{m+(n+1)}\to R^{m+n}\) be the obvious projection map, so that \(\pi\circ\pi_1:R^{m+(n+1)}\to R^m\)
is the projection on the first \(m\) coordinates

If \(C=\Gamma(f)\), then \(C_a=\Gamma(f_a)\), where \(f_a:(\pi_1C)\to R\) is defined by \(f_a(x)=f(a,x)\)

If \(C=(f,g)_D\) with \(D=\pi_1C\), then \(C_a=(f_a,g_a)_E\) where \(E=D_a\)

In both cases \(C_a\) is a cell in \(R^{n+1}\)
\end{proof}

\begin{corollary}[]
Let \(S\subseteq R^m\times R^n\) be definable. Then there is a number \(M_S\in\N\) s.t. for each \(a\in R^m\) the
set \(S_a\subseteq R^n\) has a partition into at most \(M_S\) cells. In particular, each fiber \(S_a\) has
at most \(M_S\) definably connected components
\end{corollary}

\begin{proof}
Take a decomposition \(\cald\) of \(R^{m+n}\) partitioning \(S\). Then for each \(a\in R^m\) the
decomposition \(\cald_a=\{C_a:C\in\cald,a\in\pi C\}\) of \(R^m\) consists of at most \(\abs{\cald}\) cells and
partitions \(S_a\). So we can take \(M_S=\abs{\cald}\)
\end{proof}

\begin{corollary}[]
Let \(S\subseteq R^m\times R^n\) be definable. Then there is a natural number \(M_S\) s.t. for each \(a\in R^m\)
the set \(S_a\subseteq R^n\) has at most \(M_S\) isolated points. In particular, each finite fiber \(S_a\)
has cardinality at most \(M_S\)
\end{corollary}

\section{Definable invariants: dimension and euler characteristic}
\label{sec:org746a89b}

\subsection{Dimension}
\label{sec:org8cfd4dd}
We define the \textbf{dimension} of a nonempty definable set \(X\subseteq R^m\) by
\begin{equation*}
\dim(X):=\max\{i_1+\dots+i_m:X\text{ contains an }(i_1,\dots,i_m)\text{-cell}\}
\end{equation*}
To the empty set we assign the dimension \(-\infty\)

Goal:
\begin{theorem}[]
\begin{enumerate}
\item well-defined
\item if \(f:A\to B\) is a definable bijection
\item \(\dim(A)>0\Leftrightarrow\abs{A}=\infty\)
\item \(\dim(A\times B)=\dim(A)+\dim(B)\)
\item \(\dim(R^n)=n\)
\item If \(f:A\to B\) is a definable surjection, if \(\dim(f^{-1}(b))=k\) for all \(b\in B\),
then \(\dim(A)=k+\dim(B)\)
\item If \(\varphi(\barx,\bary)\) is a formula, then
\begin{equation*}
S(k)=\{\barb\in R^m:\dim(\varphi(R^n,\barb))=k\}
\end{equation*}
is definable.
\item \(\dim(\cl(A))=\dim(A)\), \(\partial A:=\cl(A)\setminus A\), \(\dim(\partial A)<\dim(A)\)
\end{enumerate}
\end{theorem}

\begin{definition}[]
If \(A=\sqcup_{i=1}^nC_i\), \(\chi(A)=\sum_{i=1}^n(-1)^{\dim(C_i)}\)
\end{definition}

\begin{theorem}[]
\begin{itemize}
\item well-defined
\item \(\chi(A\times B)=\chi(A)\cdot\chi(B)\)
\item \(\chi(A\cup B)=\chi(A)\cdot\chi(B)-\chi(A\cap B)\)
\item If \(\abs{A}<\infty\), \(\chi(A)=\abs{A}\)
\item (7) in above and replace \(\dim\) with \(\chi\)
\item If \(f:A\to B\)
\end{itemize}
\end{theorem}

\begin{lemma}[]
If \(f:A_1\to A_2\) a definable injection, then \(m\le\dim(A_1)\Rightarrow m\le\dim(A_2)\)
\end{lemma}

\begin{theorem}[]
\begin{enumerate}
\item If \(f:A_1\to A_2\) is a definable injection, then \(\dim(A_1)\le\dim(A_2)\).
\item If \(f:A_1\to A_2\) is a definable bijection, then \(\dim(A_1)=\dim(A_2)\).
\item If \(A_1\subseteq A_2\), then \(\dim(A_1)\le\dim(A_2)\)
\end{enumerate}
\end{theorem}

\(D\subseteq R^n\) definable, \(a\in\tint(D)\Leftrightarrow\exists\text{Box } B\ni a, B\subseteq D\).

\(\tint(D)\neq\emptyset\Leftrightarrow\exists\text{Box }B, B\subseteq D\)

\begin{remark}
If \(C\subseteq R^n\) is an \((i_1,\dots,i_n)\)-cell, then
\begin{itemize}
\item if \(i_1=i_2=\dots=i_n=1\), then \(C\) is open
\item otherwise, \(\tint(C)=\emptyset\) and \(C\) is nowhere dense.
\end{itemize}
\end{remark}

\begin{definition}[]
\(D\) is \textbf{nowhere dense} if \(\forall\) box \(B\), \(\exists\) box \(B'\subseteq B\), \(B'\cap D=\emptyset\).
\end{definition}

\begin{lemma}[]
If \(D_1,D_2\) are nowhere dense, then \(D_1\cup D_2\) is nowhere dense.
\end{lemma}

\begin{proof}
Given a box \(B_1\), \(\exists B_2\subseteq B_1\), \(B_2\cap D_1=\emptyset\), \(\exists B_3\subseteq B_2\), \(B_3\cap D_2=\emptyset\), \(B_3\cap(D_1\cup D_2)=\emptyset\).
\end{proof}

If \(A\subseteq R^m\) definable, \(A=\bigcup_{i=1}^nC_i\), \(C_i\) are cells, then either
\begin{itemize}
\item some \(C_i\) is open, then \(\tint(A)\neq\emptyset\), or
\item all \(C_i\) are nowhere dense, so \(A\) is nowhere
\end{itemize}

\begin{corollary}[]
If \(D_1,\dots,D_n\subseteq R^m\) are definable and \(\tint(D_i)=\emptyset\), then \(\tint(\bigcup_{i=1}^mD_i)=\emptyset\).
\end{corollary}

\(\R=\Q\cup(\R\setminus\Q)\), \(\tint(\Q)=\tint(\R\setminus\Q)=\emptyset\), \(\tint(\R)=\R\)

\begin{proof}
\(\tint(D_i)=\emptyset\Rightarrow\) \(D_i\) is nowhere dense.
\end{proof}

\begin{theorem}[]
If \(D_1,D_2\subseteq R^n\) definable, then \(\dim(D_1\cup D_2)=\max(\dim(D_1),\dim(D_2))\)
\end{theorem}

\begin{proof}
\(\max(\dim(D_1),\dim(D_2))\le\dim(D_1\cup D_2)\).

\textbf{Claim}: If \(m\le\dim(D_1\cup D_2)\), then \(m\le\max(\dim(D_1),\dim(D_2))\)

Take \(B\subseteq R^m\), definable injection \(f:B\hookrightarrow D_1\cup D_2\). If \(x\in B\), \(f(x)\in D_1\) or \(f(x)\in D_2\).
So \(B=f^{-1}(D_1)\cup f^{-1}(D_2)\). \(\tint(B)=B\neq\emptyset\), so \(\tint(f^{-1}(D_1))\neq\emptyset\)
or \(\tint(f^{-1}(D_2))\neq\emptyset\).

\(\exists\) box \(B'\subseteq f^{-1}(D_1)\) and \(f|_{B'}:B'\hookrightarrow D_1\), so \(m\le\dim(D_i)\le\max(\dim(D_1),\dim(D_2))\)
\end{proof}

\begin{theorem}[]
If \(D\subseteq R^n\) definable and \(\tint(D)\neq\emptyset\)
\begin{enumerate}
\item (\(\rom{1}_n\)): \(\not\exists\) definable injection \(f:D\to R^{n-1}\)
\item (\(\rom{2}_n\)): If \(f:D\to R^n\) a definable injection, then \(\tint(f(D))\neq\emptyset\).
\end{enumerate}
\end{theorem}

\begin{proof}
\(n=1\), \(\rom{2}_1\), \(\abs{D}=\infty\), \(\abs{f(D)}=\infty\), then o-minimality says \(f(D)\) contains
an interval

\(\rom{1}_n\): By Cell decomposition, can shrink \(B\), get \(f|_B\) continuous. And at least
one of cell is open, take box \(B\) from it.

\(B=B_0\times(a,b)\), \(B_0\subseteq R^{n-1}\), take \(c\in(a,b)\), let \(g:B_0\to R^{n-1}\), \(g(x)=f(x,c)\).
By \(\rom{2}_{n-1}\), \(g(B_0)\) has interior, take \(p=g(x_0)\in\tint(g(B_0))\), \(f\) is
continuous, \(\exists c'\in(a,b)\)
s.t. \(c'\neq c\), \(f(x_0,c')\in\tint(g(B_0))\), \(f(x_0,c')=g(x_1)=f(x_1,c)\) for some \(x_1\in B_0\),
but \(f\) is injective.

\(\rom{2}_n\): \(f:D\to R^n\) injection, \(\tint(D)\neq\emptyset\). Assume \(\tint(f(D))=\emptyset\).

\(f(D)=\sqcup_{i=1}^nC_i\) cells, no \(C_i\) is open. \(D=\bigcup_{i=1}^nf^{-1}(C_i)\), \(\tint(D)\neq\emptyset\). There
is \(i\) s.t. \(\tint(f^{-1}(C_i))\neq\emptyset\), there is box \(B\subseteq f^{-1}(C_i)\), therefore there is
definable injection \(B\to C_i\to\pi(C_i)\subseteq R^{n-1}\), contradicts \(\rom{1}_n\)
\end{proof}

\begin{corollary}[]
\(\dim(R^n)=n\)

If \(D\subseteq R^n\), \(\tint(D)\neq\emptyset\), then \(\dim(D)=n\)

\(\tint(D)\neq\emptyset\Leftrightarrow\dim(D)=n\)
\end{corollary}

dimension theory rules out things like space-filling curves. No definable surjection
\begin{equation*}
[0,1]\to[0,1]^2
\end{equation*}
No Hilbert curves,


\begin{lemma}[]
\label{4.1.2}
If \(A\subseteq R^m\) is an open cell and \(f:A\to R^m\) an injective definable map, then \(f(A)\) contains
an open cell
\end{lemma}

\begin{proof}
Clearly for \(m=1\). Let \(m>1\) and assume inductively the lemma holds for lower values
of \(m\). Taking a decomposition of \(R^m\) that partitions \(f(A)\) we have
\begin{equation*}
f(A)=C_1\cup\dots\cup C_k\text{ for cells $C_i$ in }R^m
\end{equation*}
Then
\begin{equation*}
A=f^{-1}(C_1)\cup\dots\cup f^{-1}(C_k)
\end{equation*}
so at least one of the \(f^{-1}(C_i)\), say \(f^{-1}(C_1)\), contains a box \(B\) , and by
taking \(B\) suitably small we may assume that \(f|B\) is continuous. We now claim that \(C_1\)
is open.

If not, then by composing \(f|B:B\to C_1\) with a definable homeomorphism of \(C_1\) with a
cell in \(R^{m-1}\) we obtain a definable continuous injective map \(g:B\to R^{m-1}\).
Write \(B=B'\times(a,b)\)

Take \(c\) with \(a<c<b\) and consider the map \(h:B'\to R^{m-1}\) given by \(h(x)=f(x,c)\). By
the inductive assumption applied to \(h\) we get \(h(B')\supseteq D\) for some box \(D\) in \(R^{m-1}\).
Let \(y\) be a point in \(D\) and take \(x\) in \(B'\) with \(h(x)=y\)

If \(c'\neq c\) is sufficiently close to \(c\), then \(g(x,c')\) will be in \(D\),
so \(g(x,c')=h(x')=g(x',c)\) for some \(x'\in B'\). This contradicts the injectivity of \(g\)
\end{proof}

Box is a cell

\begin{proposition}[]
\begin{enumerate}
\item If \(X\subseteq Y\subseteq R^m\) and \(X,Y\) are definable, then \(\dim X\le\dim Y\le m\)
\item If \(X\subseteq R^m\) and \(Y\subseteq R^n\) are definable and there is a definable bijection between \(X\)
and \(Y\), then \(\dim X=\dim Y\)
\item If \(X,Y\subseteq R^m\) are definable, then \(\dim(X\cup Y)=\max\{\dim X,\dim Y\}\)
\end{enumerate}
\end{proposition}

\begin{proof}
\begin{enumerate}
\setcounter{enumi}{1}
\item Let \(f:X\to Y\) be a definable bijection and \(d=\dim X\), \(e=\dim Y\). It is enough to
show \(d\le e\).

Let \(A\) be an \((i_1,\dots,i_m)\)-cell contained in \(X\), with \(d=i_1+\dots+i_m\).
Then \(f\circ(p_A^{-1}):p(A)\to Y\) is an injective map and \(p(A)\) an open cell.
Replacing \(X\) by \(p(A)\), \(Y\) by \(f(A)\) and \(f\) by \(f\circ(p_A^{-1})\) we may as well
assume that \(d=m\) and that \(X\) is an open cell in \(R^d\). Let \(Y=C_1\cup\dots\cup C_k\) be a
partition of \(Y=f(X)\) into cells. Then \(X=f^{-1}(C_1)\cup\dots\cup f^{-1}(C_k)\), so by the cell
decomposition theorem  \(f^{-1}(C_i)\) contains an open cell \(B\) since \(X\) is open, for
some \(i\). Fix such \(i\) and \(B\)

Let \(C_i=C\subseteq R^n\) be a \((j_1,\dots,j_n)\)-cell. We shall prove that \(d\le j_1+\dots+j_n\).

Suppose \(d>j_1+\dots+j_n\), the composition
\begin{equation*}
B\xrightarrow{f|B}C\xrightarrow{p_C}p(C)\subseteq R^{j_1+\dots+j_n}
\end{equation*}
is an injective map. Identify \(R^{j_1+\dots+j_n}\) with a non-open cell \((R^{j_1+\dots+j_n})\times\{p\}\)
in \(R^d\), where \(p\in R^{d-(j_1+\dots+j_n)}\), we obtain a contradiction with lemma \ref{3.1.2}
\item Let \(d=\dim(X\cup Y)\), and let \(A\) be an \((i_1,\dots,i_m)\)-cell contained in \(X\cup Y\)
with \(d=i_1+\dots+i_m\). The open cell \(pA\subseteq R^d\) is the union of \(p_A(A\cap X)\)
and \(p_A(A\cap Y)\), so by the cell decomposition theorem, one of these sets,
say \(p_A(A\cap X)\), contains a box \(B\) in \(R^d\). Then \(p_A^{-1}(B)\) is
an \((i_1,\dots,i_m)\)-cell contained in \(X\), so that
\begin{equation*}
\dim X\ge d\ge \dim X
\end{equation*}
\end{enumerate}
\end{proof}

\begin{theorem}[]
If \(C\) is an \((i_1,\dots,i_n)\)-cell, then \(\dim(C)=\sum_{j=1}^ni_j\).
\end{theorem}

\begin{proof}
there is \(\pi:R^n\to R^m\) s.t. \(C\to\pi(C)\) is a homeomorphism, \(\pi(C)\) is open.
\end{proof}

\begin{lemma}[]
If \(C\subseteq R^n\) is an \((i_1,\dots,i_n)\)-cell, \(C'\subseteq R^m\) is a \((j_1,\dots,j_m)\)-cell, then
\(C\times C'\) is an \((i_1,\dots,i_n,j_1,\dots,j_m)\)-cell
\end{lemma}

\begin{theorem}[]
If \(A\subseteq R^n\), \(B\subseteq R^m\) definable, then
\begin{equation*}
\dim(A\times B)=\dim(A)+\dim(B)
\end{equation*}
\end{theorem}

\begin{proof}
Do cell decomposition on each one
\end{proof}

\begin{theorem}[]
If \(A\subseteq R^{n+m}\), if \(A_x=\{y\in R^m:(x,y\}\in A\) for all \(x\in R^n\), if
\(S(k)=\{x:\dim(A_x)=k\}\), then \(S(k)\) is definable and
\(\dim(A)=\max_k(k+\dim(S(k)))\)
\end{theorem}

\begin{lemma}[]
If \(D\subseteq R^n\times R^m\), if \(\pi(\barx,\bary)=\barx\), \(\dim(D)\ge\dim(\pi(D))\)
\end{lemma}

\begin{proof}
If \(C\subseteq R^{n+m}\) is a cell, then \(\pi(C)\) is an cell
\end{proof}

\begin{theorem}[]
If \(f:A\to B\) is a definable surjection, then \(\dim(A)\ge\dim(B)\)
\end{theorem}

\begin{proof}

\end{proof}

\begin{theorem}[]
If \(D\subseteq R^n\), \(m\le n\), then \(\dim(D)\ge m\) iff there is some coordinate projection \(\pi:R^n\to R^m\)
s.t. \(\tint(\pi(D))\neq\emptyset\).
\end{theorem}

\begin{proof}
If \(\exists\pi:R^n\to R^m\) and \(\tint(\pi(D))\neq\emptyset\), then \(\dim(D)\ge\dim(\pi(D))=m\)

If \(\dim(D)\ge m\), then \(\exists\) cell \(C\subseteq D\) s.t. \(C\) is an \((i_1,\dots,i_n)\)-cell
and \(l=i_1+\dots+i_n\ge m\).

There is \(\pi_0:R^n\to R^l\) s.t. \(\pi_0(C)\) is an open cell.

Then take \(\pi_1:R^l\to R^m\)
\end{proof}

\(\partial D=\cl(D)\setminus D\)


\begin{theorem}[]
If \(A\subseteq R^n\) definable, \(A\neq\emptyset\), then
\(\dim(\partial A)<\dim(A)\)
\end{theorem}

\begin{proof}

\end{proof}





The next result says among other things that the dimension of a set from a definable family
depends ``definably'' on its parameters

\begin{proposition}[]
Let \(S\subseteq R^m\times R^n\) be definable. For \(d\in\{-\infty,0,1,\dots,n\}\) put
\begin{equation*}
S(d):=\{a\in R^m:\dim S_a=d\}
\end{equation*}
Then \(S(d)\) is definable and the part of \(S\) above \(S(d)\) has dimension given by
\begin{equation*}
\dim\left( \bigcup_{a\in S(d)}\{a\}\times S_a \right)=\dim(S(d))+d
\end{equation*}
\end{proposition}

\begin{proof}
Let \(\cald\) be a decomposition of \(R^{m+n}\) partitioning \(S\)
\end{proof}

\(D=\bigsqcup_{i=1}^n C_i\), \(\chi(D)=\sum_{i=1}^n(-1)^{\dim(C_i)}\)

If \(K\) is a definable filed in o-minimal \(R\), then \(K\) is not algebraically closed with \(char(K)=p\)

\section{Tame topology}
\label{sec:org7024657}
\((R,+,0,1,\le,\dots)\) o-minimal ordered abelian group.

Note if \(n\in\N\), \(n>0\), then \(f(x)=nx\) is onto because \(\tbd(nR)\) is empty:
take \(c\in\tbd(nR)\), \((c-\epsilon,c)\subseteq nR\), \((c,c+\epsilon)\cap nR=\emptyset\) (maybe). Take \(\delta\) so small that \(n\delta<\epsilon\).
Take \(c-n\delta\in nR\) and \(c+n\delta\notin nR\), impossible since \(c-n\delta\in nR\) and \(2n\delta\in nR\)

\((R,+,0,\le)=\equiv(\Q,+,0,\le)\equiv(\R,+,0,\le)\)

\begin{definition}[]
If \(D\subseteq R^n\) definable, \(D\neq\emptyset\), \(\gamma(D)\in D\) is defined as follows:
\begin{enumerate}
\item if \(D\subseteq R^1\),
\begin{enumerate}
\item if \(D=\{a\}\), take \(\gamma(D)=a\)
\item if \(D=(a,b)\), take \(\gamma(D)=\frac{a+b}{2}\)
\item if \((D=(-\infty,a))\), take \(\gamma(D)=a-1\)
\item if \(D=(a,+\infty)\), take \(a+1\)
\item if \(D=(-\infty,\infty)\), take \(0\)
\item if \(D\) is arbitrary, suppose \(\tbd(D)=\{c_1,\dots,c_m\}\), \(c_1<\dots<c_m\), and \(D\) is a union
of some of \(S_0=(-\infty,c_1)\), \(S_1=\{c_1\}\), \(S_2=(c_1,c_2)\),\(\dots\), \(S_{2m+1}=(c_m,+\infty)\), then
take minimal \(i\) s.t. \(S_i\subseteq D\) and let \(\gamma(D)=\gamma(S_i)\).
\end{enumerate}
\item if \(D\subseteq R^{n+1}\), \(n\ge 1\), let \(\pi(D)\subseteq R^n\), \(\pi(\barx,y)=\barx\), let \(\bara=\gamma(\pi(D))\),
let \(D_{\bara}=\{y:(\bara,y\}\in D\}\neq\emptyset\), let \(b=\gamma(D_{\bara})\)
\end{enumerate}
\end{definition}

\begin{theorem}[]
\(\gamma(D)\in D\), if \(D\neq\emptyset\) and if \(\{D_{\bara}\}_{\bara\in Y}\) is definable,
then \(\bara\mapsto\gamma(D_{\bara})\) is definable
\end{theorem}

\begin{theorem}[]
if \(D\) is definable, \(E\) is a definable equivalence relation on \(D\), then \(\exists f:D\to D'\) surjective
definable s.t. \(f(x)=f(y)\Leftrightarrow xEy\). (so \(f\) induces a bijection \(D/E\to D'\))
\end{theorem}

\begin{proof}
let \(f(x)=\gamma([x]_E)\), if \(xEy\), \(f(x)=f(y)\). if \(f(x)=f(y)\), then \(xEy\).
\end{proof}

\begin{theorem}[]
If \(\barc\in\cl(D)\), \(D\subseteq R^n\), then \(\barc=\lim_{x\to 0}f(x)\) where \(f:(0,1)\to D\).
\end{theorem}

\begin{proof}
Let \(D_\epsilon=\{\barx\in D:\norm{\barx-\barc}<\epsilon\}\), \(D_\epsilon\neq\emptyset\) as \(\barc\in\cl(D)\).

Take \(f(\epsilon)=\gamma(D_\epsilon)\), \(f:R_{>0}\to D\), we can make \(f\) continuous by scale it and move it
\end{proof}

Assume \(R\) is ordered field.
\begin{definition}[]
\(D\subseteq R^n\) is definably compact if \(\forall\) continuous \(f:(0,1)\to D\), \(\lim_{x\to 0}f(x)\in D\)
\end{definition}

\begin{theorem}[]
\(D\) is definably compact iff \(D\) is closed and bounded
\end{theorem}

\begin{proof}
If \(\cl(D)\supsetneq D\), take \(\barc\in\cl(D)\setminus D\), so \(D\) is not definably compact

if \(D\) is not bounded, take \(A_N=\{\barx\in D:\norm{\barx}>N\}\), \(A_N\neq\emptyset\), \(\forall N>0\), so
let \(f(x)=\gamma(A_{1/x})\), \(\lim f(x)\) doesn't exist.

If \(D\) is closed and bounded, and \(f:(0,1)\to D\) is continuous, then \(\lim_{x\to 0}f(x)\)
(monotonicity theorem)
exists in \(R^n\), \(D\) is closed and therefore \(\barc\in D\)
\end{proof}

\begin{theorem}[]
If \(f:D\to R^n\) continuous definable, \(D\) is definably compact, then \(f(D)\) is definably compact
\end{theorem}

\begin{proof}
If not, take \(g:(0,1)\to f(D)\) continuous and \(\lim_{x\to 0}g(x)\notin f(D)\)

let \(h(x)=\gamma(\{y\in D:f(y)=g(x)\})\) definable, \(f(h(x))=g(x)\). \(h\) is continuous on \((0,\epsilon)\),
definable compactness implies
that \(\lim_{x\to 0}h(x)\in D\), \(f(\lim_{x\to 0}h(x))=\lim_{x\to 0}f(h(x))\in D\)
\end{proof}

\begin{corollary}[]
If \(f:D\to R\) continuous, \(D\) definably compact, \(D\neq\emptyset\), then \(\max(f(D))\), \(\min(f(D))\)
exist.
\end{corollary}

\begin{theorem}[]
If \(D\) is definably compact and \(f:D\to R^m\) is definable, continuous, injective,
then \(D\to f(D)\) is a homeomorphism
\end{theorem}

\begin{definition}[]
\(D\) is \textbf{definably path connected} if \(\forall a,b\in D\) \(\exists\) continuous
definable \(f:[0,1]\to D\), \(f(0)=a\), \(f(1)=b\).

\(D\) is \textbf{definably connected} if if \(f:D\to\{0,1\}\) is continuous, then \(f\) is constant
\end{definition}

\begin{remark}
If \(D\) is definably path connected, then \(D\) is definably connected

cells are path connected, definably homeomorphic to a box
\end{remark}

\begin{lemma}[]
If \(D_1\cap D_2=\emptyset\) and \(D_1\) and \(D_2\) are path-connected, \(\cl(D_1)\cap D_2\neq\emptyset\), then \(D_1\cup D_2\) is
definably path connected.
\end{lemma}

\begin{proof}
take \(\barc\in\cl(D_1)\cap D_2\), there is path from \(D_1\) to \(\barc\)
\end{proof}

\begin{lemma}[]
If \(D\) is definable and \(D_1,\dots,D_n\) are the path connected components, each \(D_i\) is
definable, clopen in \(D\)
\end{lemma}

\begin{corollary}[]
\(D\) is path connected \(\Leftrightarrow\) \(D\) is connected.
\end{corollary}

over \(\R\)

Fix o-minimal \((R,+,\cdot,\le,\dots)\) where \((R,+,\cdot,\le)\) is an o-minimal field

If \(f:I\to R\) is definable
\begin{gather*}
f'(x)=\lim_{y\to x}\frac{f(y)-f(x)}{y-x}\\
f^+(x)=\lim_{y\to x^+}\frac{f(y)-f(x)}{y-x}\\
f^-(x)=\lim_{y\to x^-}\frac{f(y)-f(x)}{y-x}
\end{gather*}

\(f^+(x),f^-(x)\) always exist in \(R_\infty=R\cup\{\pm\infty\}\)

\(f\) is differentiable at \(x\) if \(f^+(x)=f^-(x)\not\in\{\pm\infty\}\)

\begin{lemma}[]
If \(f\) has a local minimum at \(x\), then \(f^-(x)\le 0\le f^+(x)\)

similar for local maximum.

If \(f'(x)\) exists then \(f'(x)=0\) in either case
\end{lemma}

\begin{lemma}[]
If \(f:I\to R\) is definable, then \(\exists a\in I\) s.t. \(f^+(a)\le f^-(a)\) and \(f^+(x)\neq+\infty\) and \(f^-(x)\neq-\infty\)
\end{lemma}

\begin{proof}
Shrink \(I\) so that \(f\) is continuous. Take \([d,e]\subseteq I\), choose \(a,b,c\in R\) so that
if \(g(x)=f(x)+ax^2+bx+c\) then \(g(d)=0<g(\frac{d+e}{c})>g(e)=0\).
Now \(g\) has a local maximum at \(x\in(d,e)\). \(g^+(x)\le 0\le g^-(x)\), \(f^+(x)+2ax+b\le 0\le f^-(x)+2ax+b\)
\end{proof}

\begin{theorem}[]
If \(f:I\to R\) definable
\begin{enumerate}
\item \(\exists S\subseteq_fI\), \(\forall x\in I\setminus S\), \(f^+(x)\le f^-(x)\) and \(f^+(x)\neq\pm\infty\), \(f^-(x)\neq\infty\)
\end{enumerate}
\end{theorem}

\begin{theorem}[]
If \(f:[a,b]\to R\) is definable and continuous and differentiable on \((a,b)\), then
\begin{enumerate}
\item (Rolle's theorem) If \(f(a)=f(b)\) then \(\exists c\in(a,b)\), \(f'(c)=0\)
\item (Mean value theorem) \(\exists c\in(a,b)\), \(f'(c)=\frac{f(b)-f(a)}{b-a}\)
\end{enumerate}
\end{theorem}

\begin{theorem}[]
If \(f:R^n\to R\) is definable, then \(\exists S\subseteq R^n\), \(\dim(S)<n\) s.t. \(f\) is differentiable
on \(R^n\setminus S\)
\end{theorem}


\section{Triangulation}
\label{sec:org3bbc592}
An \textbf{affine subspace} of \(R^n\) is a set like \(a+V=\{a+x:x\in V\}\) where \(a\in R^n\) and \(V\) is a
linear subspace

\(a_0,\dots,a_k\in R^n\) are \textbf{affine independent} if there is no affine subspace \(A\subset R^n\)
s.t. \(\dim(A)<k\) and \(A\supseteq\{a_0,\dots,a_k\}\)

\begin{definition}[]
A \textbf{\(k\)-simplex} in \(R^n\) is a set like
\begin{equation*}
(a_0,\dots,a_k):=\{t_0a_0+\dots+t_ka_k:t_0,\dots,t_k\in R_{>0}, t_0+\dots+t_k=1\}
\end{equation*}
where \(a_0,\dots,a_k\) is affine independent.

A \textbf{face} of a \(k\)-simplex \((a_0,\dots,a_k)\) is one of the form \((b_0,\dots,b_l)\)
where \(\{b_0,\dots,b_l\}\subseteq\{a_0,\dots,a_k\}\)

A \textbf{closed simplicial complex} is a finite set \(K\) of simplices in \(R^n\) s.t.
\begin{enumerate}
\item pairwise disjoint
\item if \(\sigma\in K\) and \(\tau\) is a face of \(\sigma\), then \(\tau\in K\)
\end{enumerate}

A \textbf{subcomplex} is a subset \(K'\subseteq K\)

A \textbf{simplicial complex} is a subcomplex of a closed simplecial complex
\end{definition}

\(\abs{K}=\bigcup K\)

\(\abs{K}\) is definably compact iff \(K\) is a closed simplicial complex

\begin{theorem}[Triangulation theorem]
If \(D\subseteq R^n\) is definable, then \(\exists\) simplicial complex \(K\) and a definable
homeomorphism \(\Phi:D\to\abs{K}\)

If \(D_1,\dots,D_m\subseteq D\) definable, can choose \(K\), \(\Phi\) so that \(\exists K_1,\dots,K_n\subseteq K\) and \(\Phi(D_i)=K_i\)
\end{theorem}

Small number of homeomorphism classes

\(A\cong B\) is there is definable homeomorphism \(A\to B\)

\([A]=\{B:B\cong A\}\).

\begin{theorem}[]
\(\{[A]:A\subseteq R^n\text{ definable},n<\infty\}\) is countable
\end{theorem}

\begin{proof}
WMA \(A\) is \(\abs{K}\) for some simplicial complex \(K\). If \(a_0,\dots,a_n\) are the 0-simplices
in \(K\) then \([\abs{K}]\) is determined by the ``abstract simplicial complex''
\begin{equation*}
(\{a_0,\dots,a_n\},\{\{b_0,\dots,b_m\}\subseteq\{a_0,\dots,a_n\}:(b_0,\dots,b_m)\in K\})
\end{equation*}
\end{proof}

\begin{fact}[]
If \(\{D_a\}_{a\in X}\) is a definable family, then \(\{[D_a]:a\in X\}\) is finite.
\end{fact}

\begin{theorem}[]
If \(D\) is definable, then \(D\) is locally definably connected, locally contractible

If \(D\) is definably compact, then \(D\cong\abs{K}\), \(K\) is a closed simplicial complex

If \(R=\R\) then \(D\) is a finite CW-complex
\end{theorem}

\begin{theorem}[Extension]
If \(D\subseteq R^n\) definable and \(A\subseteq D\) definable and closed in \(D\) and \(f:A\to R\) is definable
and continuous, then \(\exists g:D\to R\), \(g\supseteq f\), \(g\) is definable and continuous
\end{theorem}


\begin{proof}
Trivial.
\end{proof}








\section{Problems}
\label{sec:org7f9ccf1}

\begin{center}
\begin{tabular}{lll}
\ref{Problem1} & \ref{Problem2} & \\\empty
\end{tabular}
\end{center}
\end{document}