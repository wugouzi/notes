% Created 2019-09-06 五 14:27
% Intended LaTeX compiler: pdflatex
\documentclass[11pt]{article}
\usepackage[utf8]{inputenc}
\usepackage[T1]{fontenc}
\usepackage{graphicx}
\usepackage{grffile}
\usepackage{longtable}
\usepackage{wrapfig}
\usepackage{rotating}
\usepackage[normalem]{ulem}
\usepackage{amsmath}
\usepackage{textcomp}
\usepackage{amssymb}
\usepackage{capt-of}
\usepackage{hyperref}
\usepackage{minted}
% TIPS
% \substack{a\\b} for multiple lines text





% pdfplots will load xolor automatically without option
\usepackage[dvipsnames]{xcolor}

\usepackage{forest}
% two-line text in node by [two \\ lines]
% \begin{forest} qtree, [..] \end{forest}
\forestset{
  qtree/.style={
    baseline,
    for tree={
      parent anchor=south,
      child anchor=north,
      align=center,
      inner sep=1pt,
    }}}
%\usepackage{flexisym}
% load order of mathtools and mathabx, otherwise conflict overbrace

\usepackage{mathtools}
%\usepackage{fourier}
\usepackage{pgfplots}
\usepackage{amsthm, mathabx,  amsmath, commath}
\usepackage{amsfonts}

\usepackage{empheq}
\usepackage{tikz}
\usetikzlibrary{arrows.meta}
\usepackage[most]{tcolorbox}

\newtheorem{theorem}{Theorem}[section]
\newtheorem{definition}{Definition}[section]
\newtheorem{corollary}{Corollary}[section]
\newtheorem{example}{Example}[section]
\newtheorem{lemma}{Lemma}[section]
\newtheorem{proposition}{Proposition}[section]

\newcommand{\bl}[1] {\boldsymbol{#1}}
\newcommand{\Wt}[1] {\stackrel{\sim}{\smash{#1}\rule{0pt}{1.1ex}}}
\newcommand{\wt}[1] {\widetilde{#1}}


%For boxed texts in align, use Aboxed{}
%otherwise use boxed{}

\DeclareMathSymbol{\widehatsym}{\mathord}{largesymbols}{"62}
\newcommand\lowerwidehatsym{%
  \text{\smash{\raisebox{-1.3ex}{%
    $\widehatsym$}}}}
\newcommand\fixwidehat[1]{%
  \mathchoice
    {\accentset{\displaystyle\lowerwidehatsym}{#1}}
    {\accentset{\textstyle\lowerwidehatsym}{#1}}
    {\accentset{\scriptstyle\lowerwidehatsym}{#1}}
    {\accentset{\scriptscriptstyle\lowerwidehatsym}{#1}}
}

\usepackage{graphicx}
    
% text on arrow for xRightarrow
\makeatletter
%\newcommand{\xRightarrow}[2][]{\ext@arrow 0359\Rightarrowfill@{#1}{#2}}
\makeatother


\def \bx {\boldsymbol{x}}
\def \ba {\boldsymbol{a}}
\def \bI {\boldsymbol{I}}
\def \bt {\boldsymbol{t}}
\def \bb {\boldsymbol{b}}
\def \bA {\boldsymbol{A}}
\def \bX {\boldsymbol{X}}
\def \bu {\boldsymbol{u}}
\def \bS {\boldsymbol{S}}
\def \bZ {\boldsymbol{Z}}
\def \bz {\boldsymbol{z}}
\def \by {\boldsymbol{y}}
\def \bw {\boldsymbol{w}}
\def \bT {\boldsymbol{T}}
\def \bS {\boldsymbol{S}}
\def \bm {\boldsymbol{m}}
\def \bW {\boldsymbol{W}}
\def \bY {\boldsymbol{Y}}
\def \bH {\boldsymbol{H}}
\def \blambda {\boldsymbol{\lambda}}
\def \bPhi {\boldsymbol{\Phi}}
\def \btheta {\boldsymbol{\theta}}
\def \bmu {\boldsymbol{\mu}}
\def \bphi {\boldsymbol{\phi}}
\def \bSigma {\boldsymbol{\Sigma}}
\def \lb {\left\{}
\def \rb {\right\}}
\def \caln {\mathcal{N}}
\def \dissum {\displaystyle\Sigma}
\def \dispro {\displaystyle\prod}
\def \E {\mathbb{E}}
\def \Q {\mathbb{Q}}
\def \V {\mathbb{V}}
\def \R {\mathbb{R}}
\def \calq {\mathcal{Q}}
\def \calg {\mathcal{G}}
\def \caln {\mathcal{N}}
\def \calr {\mathcal{R}}
\def \calm {\mathcal{M}}
\def \calc {\mathcal{C}}
\def \bcup {\bigcup}

\author{Joseph J. Rotman}
\date{\today}
\title{Advanced Modern Algebra}
\hypersetup{
 pdfauthor={Joseph J. Rotman},
 pdftitle={Advanced Modern Algebra},
 pdfkeywords={},
 pdfsubject={},
 pdfcreator={Emacs 26.2 (Org mode 9.2.5)}, 
 pdflang={English}}
\begin{document}

\maketitle
\tableofcontents \clearpage
\section{Group \rom{1}}
\label{sec:org9716707}
\subsection{Permutations}
\label{sec:org51aec9f}
\begin{definition}[]
A \textbf{permutation} of a set \(X\) is a bijection from \(X\) to itself.
\end{definition}


\begin{definition}[]
The family of all the permutations of a set \(X\), denoted by \(S_X\) is called
the \textbf{symmetric group} on \(X\). When \(X=\lb 1,2,\dots,n\rb\), \(S_X\) is
usually denoted by \(X_n\) and is called the \textbf{symmetric group on } \(n\)
\textbf{letters} 
\end{definition}

\begin{definition}[]
Let \(i_1,i_2,\dots,i_r\) be distinct integers in \(\lb 1,2,\dots,n\rb\). If
\(\alpha\in S_n\) fixes the other integers and if
\begin{equation*}
\alpha(i_1)=i_2,\alpha(i_2)=i_3,\dots,\alpha(i_{r-1})=i_r,\alpha(i_r)=i_1
\end{equation*}
then \(\alpha\) is called an textbf\{r-cycle\}. \(\alpha\) is a cycle of
\textbf{length} \(r\) and denoted by
\begin{equation*}
\alpha=(i_1\; i_2\;\dots\; i_r)
\end{equation*}
\end{definition}

2-cycles are also called the \textbf{transpositions}.

\begin{definition}[]
Two permutations \(\alpha,\beta\in S_n\) are \textbf{disjoint} if every \(i\)
moved by one is fixed by the other.
\end{definition}

\begin{lemma}[]
Disjoint permutations \(\alpha,\beta\in S_n\) commute
\end{lemma}

\begin{proposition}[]
Every permutation \(\alpha\in S_n\) is either a cycle or a product of disjoint cycles.
\end{proposition}

\begin{proof}
Induction on the number \(k\) of points moved by \(\alpha\)
\end{proof}

\begin{definition}[]
A \textbf{complete factorization} of a permutation \(\alpha\) is a
factorization of \(\alpha\) into disjoint cycles that contains exactly one
1-cycle \((i)\) for every \(i\) fixed by \(\alpha\)
\end{definition}

\begin{theorem}[]
Let \(\alpha\in S_n\) and let \(\alpha=\beta_1\dots\beta_t\) be a complete
factorization into disjoint cycles. This factorization is unique except for
the order in which the cycles occur
\end{theorem}

\begin{proof}
for all \(i\), if \(\beta_t(i)\neq i\), then \(\beta_t^k(i)\neq\beta_t^{k-1}(i)\)
for any \(k\le 1\)
\end{proof}

\begin{lemma}[]
If \(\gamma,\alpha\in S_n\), then \(\alpha\gamma\alpha^{-1}\) has the same cycle
structure as \(\gamma\). In more detail, if the complete factorization of
\(\gamma\) is
\begin{equation*}
\gamma=\beta_1\beta_2\dots(i_1\; i_2\;\dots)\dots\beta_t
\end{equation*}
then \(\alpha\gamma\alpha^{-1}\) is permutation that is obtained from \(\gamma\)
by applying \(\alpha\) to the symbols in the cycles of \(\gamma\)
\end{lemma}

Example. Suppose
\begin{gather*}
\beta=(1\;2\;3)(4)(5)\\
\gamma=(5\;2\;4)(1)(3)
\end{gather*}
then we can easily find the \(\alpha\)
\begin{equation*}
\alpha=
\begin{pmatrix}
1&2&3&4&5\\
5&2&4&1&3
\end{pmatrix}
\end{equation*}
\begin{theorem}[]
Permutations \(\gamma\) and \(\sigma\) in \(S_n\) has the same cycle structure if
and only if there exists \(\alpha\in S_n\) with \(\sigma=\alpha\gamma\alpha^{-1}\)
\end{theorem}


\begin{proposition}[]
If \(n\le 2\) then every \(\alpha\in S_n\) is a product of tranpositions
\end{proposition}
\begin{proof}
\((1\;2\;\dots\; r)=(1\; r)(1\; r-1)\dots(1\; 2)\)
\end{proof}


\begin{definition}[]
A permutation \(\alpha\in S_n\) is \textbf{even} if it can be factored into a
product of an even number of transpositions. Otherwise \textbf{odd}
\end{definition}

\begin{definition}[]
If \(\alpha\in S_n\) and \(\alpha=\beta_1\dots\beta_t\) is a complete
factorization, then \textbf{signum} \(\alpha\) is defined by
\begin{equation*}
\sgn(\alpha)=(-1)^{n-t}
\end{equation*}
\end{definition}

\begin{theorem}[]
For all \(\alpha,\beta\in S_n\)
\begin{equation*}
\sgn(\alpha\beta)=\sgn(\alpha)\sgn(\beta)
\end{equation*}
\end{theorem}

\begin{theorem}[]
\begin{enumerate}
\item Let \(\alpha\in S_n\); if \(\sgn(\alpha)=1\) then \(\alpha\) is even. otherwise
odd
\item A permutation \(\alpha\) is odd if and only if it's a product of an odd
number of transpositions
\end{enumerate}
\end{theorem}

\begin{corollary}[]
Let \(\alpha,\beta\in S_n\). If \(\alpha\) and \(\beta\) have the same parity, then
\(\alpha\beta\) is even while if \(\alpha\) and \(\beta\) have distinct parity,
\(\alpha\beta\) is odd
\end{corollary}
\subsection{Groups}
\label{sec:orgc9f1817}
\begin{definition}[]
A \textbf{binary operation} on a set \(G\) is a function
\begin{equation*}
*:G\times G\to G
\end{equation*}
\end{definition}

\begin{definition}[]
A \textbf{group} is a set \(G\) equipped with a binary operation * s.t.
\begin{enumerate}
\item the \textbf{associative law} holds
\item \textbf{identity}
\item every \(x\in G\) has an \textbf{inverse}, there is a \(x'\in G\)  with 
\(x*x'=e=x'*x\)
\end{enumerate}
\end{definition}

\begin{definition}[]
A group \(G\) is called \textbf{abelian} if it satisfies the
\textbf{commutative law}
\end{definition}

\begin{lemma}[]
Let \(G\) be a group
\begin{enumerate}
\item The \textbf{cancellation laws} holds: if either \(x*a=x*b\) or \(a*x=b*x\), then
\(a=b\)
\item \(e\) is unique
\item Each \(x\in G\) has a unique inverse
\item \((x^{-1})^{-1}=x\)
\end{enumerate}
\end{lemma}

\begin{definition}[]
An expression \(a_1a_2\dots a_n\) \textbf{needs no parentheses} if all the ultimate
products it yields are equal
\end{definition}

\begin{theorem}[Generalized Associativity]
If \(G\) is a group and \(a_1,a_2,\dots,a_n\in G\) then the expression
\(a_1a_2\dots a_n\) needs no parentheses
\end{theorem}

\begin{definition}[]
Let \(G\) be a group and let \(a\in G\). If \(a^k=1\) for some \(k>1\) then the
smallest such exponent \(k\le 1\) is called the \tf{order} or \(a\); if no such
power exists, then one says that \(a\) has \tf{infinite order}
\end{definition}

\begin{proposition}[]
If \(G\) is a finite group, then every \(x\in G\) has finite order
\end{proposition}

\begin{definition}[]
A \tf{motion} is a distance preserving bijection \(\varphi:\R^2\to\R^2\). If
\(\pi\) is a polygon in the plane, then its \tf{symmetry group} \(\Sigma(\pi)\)
consists of all the motions \(\varphi\) for which \(\varphi(\pi)=\pi\). The
elements of \(\Sigma(\pi)\) are called the \tf{symmetries} of \(\pi\)
\end{definition}

Let \(\pi_4\) be a square. Then the group \(\Sigma(\pi_4)\) is called the
\tf{dihedral group} with 8 elements, denoted by \(D_8\)

\begin{definition}[]
If \(\pi_n\) is a regular polygon with \(n\) vertices \(v_1,\dots,v_n\) and center
\(O\), then the symmetry group \(\Sigma(\pi_n)\) is called the \tf\{dihedral
group\} with \(2n\) elements, and it's denoted by \(D_{2n}\)
\end{definition}
\subsection{Lagrange's theorem}
\label{sec:orgd6b446f}
\end{document}