% Created 2023-02-25 Sat 16:21
% Intended LaTeX compiler: pdflatex
\documentclass[11pt]{article}
\usepackage[utf8]{inputenc}
\usepackage[T1]{fontenc}
\usepackage{graphicx}
\usepackage{longtable}
\usepackage{wrapfig}
\usepackage{rotating}
\usepackage[normalem]{ulem}
\usepackage{amsmath}
\usepackage{amssymb}
\usepackage{capt-of}
\usepackage{hyperref}
\graphicspath{{../../books/}}
% wrong resolution of image
% https://tex.stackexchange.com/questions/21627/image-from-includegraphics-showing-in-wrong-image-size?rq=1

%%%%%%%%%%%%%%%%%%%%%%%%%%%%%%%%%%%%%%
%% TIPS                                 %%
%%%%%%%%%%%%%%%%%%%%%%%%%%%%%%%%%%%%%%
% \substack{a\\b} for multiple lines text
% \usepackage{expl3}
% \expandafter\def\csname ver@l3regex.sty\endcsname{}
% \usepackage{pkgloader}
\usepackage[utf8]{inputenc}

% nfss error
% \usepackage[B1,T1]{fontenc}
\usepackage{fontspec}

% \usepackage[Emoticons]{ucharclasses}
\newfontfamily\DejaSans{DejaVu Sans}
% \setDefaultTransitions{\DejaSans}{}

% pdfplots will load xolor automatically without option
\usepackage[dvipsnames]{xcolor}

%                                                             ┳┳┓   ┓
%                                                             ┃┃┃┏┓╋┣┓
%                                                             ┛ ┗┗┻┗┛┗
% \usepackage{amsmath} mathtools loads the amsmath
\usepackage{amsmath}
\usepackage{mathtools}

\usepackage{amsthm}
\usepackage{amsbsy}

%\usepackage{commath}

\usepackage{amssymb}

\usepackage{mathrsfs}
%\usepackage{mathabx}
\usepackage{stmaryrd}
\usepackage{empheq}

\usepackage{scalerel}
\usepackage{stackengine}
\usepackage{stackrel}



\usepackage{nicematrix}
\usepackage{tensor}
\usepackage{blkarray}
\usepackage{siunitx}
\usepackage[f]{esvect}

% centering \not on a letter
\usepackage{slashed}
\usepackage[makeroom]{cancel}

%\usepackage{merriweather}
\usepackage{unicode-math}
\setmainfont{TeX Gyre Pagella}
% \setmathfont{STIX}
%\setmathfont{texgyrepagella-math.otf}
%\setmathfont{Libertinus Math}
\setmathfont{Latin Modern Math}

 % \setmathfont[range={\smwhtdiamond,\enclosediamond,\varlrtriangle}]{Latin Modern Math}
\setmathfont[range={\rightrightarrows,\twoheadrightarrow,\leftrightsquigarrow,\triangledown,\vartriangle,\precneq,\succneq,\prec,\succ,\preceq,\succeq,\tieconcat}]{XITS Math}
 \setmathfont[range={\int,\setminus}]{Libertinus Math}
 % \setmathfont[range={\mathalpha}]{TeX Gyre Pagella Math}
%\setmathfont[range={\mitA,\mitB,\mitC,\mitD,\mitE,\mitF,\mitG,\mitH,\mitI,\mitJ,\mitK,\mitL,\mitM,\mitN,\mitO,\mitP,\mitQ,\mitR,\mitS,\mitT,\mitU,\mitV,\mitW,\mitX,\mitY,\mitZ,\mita,\mitb,\mitc,\mitd,\mite,\mitf,\mitg,\miti,\mitj,\mitk,\mitl,\mitm,\mitn,\mito,\mitp,\mitq,\mitr,\mits,\mitt,\mitu,\mitv,\mitw,\mitx,\mity,\mitz}]{TeX Gyre Pagella Math}
% unicode is not good at this!
%\let\nmodels\nvDash

 \usepackage{wasysym}

 % for wide hat
 \DeclareSymbolFont{yhlargesymbols}{OMX}{yhex}{m}{n} \DeclareMathAccent{\what}{\mathord}{yhlargesymbols}{"62}

%                                                               ┏┳┓•┓
%                                                                ┃ ┓┃┏┓
%                                                                ┻ ┗┛┗┗

\usepackage{pgfplots}
\pgfplotsset{compat=1.18}
\usepackage{tikz}
\usepackage{tikz-cd}
\tikzcdset{scale cd/.style={every label/.append style={scale=#1},
    cells={nodes={scale=#1}}}}
% TODO: discard qtree and use forest
% \usepackage{tikz-qtree}
\usepackage{forest}

\usetikzlibrary{arrows,positioning,calc,fadings,decorations,matrix,decorations,shapes.misc}
%setting from geogebra
\definecolor{ccqqqq}{rgb}{0.8,0,0}

%                                                          ┳┳┓•    ┓┓
%                                                          ┃┃┃┓┏┏┏┓┃┃┏┓┏┓┏┓┏┓┓┏┏
%                                                          ┛ ┗┗┛┗┗ ┗┗┗┻┛┗┗ ┗┛┗┻┛
%\usepackage{twemojis}
\usepackage[most]{tcolorbox}
\usepackage{threeparttable}
\usepackage{tabularx}

\usepackage{enumitem}
\usepackage[indLines=false]{algpseudocodex}
\usepackage[]{algorithm2e}
% \SetKwComment{Comment}{/* }{ */}
% \algrenewcommand\algorithmicrequire{\textbf{Input:}}
% \algrenewcommand\algorithmicensure{\textbf{Output:}}
% wrong with preview
\usepackage{subcaption}
\usepackage{caption}
% {\aunclfamily\Huge}
\usepackage{auncial}

\usepackage{float}

\usepackage{fancyhdr}

\usepackage{ifthen}
\usepackage{xargs}

\definecolor{mintedbg}{rgb}{0.99,0.99,0.99}
\usepackage[cachedir=\detokenize{~/miscellaneous/trash}]{minted}
\setminted{breaklines,
  mathescape,
  bgcolor=mintedbg,
  fontsize=\footnotesize,
  frame=single,
  linenos}
\usemintedstyle{xcode}
\usepackage{tcolorbox}
\usepackage{etoolbox}



\usepackage{imakeidx}
\usepackage{hyperref}
\usepackage{soul}
\usepackage{framed}

% don't use this for preview
%\usepackage[margin=1.5in]{geometry}
% \usepackage{geometry}
% \geometry{legalpaper, landscape, margin=1in}
\usepackage[font=itshape]{quoting}

%\LoadPackagesNow
%\usepackage[xetex]{preview}
%%%%%%%%%%%%%%%%%%%%%%%%%%%%%%%%%%%%%%%
%% USEPACKAGES end                       %%
%%%%%%%%%%%%%%%%%%%%%%%%%%%%%%%%%%%%%%%

%%%%%%%%%%%%%%%%%%%%%%%%%%%%%%%%%%%%%%%
%% Algorithm environment
%%%%%%%%%%%%%%%%%%%%%%%%%%%%%%%%%%%%%%%
\SetKwIF{Recv}{}{}{upon receiving}{do}{}{}{}
\SetKwBlock{Init}{initially do}{}
\SetKwProg{Function}{Function}{:}{}

% https://github.com/chrmatt/algpseudocodex/issues/3
\algnewcommand\algorithmicswitch{\textbf{switch}}%
\algnewcommand\algorithmiccase{\textbf{case}}
\algnewcommand\algorithmicof{\textbf{of}}
\algnewcommand\algorithmicotherwise{\texttt{otherwise} $\Rightarrow$}

\makeatletter
\algdef{SE}[SWITCH]{Switch}{EndSwitch}[1]{\algpx@startIndent\algpx@startCodeCommand\algorithmicswitch\ #1\ \algorithmicdo}{\algpx@endIndent\algpx@startCodeCommand\algorithmicend\ \algorithmicswitch}%
\algdef{SE}[CASE]{Case}{EndCase}[1]{\algpx@startIndent\algpx@startCodeCommand\algorithmiccase\ #1}{\algpx@endIndent\algpx@startCodeCommand\algorithmicend\ \algorithmiccase}%
\algdef{SE}[CASEOF]{CaseOf}{EndCaseOf}[1]{\algpx@startIndent\algpx@startCodeCommand\algorithmiccase\ #1 \algorithmicof}{\algpx@endIndent\algpx@startCodeCommand\algorithmicend\ \algorithmiccase}
\algdef{SE}[OTHERWISE]{Otherwise}{EndOtherwise}[0]{\algpx@startIndent\algpx@startCodeCommand\algorithmicotherwise}{\algpx@endIndent\algpx@startCodeCommand\algorithmicend\ \algorithmicotherwise}
\ifbool{algpx@noEnd}{%
  \algtext*{EndSwitch}%
  \algtext*{EndCase}%
  \algtext*{EndCaseOf}
  \algtext*{EndOtherwise}
  %
  % end indent line after (not before), to get correct y position for multiline text in last command
  \apptocmd{\EndSwitch}{\algpx@endIndent}{}{}%
  \apptocmd{\EndCase}{\algpx@endIndent}{}{}%
  \apptocmd{\EndCaseOf}{\algpx@endIndent}{}{}
  \apptocmd{\EndOtherwise}{\algpx@endIndent}{}{}
}{}%

\pretocmd{\Switch}{\algpx@endCodeCommand}{}{}
\pretocmd{\Case}{\algpx@endCodeCommand}{}{}
\pretocmd{\CaseOf}{\algpx@endCodeCommand}{}{}
\pretocmd{\Otherwise}{\algpx@endCodeCommand}{}{}

% for end commands that may not be printed, tell endCodeCommand whether we are using noEnd
\ifbool{algpx@noEnd}{%
  \pretocmd{\EndSwitch}{\algpx@endCodeCommand[1]}{}{}%
  \pretocmd{\EndCase}{\algpx@endCodeCommand[1]}{}{}
  \pretocmd{\EndCaseOf}{\algpx@endCodeCommand[1]}{}{}%
  \pretocmd{\EndOtherwise}{\algpx@endCodeCommand[1]}{}{}
}{%
  \pretocmd{\EndSwitch}{\algpx@endCodeCommand[0]}{}{}%
  \pretocmd{\EndCase}{\algpx@endCodeCommand[0]}{}{}%
  \pretocmd{\EndCaseOf}{\algpx@endCodeCommand[0]}{}{}
  \pretocmd{\EndOtherwise}{\algpx@endCodeCommand[0]}{}{}
}%
\makeatother
% % For algpseudocode
% \algnewcommand\algorithmicswitch{\textbf{switch}}
% \algnewcommand\algorithmiccase{\textbf{case}}
% \algnewcommand\algorithmiccaseof{\textbf{case}}
% \algnewcommand\algorithmicof{\textbf{of}}
% % New "environments"
% \algdef{SE}[SWITCH]{Switch}{EndSwitch}[1]{\algorithmicswitch\ #1\ \algorithmicdo}{\algorithmicend\ \algorithmicswitch}%
% \algdef{SE}[CASE]{Case}{EndCase}[1]{\algorithmiccase\ #1}{\algorithmicend\ \algorithmiccase}%
% \algtext*{EndSwitch}%
% \algtext*{EndCase}
% \algdef{SE}[CASEOF]{CaseOf}{EndCaseOf}[1]{\algorithmiccaseof\ #1 \algorithmicof}{\algorithmicend\ \algorithmiccaseof}
% \algtext*{EndCaseOf}



%\pdfcompresslevel0

% quoting from
% https://tex.stackexchange.com/questions/391726/the-quotation-environment
\NewDocumentCommand{\bywhom}{m}{% the Bourbaki trick
  {\nobreak\hfill\penalty50\hskip1em\null\nobreak
   \hfill\mbox{\normalfont(#1)}%
   \parfillskip=0pt \finalhyphendemerits=0 \par}%
}

\NewDocumentEnvironment{pquotation}{m}
  {\begin{quoting}[
     indentfirst=true,
     leftmargin=\parindent,
     rightmargin=\parindent]\itshape}
  {\bywhom{#1}\end{quoting}}

\indexsetup{othercode=\small}
\makeindex[columns=2,options={-s /media/wu/file/stuuudy/notes/index_style.ist},intoc]
\makeatletter
\def\@idxitem{\par\hangindent 0pt}
\makeatother


% \newcounter{dummy} \numberwithin{dummy}{section}
\newtheorem{dummy}{dummy}[section]
\theoremstyle{definition}
\newtheorem{definition}[dummy]{Definition}
\theoremstyle{plain}
\newtheorem{corollary}[dummy]{Corollary}
\newtheorem{lemma}[dummy]{Lemma}
\newtheorem{proposition}[dummy]{Proposition}
\newtheorem{theorem}[dummy]{Theorem}
\newtheorem{notation}[dummy]{Notation}
\newtheorem{conjecture}[dummy]{Conjecture}
\newtheorem{fact}[dummy]{Fact}
\newtheorem{warning}[dummy]{Warning}
\theoremstyle{definition}
\newtheorem{examplle}{Example}[section]
\theoremstyle{remark}
\newtheorem*{remark}{Remark}
\newtheorem{exercise}{Exercise}[subsection]
\newtheorem{problem}{Problem}[subsection]
\newtheorem{observation}{Observation}[section]
\newenvironment{claim}[1]{\par\noindent\textbf{Claim:}\space#1}{}

\makeatletter
\DeclareFontFamily{U}{tipa}{}
\DeclareFontShape{U}{tipa}{m}{n}{<->tipa10}{}
\newcommand{\arc@char}{{\usefont{U}{tipa}{m}{n}\symbol{62}}}%

\newcommand{\arc}[1]{\mathpalette\arc@arc{#1}}

\newcommand{\arc@arc}[2]{%
  \sbox0{$\m@th#1#2$}%
  \vbox{
    \hbox{\resizebox{\wd0}{\height}{\arc@char}}
    \nointerlineskip
    \box0
  }%
}
\makeatother

\setcounter{MaxMatrixCols}{20}
%%%%%%% ABS
\DeclarePairedDelimiter\abss{\lvert}{\rvert}%
\DeclarePairedDelimiter\normm{\lVert}{\rVert}%

% Swap the definition of \abs* and \norm*, so that \abs
% and \norm resizes the size of the brackets, and the
% starred version does not.
\makeatletter
\let\oldabs\abss
%\def\abs{\@ifstar{\oldabs}{\oldabs*}}
\newcommand{\abs}{\@ifstar{\oldabs}{\oldabs*}}
\newcommand{\norm}[1]{\left\lVert#1\right\rVert}
%\let\oldnorm\normm
%\def\norm{\@ifstar{\oldnorm}{\oldnorm*}}
%\renewcommand{norm}{\@ifstar{\oldnorm}{\oldnorm*}}
\makeatother

% \stackMath
% \newcommand\what[1]{%
% \savestack{\tmpbox}{\stretchto{%
%   \scaleto{%
%     \scalerel*[\widthof{\ensuremath{#1}}]{\kern-.6pt\bigwedge\kern-.6pt}%
%     {\rule[-\textheight/2]{1ex}{\textheight}}%WIDTH-LIMITED BIG WEDGE
%   }{\textheight}%
% }{0.5ex}}%
% \stackon[1pt]{#1}{\tmpbox}%
% }

% \newcommand\what[1]{\ThisStyle{%
%     \setbox0=\hbox{$\SavedStyle#1$}%
%     \stackengine{-1.0\ht0+.5pt}{$\SavedStyle#1$}{%
%       \stretchto{\scaleto{\SavedStyle\mkern.15mu\char'136}{2.6\wd0}}{1.4\ht0}%
%     }{O}{c}{F}{T}{S}%
%   }
% }

% \newcommand\wtilde[1]{\ThisStyle{%
%     \setbox0=\hbox{$\SavedStyle#1$}%
%     \stackengine{-.1\LMpt}{$\SavedStyle#1$}{%
%       \stretchto{\scaleto{\SavedStyle\mkern.2mu\AC}{.5150\wd0}}{.6\ht0}%
%     }{O}{c}{F}{T}{S}%
%   }
% }

% \newcommand\wbar[1]{\ThisStyle{%
%     \setbox0=\hbox{$\SavedStyle#1$}%
%     \stackengine{.5pt+\LMpt}{$\SavedStyle#1$}{%
%       \rule{\wd0}{\dimexpr.3\LMpt+.3pt}%
%     }{O}{c}{F}{T}{S}%
%   }
% }

\newcommand{\bl}[1] {\boldsymbol{#1}}
\newcommand{\Wt}[1] {\stackrel{\sim}{\smash{#1}\rule{0pt}{1.1ex}}}
\newcommand{\wt}[1] {\widetilde{#1}}
\newcommand{\tf}[1] {\textbf{#1}}

\newcommand{\wu}[1]{{\color{red} #1}}

%For boxed texts in align, use Aboxed{}
%otherwise use boxed{}

\DeclareMathSymbol{\widehatsym}{\mathord}{largesymbols}{"62}
\newcommand\lowerwidehatsym{%
  \text{\smash{\raisebox{-1.3ex}{%
    $\widehatsym$}}}}
\newcommand\fixwidehat[1]{%
  \mathchoice
    {\accentset{\displaystyle\lowerwidehatsym}{#1}}
    {\accentset{\textstyle\lowerwidehatsym}{#1}}
    {\accentset{\scriptstyle\lowerwidehatsym}{#1}}
    {\accentset{\scriptscriptstyle\lowerwidehatsym}{#1}}
  }


\newcommand{\cupdot}{\mathbin{\dot{\cup}}}
\newcommand{\bigcupdot}{\mathop{\dot{\bigcup}}}

\usepackage{graphicx}

\usepackage[toc,page]{appendix}

% text on arrow for xRightarrow
\makeatletter
%\newcommand{\xRightarrow}[2][]{\ext@arrow 0359\Rightarrowfill@{#1}{#2}}
\makeatother

% Arbitrary long arrow
\newcommand{\Rarrow}[1]{%
\parbox{#1}{\tikz{\draw[->](0,0)--(#1,0);}}
}

\newcommand{\LRarrow}[1]{%
\parbox{#1}{\tikz{\draw[<->](0,0)--(#1,0);}}
}


\makeatletter
\providecommand*{\rmodels}{%
  \mathrel{%
    \mathpalette\@rmodels\models
  }%
}
\newcommand*{\@rmodels}[2]{%
  \reflectbox{$\m@th#1#2$}%
}
\makeatother

% Roman numerals
\makeatletter
\newcommand*{\rom}[1]{\expandafter\@slowromancap\romannumeral #1@}
\makeatother
% \\def \\b\([a-zA-Z]\) {\\boldsymbol{[a-zA-z]}}
% \\DeclareMathOperator{\\b\1}{\\textbf{\1}}

\DeclareMathOperator*{\argmin}{arg\,min}
\DeclareMathOperator*{\argmax}{arg\,max}

\DeclareMathOperator{\bone}{\textbf{1}}
\DeclareMathOperator{\bx}{\textbf{x}}
\DeclareMathOperator{\bz}{\textbf{z}}
\DeclareMathOperator{\bff}{\textbf{f}}
\DeclareMathOperator{\ba}{\textbf{a}}
\DeclareMathOperator{\bk}{\textbf{k}}
\DeclareMathOperator{\bs}{\textbf{s}}
\DeclareMathOperator{\bh}{\textbf{h}}
\DeclareMathOperator{\bc}{\textbf{c}}
\DeclareMathOperator{\br}{\textbf{r}}
\DeclareMathOperator{\bi}{\textbf{i}}
\DeclareMathOperator{\bj}{\textbf{j}}
\DeclareMathOperator{\bn}{\textbf{n}}
\DeclareMathOperator{\be}{\textbf{e}}
\DeclareMathOperator{\bo}{\textbf{o}}
\DeclareMathOperator{\bU}{\textbf{U}}
\DeclareMathOperator{\bL}{\textbf{L}}
\DeclareMathOperator{\bV}{\textbf{V}}
\def \bzero {\mathbf{0}}
\def \bbone {\mathbb{1}}
\def \btwo {\mathbf{2}}
\DeclareMathOperator{\bv}{\textbf{v}}
\DeclareMathOperator{\bp}{\textbf{p}}
\DeclareMathOperator{\bI}{\textbf{I}}
\def \dbI {\dot{\bI}}
\DeclareMathOperator{\bM}{\textbf{M}}
\DeclareMathOperator{\bN}{\textbf{N}}
\DeclareMathOperator{\bK}{\textbf{K}}
\DeclareMathOperator{\bt}{\textbf{t}}
\DeclareMathOperator{\bb}{\textbf{b}}
\DeclareMathOperator{\bA}{\textbf{A}}
\DeclareMathOperator{\bX}{\textbf{X}}
\DeclareMathOperator{\bu}{\textbf{u}}
\DeclareMathOperator{\bS}{\textbf{S}}
\DeclareMathOperator{\bZ}{\textbf{Z}}
\DeclareMathOperator{\bJ}{\textbf{J}}
\DeclareMathOperator{\by}{\textbf{y}}
\DeclareMathOperator{\bw}{\textbf{w}}
\DeclareMathOperator{\bT}{\textbf{T}}
\DeclareMathOperator{\bF}{\textbf{F}}
\DeclareMathOperator{\bmm}{\textbf{m}}
\DeclareMathOperator{\bW}{\textbf{W}}
\DeclareMathOperator{\bR}{\textbf{R}}
\DeclareMathOperator{\bC}{\textbf{C}}
\DeclareMathOperator{\bD}{\textbf{D}}
\DeclareMathOperator{\bE}{\textbf{E}}
\DeclareMathOperator{\bQ}{\textbf{Q}}
\DeclareMathOperator{\bP}{\textbf{P}}
\DeclareMathOperator{\bY}{\textbf{Y}}
\DeclareMathOperator{\bH}{\textbf{H}}
\DeclareMathOperator{\bB}{\textbf{B}}
\DeclareMathOperator{\bG}{\textbf{G}}
\def \blambda {\symbf{\lambda}}
\def \boldeta {\symbf{\eta}}
\def \balpha {\symbf{\alpha}}
\def \btau {\symbf{\tau}}
\def \bbeta {\symbf{\beta}}
\def \bgamma {\symbf{\gamma}}
\def \bxi {\symbf{\xi}}
\def \bLambda {\symbf{\Lambda}}
\def \bGamma {\symbf{\Gamma}}

\newcommand{\bto}{{\boldsymbol{\to}}}
\newcommand{\Ra}{\Rightarrow}
\newcommand{\xrsa}[1]{\overset{#1}{\rightsquigarrow}}
\newcommand{\xlsa}[1]{\overset{#1}{\leftsquigarrow}}
\newcommand\und[1]{\underline{#1}}
\newcommand\ove[1]{\overline{#1}}
%\def \concat {\verb|^|}
\def \bPhi {\mbfPhi}
\def \btheta {\mbftheta}
\def \bTheta {\mbfTheta}
\def \bmu {\mbfmu}
\def \bphi {\mbfphi}
\def \bSigma {\mbfSigma}
\def \la {\langle}
\def \ra {\rangle}

\def \caln {\mathcal{N}}
\def \dissum {\displaystyle\Sigma}
\def \dispro {\displaystyle\prod}

\def \caret {\verb!^!}

\def \A {\mathbb{A}}
\def \B {\mathbb{B}}
\def \C {\mathbb{C}}
\def \D {\mathbb{D}}
\def \E {\mathbb{E}}
\def \F {\mathbb{F}}
\def \G {\mathbb{G}}
\def \H {\mathbb{H}}
\def \I {\mathbb{I}}
\def \J {\mathbb{J}}
\def \K {\mathbb{K}}
\def \L {\mathbb{L}}
\def \M {\mathbb{M}}
\def \N {\mathbb{N}}
\def \O {\mathbb{O}}
\def \P {\mathbb{P}}
\def \Q {\mathbb{Q}}
\def \R {\mathbb{R}}
\def \S {\mathbb{S}}
\def \T {\mathbb{T}}
\def \U {\mathbb{U}}
\def \V {\mathbb{V}}
\def \W {\mathbb{W}}
\def \X {\mathbb{X}}
\def \Y {\mathbb{Y}}
\def \Z {\mathbb{Z}}

\def \cala {\mathcal{A}}
\def \cale {\mathcal{E}}
\def \calb {\mathcal{B}}
\def \calq {\mathcal{Q}}
\def \calp {\mathcal{P}}
\def \cals {\mathcal{S}}
\def \calx {\mathcal{X}}
\def \caly {\mathcal{Y}}
\def \calg {\mathcal{G}}
\def \cald {\mathcal{D}}
\def \caln {\mathcal{N}}
\def \calr {\mathcal{R}}
\def \calt {\mathcal{T}}
\def \calm {\mathcal{M}}
\def \calw {\mathcal{W}}
\def \calc {\mathcal{C}}
\def \calv {\mathcal{V}}
\def \calf {\mathcal{F}}
\def \calk {\mathcal{K}}
\def \call {\mathcal{L}}
\def \calu {\mathcal{U}}
\def \calo {\mathcal{O}}
\def \calh {\mathcal{H}}
\def \cali {\mathcal{I}}
\def \calj {\mathcal{J}}

\def \bcup {\bigcup}

% set theory

\def \zfcc {\textbf{ZFC}^-}
\def \BGC {\textbf{BGC}}
\def \BG {\textbf{BG}}
\def \ac  {\textbf{AC}}
\def \gl  {\textbf{L }}
\def \gll {\textbf{L}}
\newcommand{\zfm}{$\textbf{ZF}^-$}

\def \ZFm {\text{ZF}^-}
\def \ZFCm {\text{ZFC}^-}
\DeclareMathOperator{\WF}{WF}
\DeclareMathOperator{\On}{On}
\def \on {\textbf{On }}
\def \cm {\textbf{M }}
\def \cn {\textbf{N }}
\def \cv {\textbf{V }}
\def \zc {\textbf{ZC }}
\def \zcm {\textbf{ZC}}
\def \zff {\textbf{ZF}}
\def \wfm {\textbf{WF}}
\def \onm {\textbf{On}}
\def \cmm {\textbf{M}}
\def \cnm {\textbf{N}}
\def \cvm {\textbf{V}}

\renewcommand{\restriction}{\mathord{\upharpoonright}}
%% another restriction
\newcommand\restr[2]{{% we make the whole thing an ordinary symbol
  \left.\kern-\nulldelimiterspace % automatically resize the bar with \right
  #1 % the function
  \vphantom{\big|} % pretend it's a little taller at normal size
  \right|_{#2} % this is the delimiter
  }}

\def \pred {\text{pred}}

\def \rank {\text{rank}}
\def \Con {\text{Con}}
\def \deff {\text{Def}}


\def \uin {\underline{\in}}
\def \oin {\overline{\in}}
\def \uR {\underline{R}}
\def \oR {\overline{R}}
\def \uP {\underline{P}}
\def \oP {\overline{P}}

\def \dsum {\displaystyle\sum}

\def \Ra {\Rightarrow}

\def \e {\enspace}

\def \sgn {\operatorname{sgn}}
\def \gen {\operatorname{gen}}
\def \Hom {\operatorname{Hom}}
\def \hom {\operatorname{hom}}
\def \Sub {\operatorname{Sub}}

\def \supp {\operatorname{supp}}

\def \epiarrow {\twoheadarrow}
\def \monoarrow {\rightarrowtail}
\def \rrarrow {\rightrightarrows}

% \def \minus {\text{-}}
% \newcommand{\minus}{\scalebox{0.75}[1.0]{$-$}}
% \DeclareUnicodeCharacter{002D}{\minus}


\def \tril {\triangleleft}

\def \ISigma {\text{I}\Sigma}
\def \IDelta {\text{I}\Delta}
\def \IPi {\text{I}\Pi}
\def \ACF {\textsf{ACF}}
\def \pCF {\textit{p}\text{CF}}
\def \ACVF {\textsf{ACVF}}
\def \HLR {\textsf{HLR}}
\def \OAG {\textsf{OAG}}
\def \RCF {\textsf{RCF}}
\DeclareMathOperator{\GL}{GL}
\DeclareMathOperator{\PGL}{PGL}
\DeclareMathOperator{\SL}{SL}
\DeclareMathOperator{\Inv}{Inv}
\DeclareMathOperator{\res}{res}
\DeclareMathOperator{\Sym}{Sym}
%\DeclareMathOperator{\char}{char}
\def \equal {=}

\def \degree {\text{degree}}
\def \app {\text{App}}
\def \FV {\text{FV}}
\def \conv {\text{conv}}
\def \cont {\text{cont}}
\DeclareMathOperator{\cl}{\text{cl}}
\DeclareMathOperator{\trcl}{\text{trcl}}
\DeclareMathOperator{\sg}{sg}
\DeclareMathOperator{\trdeg}{trdeg}
\def \Ord {\text{Ord}}

\DeclareMathOperator{\cf}{cf}
\DeclareMathOperator{\zfc}{ZFC}

%\DeclareMathOperator{\Th}{Th}
%\def \th {\text{Th}}
% \newcommand{\th}{\text{Th}}
\DeclareMathOperator{\type}{type}
\DeclareMathOperator{\zf}{\textbf{ZF}}
\def \fa {\mathfrak{a}}
\def \fb {\mathfrak{b}}
\def \fc {\mathfrak{c}}
\def \fd {\mathfrak{d}}
\def \fe {\mathfrak{e}}
\def \ff {\mathfrak{f}}
\def \fg {\mathfrak{g}}
\def \fh {\mathfrak{h}}
%\def \fi {\mathfrak{i}}
\def \fj {\mathfrak{j}}
\def \fk {\mathfrak{k}}
\def \fl {\mathfrak{l}}
\def \fm {\mathfrak{m}}
\def \fn {\mathfrak{n}}
\def \fo {\mathfrak{o}}
\def \fp {\mathfrak{p}}
\def \fq {\mathfrak{q}}
\def \fr {\mathfrak{r}}
\def \fs {\mathfrak{s}}
\def \ft {\mathfrak{t}}
\def \fu {\mathfrak{u}}
\def \fv {\mathfrak{v}}
\def \fw {\mathfrak{w}}
\def \fx {\mathfrak{x}}
\def \fy {\mathfrak{y}}
\def \fz {\mathfrak{z}}
\def \fA {\mathfrak{A}}
\def \fB {\mathfrak{B}}
\def \fC {\mathfrak{C}}
\def \fD {\mathfrak{D}}
\def \fE {\mathfrak{E}}
\def \fF {\mathfrak{F}}
\def \fG {\mathfrak{G}}
\def \fH {\mathfrak{H}}
\def \fI {\mathfrak{I}}
\def \fJ {\mathfrak{J}}
\def \fK {\mathfrak{K}}
\def \fL {\mathfrak{L}}
\def \fM {\mathfrak{M}}
\def \fN {\mathfrak{N}}
\def \fO {\mathfrak{O}}
\def \fP {\mathfrak{P}}
\def \fQ {\mathfrak{Q}}
\def \fR {\mathfrak{R}}
\def \fS {\mathfrak{S}}
\def \fT {\mathfrak{T}}
\def \fU {\mathfrak{U}}
\def \fV {\mathfrak{V}}
\def \fW {\mathfrak{W}}
\def \fX {\mathfrak{X}}
\def \fY {\mathfrak{Y}}
\def \fZ {\mathfrak{Z}}

\def \sfA {\textsf{A}}
\def \sfB {\textsf{B}}
\def \sfC {\textsf{C}}
\def \sfD {\textsf{D}}
\def \sfE {\textsf{E}}
\def \sfF {\textsf{F}}
\def \sfG {\textsf{G}}
\def \sfH {\textsf{H}}
\def \sfI {\textsf{I}}
\def \sfJ {\textsf{J}}
\def \sfK {\textsf{K}}
\def \sfL {\textsf{L}}
\def \sfM {\textsf{M}}
\def \sfN {\textsf{N}}
\def \sfO {\textsf{O}}
\def \sfP {\textsf{P}}
\def \sfQ {\textsf{Q}}
\def \sfR {\textsf{R}}
\def \sfS {\textsf{S}}
\def \sfT {\textsf{T}}
\def \sfU {\textsf{U}}
\def \sfV {\textsf{V}}
\def \sfW {\textsf{W}}
\def \sfX {\textsf{X}}
\def \sfY {\textsf{Y}}
\def \sfZ {\textsf{Z}}
\def \sfa {\textsf{a}}
\def \sfb {\textsf{b}}
\def \sfc {\textsf{c}}
\def \sfd {\textsf{d}}
\def \sfe {\textsf{e}}
\def \sff {\textsf{f}}
\def \sfg {\textsf{g}}
\def \sfh {\textsf{h}}
\def \sfi {\textsf{i}}
\def \sfj {\textsf{j}}
\def \sfk {\textsf{k}}
\def \sfl {\textsf{l}}
\def \sfm {\textsf{m}}
\def \sfn {\textsf{n}}
\def \sfo {\textsf{o}}
\def \sfp {\textsf{p}}
\def \sfq {\textsf{q}}
\def \sfr {\textsf{r}}
\def \sfs {\textsf{s}}
\def \sft {\textsf{t}}
\def \sfu {\textsf{u}}
\def \sfv {\textsf{v}}
\def \sfw {\textsf{w}}
\def \sfx {\textsf{x}}
\def \sfy {\textsf{y}}
\def \sfz {\textsf{z}}

\def \ttA {\texttt{A}}
\def \ttB {\texttt{B}}
\def \ttC {\texttt{C}}
\def \ttD {\texttt{D}}
\def \ttE {\texttt{E}}
\def \ttF {\texttt{F}}
\def \ttG {\texttt{G}}
\def \ttH {\texttt{H}}
\def \ttI {\texttt{I}}
\def \ttJ {\texttt{J}}
\def \ttK {\texttt{K}}
\def \ttL {\texttt{L}}
\def \ttM {\texttt{M}}
\def \ttN {\texttt{N}}
\def \ttO {\texttt{O}}
\def \ttP {\texttt{P}}
\def \ttQ {\texttt{Q}}
\def \ttR {\texttt{R}}
\def \ttS {\texttt{S}}
\def \ttT {\texttt{T}}
\def \ttU {\texttt{U}}
\def \ttV {\texttt{V}}
\def \ttW {\texttt{W}}
\def \ttX {\texttt{X}}
\def \ttY {\texttt{Y}}
\def \ttZ {\texttt{Z}}
\def \tta {\texttt{a}}
\def \ttb {\texttt{b}}
\def \ttc {\texttt{c}}
\def \ttd {\texttt{d}}
\def \tte {\texttt{e}}
\def \ttf {\texttt{f}}
\def \ttg {\texttt{g}}
\def \tth {\texttt{h}}
\def \tti {\texttt{i}}
\def \ttj {\texttt{j}}
\def \ttk {\texttt{k}}
\def \ttl {\texttt{l}}
\def \ttm {\texttt{m}}
\def \ttn {\texttt{n}}
\def \tto {\texttt{o}}
\def \ttp {\texttt{p}}
\def \ttq {\texttt{q}}
\def \ttr {\texttt{r}}
\def \tts {\texttt{s}}
\def \ttt {\texttt{t}}
\def \ttu {\texttt{u}}
\def \ttv {\texttt{v}}
\def \ttw {\texttt{w}}
\def \ttx {\texttt{x}}
\def \tty {\texttt{y}}
\def \ttz {\texttt{z}}

\def \bara {\bbar{a}}
\def \barb {\bbar{b}}
\def \barc {\bbar{c}}
\def \bard {\bbar{d}}
\def \bare {\bbar{e}}
\def \barf {\bbar{f}}
\def \barg {\bbar{g}}
\def \barh {\bbar{h}}
\def \bari {\bbar{i}}
\def \barj {\bbar{j}}
\def \bark {\bbar{k}}
\def \barl {\bbar{l}}
\def \barm {\bbar{m}}
\def \barn {\bbar{n}}
\def \baro {\bbar{o}}
\def \barp {\bbar{p}}
\def \barq {\bbar{q}}
\def \barr {\bbar{r}}
\def \bars {\bbar{s}}
\def \bart {\bbar{t}}
\def \baru {\bbar{u}}
\def \barv {\bbar{v}}
\def \barw {\bbar{w}}
\def \barx {\bbar{x}}
\def \bary {\bbar{y}}
\def \barz {\bbar{z}}
\def \barA {\bbar{A}}
\def \barB {\bbar{B}}
\def \barC {\bbar{C}}
\def \barD {\bbar{D}}
\def \barE {\bbar{E}}
\def \barF {\bbar{F}}
\def \barG {\bbar{G}}
\def \barH {\bbar{H}}
\def \barI {\bbar{I}}
\def \barJ {\bbar{J}}
\def \barK {\bbar{K}}
\def \barL {\bbar{L}}
\def \barM {\bbar{M}}
\def \barN {\bbar{N}}
\def \barO {\bbar{O}}
\def \barP {\bbar{P}}
\def \barQ {\bbar{Q}}
\def \barR {\bbar{R}}
\def \barS {\bbar{S}}
\def \barT {\bbar{T}}
\def \barU {\bbar{U}}
\def \barVV {\bbar{V}}
\def \barW {\bbar{W}}
\def \barX {\bbar{X}}
\def \barY {\bbar{Y}}
\def \barZ {\bbar{Z}}

\def \baralpha {\bbar{\alpha}}
\def \bartau {\bbar{\tau}}
\def \barsigma {\bbar{\sigma}}
\def \barzeta {\bbar{\zeta}}

\def \hata {\hat{a}}
\def \hatb {\hat{b}}
\def \hatc {\hat{c}}
\def \hatd {\hat{d}}
\def \hate {\hat{e}}
\def \hatf {\hat{f}}
\def \hatg {\hat{g}}
\def \hath {\hat{h}}
\def \hati {\hat{i}}
\def \hatj {\hat{j}}
\def \hatk {\hat{k}}
\def \hatl {\hat{l}}
\def \hatm {\hat{m}}
\def \hatn {\hat{n}}
\def \hato {\hat{o}}
\def \hatp {\hat{p}}
\def \hatq {\hat{q}}
\def \hatr {\hat{r}}
\def \hats {\hat{s}}
\def \hatt {\hat{t}}
\def \hatu {\hat{u}}
\def \hatv {\hat{v}}
\def \hatw {\hat{w}}
\def \hatx {\hat{x}}
\def \haty {\hat{y}}
\def \hatz {\hat{z}}
\def \hatA {\hat{A}}
\def \hatB {\hat{B}}
\def \hatC {\hat{C}}
\def \hatD {\hat{D}}
\def \hatE {\hat{E}}
\def \hatF {\hat{F}}
\def \hatG {\hat{G}}
\def \hatH {\hat{H}}
\def \hatI {\hat{I}}
\def \hatJ {\hat{J}}
\def \hatK {\hat{K}}
\def \hatL {\hat{L}}
\def \hatM {\hat{M}}
\def \hatN {\hat{N}}
\def \hatO {\hat{O}}
\def \hatP {\hat{P}}
\def \hatQ {\hat{Q}}
\def \hatR {\hat{R}}
\def \hatS {\hat{S}}
\def \hatT {\hat{T}}
\def \hatU {\hat{U}}
\def \hatVV {\hat{V}}
\def \hatW {\hat{W}}
\def \hatX {\hat{X}}
\def \hatY {\hat{Y}}
\def \hatZ {\hat{Z}}

\def \hatphi {\hat{\phi}}

\def \barfM {\bbar{\fM}}
\def \barfN {\bbar{\fN}}

\def \tila {\tilde{a}}
\def \tilb {\tilde{b}}
\def \tilc {\tilde{c}}
\def \tild {\tilde{d}}
\def \tile {\tilde{e}}
\def \tilf {\tilde{f}}
\def \tilg {\tilde{g}}
\def \tilh {\tilde{h}}
\def \tili {\tilde{i}}
\def \tilj {\tilde{j}}
\def \tilk {\tilde{k}}
\def \till {\tilde{l}}
\def \tilm {\tilde{m}}
\def \tiln {\tilde{n}}
\def \tilo {\tilde{o}}
\def \tilp {\tilde{p}}
\def \tilq {\tilde{q}}
\def \tilr {\tilde{r}}
\def \tils {\tilde{s}}
\def \tilt {\tilde{t}}
\def \tilu {\tilde{u}}
\def \tilv {\tilde{v}}
\def \tilw {\tilde{w}}
\def \tilx {\tilde{x}}
\def \tily {\tilde{y}}
\def \tilz {\tilde{z}}
\def \tilA {\tilde{A}}
\def \tilB {\tilde{B}}
\def \tilC {\tilde{C}}
\def \tilD {\tilde{D}}
\def \tilE {\tilde{E}}
\def \tilF {\tilde{F}}
\def \tilG {\tilde{G}}
\def \tilH {\tilde{H}}
\def \tilI {\tilde{I}}
\def \tilJ {\tilde{J}}
\def \tilK {\tilde{K}}
\def \tilL {\tilde{L}}
\def \tilM {\tilde{M}}
\def \tilN {\tilde{N}}
\def \tilO {\tilde{O}}
\def \tilP {\tilde{P}}
\def \tilQ {\tilde{Q}}
\def \tilR {\tilde{R}}
\def \tilS {\tilde{S}}
\def \tilT {\tilde{T}}
\def \tilU {\tilde{U}}
\def \tilVV {\tilde{V}}
\def \tilW {\tilde{W}}
\def \tilX {\tilde{X}}
\def \tilY {\tilde{Y}}
\def \tilZ {\tilde{Z}}

\def \tilalpha {\tilde{\alpha}}
\def \tilPhi {\tilde{\Phi}}

\def \barnu {\bar{\nu}}
\def \barrho {\bar{\rho}}
%\DeclareMathOperator{\ker}{ker}
\DeclareMathOperator{\im}{im}

\DeclareMathOperator{\Inn}{Inn}
\DeclareMathOperator{\rel}{rel}
\def \dote {\stackrel{\cdot}=}
%\DeclareMathOperator{\AC}{\textbf{AC}}
\DeclareMathOperator{\cod}{cod}
\DeclareMathOperator{\dom}{dom}
\DeclareMathOperator{\card}{card}
\DeclareMathOperator{\ran}{ran}
\DeclareMathOperator{\textd}{d}
\DeclareMathOperator{\td}{d}
\DeclareMathOperator{\id}{id}
\DeclareMathOperator{\LT}{LT}
\DeclareMathOperator{\Mat}{Mat}
\DeclareMathOperator{\Eq}{Eq}
\DeclareMathOperator{\irr}{irr}
\DeclareMathOperator{\Fr}{Fr}
\DeclareMathOperator{\Gal}{Gal}
\DeclareMathOperator{\lcm}{lcm}
\DeclareMathOperator{\alg}{\text{alg}}
\DeclareMathOperator{\Th}{Th}
%\DeclareMathOperator{\deg}{deg}


% \varprod
\DeclareSymbolFont{largesymbolsA}{U}{txexa}{m}{n}
\DeclareMathSymbol{\varprod}{\mathop}{largesymbolsA}{16}
% \DeclareMathSymbol{\tonm}{\boldsymbol{\to}\textbf{Nm}}
\def \tonm {\bto\textbf{Nm}}
\def \tohm {\bto\textbf{Hm}}

% Category theory
\DeclareMathOperator{\ob}{ob}
\DeclareMathOperator{\Ab}{\textbf{Ab}}
\DeclareMathOperator{\Alg}{\textbf{Alg}}
\DeclareMathOperator{\Rng}{\textbf{Rng}}
\DeclareMathOperator{\Sets}{\textbf{Sets}}
\DeclareMathOperator{\Set}{\textbf{Set}}
\DeclareMathOperator{\Grp}{\textbf{Grp}}
\DeclareMathOperator{\Met}{\textbf{Met}}
\DeclareMathOperator{\BA}{\textbf{BA}}
\DeclareMathOperator{\Mon}{\textbf{Mon}}
\DeclareMathOperator{\Top}{\textbf{Top}}
\DeclareMathOperator{\hTop}{\textbf{hTop}}
\DeclareMathOperator{\HTop}{\textbf{HTop}}
\DeclareMathOperator{\Aut}{\text{Aut}}
\DeclareMathOperator{\RMod}{R-\textbf{Mod}}
\DeclareMathOperator{\RAlg}{R-\textbf{Alg}}
\DeclareMathOperator{\LF}{LF}
\DeclareMathOperator{\op}{op}
\DeclareMathOperator{\Rings}{\textbf{Rings}}
\DeclareMathOperator{\Ring}{\textbf{Ring}}
\DeclareMathOperator{\Groups}{\textbf{Groups}}
\DeclareMathOperator{\Group}{\textbf{Group}}
\DeclareMathOperator{\ev}{ev}
% Algebraic Topology
\DeclareMathOperator{\obj}{obj}
\DeclareMathOperator{\Spec}{Spec}
\DeclareMathOperator{\spec}{spec}
% Model theory
\DeclareMathOperator*{\ind}{\raise0.2ex\hbox{\ooalign{\hidewidth$\vert$\hidewidth\cr\raise-0.9ex\hbox{$\smile$}}}}
\def\nind{\cancel{\ind}}
\DeclareMathOperator{\acl}{acl}
\DeclareMathOperator{\tspan}{span}
\DeclareMathOperator{\acleq}{acl^{\eq}}
\DeclareMathOperator{\Av}{Av}
\DeclareMathOperator{\ded}{ded}
\DeclareMathOperator{\EM}{EM}
\DeclareMathOperator{\dcl}{dcl}
\DeclareMathOperator{\Ext}{Ext}
\DeclareMathOperator{\eq}{eq}
\DeclareMathOperator{\ER}{ER}
\DeclareMathOperator{\tp}{tp}
\DeclareMathOperator{\stp}{stp}
\DeclareMathOperator{\qftp}{qftp}
\DeclareMathOperator{\Diag}{Diag}
\DeclareMathOperator{\MD}{MD}
\DeclareMathOperator{\MR}{MR}
\DeclareMathOperator{\RM}{RM}
\DeclareMathOperator{\el}{el}
\DeclareMathOperator{\depth}{depth}
\DeclareMathOperator{\ZFC}{ZFC}
\DeclareMathOperator{\GCH}{GCH}
\DeclareMathOperator{\Inf}{Inf}
\DeclareMathOperator{\Pow}{Pow}
\DeclareMathOperator{\ZF}{ZF}
\DeclareMathOperator{\CH}{CH}
\def \FO {\text{FO}}
\DeclareMathOperator{\fin}{fin}
\DeclareMathOperator{\qr}{qr}
\DeclareMathOperator{\Mod}{Mod}
\DeclareMathOperator{\Def}{Def}
\DeclareMathOperator{\TC}{TC}
\DeclareMathOperator{\KH}{KH}
\DeclareMathOperator{\Part}{Part}
\DeclareMathOperator{\Infset}{\textsf{Infset}}
\DeclareMathOperator{\DLO}{\textsf{DLO}}
\DeclareMathOperator{\PA}{\textsf{PA}}
\DeclareMathOperator{\DAG}{\textsf{DAG}}
\DeclareMathOperator{\ODAG}{\textsf{ODAG}}
\DeclareMathOperator{\sfMod}{\textsf{Mod}}
\DeclareMathOperator{\AbG}{\textsf{AbG}}
\DeclareMathOperator{\sfACF}{\textsf{ACF}}
\DeclareMathOperator{\DCF}{\textsf{DCF}}
% Computability Theorem
\DeclareMathOperator{\Tot}{Tot}
\DeclareMathOperator{\graph}{graph}
\DeclareMathOperator{\Fin}{Fin}
\DeclareMathOperator{\Cof}{Cof}
\DeclareMathOperator{\lh}{lh}
% Commutative Algebra
\DeclareMathOperator{\ord}{ord}
\DeclareMathOperator{\Idem}{Idem}
\DeclareMathOperator{\zdiv}{z.div}
\DeclareMathOperator{\Frac}{Frac}
\DeclareMathOperator{\rad}{rad}
\DeclareMathOperator{\nil}{nil}
\DeclareMathOperator{\Ann}{Ann}
\DeclareMathOperator{\End}{End}
\DeclareMathOperator{\coim}{coim}
\DeclareMathOperator{\coker}{coker}
\DeclareMathOperator{\Bil}{Bil}
\DeclareMathOperator{\Tril}{Tril}
\DeclareMathOperator{\tchar}{char}
\DeclareMathOperator{\tbd}{bd}

% Topology
\DeclareMathOperator{\diam}{diam}
\newcommand{\interior}[1]{%
  {\kern0pt#1}^{\mathrm{o}}%
}

\DeclareMathOperator*{\bigdoublewedge}{\bigwedge\mkern-15mu\bigwedge}
\DeclareMathOperator*{\bigdoublevee}{\bigvee\mkern-15mu\bigvee}

% \makeatletter
% \newcommand{\vect}[1]{%
%   \vbox{\m@th \ialign {##\crcr
%   \vectfill\crcr\noalign{\kern-\p@ \nointerlineskip}
%   $\hfil\displaystyle{#1}\hfil$\crcr}}}
% \def\vectfill{%
%   $\m@th\smash-\mkern-7mu%
%   \cleaders\hbox{$\mkern-2mu\smash-\mkern-2mu$}\hfill
%   \mkern-7mu\raisebox{-3.81pt}[\p@][\p@]{$\mathord\mathchar"017E$}$}

% \newcommand{\amsvect}{%
%   \mathpalette {\overarrow@\vectfill@}}
% \def\vectfill@{\arrowfill@\relbar\relbar{\raisebox{-3.81pt}[\p@][\p@]{$\mathord\mathchar"017E$}}}

% \newcommand{\amsvectb}{%
% \newcommand{\vect}{%
%   \mathpalette {\overarrow@\vectfillb@}}
% \newcommand{\vecbar}{%
%   \scalebox{0.8}{$\relbar$}}
% \def\vectfillb@{\arrowfill@\vecbar\vecbar{\raisebox{-4.35pt}[\p@][\p@]{$\mathord\mathchar"017E$}}}
% \makeatother
% \bigtimes

\DeclareFontFamily{U}{mathx}{\hyphenchar\font45}
\DeclareFontShape{U}{mathx}{m}{n}{
      <5> <6> <7> <8> <9> <10>
      <10.95> <12> <14.4> <17.28> <20.74> <24.88>
      mathx10
      }{}
\DeclareSymbolFont{mathx}{U}{mathx}{m}{n}
\DeclareMathSymbol{\bigtimes}{1}{mathx}{"91}
% \odiv
\DeclareFontFamily{U}{matha}{\hyphenchar\font45}
\DeclareFontShape{U}{matha}{m}{n}{
      <5> <6> <7> <8> <9> <10> gen * matha
      <10.95> matha10 <12> <14.4> <17.28> <20.74> <24.88> matha12
      }{}
\DeclareSymbolFont{matha}{U}{matha}{m}{n}
\DeclareMathSymbol{\odiv}         {2}{matha}{"63}


\newcommand\subsetsim{\mathrel{%
  \ooalign{\raise0.2ex\hbox{\scalebox{0.9}{$\subset$}}\cr\hidewidth\raise-0.85ex\hbox{\scalebox{0.9}{$\sim$}}\hidewidth\cr}}}
\newcommand\simsubset{\mathrel{%
  \ooalign{\raise-0.2ex\hbox{\scalebox{0.9}{$\subset$}}\cr\hidewidth\raise0.75ex\hbox{\scalebox{0.9}{$\sim$}}\hidewidth\cr}}}

\newcommand\simsubsetsim{\mathrel{%
  \ooalign{\raise0ex\hbox{\scalebox{0.8}{$\subset$}}\cr\hidewidth\raise1ex\hbox{\scalebox{0.75}{$\sim$}}\hidewidth\cr\raise-0.95ex\hbox{\scalebox{0.8}{$\sim$}}\cr\hidewidth}}}
\newcommand{\stcomp}[1]{{#1}^{\mathsf{c}}}

\setlength{\baselineskip}{0.5in}

\stackMath
\newcommand\yrightarrow[2][]{\mathrel{%
  \setbox2=\hbox{\stackon{\scriptstyle#1}{\scriptstyle#2}}%
  \stackunder[0pt]{%
    \xrightarrow{\makebox[\dimexpr\wd2\relax]{$\scriptstyle#2$}}%
  }{%
   \scriptstyle#1\,%
  }%
}}
\newcommand\yleftarrow[2][]{\mathrel{%
  \setbox2=\hbox{\stackon{\scriptstyle#1}{\scriptstyle#2}}%
  \stackunder[0pt]{%
    \xleftarrow{\makebox[\dimexpr\wd2\relax]{$\scriptstyle#2$}}%
  }{%
   \scriptstyle#1\,%
  }%
}}
\newcommand\yRightarrow[2][]{\mathrel{%
  \setbox2=\hbox{\stackon{\scriptstyle#1}{\scriptstyle#2}}%
  \stackunder[0pt]{%
    \xRightarrow{\makebox[\dimexpr\wd2\relax]{$\scriptstyle#2$}}%
  }{%
   \scriptstyle#1\,%
  }%
}}
\newcommand\yLeftarrow[2][]{\mathrel{%
  \setbox2=\hbox{\stackon{\scriptstyle#1}{\scriptstyle#2}}%
  \stackunder[0pt]{%
    \xLeftarrow{\makebox[\dimexpr\wd2\relax]{$\scriptstyle#2$}}%
  }{%
   \scriptstyle#1\,%
  }%
}}

\newcommand\altxrightarrow[2][0pt]{\mathrel{\ensurestackMath{\stackengine%
  {\dimexpr#1-7.5pt}{\xrightarrow{\phantom{#2}}}{\scriptstyle\!#2\,}%
  {O}{c}{F}{F}{S}}}}
\newcommand\altxleftarrow[2][0pt]{\mathrel{\ensurestackMath{\stackengine%
  {\dimexpr#1-7.5pt}{\xleftarrow{\phantom{#2}}}{\scriptstyle\!#2\,}%
  {O}{c}{F}{F}{S}}}}

\newenvironment{bsm}{% % short for 'bracketed small matrix'
  \left[ \begin{smallmatrix} }{%
  \end{smallmatrix} \right]}

\newenvironment{psm}{% % short for ' small matrix'
  \left( \begin{smallmatrix} }{%
  \end{smallmatrix} \right)}

\newcommand{\bbar}[1]{\mkern 1.5mu\overline{\mkern-1.5mu#1\mkern-1.5mu}\mkern 1.5mu}

\newcommand{\bigzero}{\mbox{\normalfont\Large\bfseries 0}}
\newcommand{\rvline}{\hspace*{-\arraycolsep}\vline\hspace*{-\arraycolsep}}

\font\zallman=Zallman at 40pt
\font\elzevier=Elzevier at 40pt

\newcommand\isoto{\stackrel{\textstyle\sim}{\smash{\longrightarrow}\rule{0pt}{0.4ex}}}
\newcommand\embto{\stackrel{\textstyle\prec}{\smash{\longrightarrow}\rule{0pt}{0.4ex}}}

% from http://www.actual.world/resources/tex/doc/TikZ.pdf

\tikzset{
modal/.style={>=stealth’,shorten >=1pt,shorten <=1pt,auto,node distance=1.5cm,
semithick},
world/.style={circle,draw,minimum size=0.5cm,fill=gray!15},
point/.style={circle,draw,inner sep=0.5mm,fill=black},
reflexive above/.style={->,loop,looseness=7,in=120,out=60},
reflexive below/.style={->,loop,looseness=7,in=240,out=300},
reflexive left/.style={->,loop,looseness=7,in=150,out=210},
reflexive right/.style={->,loop,looseness=7,in=30,out=330}
}


\makeatletter
\newcommand*{\doublerightarrow}[2]{\mathrel{
  \settowidth{\@tempdima}{$\scriptstyle#1$}
  \settowidth{\@tempdimb}{$\scriptstyle#2$}
  \ifdim\@tempdimb>\@tempdima \@tempdima=\@tempdimb\fi
  \mathop{\vcenter{
    \offinterlineskip\ialign{\hbox to\dimexpr\@tempdima+1em{##}\cr
    \rightarrowfill\cr\noalign{\kern.5ex}
    \rightarrowfill\cr}}}\limits^{\!#1}_{\!#2}}}
\newcommand*{\triplerightarrow}[1]{\mathrel{
  \settowidth{\@tempdima}{$\scriptstyle#1$}
  \mathop{\vcenter{
    \offinterlineskip\ialign{\hbox to\dimexpr\@tempdima+1em{##}\cr
    \rightarrowfill\cr\noalign{\kern.5ex}
    \rightarrowfill\cr\noalign{\kern.5ex}
    \rightarrowfill\cr}}}\limits^{\!#1}}}
\makeatother

% $A\doublerightarrow{a}{bcdefgh}B$

% $A\triplerightarrow{d_0,d_1,d_2}B$

\def \uhr {\upharpoonright}
\def \rhu {\rightharpoonup}
\def \uhl {\upharpoonleft}


\newcommand{\floor}[1]{\lfloor #1 \rfloor}
\newcommand{\ceil}[1]{\lceil #1 \rceil}
\newcommand{\lcorner}[1]{\llcorner #1 \lrcorner}
\newcommand{\llb}[1]{\llbracket #1 \rrbracket}
\newcommand{\ucorner}[1]{\ulcorner #1 \urcorner}
\newcommand{\emoji}[1]{{\DejaSans #1}}
\newcommand{\vprec}{\rotatebox[origin=c]{-90}{$\prec$}}

\newcommand{\nat}[6][large]{%
  \begin{tikzcd}[ampersand replacement = \&, column sep=#1]
    #2\ar[bend left=40,""{name=U}]{r}{#4}\ar[bend right=40,',""{name=D}]{r}{#5}\& #3
          \ar[shorten <=10pt,shorten >=10pt,Rightarrow,from=U,to=D]{d}{~#6}
    \end{tikzcd}
}


\providecommand\rightarrowRHD{\relbar\joinrel\mathrel\RHD}
\providecommand\rightarrowrhd{\relbar\joinrel\mathrel\rhd}
\providecommand\longrightarrowRHD{\relbar\joinrel\relbar\joinrel\mathrel\RHD}
\providecommand\longrightarrowrhd{\relbar\joinrel\relbar\joinrel\mathrel\rhd}
\def \lrarhd {\longrightarrowrhd}


\makeatletter
\providecommand*\xrightarrowRHD[2][]{\ext@arrow 0055{\arrowfill@\relbar\relbar\longrightarrowRHD}{#1}{#2}}
\providecommand*\xrightarrowrhd[2][]{\ext@arrow 0055{\arrowfill@\relbar\relbar\longrightarrowrhd}{#1}{#2}}
\makeatother

\newcommand{\metalambda}{%
  \mathop{%
    \rlap{$\lambda$}%
    \mkern3mu
    \raisebox{0ex}{$\lambda$}%
  }%
}

%% https://tex.stackexchange.com/questions/15119/draw-horizontal-line-left-and-right-of-some-text-a-single-line
\newcommand*\ruleline[1]{\par\noindent\raisebox{.8ex}{\makebox[\linewidth]{\hrulefill\hspace{1ex}\raisebox{-.8ex}{#1}\hspace{1ex}\hrulefill}}}

% https://www.dickimaw-books.com/latex/novices/html/newenv.html
\newenvironment{Block}[1]% environment name
{% begin code
  % https://tex.stackexchange.com/questions/19579/horizontal-line-spanning-the-entire-document-in-latex
  \noindent\textcolor[RGB]{128,128,128}{\rule{\linewidth}{1pt}}
  \par\noindent
  {\Large\textbf{#1}}%
  \bigskip\par\noindent\ignorespaces
}%
{% end code
  \par\noindent
  \textcolor[RGB]{128,128,128}{\rule{\linewidth}{1pt}}
  \ignorespacesafterend
}

\mathchardef\mhyphen="2D % Define a "math hyphen"

\def \QQ {\quad}
\def \QW {​\quad}

\usepackage[UTF8]{ctex}
\DeclareMathOperator{\Char}{Char}
\makeindex
\author{wu}
\date{\today}
\title{Valuation Field}
\hypersetup{
 pdfauthor={wu},
 pdftitle={Valuation Field},
 pdfkeywords={},
 pdfsubject={},
 pdfcreator={Emacs 28.0.92 (Org mode 9.6)}, 
 pdflang={English}}
\begin{document}

\maketitle
\tableofcontents

\section{环与理想}
\label{sec:org1d35c6d}
\subsection{介绍}
\label{sec:org2179b58}
\begin{definition}[]
称\(A\)为 \textbf{局部环} ,如果\(A\)只有一个极大理想 \(I\),称\(k=A/I\)为\(A\)的 \textbf{剩余域} (residue field)
\end{definition}

\begin{proposition}[]
\begin{enumerate}
\item 设\(A\)为环,\(I\subsetneq A\)为理想,若每个\(x\in A\setminus I\)均是单位元则\(A\)是局部环,\(I\)是极大
理想
\item 若\(A\)为环,\(I\subseteq A\)为极大理想,若\(\forall a\in I\),有\(1+a\)均是单位元,则\(A\)是局部环
\end{enumerate}
\end{proposition}


\subsection{分式化}
\label{sec:org967e449}
\begin{definition}[]
设\(A\)是一个整环,令\(A^\times=A\setminus\{0\}\),在\(A\times A^\times\)上定义关系\(\sim\)为
\begin{equation*}
(a,s)\sim(b,t)\Leftrightarrow at-bs=0
\end{equation*}
\end{definition}

\begin{definition}[]
称\(S\subseteq A\)为 \textbf{乘法子集} ,如果\(1\in S\)且\(a,b\in S\Rightarrow ab\in S\)
\end{definition}

\begin{definition}[]
设\(S\subseteq A\)是乘法子集,定义\(A\times S\)上的等价关系\(\sim\)为
\begin{equation*}
(a,s)\sim(b,t)\Leftrightarrow\exists u\in S(u(at-bs)=0)
\end{equation*}
将\((a,s)\)的等价类记作\(\frac{a}{s}\),定义
\begin{equation*}
\frac{a}{s}+\frac{b}{t}=\frac{at+bs}{st},\quad\frac{a}{s}\frac{b}{t}=\frac{ab}{st}
\end{equation*}
则\(A\times S/\sim\)是一个环,记作\(S^{-1}A\)
\end{definition}

\begin{remark}
\begin{itemize}
\item \(\forall x\in A\),\(\frac{xa}{xs}=\frac{a}{s}\)
\item 若\(S\)有零因子,则\(S^{-1}A=0\)平凡
\item \(A\to S^{-1}A\), \(a\mapsto\frac{a}{1}\)是同态
\item 若\(A\)是整环,\(S=A^\times\),则\(S^{-1}=\Frac(A)\)
\end{itemize}
\end{remark}

\begin{examplle}[]
若\(\fp\)是素理想,\(S=A\setminus \fp\)是乘法子集
\begin{itemize}
\item 令\(A_{\fp}=S^{-1}A\)
\item 令\(\fm=\{\frac{a}{s}\mid a\in\fp,s\notin\fp\}=pA_{\fp}=\fp S^{-1}\),则\(A_{\fp}\)是局部环,\(\fm\)是\(A_{\fp}\)的极大理想
\end{itemize}
\end{examplle}



\subsection{多项式环}
\label{sec:org0b05355}
设\(A\)是一个环,则多项式环\(A[X]\)的元素都形如
\begin{equation*}
\sum_{i=0}^na_ix^i,\quad a_i\in A,i\in\N
\end{equation*}

\begin{definition}[]
设\(A\)是环,\(a\in A\) \textbf{不可约} 如果\(a\neq 0\)不是单位元且\(\forall b,c\in A(a=bc\Rightarrow) b\)或\(c\)为单位元

一个整环\(A\)是 \textbf{唯一因子分解环} ,如果\(\forall a\in A\),存在不可约元\(b_1,\dots,b_n\in A\)使得\(a=b_1\dotsb_n\)
并且若存在不可约元\(c_1,\dots,c_m\)使得\(a=c_1\dots c_m\)则\(m=n\),则\(\forall i<n\exists j<n(b_i=u_{ij}c_j)\),其
中\(u_{ij}\)是单位元
\end{definition}

\begin{proposition}[]
若\(A\)是唯一因子分解环,则\(A[X]\)也是
\end{proposition}

\begin{corollary}[]
若\(k\)是域,则\(k[X_1,\dots,X_n]\)是唯一因子分解环
\end{corollary}

\begin{corollary}[]
\(k\)是域,\(f\in k[X_1,\dots,X_n]\),则\((f)\)是素理想\(\Leftrightarrow f\)不可约
\end{corollary}

\begin{proof}
\(\Rightarrow\):\(k[X_1,\dots,X_n]/(f)\)是整环,如果\(f\)可约,则\(f=gh\),其中\(g,h\in k[X_1,\dots,X_n]\)且不是单位
元,于是\(g+(f),h+(f)\)非零,而\((g+(f))(h+(f))=0+(f)\),矛盾

\(\Leftarrow\):对于任意\(g,h,p\in k[X_1,\dots,X_n]\),若\(gh=fp\),因为\(k[X_1,\dots,X_n]\)是唯一因子分解环,
于是\(f\)整除\(g\)或者\(f\)整除\(h\)
\end{proof}
\section{局部环}
\label{sec:org34de702}
一个环是局部环当且仅当所有非单位元构成一个理想。等价地,一个环是局部环当且仅当所有非单位元构成一个理想。

在环的语言\(\call_{ring}=\{+,\times,0,1\}\)中局部环可以公理为
\begin{enumerate}
\item \(R\)是环。
\item 所有的非单位元构成一个集合\(\fm\)是理想,即\(\fm\)关于``+''封闭,关于``\(\times\)''吸收。
\end{enumerate}


但是非单位元关于``\(\times\)''总是吸收的,故而(2)可以改为
\begin{enumerate}
\setcounter{enumi}{1}
\item 所有非单位元关于“+”封闭,即\(\fm\)是一个群。
\end{enumerate}

\begin{remark}
\begin{itemize}
\item \(0\in\R\)出解析函数的函数芽的环\(A\)是局部环
\item 一个函数\(f\)在\(0\in\R\)处解析\(\Leftrightarrow\)存在开邻域\(U\ni 0\)使得\(f\)在\(U\)上是个幂级数,即
\(f\uhr_U=\sum_{n=0}^\infty a_nx^n\),其中\(a_n\in\R\)。
\item 显然,\(\sum a_nx^n\sim\sum b_nx^n\Leftrightarrow\forall n(a_n=b_n)\),故而
\begin{equation*}
A=\{f\mid f\text{是幂级数且收敛半径}>0\}
\end{equation*}
\item \(\fm=x A=\{xf\mid f\in A\}\)是唯一的极大理想,其中极大是因为\(A/\fm\cong\R\)。
\end{itemize}
\end{remark}

\begin{examplle}[]
设\(R\)是一个环,称\(\sum_{n=0}^\infty r_nx^n\) (\(r_n\in R\))的元素为\(R\)上的形式幂级数,令\(R[[x]]\)为\(R\)
上所有形式幂级数构成的集合,定义
\begin{enumerate}
\item \(\sum r_nx^n+\sum s_nx^n=\sum(r_n+s_n)x^n\)
\item \(\sum r_nx^n\sum s_nx^n=\sum_n(\sum_{i+j=n}r_is_j)x^n\)
\end{enumerate}


则\((R[[x]],+,\times,0_R,1_R)\)是一个环。
\end{examplle}

\begin{definition}[]
设\(R\)是一个环,称\(R[[x]]\)为\(R\)的 \textbf{形式幂级数环} ,若\(g=\sum r_nx^n\in R[[x]]\),则\(g\)的 \textbf{度数} 记
作\(\deg(g)\),定义为
\begin{equation*}
\deg(g)=\min(n\in\N\mid r_n\neq 0)
\end{equation*}
定义\(\deg(0)=\infty\)。(因此\(\deg(g)\ge 0\))
\end{definition}


\begin{lemma}[]
假设\(R是整环\)
\begin{enumerate}
\item 若\(f\in R[[x]]\),且\(\deg(f)=n\),则
\begin{equation*}
f=x^n(\sum r_kx^k)
\end{equation*}
其中\(r_0\neq 0\),即\(f=x^ng\)其中\(\deg(g)=0\)
\item 若\(f,g\in R[[x]]\),则\(\deg(fg)=\deg(f)+\deg(g)\)
\item 若\(f=\sum r_nx^n\), \(g=\sum s_nx^n\),则\(fg=1\Rightarrow r_0s_0=1\)
\item 若\(f=\sum r_nx^n\),则\(f\)是单位\(\Rightarrow\) \(r_0\)是单位(\(r_0\neq 0\))
\end{enumerate}
\end{lemma}

\begin{proof}
\begin{enumerate}
\item 由定义,若\(f=\sum s_kx^k\)且\(\deg(f)=n\),则\(s_0=\dots=s_{n-1}=0\)且\(s_n\neq 0\),因此
\(f=x^n(\sum_{k=n}^\infty s_kx^k)\),对任意\(i\in\N\),令\(r_i=s_{i+n}\),则\(f=x^n(\sum r_kx^k)\),其
中\(r_0\neq 0\)。
\item 假设\(\deg(f)=n\),\(\deg(g)=m\),则由(1),\(f=x^n(\sum r_kx^k)\),\(g=x^m(\sum s_kx^k)\),其
中\(r_0,s_0\neq 0\),因此\(fg=x^{n+m}\sum_{n=0}^\infty(\sum_{i+j=n}r_is_j)x^n\),因为\(r_0,s_0\neq 0\),\(R\)是
整环,因此\(r_0s_0\neq 0\),因此\(\deg(fg)=n+m=\deg(f)+\deg(g)\)。
\item 由定义,\(fg=\sum_{n=0}^\infty(\sum_{i+j=n}r_is_j)x^n=1\),因此\(r_0s_0=1\)
\item 如果\(f\)是单位,则存在\(g\in R[[x]]\)使得\(fg=1\),由(3),\(r_0\)是单位。
\end{enumerate}
\end{proof}

\begin{proposition}[]
若\(R\)是局部环,则\(R[[x]]\)也是局部环。
\end{proposition}

\begin{proof}
\begin{itemize}
\item 只需验证非单位元关于加法封闭。
\item 设\(f\in R[[x]]\)是单位元,则\(f=r_0+g\),其中\(r_0\)是\(R\)的单位,\(\deg(g)\ge 1\)。
\item 令一方面,若\(f=r_0+g\)且\(r_0\in R\)是单位,\(\deg(g)\ge 1\),取\(s_0\in R\)使得\(s_0r_0=1_R\),则
\(s_0f=1+s_0g\),令\(h=-s_0g\)。
\end{itemize}
\begin{claim}
\(h+h^2+h^3+\dots\in R[[x]]\)
\end{claim}

\begin{proof}
设\(h=\sum s_kx^k\),其中\(s_0=0\),令\(g=\sum_{n=1}^\infty h^n=\sum r_kx^k\),于是\(r_0\in R\),若\(r_0,\dots,r_n\in R\),
则\(r_{n+1}=s_{n+1}+\sum_{i=1}^{n-1}s_ir_{n-i}\in R\),因此对于任意\(k\in\N\),\(r_k\in R\),因此\(g\in R[[x]]\)。
\end{proof}
\begin{itemize}
\item 考虑等式\((1-h)(1+h+h^2+\dots)=1\),则\(s_0f(1+h+h^2+\dots)=1\),故\(f\)是单位,因此
\item \(f\in R[[x]]\)是单位\(\Leftrightarrow f=r_0+g\),其中\(r_0\)是单位且\(\deg(g)\ge 1\)。
\item \(f\in R[[x]]\)不是单位\(\Leftrightarrow \deg(f)\ge 1\)或\(f=r+g\),其中\(r\)不是单位且\(\deg(g)\ge 1\)。
\item \(f\)不是单位\(\Leftrightarrow f\in \fm_0+xR[[x]]=\{r+g\mid r\in\fm_0,g\in xR[x]\}\),其中\(\fm_0\)是\(R\)的极大理想。
\item 显然\(\fm_0+xR[[x]]\)是“+”封闭的,故\(R[[x]]\)是局部环。
\end{itemize}
\end{proof}

\begin{corollary}[]
若\(R\)是局部环,\(\fm_0\)为\(R\)的极大理想,则
\begin{enumerate}
\item \(R[[x]]\)是局部环,其极大理想为
\begin{equation*}
\fm_0+(x)
\end{equation*}
\item 若\(k\)是域,则\(k[[x]]\)中的理想排成一个降链
\begin{equation*}
I_0=\fm_0+(x)\supseteq I_1=(x)\supseteq\dots\supseteq I_n=(x^n)\supseteq\dots
\end{equation*}
\end{enumerate}
\end{corollary}

\begin{proof}
\begin{enumerate}
\item 已证。
\item 设\(J\)是\(k[[x]]\)的理想,令\(n=\min\{\deg(f)\mid f\in J\}\),若\(n=\infty\),则\(J=(0)\)。

若\(n<\infty\)且\(f=x^ng\in J\)其中\(\deg(g)=0\),由于\(g\)的首项是单位,因此\(g\)是单位,令\(h\in R[[x]]\)
使得\(hg=1\),则\(x^n=hf=hgx^n\in J\),因此\((x^n)\subseteq J\),又由\(n\)的定义,\(J\subseteq(x^n)\),所以\(J=(x^n)\)。
\end{enumerate}
\end{proof}

\begin{corollary}[]
若\(k\)是域,则\(k[[x]]\)是局部环,其极大理想为\((x)=xk[[x]]\),剩余域为\(k\)。
\end{corollary}

\begin{corollary}[]
定义\(k[[X_1,\dots,X_{n+1}]]=k[[X_1,\dots,X_n]][[X_{n+1}]]\),则\(k[[X_1,\dots,X_{n+1}]]\)为局部环,其极大理
想\(\fm\)为\((X_1,\dots,X_{n+1})\),剩余域为\(k\)。
\end{corollary}

\begin{examplle}[]
令\(p\in\Z\)是一个素数,
\begin{enumerate}
\item \(\Z/p\Z\)是一个域,这是因为若\(0<r<p\),则\((r,p)=1\),故存在\(m,n\)使得
\begin{equation*}
mr+np=1\Rightarrow mr\equiv_p1
\end{equation*}
故\(\Z/p\Z\)是一个局部环
\item 对每个\(n\in\N^+\),\(\Z/p^n\Z\)是局部环
\begin{itemize}
\item \(\Z\)中包含\((p^n)\)的理想与\(\Z/p^n\Z\)中的理想一一对应
\item \(\Z\)中的理想均形如\((k)\)
\item \((p^n)\subseteq(k)\Leftrightarrow k\mid p^n\Rightarrow k=p^m\),其中\(m\le n\)
\item 故\(\Z/p^n\Z\)中的理想为
\begin{equation*}
p^n\Z/p^n\Z=(0)\subseteq p^{n-1}\Z/p^n\Z\subseteq\dots\subseteq p\Z/p^n\Z
\end{equation*}
\item 故\(p\Z/p^n\Z\)为\(\Z/p^n\Z\)的唯一极大理想,显然\(\Z/p^n\Z\)中有\(p^n\)个元素。

\item \(\Z/p^n\Z\)的元素可唯一表示为
\end{itemize}
\begin{equation*}
a_0+a_1p+\dots+a_{n-1}p^{n-1}
\end{equation*}
其中 \(a_i\in\{0,\dots,p-1\}\)。
\item 若\(m>n\),则\(\Z\to\Z/p^m\Z\)和\(\Z\to\Z/p^n\Z\)诱导了
\begin{center}\begin{tikzcd}
\Z\ar[r]\ar[d]&\Z/p^m\Z\ar[dl]\\
\Z/p^n\Z
\end{tikzcd}\end{center}
\begin{itemize}
\item \(\forall m>n\),令\(\pi_{mn}\)为\(\Z/(p^m)\)到\(\Z/(p^n)\)的自然同态,即
\end{itemize}
\begin{equation*}
\pi_{mn}(a_0+a_1p+\dots+a_{m-1}p^{m-1})=a_0+\dots+a_{n-1}p^{n-1}
\end{equation*}
\begin{itemize}
\item 令\(\Z^*=\prod_{n=1}^\infty\Z/(p^n)=\{(x_1,x_2,\dots)\mid x_n\in\Z/(p^n)\}\),
\item 将\(x_n\)看作\(a_0+\dots+a_{n-1}p^{n-1}\)或序列\((a_0,\dots,a_{n-1})\)
\item 定义\(\Z_p\subseteq\Z^*\)为
\end{itemize}
\begin{equation*}
\{(x_1,x_2,\dots,)\mid\pi_{mn}(x_m)=x_n,m>n\}
\end{equation*}
\begin{itemize}
\item 将\((x_1,x_2,\dots)\)中的每个\(x_n\)看作\(a_0+\dots+a_{n-1}p^{n-1}\),则\((x_1,x_2,\dots)\in \Z_p\Leftrightarrow\forall m>n\), \(x_m\)是\(x_n\)的延长

\item 故而\((x_1,x_2,\dots)\in\Z_p\)唯一对应一个幂级数\(a_0+a_1p+a_2p^2+\dots\)

\item 定义\(\Z^*\)中的+为
\end{itemize}
\begin{equation*}
(x_1,x_2,\dots)+(y_1,y_2,\dots)=(x_1+y_1,x_2+y_2,\dots)
\end{equation*}
\begin{itemize}
\item 定义\(\Z^*\)中的``\(\times\)''为
\end{itemize}
\begin{equation*}
(x_1,x_2,\dots)\cdot(y_1,y_2,\dots)=(x_1y_1,x_2y_2,\dots)
\end{equation*}
\begin{itemize}
\item 定义零为\((0,0,\dots,)\),幺为\((1,1,\dots)\),则\(\Z^*\)为环。

\item 由于每个\(\pi_{mn}\)是同态,故\(\Z_p\)对“+”与“\(\times\)”封闭:对任意\((x_1,x_2,\dots),(y_1,y_2,\dots)\in\Z_p\),
对任意\(m>n\),因为\(\pi_{mn}\)是同态,有\(\pi_{mn}(x_m+y_m)=\pi_{mn}(x_m)+\pi_{mn}(y_m)=x_n+y_n\),
\(\pi_{mn}(x_m\cdot y_m)=\pi_{mn}(x_m)\cdot\pi_{mn}(y_m)=x_n\cdot y_n\),故\((x_1,x_2,\dots)+(y_1,y_2,\dots),(x_1,x_2,\dots)\cdot(y_1,y_2,\dots)\in\Z_p\)。
\item 故\(\Z_p\)是一个环,称其为 \textbf{\(p\)-进整数环} 。
\item \(\Z_p\)也称为\(\Z/(p^n)\)的逆极限,即\(\Z_p=\varprojlim\Z/(p^n)\)
\end{itemize}
\end{enumerate}
\end{examplle}

\begin{remark}
设\(x=(x_1,x_2,\dots)\in\Z_p\),则\(x\)可以记作\(a_0+a_1p+a_2p^2+\dots\),其中每个\(a_i\in\{0,\dots,p-1\}\),因此
\(x_1=a_0\), \(x_2=a_0+a_1p\),\(\dots\),\(x_n=\sum_{k=0}^{n-1}a_kp^k\)。

设\(y=(y_1,y_2,\dots)\in\Z_p\),设它可写作\(b_0+b_1p+\dots\),令\(z=x+y=(x_1+y_1,x_2+y_2,\dots)\),
将\(z\)写作\(\sum_{k=0}^\infty c_kp^k\),则
\begin{equation*}
z_n=x_n+y_n=(\sum_{k=0}^{n-1}a_kp^k+\sum_{k=0}^{n-1}b_kp^k)(\mod p^k)
\end{equation*}
即\(z_n\)是\(x_n+y_n\)的\(p\)-进制展开的前\(n\)项。

同理若\(z=xy\),则\(z_n\)是\(x_ny_n\)的\(p\)-进制展开的前\(n\)项。

故\(\Z_p\)中的运算是“\(p\)-进制”运算。
\end{remark}

\begin{lemma}[]
label:6
若\(A,B\)是局部环,则\(f:A\to B\)是满同态,则\(a\in A\)是单位\(\Leftrightarrow f(a)\in B\)是单位
\end{lemma}

\begin{proof}
\begin{itemize}
\item 令\(\fm\)是\(B\)的极大理想,
\item 则\(\barf:A/f^{-1}(\fm)\to B/\fm\)是同构,
\item 而\(B/\fm\)是域,故\(A/f^{-1}(\fm)\)是域,故\(f^{-1}(\fm)\)是极大理想,
\item 故\(a\in A\)是单位\(\Leftrightarrow a\notin f^{-1}(\fm) \Leftrightarrow\) \(f(a)\notin\fm\)是\(B\)的单位。
\end{itemize}
\end{proof}

\begin{proposition}[]
\begin{enumerate}
\item \(\Z_p\)是局部环
\item \(\Z_p\)的理想排成降链
\begin{equation*}
p\Z_p\supseteq p^2\Z_p\supseteq\dots
\end{equation*}
\item \(\Z_p/p^n\Z_p\cong\Z/p^n\Z\)
\end{enumerate}
\end{proposition}

\begin{proof}
\begin{enumerate}
\item 设\(x=(x_1,x_2,\dots)=a_0+a_1p+\dots\in\Z_p\),即\(x_1=a_0\),\(x_2=a_0+a_1p\),\(\dots\)。

\begin{claim}
\(x\)是单位\(\Leftrightarrow\) \(a_0\neq 0\)
\end{claim}

\begin{proof}
\(\Leftarrow\):
\begin{itemize}
\item 若\(a_0\neq 0\),则\(a_0\in\Z/p\Z\)是单位,
\item 故存在\(b_0\in\Z/p\Z\)使得\(a_0b_0\equiv 1\mod p\)。
\item 由于 \(\pi_{21}\)是同态,而\(a_0=\pi_{21}(a_0+a_1p)\)是单位,由引理\ref{6},\(a_0+a_1p\in\Z/p^2\Z\)也是单位,
\item 同理,\(\forall b_1\in\{0,\dots,p-1\}\),\(b_0+b_1p\in\Z/p^2\Z\)是单位,
\item 令\(c_0+c_1p\in\Z/p^2\Z\)使得
\begin{equation*}
(a_0+a_1p)(c_0+c_1p)=1\in\Z/p^2\Z
\end{equation*}
\item 则\(\pi_{21}((a_0+a_1p)(c_0+c_1p))=a_0c_0=1=a_0b_0\)。
\item 故\(a_0c_0-a_0b_0\equiv 0\mod p\),因此\(c_0\equiv b_0\mod p\),所以\(c_0=b_0\)。
\item 一般地,设\(b_0+b_1x+\dots+b_{n-1}x^{n-1}\in\Z/(p^n)\)使得
\((a_0+\dots+a_{n-1}x^{n-1})(b_0+\dots+b_{n-1}x^{n-1})=1\in\Z/p^n\Z\),
\item 则存在\(b_n\in\{0,\dots,p-1\}\)使得在\(\Z/(p^{n+1})\)中有
\((a_0+\dots+a_nx^n)(b_0+\dots+b_nx^n)=1\)。
\item 令\(y=b_0+b_1+\dots=(y_1,y_2,\dots)\),则\(xy=1\),故\(x\)是单位。
\end{itemize}

\(\Rightarrow\):若\(a_0=0\),则\(x=(0,x_2,\dots)\)显然不是单位。
\end{proof}

以上断言表明,所有非单位元形如\(x=(0,x_2,x_3,\dots)\)是一个加法群,故而是极大理想,恰好是\(p\Z_p\)

\item 设\(J\subseteq\Z_p\)是一个非平凡理想
\begin{itemize}
\item 令\(k=\min\{n\in\N\mid p^n\in J\}\),显然\(k>0\),\(p^k\Z_p\subseteq J\)
\item 断言\(p^k\Z_p=J\)。
\item 设\(x=a_0+a_1p+\dots\in J\),令\(a_m\)是第一个非零系数
\item 则\(x=p^m(a_m+a_{m+1}p+\dots)\),
\item 因为\(a_m\neq 0\),\(a_m+a_{m+1}p+\dots\)是单位,故存在\(y\in\Z_p\)使得\(xy=p^m\in J\)
\item 由定义,\(k\le m\Rightarrow p^m\in p^k\Z_p\Rightarrow x\in p^k\Z_p\),
\item 即\(\Z_p\)的每个非平反理想都形如\(p^k\Z_p\)。
\end{itemize}
\item 投射函数诱导了一个同态
\begin{center}\begin{tikzcd}
\Z^*\ar[r,"\pi_n"]&\Z/(p^n)\\
\Z_p\ar[u]\ar[ur,"\pi_n"']
\end{tikzcd}\end{center}
其中 \(\pi_n:\Z_p\to\Z/(p^n)\),\(x=(x_1,\dots,x_n,\dots)\mapsto x_n\),于是
\begin{align*}
x\in\ker(\pi_n)&\Leftrightarrow x_n=0\\
&\Leftrightarrow x=(0,\dots,0,x_{n+1},\dots)\\
&\Leftrightarrow x=a_{n}p^n+a_{n+1}p^{n+1}\dots\\
&\Leftrightarrow x\in p^n\Z_p
\end{align*}
\end{enumerate}
\end{proof}

\begin{remark}
证明\(\Z_p\)是局部环的关键是验证
\begin{equation*}
x=a_0+a_1p+\dots\text{是单位}\Leftrightarrow a_0\neq 0
\end{equation*}
\end{remark}

以下证明更简洁:
\begin{itemize}
\item 设\(x=(x_1,x_2,\dots)\in\Z_p\subseteq\prod\Z/(p^n)\),\(x_1=a_0,\dots,x_n=a_0+a_1p+\dots+a_{n-1}p^{n-1},\dots\)
\item 由于每个\(\Z/(p^n)\)都是局部环且\(p\Z/(p^n)\)是其极大理想,
\item 故每个\(x_n\)在\(\Z/(p^n)\)中可逆,令\(y_n\)是\(x_n\)在\(\Z/(p^n)\)的逆
\item \(\pi_{mn}(x_my_m)=\pi_{mn}(x_m)\pi_{mn}(y_m)=x_n\pi_{mn}(y_m)=1\),
\item 故\(\forall n<m\),\(\pi_{mn}(y_m)\)都是\(x_n\)的逆
\item 断言:\(\pi_{mn}(y_m)=y_n\)
\item \(x_n(y_n-\pi_{mn}(y_m))=0\Rightarrow y_nx_n(y_n-\pi_{mn}(y_m))=0\),
\item 故\(y=(y_1,y_2,\dots)\)是\(x\)的逆
\end{itemize}

更加简洁的方法:
\begin{itemize}
\item 取\(b\in\{0,\dots,p-1\}\)使得\(a_0\cdot b\equiv 1\mod p\),
\item 则\(bx=1+p(b_0+b_1p+\dots)=1-py\),
\item 令\(c=1+py+p^2y^2+\dots\in\Z_p\),
\item 则\(bxc=(1-py)(1+py+(py)^2+\dots)=1\)
\end{itemize}

\begin{remark}
\begin{itemize}
\item \(\Z\mapsto\Z_p\), \(x\mapsto x\)的\(p\)-进制展开是一个单同态。
\item \(\Z\)中不能被\(p\)整除的元素都是\(\Z_p\)的单位。
\item 令\(S=\Z-(p)\),则\(S\)是乘法集,\(\Z\)关于\((p)\)的局部化\(\Z_{(p)}=S^{-1}\Z\subseteq\Q\)是局部环,
且\(pS^{-1}\Z\)是极大理想
\item \(\Z_{(p)}=\{\frac{a}{b}:a,b\in\Z,b\nmid b\}\subseteq\Q\)
\item \(\Z\)到\(\Z_p\)的嵌入自然地扩张为\(\Z_{(p)}\)到\(\Z_p\)的嵌入
\begin{center}\begin{tikzcd}
f:\Z\to\Z_p\ar[d]\\
\parbox{3cm}{\centering \(\tilde{f}:S^{-1}\Z\to\Z_p\) \(\frac{a}{b}\mapsto(f(b))^{-1}a\)}
\end{tikzcd}\end{center}
\item \(\Z_p\cap\Q=\Z_{(p)}\)
\item 在形式上,\(\Z_p\)与\(\F_p[[X]]\)有相似之处,然而\(\Char(\Z_p)=0\),而\(\Char(\F_p[[X]])=p\)
\end{itemize}
\end{remark}
\section{亨泽尔局部环}
\label{sec:org08e8c15}
\subsection{亨泽尔局部环(Henselian)}
\label{sec:org9787992}
\begin{definition}[]
\(R\)局部环,\(\fm\)极大理想,\(R\)是 \textbf{亨泽尔环} 如果对每个多项式\(f(x)\in R[x]\),\(a\in R\),有
\begin{equation*}
f(a)\in\fm\wedge f'(a)\notin\fm
\end{equation*}
则存在\(b\in R\)使得\(f(b)=0\)且\(a\equiv b\mod\fm\)
\end{definition}

\begin{remark}
\begin{enumerate}
\item 我们可以把\(\fm\)中的元素看作\(R\)中的“无穷小量”,则\(f(a)\in\fm\)且\(f'(a)\notin\fm\)可理解为\(f(a)\)在
“0”点附近,而\(f(x)\)在“a”处的斜率不为“0”,此时\(f(x)=0\)在\(a\)点附近可能有解
\item 设\(k=R/\fm\)是\(R\)的剩余域,设\(f(x)=\sum_{k=1}^nc_kx^k\in R[x]\),定义,定义\(\barf(x)\in k[x]\)为
\(\sum_{k=1}^n\barc_kx^k\),其中
\begin{equation*}
\barc_k=c_k+\fm
\end{equation*}
则\(f(a)\in\fm\)且\(f'(a)\notin\fm \Leftrightarrow \barf(\bara)=0\)且\(\barf'(\bara)\neq 0 \Leftrightarrow \bara\)
是\(\barf(x)\)的非奇异零点(不是重根)
\end{enumerate}
\end{remark}

\begin{lemma}[]
设\(R\)是一个局部环,\(f(x)\in R[x]\), \(a\in R\),若\(f(a)\in\fm\)且\(f'(a)\notin\fm\),则至多有一个\(b\in R\)使
得\(f(b)=0\)且\(a\equiv b\mod\fm\)
\end{lemma}

\begin{proof}
设\(b\in R\)使得\(f(b)=0\)且\(a\equiv b\mod\fm\),则\(\bara=\barb\),故\(\barf'(\bara)=\barf'(\barb)\neq 0\),
故\(f'(b)\notin\fm\)是一个单位,考虑\(f(x)\)在\(b\)点的泰勒展开
\begin{equation*}
f(x+b)=f(b)+f'(b)x+cx^2
\end{equation*}
若\(x_0\in\fm\),则
\begin{equation*}
f(x_0+b)=f'(b)x_0+cx_0^2=x_0(f'(b)+cx_0)
\end{equation*}
因为\(f'(b)\)是单位,因此\(f'(b)+cx_0\)是单位,故\(f(x_0+b)=0\Leftrightarrow x_0=0\)
\end{proof}
\subsection{剩余域的提升}
\label{sec:orga9fdf92}
\begin{examplle}[]
设\(k\)是一个域,\(R=k[[x]]\),则\(R\)是一个局部环,\(\fm=(x)\)是极大理想

\(a\in k\mapsto\bara=a+(x)\)是\(k\)到\(R/\fm\)的同构,即\(R\)中存在一个子域\(k\)使得自然投射\(x\mapsto\barx\)
在\(k\)上是同构

称\(k\)是\(R\)的剩余域的提升
\end{examplle}

\begin{examplle}[]
设\(R=\Z/p^2\Z\),其中\(p\)是素数,\(\fm=p\Z/p^2\Z\),\(R/\fm\cong\F_p\),而\(R\)中没有子域,故\(R/\fm\)在\(R\)中
没有提升

若有子域,一定有1,但是1可以生成整个\(R\)
\end{examplle}

\begin{examplle}[]
考虑局部环\(\Z_p\),\(\fm=p\Z_p\),\(k=\Z_p/\fm\cong\F_p\),\(\Char\Z_p=0\Rightarrow\F_p\not\subseteq\Z_p\),故\(\Z_p/\fm\)在\(\Z_p\)
中没有提升
\end{examplle}

\begin{definition}[]
设\(R\)是一个局部环,\(\fm\)和\(k\)分别为其极大理想和剩余域,若存在\(R\)的子域\(E\)使
得\(\barE=\{\barx=x+\fm\mid x\in E\}=k\),则称\(E\)是\(k\)的提升
\end{definition}

\begin{remark}
\begin{itemize}
\item 若\(E\)是\(k\)的提升,则\(\pi:E\to\barE\)是同构,\(x\in\ker\pi\Leftrightarrow x\in\fm\Leftrightarrow x\)不可逆即\(x=0\)
\item 故而若\(k\)有提升,则提升唯一
\end{itemize}
\end{remark}

\begin{theorem}[提升定理]
设\(R,\fm,k\)如上,若\(R\)是亨泽尔的,且\(\Char k=0\),则\(k\)在\(R\)中有提升
\end{theorem}

\begin{proof}

\end{proof}
\subsection{域的扩张理论}
\label{sec:orgd2ba1fd}
\begin{definition}[]
设\(K,L\)是两个域,若\(K\)是\(L\)的子域,则称\(L\)是\(K\)的一个扩张,记作\(L/K\)
\end{definition}

\begin{definition}[]
设\(L/K\)是一个域扩张,\(X\subseteq L\),则
\begin{enumerate}
\item \(K[X]\)表示由\(K\bigcup X\)生成的\(L\)的子环,
\begin{equation*}
K[X]=\la K\cup X\ra_L
\end{equation*}
\item \(K(X)\)表示\(K[X]\)的分式域
\item 若\(X=\{a_1,\dots,a_n\}\)有穷,则\(K[X]\)记作\(K[a_1,\dots,a_n]\),\(K(X)\)记作\(K(a_1,\dots,a_n)\)
\end{enumerate}
\end{definition}

\begin{proposition}[]
若\(L/K\)是域扩张,\(a_1,\dots,a_n\in L\),则
\begin{align*}
&K[a_1,\dots,a_n]=\{f(a_1,\dots,a_n):f\in K[X_1,\dots,X_n]\}\\
&K(a_1,\dots,a_n)=\{\frac{f(a_1,\dots,a_n)}{g(a_1,\dots,a_n)}\mid f,g\in K[X_1,\dots,X_n],g(a_1,\dots,a_n)\neq 0\}
\end{align*}
\end{proposition}

\begin{definition}[]
设\(L/K\)是一个域扩张,\(a\in L\),称\(a\)在\(K\)上是代数的,如果存在一个非零多项式\(f(x)\in K[X]\)
使得\(f(a)=0\),如果\(a\)不是代数的,则\(a\)在\(K\)上是超越的
\end{definition}

\begin{definition}[]
设\(L/K\)是域扩张,\(a\in L\)在\(K\)上代数,若\(p(x)\in K[x]\)是使得\(p(a)=0\)的次数最小的首一多项式,
则称\(p(x)\)是\(a\)在\(K\)上的极小多项式,记作\(\min(K,a)\)
\end{definition}

\begin{remark}
\begin{itemize}
\item 显然\(I=\{f(x)\in K[X]\mid f(a)=0\}\)是\(k[x]\)的一个理想
\item 由于\(K[x]\)是主理想整环,即每个理想都形如\((g(x))\),故\(I=(p(x))\),\(p\in I\)且\(\deg(p)\)最
小,若要求\(p(x)\)首项为1,则\(p(x)\)唯一
\item 显然\(p(x)\)在\(K[X]\)中不可约
\item 设\(p=x^n+\sum_{k=0}^{n-1}c_kx^k\)
\item 将\(K[a]=\{f(a)\mid f\in K[x]\}\)视作\(K\)上的向量空间
\item 由于\(p\)是使得\(p(a)=0\)的次数最小的多项式
\item 故\(1,a,a^2,\dots,a^{n-1}\)在\(K\)上线性无关
\item \(a^n\)是\(\{1,\dots,a^{n-1}\}\)的线性组合
\item \(a^{n+1}\)也类似
\item 故\(\{1,a,\dots,a^{n-1}\}\)是\(k[a]\)的一组基
\item 现在\(K[a]\)是一个环,同时是\(K\)上的\(n\)维向量空间,基为\(\{1,\dots,a^{n-1}\}\)
\item \(\forall f(x)\in K[x]\), \(f(x)\)与\(p(x)\)互素
\item 故存在\(s(x),t(x)\in K[x]\)使得\(s(x)f(x)+t(x)p(x)=1\),故每个\(f(a)\in K[a]\)都可逆,\(K[a]\)是一
个域
\end{itemize}
\end{remark}

\begin{definition}[]
设\(L/K\)是一个域扩张,则\(L\)是\(K\)上的向量空间,\([L:K]\)表示\(L\)作为\(K\)空间的维数,
称\(L/K\)是一个有穷扩张如果\([L:K]<\infty\)
\end{definition}

\begin{proposition}[]
设\(L/K\)是一个域扩张,且\(a\in L\),在\(K\)上代数
\begin{enumerate}
\item \(\min(K,a)\)是\(K\)上的不可约多项式
\item \(\forall g(x)\in K[x]\),\(g(a)=0\Leftrightarrow\min(K,a)\mid g(x)\)
\item 若\(\min(K,a)\)的次数为\(n\),则\(\{1,\dots,a^{n-1}\}\)是\(K[a]\)在\(K\)上的一组基
\item \(K[a]=K(a)\)是域,\([K(a):K]=n\)
\item \(K[a]\cong K[x]/\min(K,a)\)
\end{enumerate}
\end{proposition}

\begin{proposition}[]
设\(F\subseteq K\subseteq L\)是域扩张,则
\begin{equation*}
[L:F]=[L:K][K:F]
\end{equation*}
\end{proposition}

\begin{proof}
设\(\{a_i\mid i\in I\}\)是\(K/F\)的一组基, \(\{b_j\mid j\in J\}\)是\(L/K\)的一组基

证明\(\{a_ib_j\mid i\in I,j\in J\}\)是\(L/F\)的基
\end{proof}

\begin{definition}[]
设\(L/K\)是域扩张,若每个\(a\in L\)都在\(K\)上代数,则称\(L\)是\(K\)的代数扩张
\end{definition}

\begin{lemma}[]
若\(L/K\)是有穷扩张,则\(L\)是\(K\)的代数扩张且存在\(a_1,\dots,a_n\)使得
\begin{equation*}
L=K(a_1,\dots,a_n)
\end{equation*}
\end{lemma}

\begin{proof}
对\([L:K]\)归纳

若\(L=K\),则证明结束

否则,取\(a\in L\setminus K\),则\(1<[K(a):K]\le[L:K]<\infty\)

故存在\(n\)使得\(\{1,a,\dots,a^{n-1}\}\)线性无关,\(a\)在\(K\)上是代数,故\(L/K\)是代数扩张,
\end{proof}

\begin{remark}
\(L\)可以由“更少”的元素生成,取\(b_1,\dots,b_m\in L\)使得\(b_1\notin K\), \(b_2\notin K(b_1)\),
\(\dots\), \(b_m\notin K(b_1,\dots,b_{m-1})\),则
\([K[b_1]:K]\ge 2\),故\([K(b_1,\dots,b_m):K]\ge 2^M\)
\end{remark}

\begin{lemma}[]
若\(L/K\)是域扩张,\(a_1,\dots,a_n\in L\),若每个\(a_i\)都在\(K\)上代数,则
\(E=K[a_1,\dots,a_n]\)是域且\([K[a_1,\dots,a_n]:K]\le\prod_{i=1}^n[K(a_i):K]\)
\end{lemma}

\begin{proof}
\(a_2\)在\(K\)上代数推出\(a_2\)在\(K[a_1]\)代数

若\(m\)时满足,令\(E=K[a_1,\dots,a_m]\),则
\begin{equation*}
[E[a_{m+1}]:K]=[E[a_{m+1}]:E][E:K]\le\prod_{i1}^m[K[a_i]:K][E[a_{m+1}]:E]
\end{equation*}
令\(p(x)\)为\(\min(E,a_{m+1})\),\(q(x)\)为\(\min(K,a_{m+1})\),当然\(\deg(p)\le\deg(q)\)

于是\([E[a_{m+1}]:E]\le[K[a_{m+1}]:K]\)

从而\([E[a_{m+1}]:K]\le\prod_{i=1}^{m+1}[K[a_i]:K]\)
\end{proof}

\begin{corollary}[]
\begin{itemize}
\item 设\(L/K\)是域扩张,\(a\in L\),则\(a\)在\(K\)上代数当且仅当\([K(a):K]<\infty\)
\item \(L\)在\(K\)上代数当且仅当对每个有穷的\(X\subseteq L\),都有\([K(X):X]<\infty\)
\item \(X\subseteq L\)使得每个\(a\in X\)都在\(K\)上代数,则\(K(X)/K\)是代数扩张
\end{itemize}
\end{corollary}

\begin{remark}
设\(a\in L\)在\(K\)上超越,则映射\(\ev_a:F[X]\to F[a]\)是同构
\end{remark}

\begin{proposition}[]
设\(F\subseteq K\subseteq L\)是域扩张,若\(K/F\)和\(L/K\)均是代数扩张,则\(L/F\)也是代数扩张
\end{proposition}

\begin{proof}
设\(a\in L\),令\(f(x)=k_0+\dots+x^n\)是\(a\)在\(K\)的极小多项式,显然\(f(x)\in F[k_0,\dots,k_n]\),故\(a\)
在\(F[k_0,\dots,k_n]\)上代数,从而\([F[k_0,\dots,k_n][a]:F[k_0,\dots,k_n]]<\infty\),故
\([F[k_0,\dots,k_n,a]:F]<\infty\),故\(F[k_0,\dots,k_n,a]/F\)是代数扩张,故\(a\)在\(F\)上代数。
\end{proof}

\begin{definition}[]
设\(L/K\)是一个域扩张,则称\(\{a\in L\mid a\text{在}K\text{上是代数的}\}\)为\(K\)在\(L\)中的 \textbf{代数闭包}
,若该闭包是\(K\)子自己,则称\(K\)在\(L\)中代数闭
\end{definition}

\begin{corollary}[]
设\(L/K\)是一个域扩张,令\(E\)为\(K\)在\(L\)中的代数闭包,则\(E\)是一个域,从而是\(K\)在\(L\)中
最大的代数扩张
\end{corollary}

\begin{proof}
只需验证\(E\)中的元素关于加法乘法封闭

设\(a,b\in E\),则\([K[a]:K],[K[b]:K]<\infty\),故\([K[a,b]:K]<\infty\),
\([K[a,b],K]\le[K[a]:K][K[b]:K]<\infty\)
\end{proof}

\begin{definition}[]
设\(K\)是一个域,\(K\)是 \textbf{代数闭域} ,如果\(K\)的任何真扩张都不是代数扩张

\(E\supseteq K\)是\(K\)的 \textbf{代数闭包} 如果\(E\)是代数闭的,且\(E\)的包含\(K\)的真子域都不是代数闭的
\end{definition}

\begin{remark}
\begin{enumerate}
\item \(K\)是代数闭域\(\Leftrightarrow\)任何非常数\(f(x)\in K[x]\)在\(K\)中有根\(\Leftrightarrow\)只有\(\deg\le 1\)的\(f(x)\in K[x]\)
不可约
\item 若\(L\)是代数闭的且\(K\subseteq L\),则\(E=\{a\in L\mid a\text{在$K$上代数}\}\)是\(K\)的代数闭包
\end{enumerate}
\end{remark}

下面给出代数闭包的构造

设\(K\)是一个域且\(\lambda=\abs{K}+\omega\),令\(\{f_i(x)\mid i<\lambda\}\)是\(K[x]\)的一个枚举(选择公理),令\(K_0=K\)

若\(f_0(x)\)在\(K_0\)上可约,则\(K_1=K_0\)

若不可约,则\(K_1=K_0[x]/(f_0(x))\),于是\(f_0(x)\)在\(K_1\)中有根(\(x+(f_0(x))\))

一般地,若\(\{K_i\mid i<\alpha\}\)已构造,若\(\alpha=\beta+1\),则\(K_\alpha=K_\beta\)或\(K_\alpha=K_\beta[x]/(f_\beta(x))\)

若\(\alpha\)是极限序数,则\(K_\alpha=\bigcup_{\beta<\alpha}K_\beta\)

于是每个\(K_{i+1}/K_i\)是代数扩张,每个\(f_i(x)\in K[X]\)在\(K_{i+1}\)中可约

令\(E=\bigcup_{\alpha<\lambda}K_\alpha\),断言\(E/K\)是代数的

设\(a\in E\),则\(\exists\alpha<\lambda\)使得\(a\in K_\alpha\)

令\(\beta_0=\min\{\beta<\alpha\mid\alpha\text{在$K_\beta$上代数}\}\),

则存在\(c_0,\dots,c_{n-1}\in K_{\beta_0}\)使得\(\sum c_ia^i=0\),

若\(\beta_0\neq 0\),则\(\exists\alpha_0<\beta_0\)使得\(c_0,\dots,c_{n-1}\)在\(K_{\alpha_0}\)上代数,从而\(a\)
在\(K_{\alpha_0}[c_0,\dots,c_{n-1}]\)上代数,由传递性(index),\(a\)在\(K_{\alpha_0}\)上代数,与\(\beta_0\)的极小性矛盾,
故\(\beta_0=0\)

同理\(E\)是代数闭的,因为每个代数扩张对应一个极小多项式,但是在构造过程中多项式已经被用完了

\(E\)是\(K\)的代数闭包

\begin{proposition}[]
任何域\(K\)都有代数闭包,且其代数闭包相互同构,记作\(K^{\alg}\)
\end{proposition}

若\(E'\)是\(K\)的代数闭包,考虑\(E\to E'\)的部分同构,back-and-forth一步一步抓每个元素,极大同构就
是真的同构
\subsection{提升定理}
\label{sec:orgc7ef5f2}
设\(R\)是亨泽尔局部环,\(\fm\subseteq R\)是极大理想,\(k=R/\fm\)是剩余域,若\(\Char k=0\),则\(k\)可以被提升,
即存在子域\(E\subseteq R\)使得
\begin{equation*}
k=\barE=\{a+\fm\mid a\in E\}
\end{equation*}

\begin{proof}
令\(n_R=\underbrace{1_R+\dots+1_R}_{n}\),令\(n_k\)表示\(\underbrace{1_k+\dots+1_k}_{n}\),
则\(n_k=\barn_R=n_R+\fm\),由于\(\Char k=0\),故\(n_k\neq 0\),从而\(n_R\notin\fm\),故\(R\)的特征为0

不妨假设\(\Z\subseteq R\),\(\forall n\in\Z\),\(n\notin\fm\),由于\(R\)是局部环每个\(n\neq 0\)均可逆,故
\(\Q\subseteq R\)

令\(\calf=\{E\mid E\text{是}R\text{的子域}\}\)

注意到每个\(E\in\calf\)中的非零元素都可逆,故而\(E\)到\(k\)都是单同态,\(\ker(\pi)\subseteq\fm\),令\(E^*\)是\(\calf\)
在\(\subseteq\)下的极大元,证明\(E^*\)就是\(k\)的提升

\textbf{断言1}:\(E^*\)在\(R\)中代数闭

否则,\(a\in R\setminus E^*\)在\(E^*\)上代数,则\(E^*[a]\)是\(E^*\)的真域扩张

下面证明\(\barE^*=\{a+\fm\mid a\in E^*\}\)是\(k=R/\fm\)

否则,设\(\barb=b+\fm\in k\setminus\barE^*\),则
\(\barb\)在\(\barE^*\)上代数或超越

若\(\barb\)在\(\barE^*\)上代数,则存在\(f(x)\)使得\(\barf(x)\)是\(\barb\)在\(\barE^*\)上的极小多
项式,即\(\barf(\barb)=0\)且\(\barf'(\barb)\neq 0\),即\(f(b)\in\fm\)且\(f'(b)\notin\fm\),由亨泽尔性,存
在\(\epsilon\in\fm\)使得\(f(b+\epsilon)=0\),即\(b+\epsilon\)在\(E^*\)上代数,而\(\ove{b+\epsilon}=\barb\notin\barE^*\),于
是\(\barE^*\)不是代数闭,矛盾

若\(\barb\in k\setminus\barE^*\)是超越的,于是\(\forall f(x)\in E^*[X]\),\(f(b)\notin\fm\),即\(E^*[b]\)中每个非零元都不
属于\(\fm\),从而可
逆,故\(E^*(b)\)是\(R\)的一个子域,是\(E^*\)的真扩张,矛盾

故\(\barE^*=k=R/\fm\)
\end{proof}
\section{超积与Ax-Kochen原理}
\label{sec:org540a4cc}
\subsection{环的一阶语言}
\label{sec:org3b36011}
考虑环的一阶语言\(\call_{ring}=\{+,\times,0,1\}\)
\subsection{Łoś超积定理}
\label{sec:org489d787}
\subsection{局部Ax-Kochen原理}
\label{sec:org5586aea}
观察:\(\Z_p=\{\sum_{n=0}^\infty a_np^n\mid a_n\in\{0,\dots,p-1\}\}\)与
\(\F_p[[t]]=\{\sum_{n=0}^\infty a_nt^n\mid a_n\in\{0,\dots,p-1\}\}\)的相似之处:
\begin{enumerate}
\item \(\Z_p/(p)=\F_p=\F_p[[t]]/(t)\)
\item (局部)\(\Z_p\)是\(\{\Z/(p^n)\mid n\in\N^+\}\)的逆向极限
\item (局部)\(\F_p[[t]]\)是\(\{\F_p[t]/(t^n)\mid n\in\N^+\}\)的逆向极限
\item \(\Z\)在\(\Z_p\)稠密,\(\F_p[t]\)在\(\F_p[[t]]\)中稠密
\end{enumerate}
差异:
\begin{enumerate}
\item \(\Char\Z_p=0\),\(\Char(\F_p[[t]])=p\)
\item \(\Char(\Z/p^n)=p^n\),\(\Char(\F_p[t]/(t^n))=p\)
\end{enumerate}


\begin{theorem}[局部Ax-Kochen同构定理]
令\(\calu\)是素数集上的一个非主超滤,则对每个\(n\in\N^+\),有
\begin{equation*}
\prod_{\calu}(\Z/(p^n))\cong\prod_{\calu}(\F_p[t]/(t^n))
\end{equation*}
\end{theorem}

\begin{lemma}[]
设\(\{A_i\mid i\in I\}\)是一组亨泽尔局部环,\(A_i\)的极大理想为\(\fm_i\),剩余域为\(k_i\),令\(\calu\)是\(I\)上的
一个超滤,则
\begin{enumerate}
\item \(\prod_{\calu}A_i\)是一个亨泽尔局部环
\item \(\prod_{\calu}\fm_i=\{[(a_i)_{i\in I}]\mid a_i\in\fm_i\}\)是极大理想
\item \(\prod_{\calu}k_i\)同构于\(\prod_{\calu}A_i/\prod_{\calu}\fm_i\)
\end{enumerate}
\end{lemma}

\begin{proof}
\begin{enumerate}
\item 亨泽尔局部环是一阶句子
\item 设\([a]\in\prod_{\calu}A_i\),则
\begin{align*}
[a]\text{是单位}&\Leftrightarrow\exists[b],[a][b]=[1]\\
&\Leftrightarrow\{i\in I\mid a_ib_i=1_i\}\in\calu\\
&\Leftrightarrow\{i\in I\mid a_i\text{是单位}\}\in\calu\\
&\Rightarrow[a]\notin\prod_{\calu}\fm_i
\end{align*}
若\([a]\notin\prod_{\calu}\fm_i\),则显然\(\{i\in I\mid a_i\notin\fm_i\}\in\calu\),故
\(\pi(\prod_{i\in I}\fm_i)=\prod_{\calu}\fm_i\)是其极大理想
\item 设\(k_i=A_i/\fm_i\),令\(R_i:A_i\to k_i\)为自然投射,令\(\prod_{\calu}R_i:\prod_{\calu}A_i\to\prod_{\calu}(A_i/\fm_i)\),
\([(a_i)_{i\in I}]\mapsto[(R_i(a_i))_{i\in I}]\),则\(\prod_{\calu}R_i\)是良定义的满同态,且\(\ker(\prod_{\calu}R_i)=\prod_{\calu}\fm_i\)
\end{enumerate}
\end{proof}

\begin{lemma}[]
\label{12}
若\(f(x)\in \Z[x]\),\(n>0\),\(a\in\Z\)使得\(f(a)\equiv 0\mod p^n\),\(f'(a)\not\equiv 0\mod p\),则存在
\(b\in\Z\)使得\(a\equiv b\mod p^n\)且\(f(b)\equiv 0\mod p^{n+1}\)
\end{lemma}

\begin{proof}
对\(n\)归纳证明:
\begin{enumerate}
\item 若\(n=1\),考虑同态\(\pi:\Z\to\Z/(p^2)\),\(f'(a)\not\equiv 0\mod p\),于是\(\pi(f'(a))\)是\(\Z/(p^2)\)的单位,
令\(c\in\Z/(p^2)\)是\(\pi(f'(a))\)的逆

任取\(\tilc\in\Z\)为\(c\)的提升,令\(\epsilon_1=-\tilc f(a)\),令\(b=a+\epsilon_1\),则

\(f(a)\equiv 0\mod p\Rightarrow\epsilon_1\equiv 0\mod p\Rightarrow a\equiv b\mod p\)

\(f(b)=f(a+\epsilon_1)=f(a)+f'(a)\epsilon_1+\epsilon_1^2r\),\(\pi(f(b))=\pi(f(a))+\pi(f'(a)\epsilon_1)+\pi(\epsilon_1^2r)=\pi(\epsilon_1^2r)\)

\(\epsilon_1\equiv 0\mod p\),因此\(\epsilon_1^2\equiv 0\mod p^2\)
\item \(f(a)\equiv 0\mod p^n\),\(f'(a)\not\equiv 0\mod p\),令\(\pi_{n+1}:\Z\to\Z/(p^{n+1})\),令\(c\in\Z/(p^{n+1})\)
为\(\pi_{n+1}(f'(a))\)的逆,令\(\tilc\)为\(c\)在\(\Z\)的一个提升,令\(\epsilon_n=-\tilc\cdot f(a)\),
则\(\epsilon_n\equiv 0\mod p^n\),令\(b=a+\epsilon_n\),则\(a\equiv b\mod p^n\),且
\(f(a+\epsilon_n)=f(a)+f'(a)\epsilon_n+\epsilon_n^2\cdot r\),
\begin{align*}
\pi_{n+1}(f(b))&=\pi_{n+1}(f(a))+\pi_{n+1}(f'(a))\pi_{n+1}(\epsilon_n)+0\\
&=0
\end{align*}
\end{enumerate}
\end{proof}

\begin{corollary}[]
设\(f(x)\in\Z[x]\), \(a\in\Z\)使得\(f(a)\equiv 0\mod p\), \(f'(a)\not\equiv 0\mod p\),则对任意\(n>0\),存在整数
序列\(b_1=a,b_2,\dots,b_n\)使得\(b_k\equiv b_{k+1}\mod p^k\)且\(f(b_k)\equiv 0\mod p^k\)
\end{corollary}

\begin{proof}

\end{proof}

若要求\(b_k<p^k\),则序列唯一

\begin{corollary}[]
对每个\(n>0\),\(\Z/(p^n)\)都是亨泽尔局部环
\end{corollary}

\begin{proof}
已知\(\Z/(p^n)\)是局部环,下面证明\(\Z/(p^n)\)的亨泽尔性。

同态\(\pi_n:\Z\to\Z/(p^n)\)可以自然扩张为
\begin{equation*}
\Z[x]\to\Z/(p^n)[x]
\end{equation*}
记作\(\pi_n\),用\(\tilf\)表示\(f(x)\in\Z/(p^n)[x]\)在\(\Z[x]\)中的提升,用\(\tila\)表示\(a\in\Z/(p^n)\)
在\(\Z\)的一个提升,显然对任意\(f(x)\in\Z/(p^n)[x]\)以及\(a\in\Z/(p^n)\)有
\begin{enumerate}
\item \(f(a)\in\fm\Leftrightarrow\tilf(\tila)\equiv\mod p\)
\item \(f'(a)\notin\fm\Leftrightarrow\tilf'(\tila)\not\equiv 0\mod p\)
\end{enumerate}
设\(f,a\)满足条件1,2,则由引理 \ref{12} 存在\(\b^*\in\Z\)使得
\end{proof}

\begin{theorem}[]
若\(R\)是一个局部环,如果存在\(t\in R\)使得\(\fm=tR\)是极大理想,则存在\(n>0\)使得\(t^n=0\),则\(R\)
是一个亨泽尔环
\end{theorem}

\begin{remark}
设\(R\)是局部环,\(\fm\)是极大理想,\(t\in R\)使得\(\fm=(t)\),\(t^{n-1}\neq 0\),\(t^n=0\),则
\begin{equation*}
R=(t^0)\supsetneq(t)\supsetneq\dots\supsetneq(t^n)\supsetneq(t^{n+1}=\emptyset)
\end{equation*}
是一个严格降链

对每个\(r\in R\),存在\(m\le n\)使得
\begin{equation*}
r\in(t^m)\setminus(t^{m+1})
\end{equation*}
定义\(r\)的 \textbf{范数} \(\abs{r}\)为
\begin{align*}
&r\neq 0\Rightarrow\abs{r}=2^{-m}\\
&r=0\Rightarrow\abs{r}=0
\end{align*}

则\((R,\abs{})\)是一个完备的(超度量)空间
\begin{enumerate}
\item \(r=0\Leftrightarrow\abs{r}=0\),\(\abs{1}=1\)
\item \(\abs{r_1+r_2}\le\max\{\abs{r_1},\abs{r_2}\}\)
\item \(\abs{r_1r_2}\le\abs{r_1}\abs{r_2}\)
\end{enumerate}


这种范数称为 \textbf{超范数} ,对应的度量称为 \textbf{超度量} 。
\end{remark}

\(\Z/(p^n)\)和\(k[t]/t^n\)都是完备的超度量空间

\begin{remark}
若\(R\)是一个局部整环,\(t\in R\)使得\(\fm=(t)\)且\((t^n)\neq 0\)
\begin{equation*}
\bigcap_{n=0}^\infty(t^n)=\{0\}
\end{equation*}
则\((t^0)\supsetneq\dots\supsetneq(t^n)\supsetneq\dots\)是一个严格降链,设\(r\in R\),定义
\begin{equation*}
\abs{r}=
\begin{cases}
2^{-m}&r\in(t^m)\setminus(t^{m-1})\\
0^r=0
\end{cases}
\end{equation*}
则\((R,\abs{})\)是一个超度量空间
\begin{enumerate}
\item 

\item 

\item \(\abs{r_1r_2}=\abs{r_1}\abs{r_2}\)
\end{enumerate}

但\(R\)不一定完备
\end{remark}

\begin{lemma}[]
设\(R\)是一个亨泽尔局部环,\(\fm\)是极大理想,\(k\)是剩余域,若\(t\in R\)使得\(\fm=(t)\)且
\begin{equation*}
\Char(k)=0,t^{n-1}\neq 0,t^n=0
\end{equation*}
则\(R\cong k[X]/(x^n)\)
\end{lemma}

\begin{proof}
由提升定理,\(k\)在\(R\)中有提升\(E\),则\(R\)是\(E\)上的向量空间。

\textbf{断言1} :\(\{1,t,\dots,t^{n-1}\}\)是\(R\)在\(E\)上的一组基。

\begin{enumerate}
\item 线性无关:设\(e_0+\dots+e_{n-1}t^{n-1}=0\),若\(e_0,\dots,e_{n-1}\in E\)不全为0,令\(i=\min\{k\mid e_k\neq 0\}\),
则\(e^it^i=-(e_{i+1}t^{i+1}+\dots+e_{n-1}t^{n-1})\),两边乘\(t^{n-i-1}\),
则\(e_it^{n-1}=0\Rightarrow t^{n-1}=0\),矛盾
\item 设\(r\in R\),我们找出

断言2:若\(s\in(t^k)\),则存在\(e\in E\)使得
\begin{equation*}
s-et^k\in(t^{k+1})
\end{equation*}

若\(s\in(t^{k+1})\),则\(e=0\);若\(s\notin(t^{k+1})\),则\(s=at^k\),\(a\notin(t)\),由于\(E/\fm=R/\fm\),故
存在\(e\in E\)使得\(e/\fm=a/\fm\),故\(et^k-s=et^k-at^k=(e-a)t^k\in(t^{k+1})\)

由以上断言,可递归构造\(e_0,e_1,\dots,e_{n-1}\)如下

\begin{itemize}
\item 取\(e_0\in E\)使得\(r-e_0\in(t)\)
\item 取\(e_1\in E\)使得\(r-e_0-e_1t\in(t^2)\)
\item 取\(e_{n-1}\in E\)使得\(r-e_0-\dots-e_{n-1}t\in(t^n)=\{0\}\)
\end{itemize}
即\(r=e_0+e_1+\dots+e_{n-1}t^{n-1}\)
\end{enumerate}

定义\(\pi:E[X]\to R\),\(f(x)\mapsto f(t)\)。

则断言1保证\(\pi\)是满同态且\(\ker(\pi)=(x^n)\),即
\begin{equation*}
R\cong E[X]/(x^n)\cong k[X]/(x^n)
\end{equation*}
\end{proof}

若\(k\subseteq R\), \(a_1,\dots,a_n\in R\), \(R=k[a_1,\dots,a_n]\),则

\begin{theorem}[局部Ax-Kochen同构定理]
令\(\calu\)是素数集上的一个非主超滤,则对每个\(n\in\N^+\),有
\begin{equation*}
\prod_{\calu}(\Z/(p^n))\cong\prod_{\calu}(\F_p[t]/(t^n))
\end{equation*}
\end{theorem}

\begin{proof}
令\(\calu\)是素数集\(\calp\)上的一个非主超滤,则\(\prod_{\calu}\F_p\)是\(\prod_{\calu}\Z/(p^n)\)与\(\prod_{\calu}\F_p[t_p]/(t_p^n)\)的剩余
域

对每个\(n>0\),\(p>n\)能推出\(\F_p\vDash n\neq 0\),故
\(\{p\in\calp\mid\F_p\vDash n\neq 0\}\)是\(\calp\)的余有穷集,故\(\forall n>0\),有\(\prod_{\calu}\F_p\vDash n\neq 0\),
故
\begin{enumerate}
\item \(\Char\prod_{\calu}\F_p=0\)
\item 同理\(a=[(p)_{p\in\calp}]\)满足
\(a\)是\(\prod_{\calu}\Z/(p^n)\)的极大理想,且\(a^{n-1}\neq 0\),\(a^n=0\)
\item 
\end{enumerate}
\end{proof}

\begin{theorem}[局部Ax-Kochen转移原理]
给定\(n>0\)以及一个\(\call_{ring}\)-句子\(\sigma\)存在有限的素数集\(E_\sigma\)使得对每个\(p\notin E_\sigma\)有
\begin{equation*}
\Z/p^n\vDash\sigma\Leftrightarrow(\F_p[t]/t^n)\vDash\sigma
\end{equation*}
\end{theorem}

\begin{proof}
否则,
\begin{equation*}
X_\sigma=\{p\in\calp\mid\Z/p^n\vDash\sigma\Leftrightarrow(\F_p[t]/t^n)\vDash\sigma\}
\end{equation*}
的补集\(Y_\sigma=\calp\setminus X_\sigma\)是无穷集。

则存在非主超滤\(\calu\)使得\(Y_\sigma\in\calu\)

令\(Z_\sigma=\{p\in\calp\mid\Z/p^n\vDash\sigma\}\),\(F_\sigma=\{p\in\calp\mid\F_p[t]/t^n\vDash\sigma\}\),则
\(Z_\sigma\cap F_\sigma\cap Y_\sigma=\emptyset\),故\(Z_\sigma\in\calu\Rightarrow F_\sigma\notin\calu\),

若\(\prod_{\calu}\Z/\)
\end{proof}
\end{document}