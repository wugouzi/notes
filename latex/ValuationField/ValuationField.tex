% Created 2022-06-13 Mon 16:17
% Intended LaTeX compiler: pdflatex
\documentclass[11pt]{article}
\usepackage[utf8]{inputenc}
\usepackage[T1]{fontenc}
\usepackage{graphicx}
\usepackage{longtable}
\usepackage{wrapfig}
\usepackage{rotating}
\usepackage[normalem]{ulem}
\usepackage{amsmath}
\usepackage{amssymb}
\usepackage{capt-of}
\usepackage{hyperref}
\graphicspath{{../../books/}}
% TIPS
% \substack{a\\b} for multiple lines text





% pdfplots will load xolor automatically without option
\usepackage[dvipsnames]{xcolor}

\usepackage{forest}
% two-line text in node by [two \\ lines]
% \begin{forest} qtree, [..] \end{forest}
\forestset{
  qtree/.style={
    baseline,
    for tree={
      parent anchor=south,
      child anchor=north,
      align=center,
      inner sep=1pt,
    }}}
%\usepackage{flexisym}
% load order of mathtools and mathabx, otherwise conflict overbrace

\usepackage{mathtools}
%\usepackage{fourier}
\usepackage{pgfplots}
\usepackage{amsthm, mathabx,  amsmath, commath}
\usepackage{amsfonts}

\usepackage{empheq}
\usepackage{tikz}
\usetikzlibrary{arrows.meta}
\usepackage[most]{tcolorbox}

\newtheorem{theorem}{Theorem}[section]
\newtheorem{definition}{Definition}[section]
\newtheorem{corollary}{Corollary}[section]
\newtheorem{example}{Example}[section]
\newtheorem{lemma}{Lemma}[section]
\newtheorem{proposition}{Proposition}[section]

\newcommand{\bl}[1] {\boldsymbol{#1}}
\newcommand{\Wt}[1] {\stackrel{\sim}{\smash{#1}\rule{0pt}{1.1ex}}}
\newcommand{\wt}[1] {\widetilde{#1}}


%For boxed texts in align, use Aboxed{}
%otherwise use boxed{}

\DeclareMathSymbol{\widehatsym}{\mathord}{largesymbols}{"62}
\newcommand\lowerwidehatsym{%
  \text{\smash{\raisebox{-1.3ex}{%
    $\widehatsym$}}}}
\newcommand\fixwidehat[1]{%
  \mathchoice
    {\accentset{\displaystyle\lowerwidehatsym}{#1}}
    {\accentset{\textstyle\lowerwidehatsym}{#1}}
    {\accentset{\scriptstyle\lowerwidehatsym}{#1}}
    {\accentset{\scriptscriptstyle\lowerwidehatsym}{#1}}
}

\usepackage{graphicx}
    
% text on arrow for xRightarrow
\makeatletter
%\newcommand{\xRightarrow}[2][]{\ext@arrow 0359\Rightarrowfill@{#1}{#2}}
\makeatother


\def \bx {\boldsymbol{x}}
\def \ba {\boldsymbol{a}}
\def \bI {\boldsymbol{I}}
\def \bt {\boldsymbol{t}}
\def \bb {\boldsymbol{b}}
\def \bA {\boldsymbol{A}}
\def \bX {\boldsymbol{X}}
\def \bu {\boldsymbol{u}}
\def \bS {\boldsymbol{S}}
\def \bZ {\boldsymbol{Z}}
\def \bz {\boldsymbol{z}}
\def \by {\boldsymbol{y}}
\def \bw {\boldsymbol{w}}
\def \bT {\boldsymbol{T}}
\def \bS {\boldsymbol{S}}
\def \bm {\boldsymbol{m}}
\def \bW {\boldsymbol{W}}
\def \bY {\boldsymbol{Y}}
\def \bH {\boldsymbol{H}}
\def \blambda {\boldsymbol{\lambda}}
\def \bPhi {\boldsymbol{\Phi}}
\def \btheta {\boldsymbol{\theta}}
\def \bmu {\boldsymbol{\mu}}
\def \bphi {\boldsymbol{\phi}}
\def \bSigma {\boldsymbol{\Sigma}}
\def \lb {\left\{}
\def \rb {\right\}}
\def \caln {\mathcal{N}}
\def \dissum {\displaystyle\Sigma}
\def \dispro {\displaystyle\prod}
\def \E {\mathbb{E}}
\def \Q {\mathbb{Q}}
\def \V {\mathbb{V}}
\def \R {\mathbb{R}}
\def \calq {\mathcal{Q}}
\def \calg {\mathcal{G}}
\def \caln {\mathcal{N}}
\def \calr {\mathcal{R}}
\def \calm {\mathcal{M}}
\def \calc {\mathcal{C}}
\def \bcup {\bigcup}

\usepackage[UTF8]{ctex}
\DeclareMathOperator{\Char}{Char}
\makeindex
\author{wu}
\date{\today}
\title{Valuation Field}
\hypersetup{
 pdfauthor={wu},
 pdftitle={Valuation Field},
 pdfkeywords={},
 pdfsubject={},
 pdfcreator={Emacs 28.0.92 (Org mode 9.6)}, 
 pdflang={English}}
\begin{document}

\maketitle
\tableofcontents

\section{环与理想}
\label{sec:org87be0fc}
\subsection{介绍}
\label{sec:org40c5303}
\begin{definition}[]
称\(A\)为 \textbf{局部环} ,如果\(A\)只有一个极大理想 \(I\),称\(k=A/I\)为\(A\)的 \textbf{剩余域} (residue field)
\end{definition}

\begin{proposition}[]
\begin{enumerate}
\item 设\(A\)为环,\(I\subsetneq A\)为理想,若每个\(x\in A\setminus I\)均是单位元则\(A\)是局部环,\(I\)是极大
理想
\item 若\(A\)为环,\(I\subseteq A\)为极大理想,若\(\forall a\in I\),有\(1+a\)均是单位元,则\(A\)是局部环
\end{enumerate}
\end{proposition}


\subsection{分式化}
\label{sec:org26f3866}
\begin{definition}[]
设\(A\)是一个整环,令\(A^\times=A\setminus\{0\}\),在\(A\times A^\times\)上定义关系\(\sim\)为
\begin{equation*}
(a,s)\sim(b,t)\Leftrightarrow at-bs=0
\end{equation*}
\end{definition}

\begin{definition}[]
称\(S\subseteq A\)为 \textbf{乘法子集} ,如果\(1\in S\)且\(a,b\in S\Rightarrow ab\in S\)
\end{definition}

\begin{definition}[]
设\(S\subseteq A\)是乘法子集,定义\(A\times S\)上的等价关系\(\sim\)为
\begin{equation*}
(a,s)\sim(b,t)\Leftrightarrow\exists u\in S(u(at-bs)=0)
\end{equation*}
将\((a,s)\)的等价类记作\(\frac{a}{s}\),定义
\begin{equation*}
\frac{a}{s}+\frac{b}{t}=\frac{at+bs}{st},\quad\frac{a}{s}\frac{b}{t}=\frac{ab}{st}
\end{equation*}
则\(A\times S/\sim\)是一个环,记作\(S^{-1}A\)
\end{definition}

\begin{remark}
\begin{itemize}
\item \(\forall x\in A\),\(\frac{xa}{xs}=\frac{a}{s}\)
\item 若\(S\)有零因子,则\(S^{-1}A=0\)平凡
\item \(A\to S^{-1}A\), \(a\mapsto\frac{a}{1}\)是同态
\item 若\(A\)是整环,\(S=A^\times\),则\(S^{-1}=\Frac(A)\)
\end{itemize}
\end{remark}

\begin{examplle}[]
若\(\fp\)是素理想,\(S=A\setminus \fp\)是乘法子集
\begin{itemize}
\item 令\(A_{\fp}=S^{-1}A\)
\item 令\(\fm=\{\frac{a}{s}\mid a\in\fp,s\notin\fp\}=pA_{\fp}=\fp S^{-1}\),则\(A_{\fp}\)是局部环,\(\fm\)是\(A_{\fp}\)的极大理想
\end{itemize}
\end{examplle}



\subsection{多项式环}
\label{sec:orgc0a0e54}
设\(A\)是一个环,则多项式环\(A[X]\)的元素都形如
\begin{equation*}
\sum_{i=0}^na_ix^i,\quad a_i\in A,i\in\N
\end{equation*}

\begin{definition}[]
设\(A\)是环,\(a\in A\) \textbf{不可约} 如果\(a\neq 0\)不是单位元且\(\forall b,c\in A(a=bc\Rightarrow) b\)或\(c\)为单位元

一个整环\(A\)是 \textbf{唯一因子分解环} ,如果\(\forall a\in A\),存在不可约元\(b_1,\dots,b_n\in A\)使得\(a=b_1\dotsb_n\)
并且若存在不可约元\(c_1,\dots,c_m\)使得\(a=c_1\dots c_m\)则\(m=n\),则\(\forall i<n\exists j<n(b_i=u_{ij}c_j)\),其
中\(u_{ij}\)是单位元
\end{definition}

\begin{proposition}[]
若\(A\)是唯一因子分解环,则\(A[X]\)也是
\end{proposition}

\begin{corollary}[]
若\(k\)是域,则\(k[X_1,\dots,X_n]\)是唯一因子分解环
\end{corollary}

\begin{corollary}[]
\(k\)是域,\(f\in k[X_1,\dots,X_n]\),则\((f)\)是素理想\(\Leftrightarrow f\)不可约
\end{corollary}

\begin{proof}
\(\Rightarrow\):\(k[X_1,\dots,X_n]/(f)\)是整环,如果\(f\)可约,则\(f=gh\),其中\(g,h\in k[X_1,\dots,X_n]\)且不是单位
元,于是\(g+(f),h+(f)\)非零,而\((g+(f))(h+(f))=0+(f)\),矛盾

\(\Leftarrow\):对于任意\(g,h,p\in k[X_1,\dots,X_n]\),若\(gh=fp\),因为\(k[X_1,\dots,X_n]\)是唯一因子分解环,
于是\(f\)整除\(g\)或者\(f\)整除\(h\)
\end{proof}
\section{局部环}
\label{sec:orgef8a678}
一个环是局部环当且仅当所有非单位元构成一个理想。等价地,一个环是局部环当且仅当所有非单位元构成一个理想。

在环的语言\(\call_{ring}=\{+,\times,0,1\}\)中局部环可以公理为
\begin{enumerate}
\item \(R\)是环。
\item 所有的非单位元构成一个集合\(\fm\)是理想,即\(\fm\)关于``+''封闭,关于``\(\times\)''吸收。
\end{enumerate}


但是非单位元关于``\(\times\)''总是吸收的,故而(2)可以改为
\begin{enumerate}
\setcounter{enumi}{1}
\item 所有非单位元关于“+”封闭,即\(\fm\)是一个群。
\end{enumerate}

\begin{remark}
\begin{itemize}
\item \(0\in\R\)出解析函数的函数芽的环\(A\)是局部环
\item 一个函数\(f\)在\(0\in\R\)处解析\(\Leftrightarrow\)存在开邻域\(U\ni 0\)使得\(f\)在\(U\)上是个幂级数,即
\(f\uhr_U=\sum_{n=0}^\infty a_nx^n\),其中\(a_n\in\R\)。
\item 显然,\(\sum a_nx^n\sim\sum b_nx^n\Leftrightarrow\forall n(a_n=b_n)\),故而
\begin{equation*}
A=\{f\mid f\text{是幂级数且收敛半径}>0\}
\end{equation*}
\item \(\fm=x A=\{xf\mid f\in A\}\)是唯一的极大理想,其中极大是因为\(A/\fm\cong\R\)。
\end{itemize}
\end{remark}

\begin{examplle}[]
设\(R\)是一个环,称\(\sum_{n=0}^\infty r_nx^n\) (\(r_n\in R\))的元素为\(R\)上的形式幂级数,令\(R[[x]]\)为\(R\)
上所有形式幂级数构成的集合,定义
\begin{enumerate}
\item \(\sum r_nx^n+\sum s_nx^n=\sum(r_n+s_n)x^n\)
\item \(\sum r_nx^n\sum s_nx^n=\sum_n(\sum_{i+j=n}r_is_j)x^n\)
\end{enumerate}


则\((R[[x]],+,\times,0_R,1_R)\)是一个环。
\end{examplle}

\begin{definition}[]
设\(R\)是一个环,称\(R[[x]]\)为\(R\)的 \textbf{形式幂级数环} ,若\(g=\sum r_nx^n\in R[[x]]\),则\(g\)的 \textbf{度数} 记
作\(\deg(g)\),定义为
\begin{equation*}
\deg(g)=\min(n\in\N\mid r_n\neq 0)
\end{equation*}
定义\(\deg(0)=\infty\)。(因此\(\deg(g)\ge 0\))
\end{definition}


\begin{lemma}[]
假设\(R是整环\)
\begin{enumerate}
\item 若\(f\in R[[x]]\),且\(\deg(f)=n\),则
\begin{equation*}
f=x^n(\sum r_kx^k)
\end{equation*}
其中\(r_0\neq 0\),即\(f=x^ng\)其中\(\deg(g)=0\)
\item 若\(f,g\in R[[x]]\),则\(\deg(fg)=\deg(f)+\deg(g)\)
\item 若\(f=\sum r_nx^n\), \(g=\sum s_nx^n\),则\(fg=1\Rightarrow r_0s_0=1\)
\item 若\(f=\sum r_nx^n\),则\(f\)是单位\(\Rightarrow\) \(r_0\)是单位(\(r_0\neq 0\))
\end{enumerate}
\end{lemma}

\begin{proof}
\begin{enumerate}
\item 由定义,若\(f=\sum s_kx^k\)且\(\deg(f)=n\),则\(s_0=\dots=s_{n-1}=0\)且\(s_n\neq 0\),因此
\(f=x^n(\sum_{k=n}^\infty s_kx^k)\),对任意\(i\in\N\),令\(r_i=s_{i+n}\),则\(f=x^n(\sum r_kx^k)\),其
中\(r_0\neq 0\)。
\item 假设\(\deg(f)=n\),\(\deg(g)=m\),则由(1),\(f=x^n(\sum r_kx^k)\),\(g=x^m(\sum s_kx^k)\),其
中\(r_0,s_0\neq 0\),因此\(fg=x^{n+m}\sum_{n=0}^\infty(\sum_{i+j=n}r_is_j)x^n\),因为\(r_0,s_0\neq 0\),\(R\)是
整环,因此\(r_0s_0\neq 0\),因此\(\deg(fg)=n+m=\deg(f)+\deg(g)\)。
\item 由定义,\(fg=\sum_{n=0}^\infty(\sum_{i+j=n}r_is_j)x^n=1\),因此\(r_0s_0=1\)
\item 如果\(f\)是单位,则存在\(g\in R[[x]]\)使得\(fg=1\),由(3),\(r_0\)是单位。
\end{enumerate}
\end{proof}

\begin{proposition}[]
若\(R\)是局部环,则\(R[[x]]\)也是局部环。
\end{proposition}

\begin{proof}
\begin{itemize}
\item 只需验证非单位元关于加法封闭。
\item 设\(f\in R[[x]]\)是单位元,则\(f=r_0+g\),其中\(r_0\)是\(R\)的单位,\(\deg(g)\ge 1\)。
\item 令一方面,若\(f=r_0+g\)且\(r_0\in R\)是单位,\(\deg(g)\ge 1\),取\(s_0\in R\)使得\(s_0r_0=1_R\),则
\(s_0f=1+s_0g\),令\(h=-s_0g\)。
\end{itemize}
\begin{claim}
\(h+h^2+h^3+\dots\in R[[x]]\)
\end{claim}

\begin{proof}
设\(h=\sum s_kx^k\),其中\(s_0=0\),令\(g=\sum_{n=1}^\infty h^n=\sum r_kx^k\),于是\(r_0\in R\),若\(r_0,\dots,r_n\in R\),
则\(r_{n+1}=s_{n+1}+\sum_{i=1}^{n-1}s_ir_{n-i}\in R\),因此对于任意\(k\in\N\),\(r_k\in R\),因此\(g\in R[[x]]\)。
\end{proof}
\begin{itemize}
\item 考虑等式\((1-h)(1+h+h^2+\dots)=1\),则\(s_0f(1+h+h^2+\dots)=1\),故\(f\)是单位,因此
\item \(f\in R[[x]]\)是单位\(\Leftrightarrow f=r_0+g\),其中\(r_0\)是单位且\(\deg(g)\ge 1\)。
\item \(f\in R[[x]]\)不是单位\(\Leftrightarrow \deg(f)\ge 1\)或\(f=r+g\),其中\(r\)不是单位且\(\deg(g)\ge 1\)。
\item \(f\)不是单位\(\Leftrightarrow f\in \fm_0+xR[[x]]=\{r+g\mid r\in\fm_0,g\in xR[x]\}\),其中\(\fm_0\)是\(R\)的极大理想。
\item 显然\(\fm_0+xR[[x]]\)是“+”封闭的,故\(R[[x]]\)是局部环。
\end{itemize}
\end{proof}

\begin{corollary}[]
若\(R\)是局部环,\(\fm_0\)为\(R\)的极大理想,则
\begin{enumerate}
\item \(R[[x]]\)是局部环,其极大理想为
\begin{equation*}
\fm_0+(x)
\end{equation*}
\item 若\(k\)是域,则\(k[[x]]\)中的理想排成一个降链
\begin{equation*}
I_0=\fm_0+(x)\supseteq I_1=(x)\supseteq\dots\supseteq I_n=(x^n)\supseteq\dots
\end{equation*}
\end{enumerate}
\end{corollary}

\begin{proof}
\begin{enumerate}
\item 已证。
\item 设\(J\)是\(k[[x]]\)的理想,令\(n=\min\{\deg(f)\mid f\in J\}\),若\(n=\infty\),则\(J=(0)\)。

若\(n<\infty\)且\(f=x^ng\in J\)其中\(\deg(g)=0\),由于\(g\)的首项是单位,因此\(g\)是单位,令\(h\in R[[x]]\)
使得\(hg=1\),则\(x^n=hf=hgx^n\in J\),因此\((x^n)\subseteq J\),又由\(n\)的定义,\(J\subseteq(x^n)\),所以\(J=(x^n)\)。
\end{enumerate}
\end{proof}

\begin{corollary}[]
若\(k\)是域,则\(k[[x]]\)是局部环,其极大理想为\((x)=xk[[x]]\),剩余域为\(k\)。
\end{corollary}

\begin{corollary}[]
定义\(k[[X_1,\dots,X_{n+1}]]=k[[X_1,\dots,X_n]][[X_{n+1}]]\),则\(k[[X_1,\dots,X_{n+1}]]\)为局部环,其极大理
想\(\fm\)为\((X_1,\dots,X_{n+1})\),剩余域为\(k\)。
\end{corollary}

\begin{examplle}[]
令\(p\in\Z\)是一个素数,
\begin{enumerate}
\item \(\Z/p\Z\)是一个域,这是因为若\(0<r<p\),则\((r,p)=1\),故存在\(m,n\)使得
\begin{equation*}
mr+np=1\Rightarrow mr\equiv_p1
\end{equation*}
故\(\Z/p\Z\)是一个局部环
\item 对每个\(n\in\N^+\),\(\Z/p^n\Z\)是局部环
\begin{itemize}
\item \(\Z\)中包含\((p^n)\)的理想与\(\Z/p^n\Z\)中的理想一一对应
\item \(\Z\)中的理想均形如\((k)\)
\item \((p^n)\subseteq(k)\Leftrightarrow k\mid p^n\Rightarrow k=p^m\),其中\(m\le n\)
\item 故\(\Z/p^n\Z\)中的理想为
\begin{equation*}
p^n\Z/p^n\Z=(0)\subseteq p^{n-1}\Z/p^n\Z\subseteq\dots\subseteq p\Z/p^n\Z
\end{equation*}
\item 故\(p\Z/p^n\Z\)为\(\Z/p^n\Z\)的唯一极大理想,显然\(\Z/p^n\Z\)中有\(p^n\)个元素。

\item \(\Z/p^n\Z\)的元素可唯一表示为
\end{itemize}
\begin{equation*}
a_0+a_1p+\dots+a_{n-1}p^{n-1}
\end{equation*}
其中 \(a_i\in\{0,\dots,p-1\}\)。
\item 若\(m>n\),则\(\Z\to\Z/p^m\Z\)和\(\Z\to\Z/p^n\Z\)诱导了
\begin{center}\begin{tikzcd}
\Z\ar[r]\ar[d]&\Z/p^m\Z\ar[dl]\\
\Z/p^n\Z
\end{tikzcd}\end{center}
\begin{itemize}
\item \(\forall m>n\),令\(\pi_{mn}\)为\(\Z/(p^m)\)到\(\Z/(p^n)\)的自然同态,即
\end{itemize}
\begin{equation*}
\pi_{mn}(a_0+a_1p+\dots+a_{m-1}p^{m-1})=a_0+\dots+a_{n-1}p^{n-1}
\end{equation*}
\begin{itemize}
\item 令\(\Z^*=\prod_{n=1}^\infty\Z/(p^n)=\{(x_1,x_2,\dots)\mid x_n\in\Z/(p^n)\}\),
\item 将\(x_n\)看作\(a_0+\dots+a_{n-1}p^{n-1}\)或序列\((a_0,\dots,a_{n-1})\)
\item 定义\(\Z_p\subseteq\Z^*\)为
\end{itemize}
\begin{equation*}
\{(x_1,x_2,\dots,)\mid\pi_{mn}(x_m)=x_n,m>n\}
\end{equation*}
\begin{itemize}
\item 将\((x_1,x_2,\dots)\)中的每个\(x_n\)看作\(a_0+\dots+a_{n-1}p^{n-1}\),则\((x_1,x_2,\dots)\in \Z_p\Leftrightarrow\forall m>n\), \(x_m\)是\(x_n\)的延长

\item 故而\((x_1,x_2,\dots)\in\Z_p\)唯一对应一个幂级数\(a_0+a_1p+a_2p^2+\dots\)

\item 定义\(\Z^*\)中的+为
\end{itemize}
\begin{equation*}
(x_1,x_2,\dots)+(y_1,y_2,\dots)=(x_1+y_1,x_2+y_2,\dots)
\end{equation*}
\begin{itemize}
\item 定义\(\Z^*\)中的``\(\times\)''为
\end{itemize}
\begin{equation*}
(x_1,x_2,\dots)\cdot(y_1,y_2,\dots)=(x_1y_1,x_2y_2,\dots)
\end{equation*}
\begin{itemize}
\item 定义零为\((0,0,\dots,)\),幺为\((1,1,\dots)\),则\(\Z^*\)为环。

\item 由于每个\(\pi_{mn}\)是同态,故\(\Z_p\)对“+”与“\(\times\)”封闭:对任意\((x_1,x_2,\dots),(y_1,y_2,\dots)\in\Z_p\),
对任意\(m>n\),因为\(\pi_{mn}\)是同态,有\(\pi_{mn}(x_m+y_m)=\pi_{mn}(x_m)+\pi_{mn}(y_m)=x_n+y_n\),
\(\pi_{mn}(x_m\cdot y_m)=\pi_{mn}(x_m)\cdot\pi_{mn}(y_m)=x_n\cdot y_n\),故\((x_1,x_2,\dots)+(y_1,y_2,\dots),(x_1,x_2,\dots)\cdot(y_1,y_2,\dots)\in\Z_p\)。
\item 故\(\Z_p\)是一个环,称其为 \textbf{\(p\)-进整数环} 。
\item \(\Z_p\)也称为\(\Z/(p^n)\)的逆极限,即\(\Z_p=\varprojlim\Z/(p^n)\)
\end{itemize}
\end{enumerate}
\end{examplle}

\begin{remark}
设\(x=(x_1,x_2,\dots)\in\Z_p\),则\(x\)可以记作\(a_0+a_1p+a_2p^2+\dots\),其中每个\(a_i\in\{0,\dots,p-1\}\),因此
\(x_1=a_0\), \(x_2=a_0+a_1p\),\(\dots\),\(x_n=\sum_{k=0}^{n-1}a_kp^k\)。

设\(y=(y_1,y_2,\dots)\in\Z_p\),设它可写作\(b_0+b_1p+\dots\),令\(z=x+y=(x_1+y_1,x_2+y_2,\dots)\),
将\(z\)写作\(\sum_{k=0}^\infty c_kp^k\),则
\begin{equation*}
z_n=x_n+y_n=(\sum_{k=0}^{n-1}a_kp^k+\sum_{k=0}^{n-1}b_kp^k)(\mod p^k)
\end{equation*}
即\(z_n\)是\(x_n+y_n\)的\(p\)-进制展开的前\(n\)项。

同理若\(z=xy\),则\(z_n\)是\(x_ny_n\)的\(p\)-进制展开的前\(n\)项。

故\(\Z_p\)中的运算是“\(p\)-进制”运算。
\end{remark}

\begin{lemma}[]
label:6
若\(A,B\)是局部环,则\(f:A\to B\)是满同态,则\(a\in A\)是单位\(\Leftrightarrow f(a)\in B\)是单位
\end{lemma}

\begin{proof}
\begin{itemize}
\item 令\(\fm\)是\(B\)的极大理想,
\item 则\(\barf:A/f^{-1}(\fm)\to B/\fm\)是同构,
\item 而\(B/\fm\)是域,故\(A/f^{-1}(\fm)\)是域,故\(f^{-1}(\fm)\)是极大理想,
\item 故\(a\in A\)是单位\(\Leftrightarrow a\notin f^{-1}(\fm) \Leftrightarrow\) \(f(a)\notin\fm\)是\(B\)的单位。
\end{itemize}
\end{proof}

\begin{proposition}[]
\begin{enumerate}
\item \(\Z_p\)是局部环
\item \(\Z_p\)的理想排成降链
\begin{equation*}
p\Z_p\supseteq p^2\Z_p\supseteq\dots
\end{equation*}
\item \(\Z_p/p^n\Z_p\cong\Z/p^n\Z\)
\end{enumerate}
\end{proposition}

\begin{proof}
\begin{enumerate}
\item 设\(x=(x_1,x_2,\dots)=a_0+a_1p+\dots\in\Z_p\),即\(x_1=a_0\),\(x_2=a_0+a_1p\),\(\dots\)。

\begin{claim}
\(x\)是单位\(\Leftrightarrow\) \(a_0\neq 0\)
\end{claim}

\begin{proof}
\(\Leftarrow\):
\begin{itemize}
\item 若\(a_0\neq 0\),则\(a_0\in\Z/p\Z\)是单位,
\item 故存在\(b_0\in\Z/p\Z\)使得\(a_0b_0\equiv 1\mod p\)。
\item 由于 \(\pi_{21}\)是同态,而\(a_0=\pi_{21}(a_0+a_1p)\)是单位,由引理\ref{6},\(a_0+a_1p\in\Z/p^2\Z\)也是单位,
\item 同理,\(\forall b_1\in\{0,\dots,p-1\}\),\(b_0+b_1p\in\Z/p^2\Z\)是单位,
\item 令\(c_0+c_1p\in\Z/p^2\Z\)使得
\begin{equation*}
(a_0+a_1p)(c_0+c_1p)=1\in\Z/p^2\Z
\end{equation*}
\item 则\(\pi_{21}((a_0+a_1p)(c_0+c_1p))=a_0c_0=1=a_0b_0\)。
\item 故\(a_0c_0-a_0b_0\equiv 0\mod p\),因此\(c_0\equiv b_0\mod p\),所以\(c_0=b_0\)。
\item 一般地,设\(b_0+b_1x+\dots+b_{n-1}x^{n-1}\in\Z/(p^n)\)使得
\((a_0+\dots+a_{n-1}x^{n-1})(b_0+\dots+b_{n-1}x^{n-1})=1\in\Z/p^n\Z\),
\item 则存在\(b_n\in\{0,\dots,p-1\}\)使得在\(\Z/(p^{n+1})\)中有
\((a_0+\dots+a_nx^n)(b_0+\dots+b_nx^n)=1\)。
\item 令\(y=b_0+b_1+\dots=(y_1,y_2,\dots)\),则\(xy=1\),故\(x\)是单位。
\end{itemize}

\(\Rightarrow\):若\(a_0=0\),则\(x=(0,x_2,\dots)\)显然不是单位。
\end{proof}

以上断言表明,所有非单位元形如\(x=(0,x_2,x_3,\dots)\)是一个加法群,故而是极大理想,恰好是\(p\Z_p\)

\item 设\(J\subseteq\Z_p\)是一个非平凡理想
\begin{itemize}
\item 令\(k=\min\{n\in\N\mid p^n\in J\}\),显然\(k>0\),\(p^k\Z_p\subseteq J\)
\item 断言\(p^k\Z_p=J\)。
\item 设\(x=a_0+a_1p+\dots\in J\),令\(a_m\)是第一个非零系数
\item 则\(x=p^m(a_m+a_{m+1}p+\dots)\),
\item 因为\(a_m\neq 0\),\(a_m+a_{m+1}p+\dots\)是单位,故存在\(y\in\Z_p\)使得\(xy=p^m\in J\)
\item 由定义,\(k\le m\Rightarrow p^m\in p^k\Z_p\Rightarrow x\in p^k\Z_p\),
\item 即\(\Z_p\)的每个非平反理想都形如\(p^k\Z_p\)。
\end{itemize}
\item 投射函数诱导了一个同态
\begin{center}\begin{tikzcd}
\Z^*\ar[r,"\pi_n"]&\Z/(p^n)\\
\Z_p\ar[u]\ar[ur,"\pi_n"']
\end{tikzcd}\end{center}
其中 \(\pi_n:\Z_p\to\Z/(p^n)\),\(x=(x_1,\dots,x_n,\dots)\mapsto x_n\),于是
\begin{align*}
x\in\ker(\pi_n)&\Leftrightarrow x_n=0\\
&\Leftrightarrow x=(0,\dots,0,x_{n+1},\dots)\\
&\Leftrightarrow x=a_{n}p^n+a_{n+1}p^{n+1}\dots\\
&\Leftrightarrow x\in p^n\Z_p
\end{align*}
\end{enumerate}
\end{proof}

\begin{remark}
证明\(\Z_p\)是局部环的关键是验证
\begin{equation*}
x=a_0+a_1p+\dots\text{是单位}\Leftrightarrow a_0\neq 0
\end{equation*}
\end{remark}

以下证明更简洁:
\begin{itemize}
\item 设\(x=(x_1,x_2,\dots)\in\Z_p\subseteq\prod\Z/(p^n)\),\(x_1=a_0,\dots,x_n=a_0+a_1p+\dots+a_{n-1}p^{n-1},\dots\)
\item 由于每个\(\Z/(p^n)\)都是局部环且\(p\Z/(p^n)\)是其极大理想,
\item 故每个\(x_n\)在\(\Z/(p^n)\)中可逆,令\(y_n\)是\(x_n\)在\(\Z/(p^n)\)的逆
\item \(\pi_{mn}(x_my_m)=\pi_{mn}(x_m)\pi_{mn}(y_m)=x_n\pi_{mn}(y_m)=1\),
\item 故\(\forall n<m\),\(\pi_{mn}(y_m)\)都是\(x_n\)的逆
\item 断言:\(\pi_{mn}(y_m)=y_n\)
\item \(x_n(y_n-\pi_{mn}(y_m))=0\Rightarrow y_nx_n(y_n-\pi_{mn}(y_m))=0\),
\item 故\(y=(y_1,y_2,\dots)\)是\(x\)的逆
\end{itemize}

更加简洁的方法:
\begin{itemize}
\item 取\(b\in\{0,\dots,p-1\}\)使得\(a_0\cdot b\equiv 1\mod p\),
\item 则\(bx=1+p(b_0+b_1p+\dots)=1-py\),
\item 令\(c=1+py+p^2y^2+\dots\in\Z_p\),
\item 则\(bxc=(1-py)(1+py+(py)^2+\dots)=1\)
\end{itemize}

\begin{remark}
\begin{itemize}
\item \(\Z\mapsto\Z_p\), \(x\mapsto x\)的\(p\)-进制展开是一个单同态。
\item \(\Z\)中不能被\(p\)整除的元素都是\(\Z_p\)的单位。
\item 令\(S=\Z-(p)\),则\(S\)是乘法集,\(\Z\)关于\((p)\)的局部化\(\Z_{(p)}=S^{-1}\Z\subseteq\Q\)是局部环,
且\(pS^{-1}\Z\)是极大理想
\item \(\Z_{(p)}=\{\frac{a}{b}:a,b\in\Z,b\nmid b\}\subseteq\Q\)
\item \(\Z\)到\(\Z_p\)的嵌入自然地扩张为\(\Z_{(p)}\)到\(\Z_p\)的嵌入
\begin{center}\begin{tikzcd}
f:\Z\to\Z_p\ar[d]\\
\parbox{3cm}{\centering \(\tilde{f}:S^{-1}\Z\to\Z_p\) \(\frac{a}{b}\mapsto(f(b))^{-1}a\)}
\end{tikzcd}\end{center}
\item \(\Z_p\cap\Q=\Z_{(p)}\)
\item 在形式上,\(\Z_p\)与\(\F_p[[X]]\)有相似之处,然而\(\Char(\Z_p)=0\),而\(\Char(\F_p[[X]])=p\)
\end{itemize}
\end{remark}
\end{document}
