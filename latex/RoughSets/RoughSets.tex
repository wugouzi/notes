% Created 2019-07-12 五 13:50
% Intended LaTeX compiler: pdflatex
\documentclass[11pt]{article}
\usepackage[utf8]{inputenc}
\usepackage[T1]{fontenc}
\usepackage{graphicx}
\usepackage{grffile}
\usepackage{longtable}
\usepackage{wrapfig}
\usepackage{rotating}
\usepackage[normalem]{ulem}
\usepackage{amsmath}
\usepackage{textcomp}
\usepackage{amssymb}
\usepackage{capt-of}
\usepackage{hyperref}
\usepackage{minted}
% TIPS
% \substack{a\\b} for multiple lines text





% pdfplots will load xolor automatically without option
\usepackage[dvipsnames]{xcolor}

\usepackage{forest}
% two-line text in node by [two \\ lines]
% \begin{forest} qtree, [..] \end{forest}
\forestset{
  qtree/.style={
    baseline,
    for tree={
      parent anchor=south,
      child anchor=north,
      align=center,
      inner sep=1pt,
    }}}
%\usepackage{flexisym}
% load order of mathtools and mathabx, otherwise conflict overbrace

\usepackage{mathtools}
%\usepackage{fourier}
\usepackage{pgfplots}
\usepackage{amsthm, mathabx,  amsmath, commath}
\usepackage{amsfonts}

\usepackage{empheq}
\usepackage{tikz}
\usetikzlibrary{arrows.meta}
\usepackage[most]{tcolorbox}

\newtheorem{theorem}{Theorem}[section]
\newtheorem{definition}{Definition}[section]
\newtheorem{corollary}{Corollary}[section]
\newtheorem{example}{Example}[section]
\newtheorem{lemma}{Lemma}[section]
\newtheorem{proposition}{Proposition}[section]

\newcommand{\bl}[1] {\boldsymbol{#1}}
\newcommand{\Wt}[1] {\stackrel{\sim}{\smash{#1}\rule{0pt}{1.1ex}}}
\newcommand{\wt}[1] {\widetilde{#1}}


%For boxed texts in align, use Aboxed{}
%otherwise use boxed{}

\DeclareMathSymbol{\widehatsym}{\mathord}{largesymbols}{"62}
\newcommand\lowerwidehatsym{%
  \text{\smash{\raisebox{-1.3ex}{%
    $\widehatsym$}}}}
\newcommand\fixwidehat[1]{%
  \mathchoice
    {\accentset{\displaystyle\lowerwidehatsym}{#1}}
    {\accentset{\textstyle\lowerwidehatsym}{#1}}
    {\accentset{\scriptstyle\lowerwidehatsym}{#1}}
    {\accentset{\scriptscriptstyle\lowerwidehatsym}{#1}}
}

\usepackage{graphicx}
    
% text on arrow for xRightarrow
\makeatletter
%\newcommand{\xRightarrow}[2][]{\ext@arrow 0359\Rightarrowfill@{#1}{#2}}
\makeatother


\def \bx {\boldsymbol{x}}
\def \ba {\boldsymbol{a}}
\def \bI {\boldsymbol{I}}
\def \bt {\boldsymbol{t}}
\def \bb {\boldsymbol{b}}
\def \bA {\boldsymbol{A}}
\def \bX {\boldsymbol{X}}
\def \bu {\boldsymbol{u}}
\def \bS {\boldsymbol{S}}
\def \bZ {\boldsymbol{Z}}
\def \bz {\boldsymbol{z}}
\def \by {\boldsymbol{y}}
\def \bw {\boldsymbol{w}}
\def \bT {\boldsymbol{T}}
\def \bS {\boldsymbol{S}}
\def \bm {\boldsymbol{m}}
\def \bW {\boldsymbol{W}}
\def \bY {\boldsymbol{Y}}
\def \bH {\boldsymbol{H}}
\def \blambda {\boldsymbol{\lambda}}
\def \bPhi {\boldsymbol{\Phi}}
\def \btheta {\boldsymbol{\theta}}
\def \bmu {\boldsymbol{\mu}}
\def \bphi {\boldsymbol{\phi}}
\def \bSigma {\boldsymbol{\Sigma}}
\def \lb {\left\{}
\def \rb {\right\}}
\def \caln {\mathcal{N}}
\def \dissum {\displaystyle\Sigma}
\def \dispro {\displaystyle\prod}
\def \E {\mathbb{E}}
\def \Q {\mathbb{Q}}
\def \V {\mathbb{V}}
\def \R {\mathbb{R}}
\def \calq {\mathcal{Q}}
\def \calg {\mathcal{G}}
\def \caln {\mathcal{N}}
\def \calr {\mathcal{R}}
\def \calm {\mathcal{M}}
\def \calc {\mathcal{C}}
\def \bcup {\bigcup}

\author{wu}
\date{\today}
\title{Rough Sets: Theoretical aspects of reasoning about data}
\hypersetup{
 pdfauthor={wu},
 pdftitle={Rough Sets: Theoretical aspects of reasoning about data},
 pdfkeywords={},
 pdfsubject={},
 pdfcreator={Emacs 26.2 (Org mode 9.2.4)}, 
 pdflang={English}}
\begin{document}

\maketitle
\tableofcontents

\section{Knowledge}
\label{sec:org615808c}
\subsection{Knowledge base}
\label{sec:orgc57fb02}
Given a finite set \(U\neq \emptyset\) (the universe). Any subset \(X\subset U\)
of the universe is called a \textbf{concept} or a \textbf{category} in \(U\). And any family of
concepts in \(U\) will be referred to as \textbf{abstract knowledge} about \(U\).

\textbf{partition} or \textbf{classification} of a certain universe \(U\) is a family 
\(C=\lb X_1,X_2,\dots,X_n\rb\) s.t. \(X_i\subset U,X_i\neq\emptyset,X_i\cap
   X_j=\emptyset\) and \(\bigcup X_i=U\)

A family of classifications is called a \textbf{knowledge base} over \(U\)


\(R\) an equivalence relation over \(U\), \(U/R\) family of all equivalence classes
of \(R\), referred to be \textbf{categories} or \textbf{concepts} of \(R\), and \([x]_R\) denotes a
category in \(R\) containing an element \(x\in U\)

By a \textbf{knowledge base} we can understand a relational system \(K=(U,\bR)\), \(\bR\)
is a family of equivalence relations over \(U\)

If \(\bP\subset \bR\) and \(\bP\neq\emptyset\), then \(\bigcap\bP\) is also an
equivalence relation, and will be denoted by \(IND(\bP)\), called an
\textbf{indiscernibility relation} over \(\bP\)
\begin{equation*}
[x]_{IND(\bP)}=\bigcap_{R\in\bP}[x]_R
\end{equation*}

\(U/IND(\bP)\) called \(\bP\textbf{-basic}\) \textbf{knowledge about} \(U\) in \(K\). For
simplicity, \(U/\bP=U/IND(\bP)\) and \(\bP\) will be also called
\(\bP\textbf{-basic}\) \textbf{knowledge}
. Equivalence classes of \(IND(\bP)\) are called
\textbf{basic categories} of knowledge \(\bP\). If \(Q\in\bR\), then \(Q\) is a
\(Q\textbf{-elementary}\) \textbf{knowledge} and equivalence classes of \(Q\) are referred
to as \(Q\textbf{-elementary}\) \textbf{concepts} of knowledge \(\bR\)

The family of all \(\bP\text{-basic}\) categories for all
\(\empty\neq\bP\subset\bR\) will be called the \textbf{family of basic categories} in
knowledge base \(K=(U,\bR)\)

Let \(K=(U,\bR)\) be a knowledge base. By \(IND(K)\) we denote the family of all
equivalence relations defined in \(K\) as \(IND(K)=\lb
   IND(\bP):\emptyset\neq\bP\subseteq\bR\rb\).

Thus \(IND(K)\) is the minimal set of equivalence relations.

Every union of \(\bP\text{-basic}\) categories will be \(\bP\textbf{-category}\)

The family of all categories in the knowledge base \(K=(U,\bR)\) will be
referred to as \(K\textbf{-categories}\)
\subsection{Equivalence, generalization and specialization of knowledge}
\label{sec:org2bc94fc}
Let \(K=(U,\bP),K'=(U,\bQ)\). \(K\) and \(K'\) are \textbf{equivalent} \(K\simeq
   K',(\bP\simeq\bQ)\) if \(IND(\bP)=IND(\bQ)\). Hence \(K\simeq K'\) if both \(K\) and
\(K'\) have the same set of elememtary categories. \emph{This means that knowledge in
knowledge bases \(K\) and \(K'\) enables us to express exactly the same facts about the universe.}

If \(IND(\bP)\subset IND(\bQ)\) then knowledge \(\bP\) is \textbf{finer} than knowledge
\(\bQ\) (\textbf{coarser}). \(\bP\) is \textbf{specialization} of \(\bQ\) and \(\bQ\) is \textbf{generalization}
of \(\bP\)
\section{Imprecise categories, approximations and rough sets}
\label{sec:org189c36a}
\subsection{Rough sets}
\label{sec:org6f01bd0}
Let \(X\subseteq U\). \(X\) is \(R\textbf{-definable}\) or \(R\textbf{-exact}\) if \(X\) is the union of some
\(R\text{-basic}\) categories. otherwise
\(R\textbf{-undefinable},R\textbf{-rough},R\textbf{-inexact}\)  .
\subsection{Approximations of set}
\label{sec:org81e9578}
Given \(K=(U,\bR), R\in IND(K)\)
\begin{align*}
&\underline{R}X=\bigcup\lb Y\in U/R:Y\subseteq X\rb\\
&\overline{R}X=\bigcup\lb Y\in U/R:Y\cap X\neq\emptyset\rb\\
\end{align*}
called the \(R\textbf{-lower}\) and \(R\textbf{-upper}\) \textbf{approximation} of \(X\)

\(BN_R(X)=\overline{R}X-\underline{R}X\) is \(R\textbf{-boundary}\) of \(X\).
\(BN_R(X)\) is the set of elements which cannot be classified either to \(X\) or
to \(-X\) having knowledge \(R\)

\begin{align*}
&POS_R(X)=\underline{R}X,R\text{-positive region of } X\\
&NEG_R(X)=U-\overline{R}X,R\text{-negative region of } X\\
&BN_R(X) - R\text{-borderline region of } X\\
\end{align*}

If \(x\in POS(X)\), then \(x\) will be called an \(R\textbf{-positive}\) \textbf{example of} \(X\)

\begin{proposition}
\begin{enumerate}
\item $X$ is $R$-definable if and only if $\underline{R}X=\overline{R}X$
\item $X$ is rought w.r.t. $R$ if and only if $\underline{R}X\neq\overline{R}X$
\end{enumerate}
\end{proposition}
\subsection{Properties of approximations}
\label{sec:org88253a1}
\begin{proposition}[2.2]
\begin{enumerate}
\item \(\uR X\subseteq X\subseteq \oR X\)
\item \(\uR\emptyset=\uR\emptyset=\emptyset;\quad \uR U=\oR U=U\)
\item \(\oR(X\cup Y)=\oR X\cup \oR Y\)
\item \(\uR(X\cap Y)=\uR X\cap \uR Y\)
\item \(X\subseteq Y\) implies \(\uR X\subseteq \uR Y\)
\item \(X\subseteq Y\) implies \(\oR X\subseteq\oR Y\)
\item \(\uR(X\cup Y)\subseteq \uR X\cup \uR Y\)
\item \(\uR(-X)=-\oR X\)
\item \(\oR(-X)=-\uR X\)
\item \(\oR(-X)=-\uR X\)
\item \(\uR\uR X=\oR\uR X=\uR X\)
\item \(\oR\oR X=\uR\oR X=\oR X\)
\end{enumerate}
\end{proposition}

The equivalence relation \(R\) over \(U\) uniquely defines a topological space
\(T=(U,DIS(R))\) where \(DIS(R)\) is the familty of all open and closed set in
\(T\) and \(U/R\) is a base for \(T\). The \(R\text{-lower}\) and \(R\text{-upper}\)
approximation of \(X\) in \(A\) are \textbf{interior} and \textbf{closure} operations in the
topological space \(T\)
\subsection{Approximations and membership relation}
\label{sec:org6e137fb}
\begin{align*}
&x\underline{\in}_RX \text{if and only if } x\in\underline{R}X\\
&x\overline{\in}_RX \text{if and only if } x\in\overline{R}X\\
\end{align*}
where \(\underline{\in}_R\) read "\(x\) \textbf{surely belongs} to \(X\) w.r.t. \(R\)" and
\(\overline{\in}_R\) - "\(x\) \textbf{possibly belongs} to \(X\) w.r.t. \(R\)". The \textbf{lower} and
\textbf{upper} membership.
\begin{proposition}
\begin{enumerate}
\item $x\uin X$ implies $x\in X$ implies $x\oin X$
\item $X\subset Y$ implies ($x\uin X$ implies $x\uin Y$ and $x\oin X$ implies $x\oin Y$)
\item $x\oin(X\cup Y)$ if and only if $x\oin X$ or $x\oin Y$
\item $x\uin(X\cap Y)$ if and only if $x\uin X$ and $x\uin Y$
\item $x\uin X$ or $x\uin Y$ implies $x\uin (X\cup Y)$
\item $x\oin X\cap Y$ implies $x\oin X$ and $x\oin Y$
\item $x\uin (-X)$ if and only if non $x\oin X$
\item $x\oin (-X)$ if and only if non $x\uin X$
\end{enumerate}
\end{proposition}
\subsection{Numerical characterization of imprecision}
\label{sec:orgd5eac26}
\textbf{accuracy measure}
\begin{equation*}
\alpha_R(X)=\frac{card\;\uR}{card\;\oR}
\end{equation*}
\subsection{Topological characterization of imprecision}
\label{sec:org35c727d}
\begin{definition}[]
\begin{enumerate}
\item If \(\uR X\neq\emptyset\) and \(\oR X\neq U\), then we say that \(X\) is
\textbf{roughly R-definable}. We can decide whether some elements belong to \(X\)
or \(-X\)
\item If \(\uR X=\emptyset\) and \(\oR X\neq U\), then we say that \(X\) is
\textbf{internally R-undefinable}. We can decide whether some elemnts belong
to \(-X\)
\item If \(\uR X\neq\emptyset\) and \(\oR X=U\), then we say that \(X\) is
\textbf{externally R-undefinable}. We can decide whether some elements belong
to \(X\)
\item If \(\uR X=\emptyset\) and \(\oR X=U\), then we say that \(X\) is
\textbf{totally R-undefinable}. unable to decide
\end{enumerate}
\end{definition}

\begin{proposition}[2.4]
\begin{enumerate}
\item Set \(X\) is R-definable(roughly R-definable, totally R-undefinable) if and
only if so is \(-X\)
\item Set \(X\) is externally R-undefinable if and only if \(-X\) is internally
R-undefinable
\end{enumerate}
\end{proposition}

\begin{proof}
\begin{enumerate}
\item \begin{align*}
R\text{-definable}&\Leftrightarrow \uR X=\oR X, \uR\neq\emptyset,\oR\neq U\\
&\Leftrightarrow -\uR X=-\oR X\\
&\Leftrightarrow \oR(-X)=\uR(-X)\\
\end{align*}

\begin{align*}
X \text{ is roughly } R\text{-definable}
&\Leftrightarrow \uR X\neq \emptyset\wedge\oR X\neq U\\
&\Leftrightarrow -\uR X\neq U\wedge -\oR X\neq \emptyset\\
&\Leftrightarrow \oR(-X)\neq U\wedge \uR(-X)\neq \emptyset\\
\end{align*}
\end{enumerate}
\end{proof}
\subsection{Approximation of classifications}
\label{sec:orge6009ad}
If \(F=\lb X_1,\dots,X_n\rb\) is a family of non empty sets, then
\(\uR F=\lb \uR X_1,\dots,\uR X_n\rb\) and \(\oR F=\lb\oR X_1,\dots,\oR X_n\rb\),
called the \(R\textbf{-lower}\) \textbf{approximation} and the \(R\textbf{-upper}\)
\textbf{approximation} of the family \(F\)

The \textbf{accuracy of approximation} of \(F\) by \(R\) is
\begin{equation*}
\alpha_R(F)=\frac{\displaystyle\sum card\;\uR X_i}
{\displaystyle\sum card\;\oR X_i}
\end{equation*}

\textbf{quality of approximation} of \(F\) by \(R\)
\begin{equation*}
\gamma_R(F)=\frac{\displaystyle\sum card\;\uR X_i}{card\; U}
\end{equation*}

\begin{proposition}[2.5]
Let \(F=\lb X_1,\dots,X_n\rb\) where \(n>1\) be a classification of \(U\) and let
\(R\) be an equivalence relation. If there exists \(i\in\lb 1,2,\dots,n\rb\) s.t.
\(\uR X_i\neq\emptyset\), then for each \(j\neq i\) and \(j\in\lb 1,\dots,n\rb\),
\(\oR X_j\neq U\)
\end{proposition}

\begin{proof}
If \(\uR X_i\neq\emptyset\) then there exists \(x\in X\) s.t. \([x]_R\subseteq X\),
which implies \([x]_R\cap X_j=\emptyset\) for each \(j\neq i\). This yields \(\oR
   X_j\cap[x]_R=\emptyset\).
\end{proof}

\begin{proposition}[2.6]
Let \(F=\lb X_1,\dots,X_n\rb,n>1\) be a classification of \(U\) and let \(R\) be an
equivalence relation. If there exists \(i\in\lb 1,\dots,n\rb\) s.t. \(\oR
   X_i=U\), then for each \(j\neq i\) and \(j\in\lb 1,\dots,n\rb\) \(\uR X_j=\emptyset\)
\end{proposition}

\begin{proposition}[2.7]
Let \(F=\lb X_1,\dots,X_n\rb,n>1\) be a classification of \(U\) and let \(R\) be an
equivalence relation. If for each \(i\in\lb 1,2,\dots,n\rb\) \(\uR
   X_i\neq\emptyset\) holds, then \(\oR X_i\neq U\) for each \(i\in\lb 1,\dots,n,\rb\)
\end{proposition}

\begin{proposition}[]
Let \(F=\lb X_1,\dots,X_n\rb,n>1\) be a classification of \(U\) and let \(R\) be an
equivalence relation. If for each \(i\in\lb 1,2,\dots,n\rb\) \(\oR X_i=U\) holds,
then \(\uR X_i=\emptyset\) for each \(i\in\lb 1,\dots,n\rb\)
\end{proposition}
\subsection{Rough equality of sets}
\label{sec:org14e4f99}
\begin{definition}[]
Let \(K=(U,\bR)\) be a knowledge base, \(X,Y\subseteq U\) and \(R\in IND(K)\), then
\begin{enumerate}
\item Sets \(X\) and \(Y\) are \textbf{bottom} \(R\textbf{-equal}\) \((X\eqsim_R Y)\) if \(\uR X=\uR
      Y\)
\item Sets \(X\) and \(Y\) are \textbf{top} \(R-\textbf{equal}\) \((X\simeq_R Y)\) if \(\oR X=\oR
      Y\)
\item Sets \(X\) and \(Y\) are \(R\textbf{-equal}\) \((X\approx_R Y)\) if \(X\simeq_R Y\)
and \(X\eqsim_R Y\)
\end{enumerate}
\end{definition}

\begin{proposition}[2.9]
\begin{enumerate}
\item \(X\eqsim Y\) iff \(X\cap Y\eqsim X\) and \(X\cap Y\eqsim Y\)
\item \(X\simeq Y\) iff \(X\cup Y\simeq X\) and \(X\cup Y\simeq Y\)
\item If \(X\simeq X'\) and \(Y\simeq Y'\) then \(X\cup Y\simeq X'\cup Y'\)
\item If \(X\eqsim X'\) and \(Y\eqsim Y'\) then \(X\cap Y\eqsim X'\cap Y'\)
\item If \(X\simeq Y\), then \(X\cup -Y\simeq U\)
\item If \(X\eqsim Y\), then \(X\cap -Y\eqsim\emptyset\)
\item If \(X\subseteq Y\) and \(Y\simeq\emptyset\), then \(X\simeq\emptyset\)
\item If \(X\subseteq Y\) and \(X\subseteq U\) then \(Y\subseteq U\)
\item \(X\simeq Y\) iff \(-X\eqsim -Y\)
\item If \(X\eqsim \emptyset\) or \(Y\eqsim\emptyset\), then \(X\cap
       Y\eqsim\emptyset\)
\item If \(X\simeq U\) or \(Y\simeq U\), then \(X\cup Y\simeq U\)
\end{enumerate}
\end{proposition}

\begin{proposition}[2.10 ]
For any equivalence relation \(R\)
\begin{enumerate}
\item \(\uR X\) is the intersection of all \(Y\subseteq U\) s.t. \(X\eqsim_R Y\)
\item \(\oR\) is the union of all \(Y\subseteq U\) s.t. \(X\simeq_R Y\)
\end{enumerate}
\end{proposition}
\subsection{Rough inclusion of sets}
\label{sec:orgb5f6147}
\begin{definition}[]
Let \(K=(U,\bR)\) be a knowledge base, \(X,Y\subseteq U\) and \(R\in IND(K)\).
\begin{enumerate}
\item Set \(X\) is \textbf{bottom} \(R\textbf{-included}\) in \(Y\) \((X\subsetsim_R  Y)\) iff \(\uR
      X\subseteq\uR Y\)
\item Set \(X\) is \textbf{top} \(R\textbf{-included}\) in \(Y\) \((X\simsubset_R Y)\) iff \(\oR
      X\subseteq \oR Y\)
\item Set \(X\) is \(R\textbf{-included}\) in \(Y\) \((X\simsubsetsim_R Y)\) iff
\(X\simsubset_R Y\) and \(X\subsetsim_R Y\)
\end{enumerate}
\end{definition}

\begin{proposition}[2.11]
\begin{enumerate}
\item If \(X\subseteq Y\), then \(X\subsetsim Y, X\simsubset Y\) and \(X\simsubsetsim
      Y\)
\item If \(X\subsetsim Y\) and \(Y\subsetsim X\), then \(X\eqsim Y\)
\item If \(X\simsubset Y\) and \(Y\simsubset X\), then \(X\simeq Y\)
\item If \(X\simsubsetsim Y\) and \(Y\simsubsetsim X\) then \(X\approx Y\)
\item \(X\simsubset Y\) iff \(X\cup Y\simeq Y\)
\item \(X\subsetsim Y\) iff \(X\cap Y\eqsim X\)
\item If \(X\subseteq Y, X\eqsim X',Y\eqsim Y'\), then \(X'\subsetsim Y'\)
\item If \(X\subseteq Y, X\simeq X',Y\simeq Y'\), then \(X'\simsubset Y'\)
\item If \(X\subseteq Y, X\approx X',Y\approx Y'\), then \(X'\simsubsetsim Y'\)
\item If \(X'\simsubset X\) and \(Y'\simsubset Y\), then \(X'\cup Y'\simsubset X\cup
       Y\)
\item If \(X'\subsetsim X,Y'\subsetsim\) then \(X'\cap Y'\subsetsim X\cap Y\)
\item \(X\cap Y\subsetsim X\simsubset X\cup Y\)
\item If \(X\subsetsim Y\) and \(X\eqsim Z\) then \(Z\subsetsim Y\)
\item If \(X\simsubset Y\) and \(X\simeq Z\) then \(Z\simsubset Y\)
\item If \(X\simsubsetsim Y\) and \(X\approx\) then \(Z\simsubsetsim Y\)
\end{enumerate}
\end{proposition}
\section{Reduction of knowledge}
\label{sec:org9ba5acf}
\subsection{Reduct and Core of Knowledge}
\label{sec:org7e53603}
Let \(\bR\) be a family of equivalence relations and let \(P\in\bR\). \(R\) is
\textbf{dispensable} in \(\bR\) if \(IND(\bR)=IND(\bR-\lb R\rb)\). Otherwise \(R\) is
\textbf{indispensable} in \(\bR\). The family of \(\bR\) is \textbf{independent} if each \(R\in\bR\)
is indispensable in \(\bR\). Otherwise \(\bR\) is \textbf{dependent}

\begin{proposition}[3.1]
If \(\bR\) is independent and \(\bP\subseteq \bR\), then \(\bP\) is also independent
\end{proposition}

\begin{proof}
\(IND(\bR)=IND(\bP\cup(\bR-\bP))=IND(\bP)\cap IND(\bR-\bP)\)
\end{proof}

\(\bQ\subseteq \bR\) is a \textbf{reduct} of \(\bP\) if \(\bQ\) is independent and
\(IND(\bQ)=IND(\bP)\)


The set of all indispensable relations in \(\bP\) is called the \textbf{core} of \(\bP\)
denoted by \(CORE(\bP)\)

\begin{proposition}[3.2]
\begin{equation*}
CORE(\bP)=\bigcap RED(\bP)
\end{equation*}
where \(RED(\bP)\) is the family of all reducts of \(\bP\)
\end{proposition}

\begin{proof}
If \(\bQ\) is a reduct of \(\bP\) and \(R\in\bP-\bQ\), then \(IND(\bP)=IND(\bQ)\). If
\(\bQ\subseteq\bR\subseteq\bP\) then \(IND(\bQ)=IND(\bR)\). Assuming \(\bR=\bP-\lb
   R\rb\) then \(R\notin CORE(\bP)\)

If \(R\notin CORE(\bP)\). This means \(IND(\bP)=IND(\bP-\lb R\rb)\) which implies
that there exists an independent subset \(\bS\subseteq \bP-\lb R\rb\) s.t.
\(IND(\bS)=IND(\bP)\). Hence \(R\notin\bigcap RED(\bP)\)
\end{proof}
\subsection{Relative reduct and relative core of knowledge}
\label{sec:org81e75da}
Let \(P\) and \(Q\) be equivalence relations over \(U\)

\(P\textbf{-positive}\)
\begin{equation*}
POS_P(Q)=\displaystyle\bigcup_{X\in U/Q}\uP X
\end{equation*}
The \(P\text{-positive}\) region of \(Q\) is the set of all objects of the
universe \(U\) which can be properly classified to classes of \(U/Q\) employing
knowledge expressed by the classification \(U/P\)


Let \(\bP\) and \(\bQ\) be families of equivalence relations over \(U\)

\(R\in\bP\) is \(\bQ\textbf{-dispensable}\) in \(\bP\) if
\begin{equation*}
POS_{IND(\bP)}(IND(\bQ))=POS_{IND(\bP-\lb R\rb)}(IND(\bQ))
\end{equation*}
otherwise \(R\) is \(\bQ\text{-indispensable}\) in \(\bP\)

If every \(R\) 
\end{document}