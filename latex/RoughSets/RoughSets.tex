% Created 2019-07-10 三 17:49
% Intended LaTeX compiler: pdflatex
\documentclass[11pt]{article}
\usepackage[utf8]{inputenc}
\usepackage[T1]{fontenc}
\usepackage{graphicx}
\usepackage{grffile}
\usepackage{longtable}
\usepackage{wrapfig}
\usepackage{rotating}
\usepackage[normalem]{ulem}
\usepackage{amsmath}
\usepackage{textcomp}
\usepackage{amssymb}
\usepackage{capt-of}
\usepackage{hyperref}
\usepackage{minted}
% TIPS
% \substack{a\\b} for multiple lines text





% pdfplots will load xolor automatically without option
\usepackage[dvipsnames]{xcolor}

\usepackage{forest}
% two-line text in node by [two \\ lines]
% \begin{forest} qtree, [..] \end{forest}
\forestset{
  qtree/.style={
    baseline,
    for tree={
      parent anchor=south,
      child anchor=north,
      align=center,
      inner sep=1pt,
    }}}
%\usepackage{flexisym}
% load order of mathtools and mathabx, otherwise conflict overbrace

\usepackage{mathtools}
%\usepackage{fourier}
\usepackage{pgfplots}
\usepackage{amsthm, mathabx,  amsmath, commath}
\usepackage{amsfonts}

\usepackage{empheq}
\usepackage{tikz}
\usetikzlibrary{arrows.meta}
\usepackage[most]{tcolorbox}

\newtheorem{theorem}{Theorem}[section]
\newtheorem{definition}{Definition}[section]
\newtheorem{corollary}{Corollary}[section]
\newtheorem{example}{Example}[section]
\newtheorem{lemma}{Lemma}[section]
\newtheorem{proposition}{Proposition}[section]

\newcommand{\bl}[1] {\boldsymbol{#1}}
\newcommand{\Wt}[1] {\stackrel{\sim}{\smash{#1}\rule{0pt}{1.1ex}}}
\newcommand{\wt}[1] {\widetilde{#1}}


%For boxed texts in align, use Aboxed{}
%otherwise use boxed{}

\DeclareMathSymbol{\widehatsym}{\mathord}{largesymbols}{"62}
\newcommand\lowerwidehatsym{%
  \text{\smash{\raisebox{-1.3ex}{%
    $\widehatsym$}}}}
\newcommand\fixwidehat[1]{%
  \mathchoice
    {\accentset{\displaystyle\lowerwidehatsym}{#1}}
    {\accentset{\textstyle\lowerwidehatsym}{#1}}
    {\accentset{\scriptstyle\lowerwidehatsym}{#1}}
    {\accentset{\scriptscriptstyle\lowerwidehatsym}{#1}}
}

\usepackage{graphicx}
    
% text on arrow for xRightarrow
\makeatletter
%\newcommand{\xRightarrow}[2][]{\ext@arrow 0359\Rightarrowfill@{#1}{#2}}
\makeatother


\def \bx {\boldsymbol{x}}
\def \ba {\boldsymbol{a}}
\def \bI {\boldsymbol{I}}
\def \bt {\boldsymbol{t}}
\def \bb {\boldsymbol{b}}
\def \bA {\boldsymbol{A}}
\def \bX {\boldsymbol{X}}
\def \bu {\boldsymbol{u}}
\def \bS {\boldsymbol{S}}
\def \bZ {\boldsymbol{Z}}
\def \bz {\boldsymbol{z}}
\def \by {\boldsymbol{y}}
\def \bw {\boldsymbol{w}}
\def \bT {\boldsymbol{T}}
\def \bS {\boldsymbol{S}}
\def \bm {\boldsymbol{m}}
\def \bW {\boldsymbol{W}}
\def \bY {\boldsymbol{Y}}
\def \bH {\boldsymbol{H}}
\def \blambda {\boldsymbol{\lambda}}
\def \bPhi {\boldsymbol{\Phi}}
\def \btheta {\boldsymbol{\theta}}
\def \bmu {\boldsymbol{\mu}}
\def \bphi {\boldsymbol{\phi}}
\def \bSigma {\boldsymbol{\Sigma}}
\def \lb {\left\{}
\def \rb {\right\}}
\def \caln {\mathcal{N}}
\def \dissum {\displaystyle\Sigma}
\def \dispro {\displaystyle\prod}
\def \E {\mathbb{E}}
\def \Q {\mathbb{Q}}
\def \V {\mathbb{V}}
\def \R {\mathbb{R}}
\def \calq {\mathcal{Q}}
\def \calg {\mathcal{G}}
\def \caln {\mathcal{N}}
\def \calr {\mathcal{R}}
\def \calm {\mathcal{M}}
\def \calc {\mathcal{C}}
\def \bcup {\bigcup}

\author{wu}
\date{\today}
\title{Rough Sets: Theoretical aspects of reasoning about data}
\hypersetup{
 pdfauthor={wu},
 pdftitle={Rough Sets: Theoretical aspects of reasoning about data},
 pdfkeywords={},
 pdfsubject={},
 pdfcreator={Emacs 26.2 (Org mode 9.2.4)}, 
 pdflang={English}}
\begin{document}

\maketitle
\tableofcontents

\section{Knowledge}
\label{sec:org7c56ea3}
\subsection{Knowledge base}
\label{sec:orgaf72ad1}
Given a finite set \(U\neq \emptyset\) (the universe). Any subset \(X\subset U\)
of the universe is called a \textbf{concept} or a \textbf{category} in \(U\). And any family of
concepts in \(U\) will be referred to as \textbf{abstract knowledge} about \(U\).

\textbf{partition} or \textbf{classification} of a certain universe \(U\) is a family 
\(C=\lb X_1,X_2,\dots,X_n\rb\) s.t. \(X_i\subset U,X_i\neq\emptyset,X_i\cap
   X_j=\emptyset\) and \(\bigcup X_i=U\)

A family of classifications is called a \textbf{knowledge base} over \(U\)


\(R\) an equivalence relation over \(U\), \(U/R\) family of all equivalence classes
of \(R\), referred to be \textbf{categories} or \textbf{concepts} of \(R\), and \([x]_R\) denotes a
category in \(R\) containing an element \(x\in U\)

By a \textbf{knowledge base} we can understand a relational system \(K=(U,\bR)\), \(\bR\)
is a family of equivalence relations over \(U\)

If \(\bP\subset \bR\) and \(\bP\neq\emptyset\), then \(\bigcap\bP\) is also an
equivalence relation, and will be denoted by \(IND(\bP)\), called an
\textbf{indiscernibility relation} over \(\bP\)
\begin{equation*}
[x]_{IND(\bP)}=\bigcap_{R\in\bP}[x]_R
\end{equation*}

\(U/IND(\bP)\) called \(\bP\textbf{-basic}\) \textbf{knowledge about} \(U\) in \(K\). For
simplicity, \(U/\bP=U/IND(\bP)\) and \(\bP\) will be also called
\(\bP\textbf{-basic}\) \textbf{knowledge}
. Equivalence classes of \(IND(\bP)\) are called
\textbf{basic categories} of knowledge \(\bP\). If \(Q\in\bR\), then \(Q\) is a
\(Q\textbf{-elementary}\) \textbf{knowledge} and equivalence classes of \(Q\) are referred
to as \(Q\textbf{-elementary}\) \textbf{concepts} of knowledge \(\bR\)

The family of all \(\bP\text{-basic}\) categories for all
\(\empty\neq\bP\subset\bR\) will be called the \textbf{family of basic categories} in
knowledge base \(K=(U,\bR)\)

Let \(K=(U,\bR)\) be a knowledge base. By \(IND(K)\) we denote the family of all
equivalence relations defined in \(K\) as \(IND(K)=\lb
   IND(\bP):\emptyset\neq\bP\subseteq\bR\rb\).

Thus \(IND(K)\) is the minimal set of equivalence relations.

Every union of \(\bP\text{-basic}\) categories will be \(\bP\textbf{-category}\)

The family of all categories in the knowledge base \(K=(U,\bR)\) will be
referred to as \(K\textbf{-categories}\)
\subsection{Equivalence, generalization and specialization of knowledge}
\label{sec:orgf6b0be9}
Let \(K=(U,\bP),K'=(U,\bQ)\). \(K\) and \(K'\) are \textbf{equivalent} \(K\simeq
   K',(\bP\simeq\bQ)\) if \(IND(\bP)=IND(\bQ)\). Hence \(K\simeq K'\) if both \(K\) and
\(K'\) have the same set of elememtary categories. \emph{This means that knowledge in
knowledge bases \(K\) and \(K'\) enables us to express exactly the same facts about the universe.}

If \(IND(\bP)\subset IND(\bQ)\) then knowledge \(\bP\) is \textbf{finer} than knowledge
\(\bQ\) (\textbf{coarser}). \(\bP\) is \textbf{specialization} of \(\bQ\) and \(\bQ\) is \textbf{generalization}
of \(\bP\)
\section{Imprecise categories, approximations and rough sets}
\label{sec:org7c18e2b}
\subsection{Rough sets}
\label{sec:org9f59af6}
Let \(X\subseteq U\). \(X\) is \(R\textbf{-definable}\) or \(R\textbf{-exact}\) if \(X\) is the union of some
\(R\text{-basic}\) categories. otherwise
\(R\textbf{-undefinable},R\textbf{-rough},R\textbf{-inexact}\)  .
\subsection{Approximations of set}
\label{sec:org45afb92}
Given \(K=(U,\bR), R\in IND(K)\)
\begin{align*}
&\underline{R}X=\bigcup\lb Y\in U/R:Y\subseteq X\rb\\
&\overline{R}X=\bigcup\lb Y\in U/R:Y\cap X\neq\emptyset\rb\\
\end{align*}
called the \(R\textbf{-lower}\) and \(R\textbf{-upper}\) \textbf{approximation} of \(X\)

\(BN_R(X)=\overline{R}X-\underline{R}X\) is \(R\textbf{-boundary}\) of \(X\).
\(BN_R(X)\) is the set of elements which cannot be classified either to \(X\) or
to \(-X\) having knowledge \(R\)

\begin{align*}
&POS_R(X)=\underline{R}X,R\text{-positive region of } X\\
&NEG_R(X)=U-\overline{R}X,R\text{-negative region of } X\\
&BN_R(X) - R\text{-borderline region of } X\\
\end{align*}

If \(x\in POS(X)\), then \(x\) will be called an \(R\textbf{-positive}\) \textbf{example of} \(X\)

\begin{proposition}
\begin{enumerate}
\item $X$ is $R$-definable if and only if $\underline{R}X=\overline{R}X$
\item $X$ is rought w.r.t. $R$ if and only if $\underline{R}X\neq\overline{R}X$
\end{enumerate}
\end{proposition}
\subsection{Properties of approximations}
\label{sec:org3f6ffcb}
\begin{proposition}[2.2]
\begin{enumerate}
\item \(\uR X\subseteq X\subseteq \oR X\)
\item \(\uR\emptyset=\uR\emptyset=\emptyset;\quad \uR U=\oR U=U\)
\item \(\oR(X\cup Y)=\oR X\cup \oR Y\)
\item \(\uR(X\cap Y)=\uR X\cap \uR Y\)
\item \(X\subseteq Y\) implies \(\uR X\subseteq \uR Y\)
\item \(X\subseteq Y\) implies \(\oR X\subseteq\oR Y\)
\item \(\uR(X\cup Y)\subseteq \uR X\cup \uR Y\)
\item \(\uR(-X)=-\oR X\)
\item \(\oR(-X)=-\uR X\)
\item \(\oR(-X)=-\uR X\)
\item \(\uR\uR X=\oR\uR X=\uR X\)
\item \(\oR\oR X=\uR\oR X=\oR X\)
\end{enumerate}
\end{proposition}

The equivalence relation \(R\) over \(U\) uniquely defines a topological space
\(T=(U,DIS(R))\) where \(DIS(R)\) is the familty of all open and closed set in
\(T\) and \(U/R\) is a base for \(T\). The \(R\text{-lower}\) and \(R\text{-upper}\)
approximation of \(X\) in \(A\) are \textbf{interior} and \textbf{closure} operations in the
topological space \(T\)
\subsection{Approximations and membership relation}
\label{sec:org0223a7a}
\begin{align*}
&x\underline{\in}_RX \text{if and only if } x\in\underline{R}X\\
&x\overline{\in}_RX \text{if and only if } x\in\overline{R}X\\
\end{align*}
where \(\underline{\in}_R\) read "\(x\) \textbf{surely belongs} to \(X\) w.r.t. \(R\)" and
\(\overline{\in}_R\) - "\(x\) \textbf{possibly belongs} to \(X\) w.r.t. \(R\)". The \textbf{lower} and
\textbf{upper} membership.
\begin{proposition}
\begin{enumerate}
\item $x\uin X$ implies $x\in X$ implies $x\oin X$
\item $X\subset Y$ implies ($x\uin X$ implies $x\uin Y$ and $x\oin X$ implies $x\oin Y$)
\item $x\oin(X\cup Y)$ if and only if $x\oin X$ or $x\oin Y$
\item $x\uin(X\cap Y)$ if and only if $x\uin X$ and $x\uin Y$
\item $x\uin X$ or $x\uin Y$ implies $x\uin (X\cup Y)$
\item $x\oin X\cap Y$ implies $x\oin X$ and $x\oin Y$
\item $x\uin (-X)$ if and only if non $x\oin X$
\item $x\oin (-X)$ if and only if non $x\uin X$
\end{enumerate}
\end{proposition}
\subsection{Numerical characterization of imprecision}
\label{sec:org8ce829a}
\textbf{accuracy measure}
\begin{equation*}
\alpha_R(X)=\frac{card\;\uR}{card\;\oR}
\end{equation*}
\subsection{Topological characterization of imprecision}
\label{sec:org84376d0}
\begin{definition}[]
\begin{enumerate}
\item If \(\uR X\neq\emptyset\) and \(\oR X\neq U\), then we say that \(X\) is
\textbf{roughly R-definable}. We can decide whether some elements belong to \(X\)
or \(-X\)
\item If \(\uR X=\emptyset\) and \(\oR X\neq U\), then we say that \(X\) is
\textbf{internally R-undefinable}. We can decide whether some elemnts belong
to \(-X\)
\item If \(\uR X\neq\emptyset\) and \(\oR X=U\), then we say that \(X\) is
\textbf{externally R-undefinable}. We can decide whether some elements belong
to \(X\)
\item If \(\uR X=\emptyset\) and \(\oR X=U\), then we say that \(X\) is
\textbf{totally R-undefinable}. unable to decide
\end{enumerate}
\end{definition}

\begin{proposition}[2.4]
\begin{enumerate}
\item Set \(X\) is R-definable(roughly R-definable, totally R-undefinable) if and
only if so is \(-X\)
\item Set \(X\) is externally R-undefinable if and only if \(-X\) is internally
R-undefinable
\end{enumerate}
\end{proposition}

\begin{proof}
\begin{enumerate}
\item \begin{align*}
R\text{-definable}&\Leftrightarrow \uR X=\oR X, \uR\neq\emptyset,\oR\neq U\\
&\Leftrightarrow -\uR X=-\oR X\\
&\Leftrightarrow \oR(-X)=\uR(-X)\\
\end{align*}
\end{enumerate}
\end{proof}
\end{document}