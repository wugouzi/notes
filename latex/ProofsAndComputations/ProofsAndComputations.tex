% Created 2021-05-21 五 05:39
% Intended LaTeX compiler: pdflatex
\documentclass[11pt]{article}
\usepackage[utf8]{inputenc}
\usepackage[T1]{fontenc}
\usepackage{graphicx}
\usepackage{grffile}
\usepackage{longtable}
\usepackage{wrapfig}
\usepackage{rotating}
\usepackage{amsmath}
\usepackage{textcomp}
\usepackage{amssymb}
\usepackage{capt-of}
\usepackage{hyperref}
% TIPS
% \substack{a\\b} for multiple lines text





% pdfplots will load xolor automatically without option
\usepackage[dvipsnames]{xcolor}

\usepackage{forest}
% two-line text in node by [two \\ lines]
% \begin{forest} qtree, [..] \end{forest}
\forestset{
  qtree/.style={
    baseline,
    for tree={
      parent anchor=south,
      child anchor=north,
      align=center,
      inner sep=1pt,
    }}}
%\usepackage{flexisym}
% load order of mathtools and mathabx, otherwise conflict overbrace

\usepackage{mathtools}
%\usepackage{fourier}
\usepackage{pgfplots}
\usepackage{amsthm, mathabx,  amsmath, commath}
\usepackage{amsfonts}

\usepackage{empheq}
\usepackage{tikz}
\usetikzlibrary{arrows.meta}
\usepackage[most]{tcolorbox}

\newtheorem{theorem}{Theorem}[section]
\newtheorem{definition}{Definition}[section]
\newtheorem{corollary}{Corollary}[section]
\newtheorem{example}{Example}[section]
\newtheorem{lemma}{Lemma}[section]
\newtheorem{proposition}{Proposition}[section]

\newcommand{\bl}[1] {\boldsymbol{#1}}
\newcommand{\Wt}[1] {\stackrel{\sim}{\smash{#1}\rule{0pt}{1.1ex}}}
\newcommand{\wt}[1] {\widetilde{#1}}


%For boxed texts in align, use Aboxed{}
%otherwise use boxed{}

\DeclareMathSymbol{\widehatsym}{\mathord}{largesymbols}{"62}
\newcommand\lowerwidehatsym{%
  \text{\smash{\raisebox{-1.3ex}{%
    $\widehatsym$}}}}
\newcommand\fixwidehat[1]{%
  \mathchoice
    {\accentset{\displaystyle\lowerwidehatsym}{#1}}
    {\accentset{\textstyle\lowerwidehatsym}{#1}}
    {\accentset{\scriptstyle\lowerwidehatsym}{#1}}
    {\accentset{\scriptscriptstyle\lowerwidehatsym}{#1}}
}

\usepackage{graphicx}
    
% text on arrow for xRightarrow
\makeatletter
%\newcommand{\xRightarrow}[2][]{\ext@arrow 0359\Rightarrowfill@{#1}{#2}}
\makeatother


\def \bx {\boldsymbol{x}}
\def \ba {\boldsymbol{a}}
\def \bI {\boldsymbol{I}}
\def \bt {\boldsymbol{t}}
\def \bb {\boldsymbol{b}}
\def \bA {\boldsymbol{A}}
\def \bX {\boldsymbol{X}}
\def \bu {\boldsymbol{u}}
\def \bS {\boldsymbol{S}}
\def \bZ {\boldsymbol{Z}}
\def \bz {\boldsymbol{z}}
\def \by {\boldsymbol{y}}
\def \bw {\boldsymbol{w}}
\def \bT {\boldsymbol{T}}
\def \bS {\boldsymbol{S}}
\def \bm {\boldsymbol{m}}
\def \bW {\boldsymbol{W}}
\def \bY {\boldsymbol{Y}}
\def \bH {\boldsymbol{H}}
\def \blambda {\boldsymbol{\lambda}}
\def \bPhi {\boldsymbol{\Phi}}
\def \btheta {\boldsymbol{\theta}}
\def \bmu {\boldsymbol{\mu}}
\def \bphi {\boldsymbol{\phi}}
\def \bSigma {\boldsymbol{\Sigma}}
\def \lb {\left\{}
\def \rb {\right\}}
\def \caln {\mathcal{N}}
\def \dissum {\displaystyle\Sigma}
\def \dispro {\displaystyle\prod}
\def \E {\mathbb{E}}
\def \Q {\mathbb{Q}}
\def \V {\mathbb{V}}
\def \R {\mathbb{R}}
\def \calq {\mathcal{Q}}
\def \calg {\mathcal{G}}
\def \caln {\mathcal{N}}
\def \calr {\mathcal{R}}
\def \calm {\mathcal{M}}
\def \calc {\mathcal{C}}
\def \bcup {\bigcup}

\usepackage{ebproof}
\DeclareMathOperator{\Efq}{Efq}
\def \texists {\tilde{\exists}}
\def \tvee {\tilde{\vee}}
\DeclareMathOperator{\Stab}{Stab}
\author{Helmut Schwichtenberg \& Stanley S. Wainer}
\date{\today}
\title{Proofs and Computations}
\hypersetup{
 pdfauthor={Helmut Schwichtenberg \& Stanley S. Wainer},
 pdftitle={Proofs and Computations},
 pdfkeywords={},
 pdfsubject={},
 pdfcreator={Emacs 27.1 (Org mode 9.3)}, 
 pdflang={English}}
\begin{document}

\maketitle
\tableofcontents

\section{Logic}
\label{sec:org8e3244c}

\subsection{Natural Deduction}
\label{sec:org94529aa}
Negation is defined by
\begin{equation*}
\neg A:=(A\to\bot)
\end{equation*}

\begin{definition}[Gentzen]
\textbf{Subformulas} of \(A\) are defined by
\begin{enumerate}
\item \(A\) is a subformula of \(A\)
\item if \(B\circ C\) is a subformula of \(A\) then so are \(B,C\) for \(\circ=\to,\wedge,\vee\)
\item if \(\forall_xB(x)\) or \(\exists_xB(x)\) is a subformula of \(A\), then
so is \(B(r)\)
\end{enumerate}
\end{definition}



\begin{definition}[]
The notions of \textbf{positive}, \textbf{negative}, \textbf{strictly positive} subformula are defined
in a similar style
\begin{enumerate}
\item \(A\) is a positive and a strictly positive subformula of itself
\item if \(B\wedge C\) or \(B\vee C\) is a positive (negative, strictly
positive) subformula of \(A\), then so are \(B, C\)
\item if \(\forall_xB(x)\) or \(\exists_xB(x)\) is a positive (negative,
strictly positive) subformula of \(A\), then so is \(B(r)\)
\item if \(B\to C\) is a positive (negative) subformula of
\(A\), then \(B\) is a negative (positive)subformula of \(A\), and \(C\)
is a positive (negative)subformula of \(A\)
\item if \(B\to C\) is a strictly subformula of \(A\), then so is \(C\)
\end{enumerate}


A strictly positive subformula of \(A\) is also called a \textbf{strictly positive
part} (\textbf{s.p.p.}) of \(A\)
\end{definition}


\begin{equation*}
\begin{prooftree}[center=false]
\hypo{[u:A]}
\ellipsis{D}{B}
\infer1[\(\to^+u\)]{A\to B}
\end{prooftree}
\hspace{1cm}
\begin{prooftree}[center=false]
\hypo{}
\ellipsis{M}{A\to B}
\hypo{}
\ellipsis{N}{A}
\infer2[\(\to^-\)]{B}
\end{prooftree}
\end{equation*}


The rule \(\forall^+x\) with conclusion \(\forall_xA\) is subject to the
following \textbf{(eigen-)variable condition}: the derivation \(M\) of the premise
\(A\) should not contain any open assumption having \(x\) as a free variable

\begin{equation*}
\begin{prooftree}[center=false]
\hypo{}
\ellipsis{M}{A}
\infer1[\(\forall^+x\)]{\forall_xA}
\end{prooftree}
\hspace{1cm}
\begin{prooftree}[center=false]
\hypo{}
\ellipsis{M}{\forall_xA(x)}
\hypo{r}
\infer2[\(\forall^-\)]{A(r)}
\end{prooftree}
\end{equation*}

\begin{equation*}
\begin{prooftree}[center=false]
\hypo{}
\ellipsis{M}{A}
\infer1[\(\vee_0^+\)]{A\vee B}
\end{prooftree}\hspace{1cm}
\begin{prooftree}[center=false]
\hypo{}
\ellipsis{M}{B}
\infer1[\(\vee_1^+\)]{A\vee B}
\end{prooftree}\hspace{1cm}
\begin{prooftree}[center=false]
\hypo{}
\ellipsis{M}{A\vee B}
\hypo{[u:A]}
\ellipsis{N}{C}
\hypo{[v:B]}
\ellipsis{K}{C}
\infer3[\(\vee^-u,v\)]{C}
\end{prooftree}
\end{equation*}

\begin{equation*}
\begin{prooftree}[center=false]
\hypo{}
\ellipsis{M}{A}
\hypo{}
\ellipsis{N}{B}
\infer2[\(\wedge^+\)]{A\wedge B}
\end{prooftree}\hspace{1cm}
\begin{prooftree}[center=false]
\hypo{}
\ellipsis{M}{A\wedge B}
\hypo{[u:A]\quad [v:B]}
\ellipsis{N}{C}
\infer2[\(\wedge^-u,v\)]{C}
\end{prooftree}
\end{equation*}

\begin{equation*}
\begin{prooftree}[center=false]
\hypo{r}
\hypo{}
\ellipsis{M}{A(r)}
\infer2[\(\exists^+\)]{\exists_xA(x)}
\end{prooftree}\hspace{1cm}
\begin{prooftree}[center=false]
\hypo{}
\ellipsis{M}{\exists_xA}
\hypo{[u:A]}
\ellipsis{N}{B}
\infer2[\(\exists^- x,u(\text{var.cond.})\)]{B}
\end{prooftree}
\end{equation*}

Rule \(\exists^-x,u\) is subject to an \textbf{(eigen-)variable condition}: in the
derivation \(N\) the variable \(x\)
\begin{enumerate}
\item should not occur free in the formula of any open assumption other than \(u:A\)
\item should not occur free in \(B\)
\end{enumerate}


For each of the connectives \(\wedge, \vee, \exists\) the rules and the
following axioms are equivalent over minimal logic
\begin{equation*}
\exists^+:A\to\exists_xA,\quad \exists^-:\exists_xA\to\forall_x(A\to B)\to B(x\not\in FV(B))
\end{equation*}

\begin{lemma}[]
The following are derivable
\begin{align*}
&(A\wedge B\to C)\leftrightarrow(A\to B\to C)\\
&(A\to B\wedge C)\leftrightarrow(A\to B)\wedge(A\to C)\\
&(A\vee B\to C)\leftrightarrow(A\to C)\wedge(B\to C)\\
&(A\to B)\vee(A\to C)\to (A\to B\vee C)\\
&\exists_x(A\to B)\to(\forall_xA\to B)\quad\text{if }x\not\in FV(B)\\
&\forall_x(A\to B)\leftrightarrow(A\to\forall_xB)\quad\text{if }x\not\in FV(A)\\
&\forall_x(A\to B)\leftrightarrow(\exists_xA\to B)\quad\text{if }x\not\in FV(B)\\
&\exists_x(A\to B)\to(A\to\exists_xB)\quad\text{if }x\not\in FV(A)
\end{align*}
\end{lemma}

\begin{proof}
\begin{equation*}
\begin{prooftree}[center=false]
\hypo{u:\exists_x(A\to B)}
\hypo{x}
\hypo{w:A\to B}\hypo{v:A}
\infer2{B}
\infer2{\exists_xB}
\infer2[\(\exists^-x,w\)]{\exists_xB}
\infer1[\(\to^+v\)]{A\to\exists_xB}
\infer1[\(\to^+u\)]{\exists_x(A\to B)to A\to\exists_xB}
\end{prooftree}
\end{equation*}
\end{proof}

\textbf{weak disjuction} and \textbf{weak existence}
\begin{equation*}
A\tilde{\vee}B:=\neg A\to\neg B\to\bot,\quad
\tilde{\exists}_xA:=\neg\forall_x\neg A
\end{equation*}
These weak variants are no stronger than the proper ones
\begin{equation*}
A\vee B\to A\tilde{\vee}B,\quad\exists_xA\to\tilde{\exists}_xA
\end{equation*}
by putting \(C:=\bot\) in \(\vee^-\) and \(B:=\bot\) in \(\exists^-\)

Moreover
\begin{align*}
\tilde{\exists}_{x_1,\dots,x_n}A&:=\forall_{x_1,\dots,x_n}(A\to\bot)\to\bot\\
\tilde{\exists}_{x_1,\dots,x_n}(A_1\tilde{\wedge}\cdots\tilde{\wedge}A_m)&:=
\forall_{x_1,\dots,x_n}(A_1\to\cdots\to A_m\to\bot)\to\bot
\end{align*}

In the previous contexts falsity \(\bot\) plays no role. We may change this
and require \textbf{ex-falso-quodlibet} axioms of the form
\begin{equation*}
\forall_{\vec{x}}(\bot\to R\vec{x})
\end{equation*}
with \(R\) a relation symbol distinct from \(\bot\). Let Efq denote the set
of all such axioms. A formula \(A\) is called \textbf{intuitionistically derivable},
written \(\vdash_iA\) if \(\Efq\vdash A\). We write \(\Gamma\vdash_i B\) for
\(\Gamma\cup\Efq\vdash B\)

We may even go further and require \textbf{stability} axioms, of the form
\begin{equation*}
\forall_{\vec{x}}(\neg\neg R\vec{x}\to R\vec{x})
\end{equation*}
with \(R\) again a relation distinct from \(\bot\). Let Stab denote the set
of all these axioms. A formula \(A\) is called \textbf{classically derivable}, written
\(\vdash_c A\), if \(\Stab\vdash A\). We write \(\Gamma\vdash_cB\) for
\(\Gamma\cup\Stab\vdash B\)

\begin{theorem}[Stability, or principle of indirect proof]
\begin{enumerate}
\item \(\vdash(\neg\neg A\to A)\to(\neg\neg B\to B)\to\neg\neg(A\wedge B)\to
      A\wedge B\)
\item \(\vdash(\neg\neg B\to B)\to\neg\neg(A\to B)\to A\to B\)
\item \(\vdash(\neg\neg A\to A)\to\neg\neg\forall_xA\to A\)
\item \(\vdash_c\neg\neg A\to A\) for every formula \(A\) without \(\vee,\exists\)
\end{enumerate}
\end{theorem}

\begin{proof}
\begin{enumerate}
\item \((\neg\neg A\to A)\to\neg\neg A\to A\)
\setcounter{enumi}{1}
\item \begin{equation*}
\begin{prooftree}[center=false]
\hypo{u:\neg\neg B\to B}
\hypo{v:\neg\neg(A\to B)}
\hypo{u_1:\neg B}
\hypo{u_2:A\to B}
\hypo{w:A}
\infer2{B}
\infer2{\bot}
\infer1[\(\to^+u_2\)]{\neg(A\to B)}
\infer2{
\begin{prooftree}[center=false]
\hypo{\bot}\infer1[\(\to^+u_1\)]{\neg\neg B}
\end{prooftree}
}
\infer2{B}
\end{prooftree}
\end{equation*}
\item \begin{equation*}
\begin{prooftree}[center=false]
\hypo{u:\neg\neg A\to A}
\hypo{v:\neg\neg\forall_xA}
\hypo{u_1:\neg A}
\hypo{u_2:\forall_xA}
\hypo{x}
\infer2{A}
\infer2{
\begin{prooftree}[center=false]
\hypo{\bot}\infer1[\(\to^+u_2\)]{\neg\forall_xA}
\end{prooftree}
}
\infer2{\begin{prooftree}[center=false]
\hypo{\bot}\infer1[\(\to^+u_1\)]{\neg\neg A}
\end{prooftree}}
\infer2{A}
\end{prooftree}
\end{equation*}
\item Induction on \(A\). The case \(R\vec{t}\) with \(R\) distinct from
\(\bot\) is given by Stab. In the case \(\bot\) the desired derivation is
\begin{equation*}
\begin{prooftree}[center=false]
\hypo{v:(\bot\to\bot)\to\bot}
\hypo{u:\bot}
\infer1[\(\to^+u\)]{\bot\to\bot}
\infer2{\bot}
\end{prooftree}
\end{equation*}
In the case \(A\wedge B,A\to B\) and \(\forall_xA\) use 1,2,3 respectively
\end{enumerate}
\end{proof}

\begin{lemma}[]
The following are derivable
\begin{align*}
(\tilde{\exists}_xA\to B)\to\forall_x(A\to B)&\quad\text{ if }x\not\in FV(B)\\
(\neg\neg B\to B)\to\forall_x(A\to B)\to\tilde{\exists}_xA\to B&\quad\text{ if }x\not\in FV(B)\\
(\bot\to B[x:=c])\to(A\to\tilde{\exists}_xB)\to\texists_x(A\to B)&\quad\text{ if }x\not\in FV(A)\\
\texists_x(A\to B)\to A\to\texists_xB&\quad\text{ if }x\not \in FV(A)
\end{align*}
The last two items can also be seen as simplifying a weakly existentially
quantified implication whose premise doesn't contain the quantified variable.
In case the conclusion does not contain the quantified variable we have
align
\begin{align*}
(\neg\neg B\to B)\to\texists_x(A\to B)\to\forall_xA\to B&\quad\text{ if }x\not \in FV(A)\\
\forall_x(\neg\neg A\to A)\to(\forall_xA\to B)\to\texists_x(A\to B)&\quad\text{ if }x\not \in FV(A)
\end{align*}
\end{lemma}

\begin{proof}
\begin{enumerate}
\setcounter{enumi}{2}
\item Writing \(B_0\) for \(B[x:=c]\) we have
\begin{equation*}
\resizebox{0.9\textwidth}{!}{
\begin{prooftree}[center=false]
\hypo{\forall_x\neg(A\to B)}
\hypo{c}
\infer2{\neg(A\to B_0)}
\hypo{\bot\to B_0}
\hypo{A\to\texists_xB}
\hypo{u_2:A}
\infer2{\texists_xB}
\hypo{\forall_x\neg(A\to B)}
\hypo{x}
\infer2{\neg(A\to B)}
\hypo{u_1:B}
\infer1{A\to B}
\infer2{
\begin{prooftree}
\hypo{\bot}
\infer1[\(\to^+u\)]{\neg B}
\infer1{\forall_x\neg B}
\end{prooftree}
}
\infer2{\bot}
\infer2{B_0}
\infer1[\(\to^+u_2\)]{A\to B_0}
\infer2{\bot}
\end{prooftree}}
\end{equation*}
\item \begin{equation*}
\begin{prooftree}[center=false]
\hypo{\texists_x(A\to B)}
\hypo{\forall_x\neg B}
\hypo{x}
\infer2{\neg B}
\hypo{u_1:A\to B}
\hypo{A}
\infer2{B}
\infer2{\bot}
\infer1[\(\to^+u_1\)]{\neg(A\to B)}
\infer1{\forall_x\neg(A\to B)}
\infer2{\bot}
\end{prooftree}
\end{equation*}
\end{enumerate}
\end{proof}

An immediate consequence of 6 is the classical derivability of the "drinker
formula" \(\texists_x(Px\to\forall_xPx)\) to be read "in every non-empty bar
there is a person s.t. if this person drinks, then everybody drinks"

\begin{corollary}[]
\begin{alignat*}{2}
&\vdash_c(\texists_xA\to B)\leftrightarrow\forall_x(A\to B)\quad
&&\text{ if }x\not\in FV(B)\text{ and }B\text{ without }\forall,\exists\\
&\vdash_i(A\to\texists_xB)\leftrightarrow\texists_x(A\to B)
&&\text{ if }x\not\in FV(A)\\
&\vdash_c\texists_x(A\to B)\leftrightarrow(\forall_xA\to B)\quad
&&\text{ if }x\not\in FV(B)\text{ and }A,B\text{ without }\forall,\exists
\end{alignat*}
\end{corollary}

\begin{lemma}[]
The following are derivable
\begin{align*}
&(A\tvee B\to C)\to (A\to C)\wedge(B\to C)\\
&(\neg\neg C\to C)\to(A\to C)\to(B\to C)\to A\tvee B\to C\\
&(\bot\to B)\to(A\to B \tvee C)\to(A\to B)\tvee(A\to C)\\
&(A\to B)\tvee(A\to C)\to A\to B\tvee C\\
&(\neg\neg C\to C)\to(A\to C)\tvee(B\to C)\to A\to B\to C\\
&(\bot\to C)\to(A\to B\to C)\to(A\to C)\tvee(B\to C)
\end{align*}
\end{lemma}

\begin{proof}
\begin{enumerate}
\setcounter{enumi}{5}
\item \begin{equation*}
\begin{prooftree}[center=false]
\hypo{\neg(B\to C)}
\hypo{\bot\to C}
\hypo{\neg(A\to C)}
\hypo{A\to B\to C}
\hypo{u_1:A}
\infer2{B\to C}
\hypo{u_2:B}
\infer2{C}
\infer1[\(\to^+u_1\)]{A\to C}
\infer2{\bot}
\infer2{C}
\infer1[\(\to^+u_2\)]{B\to C}
\infer2{\bot}
\end{prooftree}
\end{equation*}
\end{enumerate}
\end{proof}

\begin{corollary}[]
\begin{alignat*}{2}
&\vdash_c(A\tvee B\to C)\leftrightarrow(A\to C)\wedge(B\to C)\quad&&\text{ for }C
\text{ without }\forall,\exists\\
&\vdash_i(A\to B\tvee C)\leftrightarrow(A\to B)\tvee(A\to C)\\
&\vdash_c(A\to C)\tvee(B\to C)\leftrightarrow(A\to B\to C)
&&\text{for }C\text{ without }\forall,\exists
\end{alignat*}
\end{corollary}

\begin{remark}
It is easy to see that weak disjuction and the weak existential quantifier
satisfy the same axioms as the strong variants, if one restricts the
conclusion of the elimination axioms to formulas without \(\forall,
   \exists\). In fact we have
\begin{align*}
& \vdash A\to A\tvee B,\quad\vdash B\to A\tvee B\\
&\vdash_c A\tvee B\to(A\to C)\to(B\to C)\to C\quad(C\text{ without }\forall,\exists)\\
&\vdash A\to\tvee_xA\\
&\vdash_c \texists_xA\to\forall_x(A\to B)\to B\quad(x\not\in FV(B),B\text{ without }\forall,\exists)
\end{align*}
\end{remark}

\begin{proof}
\begin{enumerate}
\setcounter{enumi}{1}
\item \begin{equation*}
\resizebox{0.9\textwidth}{!}{
\begin{prooftree}[center=false]
\hypo{\neg\neg C\to C}
\hypo{\neg A\to\neg B\to\bot}
\hypo{u_1:\neg C}
\hypo{A\to C}
\hypo{u_2:A}
\infer2{C}
\infer2{\bot}
\infer1[\(\to^+u_2\)]{\neg A}
\infer2{\neg B\to\bot}
\hypo{u_1:\neg C}
\hypo{B\to C}
\hypo{u_3:B}
\infer2{C}
\infer2{\bot}
\infer1[\(\to^+u_3\)]{\neg B}
\infer2{\bot}
\infer1[\(\to^+u_1\)]{\neg\neg C}
\infer2{C}
\end{prooftree}}
\end{equation*}
\end{enumerate}
\end{proof}

\(A\) is derivable in classical logic iff its translation \(A^g\) is
derivable in minimal logic

\begin{definition}[Gödel-Gentzen translation $A^g$]
\begin{align*}
(R\vec{t})^g&:=\neg\neg R\vec{t}\quad\text{ for }R\text{ distinct from }\bot\\
\bot^g&:=\bot\\
(A\vee B)^g&:=A^g\tvee B^g\\
(\exists_xA)^g&:=\texists_xA^g\\
(A\circ B)^g&:=A^g\circ B^g\quad\text{ for }\circ=\to,\wedge\\
(\forall_xA)^g&:=\forall_xA^g
\end{align*}
\end{definition}

\begin{lemma}[]
\(\vdash\neg\neg A^g\to A^g\)
\end{lemma}

\begin{proof}
Induction on \(A\)

\emph{Case} \(R\vec{t}\) with \(R\) distinct from \(\bot\). We must show
\(\neg\neg\neg\neg R\vec{t}\to\neg\neg R\vec{t}\), which is a special case of
\(\vdash\neg\neg\neg B\to\neg B\)

\emph{Case} \(\bot\). Use \(\vdash\neg\neg\bot\to\bot\)

\emph{Case} \(A\vee B\). \(\vdash\neg\neg(A^g\tvee B^g)\to A^g\tvee B^g\) is a special
case of \(\vdash\neg\neg(\neg C\to \neg D\to\bot)\to\neg C\to\neg D\to\bot\)
\begin{equation*}
\begin{prooftree}[center=false]
\hypo{\neg\neg(\neg C\to\neg D\to\bot)}
\hypo{u_1:\neg C\to\neg D\to\bot}
\hypo{\neg C}
\infer2{\neg D\to\bot}
\hypo{\neg D}
\infer2{\bot}
\infer1[\(\to^+u_1\)]{\neg(\neg C\to\neg D\to\bot)}
\infer2{\bot}
\end{prooftree}
\end{equation*}

\emph{Case} \(\exists_xA\). We need to show
\(\vdash\neg\neg\texists_xA^g\to\texists_xA^g\), and this is a special case
of \(\vdash\neg\neg\neg B\to\neg B\)
\end{proof}

\begin{theorem}[]
\begin{enumerate}
\item \(\Gamma\vdash_c A\) implies \(\Gamma^g\vdash A^g\)
\item \(\Gamma^g\vdash A^g\) implies \(\Gamma\vdash_c A\) for \(\Gamma,A\) without \(\vee,\exists\)
\end{enumerate}
\end{theorem}
\end{document}