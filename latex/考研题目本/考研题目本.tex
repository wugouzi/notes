% Created 2020-09-25 五 20:03
% Intended LaTeX compiler: pdflatex
\documentclass{article}
\usepackage[utf8]{inputenc}
\usepackage[T1]{fontenc}
\usepackage{graphicx}
\usepackage{grffile}
\usepackage{longtable}
\usepackage{wrapfig}
\usepackage{rotating}
\usepackage[normalem]{ulem}
\usepackage{amsmath}
\usepackage{textcomp}
\usepackage{amssymb}
\usepackage{capt-of}
\usepackage{hyperref}
\usepackage{minted}
% TIPS
% \substack{a\\b} for multiple lines text





% pdfplots will load xolor automatically without option
\usepackage[dvipsnames]{xcolor}

\usepackage{forest}
% two-line text in node by [two \\ lines]
% \begin{forest} qtree, [..] \end{forest}
\forestset{
  qtree/.style={
    baseline,
    for tree={
      parent anchor=south,
      child anchor=north,
      align=center,
      inner sep=1pt,
    }}}
%\usepackage{flexisym}
% load order of mathtools and mathabx, otherwise conflict overbrace

\usepackage{mathtools}
%\usepackage{fourier}
\usepackage{pgfplots}
\usepackage{amsthm, mathabx,  amsmath, commath}
\usepackage{amsfonts}

\usepackage{empheq}
\usepackage{tikz}
\usetikzlibrary{arrows.meta}
\usepackage[most]{tcolorbox}

\newtheorem{theorem}{Theorem}[section]
\newtheorem{definition}{Definition}[section]
\newtheorem{corollary}{Corollary}[section]
\newtheorem{example}{Example}[section]
\newtheorem{lemma}{Lemma}[section]
\newtheorem{proposition}{Proposition}[section]

\newcommand{\bl}[1] {\boldsymbol{#1}}
\newcommand{\Wt}[1] {\stackrel{\sim}{\smash{#1}\rule{0pt}{1.1ex}}}
\newcommand{\wt}[1] {\widetilde{#1}}


%For boxed texts in align, use Aboxed{}
%otherwise use boxed{}

\DeclareMathSymbol{\widehatsym}{\mathord}{largesymbols}{"62}
\newcommand\lowerwidehatsym{%
  \text{\smash{\raisebox{-1.3ex}{%
    $\widehatsym$}}}}
\newcommand\fixwidehat[1]{%
  \mathchoice
    {\accentset{\displaystyle\lowerwidehatsym}{#1}}
    {\accentset{\textstyle\lowerwidehatsym}{#1}}
    {\accentset{\scriptstyle\lowerwidehatsym}{#1}}
    {\accentset{\scriptscriptstyle\lowerwidehatsym}{#1}}
}

\usepackage{graphicx}
    
% text on arrow for xRightarrow
\makeatletter
%\newcommand{\xRightarrow}[2][]{\ext@arrow 0359\Rightarrowfill@{#1}{#2}}
\makeatother


\def \bx {\boldsymbol{x}}
\def \ba {\boldsymbol{a}}
\def \bI {\boldsymbol{I}}
\def \bt {\boldsymbol{t}}
\def \bb {\boldsymbol{b}}
\def \bA {\boldsymbol{A}}
\def \bX {\boldsymbol{X}}
\def \bu {\boldsymbol{u}}
\def \bS {\boldsymbol{S}}
\def \bZ {\boldsymbol{Z}}
\def \bz {\boldsymbol{z}}
\def \by {\boldsymbol{y}}
\def \bw {\boldsymbol{w}}
\def \bT {\boldsymbol{T}}
\def \bS {\boldsymbol{S}}
\def \bm {\boldsymbol{m}}
\def \bW {\boldsymbol{W}}
\def \bY {\boldsymbol{Y}}
\def \bH {\boldsymbol{H}}
\def \blambda {\boldsymbol{\lambda}}
\def \bPhi {\boldsymbol{\Phi}}
\def \btheta {\boldsymbol{\theta}}
\def \bmu {\boldsymbol{\mu}}
\def \bphi {\boldsymbol{\phi}}
\def \bSigma {\boldsymbol{\Sigma}}
\def \lb {\left\{}
\def \rb {\right\}}
\def \caln {\mathcal{N}}
\def \dissum {\displaystyle\Sigma}
\def \dispro {\displaystyle\prod}
\def \E {\mathbb{E}}
\def \Q {\mathbb{Q}}
\def \V {\mathbb{V}}
\def \R {\mathbb{R}}
\def \calq {\mathcal{Q}}
\def \calg {\mathcal{G}}
\def \caln {\mathcal{N}}
\def \calr {\mathcal{R}}
\def \calm {\mathcal{M}}
\def \calc {\mathcal{C}}
\def \bcup {\bigcup}

\usepackage[UTF8]{ctex}
\author{五狗砸}
\date{\today}
\title{考研题目本}
\hypersetup{
 pdfauthor={五狗砸},
 pdftitle={考研题目本},
 pdfkeywords={},
 pdfsubject={},
 pdfcreator={Emacs 26.3 (Org mode 9.4)}, 
 pdflang={English}}
\begin{document}

\maketitle
\tableofcontents \clearpage
\section{微积分}
\label{sec:orga8507a7}
\begin{examplle}[]
设\(f'(x)\)连续,\(f(0)=0,f'(0)\neq0\),求
\(\lim_{x\to0}\frac{\int_0^{x^2}f(x^2-t)dt}{x^3\int_0^1f(xt)dt}\)

令\(x^2-t=u,xt=u\)
\begin{align*}
\lim_{x\to0}\frac{\int_0^{x^2}f(x^2-t)dt}{x^3\int_0^1f(xt)dt}&=
\lim_{x\to0}\frac{-\int_{x^2}^0f(u)du}{x^3\int_0^xf(u)\frac{du}{x}}=
\lim_{x\to0}\frac{\int_0^{x^2}f(u)du}{x^2\int_0^xf(u)du}\\
&=\lim_{x\to0}\frac{2xf(x^2)}{2x\int_0^xf(u)du+x^2f(x)}\\
&=\lim_{x\to0}\frac{2f(x^2)}{2\int_0^xf(u)du+xf(x)}\\
&=\lim_{x\to0}\frac{4xf'(x^2)}{3f(x)+xf'(x)}\\
&=\lim_{x\to0}\frac{4f'(x^2)}{3\frac{f(x)-f(0)}{x}+f'(x)}=1
\end{align*}
\end{examplle}

\begin{examplle}[]
求\(\lim_{x\to0}\frac{\frac{x^2}{2}+1-\sqrt{1+x^2}}{(\cos x-e^{x^2})\sin
  x^2}\)

利用泰勒展开,\(\sqrt{1+x^2}=1+\frac{1}{2}x^2-\frac{1}{8}x^4+o(x^4)\),
\(\cos x=1-\frac{1}{2}x^2+o(x^2)\),\(e^{x^2}=1+x^2+o(x^2)\),因此
\begin{equation*}
\lim_{x\to0}\frac{\frac{x^2}{2}+1-\sqrt{1+x^2}}{(\cos x-e^{x^2})\sin
x^2}=\lim_{x\to0}\frac{\frac{x^4}{8}+o(x^4)}{-\frac{3}{2}x^4+o(x^4)}=-\frac{1}{12}
\end{equation*}
\end{examplle}

\begin{examplle}[]
求\(\lim_{n\to\infty}\tan^n(\frac{\pi}{4}+\frac{2}{n})\)

因为\(\lim_{x\to\infty}f(x)=A\Rightarrow\lim_{n\to\infty}f(n)=A\)
\end{examplle}

\begin{examplle}[]
suppose \(\displaystyle y_n=\left[\frac{(2n)!}{n!n^n}\right]^{\frac{1}{n+1}}\). Compute
\(\lim_{n\to\infty}y_n\)

\begin{align*}
\ln y_n&=\frac{1}{n+1}\ln\frac{(2n)!}{n!n^n}=
\frac{1}{n+1}\ln\frac{(2n)(2n-1)\dots(n+1)}{n^n}\\
&=\frac{1}{n+1}\sum_{k=1}^n\ln(1+\frac{k}{n})=
\frac{n}{n+1}\left(
\frac{1}{n}\sum_{k=1}^n\ln(1+\frac{k}{n})
\right)
\end{align*}
Hence
\begin{align*}
\lim_{n\to\infty}y_n&=\lim_{n\to\infty}\frac{n}{n+1}\left(
\frac{1}{n}\sum_{k=1}^n\ln(1+\frac{k}{n})
\right)\\
&=1\cdot\int_0^1\ln(1+x)dx=
x\ln(1+x)\rvert_0^1-\int_0^1\frac{x}{1+x}dx\\
&=\ln2-1+\ln2=\ln\frac{4}{e}
\end{align*}
\end{examplle}

\begin{examplle}[]
已知\(x\to0\)时,\(e^{-x^4}-\cos(\sqrt{2}x^2)\) 与\(ax^n\)是等价无穷小,试求
\(a,n\)
\begin{align*}
&e^{-x^4}=1-x^4+\frac{x^8}{2}+o(x^8)\\
&\cos(\sqrt{2}x^2)=1-x^4+\frac{x^8}{6}+o(x^8)
\end{align*}
Hence \(a=\frac{1}{3},n=8\)
\end{examplle}

\begin{examplle}[]
设\(\displaystyle f(x)=\frac{\sqrt{1+\sin x+\sin^2x}-(\alpha+\beta\sin x)}{\sin^2x}\),且点
\(x=0\)是\(f(x)\)的可去间断点,求\(\alpha,\beta\)

由极限存在可知,\(\alpha=1\),泰勒展开
\begin{align*}
&\frac{\sqrt{1+\sin x+\sin^2x}-(\alpha+\beta\sin x)}{\sin^2x}\\
&=\lim_{x\to0}\frac{1+\frac{1}{2}(\sin x+\sin^2x)-\frac{1}{8}(\sin x+\sin^2x)^2-(1+\beta\sin x)
+o(\sin^2x)}{\sin^2}\\
&=\lim_{x\to0}\frac{(\frac{1}{2}-\beta)\sin x+\frac{3}{8}\sin^2x}{\sin^2x}
\end{align*}
故\(\beta=\frac{1}{2}\)
\end{examplle}

\begin{examplle}[]
let \(f(x)=\lim_{n\to\infty}\frac{2x^n-3x^{-n}}{x^n+x^{-n}}\sin\frac{1}{x}\)

\begin{equation*}
f(x)=
\begin{cases}
2\sin\frac{1}{x}^x&x<-1\\
-\frac{1}{2}\sin\frac{1}{x}&x=-1\\
-3\sin\frac{1}{x}&-1<x<0\\
-3\sin\frac{1}{x}&0<x<1\\
-\frac{1}{2}\sin\frac{1}{x}&x=1\\
2\sin\frac{1}{x}^x&x>1
\end{cases}
\end{equation*}
\(x=0\)是第二类间断点,\(x=\pm1\)是第一类间断点
\end{examplle}

\begin{examplle}[]
设\(f(1)=0,f'(1)=a\),求极限
\end{examplle}
\end{document}
