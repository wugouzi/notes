% Created 2020-10-10 六 17:29
% Intended LaTeX compiler: pdflatex
\documentclass{article}
\usepackage[utf8]{inputenc}
\usepackage[T1]{fontenc}
\usepackage{graphicx}
\usepackage{grffile}
\usepackage{longtable}
\usepackage{wrapfig}
\usepackage{rotating}
\usepackage[normalem]{ulem}
\usepackage{amsmath}
\usepackage{textcomp}
\usepackage{amssymb}
\usepackage{capt-of}
\usepackage{hyperref}
\usepackage{minted}
% TIPS
% \substack{a\\b} for multiple lines text





% pdfplots will load xolor automatically without option
\usepackage[dvipsnames]{xcolor}

\usepackage{forest}
% two-line text in node by [two \\ lines]
% \begin{forest} qtree, [..] \end{forest}
\forestset{
  qtree/.style={
    baseline,
    for tree={
      parent anchor=south,
      child anchor=north,
      align=center,
      inner sep=1pt,
    }}}
%\usepackage{flexisym}
% load order of mathtools and mathabx, otherwise conflict overbrace

\usepackage{mathtools}
%\usepackage{fourier}
\usepackage{pgfplots}
\usepackage{amsthm, mathabx,  amsmath, commath}
\usepackage{amsfonts}

\usepackage{empheq}
\usepackage{tikz}
\usetikzlibrary{arrows.meta}
\usepackage[most]{tcolorbox}

\newtheorem{theorem}{Theorem}[section]
\newtheorem{definition}{Definition}[section]
\newtheorem{corollary}{Corollary}[section]
\newtheorem{example}{Example}[section]
\newtheorem{lemma}{Lemma}[section]
\newtheorem{proposition}{Proposition}[section]

\newcommand{\bl}[1] {\boldsymbol{#1}}
\newcommand{\Wt}[1] {\stackrel{\sim}{\smash{#1}\rule{0pt}{1.1ex}}}
\newcommand{\wt}[1] {\widetilde{#1}}


%For boxed texts in align, use Aboxed{}
%otherwise use boxed{}

\DeclareMathSymbol{\widehatsym}{\mathord}{largesymbols}{"62}
\newcommand\lowerwidehatsym{%
  \text{\smash{\raisebox{-1.3ex}{%
    $\widehatsym$}}}}
\newcommand\fixwidehat[1]{%
  \mathchoice
    {\accentset{\displaystyle\lowerwidehatsym}{#1}}
    {\accentset{\textstyle\lowerwidehatsym}{#1}}
    {\accentset{\scriptstyle\lowerwidehatsym}{#1}}
    {\accentset{\scriptscriptstyle\lowerwidehatsym}{#1}}
}

\usepackage{graphicx}
    
% text on arrow for xRightarrow
\makeatletter
%\newcommand{\xRightarrow}[2][]{\ext@arrow 0359\Rightarrowfill@{#1}{#2}}
\makeatother


\def \bx {\boldsymbol{x}}
\def \ba {\boldsymbol{a}}
\def \bI {\boldsymbol{I}}
\def \bt {\boldsymbol{t}}
\def \bb {\boldsymbol{b}}
\def \bA {\boldsymbol{A}}
\def \bX {\boldsymbol{X}}
\def \bu {\boldsymbol{u}}
\def \bS {\boldsymbol{S}}
\def \bZ {\boldsymbol{Z}}
\def \bz {\boldsymbol{z}}
\def \by {\boldsymbol{y}}
\def \bw {\boldsymbol{w}}
\def \bT {\boldsymbol{T}}
\def \bS {\boldsymbol{S}}
\def \bm {\boldsymbol{m}}
\def \bW {\boldsymbol{W}}
\def \bY {\boldsymbol{Y}}
\def \bH {\boldsymbol{H}}
\def \blambda {\boldsymbol{\lambda}}
\def \bPhi {\boldsymbol{\Phi}}
\def \btheta {\boldsymbol{\theta}}
\def \bmu {\boldsymbol{\mu}}
\def \bphi {\boldsymbol{\phi}}
\def \bSigma {\boldsymbol{\Sigma}}
\def \lb {\left\{}
\def \rb {\right\}}
\def \caln {\mathcal{N}}
\def \dissum {\displaystyle\Sigma}
\def \dispro {\displaystyle\prod}
\def \E {\mathbb{E}}
\def \Q {\mathbb{Q}}
\def \V {\mathbb{V}}
\def \R {\mathbb{R}}
\def \calq {\mathcal{Q}}
\def \calg {\mathcal{G}}
\def \caln {\mathcal{N}}
\def \calr {\mathcal{R}}
\def \calm {\mathcal{M}}
\def \calc {\mathcal{C}}
\def \bcup {\bigcup}

\usepackage[UTF8]{ctex}
\author{五狗砸}
\date{\today}
\title{考研题目本}
\hypersetup{
 pdfauthor={五狗砸},
 pdftitle={考研题目本},
 pdfkeywords={},
 pdfsubject={},
 pdfcreator={Emacs 26.3 (Org mode 9.4)}, 
 pdflang={English}}
\begin{document}

\maketitle
\tableofcontents \clearpage
\section{微积分}
\label{sec:orgfd6046a}
\subsection{一元函数微分}
\label{sec:org8351aa7}
\begin{examplle}[]
设\(f'(x)\)连续,\(f(0)=0,f'(0)\neq0\),求
\(\displaystyle\lim_{x\to0}\frac{\int_0^{x^2}f(x^2-t)dt}{x^3\int_0^1f(xt)dt}\)

令\(x^2-t=u,xt=u\)
\begin{align*}
\lim_{x\to0}\frac{\int_0^{x^2}f(x^2-t)dt}{x^3\int_0^1f(xt)dt}&=
\lim_{x\to0}\frac{-\int_{x^2}^0f(u)du}{x^3\int_0^xf(u)\frac{du}{x}}=
\lim_{x\to0}\frac{\int_0^{x^2}f(u)du}{x^2\int_0^xf(u)du}\\
&=\lim_{x\to0}\frac{2xf(x^2)}{2x\int_0^xf(u)du+x^2f(x)}\\
&=\lim_{x\to0}\frac{2f(x^2)}{2\int_0^xf(u)du+xf(x)}\\
&=\lim_{x\to0}\frac{4xf'(x^2)}{3f(x)+xf'(x)}\\
&=\lim_{x\to0}\frac{4f'(x^2)}{3\frac{f(x)-f(0)}{x}+f'(x)}=1
\end{align*}
\end{examplle}

\begin{examplle}[]
求\(\displaystyle\lim_{x\to0}\frac{\frac{x^2}{2}+1-\sqrt{1+x^2}}{(\cos x-e^{x^2})\sin
  x^2}\)

利用泰勒展开,\(\sqrt{1+x^2}=1+\frac{1}{2}x^2-\frac{1}{8}x^4+o(x^4)\),
\(\cos x=1-\frac{1}{2}x^2+o(x^2)\),\(e^{x^2}=1+x^2+o(x^2)\),因此
\begin{equation*}
\lim_{x\to0}\frac{\frac{x^2}{2}+1-\sqrt{1+x^2}}{(\cos x-e^{x^2})\sin
x^2}=\lim_{x\to0}\frac{\frac{x^4}{8}+o(x^4)}{-\frac{3}{2}x^4+o(x^4)}=-\frac{1}{12}
\end{equation*}
\end{examplle}

\begin{examplle}[]
求\(\displaystyle\lim_{n\to\infty}\tan^n(\frac{\pi}{4}+\frac{2}{n})\)

因为\(\lim_{x\to\infty}f(x)=A\Rightarrow\lim_{n\to\infty}f(n)=A\)
\end{examplle}

\begin{examplle}[]
suppose \(\displaystyle y_n=\left[\frac{(2n)!}{n!n^n}\right]^{\frac{1}{n+1}}\). Compute
\(\lim_{n\to\infty}y_n\)

\begin{align*}
\ln y_n&=\frac{1}{n+1}\ln\frac{(2n)!}{n!n^n}=
\frac{1}{n+1}\ln\frac{(2n)(2n-1)\dots(n+1)}{n^n}\\
&=\frac{1}{n+1}\sum_{k=1}^n\ln(1+\frac{k}{n})=
\frac{n}{n+1}\left(
\frac{1}{n}\sum_{k=1}^n\ln(1+\frac{k}{n})
\right)
\end{align*}
Hence
\begin{align*}
\lim_{n\to\infty}y_n&=\lim_{n\to\infty}\frac{n}{n+1}\left(
\frac{1}{n}\sum_{k=1}^n\ln(1+\frac{k}{n})
\right)\\
&=1\cdot\int_0^1\ln(1+x)dx=
x\ln(1+x)\rvert_0^1-\int_0^1\frac{x}{1+x}dx\\
&=\ln2-1+\ln2=\ln\frac{4}{e}
\end{align*}
\end{examplle}

\begin{examplle}[]
已知\(x\to0\)时,\(e^{-x^4}-\cos(\sqrt{2}x^2)\) 与\(ax^n\)是等价无穷小,试求
\(a,n\)
\begin{align*}
&e^{-x^4}=1-x^4+\frac{x^8}{2}+o(x^8)\\
&\cos(\sqrt{2}x^2)=1-x^4+\frac{x^8}{6}+o(x^8)
\end{align*}
Hence \(a=\frac{1}{3},n=8\)
\end{examplle}

\begin{examplle}[]
设\(\displaystyle f(x)=\frac{\sqrt{1+\sin x+\sin^2x}-(\alpha+\beta\sin x)}{\sin^2x}\),且点
\(x=0\)是\(f(x)\)的可去间断点,求\(\alpha,\beta\)

由极限存在可知,\(\alpha=1\),泰勒展开
\begin{align*}
&\frac{\sqrt{1+\sin x+\sin^2x}-(\alpha+\beta\sin x)}{\sin^2x}\\
&=\lim_{x\to0}\frac{1+\frac{1}{2}(\sin x+\sin^2x)-\frac{1}{8}(\sin x+\sin^2x)^2-(1+\beta\sin x)
+o(\sin^2x)}{\sin^2}\\
&=\lim_{x\to0}\frac{(\frac{1}{2}-\beta)\sin x+\frac{3}{8}\sin^2x}{\sin^2x}
\end{align*}
故\(\beta=\frac{1}{2}\)
\end{examplle}

\begin{examplle}[]
let \(\displaystyle f(x)=\lim_{n\to\infty}\frac{2x^n-3x^{-n}}{x^n+x^{-n}}\sin\frac{1}{x}\)

\begin{equation*}
f(x)=
\begin{cases}
2\sin\frac{1}{x}^x&x<-1\\
-\frac{1}{2}\sin\frac{1}{x}&x=-1\\
-3\sin\frac{1}{x}&-1<x<0\\
-3\sin\frac{1}{x}&0<x<1\\
-\frac{1}{2}\sin\frac{1}{x}&x=1\\
2\sin\frac{1}{x}^x&x>1
\end{cases}
\end{equation*}
\(x=0\)是第二类间断点,\(x=\pm1\)是第一类间断点
\end{examplle}

\begin{examplle}[]
设\(f(1)=0,f'(1)=a\),求极限\(\displaystyle
  \lim_{x\to0}\frac{\sqrt{1+2f(e^{x^2})}- \sqrt{1+f(1+\sin^2 x)}}{\ln\cos x}\)

由\(f(1)=0,f'(1)=a\)可知,\(\displaystyle
  f'(1)=\lim_{x\to1}\frac{f(x)-f(1)}{x-1}=\lim_{t\to0}\frac{f(1+t)}{t}=a\)

\begin{align*}
\lim_{x\to0}\frac{\sqrt{1+2f(e^{x^2})}- \sqrt{1+f(1+\sin^2 x)}}{\ln\cos x}&=
\frac{2f(e^{x^2})-f(1+\sin^2x)}{-\frac{1}{2}x^2
\left[\sqrt{1+2f(e^{x^2})}+\sqrt{1+f(1+\sin^2x)}
\right]}\\
&=\lim_{x\to0}\frac{f(1+\sin^2x)-f(e^{x^2})}{x^2}\\
&=\lim_{x\to0}\left[
\frac{f(1+\sin^2x)}{\sin^2x}\cdot\frac{\sin^2x}{x^2}-
\frac{f(e^{x^2})}{e^{x^2}-1}\cdot\frac{e^{x^2}-1}{x^2}
\right]\\
&=-a
\end{align*}
\end{examplle}

\begin{examplle}[]
设\(f(x)\)在\(x=0\)的某邻域内二阶可导,且
\(\displaystyle\lim_{x\to0}\frac{f(x)}{x}=0\),\(f''(0)\neq0\),
\(\displaystyle\lim_{x\to0^+}\frac{\int_0^xf(x)dt}{x^\alpha-\sin
  x}=\beta(\beta\neq0)\),求\(\alpha,\beta\)

因为\(\lim_{x\to0}\frac{f(x)}{x}=0\),\(f(0)=0,f'(0)=0\)

因为\(\lim_{x\to0^+}\int_0^xf(x)dt=0\),因此\(\lim_{x\to0^+}x^\alpha-\sin
  x=0\),因此\(\alpha>0\)
\begin{enumerate}
\item 若\(0<\alpha<1\)
\item 若\(\alpha>1\)
\item 若\(\alpha=1\)

\(\beta=f''(0)\)
\end{enumerate}
\end{examplle}

\begin{examplle}[]
设\(f(x)\)在\((-\infty,+\infty)\)上有定义,且\(f'(0)=1\),
\(f(x+y)=f(x)e^y+f(y)e^x\),求\(f(x)\)

\(f(0)=0\)

\begin{align*}
f'(x)&=\lim_{y\to0}\frac{f(x+y)-f(x)}{y}\\
&=\lim_{y\to0}\frac{f(x)e^y+f(y)e^x-f(x)}{y}\\
&=\lim_{y\to0}\left[
f(x)\frac{e^y-1}{y}+e^x\frac{f(y)-f(0)}{y}
\right]\\
&=f(x)+e^xf'(0)=f(x)+e^x
\end{align*}
即\(f'(x)-f(x)=e^x\),因此\(f(x)=e^x(x+C)\),又\(f(0)=0,C=0,f(x)=xe^x\)
\end{examplle}

\begin{examplle}[]
已知函数\(\displaystyle f(x)=\begin{cases}x&x\le0\\\frac{1}{n}&\frac{1}{n+1}
  <x\le\frac{1}{n}\end{cases}\)

\begin{equation*}
f_+'(0)=\lim_{x\to0^+}\frac{f(x)-f(0)}{x}=\lim_{x\to0^+}\frac{\frac{1}{n}}{x}
\left(\frac{1}{n+1}<x\le\frac{1}{n}
\right)
\end{equation*}
而\(1\le\frac{\frac{1}{n}}{x}<\frac{n+1}{n}\),由夹逼准则得\(f'_+(0)=1\),因此\(f'(0)=1\)
\end{examplle}

\begin{examplle}[]
设\(f(x)\)是可导的偶函数,它在\(x=0\)的某邻域内满足
\begin{equation*}
f(e^{x^2})-3f(1+\sin x^2)=2x^2+o(x^2)
\end{equation*}
求曲线\(y=f(x)\)在点\((-1,f(-1))\)处的切线方程

由
\begin{equation*}
\lim_{x\to0}\frac{f(e^{x^2})-3f(1+\sin x^2)-2x^2}{x^2}=0
\end{equation*}
得
\begin{equation*}
f(0)-3f(1)=0\Rightarrow f(1)=0
\end{equation*}
变形
\begin{equation*}
\lim_{x\to0}\left(
\frac{f(e^{x^2})}{e^{x^2}-1}\cdot\frac{e^{x^2}-1}{x^2}-
\frac{3f(1+\sin x^2)}{\sin x^2}\cdot\frac{\sin x^2}{x^2}-2
\right)=0
\end{equation*}
有\(f'(1)-3f'(1)-2=0\Rightarrow f'(1)=-1\)
\end{examplle}

\begin{examplle}[]
若\(y=f(x)\)存在单值反函数,且\(y'\neq0\),求\(\frac{d^2x}{dy^2}\)

根据反函数的求导法则\(\frac{dx}{dy}=\frac{1}{y'}\),于是
\begin{equation*}
\frac{d^2x}{dy^2}=\frac{d}{dy}\left(\frac{dx}{dy}\right)=
\frac{d}{dx}\left(\frac{dx}{dy}\right)\frac{dx}{dy}
\end{equation*}
因为\(\frac{1}{y'}\)是以\(x\)为变量的函数
\end{examplle}

\begin{examplle}[]
设函数\(f(x)=\arctan x-\frac{x}{1+ax^2}\),且\(f'''(0)=1\),求\(a\)

泰勒展开
\begin{align*}
f(x)&=\arctan x-\frac{x}{1+ax^2}=
\left(x-\frac{x^3}{3}+\dots
\right)-x(1-ax^2+\dots)\\
&=(a-\frac{1}{3})x^3+\dots
\end{align*}
因此\(f'''(0)/3!=a-1/3,a=1/2\)
\end{examplle}

\begin{examplle}[]
设\(f(x)\)在\([a,b]\)上连续且\(f(x)>0\),证明存在\(\xi\in(a,b)\)使得
\begin{equation*}
\int_a^\xi f(x)dx=\int_\xi^bf(x)dx=\frac{1}{2}\int_a^bf(x)dx
\end{equation*}

令\(F(x)=\int_a^xf(t)dt-\int_x^bf(t)dt\),则\(F(x)\)在\([a,b]\)上连续,且
\begin{equation*}
F(a)F(b)=-\left[\int_a^bf(t)dt\right]^2<0
\end{equation*}
故由连续函数的零点定理知:在\((a,b)\)内存在 \(\xi\) 使得\(F(\xi)=0\),即\(\int_a^\xi f(x)dx=\int_\xi^bf(x)dx\)
\end{examplle}

\begin{examplle}[]
设\(f(x),g(x)\)在\([a,b]\)上连续,证明存在\(\xi\in(a,b)\)使得
\begin{equation*}
g(\xi)\int_a^\xi f(x)dx=f(\xi)\int_\xi^bg(x)dx
\end{equation*}

令\(F'(x)=g(x)\int_a^x
  f(x)dx-f(x)\int_x^bg(x)dx=(\int^x_af(t)dt\int_b^xg(t)dt)'\),可取辅助函数
\(F(x)=\int_a^xf(t)dt\int_x^bg(t)dt\)。则\(F(a)=F(b)=0\),则存在
\(\xi\in(a,b)\)使得\(F'(\xi)=0\)
\end{examplle}

\begin{examplle}[]
设实数\(a_1,\dots,a_n\)满足关系式
\(a_1-\frac{a_2}{3}+\dots+(-1)^{n-1}\frac{a_n}{2n-1}=0\),证明方程
\(a_1\cos x+a_2\cos 3x+\dots+a_n\cos(2n-1)x=0\)在\((0,\frac{\pi}{2})\)内至少有一
实根

令\(f(x)=a_1\cos x+a_2\cos 3x+\dots+a_n\cos(2n-1)x\),但\(f(x)\)在
\([0,\frac{\pi}{2}]\)内不满足零点定理,因此考虑
\(f'(x)=a_1\cos x+a_2\cos 3x+\dots+a_n\cos(2n-1)x\),则
\(f(x)=a_1\cos x+\frac{a_2}{3}\sin 3x+\dots+\frac{a_n}{2n-1}\sin(2n-1)x\),则
\(f(0)=f(\pi/2)=0\)
\end{examplle}

\begin{examplle}[]
试确定方程\(e^x=ax^2(a>0)\)的根的个数,并指出每个根所在的范围

若直接令\(f(x)=e^x-ax^2\),\(f'(x)\)的符号不易判断。又\(x=0\)不是方程的根,于
是方程可化为等价方程\(\frac{e^x}{x^2}=a\)

令\(f(x)=\frac{e^x}{x^2}-a\),由\(f'(x)=\frac{x-2}{x^3}e^x=0\)得\(x=2\)
\end{examplle}

\begin{examplle}[]
已知方程\(\frac{1}{\ln(1+x)}-\frac{1}{x}=k\)在区间\((0,1)\)内有实根,确定常数
\(k\)的取值范围

令\(f(x)=\frac{1}{\ln(1+x)}-\frac{1}{x}-k\),\(x\in(0,1]\),则
\begin{equation*}
f'(x)=\frac{(1+x)\ln^2(1+x)-x^2}{x^2(1+x)\ln^2(1+x)}
\end{equation*}
因为\(x^2(1+x)\ln^2(1+x)>0\),因此只讨论\(g(x)=(1+x)\ln^2(1+x)-x^2\).
\begin{align*}
&g'(x)=\ln^2(1+x)+2\ln(1+x)-2x\\
&g''(x)=\frac{2\ln(1+x)}{1+x}+\frac{2}{1+x}-2=\frac{2\ln(1+x)-2x}{1+x}
\end{align*}
因此当\(x\in(0,1)\)时,\(g''(x)<0\),而\(g'(0)=0\),因此\(g(x)\)递减
\end{examplle}

\begin{examplle}[]
设\(f(x)\)在\([0,3]\)上连续,在\((0,3)\)内可导,且\(f(0)+f(1)+f(2)=3,f(3)=1\),
证明存在\(\xi\in(0,3)\)使得\(f'(\xi)=0\)

因为\(f(x)\)在\([0,3]\)上连续,所以在\([0,2]\)内必有最大值\(M\)和最小值\(m\),
于是\(m\le f(0)\le M,m\le f(1)\le M,m\le f(2)\le M\),故
\begin{equation*}
m\le\frac{f(0)+f(1)+f(2)}{3}\le M
\end{equation*}
由介值定理,至少存在一点\(\eta\in[0,2]\)使
\begin{equation*}
f(\eta)=\frac{f(0)+f(1)+f(2)}{3}=1
\end{equation*}
因此\(f(\eta)=f(3)=1\),由罗尔定理知,必存在\(\xi\in(\eta,3)\subset(0,3)\)使得\(f'(\xi)=0\)
\end{examplle}

\begin{examplle}[]
设\(f(x)\)在\([0,2]\)上连续,在\((0,2)\)内具有二阶导数且
\(\displaystyle\lim_{x\to\frac{1}{2}}\frac{f(x)}{\cos\pi x}=0\),
\(2\int_{1/2}^1f(x)dx=f(2)\),证明存在\(\xi\in(0,2)\)使得\(f''(\xi)=0\)

\(f(0.5)=0\),因此
\begin{equation*}
f'(0.5)=\lim_{x\to0.5}\frac{f(x)-f(0.5)}{x-0.5}=
\lim_{x\to0.5}\frac{f(x)}{\cos\pi x}\frac{\cos\pi x}{x-0.5}=
\lim_{x\to0.5}\frac{f(x)}{\cos\pi x}\lim_{x\to0.5}\frac{\cos\pi x}{x-0.5}=0
\end{equation*}
再由\(2\int_{0.5}^2f(x)dx=f(2)\),用积分中值定理\(\exists\xi_1\in[0.5,1]\)使得
\(2f(\xi_1)0.5=f(2)\),即\(f(\xi)=f(2)\),在\([\xi_1,2]\)上应用罗尔定理,
\(\exists\xi_2\in(\xi_1,2)\)使\(f'(\xi_2)=0\)

再在\([0.5,\xi_2]\)上对\(f'(x)\)应用罗尔定理,知\(\exists\xi\in(0.5,\xi_2)\),
使\(f''(\xi)=0\)
\end{examplle}

\begin{examplle}[]
设\(f(x)\)在\([0,1]\)上连续,\((0,1)\)内可导,且
\begin{equation*}
f(1)=k\int_0^{\frac{1}{k}}xe^{1-x}f(x)dx,k>1
\end{equation*}
证明:在\((0,1)\)内至少存在一点 \(\xi\) 使\(f'(\xi)=(1-\xi^{-1})f(\xi)\)

\begin{enumerate}
\item \(\xi\) 换为\(x\),\(f'(x)=(1-x^{-1})f(x)\)
\item 变形\(\frac{f'(x)}{f(x)}=1-x^{-1}\)
\item 两边积分\(\ln f(x)=x-\ln x+ \ln C\)
\item 分离常数\(\ln\frac{xf(x)}{e^x}=\ln C\),即\(xe^{-x}f(x)=C\),可令辅助函数
\(F(x)=xe^{-x}f(x)\)
\end{enumerate}


由积分中值定理,存在\(\xi_1\in[0,\frac{1}{k}]\)使得
\(f(1)=\xi_1e^{1-\xi_1}f(\xi_1)\),即\(1\times e^{-1}f(1)=\xi_1
  e^{-\xi_1}f(\xi_1)\)。因此\(F(x)\)满足在\([\xi_1,1]\)内的罗尔定理,因此
存在 \(\xi\) 使得 \(f'(\xi)=(1-\xi^{-1})f(\xi)\)
\end{examplle}

\begin{examplle}[]
设\(f(x)\)在\([a,b]\)上连续,在\((a,b)\)内可导,且\(f(a)=f(b)=\lambda\),证明
存在\(\xi\in(a,b)\)使得\(f'(\xi)+f(\xi)=\lambda\)

\begin{enumerate}
\item \(\xi\) 换为\(x\),\(f'(x)+f(x)=\lambda\)这是关于\(f(x)\)的一阶线性微分方程
\item 解微分方程\(f(x)=e^{-x}(\lambda e^x+C)\)
\item 分离常数\([f(x)-\lambda]e^x=C\),可令辅助函数\(F(x)=[f(x)-\lambda]e^x\)
\end{enumerate}


\(F(a)=F(b)=0\),因此存在\(\xi\in[a,b]\)使得\(F'(\xi)=0\)
\end{examplle}

\begin{examplle}[]
设\(f(x)\)在\([a,b]\)上连续,在\((a,b)\)上可导,求证:存在\(\xi\in(a,b)\)使得
\(f(b)-f(a)=\xi\ln\frac{b}{a}f'(\xi)\)

可变形为
\begin{equation*}
\frac{f(b)-f(a)}{\ln b-\ln a}=\xi f'(\xi)
\end{equation*}
令\(F(x)=\ln x\),由柯西中值定理,存在\(\xi\in(a,b)\)使得
\begin{equation*}
\frac{f(b)-f(a)}{\ln b-\ln a}=\frac{f'(\xi)}{F'(\xi)}=\xi f'(\xi)
\end{equation*}
\end{examplle}

\begin{examplle}[]
设\(f(x)\)在\([-1,1]\)上具有三阶连续导数,且\(f(-1)=0,f(1)=1,f'(0)=0\),证明:
在\((-1,1)\)内存在一点 \(\xi\) 使得\(f'''(\xi)=3\)

泰勒展开
\(f(x)=f(0)+f'(0)x+\frac{1}{2!}f''(0)x^2+\frac{1}{3!}f'''(\xi)x^3,\xi\in(0,x)\),
则
\begin{align*}
&0=f(-1)=f(0)+\frac{1}{2}f''(0)-\frac{1}{6}f'''(\xi_1),-1<\xi_1<0\\
&1=f(1)=f(0)+\frac{1}{2}f''(0)+\frac{1}{6}f'''(\xi_2),0<\xi_2<1
\end{align*}
两式相减得
\begin{equation*}
\frac{f'''(\xi_1)+f'''(\xi_2)}{2}=3
\end{equation*}
由介值定理可证存在\(\xi\in[\xi_1,\xi_2]\)有\(f'''(\xi)=\frac{f'''(\xi_1)+f'''(\xi_2)}{2}=3\)
\end{examplle}

\begin{examplle}[]
设\(f(x)\)在\([a,b]\)上连续,在\((a,b)\)内可导,\(0<a<b\),求证存在
\(\xi,\eta\in(a,b)\)使得\(f'(\xi)=\frac{f'(\eta)}{2\eta}(a+b)\)

根据拉格朗日中值定理至少存在一个\(\xi\in(a,b)\)使得
\begin{equation*}
f'(\xi)=\frac{f(b)-f(a)}{b-a}
\end{equation*}
只要再证存在\(\eta\in(a,b)\)使得
\(\frac{f(b)-f(a)}{b-a}=\frac{f'(\eta)}{2\eta}(a+b)\)即
\begin{equation*}
\frac{f(b)-f(a)}{b^2-a^2}=\frac{f'(\eta)}{2\eta}
\end{equation*}
只要用柯西中值定理
\end{examplle}

\begin{examplle}[]
已知函数\(f(x)\)在\([0,1]\)上连续,在\((0,1)\)内可导,且\(f(0)=0,f(1)=1\),证
明
\begin{enumerate}
\item 存在\(\xi\in(0,1)\)使得\(f(\xi)=1-\xi\)
\item 存在两个不同的点\(\eta,\zeta\in(0,1)\)使得\(f'(\eta)f'(\zeta)=1\)
\end{enumerate}


令\(F(x)=f(x)-1+x\),则\(F(0)=-1,F(1)=1\)

对\([0,\xi],[\xi,1]\)分别用拉格朗日中值定理,则
\begin{equation*}
f'(\eta)f'(\zeta)=\frac{f(\xi)-f(0)}{\xi-0}\frac{f(1)-f(\xi)}{1-\xi}=
\frac{f(\xi)}{\xi}\frac{1-f(\xi)}{1-\xi}=
\frac{1-\xi}{\xi}\frac{\xi}{1-\xi}=1
\end{equation*}
\end{examplle}

\begin{examplle}[]
求证\(\frac{\tan x}{x}>\frac{x}{\sin x},0<x<\frac{\pi}{2}\)

\begin{align*}
&f(x)=\sin x\tan x-x^2\\
&f'(x)=\sin x+\tan x\sec x-2x\\
&f''(x)=\cos x+\sec^3x+\tan^2x\sec x-2\\
&f'''(x)=-\sin x+5\sec^3x\tan x+\tan^3x\sec x=
\sin x(5\sec^4x-1)+\tan^3x\sec x>0
\end{align*}
\end{examplle}

\begin{examplle}[]
设\(a>0,b>0\),证明不等式
\begin{equation*}
a\ln a+b\ln b\ge(a+b)[\ln(a+b)-\ln2]
\end{equation*}

令\(f(x)=x\ln x\),则\(f'(x)=\ln x+1,f''(x)=\frac{1}{x}>0\),即曲线\(y=f(x)\)
在\((0,+\infty)\)是凹的,故对任意\(a>0,b>0\),有
\begin{equation*}
\frac{f(a)+f(b)}{2}\ge f(\frac{a+b}{2})
\end{equation*}
代入得
\begin{equation*}
\frac{a\ln a+b\ln b}{2}\ge\frac{a+b}{2}\ln\frac{a+b}{2}
\end{equation*}
\end{examplle}

\begin{examplle}[]
证明:对任意正整数\(n\),都有
\(\frac{1}{n+1}\le\ln(1+\frac{1}{n})<\frac{1}{n}\)

由拉格朗日定理,存在\(\xi\in(n,n+1)\)
\begin{gather*}
\ln(1+\frac{1}{n})=\ln(n+1)-\ln n=\frac{1}{\xi}\\
\frac{1}{n+1}<\frac{1}{\xi}<\frac{1}{n}
\end{gather*}
\end{examplle}

\begin{examplle}[]
设\(f(x)\)在\([0,1]\)上二阶可导,且\(f(0)=f(1)=0\),\(f(x)\)在\([0,1]\)上的最
小值等于\(-1\),证明:至少存在一点\(\xi\in(0,1)\)使\(f''(x)\ge8\)

存在\(a\in(0,1),f'(a)=0,f(a)=-1\),将\(f(x)\)在\(x=a\)泰勒展开
\begin{equation*}
f(x)=f(a)+f'(a)(x-a)+\frac{f''(\xi)}{2!}(x-a)^2=-1+\frac{f''(\xi)}{2}(x-a)^2(\xi\in(a,x)\text{ or }(x,a))
\end{equation*}
令\(x=0,x=1\)得
\begin{gather*}
f(0)=0=-1+\frac{f''(\xi_1)}{2}a^2,0<\xi_1<a\\
f(1)=0=-1+\frac{f''(\xi_2)}{2}(1-a)^2,a<\xi_2<1
\end{gather*}
若\(0<a<\frac{1}{2}\),则\(f''(\xi_1)>8\)


若\(\frac{1}{2}<a<1\),则\(f''(\xi_2)>8\)
\end{examplle}

\begin{examplle}[]
设函数\(f(x)\)在\([0,1]\)上二阶可导,且\(\int_0^1f(x)dx=0\),则当\(f''(x)>0\)
时

\begin{equation*}
f(x)=f(0.5)+f'(0.5)(x-0.5)+\frac{f''(\xi)}{2}(x-0.5)^2
\end{equation*}
积分
\begin{align*}
0&=f(0.5)+f'(0.5)\int_0^1(x-0.5)dx+\frac{f''(\xi)}{2}\int_0^(x-0.5)^2dx\\
&=f(0.5)+\frac{1}{2}f''(\xi)\int_0^1(x-0.5)^2dx
\end{align*}
因此\(f(0.5)<0\)
\end{examplle}

\begin{examplle}[]
设函数\(f(x)\)在点\(x=0\)可导,且\(f(0)=0\),求\(\lim_{x\to0}\frac{f(1-\cos
  x)}{\tan^2x}\)

\begin{align*}
\lim_{x\to0}\frac{f(1-\cos
x)}{\tan^2x}&=
\lim_{x\to0}\frac{f(1-\cos x)-f(0)}{1-\cos x}\frac{1-\cos x}{\tan2^x}\\
&=f'(0)\cdot\frac{1}{2}
\end{align*}
\end{examplle}

\begin{examplle}[]
设\(f(x)\)在\([a,b]\)上连续,在\((a,b)\)内可导,且\(f(a)\cdot f(b)>0,f(a)\cdot
  f(\frac{a+b}{2})<0\),证明: 对任意实数\(k\),存在\(\xi\in(a,b)\)使得$\backslash$(f'(\(\xi\))=kf(\(\xi\)))$\backslash$
\end{examplle}

\begin{examplle}[]
设\(f(x)\)在\([a,b]\)上连续,在\((a,b)\)内可导,且\(f(a)=f(b)=1\),证明:存在
两点\(\xi,\eta\in(a,b)\)使
\begin{equation*}
(e^{2a}+e^{a+b}+e^{2b})[f(\xi)+f'(\xi)]=3e^{3\eta-\xi}
\end{equation*}


\begin{align*}
&(e^{2a}+e^{a+b}+e^{2b})[f(\xi)+f'(\xi)]=3e^{3\eta-\xi}\\
&\Leftrightarrow (e^{2a}+e^{a+b}+e^{2b})[f(\xi)+f'(\xi)]e^{\xi}=3e^{3\eta}\\
&\Leftrightarrow(e^{2a}+e^{a+b}+e^{2b})[e^xf(x)]'|_{x=\xi}=
e^{3x}|_{x=\eta}
\end{align*}

令\(g(x)=e^{3x}\),则由拉格朗日中值定理
\begin{equation*}
g'(\eta)=\frac{g(b)-g(a)}{b-a}
\end{equation*}
即\(\displaystyle  3e^{3\eta}=\frac{e^{3b}-e^{3a}}{b-a}\). 令\(f(x)=e^xf(x)\),
由拉格朗日中值定理,存在\(\xi\in(a,b)\)使得
\begin{equation*}
\frac{e^bf(b)-e^af(a)}{b-a}=e^{\xi}[f(\xi)+f'(\xi)]=\frac{e^b-e^a}{b-a}
\end{equation*}
两边同乘\(e^{2a}+e^{a+b}+e^{2b}\)得
\begin{equation*}
\frac{e^{3b}-e^{3a}}{b-a}=(e^{2a}+e^{a+b}+e^{2b})e^{\xi}[f(\xi)+f'(\xi)]
\end{equation*}
\end{examplle}
\subsection{一元函数积分}
\label{sec:org825b4b8}
\begin{examplle}[]
求不定积分\(\displaystyle\int\frac{2^x\cdot 3^x}{9^x-4^x}dx\)

\begin{align*}
\int\frac{2^x\cdot 3^x}{9^x-4^x}dx&=
\int\frac{\left(\frac{3}{2}\right)^x}{\left(\frac{3}{2}\right)^{2x}-1}dx=
\frac{1}{\ln\frac{3}{2}}\int\frac{d\left[\left(\frac{3}{2}\right)^x\right]}
{\left[\left(\frac{3}{2}\right)^{2x}\right]-1}\\
&=\frac{1}{2(\ln3-\ln2)}\ln\abs{\frac{\left(\frac{3}{2}\right)^x-1}
{\left(\frac{3}{2}\right)^x+1}}
\end{align*}
\end{examplle}

\begin{examplle}[]
求\(\displaystyle\int\frac{dx}{\cos x\sqrt{\sin x}}\)

\begin{align*}
\int\frac{dx}{\cos x\sqrt{\sin x}}&=
\int\frac{\cos xdx}{(1-\sin^2x)\sqrt{\sin x}}=
2\int\frac{d(\sqrt{\sin x})}{1-(\sqrt{\sin x})^4}=2\int\frac{dt}{1-t^4}\\
&\int\left(\frac{1}{1+t^2}+\frac{1}{1-t^2}\right)dt
\end{align*}
\end{examplle}

\begin{examplle}[]
求\(\displaystyle\int\frac{dx}{\sqrt{x(4-x)}}\)

\begin{equation*}
\int\frac{dx}{\sqrt{x(4-x)}}=
\int\frac{2d(\sqrt{x})}{\sqrt{4-x}}=2\arcsin\frac{\sqrt{x}}{2}+C
\end{equation*}
\end{examplle}

\begin{examplle}[]
求\(\displaystyle\int\frac{1}{1+e^x}dx\)

\begin{equation*}
\int\frac{1}{1+e^x}dx=\int\frac{e^x}{e^x(1+e^x)}dx=
\int\left(\frac{1}{e^x}-\frac{1}{e^x+1}\right)de^x
\end{equation*}
\end{examplle}

\begin{examplle}[]
求\(\displaystyle\int\frac{xe^x}{\sqrt{e^x-1}}dx\)

令\(\sqrt{e^x-1}=t,x=\ln(1+t^2)\)
\begin{equation*}
\int\frac{xe^x}{\sqrt{e^x-1}}=2\int\ln(1+t^2)dt
\end{equation*}
\end{examplle}


\begin{examplle}[]
求\(\displaystyle\int\frac{dx}{x^4(1+x^2)}\)

\begin{align*}
\int\frac{dx}{x^4(1+x^2)}&=
\int\frac{1+x^2-x^2}{x^4(1+x^2)}dx
\end{align*}
\end{examplle}

\begin{examplle}[]
求\(\displaystyle\int\frac{3x^2-x+4}{x^3-x^2+2x-2}dx\)

\(x^3-x^2+2x-2=(x^2+2)(x-1)\),令
\begin{equation*}
\frac{3x^2-x+4}{x^3-x^2+2x-2}=
\frac{A}{x-1}+\frac{Bx+C}{x^2+2}
\end{equation*}
\end{examplle}

\begin{examplle}[]
求\(\displaystyle\int\frac{dx}{1+\sin x}\)

\begin{equation*}
\int\frac{dx}{1+\sin x}=\int\frac{1-\sin x}{\cos^2 x}=
\int\frac{dx}{\cos^2x}-\int\frac{\sin x}{\cos^2 x}=\tan x-\frac{1}{\cos x}+C
\end{equation*}
\end{examplle}

\begin{examplle}[]
求\(I_n=\int\tan^nxdx\)的递推公式

\begin{align*}
I_n&=\int\tan^{n-2}x(\sec^2x-1)dx=\int\tan^{n-2}x\sec^2 xdx-\int\tan^{n-2}xdx\\
&=\frac{1}{n-1}\tan^{n-1}x-I_{n-2}
\end{align*}
\end{examplle}

\begin{examplle}[]
求\(\displaystyle\lim_{n\to\infty}\int_0^1\frac{x^n}{1+x}dx\)

对于\(0\le x\le1\),有\(0\le\frac{x^n}{1+x}\le x\),则
\begin{equation*}
0\le\int_0^1\frac{x^n}{1+x}dx\le\int^1_0x^ndx=\frac{1}{n+1}
\end{equation*}
因此由夹逼定理,\(\displaystyle\lim_{n\to\infty}\int_0^1\frac{x^n}{1+x}dx=0\)
\end{examplle}

\begin{examplle}[]
求\(\displaystyle\lim_{n\to\infty}n(\frac{1}{1+n^2}+\dots+\frac{1}{n^2+n^2})\)

\begin{align*}
\lim_{n\to\infty}n(\frac{1}{1+n^2}+\dots+\frac{1}{n^2+n^2})&=
\lim_{n\to\infty}\left[
\frac{1}{(\frac{1}{n})^2+1}+\dots+\frac{1}{(\frac{n}{n})^2+1}
\right]\cdot\frac{1}{n}\\
&=\left.\int_0^1\frac{1}{1+x^2}dx=\arctan\right\rvert_0^1=\frac{\pi}{4}
\end{align*}
\end{examplle}

\begin{examplle}[]
证明下列不等式
\begin{equation*}
\frac{\sqrt{\pi}}{80}\pi^2<\int_0^{\frac{\pi}{4}}x\sqrt{\tan x}dx<
\frac{\pi^2}{32}
\end{equation*}

当\(0<x<\frac{\pi}{4}\)时,\(0<x<\tan x<1\),则
\begin{equation*}
\int_0^{\frac{\pi}{4}}x^{3/2}dx<\int_0^{\frac{\pi}{4}}x\sqrt{\tan x}dx
<\int^{\frac{\pi}{4}}_0xdx
\end{equation*}
\end{examplle}

\begin{examplle}[]
求\(\displaystyle\int_2^3\frac{\sqrt{3+2x-x^2}}{(x-1)^2}dx\)

\begin{align*}
\int_2^3\frac{\sqrt{3+2x-x^2}}{(x-1)^2}dx&=
\int_2^3\frac{\sqrt{4-(x-1)^2}}{(x-1)^2}dx=
\int^{\frac{\pi}{2}}_{\frac{\pi}{6}}\frac{\sqrt{4-4\sin^2t}}{4\sin^2t}2\cos tdt\\
&=\int^{\frac{\pi}{2}}_{\frac{\pi}{6}}\frac{\cos^2t}{\sin^t}dt=
\int^{\frac{\pi}{2}}_{\frac{\pi}{6}}(\csc^2t-1)dt=-\cot t\rvert^{\frac{\pi}{2}}_{\frac{\pi}{6}}
-t\rvert^{\frac{\pi}{2}}_{\frac{\pi}{6}}=\sqrt{3}-\frac{\pi}{3}
\end{align*}
\end{examplle}

\begin{examplle}[]
求\(\displaystyle\int_0^{\ln2}\sqrt{1-e^{-2x}}dx\)

令\(e^{-x}=\sin t\),则
\begin{align*}
\int_0^{\ln2}\sqrt{1-e^{-2x}}dx&=
\int_{\frac{\pi}{6}}^{\frac{\pi}{2}}\cos t\cdot\frac{\cos t}{\sin t}dt=
\int_{\frac{\pi}{6}}^{\frac{\pi}{2}}\frac{1}{\sin t}dt-
\int_{\frac{\pi}{6}}^{\frac{\pi}{2}}\sin tdt\\
&=-\ln(\csc t+\cot t)\rvert_{\frac{\pi}{6}}^{\frac{\pi}{2}}-\frac{\sqrt{3}}{2}
=\ln(2+\sqrt{3})-\frac{\sqrt{3}}{2}
\end{align*}
\end{examplle}

\begin{examplle}[]
求\(\displaystyle\int_0^3\arcsin\sqrt{\frac{x}{1+x}}dx\)

令\(\arcsin\sqrt{\frac{x}{1+x}}=t\),则
\(\sin^2u=\frac{x}{1+x},x\cos^2u=\sin^2u,x=\tan^2u\)
\begin{align*}
\int_0^3\arcsin\sqrt{\frac{x}{1+x}}dx&=
\left.\int_0^{\frac{\pi}{3}}ud(\tan^2u)=(u\cdot\tan^2u)\right\rvert_0^{\frac{\pi}{3}}
-\int_0^{\frac{\pi}{3}}1\cdot\tan^2udu\\
&\left.=\pi-\int_0^{\frac{\pi}{3}}(\sec^2u-1)du=\pi-\tan u\right\rvert_0^{\frac{\pi}{3}}
+\frac{\pi}{3}\\
&=\frac{4}{3}\pi-\sqrt{3}
\end{align*}
\end{examplle}

\begin{examplle}[]
求\(I=\displaystyle\int_{-\frac{\pi}{4}}^{\frac{\pi}{4}}\frac{\cos^2x}{1+e^{-x}}dx\)

令\(x=-t\),则
\(I=\displaystyle\int_{-\frac{\pi}{4}}^{\frac{\pi}{4}}\frac{\cos^2x}{1+e^{x}}dx\)。
因此
\begin{align*}
I&=\frac{1}{2}\int_{-\frac{\pi}{4}}^{\frac{\pi}{4}}
\left(\frac{\cos^2x}{1+e^{-x}}+\frac{\cos^2x}{1+e^{x}}
\right)dx=
\int^{\frac{\pi}{4}}_0
\left(\frac{1+e^{-x}+1+e^x}{(1+e^{-x})(1+e^x)}
\right)\cos^2xdx\\
&=\int^{\frac{\pi}{4}}_0\cos^2dx=\frac{\pi}{8}+\frac{1}{4}
\end{align*}
\end{examplle}

\begin{remark}
一般地,有如下结论:作变换\(x=a+b-t\)
\begin{equation*}
I=\int^b_af(x)dx=\int^b_af(a+b-t)dt
\end{equation*}
从而\(I=\frac{1}{2}\int^b_a[f(x)+f(a+b-x)]dx\)
\end{remark}

\begin{examplle}[]
求\(I=\displaystyle\int_0^{\frac{\pi}{2}}\frac{\sin^3x}{\sin x+\cos x}dx\)

令\(x=\frac{\pi}{2}-t\),则
\begin{align*}
I&=\int_0^{\frac{\pi}{2}}\frac{\sin^3x+\cos^3x}{\sin x+\cos x}dx=
\frac{1}{2}\int_0^{\frac{\pi}{2}}(\sin^2x-\sin x\cos x+\cos^2x)dx\\
&=\frac{1}{2}\int_0^{\frac{\pi}{2}}(1-\frac{1}{2}\sin 2x)dx=\frac{\pi-1}{4}
\end{align*}
\end{examplle}

\begin{remark}
要求\(I=\displaystyle\int^{\frac{\pi}{2}}_0f(\sin x,\cos x)dx\),可作变换
\(x=\frac{\pi}{2}-t\),则\(I=\displaystyle\int^{\frac{\pi}{2}}_0f(\cos x,\sin x)dx\)
\end{remark}

\begin{examplle}[]
求\(I=\int^\pi_0\frac{x\sin x}{1+\cos^2x}dx\)

令\(x=\pi-t\),则
\begin{align*}
I&=\int^\pi_0\frac{(\pi-t)\sin t}{1+\cos^2t}dt=
\pi\int^\pi_0\frac{\sin t}{1+\cos^2t}dt-I
\end{align*}
\end{examplle}

\begin{remark}
一般地,\(I=\int^\pi_0xf(\sin x)dx=\int^\pi_0(\pi-t)f(\sin
   t)dt=\pi\int^\pi_0f(\sin t)dt-I\)
\end{remark}

\begin{examplle}[]
求\(\int_0^1\frac{x^b-x^a}{\ln x}dx,a,b>0\)

\begin{align*}
\int_0^1\frac{x^b-x^a}{\ln x}dx,a,b>0&=
\int^1_0\left[f^b_ax^tdt
\right]dx=\int^b_a\left[\int^1_0x^tdx
\right]dt\\
&=\ln\frac{b+1}{a+1}
\end{align*}
\end{examplle}

\begin{examplle}[]
设\(\displaystyle f(x)=\int_0^x\frac{\sin t}{\pi-t}dt\),求
\(\int_0^\pi f(x)dx\)

\begin{align*}
\int^\pi_0f(x)dx&=\int_0^\pi f(x)d(x-\pi)\\
&=(x-\pi)f(x)|^\pi_0-\int_0^\pi(x-\pi)f'(x)dx\\
&=-\int_0^\pi (x-\pi)\frac{\sin x}{\pi-x}dx=2
\end{align*}
\end{examplle}

\begin{examplle}[]
证明\(\displaystyle\int_1^af(x^2+\frac{a^2}{x^2})\frac{dx}{x}=
   \int_1^af(x+\frac{a^2}{x})\frac{dx}{x}\)

\begin{align*}
\int_1^af(x^2+\frac{a^2}{x^2})\frac{dx}{x}&=\frac{1}{2}\int_1^{a^2}f(t+\frac{a^2}{t})\frac{dt}{t}\\
&=\frac{1}{2}\int_1^{a}f(t+\frac{a^2}{t})\frac{dt}{t}+
\frac{1}{2}\int_a^{a^2}f(t+\frac{a^2}{t})\frac{dt}{t}
\end{align*}
令\(t=\frac{a^2}{u}\)
\begin{align*}
\frac{1}{2}\int_a^{a^2}f(t+\frac{a^2}{t})\frac{dt}{t}&=
\int^1_af(\frac{a^2}{u}+u)\frac{u}{a^2}\left(-\frac{a^2}{u^2}\right)du\\
&=\int_1^af(u+\frac{a^2}{u})\frac{1}{u}du
\end{align*}
\end{examplle}

\begin{examplle}[]
设\(f(x)\)在\([a,b]\)上有二阶连续导数,又\(f(a)=f'(a)=0\),证明:
\begin{equation*}
\int_a^bf(x)dx=\frac{1}{2}\int_a^bf''(x)(x-b)^2dx
\end{equation*}

利用分部积分
\begin{align*}
\int_a^bf(x)dx&=\int_a^b f(x)d(x-b)=-\int_a^bf'(x)(x-b)d(x-b)\\
&=-\frac{1}{2}\int_a^bf'(x)d(x-b)^2=\frac{1}{2}\int_a^bf''(x)(x-b)^2dx
\end{align*}
\end{examplle}

\begin{examplle}[]
设\(f(x)\)在\([a,b]\)上有二阶连续导数且
\(f(a)=f(b)=0\),\(M=\displaystyle\max_{[a,b]}\abs{f''(x)}\),证明
\(\displaystyle\abs{\int^b_af(x)dx}\le\frac{(b-a)^2}{12}M\)

\begin{align*}
\int_a^bf(x)dx&=\int_a^bf(x)d(x-a)=-\int_a^bf'(x)(x-a)d(x-b)\\
&=\int_a^bf''(x)(x-a)(x-b)dx+\int_a^bf'(x)(x-b)dx\\
&=\int_a^bf''(x)(x-a)(x-b)dx+\int_a^b(x-b)df(x)\\
&=\int_a^bf''(x)(x-a)(x-b)dx-\int_a^bf(x)dx
\end{align*}
则
\begin{equation*}
\int_a^bf(x)dx=\frac{1}{2}\int_a^bf''(x)(x-a)(x-b)dx
\end{equation*}
因此
\begin{align*}
\abs{\int_a^bf(x)dx}&\le\frac{1}{2}M\int_a^b(x-a)(b-a)dx\\
&=\frac{1}{4}M\int_a^b(x-a)^2dx=\frac{(b-a)^3}{12}M
\end{align*}
\end{examplle}

\begin{examplle}[]
设\(f(x)\)在\([a,b]\)上连续且严格单调增,证明:
\begin{equation*}
(a+b)\int_a^bf(x)dx<2\int_a^bxf(x)dx
\end{equation*}

令\(F(x)=(a+x)\int^x_af(t)dt-2\int_a^xtf(t)dt,(a<x\le b)\)
\end{examplle}

\begin{examplle}[]
求
\(\displaystyle\int_{\frac{1}{2}}^{\frac{3}{2}}\frac{1}{\sqrt{\abs{x-x^2}}}dx\)

\begin{align*}
\int_{\frac{1}{2}}^{\frac{3}{2}}\frac{1}{\sqrt{\abs{x-x^2}}}dx&=
\int_{\frac{1}{2}}^1\frac{1}{\sqrt{x-x^2}}dx+
\int_{1}^{\frac{3}{2}}\frac{1}{\sqrt{x^2-x}}dx\\
&=\int_{\frac{1}{2}}^1\frac{1}{\sqrt{\frac{1}{4}-(x-\frac{1}{2})^2}}dx+
\int_{1}^{\frac{3}{2}}\frac{1}{\sqrt{(x-\frac{1}{2})^2-\frac{1}{4}}}dx\\
&=\arcsin(2x-1)\Big\rvert^1_{\frac{1}{2}}+\ln\left[
(x-\frac{1}{2})+\sqrt{(x-\frac{1}{2})-\frac{1}{4}}
\right]\Big\rvert^{\frac{3}{2}}_1
\end{align*}
\end{examplle}

\begin{examplle}[]
求\(\displaystyle\int e^x\frac{1+\sin x}{1+\cos x}dx\)

\begin{align*}
\int e^x\frac{1+\sin x}{1+\cos x}dx&=\int e^x(1+\sin x)\frac{1}{2\cos^2\frac{x}{2}}dx=
\int e^xd\tan\frac{x}{2}+\int e^x\tan\frac{x}{2}dx\\
&=e^x\tan\frac{x}{2}+C
\end{align*}
\end{examplle}

\begin{examplle}[]
设\(f(x)\)为非负连续函数,当\(x\ge0\)时,有\(\int_0^xf(x)f(x-t)dt=e^{2x}-1\),
求\(f(x)\)

\(f(x)\int)0^xf(u)du=e^{2x-1}\),令\(F(x)=\int_0^xf(t)dt\),则有
\(F'(x)F(x)=e^{2x-1},F(0)=0\),两边积分,得
\begin{equation*}
\frac{1}{2}F^2(x)=\frac{1}{2}e^{2x}-x+C
\end{equation*}
由\(F(0)=0\)得,\(C=-\frac{1}{2}\).因此\(F^2(x)=e^{2x}-x-1\),故
\begin{equation*}
f(x)=F'(x)=\frac{e^{2x}-1}{\sqrt{e^{2x}-2x-1}}
\end{equation*}
\end{examplle}

\begin{examplle}[]
设\(\displaystyle f(x)=\int_1^x\frac{\ln t}{1+t}dt(x>0)\),\(g(x)\)连续,且
\(f(x)+f(\frac{1}{x})=\int_0^1g(xt)dt\),求\(g(x)\)

\(\int_0^1g(xt)dt=\frac{1}{x}\int_0^xg(t)dt\),又
\begin{equation*}
f(\frac{1}{x})=\int_0^{\frac{1}{x}}\frac{\ln t}{1+t}dt=
\int_0^x\frac{\ln\frac{1}{u}}{1+\frac{1}{u}}(-\frac{1}{u^2})du=
\int_1^x\frac{\ln u}{u(1+u)}du
\end{equation*}
因此\(f(x)+f(\frac{1}{x})=\int_1^x\frac{\ln t}{t}dt\),于是
\(\int_0^xg(t)dt=x\int_1^x\frac{\ln t}{t}dt\),
\begin{equation*}
g(x)=\int_1^x\frac{\ln t}{t}dt+\ln x=\frac{1}{2}\ln^2x+\ln x
\end{equation*}
\end{examplle}

\begin{examplle}[]
设\(f(x)\)在\([0,+\infty)\)上连续且单调增加,证明:对任意\(a,b>0\),恒有
\begin{equation*}
\int_a^bxf(x)dx\ge\frac{1}{2}\left[
b\int_0^bf(x)dx-a\int_0^af(x)dx
\right]
\end{equation*}

令\(F(x)=x\int_0^xf(t)dt\),则\(F'(x)=\int_0^xf(t)dt+xf(x)\)
\begin{align*}
F(b)-F(a)&=\int_a^bF'(x)dx=\int_a^b
\left[\int_0^xf(t)dt+xf(x)
\right]dx\\
&\le\int_a^b[xf(x)+xf(x)]dx=2\int_a^bxf(x)dx
\end{align*}
\end{examplle}
\subsection{多元函数微积分学}
\label{sec:orga6c0685}
\begin{examplle}[]
求极限
\(\displaystyle\lim_{\substack{x\to0\\y\to0}}\frac{x^2y^2}{(x^2+y^2)^{\frac{3}{2}}}\)

\(x^2y^2\le(\frac{x^2+y^2}{2})^2\),因而
\begin{equation*}
0\le \frac{x^2y^2}{(x^2+y^2)^{\frac{3}{2}}}\le\frac{1}{4}\sqrt{x^2+y^2}
\end{equation*}
\end{examplle}

\begin{examplle}[]
讨论极限\(\displaystyle\lim_{\substack{x\to0\\y\to0}}\frac{xy^2}{x^2+y^4}\)的
存在性

当点\(P(x,y)\)沿曲线\(x=ky^2\)趋于点\((0,0)\)时
\begin{equation*}
\lim_{\substack{x\to0\\y\to0}}\frac{xy^2}{x^2+y^4}=
\lim_{y\to0}\frac{ky^4}{k^2y^4+y^4}=\frac{k}{k^2+1}
\end{equation*}
不是一个确定的常数,因此极限不存在
\end{examplle}

\begin{examplle}[]
讨论函数
\begin{equation*}
f(x,y)=
\begin{cases}
\frac{xy(x^2-y^2)}{x^2+y^2}&(x,y)\neq(0,0)\\
0&(x,y)=(0,0)
\end{cases}
\end{equation*}在\((0,0)\)处的连续性

令\(x=r\cos\theta,y=r\sin\theta\),则
\begin{equation*}
0\le\abs{f(x,y)}=\abs{\frac{r^2\sin 4\theta}{4}}\le\frac{r^2}{4}
\end{equation*}
因此连续
\end{examplle}

\begin{examplle}[]
设\(z=(s\in y^3+x^3)(x+y^4)^{\frac{y}{x}+e^{y^3x^2}}\),求\(\frac{\partial z}{\partial
   x}\Big\rvert_{(1,0)}\)

\begin{equation*}
\frac{\partial z}{\partial x}\Big\rvert_{(1,0)}=\frac{\partial z(x,0)}{\partial x}\Big\rvert_{x=1}=(x^4)'
\Big\rvert_{x=1}=4
\end{equation*}
\end{examplle}

\begin{examplle}[]
已知函数\(f(x,y)\)在点\((0,0)\)的某邻域内有定义,且\(f(0,0)=0\),
\(\displaystyle\lim_{\substack{x\to0\\y\to0}}\frac{f(x,y)}{x^2+y^2}=1\),则
\(f(x,y)\)在点\((0,0)\)处

由于\(\displaystyle\lim_{\substack{x\to0\\y\to0}}\frac{f(x,y)}{x^2+y^2}=1\),
\(\lim_{\substack{x\to0\\y\to0}}(x^2+y^2)=0\),于是
\(\lim_{\substack{x\to0\\y\to0}}f(x,y)=0\),又\(f(0,0)=0\),所以\(f(x,y)\)在
\((0,0)\)处极限存在且连续,又由
\(\lim_{\substack{x\to0\\y\to0}}\frac{f(x,y)}{x^2+y^2}=1\),得
\begin{equation*}
\lim_{x\to0}\frac{f(x,0)}{x^2}=1,
\lim_{y\to0}\frac{f(0,y)}{y^2}=1
\end{equation*}
所以
\begin{equation*}
f_x'(0,0)=\lim_{x\to0}\frac{f(x,0)-f(0,0)}{x}=
\lim_{x\to0}\frac{f(x,0)}{x}=\lim_{x\to0}\frac{f(x,0)}{x^2}x=0
\end{equation*}
同理\(f'_y(0,0)=0\),故\(f(x,y)\)在\((0,0)\)处偏导数存在

因为
\begin{align*}
\lim_{\rho\to0}\frac{\Delta z-[f_x'(0,0)\Delta x+f_y'(0,0)\Delta y]}{\rho}&=
\lim_{\substack{x\to0\\y\to0}}\frac{f(x,y)-f(0,0)}{\sqrt{x^2+y^2}}(\rho=\sqrt{x^2+y^2})\\
&=\lim_{\substack{x\to0\\y\to0}}\frac{f(x,y)}{x^2+y^2}\sqrt{x^2+y^2}=0
\end{align*}
所以\(f(x,y)\)在\((0,0)\)处可微
\end{examplle}

\begin{remark}
讨论二元函数\(f(x,y)\)在\((x_0,y_0)\)的可微性,可从如下几个方面考虑
\begin{enumerate}
\item 若二元函数\(f(x,y)\)在\((x_0,y_0)\)的偏导数至少有一个不存在,则函数不可微
\item 若二元函数\(f(x,y)\)在\((x_0,y_0)\)不连续,则函数不可微
\item 若二元函数\(f(x,y)\)在\((x_0,y_0)\)连续,两个偏导数存在,则考虑
\begin{equation*}
\lim_{\rho\to0}\frac{\Delta z-[f_x'(x_0,y_0)\Delta x+f_y'(x_0,y_0)\Delta y]}{\rho},
\rho=\sqrt{(\Delta x)^2+(\Delta y)^2}
\end{equation*}
若极限为 0,则函数在\((x_0,y_0)\)可微,否则不可微
\end{enumerate}
\end{remark}

\begin{examplle}[]
设\(z=(\frac{y}{2})^{\frac{x}{y}}\),求\(dz\Big\rvert_{(1,2)}\)

取对数,有
\begin{equation*}
\ln z=\frac{x}{y}\ln\frac{y}{x}\Rightarrow
y\ln z=x(\ln y-\ln x)
\end{equation*}
\end{examplle}


\begin{examplle}[]
设\(u=f(\frac{x}{y},\frac{y}{z}),u=f(s,t)\)有二阶连续偏导数,求
\(du,\frac{\partial^2 u}{\partial y\partial z}\)

\begin{align*}
du&=f_1'd(\frac{x}{y})+f_2'd(\frac{y}{z})=f_1'\frac{ydx-xdy}{y^2}
+f_2'\frac{zdy-ydz}{z^2}\\
&=\frac{1}{y}f_1'dx+(-\frac{x}{y^2}f_1'+\frac{1}{z}f_2')dy-\frac{y}{z^2}f_2'dz
\end{align*}
\end{examplle}

\begin{examplle}[]
已知\((axy^3-y^2\cos x)dx+(1+by\sin x+3x^2y^2)dy\)为某一函数\(f(x,y)\)的全微
分,求\(a,b\)

由题意知,\(\frac{\partial f}{\partial x}=axy^3-y^2\cos x,\frac{\partial f}{\partial y}=1+by\sin
   x+3x^2y^2\),从而有
\(\frac{\partial ^2f}{\partial x\partial y}=3axy^2-2y\cos x,\frac{\partial^2f}{\partial y\partial
   x}=by\cos x+6xy^2\),显然\(\frac{\partial^2f}{\partial x\partial y},\frac{\partial^2f}{\partial y\partial
   x}\)均连续,所以\(\frac{\partial^2f}{\partial x\partial y}=\frac{\partial^2f}{\partial y\partial x}\),即
\(by\cos x+6xy^2=3axy^2-2y\cos x\),因此\(a=2,b=-2\)
\end{examplle}

\begin{examplle}[]
设\(z=f(x,y)\)满足\(\frac{\partial^2f}{\partial y^2}=2x,f(x,1)=0,\frac{\partial f(x,0)}{\partial y}=\sin
   x\),求\(f(x,y)\)

\(f(x,y)=xy^2+\varphi(x)y+\psi(x)\),从
\(\frac{\partial f(x,0)}{\partial y}=\sin x\),即\([2xy+\varphi(x)]\Big\rvert_{y=0}=\sin
   x\),得\(\varphi(x)=\sin x\)
\end{examplle}

\begin{examplle}[]
设函数\(u=f(\ln\sqrt{x^2+y^2})\),满足\(\frac{\partial^2u}{\partial x^2}+\frac{\partial^2u}{\partial
   y^2}=(x^2+y^2)^{3/2}\),求函数\(f\)的表达式

设\(t=\ln\sqrt{x^2+y^2}\),则\(x^2+y^2=e^{2t}\)
\begin{gather*}
\frac{\partial u}{\partial x}=f'(t)\frac{x}{x^2+y^2},\frac{\partial u}{\partial y}=f'(t)\frac{y}{x^2+y^2}\\
\frac{\partial^2 u}{\partial x^2}=f''(t)\frac{x^2}{(x^2+y^2)^2}+f'(t)\frac{y^2-x^2}{(x^2+y^2)^2}\\
\frac{\partial^2 u}{\partial y^2}=f''(t)\frac{y^2}{(x^2+y^2)^2}+f'(t)\frac{x^2-y^2}{(x^2+y^2)^2}\\
\end{gather*}
代入得\(f''(t)=(x^2+y^2)^{5/2}=e^{5t}\),因此有
\begin{equation*}
f(t)=\frac{1}{25}e^{5t}+C_1t+C_2
\end{equation*}
\end{examplle}

\begin{examplle}[]
已知函数\(f(x,y)\)在点\((0,0)\)的某个邻域内连续,且
\(\displaystyle\lim_{\substack{x\to0\\y\to0}}\frac{f(x,y)-xy}{(x^2+y^2)^2}=1\)
,则
\begin{enumerate}
\item 点\((0,0)\)不是\(f(x,y)\)的极值点
\item 点\((0,0)\)是\(f(x,y)\)的极大值点
\item 点\((0,0)\)是\(f(x,y)\)的极小值点
\item 根据所给条件无法判断点\((0,0)\)是否为\(f(x,y)\)的极值点
\end{enumerate}


分子的极限为 0,从而有\(f(0,0)=0\),且由极限的性质知,
\(\frac{f(x,y)-xy}{(x^2+y^2)^2}=1+\alpha(x,y)\),这里
\(\displaystyle\lim_{\substack{x\to0\\y\to0}}\alpha(x,y)=0\),因而
\(f(x,y)=xy+(x^2+y^2)^2[1+\alpha(x,y)]\),在点\((0,0)\)的某充分小去心邻域内,
取\(y=x\)且\(\abs{x}\)充分小时,\(f(x,y)=x^2+4x^4[1+\alpha(x,x)]>0=f(0,0)\),
在点\((0,0)\)的某充分小去心邻域内,取\(y=-x\)且\(\abs{x}\)充分小时,
\(f(x,y)=-x^2+4x^4[1+\alpha(x,-x)]<0=f(0,0)\),故点\((0,0)\)不是\(f(x,y)\)的
极值点
\end{examplle}

\begin{examplle}[]
讨论二元函数\(z=x^3+y^3-2(x^2+y^2)\)的极值

\begin{equation*}
\begin{cases}
\frac{\partial z}{\partial x}=3x^2-4x=0\\
\frac{\partial z}{\partial y}=3y^2-4y=0\\
\end{cases}
\end{equation*}
得驻点\((0,0),(4/3,0),(0,4/3),(4/3,4/3)\).进而
\begin{gather*}
A=\frac{\partial^2z}{\partial x^2}=6x-4,B=\frac{\partial^2z}{\partial xy}=0,C=\frac{\partial^2z}{\partial y^2}=6y-4\\
AC-B^2=16+36xy-24(x+y)
\end{gather*}
在点\((0,0)\)时\(AC-B^2>0\)且\(A<0\)有极大值

在点\((4/3,4/3)\)时\(AC-B^2>0\)且\(A>0\)有极小值
\end{examplle}

\begin{examplle}[]
求椭圆\(x^2+2xy+3y^2-8y=0\)与直线\(x+y=8\)之间的最短距离

椭圆上任意一点\(P(x,y)\)到直线\(x+y=8\)的距离的平方为
\begin{equation*}
d^2=\frac{(x+y-8)^2}{2}
\end{equation*}
令\(\displaystyle F(x,y)=\frac{1}{2}(x+y-8)^2+\lambda(x^2+2xy+3y^2-8y)\)
则有方程组
\begin{equation*}
\begin{cases}
&F_x'=x+y-8+(2\lambda x+2\lambda y)=0\\
&F_y' =x+y-8+\lambda(2x+6y-8)=0\\
&x^2+2xy+3y^2-8y = 0
\end{cases}
\end{equation*}
解得
\begin{equation*}
\begin{cases}
x=-2+2\sqrt{2}\\
y=2
\end{cases}\quad\text{ or }\quad
\begin{cases}
x=-2-2\sqrt{2}\\
y=2
\end{cases}
\end{equation*}
且\(d_1=4\sqrt{2}-2,d_2=4\sqrt{2}+2\),所以所求最短距离为\(4\sqrt{2}-2\)
\end{examplle}

\begin{examplle}[]
求函数\(f(x,y)=x^2+2y^2-x^2y^2\)在区域\(D=\{(x,y)\mid x^2+y^2\le4,y\ge0\}\)上
的最大值和最小值

解方程组
\begin{equation*}
\begin{cases}
f_x'=2x-2xy^2=0\\
f_y' =4y-2x^2y=0
\end{cases}
\end{equation*}
得开区域内的可能极值点为\((\pm\sqrt{2},1)\),其对应函数值为
\(f(\pm\sqrt{2},1)=2\)

当\(y=0\)时,\(f(x,y)=x^2\)在\(-2\le x\le 2\)上的最大值为 4,最小值为 0

当\(x^2+y^2=4,y>0,-2<x<2\)时,构造拉格朗日函数
\begin{equation*}
F(x,y,\lambda)=x^2+2y^2-x^2y^2+\lambda(x^2+y^2-4)
\end{equation*}
解方程组
\begin{equation*}
\begin{cases}
F_x'=2x-2xy^2+2\lambda x=0\\
F_y' =4y-2x^2y+2\lambda y=0\\
F_\lambda' =x^2+y^2-4=0
\end{cases}
\end{equation*}
得可能极值点:
\(\displaystyle(0,2),\left(\pm\sqrt{\frac{5}{2}},\sqrt{\frac{3}{2}}\right)\),
其对应函数值为
\(f(0,2)=8,f\left(\pm\sqrt{\frac{5}{2}},\sqrt{\frac{3}{2}}\right)=\frac{7}{4}\)

因此\(f(x,y)\)在\(D\)上的最大值为 8,最小值 0
\end{examplle}

\begin{examplle}[]
设\(f(x,y)\)有二阶连续偏导数,\(g(x,y)=f(e^{xy},x^2+y^2)\),且
\begin{equation*}
f(x,y)=1-x-y+o(\sqrt{(x-1)^2+y^2})
\end{equation*}
证明\(g(x,y)\)在\((0,0)\)取得极值,判断此极值是极大值还是极小值,并求出此极值

由\(f(x,y)=-(x-1)-y+o(\sqrt{(x-1)^2+y^2})\),由全微分的定义得
\begin{gather*}
f(1,0)=0,f_x'(1,0)=f_y'(1,0)=-1
\end{gather*}
计算得\(g_x'=f_1'\cdot e^{xy}y+f_2'\cdot 2x,g_y'=f_1'\cdot e^{xy}x+f_2'\cdot
   2y\),有
\begin{gather*}
g_x'(0,0)=0,g_y'(0,0)=0
\end{gather*}
再求二阶导数
\begin{align*}
&g''_{xx}=(f_{11}''\cdot e^{xy}y+f_{12}''\cdot 2x)e^{xy}y+f_1'\cdot e^{xy}y^2+
(f_{21}''\cdot e^{xy}y+f_{22}''\cdot 2x)2x+2f_2'\\
&g_{xy}''=(f_{11}''\cdot e^{xy}x+f_{12}''\cdot 2y)e^{xy}y+f_1'\cdot(e^{xy}xy+e^{xy})+
(f_{21}''\cdot e^{xy}x+f_{22}''\cdot 2y)2x\\
&g''_{yy}=(f_{11}''\cdot e^{xy}x+f_{12}''\cdot 2y)e^{xy}x+f_1'\cdot e^{xy}x^2+
(f_{21}''\cdot e^{xy}x+f_{22}''\cdot 2y)2x+2f_2'
\end{align*}
因此
\(A=g_{xx}''(0,0)=2f_2'(1,0)=-2,B=g_{xy}''(0,0)=f_1'(1,0)=-1\),\(C=g_{yy}''(0,0)=2f_2'(1,0)=-2\)
,进而\(AC-B^2>0\),因此\(g(0,0)=f(1,0)=0\)是极大值
\end{examplle}

\begin{examplle}[]
已知\(x,y,z\)为实数,且\(e^x+y^2+\abs{z}=3\),求证\(e^xy^2\abs{z}\le1\)

\emph{证明 1} 。在\(e^x+y^2+\abs{z}=3\)约束条件下求函数\(u=e^xy^2\abs{z}\)的最值问题,
转化为无条件极值\(u=e^xy^2(3-e^x-y^2)\)

\emph{证明 2} 。可化为以下等价问题:已知\(X>0,Y\ge0,Z\ge0\),且\(X+Y+Z=3\),求
\(XYZ\le1\)。因此用拉格朗日乘数法
\end{examplle}

\begin{examplle}[]
设闭区域\(D:x^2+y^2\le y,x\ge0\),\(f(x,y)\)为\(D\)上的连续函数,且
\begin{equation*}
f(x,y)=\sqrt{1-x^2-y^2}-\frac{8}{\pi}\iint_Df(u,v)dudv
\end{equation*}
求\(f(u,v)\)

设\(\iint_Df(u,v)dudv=A\),在已知等式两边求区域\(D\)的二重积分
\begin{gather*}
\iint_Df(x,y)dxdy=\iint_D\sqrt{1-x^2-y^2}dxdy-\frac{8A}{\pi}\iint_Ddxdy\\
A=\iint_D\sqrt{1-x^2-y^2}dxdy-A\\
2A=\int^{\frac{\pi}{2}}_0d\theta\int_0^{\sin\theta}\sqrt{1-r^2}rdr=\frac{1}{3}(\frac{\pi}{2}-\frac{2}{3})
\end{gather*}
\end{examplle}

\begin{examplle}[]
设区域\(D=\{(x,y)\mid x^2+y^2\le4,x\ge0,y\ge0\}\),\(f(x)\)为\(D\)上的正值连续
函数,\(a,b\)为常数,求
\(\displaystyle\iint_D\frac{a\sqrt{f(x)}+b\sqrt{f(y)}}{\sqrt{f(x)}+\sqrt{f(y)}}d\sigma\)

由轮换对称性
\begin{align*}
&\iint_D\frac{a\sqrt{f(x)}+b\sqrt{f(y)}}{\sqrt{f(y)}+\sqrt{f(x)}}d\sigma=
\iint_D\frac{a\sqrt{f(y)}+b\sqrt{f(x)}}{\sqrt{f(x)}+\sqrt{f(y)}}d\sigma\\
&=\frac{1}{2}\iint_D\left[
\frac{a\sqrt{f(x)}+b\sqrt{f(y)}}{\sqrt{f(x)}+\sqrt{f(y)}}+
\frac{a\sqrt{f(y)}+b\sqrt{f(x)}}{\sqrt{f(y)}+\sqrt{f(x)}}
\right]\\
&=\frac{a+b}{2}\iint_Dd\sigma=\frac{a+b}{2}\pi
\end{align*}
\end{examplle}

\begin{examplle}[]
计算二重积分\(I=\iint_Dx[1+yf(x^2+y^2)]dxdy\),其中积分区域\(D\)为
\(y=x^3,y=1,x=-1\)所围成的平面区域,\(f\)连续

补充曲线\(y=-x^3\),拆分积分区域\(D\)分别关于\(x,y\)坐标轴对称
\begin{center}
\begin{tikzpicture}[line cap=round,line join=round,>=triangle 45,x=1cm,y=1cm]
\begin{axis}[
x=1cm,y=1cm,
axis lines=middle,
xmin=-1.5,
xmax=1.5,
ymin=-1.5,
ymax=1.5,
xtick={-1,0,1},
ytick={-1,0,1},]
\draw [line width=0.2pt] (-1,-1.5) -- (-1,1.5);
\draw [line width=0.2pt,domain=-1.5:1.5] plot(\x,{(--1-0*\x)/1});
\draw[line width=0.2pt,smooth,samples=100,domain=-1.5:1.5] plot(\x,{(\x)^(3)});
\draw[line width=0.2pt,color=ccqqqq,smooth,samples=100,domain=-1.5:1.5] plot(\x,{0-(\x)^(3)});
\draw (-0.6,0.7360515021459239) node[anchor=north west] {$D_2$};
\draw (-1.1,0.36266094420600964) node[anchor=north west] {$D_1$};
\end{axis}
\end{tikzpicture}
\end{center}
\begin{align*}
I&=\iint_D=\iint_{D_1}+\iint_{D_2}\\
&=\iint_{D_1}[x+xyf(x^2+y^2)]dxdy+\iint_{D_2}x[1+yf(x^2+y^2)]dxdy\\
&=\iint_{D_1}xdxdy=2\int^0_{-1}dx\int^{-x^3}_0xdy\\
&=-\frac{2}{5}
\end{align*}
\end{examplle}

\begin{examplle}[]
设平面区域\(D=\{(x,y)\mid1\le x^2+y^2\le4,x\ge0,y\ge0\}\),计算
\begin{equation*}
\iint_D\frac{x\sin(\pi\sqrt{x^2+y^2})}{x+y}dxdy
\end{equation*}

由轮换对称性
\begin{align*}
&\iint_D\frac{x\sin(\pi\sqrt{x^2+y^2})}{x+y}dxdy\\
&=\frac{1}{2}\left[
\iint_D\frac{x\sin(\pi\sqrt{x^2+y^2})}{x+y}dxdy+
\iint_D\frac{y\sin(\pi\sqrt{x^2+y^2})}{x+y}dxdy
\right]\\
&=\frac{1}{2}\iint_D\sin(\pi\sqrt{x^2+y^2})dxdy=
\frac{1}{2}\int_0^{\frac{\pi}{2}}d\theta\int_1^2r\sin\pi rdr\\
&=-\frac{3}{4}
\end{align*}
\end{examplle}


\begin{examplle}[]
计算二重积分\(\iint_D\abs{x^2+y^2-1}d\sigma\),其中\(D=\{(x,y)\mid0\le
   x\le1,0\le y\le1\}\)

记\(D_1=\{(x,y)\mid x^2+y^2\le1,(x,y)\in D\},D_2=\{(x,y)\mid
   x^2+y^2>1,(x,y)\in D\}\),则

\begin{align*}
\iint_D\abs{x^2+y^2-1}d\sigma&=
-\iint_{D_1}(x^2+y^2-1)dxdy+\iint_{D_2}(x^2+y^2-1)dxdy\\
&=-2\iint_{D_1}(x^2+y^2-1)dxdy+\iint_D(x^2+y^2-1)dxdy
\end{align*}
\end{examplle}

\begin{examplle}[]
设\(f(x)\)为连续函数,\(F(t)=\int_1^tdy\int_y^tf(x)dx\),求\(F'(2)\)

交换积分次序得
\begin{equation*}
F(t)=\int_1^tdy\int_y^tf(x)dx=\int_1^t[\int_1^xf(x)dy]dx=\int_1^tf(x)(x-1)dx
\end{equation*}
因此\(F'(2)=f(2)(x-1)=f(2)\)
\end{examplle}

\begin{examplle}[]
计算二重积分\(\iint_Dr^2\sin\theta\sqrt{1-r^2\cos2\theta}drd\theta\),其中
\(D=\{(r,\theta)\mid 0\le r\le\sec\theta,0\le\theta\le\frac{\pi}{4}\}\)

直角坐标系下\(D=\{(x,y)\mid0\le x\le1,0\le y\le x\}\)
\end{examplle}

\begin{examplle}[]
已知函数\(f(x,y)\)具有二阶连续偏导数,且\(f(1,y)=0,f(x,1)=0\),
\(\iint_Df(x,y)dxdy=a\),其中\(D=\{(x,y)\mid0\le x\le1,0\le y\le1\}\),计算二
重积分
\begin{equation*}
\iint_Dxyf''_{xy}(x,y)dxdy
\end{equation*}

\begin{gather*}
\iint_Dxyf''_{xy}(x,y)dxdy=\int_0^1x(\int_0^1yf''_{xy}(x,y)dy)dx=\int_0^1x
(\int_0^1ydf_x'(x,y))dx\\
\int_0^1ydf_x'(x,y)=yf_x'(x,y)\Big\rvert_0^1-\int_0^1f'_x(x,y)dy=
-\int_0^1f_x'(x,y)dy\\
\int_0^1x(\int_0^1ydf_x'(x,y))dx=-\int_0^1x(\int_0^1f_x'(x,y)dy)dx=
-\int_0^1(\int_0^1xf_x'(x,y)dx)dy\\
\int_0^1xf_x'(x,y)dx=\int_0^1xdf(x,y)=xf(x,y)\Big\rvert_0^1-\int_0^1f(x,y)dx=
-\int_0^1f(x,y)dx\\
\iint_Dxyf''_{xy}dxdy=\int_0^1dy\int_0^1f(x,y)dx=a
\end{gather*}
\end{examplle}

\begin{examplle}[]
求积分\(\int_0^1dy\int_y^1\left(\frac{e^{x^2}}{x}-e^{y^2}\right)dx\)

\begin{align*}
\int_0^1dy\int_y^1\left(\frac{e^{x^2}}{x}-e^{y^2}\right)dx&=
\int_0^1dxf_0^x\left(\frac{e^{x^2}}{x}-e^{y^2}\right)dy=
\int_0^1(e^{x^2}-\int_0^xe^{y^2}dy)dx\\
&=\int_0^1e^{x^2}dx-\int_0^1(\int_0^xe^{y^2}dy)dx\\&=
\int_0^1e^{x^2}dx-x\int_0^xe^{y^2}dy\Big\rvert_0^1+
\int_0^1e^{x^2}\cdot xdx\\
&=\int_0^1e^{x^2}dx-\int_0^1e^{y^2}dy+\frac{1}{2}e^{x^2}\Big\rvert_0^1\\
&=\frac{e-1}{2}
\end{align*}
\end{examplle}

\begin{examplle}[]
设\(f(x)\)在\([a,b]\)上连续,且\(f(x)>0\),证明
\begin{equation*}
\int_a^bf(x)dx\int_a^b\frac{1}{f(x)}dx\ge(b-a)^2
\end{equation*}

令\(D=\{(x,y)\mid a\le x,y\le b\}\)
\begin{align*}
\int_a^bf(x)dx\int_a^b\frac{1}{f(x)}dx\ge(b-a)^2&=
\int_a^bf(x)dx\int_a^b\frac{1}{f(y)}dy\\
&=\iint_D\frac{f(x)}{f(y)}dxdy=\iint_D\frac{f(y)}{f(x)}dxdy\\
&=\frac{1}{2}\iint_D\left[\frac{f(x)}{f(y)}+\frac{f(y)}{f(x)}\right]dxdy\\
&\ge\iint_Ddxdy=(b-a)^2
\end{align*}
\end{examplle}

\begin{examplle}[]
证明
\(\left(\int_0^1e^{-x^2}dx\right)^2>\frac{\pi}{4}(1-\frac{1}{e})\)

令\(D=\{(x,y)\mid0\le x,y\le1\},D_1=\{(x,y)\mid x^2+y^2\le1,x\ge0,y\ge0\}\)
\begin{align*}
\left(\int_0^1e^{-x^2}dx\right)^2&=
\int_0^1e^{-x^2}dx\int_0^1e^{-y^2}dy=\iint_De^{-x^2-y^2}dxdy\\
&>\iint_De^{-(x^2+y^2)}dxdy=\int_0^{\frac{\pi}{2}}d\theta\int_0^1e^{-r^2}rdr=\frac{\pi}{4}(1-\frac{1}{e})
\end{align*}
\end{examplle}

\begin{examplle}[]
设函数\(z=f(x,y)\)具有二阶连续偏导数,且满足\(f_{xx}''=f_{yy}''\),又由
\(f(x,2x)=x\),\(f_x'(x,2x)=x^2\),试求二阶偏导数
\(f_{xx}''(x,2x),f_{xy}''(x,2x)\)

因为\(f_x'\cdot1+f_y'\cdot2=1\)所以\(2f_y'=1-x^2\)。又因为
\(2(f_{yx}''\cdot1+f_{yy}''\cdot2) =-2x\),由条件知\(f_x'(x,2x)=x^2\),则
\(f_{xx}''\cdot1+f_{xy}''\cdot2=2x\),解得
\(f_{xx}''(x,2x)=-\frac{4}{3}x,f_{xy}''(x,2x)=\frac{5}{3}x\)
\end{examplle}

\begin{examplle}[]
设函数\(u=u(x,y)\)由方程\(u=f(x,y,z,t),g(y,z,t)=0,h(z,t)=0\)所确定,求
\(\frac{\partial u}{\partial x},\frac{\partial u}{\partial y}\)

由方程组
\begin{equation*}
\begin{cases}
f(x,y,z,t)-u=0\\
g(y,z,t)=0\\
h(z,t)=0
\end{cases}
\end{equation*}
有
\begin{equation*}
\begin{cases}
\frac{\partial f}{\partial y}+\frac{\partial f}{\partial z}\frac{\partial z}{\partial y}+\frac{\partial f}{\partial t}\frac{\partial t}{\partial y}-\frac{\partial u}{\partial y}=0\\
\frac{\partial g}{\partial y}+\frac{\partial g}{\partial z}\frac{\partial z}{\partial y}+\frac{\partial g}{\partial t}\frac{\partial t}{\partial y}=0\\
\frac{\partial h}{\partial z}\frac{\partial z}{\partial y}+\frac{\partial h}{\partial t}\frac{\partial t}{\partial y}=0
\end{cases}
\end{equation*}
解
\end{examplle}
\subsection{无穷级数}
\label{sec:org387ccff}
\begin{examplle}[]
判断级数
\(\displaystyle\sum_{n=1}^\infty\int_0^{\frac{1}{n}}\frac{\sqrt{x}}{1+x^2}dx\)
的敛散性

由于
\begin{equation*}
0<\sum_{n=1}^\infty\int_0^{\frac{1}{n}}\frac{\sqrt{x}}{1+x^2}dx<
\int_0^{\frac{1}{n}}\sqrt{x}dx=\frac{2}{3}\frac{1}{n^{3/2}}
\end{equation*}
而级数\(\displaystyle\sum_{n=1}^\infty\frac{1}{n^{3/2}}\)收敛,因此级数收敛
\end{examplle}

\begin{examplle}[]
判断级数
\(\displaystyle\sum_{n=1}^\infty\left(\frac{1}{n}-\ln\frac{n+1}{n}\right)\)的
收敛性

由泰勒公式
\(\ln\frac{n+1}{n}=\frac{1}{n}-\frac{1}{2}\frac{1}{n^2}+o(\frac{1}{n^2})\),
则 \(\frac{1}{n}-\ln\frac{n+1}{n}\sim\frac{1}{2}\cdot\frac{1}{n^2}\),而
\(\displaystyle\sum_{n=1}^n\frac{1}{n^2}\)收敛
\end{examplle}

\begin{examplle}[]
判定级数\(\displaystyle\sum_{n=1}^\infty\sin(n\pi+\frac{1}{n-\ln n})\)的敛散
性

\(\displaystyle\sum_{n=1}^\infty\sin(n\pi+\frac{1}{n-\ln
   n})=\sum_{n=1}^\infty(-1)^n\sin\frac{1}{n-\ln n}\),令\(f(x)=\frac{1}{x-\ln
   x}\),则\(f'(x)<0\)且\(\lim_{n\to\infty}u_n=0\)
\end{examplle}

\begin{examplle}[]
判定级数\(\displaystyle\sum_{n=1}^\infty\frac{(-1)^n}{\sqrt{n}+(-1)^n}\)的敛
散性

\begin{equation*}
\frac{(-1)^n}{\sqrt{n}+(-1)^n}=
\frac{(-1)^n[\sqrt{n}-(-1)^n]}{n-1}=
\frac{(-1)^n\sqrt{n}}{n-1}-\frac{1}{n-1}
\end{equation*}
\end{examplle}

\begin{examplle}[]
设\(a_1=2,a_{n+1}=\frac{1}{n}\left(a_n+\frac{1}{a_n}),n=1,2,\dots\),证明
\begin{enumerate}
\item \(\lim a_n\)存在
\item 级数\(\displaystyle\sum_{n=1}^\infty(\frac{a_n}{a_{n+1}}-1)\)收敛
\end{enumerate}


\begin{enumerate}
\item 显然\(a_n\ge0\),\(\{a_n\}\)单调减少且有下界,因此\(\lim a_n\)存在
\item 由于数列单调减少,所以有
\(0\le\frac{a_n}{a_{n+1}}-1=\frac{a_n-a_{n+1}}{a_{n+1}}\le a_n-a_{n+1}\),
而\(\displaystyle\sum_{n=1}^\infty(a_n-a_{n+1})=a_1-\lim_{n\to\infty}a_n\)
收敛
\end{enumerate}
\end{examplle}

\begin{examplle}[]
设正项数列\(\{a_n\}\)单调减少,且\(\displaystyle\sum_{n=1}^\infty(-1)^na_n\)
发散,试问级数
\(\displaystyle\sum_{n=1}^\infty\left(\frac{1}{a_n+1}\right)^n\)是否收敛
\end{examplle}

\begin{proof}
由已知正项数列\(\{a_n\}\)单调减少,根据单调有界数列必有极限知,极限
\(\lim_{n\to\infty}a_n\)存在,记\(a=\lim_{n\to\infty}a_n\),则有\(a_n\ge
   a\ge0\),若\(a=0\),则交错级数收敛,矛盾,因此\(a>0\)

又由于\(\left(\frac{1}{a_n+1}\right)^n\le\left(\frac{1}{a+1}\right)^n\),而
\(\frac{1}{a+1}<1\),几何级数\(\sum_{n=1}^\infty(\frac{1}{a+1})^n\)收敛,因此
级数收敛
\end{proof}
\section{线性代数}
\label{sec:org65aa4cf}
\subsection{行列式}
\label{sec:orge519f5d}
\begin{examplle}[]
计算行列式
\begin{equation*}
D_n=
\begin{vmatrix}
a_1&-1&0&\cdots&0&0\\
a_2&x&-1&\cdots&0&0\\
a_3&0&x&\cdots&0&0\\
\vdots&\vdots&\vdots&&\vdots&\vdots\\
a_{n-1}&0&0&\cdots&x&-1\\
a_n&0&0&\cdots&0&x
\end{vmatrix}
\end{equation*}

\begin{align*}
D_n&=
\begin{vmatrix}
a_1&-1&0&\cdots&0&0\\
a_2+a_1x&0&-1&\cdots&0&0\\
a_3&0&x&\cdots&0&0\\
\vdots&\vdots&\vdots&&\vdots&\vdots\\
a_{n-1}&0&0&\cdots&x&-1\\
a_n&0&0&\cdots&0&x
\end{vmatrix}\\
&=\cdots=
\begin{vmatrix}
a_1&-1&0&\cdots&0&0\\
a_2+a_1x&0&-1&\cdots&0&0\\
a_3+a_2x+a_1x^2&0&0&\cdots&0&0\\
\vdots&\vdots&\vdots&&\vdots&\vdots\\
a_{n-1}+\dots+a_1x^{n-2}&0&0&\cdots&0&-1\\
a_n+\dots+a_1x^{n-1}&0&0&\cdots&0&0
\end{vmatrix}\\
&=a_1x^{n-1}+\dots+a_n
\end{align*}
\end{examplle}

\begin{examplle}[]
计算
\begin{equation*}
\begin{vmatrix}
a&b&c&d\\x&0&0&y\\
y&0&0&x\\d&c&b&a
\end{vmatrix}
\end{equation*}

\begin{equation*}
\begin{vmatrix}
a&b&c&d\\x&0&0&y\\
y&0&0&x\\d&c&b&a
\end{vmatrix}=-
\begin{vmatrix}
y&0&0&x\\x&0&0&y\\a&b&c&d\\d&c&b&a
\end{vmatrix}=
\begin{vmatrix}
y&x&0&0\\x&y&0&0\\
a&d&c&b\\d&a&b&c
\end{vmatrix}=(x^2-y^2)(b^2-c^2)
\end{equation*}
\end{examplle}

\begin{examplle}[]
已知\(\bA,\bB\)均为\(n\)阶矩阵,若
\(\abs{\bA}=3,\abs{\bB}=2,\abs{\bA^{-1}+\bB}=2\),求
\(\abs{\bA+\bB^{-1}}\)

\begin{align*}
\abs{\bA+\bB^{-1}}&=\abs{\bE\bA+\bB^{-1}\bE}=\abs{(\bB^{-1}\bB)\bA+\bB^{-1}(\bA^{-1}\bA)}\\
&=\abs{\bB^{-1}(\bB+\bA^{-1})\bA}=3
\end{align*}
\end{examplle}

\begin{examplle}[]
已知 4 阶矩阵\(\bA\)相似于\(\bB,\bA\)的特征值为 2,3,4,5,\(\bE\)为 4 阶单位矩阵,
求\(\abs{\bB-\bE}\)
\end{examplle}
\end{document}
