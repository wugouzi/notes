% Created 2019-07-26 五 18:42
% Intended LaTeX compiler: pdflatex
\documentclass[11pt]{article}
\usepackage[utf8]{inputenc}
\usepackage[T1]{fontenc}
\usepackage{graphicx}
\usepackage{grffile}
\usepackage{longtable}
\usepackage{wrapfig}
\usepackage{rotating}
\usepackage[normalem]{ulem}
\usepackage{amsmath}
\usepackage{textcomp}
\usepackage{amssymb}
\usepackage{capt-of}
\usepackage{hyperref}
\usepackage{minted}
% TIPS
% \substack{a\\b} for multiple lines text





% pdfplots will load xolor automatically without option
\usepackage[dvipsnames]{xcolor}

\usepackage{forest}
% two-line text in node by [two \\ lines]
% \begin{forest} qtree, [..] \end{forest}
\forestset{
  qtree/.style={
    baseline,
    for tree={
      parent anchor=south,
      child anchor=north,
      align=center,
      inner sep=1pt,
    }}}
%\usepackage{flexisym}
% load order of mathtools and mathabx, otherwise conflict overbrace

\usepackage{mathtools}
%\usepackage{fourier}
\usepackage{pgfplots}
\usepackage{amsthm, mathabx,  amsmath, commath}
\usepackage{amsfonts}

\usepackage{empheq}
\usepackage{tikz}
\usetikzlibrary{arrows.meta}
\usepackage[most]{tcolorbox}

\newtheorem{theorem}{Theorem}[section]
\newtheorem{definition}{Definition}[section]
\newtheorem{corollary}{Corollary}[section]
\newtheorem{example}{Example}[section]
\newtheorem{lemma}{Lemma}[section]
\newtheorem{proposition}{Proposition}[section]

\newcommand{\bl}[1] {\boldsymbol{#1}}
\newcommand{\Wt}[1] {\stackrel{\sim}{\smash{#1}\rule{0pt}{1.1ex}}}
\newcommand{\wt}[1] {\widetilde{#1}}


%For boxed texts in align, use Aboxed{}
%otherwise use boxed{}

\DeclareMathSymbol{\widehatsym}{\mathord}{largesymbols}{"62}
\newcommand\lowerwidehatsym{%
  \text{\smash{\raisebox{-1.3ex}{%
    $\widehatsym$}}}}
\newcommand\fixwidehat[1]{%
  \mathchoice
    {\accentset{\displaystyle\lowerwidehatsym}{#1}}
    {\accentset{\textstyle\lowerwidehatsym}{#1}}
    {\accentset{\scriptstyle\lowerwidehatsym}{#1}}
    {\accentset{\scriptscriptstyle\lowerwidehatsym}{#1}}
}

\usepackage{graphicx}
    
% text on arrow for xRightarrow
\makeatletter
%\newcommand{\xRightarrow}[2][]{\ext@arrow 0359\Rightarrowfill@{#1}{#2}}
\makeatother


\def \bx {\boldsymbol{x}}
\def \ba {\boldsymbol{a}}
\def \bI {\boldsymbol{I}}
\def \bt {\boldsymbol{t}}
\def \bb {\boldsymbol{b}}
\def \bA {\boldsymbol{A}}
\def \bX {\boldsymbol{X}}
\def \bu {\boldsymbol{u}}
\def \bS {\boldsymbol{S}}
\def \bZ {\boldsymbol{Z}}
\def \bz {\boldsymbol{z}}
\def \by {\boldsymbol{y}}
\def \bw {\boldsymbol{w}}
\def \bT {\boldsymbol{T}}
\def \bS {\boldsymbol{S}}
\def \bm {\boldsymbol{m}}
\def \bW {\boldsymbol{W}}
\def \bY {\boldsymbol{Y}}
\def \bH {\boldsymbol{H}}
\def \blambda {\boldsymbol{\lambda}}
\def \bPhi {\boldsymbol{\Phi}}
\def \btheta {\boldsymbol{\theta}}
\def \bmu {\boldsymbol{\mu}}
\def \bphi {\boldsymbol{\phi}}
\def \bSigma {\boldsymbol{\Sigma}}
\def \lb {\left\{}
\def \rb {\right\}}
\def \caln {\mathcal{N}}
\def \dissum {\displaystyle\Sigma}
\def \dispro {\displaystyle\prod}
\def \E {\mathbb{E}}
\def \Q {\mathbb{Q}}
\def \V {\mathbb{V}}
\def \R {\mathbb{R}}
\def \calq {\mathcal{Q}}
\def \calg {\mathcal{G}}
\def \caln {\mathcal{N}}
\def \calr {\mathcal{R}}
\def \calm {\mathcal{M}}
\def \calc {\mathcal{C}}
\def \bcup {\bigcup}

\author{gouziwu}
\date{\today}
\title{Rough Set Theory: A True Landmark in Data Analysis}
\hypersetup{
 pdfauthor={gouziwu},
 pdftitle={Rough Set Theory: A True Landmark in Data Analysis},
 pdfkeywords={},
 pdfsubject={},
 pdfcreator={Emacs 26.2 (Org mode 9.2.4)}, 
 pdflang={English}}
\begin{document}

\maketitle
\tableofcontents \clearpage\section{Rough Sets on Fuzzy Approximation Spaces and Intuitionistic Fuzzy Approximation Spaces}
\label{sec:orga6aab5d}
\subsection{Introduction}
\label{sec:org7de1383}
\subsubsection{Fuzzy Sets}
\label{sec:org6b9a5b5}
\subsubsection{Intuitionistic Fuzzy Sets}
\label{sec:orgde6098d}
the membership and nonmembership values of an element with respect to a
collection of elements from a universe may not add up to 1 in all possible
cases   

\begin{definition}[]
An \emph{intuitionistic fuzzy set} \(A\) on a universe \(U\) is defined by two
functions: \emph{membership function} \(\mu_A\) and \emph{non-membership function} \(\nu\) s.t.
\begin{equation*}
\mu_A, \nu_A: U\to [0,1]
\end{equation*}
where \(0\le \mu_A(x)+\nu_A(x)\le 1\) for all \(x\in U\).
\end{definition}

The \emph{hesitation function} \(\Pi_A\) for an intuitionistic fuzzy set is given by
\begin{equation*}
\Pi_A(x)=1-\mu_A(x)-\nu_A(x)
\end{equation*}
\subsubsection{Rough set}
\label{sec:orgc506de4}
A knowledge base is also called an \textbf{approximation space}
\subsubsection{Motivation}
\label{sec:orge8ab089}
\subsubsection{Fuzzy proximity relation}
\label{sec:org9ac2106}
\begin{definition}[]
Let \(U\) be a universal set and \(X\subseteq U\). Then a \emph{fuzzy relation} on \(X\)
is defined as any fuzzy set defined on \(X\times X\)
\end{definition}
\begin{definition}[]
A fuzzy relation \(R\) is said to be \emph{fuzzy reflexive} on \(X\subseteq U\) if it
satisfies
\begin{equation*}
\mu_R(x,x)=1\quad\text{for all } x\
\end{equation*}
\end{definition}
\begin{definition}[]
A fuzzy relation \(R\) is said to be \emph{fuzzy symmetric} on \(X\subseteq U\) if it
satisfies
\begin{equation*}
\mu_R(x,y)=\mu_R(y,x)\quad\text{for all } x,y\in X
\end{equation*}
\end{definition}
\begin{definition}[]
A fuzzy relation on \(X\subseteq U\) is said to be a \emph{fuzzy proximity relation}
if it is fuzzy reflexive and fuzzy symmetric.
\end{definition}
\begin{definition}[]
Let \(X,Y\subseteq U\). A fuzzy relation from \(X\) to \(Y\) is a fuzzy set
defined on \(X\times Y\) characterized by the membership function \(\mu_R:X\times Y\to
    [0,1]\) 
\end{definition}
\begin{definition}[]
For any \(\alpha\in [0,1]\), the \(\alpha\textit{-cut}\) of \(R\), denoted by \(R_\alpha\) is a subset
of \(X\times Y\) given by \(R_\alpha=\lb(x,y):\mu_R(x,y)\ge\alpha\rb\)
\end{definition}
Let \(R\) be a fuzzy proximity relation on \(U\). Then for any \(\alpha\in [0,1]\) the
elements of \(R_\alpha\) are said to be \(\alpha\textit{-similar}\) to each
other. \(xR_\alpha y\). 

Two elements \(x\) and \(y\) in \(U\) are said to be \(\alpha\textit{-identical}\)
w.r.t. \(R\) (\(xR(\alpha)y\)) if either \(x\) and \(y\) are \(\alpha\text{-similar}\) or \(x\)
and \(y\) are \emph{transitively} \(\alpha\textit{-similar}\), that is, there exists a
sequence of elements \(u_1, u_2, \dots,u_n\) in \(U\) s.t. \linebreak
\(xR_\alpha u_1, u_1 R_\alpha u_2
    ,\dots,u_n R_\alpha y\)
\subsubsection{Intuitionistic fuzzy proximity relation}
\label{sec:org09e52d6}
\begin{definition}[]
An \emph{intuitionistic fuzzy relation} on a universal set \(U\) is an intuitionistic
fuzzy set defined on \(U\times U\)
\end{definition}
\begin{definition}[]
An intuitionistic fuzzy relation \(R\) on a universal set \(U\) is said to be 
\emph{intuitionstic fuzzy reflexive} if 
\begin{equation*}
\mu_R(x,x)=1\text{ and } \nu_R(x,x)=0\quad\text{for all } x\in X
\end{equation*}
\end{definition}
\begin{definition}[]
An intuitionistic fuzzy relation \(R\) on a universal set \(U\) is said to be 
\emph{intuitionistic fuzzy symmetric} if
\begin{equation*}
\mu_R(x,y)=\mu_R(y,x)\text{ and } \nu_R(x,y)=\nu_R(y,x)\quad\text{for all } x,y\in X
\end{equation*}
\end{definition}
\begin{definition}[]
\emph{intuitionistic fuzzy proximity}
\end{definition}
Define
\begin{equation*}
J=\lb(m,n)\mid m,n\in[0,1] \text{ and } 0\le m+n\le 1\rb
\end{equation*}
\begin{definition}[]
Le \(R\) be an IF-proximity relation on \(U\). Then for any \((\alpha,\beta)\in J\) the 
\((\alpha,\beta)\textit{-cut}\) of \(R\), denoted by R\textsubscript{\(\alpha\),\(\beta\)} is 
\begin{equation*}
R_{\alpha,\beta}=\lb(x,y)\mid\mu_R(x,y)\le\alpha\text{ and } \nu_R(x,y)\le\beta\rb
\end{equation*}
\end{definition}
The relation \(R(\alpha,\beta)\) is an equivalence relation.
\subsection{Rough Sets on Fuzzy Approximation}
\label{sec:org6636b93}
\subsubsection{Preliminaries}
\label{sec:org05193a7}
\begin{definition}[]
For any set of fuzzy proximity relation \(K=(U,\mathfrak{R})\) is called a
\emph{fuzzy approximation space}
\end{definition}

For any fixed \(\alpha\in[0,1]\), \(\fR\) generates a set of equivalence relation
\(\fR(\alpha)\) and we call the associated space \(\bK(\alpha)=(U,\fR(\alpha))\) as the 
\emph{generated approximation space} corresponding to \(\bK\) and \(\alpha\)
\subsubsection{Properties}
\label{sec:org21736f6}
\subsubsection{Reduction of Knowledge in Fuzzy Approximation Spaces}
\label{sec:orga8f0600}
\begin{definition}[]
Let \(\fR\) be a family of fuzzy proximity relations on \(U\) and \(\alpha\in[0,1]\).
For any \(R\in \fR\), we say that \(R\) is \(\alpha\textit{-dispensable}\) or 
\(\alpha\textit{-superfluous}\) in \(\fR\) if and only if 
\(IND(\fR(\alpha))=IND(\fR(\alpha)-R(\alpha))\)
\end{definition}


Consider \(U=\lb x_1,\dots,x_n\rb\). Define the fuzzy proximity relations
\(P,Q,R\) and \(S\) over \(U\) corresponding to the attributes \(a,b,c\) and \(d\)
respectively.

\begin{table}[htbp]
\caption{Fuzzy proximity relation for attribute \(R\)}
\centering
\begin{tabular}{lrrrrr}
\hline
P & x\textsubscript{1} & x\textsubscript{2} & x\textsubscript{3} & x\textsubscript{4} & x\textsubscript{5}\\
\hline
x\textsubscript{1} & 1 & 0.3 & 0.6 & 0.8 & 0.5\\
x\textsubscript{2} & 0.3 & 1 & 0.7 & 0.4 & 0.4\\
x\textsubscript{3} & 0.6 & 0.7 & 1 & 0.2 & 0.8\\
x\textsubscript{4} & 0.8 & 0.4 & 0.2 & 1 & 0.5\\
x\textsubscript{5} & 0.5 & 0.4 & 0.8 & 0.5 & 1\\
\hline
\end{tabular}
\end{table}

\begin{table}[htbp]
\caption{Fuzzy proximity relation for attribute \(Q\)}
\centering
\begin{tabular}{lrrrrr}
\hline
P & x\textsubscript{1} & x\textsubscript{2} & x\textsubscript{3} & x\textsubscript{4} & x\textsubscript{5}\\
\hline
x\textsubscript{1} & 1 & 0.3 & 0.4 & 0.2 & 0.5\\
x\textsubscript{2} & 0.3 & 1 & 0.8 & 0.6 & 0.6\\
x\textsubscript{3} & 0.4 & 0.8 & 1 & 0.3 & 0.9\\
x\textsubscript{4} & 0.2 & 0.6 & 0.3 & 1 & 0.7\\
0.5 & 0.2 & 0.2 & 0.9 & 0.7 & 1\\
\hline
\end{tabular}
\end{table}

\begin{table}[htbp]
\caption{Fuzzy proximity relation for attribute \(R\)}
\centering
\begin{tabular}{lrrrrr}
\hline
R & x\textsubscript{1} & x\textsubscript{2} & x\textsubscript{3} & x\textsubscript{4} & x\textsubscript{5}\\
\hline
x\textsubscript{1} & 1 & 0.3 & 0.2 & 0.8 & 0.7\\
x\textsubscript{2} & 0.3 & 1 & 0.5 & 0.3 & 0.5\\
x\textsubscript{3} & 0.2 & 0.5 & 1 & 0.6 & 0.4\\
x\textsubscript{4} & 0.8 & 0.3 & 0.6 & 1 & 0.9\\
x\textsubscript{5} & 0.7 & 0.5 & 0.4 & 0.9 & 1\\
\hline
\end{tabular}
\end{table}

\begin{table}[htbp]
\caption{Fuzzy proximity relation for attribute \(S\)}
\centering
\begin{tabular}{lrrrrr}
\hline
S & x\textsubscript{1} & x\textsubscript{2} & x\textsubscript{3} & x\textsubscript{4} & x\textsubscript{5}\\
\hline
x\textsubscript{1} & 1 & 0.3 & 0.2 & 0.2 & 0.5\\
x\textsubscript{2} & 0.3 & 1 & 0.5 & 0.3 & 0.2\\
x\textsubscript{3} & 0.2 & 0.5 & 1 & 0.2 & 0.4\\
x\textsubscript{4} & 0.2 & 0.3 & 0.2 & 1 & 0.5\\
x\textsubscript{5} & 0.5 & 0.4 & 0.4 & 0.5 & 1\\
\hline
\end{tabular}
\end{table}

\begin{table}[htbp]
\caption{Fuzzy proximity relation for \(IND(\fR(\alpha))\)}
\centering
\begin{tabular}{crrrrrl}
\hline
\(IND(\fR(\alpha))\) & x\textsubscript{1} & x\textsubscript{2} & x\textsubscript{3} & x\textsubscript{4} & x\textsubscript{5} & \\
\hline
x\textsubscript{1} & 1 & 0.3 & 0.2 & 0.2 & 0.5 & \\
x\textsubscript{2} & 0.3 & 1 & 0.3 & 0.3 & 0.2 & \\
x\textsubscript{3} & 0.2 & 0.3 & 1 & 0.2 & 0.4 & \\
x\textsubscript{4} & 0.2 & 0.3 & 0.2 & 1 & 0.4 & \\
x\textsubscript{5} & 0.5 & 0.2 & 0.4 & 0.4 & 1 & \\
\hline
\end{tabular}
\end{table}

Suppose \(\alpha=0.6\), then we get
\begin{align*}
  &U/P(\alpha)=\{\{x_1,x_2,x_3,x_4,x_5\}\}\\
  &U/Q(\alpha)=\{\{x_1\},\{x_2,x_3,x_4,x_5\}\}\\
  &U/R(\alpha)=\{\{x_1,x_3,x_4,x_5\},\{x_2\}\}\\
  &U/S(\alpha)=\{\{x_1\},\{x_2\},\{x_3\},\{x_4\},\{x_5\}\}\\
\end{align*}
\subsubsection{Relative reducts and relative core of knowledge in fuzzy approximation spaces}
\label{sec:orgff81e26}
\begin{definition}[]
Let \(P\) and \(Q\) be two fuzzy proximity relations over the universe \(U\). For
every fixed \(\alpha\in[0,1]\), the \(\alpha\textit{-positive region}\) of \(P\) w.r.t. \(Q\)
can be defined as
\begin{equation*}
\alpha\text{-}POS_P Q=\displaystyle\bigcup_{X_\alpha\in U/Q} \underline{P}X_\alpha
\end{equation*}
\end{definition}
\begin{definition}[]
Let \(\bP\) and \(\bQ\) be two families of fuzzy proximity relations on \(U\). For
every fixed \(\alpha\in [0,1]\) and \(R\in \bP\) ,\(R\) is
\((\bQ,\alpha)\textit{-dispensable}\) in \(\bP\) if 
\begin{equation*}
\alpha\text{-}POS_IND (\bQ)=\alpha\text{-}POS_{IND(\bP-\lb R\rb)}IND(\bQ)
\end{equation*}
\end{definition}

\subsubsection{Dependency of knowledge in fuzzy approximation spaces}
\label{sec:org00ede04}
\subsubsection{partial dependency of knowledge in fuzzy approximation spaces}
\label{sec:orgac210d1}
\end{document}