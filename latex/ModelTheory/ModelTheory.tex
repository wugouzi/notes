% Created 2020-09-19 六 12:54
% Intended LaTeX compiler: pdflatex
\documentclass[11pt]{article}
\usepackage[utf8]{inputenc}
\usepackage[T1]{fontenc}
\usepackage{graphicx}
\usepackage{grffile}
\usepackage{longtable}
\usepackage{wrapfig}
\usepackage{rotating}
\usepackage[normalem]{ulem}
\usepackage{amsmath}
\usepackage{textcomp}
\usepackage{amssymb}
\usepackage{capt-of}
\usepackage{hyperref}
\usepackage{minted}
% TIPS
% \substack{a\\b} for multiple lines text





% pdfplots will load xolor automatically without option
\usepackage[dvipsnames]{xcolor}

\usepackage{forest}
% two-line text in node by [two \\ lines]
% \begin{forest} qtree, [..] \end{forest}
\forestset{
  qtree/.style={
    baseline,
    for tree={
      parent anchor=south,
      child anchor=north,
      align=center,
      inner sep=1pt,
    }}}
%\usepackage{flexisym}
% load order of mathtools and mathabx, otherwise conflict overbrace

\usepackage{mathtools}
%\usepackage{fourier}
\usepackage{pgfplots}
\usepackage{amsthm, mathabx,  amsmath, commath}
\usepackage{amsfonts}

\usepackage{empheq}
\usepackage{tikz}
\usetikzlibrary{arrows.meta}
\usepackage[most]{tcolorbox}

\newtheorem{theorem}{Theorem}[section]
\newtheorem{definition}{Definition}[section]
\newtheorem{corollary}{Corollary}[section]
\newtheorem{example}{Example}[section]
\newtheorem{lemma}{Lemma}[section]
\newtheorem{proposition}{Proposition}[section]

\newcommand{\bl}[1] {\boldsymbol{#1}}
\newcommand{\Wt}[1] {\stackrel{\sim}{\smash{#1}\rule{0pt}{1.1ex}}}
\newcommand{\wt}[1] {\widetilde{#1}}


%For boxed texts in align, use Aboxed{}
%otherwise use boxed{}

\DeclareMathSymbol{\widehatsym}{\mathord}{largesymbols}{"62}
\newcommand\lowerwidehatsym{%
  \text{\smash{\raisebox{-1.3ex}{%
    $\widehatsym$}}}}
\newcommand\fixwidehat[1]{%
  \mathchoice
    {\accentset{\displaystyle\lowerwidehatsym}{#1}}
    {\accentset{\textstyle\lowerwidehatsym}{#1}}
    {\accentset{\scriptstyle\lowerwidehatsym}{#1}}
    {\accentset{\scriptscriptstyle\lowerwidehatsym}{#1}}
}

\usepackage{graphicx}
    
% text on arrow for xRightarrow
\makeatletter
%\newcommand{\xRightarrow}[2][]{\ext@arrow 0359\Rightarrowfill@{#1}{#2}}
\makeatother


\def \bx {\boldsymbol{x}}
\def \ba {\boldsymbol{a}}
\def \bI {\boldsymbol{I}}
\def \bt {\boldsymbol{t}}
\def \bb {\boldsymbol{b}}
\def \bA {\boldsymbol{A}}
\def \bX {\boldsymbol{X}}
\def \bu {\boldsymbol{u}}
\def \bS {\boldsymbol{S}}
\def \bZ {\boldsymbol{Z}}
\def \bz {\boldsymbol{z}}
\def \by {\boldsymbol{y}}
\def \bw {\boldsymbol{w}}
\def \bT {\boldsymbol{T}}
\def \bS {\boldsymbol{S}}
\def \bm {\boldsymbol{m}}
\def \bW {\boldsymbol{W}}
\def \bY {\boldsymbol{Y}}
\def \bH {\boldsymbol{H}}
\def \blambda {\boldsymbol{\lambda}}
\def \bPhi {\boldsymbol{\Phi}}
\def \btheta {\boldsymbol{\theta}}
\def \bmu {\boldsymbol{\mu}}
\def \bphi {\boldsymbol{\phi}}
\def \bSigma {\boldsymbol{\Sigma}}
\def \lb {\left\{}
\def \rb {\right\}}
\def \caln {\mathcal{N}}
\def \dissum {\displaystyle\Sigma}
\def \dispro {\displaystyle\prod}
\def \E {\mathbb{E}}
\def \Q {\mathbb{Q}}
\def \V {\mathbb{V}}
\def \R {\mathbb{R}}
\def \calq {\mathcal{Q}}
\def \calg {\mathcal{G}}
\def \caln {\mathcal{N}}
\def \calr {\mathcal{R}}
\def \calm {\mathcal{M}}
\def \calc {\mathcal{C}}
\def \bcup {\bigcup}

\author{C. C. Chang \& H. Jerome Keisler}
\date{\today}
\title{Model Theory}
\hypersetup{
 pdfauthor={C. C. Chang \& H. Jerome Keisler},
 pdftitle={Model Theory},
 pdfkeywords={},
 pdfsubject={},
 pdfcreator={Emacs 26.3 (Org mode 9.4)}, 
 pdflang={English}}
\begin{document}

\maketitle
\tableofcontents \clearpage
\section{Models Constructed From Constants}
\label{sec:orgf683279}

\subsection{Completeness and Compactness}
\label{sec:org1f7da0f}
\begin{definition}[]
Let \(T\) be a set of sentences of \(\call\) and let \(C\) be a set of
constant symbols of \(\call\). We say that \(C\) is a \textbf{set of witnesses} for
\(T\) iff for every formula \(\varphi\) of \(\call\) with at most one free variable,
say ,\(x\), there is a constant \(c\in C\) s.t.
\begin{equation*}
T\vdash(\exists x)\varphi \to\varphi(c)
\end{equation*}
We say that \(T\) \textbf{has witnesses} in \(\call\) iff \(T\) has some set \(C\) of
witness in \(\call\)
\end{definition}

\begin{lemma}[]
\label{lemma2.1.1}
Let \(T\) be a consistent set of sentences of \(\call\). Let \(C\) be a set
of new constant symbols of power \(\abs{C}=\norm{\call}\), and let
\(\bbar{\call}=\call\cup C\) be the simple extension of \(\call\) formed by
adding \(C\). Then \(T\) can be extended to a consistent set of sentences
\(\bbar{T}\) in \(\bbar{\call}\) which has \(C\) as a set of witnesses in \(\bbar{\call}\)
\end{lemma}

\begin{proof}
Let \(\alpha=\norm{\call}\). For each \(\beta<\alpha\), let \(c_\beta\) be a
constant symbol which does not occur in \(\call\) and s.t. \(\c_\beta\neq
   c_\gamma\) if \(\beta<\gamma<\alpha\). Let \(C=\{c_\beta:\beta<\alpha\}\),
\(\bbar{\call}=\call\cup C\). Clearly \(\norm{\bbar{\call}}=\alpha\), so we
may arrange all formulas of \(\bbar{\call}\) with at most one free variable
in a sequence \(\varphi_\xi,\xi<\alpha\). We now define an increasing sequence
of sets of sentences of \(\bbar{\call}\):
\begin{equation*}
T=T_0\subset T_1\subset\dots\subset T_\xi\subset\dots,\quad\xi<\alpha
\end{equation*}
and a sequence \(d_\xi,\xi<\alpha\) of constants from \(C\) s.t.
\begin{enumerate}
\item each \(T_\xi\) is consistent in \(\bbar{\call}\)
\item if \(\xi=\xi+1\), then \(T_\xi=T_\zeta\cup\{(\exists
      x_\zeta)\varphi_\zeta\to\varphi_\zeta(d_\zeta)\}\); \(\xi_\zeta\) is the
free variable in \(\varphi_\zeta\) if it has one, otherwise \(x_\xi=v_0\)
\end{enumerate}
3.if \(\xi\) is a limit ordinal different from 0, then
\(T_\xi=\bigcup_{\zeta<\xi}T_\zeta\)


Let \(d_\zeta\) be the first element of \(C\) which has not yet occurred in
\(T_\zeta\). We show that
\begin{equation*}
T_{\zeta+1}=T_\zeta\cup\{(\exists x_\zeta)\varphi_\zeta\to\varphi_\zeta(d_\zeta)\}
\end{equation*}
is consistent. If this were not the case, then
\begin{equation*}
   T_\zeta\vdash\neg((\exists x_\zeta)\varphi_\zeta\to\varphi_\zeta(d_\zeta))
\end{equation*}
By propositional logic
\begin{equation*}
T_\zeta\vdash(\exists x_\zeta)\varphi_\zeta\wedge\neg\varphi_\zeta(d_\zeta)
\end{equation*}
As \(d_\zeta\) does not occur in \(T_\zeta\), we have by predicate logic
\begin{gather*}
T_\zeta\vdash(\forall x_\zeta)((\exists x_\zeta)\varphi_\zeta\wedge\neg\varphi_\zeta
(x_\zeta))\\
T_\zeta\vdash(\exists x_\zeta)\varphi_\zeta\wedge\neg(\exists x_\zeta)\varphi_\zeta
\end{gather*}
which contradicts the consistency of \(T_\zeta\). If \(\xi\) is a nonzero limit
ordinal, and each member of the increasing chain \(T_\zeta,\zeta<\xi\) is
consistent, then \(T_\xi\) is consistent.

Now let \(\bbar{T}=\bigcup_{\xi<\alpha}T_\xi\). Suppose \(\varphi\) is a formula of
\(\bbar{\call}\) with at most the variable \(x\) free. Then we may assume
that \(\varphi=\varphi_xi\) and \(x=x_\xi\) for some \(\xi<\alpha\). Whence the
sentence
\begin{equation*}
(\exists x_\xi)\varphi_xi\to\varphi_\xi(d_\xi)
\end{equation*}
belongs to \(T_{\xi+1}\) and so to \(\bbar{T}\)
\end{proof}

\begin{lemma}[]
\label{lemma2.1.2}
Let \(T\) be a consistent set of sentences and \(C\) be a set of witnesses
for \(T\) in \(\call\). Then \(T\) has a model \(\fA\) s.t. every element of
\(\fA\) is an interpretation of a constant \(c\in C\)
\end{lemma}

\begin{proof}
If a set of sentences \(T\) has a set \(C\) of witnesses in \(\call\), then
\(C\) is also a set of witnesses for every extension of \(T\). Second, if an
extension of \(T\) has a model \(\fA\), then \(fA\) is also a model of \(T\).
So we may assume that \(T\) is maximal consistent in \(\call\)

For two constants \(c,d\in C\), define
\begin{equation*}
c\sim d \quad\text{ iff }\quad
c\equiv d\in T
\end{equation*}
Because \(T\) is maximal consistent, we see that \(\sim\) is an equivalence
relation on \(C\). For each \(c\in C\), let
\begin{equation*}
\widetilde{c}=\{d\in C:d\sim c\}
\end{equation*}
be the equivalence class of \(c\). We propose to construct a model \(\fA\)
whose set of elements \(A\) is the set of all these equivalence classes
\(\widetilde{c}\), for \(c\in C\); so we define
\begin{enumerate}
\item \(A=\{\widetilde{c}:c\in C\}\)
\end{enumerate}



\begin{enumerate}
\setcounter{enumi}{1}
\item For each \(n\)-placed relation symbol \(P\) in \(\call\), we define an
\(n\)-placed relation \(R'\) on the set \(C\) by: for all \(c_1,\dots,c_n\in
      C\)

\(R'(c_1,\dots, c_n)\) iff \(P(c_1,\dots,c_n)\in T\)

By our axioms of identity, we have

\begin{equation*}
\vdash P(c_1,\dots,c_n)\wedge c_1\equiv d_1\wedge\dots\wedge c_n\equiv d_n\to
P(d_1,\dots,d_n)
\end{equation*}
So \(\sim\) is what is called a \textbf{congruence relation}.
\(R(\widetilde{c}_1,\dots,\widetilde{c}_n)\) iff
\(P(c_1,\dots,c_n)\in T\)
\item Now consider a constant symbol \(d\) of \(\call\). From predicate logic, we
have
\begin{equation*}
\vdash(\exists v_0)(d\equiv v_0)
\end{equation*}
So \((\exists v_0)(d\equiv v_0)\in T\), and because \(T\) has witnesses,
there is a constant \(c\in C\) s.t.
\begin{equation*}
(d\equiv c)\in T
\end{equation*}

the constant \(c\) may not be unique, but its equivalence class is unique
 because
\begin{equation*}
\vdash(d\equiv c\wedge d\equiv c'\to c\equiv c')
\end{equation*}

\item Let \(F\) be any \(m\)-placed function symbol of \(\call\), and let
\(c_1,\dots,c_m\in C\). We have
\begin{equation*}
(\exists v_0)(F(c_1,\dots,c_m)\equiv v_0)\in T
\end{equation*}
hence there is a constant \(c\in C\) s.t.
\begin{equation*}
(F(c_1,\dots ,c_m)\equiv c)\in T
\end{equation*}
We use our axioms of identity to obtain
\begin{equation*}
\vdash (F(c_1\dots c_m)\equiv c\wedge
c_1\equiv d_1\wedge\dots\wedge c_m\equiv d_m\wedge c\equiv d)\to
F(d_1\dots d_m)\equiv d
\end{equation*}
Hence we define

\(G(\widetilde{c}_1\dots\widetilde{c}_m)\) iff
\((F(c_1\dots c_m)\equiv c)\in T\)

By induction
\begin{equation*}
\fA\models t\equiv c \quad\text{ iff }\quad
(t\equiv c)\in T
\end{equation*}
Since \(C\) is a set of witness for \(T\), we have: for any terms
\(t_1,t_2\) of \(\call\) with no free variables
\begin{equation*}
\fA\models t_1\equiv t_2 \quad\text{ iff }\quad
(t_1\equiv t_2)\in T
\end{equation*}

for any atomic formula \(P(t_1\dots t_n)\) of \(\call\) containing no free
variables
\begin{equation*}
\fA\models P(t_1\dots t_n) \quad\text{ iff }\quad
P(t_1\dots t_n)\in T
\end{equation*}

Hence for any sentence \(\varphi\) of \(\call\)
\begin{equation*}
\fA\models\varphi \quad\text{ iff }\quad
\varphi\in T
\end{equation*}

Suppose \(\varphi=(\exists x)\psi\). If \(fA\models\varphi\), then for
some \(\widetilde{c}\in A,\fA\models\psi[\widetilde{c}]\). This means that
\(\fA\models\psi(c)\). So \(\psi(c)\in T\) and because
\begin{equation*}
\vdash\psi(c)\to(\exists x)\psi
\end{equation*}
we have \(\varphi\in T\). On the other hand, if \(\varphi\in T\), then
because \(T\) has witnesses, there exists a constant \(c\in C\) s.t.
\(\psi(c)\in T\), so \(\fA\models\psi(c)\). This gives
\(\fA\models\psi[\widetilde{c}]\) and \(\fA\models\varphi\)
\end{enumerate}
\end{proof}

\begin{lemma}[]
\label{lemma2.1.3}
Let \(C\) be a set of constant symbols of \(\call\), and let \(T\) be a
set of sentences of \(\call\). If \(T\) has a model \(\fA\) s.t. every
element of \(\fA\) is an interpretation of some constant \(c\in C\), then
\(T\) can be extended to a consistent \(\bbar{T}\) in \(\call\) for which
\(C\) is a set of witnesses
\end{lemma}

\begin{proof}
Let \(\bbar{T}\) be the sentences of \(\call\) true in \(\fA\)
\end{proof}

\begin{theorem}[Extended Completeness Theorem]
   \label{thm1.3.21} 
Let \(\Sigma\) be a set of sentences of \(\call\). Then \(\Sigma\) is consistent iff \(\Sigma\) has a model
\end{theorem}

\begin{proof}
Assume \(\Sigma\) is consistent.  By Lemma \ref{lemma2.1.1} we consider extensions
\(\bbar{\Sigma}\) of \(\Sigma\) and \(\bbar{\call}\) of \(\call\), so that \(\bbar{\Sigma}\) has
witnesses in \(\bbar{\call}\). By Lemma \ref{lemma2.1.2} let \(\fA\) be the
model of \(\bbar{\Sigma}\). Let \(\fB\) be the model for \(\call\) which is the
reduct of \(\fA\) to \(\call\).
\end{proof}

\begin{corollary}[Downward Löwenheim–Skolem Theorem]


Every consistent theory \(T\) in \(\call\) has a model of power at most \(\norm{\call}\)
\end{corollary}

\begin{proof}
Choose \(\fA\) so that every element is a constant.

\(\abs{B}=\abs{A}\le\norm{\bbar{\call}}=\norm{\call}\)
\end{proof}

\begin{theorem}[Gödel's Completeness Theorem]
\label{thm1.3.20}
A sentence of \(\call\) is a theorem of \(\call\) iff it is valid
\end{theorem}

\begin{proof}
If a sentence \(\sigma\) is not a theorem of \(\call\), then \(\{\neg\sigma\}\) is
consistent in \(\call\). By Theorem \ref{thm1.3.21}, \(\{\neg\sigma\}\) will
have a model where \(\sigma\) cannot hold. Hence \(\sigma\) is not valid
\end{proof}

\begin{theorem}[Compactness Theorem]
\label{thm1.3.22}
A set of sentences \(\Sigma\) has a model iff every finite subset of \(\Sigma\) has a model
\end{theorem}

\begin{proof}
If every finite subset of \(\Sigma\) has a model, then every finite subset of \(\Sigma\) is
consistent. So \(\Sigma\) is consistent and has a model by Theorem \ref{thm1.3.21}
\end{proof}

\begin{corollary}[]
If a theory \(T\) has arbitrarily large finite models, then it has an
infinite model
\end{corollary}

\begin{proof}
Consider the expansion \(\call'=\call\cup\{c_n:n\in\omega\}\)  where \(c_n\)
is a list of distinct constant symbols not in \(\call\). Consider the set \(\Sigma\)
of \(\call'\) defined by
\begin{equation*}
\Sigma=T\cup\{\neg(c_n\equiv c_m):n<m<\omega\}
\end{equation*}
Any finite subset \(\Sigma'\) of \(\Sigma\) will involve at most the constants
\(c_0,\dots,c_m\) for some \(m\). Let \(\fA\) be  a model of \(T\) with at
least \(m+1\) elements, and let \(a_0,\dots,a_m\) be a list of \(m+1\)
distinct elements of \(\fA\). The model \((\fA,a_0,\dots,a_m)\) for the
finite expansion \(\call''=\call\cup\{c_0,\dots,c_m\}\) of \(\call\) is a
model of (\(\Sigma\)'). So by Theorem \ref{thm1.3.22} \(\Sigma\) has a model.
\end{proof}

\begin{corollary}[Upward Löwenheim–Skolem-Tarski Theorem]
If \(T\) has infinite models, then it has infinite models of any given power \(\alpha\ge\norm{\call}\)
\end{corollary}


\textbf{Method of diagrams}. Let \(\fA\) be a model of \(\call\). We expand the
language \(\call\) to a new language
\begin{equation*}
\call_A=\call\cup\{c_a:a\in A\}
\end{equation*}
by If \(a\neq b\) and \(c_a,c_b\) are different symbols, we may then expand
\(\fA\) to the model
\begin{equation*}
\fA_A=(\fA,a)_{a\in A}
\end{equation*}
The \textbf{diagram of} \(\fA\), denote by \(\varlrtriangle_{\fA}\), is the set of all
atomic sentences and negations of atomic sentences of \(\call_A\) which hold
in the model \(\fA_A\)

If \(X\) is a subset of \(A\), then we let \(\call_X=\call\cup\{c_a:a\in
   X\}\) and \(\fA_X=(\fA,a)_{a\in X}\). If \(f\) is a mapping from \(X\) into
the set of elements \(B\) of a model \(\fB\) for \(\call\), then
\((\fB,fa)_{a\in X}\) is the expansion of \(\fB\) to a model for \(\call_X\)

\begin{proposition}[]
Let \(\fA,\fB\) be models for \(\call\) and let \(f:A\to B\). Then the
following are equivalent:
\begin{enumerate}
\item \(f\) is an isomorphic embedding of \(\fA\) into \(\fB\)
\item There is an extension \(\fC\supset\fA\) and an isomorphism
\(g:\fC\cong\fB\) s.t. \(g\supset f\)
\item \((\fB,fa)_{a\in A}\) is a model of the diagram of \(\fA\)
\end{enumerate}
\end{proposition}

\begin{proof}
\(1\to2\). Extend the set \(A\) to a set \(C\) and extend the function \(f\)
to a one-to-one function \(g\) from \(C\) onto \(B\). Then define the
relations
\begin{equation*}
\fC\models R[c_1\dots c_n] \quad\text{ iff }\quad
\fB\models R[gc_1\dots gc_n]
\end{equation*}

\(1\leftrightarrow2\). For each formula \(\varphi(x_1\dots x_n)\) and all
\(a_1,\dots,a_n\in A\)
\begin{equation*}
\fA\models\varphi[a_1\dots a_n] \quad\text{ iff }\quad
\fA_A\models\varphi(a_1\dots a_n)
\end{equation*}
and
\begin{equation*}
\fB\models\varphi[fa_1\dots fa_n] \quad\text{ iff }\quad
(\fB,fa)_{a\in A}\models\varphi(a_1\dots a_n)
\end{equation*}
\end{proof}

\begin{corollary}[]
Suppose that \(\call\) has no function or constant symbols. Let \(T\) be a
theory in \(\call\) and \(\fA\) be a model for \(\call\). Then \(\fA\) is
isomorphically embedded in some model of \(T\) iff every finite submodel of
\(\fA\) is isomorphically embedded in some model of \(T\)
\end{corollary}

\begin{proof}
Suppose every finite submodel of \(\fA\) is isomorphically embedded in some
model of \(T\). We show that the set \(\Sigma=T\cup\varlrtriangle_{\fA}\) is
consistent. Every finite subset \(\Sigma'\) of \(\Sigma\) contains at most a finite
number of the new constants
\end{proof}
\end{document}
