% Created 2019-07-31 三 20:44
% Intended LaTeX compiler: pdflatex
\documentclass[11pt]{article}
\usepackage[utf8]{inputenc}
\usepackage[T1]{fontenc}
\usepackage{graphicx}
\usepackage{grffile}
\usepackage{longtable}
\usepackage{wrapfig}
\usepackage{rotating}
\usepackage[normalem]{ulem}
\usepackage{amsmath}
\usepackage{textcomp}
\usepackage{amssymb}
\usepackage{capt-of}
\usepackage{hyperref}
\usepackage{minted}
% TIPS
% \substack{a\\b} for multiple lines text





% pdfplots will load xolor automatically without option
\usepackage[dvipsnames]{xcolor}

\usepackage{forest}
% two-line text in node by [two \\ lines]
% \begin{forest} qtree, [..] \end{forest}
\forestset{
  qtree/.style={
    baseline,
    for tree={
      parent anchor=south,
      child anchor=north,
      align=center,
      inner sep=1pt,
    }}}
%\usepackage{flexisym}
% load order of mathtools and mathabx, otherwise conflict overbrace

\usepackage{mathtools}
%\usepackage{fourier}
\usepackage{pgfplots}
\usepackage{amsthm, mathabx,  amsmath, commath}
\usepackage{amsfonts}

\usepackage{empheq}
\usepackage{tikz}
\usetikzlibrary{arrows.meta}
\usepackage[most]{tcolorbox}

\newtheorem{theorem}{Theorem}[section]
\newtheorem{definition}{Definition}[section]
\newtheorem{corollary}{Corollary}[section]
\newtheorem{example}{Example}[section]
\newtheorem{lemma}{Lemma}[section]
\newtheorem{proposition}{Proposition}[section]

\newcommand{\bl}[1] {\boldsymbol{#1}}
\newcommand{\Wt}[1] {\stackrel{\sim}{\smash{#1}\rule{0pt}{1.1ex}}}
\newcommand{\wt}[1] {\widetilde{#1}}


%For boxed texts in align, use Aboxed{}
%otherwise use boxed{}

\DeclareMathSymbol{\widehatsym}{\mathord}{largesymbols}{"62}
\newcommand\lowerwidehatsym{%
  \text{\smash{\raisebox{-1.3ex}{%
    $\widehatsym$}}}}
\newcommand\fixwidehat[1]{%
  \mathchoice
    {\accentset{\displaystyle\lowerwidehatsym}{#1}}
    {\accentset{\textstyle\lowerwidehatsym}{#1}}
    {\accentset{\scriptstyle\lowerwidehatsym}{#1}}
    {\accentset{\scriptscriptstyle\lowerwidehatsym}{#1}}
}

\usepackage{graphicx}
    
% text on arrow for xRightarrow
\makeatletter
%\newcommand{\xRightarrow}[2][]{\ext@arrow 0359\Rightarrowfill@{#1}{#2}}
\makeatother


\def \bx {\boldsymbol{x}}
\def \ba {\boldsymbol{a}}
\def \bI {\boldsymbol{I}}
\def \bt {\boldsymbol{t}}
\def \bb {\boldsymbol{b}}
\def \bA {\boldsymbol{A}}
\def \bX {\boldsymbol{X}}
\def \bu {\boldsymbol{u}}
\def \bS {\boldsymbol{S}}
\def \bZ {\boldsymbol{Z}}
\def \bz {\boldsymbol{z}}
\def \by {\boldsymbol{y}}
\def \bw {\boldsymbol{w}}
\def \bT {\boldsymbol{T}}
\def \bS {\boldsymbol{S}}
\def \bm {\boldsymbol{m}}
\def \bW {\boldsymbol{W}}
\def \bY {\boldsymbol{Y}}
\def \bH {\boldsymbol{H}}
\def \blambda {\boldsymbol{\lambda}}
\def \bPhi {\boldsymbol{\Phi}}
\def \btheta {\boldsymbol{\theta}}
\def \bmu {\boldsymbol{\mu}}
\def \bphi {\boldsymbol{\phi}}
\def \bSigma {\boldsymbol{\Sigma}}
\def \lb {\left\{}
\def \rb {\right\}}
\def \caln {\mathcal{N}}
\def \dissum {\displaystyle\Sigma}
\def \dispro {\displaystyle\prod}
\def \E {\mathbb{E}}
\def \Q {\mathbb{Q}}
\def \V {\mathbb{V}}
\def \R {\mathbb{R}}
\def \calq {\mathcal{Q}}
\def \calg {\mathcal{G}}
\def \caln {\mathcal{N}}
\def \calr {\mathcal{R}}
\def \calm {\mathcal{M}}
\def \calc {\mathcal{C}}
\def \bcup {\bigcup}

\author{Miaomiaomiao}
\date{\today}
\title{Non-standard Logics for Automated Reasoning}
\hypersetup{
 pdfauthor={Miaomiaomiao},
 pdftitle={Non-standard Logics for Automated Reasoning},
 pdfkeywords={},
 pdfsubject={},
 pdfcreator={Emacs 26.2 (Org mode 9.2.4)}, 
 pdflang={English}}
\begin{document}

\maketitle
\tableofcontents \clearpage
\section{Belief functions}
\label{sec:org22eb89a}
\subsection{Introduction}
\label{sec:org19a8f87}
\subsubsection{Possibility}
\label{sec:org6f22a50}
The information that "John's height is over 170cm" implies that, in
describing John, any height \(h\) over 170 is possible and any height equal to
or below 170 is impossible. This can be represented by a possibility
function on the height domain whose value is 0 for \(h\le 170\) and 1 for
\(h>170\). \textbf{Ignorance} is due to the lack of precision of specificity of the
information "over 170"
\subsubsection{Probability}
\label{sec:orgbd97aeb}
When throwing a dice, the probability that the outcome is one is \(1/6\).

This model can be generalized by considering that the probability of each
event is not known as a real value between 0 and 1, but as belonging to an
interval. 
\subsubsection{Credibility}
\label{sec:org2eee551}
Belief functions aim to model and to quantify the subjective, personal
credibility (called belief) induced in us by evidence. 
\subsection{The frame of discernment}
\label{sec:orgcd51fe5}
\subsubsection{open- and closed-world assumptions}
\label{sec:org6c2f6d9}
frame of discernment \(\delta\) (also called the Universe of discourse or the
domain of reference) where evidence induces some belief.

\emph{UP} as \emph{unknown propositions}, \emph{KP} as \emph{Known as Possible}, \emph{KI} as \emph{Known as
Impossible}.

\emph{closed-world assumption} postulates an empty \(UP\) set. The \emph{open-world
assumption} admits the existentce of a non-empty \(UP\) set.
\subsubsection{Notation}
\label{sec:org96b218e}
let \(\Omega\) be the boolean algebra of propositions derived from \(\Delta\). Let 
\(\bl{1}_\Omega\) be the tautology relative to \(\Omega\), i.e. \(\bl{1}_\Omega\) is
the disjunction of all elementary propositions of \(\Delta\). Let
\(\bl{0}_\Omega\) be the contradiction relative to \(\Omega\), i.e. none of the
propositions of \(\Delta\) implies \(\bl{0}_\Omega\). 

The set \(UP\) will be denoted by \(\Theta\)
\end{document}