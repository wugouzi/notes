% Created 2021-12-07 Tue 13:29
% Intended LaTeX compiler: pdflatex
\documentclass[11pt]{article}
\usepackage[utf8]{inputenc}
\usepackage[T1]{fontenc}
\usepackage{graphicx}
\usepackage{longtable}
\usepackage{wrapfig}
\usepackage{rotating}
\usepackage[normalem]{ulem}
\usepackage{amsmath}
\usepackage{amssymb}
\usepackage{capt-of}
\usepackage{hyperref}
\graphicspath{{../../books/}}
% wrong resolution of image
% https://tex.stackexchange.com/questions/21627/image-from-includegraphics-showing-in-wrong-image-size?rq=1

%%%%%%%%%%%%%%%%%%%%%%%%%%%%%%%%%%%%%%
%% TIPS                                 %%
%%%%%%%%%%%%%%%%%%%%%%%%%%%%%%%%%%%%%%
% \substack{a\\b} for multiple lines text
% \usepackage{expl3}
% \expandafter\def\csname ver@l3regex.sty\endcsname{}
% \usepackage{pkgloader}
\usepackage[utf8]{inputenc}

% nfss error
% \usepackage[B1,T1]{fontenc}
\usepackage{fontspec}

% \usepackage[Emoticons]{ucharclasses}
\newfontfamily\DejaSans{DejaVu Sans}
% \setDefaultTransitions{\DejaSans}{}

% pdfplots will load xolor automatically without option
\usepackage[dvipsnames]{xcolor}

%                                                             ┳┳┓   ┓
%                                                             ┃┃┃┏┓╋┣┓
%                                                             ┛ ┗┗┻┗┛┗
% \usepackage{amsmath} mathtools loads the amsmath
\usepackage{amsmath}
\usepackage{mathtools}

\usepackage{amsthm}
\usepackage{amsbsy}

%\usepackage{commath}

\usepackage{amssymb}

\usepackage{mathrsfs}
%\usepackage{mathabx}
\usepackage{stmaryrd}
\usepackage{empheq}

\usepackage{scalerel}
\usepackage{stackengine}
\usepackage{stackrel}



\usepackage{nicematrix}
\usepackage{tensor}
\usepackage{blkarray}
\usepackage{siunitx}
\usepackage[f]{esvect}

% centering \not on a letter
\usepackage{slashed}
\usepackage[makeroom]{cancel}

%\usepackage{merriweather}
\usepackage{unicode-math}
\setmainfont{TeX Gyre Pagella}
% \setmathfont{STIX}
%\setmathfont{texgyrepagella-math.otf}
%\setmathfont{Libertinus Math}
\setmathfont{Latin Modern Math}

 % \setmathfont[range={\smwhtdiamond,\enclosediamond,\varlrtriangle}]{Latin Modern Math}
\setmathfont[range={\rightrightarrows,\twoheadrightarrow,\leftrightsquigarrow,\triangledown,\vartriangle,\precneq,\succneq,\prec,\succ,\preceq,\succeq,\tieconcat}]{XITS Math}
 \setmathfont[range={\int,\setminus}]{Libertinus Math}
 % \setmathfont[range={\mathalpha}]{TeX Gyre Pagella Math}
%\setmathfont[range={\mitA,\mitB,\mitC,\mitD,\mitE,\mitF,\mitG,\mitH,\mitI,\mitJ,\mitK,\mitL,\mitM,\mitN,\mitO,\mitP,\mitQ,\mitR,\mitS,\mitT,\mitU,\mitV,\mitW,\mitX,\mitY,\mitZ,\mita,\mitb,\mitc,\mitd,\mite,\mitf,\mitg,\miti,\mitj,\mitk,\mitl,\mitm,\mitn,\mito,\mitp,\mitq,\mitr,\mits,\mitt,\mitu,\mitv,\mitw,\mitx,\mity,\mitz}]{TeX Gyre Pagella Math}
% unicode is not good at this!
%\let\nmodels\nvDash

 \usepackage{wasysym}

 % for wide hat
 \DeclareSymbolFont{yhlargesymbols}{OMX}{yhex}{m}{n} \DeclareMathAccent{\what}{\mathord}{yhlargesymbols}{"62}

%                                                               ┏┳┓•┓
%                                                                ┃ ┓┃┏┓
%                                                                ┻ ┗┛┗┗

\usepackage{pgfplots}
\pgfplotsset{compat=1.18}
\usepackage{tikz}
\usepackage{tikz-cd}
\tikzcdset{scale cd/.style={every label/.append style={scale=#1},
    cells={nodes={scale=#1}}}}
% TODO: discard qtree and use forest
% \usepackage{tikz-qtree}
\usepackage{forest}

\usetikzlibrary{arrows,positioning,calc,fadings,decorations,matrix,decorations,shapes.misc}
%setting from geogebra
\definecolor{ccqqqq}{rgb}{0.8,0,0}

%                                                          ┳┳┓•    ┓┓
%                                                          ┃┃┃┓┏┏┏┓┃┃┏┓┏┓┏┓┏┓┓┏┏
%                                                          ┛ ┗┗┛┗┗ ┗┗┗┻┛┗┗ ┗┛┗┻┛
%\usepackage{twemojis}
\usepackage[most]{tcolorbox}
\usepackage{threeparttable}
\usepackage{tabularx}

\usepackage{enumitem}
\usepackage[indLines=false]{algpseudocodex}
\usepackage[]{algorithm2e}
% \SetKwComment{Comment}{/* }{ */}
% \algrenewcommand\algorithmicrequire{\textbf{Input:}}
% \algrenewcommand\algorithmicensure{\textbf{Output:}}
% wrong with preview
\usepackage{subcaption}
\usepackage{caption}
% {\aunclfamily\Huge}
\usepackage{auncial}

\usepackage{float}

\usepackage{fancyhdr}

\usepackage{ifthen}
\usepackage{xargs}

\definecolor{mintedbg}{rgb}{0.99,0.99,0.99}
\usepackage[cachedir=\detokenize{~/miscellaneous/trash}]{minted}
\setminted{breaklines,
  mathescape,
  bgcolor=mintedbg,
  fontsize=\footnotesize,
  frame=single,
  linenos}
\usemintedstyle{xcode}
\usepackage{tcolorbox}
\usepackage{etoolbox}



\usepackage{imakeidx}
\usepackage{hyperref}
\usepackage{soul}
\usepackage{framed}

% don't use this for preview
%\usepackage[margin=1.5in]{geometry}
% \usepackage{geometry}
% \geometry{legalpaper, landscape, margin=1in}
\usepackage[font=itshape]{quoting}

%\LoadPackagesNow
%\usepackage[xetex]{preview}
%%%%%%%%%%%%%%%%%%%%%%%%%%%%%%%%%%%%%%%
%% USEPACKAGES end                       %%
%%%%%%%%%%%%%%%%%%%%%%%%%%%%%%%%%%%%%%%

%%%%%%%%%%%%%%%%%%%%%%%%%%%%%%%%%%%%%%%
%% Algorithm environment
%%%%%%%%%%%%%%%%%%%%%%%%%%%%%%%%%%%%%%%
\SetKwIF{Recv}{}{}{upon receiving}{do}{}{}{}
\SetKwBlock{Init}{initially do}{}
\SetKwProg{Function}{Function}{:}{}

% https://github.com/chrmatt/algpseudocodex/issues/3
\algnewcommand\algorithmicswitch{\textbf{switch}}%
\algnewcommand\algorithmiccase{\textbf{case}}
\algnewcommand\algorithmicof{\textbf{of}}
\algnewcommand\algorithmicotherwise{\texttt{otherwise} $\Rightarrow$}

\makeatletter
\algdef{SE}[SWITCH]{Switch}{EndSwitch}[1]{\algpx@startIndent\algpx@startCodeCommand\algorithmicswitch\ #1\ \algorithmicdo}{\algpx@endIndent\algpx@startCodeCommand\algorithmicend\ \algorithmicswitch}%
\algdef{SE}[CASE]{Case}{EndCase}[1]{\algpx@startIndent\algpx@startCodeCommand\algorithmiccase\ #1}{\algpx@endIndent\algpx@startCodeCommand\algorithmicend\ \algorithmiccase}%
\algdef{SE}[CASEOF]{CaseOf}{EndCaseOf}[1]{\algpx@startIndent\algpx@startCodeCommand\algorithmiccase\ #1 \algorithmicof}{\algpx@endIndent\algpx@startCodeCommand\algorithmicend\ \algorithmiccase}
\algdef{SE}[OTHERWISE]{Otherwise}{EndOtherwise}[0]{\algpx@startIndent\algpx@startCodeCommand\algorithmicotherwise}{\algpx@endIndent\algpx@startCodeCommand\algorithmicend\ \algorithmicotherwise}
\ifbool{algpx@noEnd}{%
  \algtext*{EndSwitch}%
  \algtext*{EndCase}%
  \algtext*{EndCaseOf}
  \algtext*{EndOtherwise}
  %
  % end indent line after (not before), to get correct y position for multiline text in last command
  \apptocmd{\EndSwitch}{\algpx@endIndent}{}{}%
  \apptocmd{\EndCase}{\algpx@endIndent}{}{}%
  \apptocmd{\EndCaseOf}{\algpx@endIndent}{}{}
  \apptocmd{\EndOtherwise}{\algpx@endIndent}{}{}
}{}%

\pretocmd{\Switch}{\algpx@endCodeCommand}{}{}
\pretocmd{\Case}{\algpx@endCodeCommand}{}{}
\pretocmd{\CaseOf}{\algpx@endCodeCommand}{}{}
\pretocmd{\Otherwise}{\algpx@endCodeCommand}{}{}

% for end commands that may not be printed, tell endCodeCommand whether we are using noEnd
\ifbool{algpx@noEnd}{%
  \pretocmd{\EndSwitch}{\algpx@endCodeCommand[1]}{}{}%
  \pretocmd{\EndCase}{\algpx@endCodeCommand[1]}{}{}
  \pretocmd{\EndCaseOf}{\algpx@endCodeCommand[1]}{}{}%
  \pretocmd{\EndOtherwise}{\algpx@endCodeCommand[1]}{}{}
}{%
  \pretocmd{\EndSwitch}{\algpx@endCodeCommand[0]}{}{}%
  \pretocmd{\EndCase}{\algpx@endCodeCommand[0]}{}{}%
  \pretocmd{\EndCaseOf}{\algpx@endCodeCommand[0]}{}{}
  \pretocmd{\EndOtherwise}{\algpx@endCodeCommand[0]}{}{}
}%
\makeatother
% % For algpseudocode
% \algnewcommand\algorithmicswitch{\textbf{switch}}
% \algnewcommand\algorithmiccase{\textbf{case}}
% \algnewcommand\algorithmiccaseof{\textbf{case}}
% \algnewcommand\algorithmicof{\textbf{of}}
% % New "environments"
% \algdef{SE}[SWITCH]{Switch}{EndSwitch}[1]{\algorithmicswitch\ #1\ \algorithmicdo}{\algorithmicend\ \algorithmicswitch}%
% \algdef{SE}[CASE]{Case}{EndCase}[1]{\algorithmiccase\ #1}{\algorithmicend\ \algorithmiccase}%
% \algtext*{EndSwitch}%
% \algtext*{EndCase}
% \algdef{SE}[CASEOF]{CaseOf}{EndCaseOf}[1]{\algorithmiccaseof\ #1 \algorithmicof}{\algorithmicend\ \algorithmiccaseof}
% \algtext*{EndCaseOf}



%\pdfcompresslevel0

% quoting from
% https://tex.stackexchange.com/questions/391726/the-quotation-environment
\NewDocumentCommand{\bywhom}{m}{% the Bourbaki trick
  {\nobreak\hfill\penalty50\hskip1em\null\nobreak
   \hfill\mbox{\normalfont(#1)}%
   \parfillskip=0pt \finalhyphendemerits=0 \par}%
}

\NewDocumentEnvironment{pquotation}{m}
  {\begin{quoting}[
     indentfirst=true,
     leftmargin=\parindent,
     rightmargin=\parindent]\itshape}
  {\bywhom{#1}\end{quoting}}

\indexsetup{othercode=\small}
\makeindex[columns=2,options={-s /media/wu/file/stuuudy/notes/index_style.ist},intoc]
\makeatletter
\def\@idxitem{\par\hangindent 0pt}
\makeatother


% \newcounter{dummy} \numberwithin{dummy}{section}
\newtheorem{dummy}{dummy}[section]
\theoremstyle{definition}
\newtheorem{definition}[dummy]{Definition}
\theoremstyle{plain}
\newtheorem{corollary}[dummy]{Corollary}
\newtheorem{lemma}[dummy]{Lemma}
\newtheorem{proposition}[dummy]{Proposition}
\newtheorem{theorem}[dummy]{Theorem}
\newtheorem{notation}[dummy]{Notation}
\newtheorem{conjecture}[dummy]{Conjecture}
\newtheorem{fact}[dummy]{Fact}
\newtheorem{warning}[dummy]{Warning}
\theoremstyle{definition}
\newtheorem{examplle}{Example}[section]
\theoremstyle{remark}
\newtheorem*{remark}{Remark}
\newtheorem{exercise}{Exercise}[subsection]
\newtheorem{problem}{Problem}[subsection]
\newtheorem{observation}{Observation}[section]
\newenvironment{claim}[1]{\par\noindent\textbf{Claim:}\space#1}{}

\makeatletter
\DeclareFontFamily{U}{tipa}{}
\DeclareFontShape{U}{tipa}{m}{n}{<->tipa10}{}
\newcommand{\arc@char}{{\usefont{U}{tipa}{m}{n}\symbol{62}}}%

\newcommand{\arc}[1]{\mathpalette\arc@arc{#1}}

\newcommand{\arc@arc}[2]{%
  \sbox0{$\m@th#1#2$}%
  \vbox{
    \hbox{\resizebox{\wd0}{\height}{\arc@char}}
    \nointerlineskip
    \box0
  }%
}
\makeatother

\setcounter{MaxMatrixCols}{20}
%%%%%%% ABS
\DeclarePairedDelimiter\abss{\lvert}{\rvert}%
\DeclarePairedDelimiter\normm{\lVert}{\rVert}%

% Swap the definition of \abs* and \norm*, so that \abs
% and \norm resizes the size of the brackets, and the
% starred version does not.
\makeatletter
\let\oldabs\abss
%\def\abs{\@ifstar{\oldabs}{\oldabs*}}
\newcommand{\abs}{\@ifstar{\oldabs}{\oldabs*}}
\newcommand{\norm}[1]{\left\lVert#1\right\rVert}
%\let\oldnorm\normm
%\def\norm{\@ifstar{\oldnorm}{\oldnorm*}}
%\renewcommand{norm}{\@ifstar{\oldnorm}{\oldnorm*}}
\makeatother

% \stackMath
% \newcommand\what[1]{%
% \savestack{\tmpbox}{\stretchto{%
%   \scaleto{%
%     \scalerel*[\widthof{\ensuremath{#1}}]{\kern-.6pt\bigwedge\kern-.6pt}%
%     {\rule[-\textheight/2]{1ex}{\textheight}}%WIDTH-LIMITED BIG WEDGE
%   }{\textheight}%
% }{0.5ex}}%
% \stackon[1pt]{#1}{\tmpbox}%
% }

% \newcommand\what[1]{\ThisStyle{%
%     \setbox0=\hbox{$\SavedStyle#1$}%
%     \stackengine{-1.0\ht0+.5pt}{$\SavedStyle#1$}{%
%       \stretchto{\scaleto{\SavedStyle\mkern.15mu\char'136}{2.6\wd0}}{1.4\ht0}%
%     }{O}{c}{F}{T}{S}%
%   }
% }

% \newcommand\wtilde[1]{\ThisStyle{%
%     \setbox0=\hbox{$\SavedStyle#1$}%
%     \stackengine{-.1\LMpt}{$\SavedStyle#1$}{%
%       \stretchto{\scaleto{\SavedStyle\mkern.2mu\AC}{.5150\wd0}}{.6\ht0}%
%     }{O}{c}{F}{T}{S}%
%   }
% }

% \newcommand\wbar[1]{\ThisStyle{%
%     \setbox0=\hbox{$\SavedStyle#1$}%
%     \stackengine{.5pt+\LMpt}{$\SavedStyle#1$}{%
%       \rule{\wd0}{\dimexpr.3\LMpt+.3pt}%
%     }{O}{c}{F}{T}{S}%
%   }
% }

\newcommand{\bl}[1] {\boldsymbol{#1}}
\newcommand{\Wt}[1] {\stackrel{\sim}{\smash{#1}\rule{0pt}{1.1ex}}}
\newcommand{\wt}[1] {\widetilde{#1}}
\newcommand{\tf}[1] {\textbf{#1}}

\newcommand{\wu}[1]{{\color{red} #1}}

%For boxed texts in align, use Aboxed{}
%otherwise use boxed{}

\DeclareMathSymbol{\widehatsym}{\mathord}{largesymbols}{"62}
\newcommand\lowerwidehatsym{%
  \text{\smash{\raisebox{-1.3ex}{%
    $\widehatsym$}}}}
\newcommand\fixwidehat[1]{%
  \mathchoice
    {\accentset{\displaystyle\lowerwidehatsym}{#1}}
    {\accentset{\textstyle\lowerwidehatsym}{#1}}
    {\accentset{\scriptstyle\lowerwidehatsym}{#1}}
    {\accentset{\scriptscriptstyle\lowerwidehatsym}{#1}}
  }


\newcommand{\cupdot}{\mathbin{\dot{\cup}}}
\newcommand{\bigcupdot}{\mathop{\dot{\bigcup}}}

\usepackage{graphicx}

\usepackage[toc,page]{appendix}

% text on arrow for xRightarrow
\makeatletter
%\newcommand{\xRightarrow}[2][]{\ext@arrow 0359\Rightarrowfill@{#1}{#2}}
\makeatother

% Arbitrary long arrow
\newcommand{\Rarrow}[1]{%
\parbox{#1}{\tikz{\draw[->](0,0)--(#1,0);}}
}

\newcommand{\LRarrow}[1]{%
\parbox{#1}{\tikz{\draw[<->](0,0)--(#1,0);}}
}


\makeatletter
\providecommand*{\rmodels}{%
  \mathrel{%
    \mathpalette\@rmodels\models
  }%
}
\newcommand*{\@rmodels}[2]{%
  \reflectbox{$\m@th#1#2$}%
}
\makeatother

% Roman numerals
\makeatletter
\newcommand*{\rom}[1]{\expandafter\@slowromancap\romannumeral #1@}
\makeatother
% \\def \\b\([a-zA-Z]\) {\\boldsymbol{[a-zA-z]}}
% \\DeclareMathOperator{\\b\1}{\\textbf{\1}}

\DeclareMathOperator*{\argmin}{arg\,min}
\DeclareMathOperator*{\argmax}{arg\,max}

\DeclareMathOperator{\bone}{\textbf{1}}
\DeclareMathOperator{\bx}{\textbf{x}}
\DeclareMathOperator{\bz}{\textbf{z}}
\DeclareMathOperator{\bff}{\textbf{f}}
\DeclareMathOperator{\ba}{\textbf{a}}
\DeclareMathOperator{\bk}{\textbf{k}}
\DeclareMathOperator{\bs}{\textbf{s}}
\DeclareMathOperator{\bh}{\textbf{h}}
\DeclareMathOperator{\bc}{\textbf{c}}
\DeclareMathOperator{\br}{\textbf{r}}
\DeclareMathOperator{\bi}{\textbf{i}}
\DeclareMathOperator{\bj}{\textbf{j}}
\DeclareMathOperator{\bn}{\textbf{n}}
\DeclareMathOperator{\be}{\textbf{e}}
\DeclareMathOperator{\bo}{\textbf{o}}
\DeclareMathOperator{\bU}{\textbf{U}}
\DeclareMathOperator{\bL}{\textbf{L}}
\DeclareMathOperator{\bV}{\textbf{V}}
\def \bzero {\mathbf{0}}
\def \bbone {\mathbb{1}}
\def \btwo {\mathbf{2}}
\DeclareMathOperator{\bv}{\textbf{v}}
\DeclareMathOperator{\bp}{\textbf{p}}
\DeclareMathOperator{\bI}{\textbf{I}}
\def \dbI {\dot{\bI}}
\DeclareMathOperator{\bM}{\textbf{M}}
\DeclareMathOperator{\bN}{\textbf{N}}
\DeclareMathOperator{\bK}{\textbf{K}}
\DeclareMathOperator{\bt}{\textbf{t}}
\DeclareMathOperator{\bb}{\textbf{b}}
\DeclareMathOperator{\bA}{\textbf{A}}
\DeclareMathOperator{\bX}{\textbf{X}}
\DeclareMathOperator{\bu}{\textbf{u}}
\DeclareMathOperator{\bS}{\textbf{S}}
\DeclareMathOperator{\bZ}{\textbf{Z}}
\DeclareMathOperator{\bJ}{\textbf{J}}
\DeclareMathOperator{\by}{\textbf{y}}
\DeclareMathOperator{\bw}{\textbf{w}}
\DeclareMathOperator{\bT}{\textbf{T}}
\DeclareMathOperator{\bF}{\textbf{F}}
\DeclareMathOperator{\bmm}{\textbf{m}}
\DeclareMathOperator{\bW}{\textbf{W}}
\DeclareMathOperator{\bR}{\textbf{R}}
\DeclareMathOperator{\bC}{\textbf{C}}
\DeclareMathOperator{\bD}{\textbf{D}}
\DeclareMathOperator{\bE}{\textbf{E}}
\DeclareMathOperator{\bQ}{\textbf{Q}}
\DeclareMathOperator{\bP}{\textbf{P}}
\DeclareMathOperator{\bY}{\textbf{Y}}
\DeclareMathOperator{\bH}{\textbf{H}}
\DeclareMathOperator{\bB}{\textbf{B}}
\DeclareMathOperator{\bG}{\textbf{G}}
\def \blambda {\symbf{\lambda}}
\def \boldeta {\symbf{\eta}}
\def \balpha {\symbf{\alpha}}
\def \btau {\symbf{\tau}}
\def \bbeta {\symbf{\beta}}
\def \bgamma {\symbf{\gamma}}
\def \bxi {\symbf{\xi}}
\def \bLambda {\symbf{\Lambda}}
\def \bGamma {\symbf{\Gamma}}

\newcommand{\bto}{{\boldsymbol{\to}}}
\newcommand{\Ra}{\Rightarrow}
\newcommand{\xrsa}[1]{\overset{#1}{\rightsquigarrow}}
\newcommand{\xlsa}[1]{\overset{#1}{\leftsquigarrow}}
\newcommand\und[1]{\underline{#1}}
\newcommand\ove[1]{\overline{#1}}
%\def \concat {\verb|^|}
\def \bPhi {\mbfPhi}
\def \btheta {\mbftheta}
\def \bTheta {\mbfTheta}
\def \bmu {\mbfmu}
\def \bphi {\mbfphi}
\def \bSigma {\mbfSigma}
\def \la {\langle}
\def \ra {\rangle}

\def \caln {\mathcal{N}}
\def \dissum {\displaystyle\Sigma}
\def \dispro {\displaystyle\prod}

\def \caret {\verb!^!}

\def \A {\mathbb{A}}
\def \B {\mathbb{B}}
\def \C {\mathbb{C}}
\def \D {\mathbb{D}}
\def \E {\mathbb{E}}
\def \F {\mathbb{F}}
\def \G {\mathbb{G}}
\def \H {\mathbb{H}}
\def \I {\mathbb{I}}
\def \J {\mathbb{J}}
\def \K {\mathbb{K}}
\def \L {\mathbb{L}}
\def \M {\mathbb{M}}
\def \N {\mathbb{N}}
\def \O {\mathbb{O}}
\def \P {\mathbb{P}}
\def \Q {\mathbb{Q}}
\def \R {\mathbb{R}}
\def \S {\mathbb{S}}
\def \T {\mathbb{T}}
\def \U {\mathbb{U}}
\def \V {\mathbb{V}}
\def \W {\mathbb{W}}
\def \X {\mathbb{X}}
\def \Y {\mathbb{Y}}
\def \Z {\mathbb{Z}}

\def \cala {\mathcal{A}}
\def \cale {\mathcal{E}}
\def \calb {\mathcal{B}}
\def \calq {\mathcal{Q}}
\def \calp {\mathcal{P}}
\def \cals {\mathcal{S}}
\def \calx {\mathcal{X}}
\def \caly {\mathcal{Y}}
\def \calg {\mathcal{G}}
\def \cald {\mathcal{D}}
\def \caln {\mathcal{N}}
\def \calr {\mathcal{R}}
\def \calt {\mathcal{T}}
\def \calm {\mathcal{M}}
\def \calw {\mathcal{W}}
\def \calc {\mathcal{C}}
\def \calv {\mathcal{V}}
\def \calf {\mathcal{F}}
\def \calk {\mathcal{K}}
\def \call {\mathcal{L}}
\def \calu {\mathcal{U}}
\def \calo {\mathcal{O}}
\def \calh {\mathcal{H}}
\def \cali {\mathcal{I}}
\def \calj {\mathcal{J}}

\def \bcup {\bigcup}

% set theory

\def \zfcc {\textbf{ZFC}^-}
\def \BGC {\textbf{BGC}}
\def \BG {\textbf{BG}}
\def \ac  {\textbf{AC}}
\def \gl  {\textbf{L }}
\def \gll {\textbf{L}}
\newcommand{\zfm}{$\textbf{ZF}^-$}

\def \ZFm {\text{ZF}^-}
\def \ZFCm {\text{ZFC}^-}
\DeclareMathOperator{\WF}{WF}
\DeclareMathOperator{\On}{On}
\def \on {\textbf{On }}
\def \cm {\textbf{M }}
\def \cn {\textbf{N }}
\def \cv {\textbf{V }}
\def \zc {\textbf{ZC }}
\def \zcm {\textbf{ZC}}
\def \zff {\textbf{ZF}}
\def \wfm {\textbf{WF}}
\def \onm {\textbf{On}}
\def \cmm {\textbf{M}}
\def \cnm {\textbf{N}}
\def \cvm {\textbf{V}}

\renewcommand{\restriction}{\mathord{\upharpoonright}}
%% another restriction
\newcommand\restr[2]{{% we make the whole thing an ordinary symbol
  \left.\kern-\nulldelimiterspace % automatically resize the bar with \right
  #1 % the function
  \vphantom{\big|} % pretend it's a little taller at normal size
  \right|_{#2} % this is the delimiter
  }}

\def \pred {\text{pred}}

\def \rank {\text{rank}}
\def \Con {\text{Con}}
\def \deff {\text{Def}}


\def \uin {\underline{\in}}
\def \oin {\overline{\in}}
\def \uR {\underline{R}}
\def \oR {\overline{R}}
\def \uP {\underline{P}}
\def \oP {\overline{P}}

\def \dsum {\displaystyle\sum}

\def \Ra {\Rightarrow}

\def \e {\enspace}

\def \sgn {\operatorname{sgn}}
\def \gen {\operatorname{gen}}
\def \Hom {\operatorname{Hom}}
\def \hom {\operatorname{hom}}
\def \Sub {\operatorname{Sub}}

\def \supp {\operatorname{supp}}

\def \epiarrow {\twoheadarrow}
\def \monoarrow {\rightarrowtail}
\def \rrarrow {\rightrightarrows}

% \def \minus {\text{-}}
% \newcommand{\minus}{\scalebox{0.75}[1.0]{$-$}}
% \DeclareUnicodeCharacter{002D}{\minus}


\def \tril {\triangleleft}

\def \ISigma {\text{I}\Sigma}
\def \IDelta {\text{I}\Delta}
\def \IPi {\text{I}\Pi}
\def \ACF {\textsf{ACF}}
\def \pCF {\textit{p}\text{CF}}
\def \ACVF {\textsf{ACVF}}
\def \HLR {\textsf{HLR}}
\def \OAG {\textsf{OAG}}
\def \RCF {\textsf{RCF}}
\DeclareMathOperator{\GL}{GL}
\DeclareMathOperator{\PGL}{PGL}
\DeclareMathOperator{\SL}{SL}
\DeclareMathOperator{\Inv}{Inv}
\DeclareMathOperator{\res}{res}
\DeclareMathOperator{\Sym}{Sym}
%\DeclareMathOperator{\char}{char}
\def \equal {=}

\def \degree {\text{degree}}
\def \app {\text{App}}
\def \FV {\text{FV}}
\def \conv {\text{conv}}
\def \cont {\text{cont}}
\DeclareMathOperator{\cl}{\text{cl}}
\DeclareMathOperator{\trcl}{\text{trcl}}
\DeclareMathOperator{\sg}{sg}
\DeclareMathOperator{\trdeg}{trdeg}
\def \Ord {\text{Ord}}

\DeclareMathOperator{\cf}{cf}
\DeclareMathOperator{\zfc}{ZFC}

%\DeclareMathOperator{\Th}{Th}
%\def \th {\text{Th}}
% \newcommand{\th}{\text{Th}}
\DeclareMathOperator{\type}{type}
\DeclareMathOperator{\zf}{\textbf{ZF}}
\def \fa {\mathfrak{a}}
\def \fb {\mathfrak{b}}
\def \fc {\mathfrak{c}}
\def \fd {\mathfrak{d}}
\def \fe {\mathfrak{e}}
\def \ff {\mathfrak{f}}
\def \fg {\mathfrak{g}}
\def \fh {\mathfrak{h}}
%\def \fi {\mathfrak{i}}
\def \fj {\mathfrak{j}}
\def \fk {\mathfrak{k}}
\def \fl {\mathfrak{l}}
\def \fm {\mathfrak{m}}
\def \fn {\mathfrak{n}}
\def \fo {\mathfrak{o}}
\def \fp {\mathfrak{p}}
\def \fq {\mathfrak{q}}
\def \fr {\mathfrak{r}}
\def \fs {\mathfrak{s}}
\def \ft {\mathfrak{t}}
\def \fu {\mathfrak{u}}
\def \fv {\mathfrak{v}}
\def \fw {\mathfrak{w}}
\def \fx {\mathfrak{x}}
\def \fy {\mathfrak{y}}
\def \fz {\mathfrak{z}}
\def \fA {\mathfrak{A}}
\def \fB {\mathfrak{B}}
\def \fC {\mathfrak{C}}
\def \fD {\mathfrak{D}}
\def \fE {\mathfrak{E}}
\def \fF {\mathfrak{F}}
\def \fG {\mathfrak{G}}
\def \fH {\mathfrak{H}}
\def \fI {\mathfrak{I}}
\def \fJ {\mathfrak{J}}
\def \fK {\mathfrak{K}}
\def \fL {\mathfrak{L}}
\def \fM {\mathfrak{M}}
\def \fN {\mathfrak{N}}
\def \fO {\mathfrak{O}}
\def \fP {\mathfrak{P}}
\def \fQ {\mathfrak{Q}}
\def \fR {\mathfrak{R}}
\def \fS {\mathfrak{S}}
\def \fT {\mathfrak{T}}
\def \fU {\mathfrak{U}}
\def \fV {\mathfrak{V}}
\def \fW {\mathfrak{W}}
\def \fX {\mathfrak{X}}
\def \fY {\mathfrak{Y}}
\def \fZ {\mathfrak{Z}}

\def \sfA {\textsf{A}}
\def \sfB {\textsf{B}}
\def \sfC {\textsf{C}}
\def \sfD {\textsf{D}}
\def \sfE {\textsf{E}}
\def \sfF {\textsf{F}}
\def \sfG {\textsf{G}}
\def \sfH {\textsf{H}}
\def \sfI {\textsf{I}}
\def \sfJ {\textsf{J}}
\def \sfK {\textsf{K}}
\def \sfL {\textsf{L}}
\def \sfM {\textsf{M}}
\def \sfN {\textsf{N}}
\def \sfO {\textsf{O}}
\def \sfP {\textsf{P}}
\def \sfQ {\textsf{Q}}
\def \sfR {\textsf{R}}
\def \sfS {\textsf{S}}
\def \sfT {\textsf{T}}
\def \sfU {\textsf{U}}
\def \sfV {\textsf{V}}
\def \sfW {\textsf{W}}
\def \sfX {\textsf{X}}
\def \sfY {\textsf{Y}}
\def \sfZ {\textsf{Z}}
\def \sfa {\textsf{a}}
\def \sfb {\textsf{b}}
\def \sfc {\textsf{c}}
\def \sfd {\textsf{d}}
\def \sfe {\textsf{e}}
\def \sff {\textsf{f}}
\def \sfg {\textsf{g}}
\def \sfh {\textsf{h}}
\def \sfi {\textsf{i}}
\def \sfj {\textsf{j}}
\def \sfk {\textsf{k}}
\def \sfl {\textsf{l}}
\def \sfm {\textsf{m}}
\def \sfn {\textsf{n}}
\def \sfo {\textsf{o}}
\def \sfp {\textsf{p}}
\def \sfq {\textsf{q}}
\def \sfr {\textsf{r}}
\def \sfs {\textsf{s}}
\def \sft {\textsf{t}}
\def \sfu {\textsf{u}}
\def \sfv {\textsf{v}}
\def \sfw {\textsf{w}}
\def \sfx {\textsf{x}}
\def \sfy {\textsf{y}}
\def \sfz {\textsf{z}}

\def \ttA {\texttt{A}}
\def \ttB {\texttt{B}}
\def \ttC {\texttt{C}}
\def \ttD {\texttt{D}}
\def \ttE {\texttt{E}}
\def \ttF {\texttt{F}}
\def \ttG {\texttt{G}}
\def \ttH {\texttt{H}}
\def \ttI {\texttt{I}}
\def \ttJ {\texttt{J}}
\def \ttK {\texttt{K}}
\def \ttL {\texttt{L}}
\def \ttM {\texttt{M}}
\def \ttN {\texttt{N}}
\def \ttO {\texttt{O}}
\def \ttP {\texttt{P}}
\def \ttQ {\texttt{Q}}
\def \ttR {\texttt{R}}
\def \ttS {\texttt{S}}
\def \ttT {\texttt{T}}
\def \ttU {\texttt{U}}
\def \ttV {\texttt{V}}
\def \ttW {\texttt{W}}
\def \ttX {\texttt{X}}
\def \ttY {\texttt{Y}}
\def \ttZ {\texttt{Z}}
\def \tta {\texttt{a}}
\def \ttb {\texttt{b}}
\def \ttc {\texttt{c}}
\def \ttd {\texttt{d}}
\def \tte {\texttt{e}}
\def \ttf {\texttt{f}}
\def \ttg {\texttt{g}}
\def \tth {\texttt{h}}
\def \tti {\texttt{i}}
\def \ttj {\texttt{j}}
\def \ttk {\texttt{k}}
\def \ttl {\texttt{l}}
\def \ttm {\texttt{m}}
\def \ttn {\texttt{n}}
\def \tto {\texttt{o}}
\def \ttp {\texttt{p}}
\def \ttq {\texttt{q}}
\def \ttr {\texttt{r}}
\def \tts {\texttt{s}}
\def \ttt {\texttt{t}}
\def \ttu {\texttt{u}}
\def \ttv {\texttt{v}}
\def \ttw {\texttt{w}}
\def \ttx {\texttt{x}}
\def \tty {\texttt{y}}
\def \ttz {\texttt{z}}

\def \bara {\bbar{a}}
\def \barb {\bbar{b}}
\def \barc {\bbar{c}}
\def \bard {\bbar{d}}
\def \bare {\bbar{e}}
\def \barf {\bbar{f}}
\def \barg {\bbar{g}}
\def \barh {\bbar{h}}
\def \bari {\bbar{i}}
\def \barj {\bbar{j}}
\def \bark {\bbar{k}}
\def \barl {\bbar{l}}
\def \barm {\bbar{m}}
\def \barn {\bbar{n}}
\def \baro {\bbar{o}}
\def \barp {\bbar{p}}
\def \barq {\bbar{q}}
\def \barr {\bbar{r}}
\def \bars {\bbar{s}}
\def \bart {\bbar{t}}
\def \baru {\bbar{u}}
\def \barv {\bbar{v}}
\def \barw {\bbar{w}}
\def \barx {\bbar{x}}
\def \bary {\bbar{y}}
\def \barz {\bbar{z}}
\def \barA {\bbar{A}}
\def \barB {\bbar{B}}
\def \barC {\bbar{C}}
\def \barD {\bbar{D}}
\def \barE {\bbar{E}}
\def \barF {\bbar{F}}
\def \barG {\bbar{G}}
\def \barH {\bbar{H}}
\def \barI {\bbar{I}}
\def \barJ {\bbar{J}}
\def \barK {\bbar{K}}
\def \barL {\bbar{L}}
\def \barM {\bbar{M}}
\def \barN {\bbar{N}}
\def \barO {\bbar{O}}
\def \barP {\bbar{P}}
\def \barQ {\bbar{Q}}
\def \barR {\bbar{R}}
\def \barS {\bbar{S}}
\def \barT {\bbar{T}}
\def \barU {\bbar{U}}
\def \barVV {\bbar{V}}
\def \barW {\bbar{W}}
\def \barX {\bbar{X}}
\def \barY {\bbar{Y}}
\def \barZ {\bbar{Z}}

\def \baralpha {\bbar{\alpha}}
\def \bartau {\bbar{\tau}}
\def \barsigma {\bbar{\sigma}}
\def \barzeta {\bbar{\zeta}}

\def \hata {\hat{a}}
\def \hatb {\hat{b}}
\def \hatc {\hat{c}}
\def \hatd {\hat{d}}
\def \hate {\hat{e}}
\def \hatf {\hat{f}}
\def \hatg {\hat{g}}
\def \hath {\hat{h}}
\def \hati {\hat{i}}
\def \hatj {\hat{j}}
\def \hatk {\hat{k}}
\def \hatl {\hat{l}}
\def \hatm {\hat{m}}
\def \hatn {\hat{n}}
\def \hato {\hat{o}}
\def \hatp {\hat{p}}
\def \hatq {\hat{q}}
\def \hatr {\hat{r}}
\def \hats {\hat{s}}
\def \hatt {\hat{t}}
\def \hatu {\hat{u}}
\def \hatv {\hat{v}}
\def \hatw {\hat{w}}
\def \hatx {\hat{x}}
\def \haty {\hat{y}}
\def \hatz {\hat{z}}
\def \hatA {\hat{A}}
\def \hatB {\hat{B}}
\def \hatC {\hat{C}}
\def \hatD {\hat{D}}
\def \hatE {\hat{E}}
\def \hatF {\hat{F}}
\def \hatG {\hat{G}}
\def \hatH {\hat{H}}
\def \hatI {\hat{I}}
\def \hatJ {\hat{J}}
\def \hatK {\hat{K}}
\def \hatL {\hat{L}}
\def \hatM {\hat{M}}
\def \hatN {\hat{N}}
\def \hatO {\hat{O}}
\def \hatP {\hat{P}}
\def \hatQ {\hat{Q}}
\def \hatR {\hat{R}}
\def \hatS {\hat{S}}
\def \hatT {\hat{T}}
\def \hatU {\hat{U}}
\def \hatVV {\hat{V}}
\def \hatW {\hat{W}}
\def \hatX {\hat{X}}
\def \hatY {\hat{Y}}
\def \hatZ {\hat{Z}}

\def \hatphi {\hat{\phi}}

\def \barfM {\bbar{\fM}}
\def \barfN {\bbar{\fN}}

\def \tila {\tilde{a}}
\def \tilb {\tilde{b}}
\def \tilc {\tilde{c}}
\def \tild {\tilde{d}}
\def \tile {\tilde{e}}
\def \tilf {\tilde{f}}
\def \tilg {\tilde{g}}
\def \tilh {\tilde{h}}
\def \tili {\tilde{i}}
\def \tilj {\tilde{j}}
\def \tilk {\tilde{k}}
\def \till {\tilde{l}}
\def \tilm {\tilde{m}}
\def \tiln {\tilde{n}}
\def \tilo {\tilde{o}}
\def \tilp {\tilde{p}}
\def \tilq {\tilde{q}}
\def \tilr {\tilde{r}}
\def \tils {\tilde{s}}
\def \tilt {\tilde{t}}
\def \tilu {\tilde{u}}
\def \tilv {\tilde{v}}
\def \tilw {\tilde{w}}
\def \tilx {\tilde{x}}
\def \tily {\tilde{y}}
\def \tilz {\tilde{z}}
\def \tilA {\tilde{A}}
\def \tilB {\tilde{B}}
\def \tilC {\tilde{C}}
\def \tilD {\tilde{D}}
\def \tilE {\tilde{E}}
\def \tilF {\tilde{F}}
\def \tilG {\tilde{G}}
\def \tilH {\tilde{H}}
\def \tilI {\tilde{I}}
\def \tilJ {\tilde{J}}
\def \tilK {\tilde{K}}
\def \tilL {\tilde{L}}
\def \tilM {\tilde{M}}
\def \tilN {\tilde{N}}
\def \tilO {\tilde{O}}
\def \tilP {\tilde{P}}
\def \tilQ {\tilde{Q}}
\def \tilR {\tilde{R}}
\def \tilS {\tilde{S}}
\def \tilT {\tilde{T}}
\def \tilU {\tilde{U}}
\def \tilVV {\tilde{V}}
\def \tilW {\tilde{W}}
\def \tilX {\tilde{X}}
\def \tilY {\tilde{Y}}
\def \tilZ {\tilde{Z}}

\def \tilalpha {\tilde{\alpha}}
\def \tilPhi {\tilde{\Phi}}

\def \barnu {\bar{\nu}}
\def \barrho {\bar{\rho}}
%\DeclareMathOperator{\ker}{ker}
\DeclareMathOperator{\im}{im}

\DeclareMathOperator{\Inn}{Inn}
\DeclareMathOperator{\rel}{rel}
\def \dote {\stackrel{\cdot}=}
%\DeclareMathOperator{\AC}{\textbf{AC}}
\DeclareMathOperator{\cod}{cod}
\DeclareMathOperator{\dom}{dom}
\DeclareMathOperator{\card}{card}
\DeclareMathOperator{\ran}{ran}
\DeclareMathOperator{\textd}{d}
\DeclareMathOperator{\td}{d}
\DeclareMathOperator{\id}{id}
\DeclareMathOperator{\LT}{LT}
\DeclareMathOperator{\Mat}{Mat}
\DeclareMathOperator{\Eq}{Eq}
\DeclareMathOperator{\irr}{irr}
\DeclareMathOperator{\Fr}{Fr}
\DeclareMathOperator{\Gal}{Gal}
\DeclareMathOperator{\lcm}{lcm}
\DeclareMathOperator{\alg}{\text{alg}}
\DeclareMathOperator{\Th}{Th}
%\DeclareMathOperator{\deg}{deg}


% \varprod
\DeclareSymbolFont{largesymbolsA}{U}{txexa}{m}{n}
\DeclareMathSymbol{\varprod}{\mathop}{largesymbolsA}{16}
% \DeclareMathSymbol{\tonm}{\boldsymbol{\to}\textbf{Nm}}
\def \tonm {\bto\textbf{Nm}}
\def \tohm {\bto\textbf{Hm}}

% Category theory
\DeclareMathOperator{\ob}{ob}
\DeclareMathOperator{\Ab}{\textbf{Ab}}
\DeclareMathOperator{\Alg}{\textbf{Alg}}
\DeclareMathOperator{\Rng}{\textbf{Rng}}
\DeclareMathOperator{\Sets}{\textbf{Sets}}
\DeclareMathOperator{\Set}{\textbf{Set}}
\DeclareMathOperator{\Grp}{\textbf{Grp}}
\DeclareMathOperator{\Met}{\textbf{Met}}
\DeclareMathOperator{\BA}{\textbf{BA}}
\DeclareMathOperator{\Mon}{\textbf{Mon}}
\DeclareMathOperator{\Top}{\textbf{Top}}
\DeclareMathOperator{\hTop}{\textbf{hTop}}
\DeclareMathOperator{\HTop}{\textbf{HTop}}
\DeclareMathOperator{\Aut}{\text{Aut}}
\DeclareMathOperator{\RMod}{R-\textbf{Mod}}
\DeclareMathOperator{\RAlg}{R-\textbf{Alg}}
\DeclareMathOperator{\LF}{LF}
\DeclareMathOperator{\op}{op}
\DeclareMathOperator{\Rings}{\textbf{Rings}}
\DeclareMathOperator{\Ring}{\textbf{Ring}}
\DeclareMathOperator{\Groups}{\textbf{Groups}}
\DeclareMathOperator{\Group}{\textbf{Group}}
\DeclareMathOperator{\ev}{ev}
% Algebraic Topology
\DeclareMathOperator{\obj}{obj}
\DeclareMathOperator{\Spec}{Spec}
\DeclareMathOperator{\spec}{spec}
% Model theory
\DeclareMathOperator*{\ind}{\raise0.2ex\hbox{\ooalign{\hidewidth$\vert$\hidewidth\cr\raise-0.9ex\hbox{$\smile$}}}}
\def\nind{\cancel{\ind}}
\DeclareMathOperator{\acl}{acl}
\DeclareMathOperator{\tspan}{span}
\DeclareMathOperator{\acleq}{acl^{\eq}}
\DeclareMathOperator{\Av}{Av}
\DeclareMathOperator{\ded}{ded}
\DeclareMathOperator{\EM}{EM}
\DeclareMathOperator{\dcl}{dcl}
\DeclareMathOperator{\Ext}{Ext}
\DeclareMathOperator{\eq}{eq}
\DeclareMathOperator{\ER}{ER}
\DeclareMathOperator{\tp}{tp}
\DeclareMathOperator{\stp}{stp}
\DeclareMathOperator{\qftp}{qftp}
\DeclareMathOperator{\Diag}{Diag}
\DeclareMathOperator{\MD}{MD}
\DeclareMathOperator{\MR}{MR}
\DeclareMathOperator{\RM}{RM}
\DeclareMathOperator{\el}{el}
\DeclareMathOperator{\depth}{depth}
\DeclareMathOperator{\ZFC}{ZFC}
\DeclareMathOperator{\GCH}{GCH}
\DeclareMathOperator{\Inf}{Inf}
\DeclareMathOperator{\Pow}{Pow}
\DeclareMathOperator{\ZF}{ZF}
\DeclareMathOperator{\CH}{CH}
\def \FO {\text{FO}}
\DeclareMathOperator{\fin}{fin}
\DeclareMathOperator{\qr}{qr}
\DeclareMathOperator{\Mod}{Mod}
\DeclareMathOperator{\Def}{Def}
\DeclareMathOperator{\TC}{TC}
\DeclareMathOperator{\KH}{KH}
\DeclareMathOperator{\Part}{Part}
\DeclareMathOperator{\Infset}{\textsf{Infset}}
\DeclareMathOperator{\DLO}{\textsf{DLO}}
\DeclareMathOperator{\PA}{\textsf{PA}}
\DeclareMathOperator{\DAG}{\textsf{DAG}}
\DeclareMathOperator{\ODAG}{\textsf{ODAG}}
\DeclareMathOperator{\sfMod}{\textsf{Mod}}
\DeclareMathOperator{\AbG}{\textsf{AbG}}
\DeclareMathOperator{\sfACF}{\textsf{ACF}}
\DeclareMathOperator{\DCF}{\textsf{DCF}}
% Computability Theorem
\DeclareMathOperator{\Tot}{Tot}
\DeclareMathOperator{\graph}{graph}
\DeclareMathOperator{\Fin}{Fin}
\DeclareMathOperator{\Cof}{Cof}
\DeclareMathOperator{\lh}{lh}
% Commutative Algebra
\DeclareMathOperator{\ord}{ord}
\DeclareMathOperator{\Idem}{Idem}
\DeclareMathOperator{\zdiv}{z.div}
\DeclareMathOperator{\Frac}{Frac}
\DeclareMathOperator{\rad}{rad}
\DeclareMathOperator{\nil}{nil}
\DeclareMathOperator{\Ann}{Ann}
\DeclareMathOperator{\End}{End}
\DeclareMathOperator{\coim}{coim}
\DeclareMathOperator{\coker}{coker}
\DeclareMathOperator{\Bil}{Bil}
\DeclareMathOperator{\Tril}{Tril}
\DeclareMathOperator{\tchar}{char}
\DeclareMathOperator{\tbd}{bd}

% Topology
\DeclareMathOperator{\diam}{diam}
\newcommand{\interior}[1]{%
  {\kern0pt#1}^{\mathrm{o}}%
}

\DeclareMathOperator*{\bigdoublewedge}{\bigwedge\mkern-15mu\bigwedge}
\DeclareMathOperator*{\bigdoublevee}{\bigvee\mkern-15mu\bigvee}

% \makeatletter
% \newcommand{\vect}[1]{%
%   \vbox{\m@th \ialign {##\crcr
%   \vectfill\crcr\noalign{\kern-\p@ \nointerlineskip}
%   $\hfil\displaystyle{#1}\hfil$\crcr}}}
% \def\vectfill{%
%   $\m@th\smash-\mkern-7mu%
%   \cleaders\hbox{$\mkern-2mu\smash-\mkern-2mu$}\hfill
%   \mkern-7mu\raisebox{-3.81pt}[\p@][\p@]{$\mathord\mathchar"017E$}$}

% \newcommand{\amsvect}{%
%   \mathpalette {\overarrow@\vectfill@}}
% \def\vectfill@{\arrowfill@\relbar\relbar{\raisebox{-3.81pt}[\p@][\p@]{$\mathord\mathchar"017E$}}}

% \newcommand{\amsvectb}{%
% \newcommand{\vect}{%
%   \mathpalette {\overarrow@\vectfillb@}}
% \newcommand{\vecbar}{%
%   \scalebox{0.8}{$\relbar$}}
% \def\vectfillb@{\arrowfill@\vecbar\vecbar{\raisebox{-4.35pt}[\p@][\p@]{$\mathord\mathchar"017E$}}}
% \makeatother
% \bigtimes

\DeclareFontFamily{U}{mathx}{\hyphenchar\font45}
\DeclareFontShape{U}{mathx}{m}{n}{
      <5> <6> <7> <8> <9> <10>
      <10.95> <12> <14.4> <17.28> <20.74> <24.88>
      mathx10
      }{}
\DeclareSymbolFont{mathx}{U}{mathx}{m}{n}
\DeclareMathSymbol{\bigtimes}{1}{mathx}{"91}
% \odiv
\DeclareFontFamily{U}{matha}{\hyphenchar\font45}
\DeclareFontShape{U}{matha}{m}{n}{
      <5> <6> <7> <8> <9> <10> gen * matha
      <10.95> matha10 <12> <14.4> <17.28> <20.74> <24.88> matha12
      }{}
\DeclareSymbolFont{matha}{U}{matha}{m}{n}
\DeclareMathSymbol{\odiv}         {2}{matha}{"63}


\newcommand\subsetsim{\mathrel{%
  \ooalign{\raise0.2ex\hbox{\scalebox{0.9}{$\subset$}}\cr\hidewidth\raise-0.85ex\hbox{\scalebox{0.9}{$\sim$}}\hidewidth\cr}}}
\newcommand\simsubset{\mathrel{%
  \ooalign{\raise-0.2ex\hbox{\scalebox{0.9}{$\subset$}}\cr\hidewidth\raise0.75ex\hbox{\scalebox{0.9}{$\sim$}}\hidewidth\cr}}}

\newcommand\simsubsetsim{\mathrel{%
  \ooalign{\raise0ex\hbox{\scalebox{0.8}{$\subset$}}\cr\hidewidth\raise1ex\hbox{\scalebox{0.75}{$\sim$}}\hidewidth\cr\raise-0.95ex\hbox{\scalebox{0.8}{$\sim$}}\cr\hidewidth}}}
\newcommand{\stcomp}[1]{{#1}^{\mathsf{c}}}

\setlength{\baselineskip}{0.5in}

\stackMath
\newcommand\yrightarrow[2][]{\mathrel{%
  \setbox2=\hbox{\stackon{\scriptstyle#1}{\scriptstyle#2}}%
  \stackunder[0pt]{%
    \xrightarrow{\makebox[\dimexpr\wd2\relax]{$\scriptstyle#2$}}%
  }{%
   \scriptstyle#1\,%
  }%
}}
\newcommand\yleftarrow[2][]{\mathrel{%
  \setbox2=\hbox{\stackon{\scriptstyle#1}{\scriptstyle#2}}%
  \stackunder[0pt]{%
    \xleftarrow{\makebox[\dimexpr\wd2\relax]{$\scriptstyle#2$}}%
  }{%
   \scriptstyle#1\,%
  }%
}}
\newcommand\yRightarrow[2][]{\mathrel{%
  \setbox2=\hbox{\stackon{\scriptstyle#1}{\scriptstyle#2}}%
  \stackunder[0pt]{%
    \xRightarrow{\makebox[\dimexpr\wd2\relax]{$\scriptstyle#2$}}%
  }{%
   \scriptstyle#1\,%
  }%
}}
\newcommand\yLeftarrow[2][]{\mathrel{%
  \setbox2=\hbox{\stackon{\scriptstyle#1}{\scriptstyle#2}}%
  \stackunder[0pt]{%
    \xLeftarrow{\makebox[\dimexpr\wd2\relax]{$\scriptstyle#2$}}%
  }{%
   \scriptstyle#1\,%
  }%
}}

\newcommand\altxrightarrow[2][0pt]{\mathrel{\ensurestackMath{\stackengine%
  {\dimexpr#1-7.5pt}{\xrightarrow{\phantom{#2}}}{\scriptstyle\!#2\,}%
  {O}{c}{F}{F}{S}}}}
\newcommand\altxleftarrow[2][0pt]{\mathrel{\ensurestackMath{\stackengine%
  {\dimexpr#1-7.5pt}{\xleftarrow{\phantom{#2}}}{\scriptstyle\!#2\,}%
  {O}{c}{F}{F}{S}}}}

\newenvironment{bsm}{% % short for 'bracketed small matrix'
  \left[ \begin{smallmatrix} }{%
  \end{smallmatrix} \right]}

\newenvironment{psm}{% % short for ' small matrix'
  \left( \begin{smallmatrix} }{%
  \end{smallmatrix} \right)}

\newcommand{\bbar}[1]{\mkern 1.5mu\overline{\mkern-1.5mu#1\mkern-1.5mu}\mkern 1.5mu}

\newcommand{\bigzero}{\mbox{\normalfont\Large\bfseries 0}}
\newcommand{\rvline}{\hspace*{-\arraycolsep}\vline\hspace*{-\arraycolsep}}

\font\zallman=Zallman at 40pt
\font\elzevier=Elzevier at 40pt

\newcommand\isoto{\stackrel{\textstyle\sim}{\smash{\longrightarrow}\rule{0pt}{0.4ex}}}
\newcommand\embto{\stackrel{\textstyle\prec}{\smash{\longrightarrow}\rule{0pt}{0.4ex}}}

% from http://www.actual.world/resources/tex/doc/TikZ.pdf

\tikzset{
modal/.style={>=stealth’,shorten >=1pt,shorten <=1pt,auto,node distance=1.5cm,
semithick},
world/.style={circle,draw,minimum size=0.5cm,fill=gray!15},
point/.style={circle,draw,inner sep=0.5mm,fill=black},
reflexive above/.style={->,loop,looseness=7,in=120,out=60},
reflexive below/.style={->,loop,looseness=7,in=240,out=300},
reflexive left/.style={->,loop,looseness=7,in=150,out=210},
reflexive right/.style={->,loop,looseness=7,in=30,out=330}
}


\makeatletter
\newcommand*{\doublerightarrow}[2]{\mathrel{
  \settowidth{\@tempdima}{$\scriptstyle#1$}
  \settowidth{\@tempdimb}{$\scriptstyle#2$}
  \ifdim\@tempdimb>\@tempdima \@tempdima=\@tempdimb\fi
  \mathop{\vcenter{
    \offinterlineskip\ialign{\hbox to\dimexpr\@tempdima+1em{##}\cr
    \rightarrowfill\cr\noalign{\kern.5ex}
    \rightarrowfill\cr}}}\limits^{\!#1}_{\!#2}}}
\newcommand*{\triplerightarrow}[1]{\mathrel{
  \settowidth{\@tempdima}{$\scriptstyle#1$}
  \mathop{\vcenter{
    \offinterlineskip\ialign{\hbox to\dimexpr\@tempdima+1em{##}\cr
    \rightarrowfill\cr\noalign{\kern.5ex}
    \rightarrowfill\cr\noalign{\kern.5ex}
    \rightarrowfill\cr}}}\limits^{\!#1}}}
\makeatother

% $A\doublerightarrow{a}{bcdefgh}B$

% $A\triplerightarrow{d_0,d_1,d_2}B$

\def \uhr {\upharpoonright}
\def \rhu {\rightharpoonup}
\def \uhl {\upharpoonleft}


\newcommand{\floor}[1]{\lfloor #1 \rfloor}
\newcommand{\ceil}[1]{\lceil #1 \rceil}
\newcommand{\lcorner}[1]{\llcorner #1 \lrcorner}
\newcommand{\llb}[1]{\llbracket #1 \rrbracket}
\newcommand{\ucorner}[1]{\ulcorner #1 \urcorner}
\newcommand{\emoji}[1]{{\DejaSans #1}}
\newcommand{\vprec}{\rotatebox[origin=c]{-90}{$\prec$}}

\newcommand{\nat}[6][large]{%
  \begin{tikzcd}[ampersand replacement = \&, column sep=#1]
    #2\ar[bend left=40,""{name=U}]{r}{#4}\ar[bend right=40,',""{name=D}]{r}{#5}\& #3
          \ar[shorten <=10pt,shorten >=10pt,Rightarrow,from=U,to=D]{d}{~#6}
    \end{tikzcd}
}


\providecommand\rightarrowRHD{\relbar\joinrel\mathrel\RHD}
\providecommand\rightarrowrhd{\relbar\joinrel\mathrel\rhd}
\providecommand\longrightarrowRHD{\relbar\joinrel\relbar\joinrel\mathrel\RHD}
\providecommand\longrightarrowrhd{\relbar\joinrel\relbar\joinrel\mathrel\rhd}
\def \lrarhd {\longrightarrowrhd}


\makeatletter
\providecommand*\xrightarrowRHD[2][]{\ext@arrow 0055{\arrowfill@\relbar\relbar\longrightarrowRHD}{#1}{#2}}
\providecommand*\xrightarrowrhd[2][]{\ext@arrow 0055{\arrowfill@\relbar\relbar\longrightarrowrhd}{#1}{#2}}
\makeatother

\newcommand{\metalambda}{%
  \mathop{%
    \rlap{$\lambda$}%
    \mkern3mu
    \raisebox{0ex}{$\lambda$}%
  }%
}

%% https://tex.stackexchange.com/questions/15119/draw-horizontal-line-left-and-right-of-some-text-a-single-line
\newcommand*\ruleline[1]{\par\noindent\raisebox{.8ex}{\makebox[\linewidth]{\hrulefill\hspace{1ex}\raisebox{-.8ex}{#1}\hspace{1ex}\hrulefill}}}

% https://www.dickimaw-books.com/latex/novices/html/newenv.html
\newenvironment{Block}[1]% environment name
{% begin code
  % https://tex.stackexchange.com/questions/19579/horizontal-line-spanning-the-entire-document-in-latex
  \noindent\textcolor[RGB]{128,128,128}{\rule{\linewidth}{1pt}}
  \par\noindent
  {\Large\textbf{#1}}%
  \bigskip\par\noindent\ignorespaces
}%
{% end code
  \par\noindent
  \textcolor[RGB]{128,128,128}{\rule{\linewidth}{1pt}}
  \ignorespacesafterend
}

\mathchardef\mhyphen="2D % Define a "math hyphen"

\def \QQ {\quad}
\def \QW {​\quad}

\makeindex
\usepackage[UTF8]{ctex}
\def \btu {\bigtriangleup}
\def \btd {\bigtriangledown}
\author{郝兆宽}
\date{\today}
\title{Set Theory}
\hypersetup{
 pdfauthor={郝兆宽},
 pdftitle={Set Theory},
 pdfkeywords={},
 pdfsubject={},
 pdfcreator={Emacs 27.2 (Org mode 9.6)}, 
 pdflang={English}}
\begin{document}

\maketitle
\tableofcontents

\section{Intro}
\label{sec:org0568da2}
习题 40\%

考试 闭卷

期中 8周

集合 \(\{x,y,z,\dots\}\) 外延 \(\{x\mid x\text{ is ...}\}\) 内涵

\begin{theorem}[Cantor-Bendixen]
闭集\(X\subseteq\R\) uncountable, then \(X=Y_1\cup Y_2\), where \(Y_1\) is countable and \(Y_2\) is perfect
\end{theorem}

if \(Y_2\) is perfect, then \(\abs{Y_2}=\abs{\R}\).

Hence CH is true for close set

Suppose \(X\) is a set and \(Y=\{x\mid x\in X\wedge x\not\in x\}\) is a set

\(Y\in Y\) => \(Y\in X\) and \(Y\not\in Y\), a contradiction. Hence \(Y\not\in Y\).

\(Y\not\in Y\) => \(Y\not\in X\) or \(Y\in Y\) => \(Y\not\in X\)

Thus we have
\begin{proposition}[]
\begin{enumerate}
\item For any set \(X\), there exists a set \(Y\) s.t. \(Y\not\in X\).
\item collection of all sets is not a set
\end{enumerate}
\end{proposition}

Notation: we call \(\{x\mid\varphi(x)\}\) a class

\(V=\{x\mid x=x\}\), \(\{x\mid x\not\in x\}\) is not a set

\begin{proposition}[]
\(\{x\mid x\neq x\}\) is a set
\end{proposition}

\begin{proof}
From Existential Axiom, there is a set \(X_0\).

Claim: \(\{x\mid x\neq x\}=\{x\mid x\in X_0\wedge x\neq x\}\)

for any \(z\), \(z\neq z\), we need to prove \(z\in X_0\). But \(z\neq z\) is always false.
\end{proof}

We write \(\emptyset=\{x\mid x\neq x\}\)

For any set \(X\neq\emptyset\), its arbitrary intersection
\begin{equation*}
\bigcap X=\{u\mid\forall Y(Y\in X\to u\in Y)\}
\end{equation*}

\begin{definition}[]
\((x,y)=\{\{x\},\{x,y\}\}\)
\end{definition}

\begin{proposition}[]
\((x,y)=(x',y')\Leftrightarrow x=x'\wedge y=y'\)
\end{proposition}

\begin{proof}
作业
\end{proof}


\uline{partial ordering + well-founded => linear ordering}

since for arbitrary \(\{x,y\}\), it has a minimal element

\begin{definition}[]
For any set \(\alpha\), if \(\in\) is a well-ordering on \(\alpha\), then \(\alpha\) is an \textbf{ordinal}
\end{definition}

\begin{definition}[]
\textbf{successor} of \(x\) is \(S(x)=x\cup\{x\}\), or \(x^+\)
\end{definition}

ordinal that is not 0 and successor is a limit ordinal

极限是否存在

let \(\omega=\{n\mid n=0\vee n\text{ is successor and }\forall m<n(m\text{ is successor})\}\)

Then we need to show that \(\omega\) is a set. So we need new axiom

\begin{proposition}[]
\(X\) 是归纳集,则\(\omega\subseteq X\)
\end{proposition}

\begin{proof}
Otherwise we have least \(n\in\omega\) and \(n\not\in X\). Let \(n=S(m)\), then \(m\in X\). Hence we get a contradiction
\end{proof}

\begin{theorem}[]
For any \(X\subset\omega\), if \(X\) is inductive, then \(X=\omega\)
\end{theorem}

\([\varphi(0)\wedge\varphi(n)\to\varphi(n+1)]\Rightarrow\forall n\varphi(n)\)

\begin{theorem}[]
\(\omega\) is a ordinal and is a limit ordinal
\end{theorem}

\begin{proof}
\(\in\) is a well-ordering on \(\omega\), then we need show
\begin{enumerate}
\item \(\in\) is a partial ordering
\begin{enumerate}
\item transitive. let \(\varphi(x)\): if \(m<n\), then for any \(x\), \(n<x\to m<x\)

\(x=0\)

\(x=k+1\), \(n<k+1\), \(n\in k\cup\{k\}\). then \(n=k\) or \(n\in k\)
\end{enumerate}
\item \(\in\) is well-founded

\(X\subseteq\omega\), \(X\neq\emptyset\), then \(\exists x_0\in X\). consider \(\omega-X\). Suppose \(X\) has no minimum element
\begin{enumerate}
\item \(0\not\in X\Rightarrow 0\in\omega\)

\item if \(n\in\omega-X\), then \(S(n)\in\omega-X\). Suppose \(n\not\in X\) and \(S(n)\in X\). Then \(S(n)\) is minimum
\end{enumerate}

then \(\omega-X\) is inductive
\end{enumerate}



for any \(\alpha<\omega\), \(S(\alpha)\neq\omega\), hence \(\omega\) is limit
\end{proof}
\section{序数}
\label{sec:orge0799da}
\begin{definition}[]
一个集合\(X\)成为 \textbf{传递} 的,如果对任意\(x\in X\),都有\(x\subseteq X\)
\end{definition}

\begin{proposition}[]
假设\(X\)是传递集。如果\(X\)的所有元素也是传递集,则\(\in\)在\(X\)上是一个传递关系。反之亦然。
\end{proposition}

\begin{proof}
注意在证明反方向时,\(\in\)是\(X\)上的传递关系,如果\(x\in b\), \(b\in a\),我们先说明\(x,b,a\in X\),然
后因为\(\in\)是传递关系,于是\(x\in a\)
\end{proof}

\begin{exercise}
如果\((\calt,\in)\)是传递集,则外延公理在\(\calt\)中成立,即:对任意\(X,Y\in\calt\),\(X=Y\)当且仅当对任
意\(a\in\calt\),\(a\in X\)当且仅当\(a\in Y\)。即\(X\cap\calt=Y\cap\calt\)
\end{exercise}

如果\(\calt\)不传递,存在\(X\in\calt\),有\(X\not\subseteq\calt\),有\(a\in X\)且\(a\notin\calt\),假设\(X\)与\(Y\)仅有\(a\)差别,
但是\(\calt\)分辨不出

\begin{definition}[]
对任意集合\(\alpha\),如果\(\in\)是\(\alpha\)上的良序,就称\(\alpha\)是 \textbf{序数}
\end{definition}

\(\{\omega,\omega+1,\omega+2,\dots\}\)是集合吗?这需要替换公理

考虑\(V_{\omega+\omega}\),我们定义集合是它的元素,替换公理在这里不对,\(f(\omega)\notin V_{\omega+\omega}\)不是集合。因此我们
需要替换公理来保证它是集合
\section{基数与选择公理}
\label{sec:org9bfe7de}
\begin{proposition}[]
TFAE
\begin{enumerate}
\item set \(X\) is finite
\item there is a linear order \(\le\) on \(X\) satisfying for any nonempty subset there is a maximum
and a minimum
\item \(\forall Y\subseteq\calp(X)\) and \(Y\neq\emptyset\), it has a maximal under \(\subseteq\)
\end{enumerate}
\end{proposition}

\begin{proof}
\(2\to 1\). Let \(x_0=\inf(X)\). For any \(k\in\N\), let \(x_{k+1}=\inf(X-\{x_0,\dots,x_k\})\)
if \(x_k\neq\sup(X)\).

\(1\to 3\). Find a maximum cardinality

\(3\to 1\). If \(X\) is infinite. \(Y=\{Z\subseteq X\mid Z\text{ is finite}\}\)
\end{proof}

\begin{theorem}[]
一个序数是\(\alpha\)是至多可数的,当且仅当存在\(\R\)的子集\(A\),\(ot(A)=\alpha\)
\end{theorem}

\begin{proof}
首先假设\(A\subseteq\R\)并且\(ot(A)=\alpha\),即\(A=\{a_\beta\mid\beta<\alpha\}\),并且\(a_\beta<a_\gamma\)当且仅当\(\beta<\gamma\)。对任意\(\beta<\alpha\),
令\(I_\beta=(a_\beta,a_{\beta+1})\)为实数的区间。如果\(\alpha=\eta+1\)是后继序数,则令\(I_\eta=(a_\eta,a_\eta+1)\)。这样的区间
只有可数多
\wu{
因为每个区间都有有理数,但是有理数只有可数多
}
\end{proof}

\begin{proposition}[2.2.15]
对任意无穷基数\(\kappa,\lambda\)
\begin{enumerate}
\item \(\kappa^{<\lambda}=\sup\{\kappa^\eta\mid\eta\text{是基数并且}\eta<\lambda\}\)
\end{enumerate}
\end{proposition}

\begin{proof}
\begin{enumerate}
\item TTT
\begin{align*}
\kappa^{<\lambda}&=\abs{\bigcup\{X^\beta\mid\beta<\lambda\}}\\
&=\{\bigcup_{\beta<\lambda}\kappa^\beta\}=\bigoplus_{\beta<\lambda}\abs{\kappa^\beta}\\
&=\sup\{\abs{\kappa^\beta}\mid\beta<\lambda\}\\
&=\sup\{\kappa^{\abs{\beta}}\mid\beta<\lambda\}
\end{align*}
\item 对于\(f\in\kappa^\lambda\),\(f\)都是\(\lambda\otimes\kappa\)的子集且 \(f\in\{X\subseteq\lambda\otimes\kappa\mid\abs{X}=\lambda\}\)
\end{enumerate}
\end{proof}

\begin{exercise}
若\(\kappa\)是不可达的,则\(V_\kappa\vDash ZF\)。
\end{exercise}

\begin{corollary}[]
\(ZF\not\vDash\exists\kappa(\kappa\text{是inaccessible})\)
\end{corollary}

\begin{proof}
由 Gödel 第二不完全性定理,如果存在了,就能证明有模型了,就证明一致了
\end{proof}

大基数:\(\kappa\)是基数且\(ZF\not\vDash\exists\kappa\)

\begin{proposition}[]
\(\kappa\) 不可达,\(\abs{V_\kappa}=\kappa\).
\end{proposition}

\begin{proof}
\(\kappa\le\abs{V_\kappa}\)

证明对任意\(\alpha<\kappa\), \(\abs{V_\alpha}<\kappa\),

根据\(2^\alpha<\kappa\)
\end{proof}

证明\(f:\abs{X}=\alpha\to V_\kappa\)有界

若\(\beta<\alpha\cdot\omega\),\(\beta=\alpha\cdot\xi+\eta\),其中\(\xi\)有穷

\(\beta>\alpha\), \(\beta=\alpha\cdot\xi+\eta\). \(\alpha+\beta=\alpha+\alpha\cdot\xi+\eta=\alpha\cdot\xi+\eta\).  \(\alpha+\alpha\cdot\xi=\alpha\cdot\xi\).
只要证明\(\alpha\cdot(\xi+1)=\alpha+\alpha\cdot\xi\)

\begin{lemma}[]
\(\abs{\bigcup_{\gamma<\omega_\alpha}V_\gamma}=2^{\aleph_\beta}\)
\end{lemma}

\begin{proof}
\(2^{\aleph_\beta}=\sup\{2^{\aleph_\gamma}\mid\gamma<\beta\}\). Thus \(2^{\aleph_\beta}\le\abs{\bigcup_{\gamma<\omega_\alpha}V_\gamma}\)
\end{proof}

\(\abs{V_{\omega+1}}=2^{\aleph_0},\abs{V_{\omega+2}}=2^{\abs{2^{\aleph_0}}}>2^{\aleph_0}\)

\(\abs{V_{\omega_1}}\ge\abs{V_{\omega+2}}>2^{\aleph_0}=\beth(1)\)

\(\abs{V_{\omega_1}}=\beth_{\omega_1}\)

\begin{proposition}[]
若\(\kappa\)不可达,\(X\in V_\kappa\),\(f:X\to V_\kappa\),则\(f[X]\in V_\kappa\)
\end{proposition}

\begin{proof}
\(\abs{X}<\kappa\),对不可达基数\(\abs{V_\kappa}=\kappa\)

因此\(\abs{f(X)}<\kappa\)

已知\(f[X]\subseteq V_\kappa\)。令\(\lambda=\sup\{\rank(y)\mid y\in f[X]\}\),因为\(y\in V_\kappa\)而\(\kappa\)是极限序数,因此存
在\(\alpha<\kappa\)使得\(y\in V_\alpha\),于是\(\rank(y)<\alpha+1<\kappa\),因此\(\lambda\)是\(<\kappa\)个小于\(\kappa\)的上界,又因为\(\kappa\)
是正则的,\(\lambda<\kappa\),于是\(f[X]\subseteq V_\lambda\),于是\(f[X]\in V_{\lambda+1}\subseteq V_\kappa\)
\end{proof}

\begin{proposition}[]
令\(\beta\)为任意序数,\(\alpha\)为任意极限序数,证明:如果\(\alpha+\beta=\beta\),则\(\beta\ge\alpha\cdot\omega\)
\end{proposition}

\begin{proof}
\(\alpha+\beta=\beta\Rightarrow\alpha\le\beta\Rightarrow\exists\delta,\gamma(\beta=\alpha\cdot\delta+\gamma\wedge\gamma<\alpha)\)

若\(\delta\ge\omega\)就对了

若\(\delta<\omega\),\(\alpha+\beta=\alpha+(\alpha\cdot\delta)+\gamma=\alpha(1+\delta)+\delta>\alpha\)
\end{proof}

\(\aleph_1\le 2^{\aleph_0}\),因此\(\aleph_1^{\aleph_0}\le 2^{\aleph_0\cdot\aleph_0}=2^{\aleph_0}\)

\begin{proposition}[]
令\(X=\{f:\omega\to\omega_1\mid f\text{ 1-1}\}\),证明\(\abs{X}=2^\omega\)
\end{proposition}

\begin{proof}
\(Y=\{F\mid F:\aleph_0\to\aleph_0\times\aleph_1, \text{1-1}\}\)

\(G=\aleph_1^{\aleph_0}\to Y\) s.t. \(G(f)=F\in Y\) s.t. \(F(n)=(n,f(n))\), then \(G\) is 1-1

Hence \(\aleph_1^{\aleph_0}\le\abs{Y}=(\aleph_0\times\aleph_1)^{\aleph_0}=\aleph_1^{\aleph_0}\)
\end{proof}

\begin{proof}

\end{proof}



\section{滤、理想与无界闭集}
\label{sec:orga260981}
\subsection{滤}
\label{sec:orge68335d}
\begin{proposition}[]
\begin{enumerate}
\item \(-0=1\)
\item \(-1=0\)
\item \(a\cdot 1=a\)
\item \(a+0=a\)
\item \(a+a=a\)
\item \(a\cdot a=a\)
\item \(1+a=1\)
\item \(0\cdot a=0\)
\item \(a+b=1\wedge a\cdot b=0\Rightarrow b=-a\)
\item \(-(a\cdot b)=(-a)+(-b)\)
\item \(-(a+b)=(-a)\cdot(-b)\)
\end{enumerate}
\end{proposition}

\begin{proof}
\begin{enumerate}
\item \(1=0+(-0)=(0\cdot(-0))+(-0)=-0\)
\item \(0=1\cdot(-1)=(1+(-1))\cdot(-1)=-1\)
\item \(a\cdot 1=a\cdot(a+(-a))=a\)
\setcounter{enumi}{4}
\item \(a+a=a+(a\cdot 1)=a\)
\setcounter{enumi}{6}
\item \(1+a=(a+1)\cdot 1=(a+1)\cdot(a+-a)=a\cdot a+0+a+-a=a+-a=1\)
\item \(0\cdot a=(a\cdot (-a))\cdot a=a\cdot a\cdot (-a)=a\cdot (-a)=0\)
\setcounter{enumi}{8}
\item \(-a=(-a)\cdot 1=(-a)\cdot(a+b)=(-a)\cdot a+(-a)\cdot b=(-a)\cdot b\).

\(ab+(-a)b=-a\). \(b(a+(-a))=b=-a\)
\item \(ab+(-a)+(-b)=ab+(-a)+(-b)\cdot 1=ab+(-a)+(-b)a+(-b)(-a)=a(b+(-b))+(-a)+(-b)(-a)=1+(-b)(-a)=1\)
\end{enumerate}
\end{proof}

\(G\)有有穷交性质
\(F=\{b\in B\mid\exists g\in G(g\le b)\}\)

若\(a,b\in F\),则\(g_1\le a,g_2\le b\)

Rasiowa-Sikorski [mathematics for metamathematics] 是为了证明一阶完全性,\(\sum D\)等价于\(\exists x\)

令\(\phi(x,\bary)\), \(M_\phi=\{[\varphi(t,\bary)]\mid t\text{ a term}\}\subset B\)

\textbf{Claim} \(\sum M_\phi=[\exists x\varphi(x,\bary)]\)

那么如果\(U\)是完全的,那么\(\sum M_\phi\in U\),于是\(\exists t[\phi(t,\bary)]\in U\)。类似于极大一致Henkin集

under CH \(\abs{A}<2^{\aleph_0}=\aleph_1\) iff \(\abs{A}\le\aleph_0\)

cichon diagram

\(f:M\to N\), \(f\upharpoonright A:A\to N\), \(\forall a\in A, f\upharpoonright A(a)=f(a)\)
\subsection{Clut set}
\label{sec:org3b7ceda}
\begin{definition}[]
\(\alpha\) limit, \(C\subseteq\alpha\) is a \textbf{club set} in \(\alpha\) if
\begin{enumerate}
\item unbounded  \(\sup C=\alpha\), that is, for any \(\beta<\alpha\), there is \(\gamma\in C\) s.t. \(\beta<\gamma\)
\item closed: for any limit \(\gamma<\alpha\), \(\sup(C\cap\gamma)=\gamma\Rightarrow\gamma\in C\)
\end{enumerate}
\end{definition}

If \(A\subseteq C\subseteq\alpha\) and \(\sup C=\gamma<\alpha\), then \(\gamma\) is the \textbf{limit point} of \(C\). \(C\) is closed in \(\alpha\) iff
all limit point of \(C\) belong to \(C\)

\begin{lemma}[]
\label{3.3.4}
\(\alpha\) limit and \(\cf(\alpha)>\omega\), then
\begin{enumerate}
\item \(\alpha\) is a club set of itself
\item \(\forall\beta<\alpha\), \(\{\beta<\alpha\mid\delta>\beta\}\) is a club set in \(\alpha\)
\item \(X=\{\beta<\alpha\mid\beta\text{ limit}\}\) is a club set in \(\alpha\)
\item If \(X\) is unbounded in \(\alpha\), then \(X'=\{\gamma\in X\mid\gamma<\alpha\wedge\gamma\text{ is a limit point of }X\}\) is a club
set in \(\alpha\)
\end{enumerate}
\end{lemma}

\begin{proof}
\begin{enumerate}
\setcounter{enumi}{2}
\item \(X\) is closed. For any \(\xi\in\alpha\), define
\begin{equation*}
\xi=\xi_0,\xi_1,\cdots,\xi_n,\cdots\quad(n\in\omega)
\end{equation*}
s.t. \(\xi_{n+1}=\min\{\alpha-\xi_n\}\). Let \(\eta=\sup\xi_i\). Then \(\xi<\eta\in X\). \(\eta<\alpha\) since \(\cf(\alpha)>\omega\)
\item Like 3, for any \(\xi\in\alpha\), define \(\xi_{n+1}=\min\{\xi'>\xi:X-\{\xi_1,\dots,\xi_n\}\}\), this works
since \(X\) is unbounded.

For any limit point \(\eta<\alpha\) of \(X'\), that is, \(\sup(X'\cap\eta)=\eta\), then for any \(\sigma<\eta\), there
is limit point \(\xi<\eta\) of \(X\) s.t. \(\sigma+1<\xi\). By definition of limit point, \(\exists\mu\in X\cap\xi\)
s.t. \(\sigma<\mu\), so \(\sup(X\cap\eta)=\eta\) and \(\eta\) is a limit point of \(X\), thus \(\eta\in X'\)

Limit of limits of \(X\) is still a limit of \(X\)
\end{enumerate}
\end{proof}

\begin{lemma}[]
\label{3.3.5}
if \(\alpha\) is limit and \(\cf(\alpha)>\omega\), and \(f:\alpha\to\alpha\) is strictly increasing and continuous, that is,
for any limit \(\beta<\alpha\), \(f(\beta)=\bigcup_{\gamma<\beta}f(\gamma)\), then
\begin{enumerate}
\item \(\im(f)\) is a club set in \(\alpha\)
\item if \(\alpha\) is regular, then every club set \(C\) in \(\alpha\) is the image of such a function
\end{enumerate}
\end{lemma}

\begin{proof}
\begin{enumerate}
\setcounter{enumi}{1}
\item suppose \(ot(C)=\tau\). \(f:(\tau,<)\cong(C,<)\), then \(f\) is strictly increasing and
continuous. Since \(C\) is unbounded and \(\alpha\) is regular, \(\tau\ge\cf(\alpha)=\alpha\). \(\forall\eta<\tau\), \(\eta\le f(\eta)\),
so \(\tau\le\sup f(\eta)=\alpha\), thus \(\tau=\alpha\)
\end{enumerate}
\end{proof}

\begin{proposition}[]
\label{3.3.7}
Suppose \(\alpha\) is a limit ordinal and \(\cf(\alpha)>\omega\), then for any \(\gamma<\cf(\alpha)\), if \((C_\xi)_{\xi<\gamma}\) is
a sequence of club sets in \(\alpha\), then \(\bigcap_{\xi<\gamma}C_\xi\) is a club set in \(\alpha\)
\end{proposition}

\begin{proof}
Suppose \(\gamma=2\). Intersection of closed sets is still closed. We prove that \(C_1\cap C_2\) is
unbounded in \(\alpha\). \(\forall\delta<\kappa\), \(\exists\xi\in C_1,\eta\in C_2\) s.t. \(\delta<\xi<\eta\), let
\begin{equation*}
\xi_0<\eta_0<\xi_1<\eta_1<\dots
\end{equation*}
where \(\xi_0=\xi,\eta_0=\eta\) and for any \(n\in\omega\), \(\xi_n\in C_1\), \(\eta_n\in C_2\). Let \(\mu\) be the limit of this
sequence, then \(\sup(C_1\cap\mu)=\mu\) and \(\sup(C_2\cap \mu)=\mu\), hence \(\mu\in C_1\cap C_2\).

Suppose \(\gamma\) is a successor ordinal

Suppose \(\gamma\) is a limit ordinal, let \(D=\bigcap_{\xi<\gamma}C_\xi\), we prove it is unbounded. For any \(\eta<\gamma\),
if \(D_\eta=\bigcap\{C_\xi\mid\xi<\eta\}\), then \(\eta_n\) is a club set and \(D=\bigcap_{\eta<\gamma}D_\eta\) and \(\eta<\eta'<\gamma\)
implies \(D_\eta\supset D_{\eta'}\). For any \(\mu<\alpha\), let
\begin{equation*}
\xi_0<\xi_1<\cdots<\xi_\eta<\cdots
\end{equation*}
where \(\xi_0>\mu\), and for any \(\eta<\gamma\), \(\xi_\eta\in D_\eta\) is the minimum element larger
than \(\sup\{\xi_\alpha\mid\alpha<\eta\}\). Since \(\cf(\alpha)>\gamma\), \(\xi=\sup\{\xi_n\mid\eta<\gamma\}<\alpha\). for any \(\eta<\lambda\), \(\xi\in D_\eta\),
thus \(\xi\in D\) and \(\mu<\xi\)
\end{proof}

\begin{definition}[]
For any limit \(\cf(\alpha)>\omega\)
\begin{equation*}
F_{CB}(\alpha)=\{X\subseteq\alpha\mid\exists C(C\text{ is a club set in }\alpha\wedge C\subseteq X)\}
\end{equation*}
is a filter, called \textbf{club filter} in \(\alpha\)
\end{definition}

\begin{corollary}[]
\label{3.3.9}
If \(\kappa\) is uncountable regular cardinal, then club filter in \(\kappa\) is \(\kappa\)-complete
\end{corollary}

\begin{definition}[]
for any ordinal \(\alpha\), \((X_\xi\mid\xi<\alpha)\) is a sequence of subsets of \(\alpha\)
\begin{enumerate}
\item \textbf{diagonal intersection} of \(X_\xi\)
\begin{equation*}
\bigtriangleup_{\xi<\alpha}X_\xi=\{\eta<\alpha\mid\eta\in\bigcap_{\xi<\eta}X_\xi\}
\end{equation*}
\item \textbf{diagonal union} of \(X_\xi\)
\begin{equation*}
\bigtriangledown_{\xi<\alpha}X_\xi=\{\eta<\alpha\mid\eta\in\bigcup_{\xi<\eta}X_\xi\}
\end{equation*}
\end{enumerate}
\end{definition}

\begin{remark}
Let \(Y_\xi=\{\eta\in X_\xi\mid\eta>\xi\}\), then \(\bigtriangleup_{\xi<\alpha}X_\xi=\bigtriangleup_{\xi<\alpha}Y_\xi\)
\end{remark}

\begin{proposition}[]
for any uncountable regular \(\kappa\), and a sequence of club sets \((X_\gamma\mid\gamma<\kappa)\) in \(\kappa\), \(\bigtriangleup_{\gamma<\kappa}X_\gamma\) is a
club set in \(\kappa\).
\end{proposition}

\begin{proof}
Let \(C_\gamma=\bigcap_{\xi<\gamma}X_\xi\), then \(\bigtriangleup X_\gamma=\bigtriangleup C_\gamma\)
\wu{
\begin{align*}
\eta\in\btu C_\gamma&\Leftrightarrow\forall\xi<\eta,\eta\in C_\xi=\bigcap_{\zeta<\xi}X_\zeta\\
&\Leftrightarrow\forall\zeta<\xi<\eta,\eta\in X_\zeta
\end{align*}
guess should be \(C_\gamma=\bigcap_{\xi\le\gamma}X_\xi\)
}


let
\begin{equation*}
C_0\supset C_1\supset\cdots\supset C_\gamma\supset\cdots\quad(\gamma<\kappa)
\end{equation*}
Define \(C=\btu C_\gamma\). To prove \(C\) is closed, let \(\eta\) be the limit point of \(C\). We need to
prove \(\eta\in C\), that is, \(\forall\xi<\eta\), \(\eta\in C_\xi\). For any \(\xi<\eta\), define \(X=\{\nu\in C\mid\xi<\nu<\eta\}\), then \(X\subset C_\xi\); by
Theorem \ref{3.3.9}, \(C_\xi\) is a club in \(\kappa\), therefore \(\eta=\sup X\in C_\xi\), hence \(\eta\in C\)

Unboundedness: for any \(\mu<\kappa\), define \((\beta_n\mid n\in\omega)\): let \(\mu<\beta_0\in C_0\),
and \(\beta_n<\beta_{n+1}\in C_{\beta_n}\). Since \(C_{\beta_n}\) is unbounded, such \(\beta_{n+1}\) can always be
found. Also
\begin{equation*}
C_{\beta_0}\supset C_{\beta_1}\supset C_{\beta_2}\supset\cdots
\end{equation*}
thus for any \(m>n\), \(\beta_m\in C_{\beta_{m+1}}\subset C_{\beta_n}\). Now we prove \(\beta=\sup\{\beta_n\mid n\in\omega\}\in C\), which
is suffice to show that for any \(\xi<\beta\), \(\beta\in C_\xi\). But if \(\xi<\beta\), there is \(n\) s.t. \(\xi<\beta_n\) and
for any \(m>n\), \(\beta_m\in C_{\beta_n}\subset C_\xi\). Since \(C_\xi\) is closed, \(\beta\in C_\xi\). Thus \(\beta\in C\)
\end{proof}

\begin{corollary}[]
For any uncountable regular cardinal \(\kappa\), if \(f:\kappa\to\kappa\) is a function ,then
\begin{equation*}
D=\{\alpha<\kappa\mid\forall\beta<\alpha(f(\beta)<\alpha)\}
\end{equation*}
is a club set
\end{corollary}

\begin{proof}
For any \(\alpha<\kappa\), let \(C_\alpha=\{\beta<\kappa\mid f(\alpha)<\beta\}\), which is a club set. Then \(D=\btu C_\alpha\)
\end{proof}

\begin{definition}[]
\(\alpha\) limit and \(\cf(\alpha)>\omega)\)
\begin{enumerate}
\item If \(S\subseteq\alpha\) and for any club set \(C\) in \(\alpha\) \(S\cap C\neq\emptyset\), then \(S\) is called \textbf{stationary set} in \(\alpha\)
\item \(I_{NS}(\alpha)=\{X\subseteq\alpha\mid\exists C(C\text{ is a club set in $\alpha$}\wedge X\cap C=\emptyset)\}\) is called a \textbf{non-stationary
ideal} in \(\alpha\)
\end{enumerate}
\end{definition}

\begin{proposition}[]
limit ordinal \(\alpha\) with \(\cf(\alpha)>\omega\)
\begin{enumerate}
\item club set in \(\alpha\) is stationary. if \(S\) is stationary and \(S\subseteq T\subseteq\alpha\), then \(T\) is stationary
\item stationary set in \(\alpha\) is unbounded
\item there is unbounded \(T\subseteq\alpha\) that is not stationary
\end{enumerate}
\end{proposition}

\begin{proof}
\begin{enumerate}
\item \ref{3.3.7}
\item If \(S\) is stationary, for any \(\beta<\alpha\), \(\{\gamma<\alpha\mid\beta<\gamma\}\) is a club set in \(\alpha\) and the elements of
the intersection of it with \(S\) is larger than \(\beta\)
\item \(T=\{\alpha+1\mid\alpha<\kappa\}\) is unbounded but not stationary, since the club set of all limit ordinal
doesn't intersect with it
\end{enumerate}
\end{proof}

\begin{proposition}[]
limit ordinal \(\alpha\) with \(\cf(\alpha)>\omega\) and \(\lambda<\cf(\alpha)\) is regular, then
\begin{equation*}
E_\lambda^\alpha=\{\beta<\alpha\mid\cf(\beta)=\lambda\}
\end{equation*}
is stationary in \(\alpha\)
\end{proposition}

\begin{proof}
For any club set \(C\) in \(\alpha\), define a strictly increasing sequence of \(C\):
\begin{equation*}
\alpha_0<\alpha_1<\cdots<\alpha_\xi<\cdots\quad (\xi<\lambda)
\end{equation*}
such sequence exists since \(\lambda\) is regular, \(\lambda<\cf(\alpha)\) and \(C\) is unbounded. suppose \(\delta\) is the
supremem of the sequence. Since \(C\) is closed, \(\delta\in C\), Since \(\cf(\delta)=\lambda\), \(\delta\in E_\lambda^\alpha\)
\end{proof}

\begin{proposition}[]
for any uncountable regular cardinal \(\kappa\), if \((X_\xi\mid\xi<\kappa)\) is a sequence of non-stationary sets,
then \(\btd_{\xi<\kappa}X_\xi\) is non-stationary. That is, \(I_{NS}(\kappa)\) is closed under diagonal intersection
\end{proposition}

\begin{proof}
For any \(X_\xi\), there is \(C_\xi\) s.t. \(X_\xi\cap C_\xi=\emptyset\). Let \(C=\btu C_\xi\), then \(C\) is a club
set. Let \(X=\btd X_\xi\), then \(X\cap C=\emptyset\)
\end{proof}

\begin{definition}[]
For a ordinal set \(S\) and \(\dom(f)=S\), if for any \(0\neq\alpha\in S\), \(f(\alpha)<\alpha\), then \(f\) is \textbf{regressive}
\end{definition}

\begin{theorem}[Fodor]
For any uncountable regular cardinal \(\kappa\), stationary \(S\subseteq\kappa\), if \(\dom(f)=S\) is regressive, then
there is a stationary \(T\subseteq S\) and ordinal \(\gamma<\kappa\) s.t. for any \(\alpha\in T\), \(f(\alpha)=\gamma\)
\end{theorem}

\begin{proof}
If for any \(\gamma<\kappa\), \(A_\gamma=\{\alpha\in S\mid f(\alpha)=\gamma\}\) is non-stationary, and there is a club set \(C_\gamma\)
s.t. \(A_\gamma\cap C_\gamma=\emptyset\), that is, for any \(\alpha\in S\cap C_\gamma\), \(f(\alpha)\neq\gamma\). Let \(C=\btu_{\gamma<\kappa}C_\gamma\).
Then \(\alpha\in C\) iff \(\forall\gamma<\alpha\), \(\alpha\in C_\gamma\) iff \(\forall\gamma<\alpha\), \(f(\alpha)\neq\gamma\). Hence for
any \(\alpha\in C\), \(f(\alpha)\ge\alpha\). Since \(C\) is a club set, \(S\cap C\neq\emptyset\), but for any \(\alpha\in S\), \(f(\alpha)<\alpha\)
\end{proof}

\begin{lemma}[]
\label{3.3.24}
uncountable regular cardinal \(\kappa\), \(S\subseteq\kappa\) stationary, \(f\) is a regressive function on \(S\). If
for any \(\eta<\kappa\),
\begin{equation*}
X_\eta=\{\alpha\in S\mid f(\alpha)\ge\eta\}
\end{equation*}
is stationary, then \(S\) can be partitioned into \(\kappa\) disjoint stationary sets
\end{lemma}

\begin{proof}
For any \(\eta<\kappa\), \(f\uhr X_\eta\) is a regressive function on \(X_\eta\). By Fodor's, there
is \(\eta<\gamma_\eta<\kappa\) s.t. \(S_{\gamma_n}=\{\alpha\in S\mid f(\alpha)=\gamma_n\}\) is stationary

Define \(g:\kappa\to\kappa\): \(g(0)=0\), \(g(\eta)=\sup\{\gamma_{g(\xi)}+1\mid\xi<\eta\}\). If \(\xi<\eta<\kappa\),
then \(\gamma_{g(\xi)}<g(\eta)\le\gamma_{g(\eta)}\), hence \(\eta\mapsto\gamma_{g(\eta)}\) is a increasing cofinal function from
\(\kappa\) to \(\kappa\). Thus \(\{S_{\gamma_{g(\eta)}}\mid\eta<\kappa\}\) has cardinality \(\kappa\) and is pairwise disjoint
\end{proof}

\begin{lemma}[]
uncountable regular \(\kappa\), \(\lambda<\kappa\) is regular, any stationary subset of
\begin{equation*}
E_\lambda^\kappa=\{\alpha<\kappa\mid\cf(\alpha)=\lambda\}
\end{equation*}
can be partitioned into \(\kappa\) disjoint stationary subsets
\end{lemma}

\begin{proof}
Stationary \(S\subseteq E^\kappa_\lambda\), \(\forall\alpha\in S\), choose a strictly increasing cofinal
function \(f_\alpha:\lambda\to\alpha\). \(\forall\xi<\lambda\), define \(g_\xi:\kappa\to\kappa\):
\begin{equation*}
g_\xi(\alpha)=
\begin{cases}
0&\alpha\notin S\\
f_\alpha(\xi)&\alpha\in S
\end{cases}
\end{equation*}
\(g_\xi\uhr S\) is regressive

\(\forall\eta<\kappa\forall\xi<\lambda\), let
\begin{equation*}
X_\xi^\eta=\{\alpha\in S\mid g_\xi(\alpha)\ge\eta\}
\end{equation*}
We prove: \(\exists\xi<\lambda\forall\eta<\kappa\), \(X_\xi^\eta\) is stationary. Otherwise, \(\forall\xi<\lambda\), there is a club \(C_\xi\)
and a ordinal \(\eta_\xi<\kappa\) s.t. \(C_\xi\cap X_\xi^{\eta_\xi}=\emptyset\). Let \(C=\bigcap_{\xi<\lambda}C_\xi\), \(\eta=\sup\{\eta_\xi\mid\xi<\lambda\}\),
then \(C\) is a club. But for any \(\alpha\in C\cap S\), \(\forall\xi<\lambda\), \(g_\xi(\alpha)<\eta\) since \(C\cap X_\xi^\eta=\emptyset\), therefore \(C\cap S\subseteq\eta\), a
contradiction since \(C\) is a club

Fix a \(\xi<\lambda\) s.t. for any \(\eta<\kappa\), \(X_\xi^\eta\) is stationary. By \ref{3.3.24}, \(S\) can be
partitioned into \(\kappa\) disjoint stationary sets
\end{proof}

\begin{corollary}[]
\label{3.3.26}
uncountable regular \(\kappa\), \(X=\{\alpha<\kappa\mid\cf(\alpha)<\alpha\}\). If \(S\subseteq X\) is stationary, then \(S\) can be
partitioned into \(\kappa\) disjoint stationary sets
\end{corollary}

\begin{proof}
Let \(f:\kappa\to\kappa\) be \(f(\alpha)=\cf(\alpha)\). Then \(f\uhr S\) is regressive. By Fodor's lemma, there
is \(\lambda<\kappa\), \(S_\lambda=\{\alpha\in S\mid f(\alpha)=\lambda\}\) is stationary. Note that \(S_\lambda\subseteq E_\lambda^\kappa\), hence \(S_\lambda\) can be
partitioned into \(\kappa\) disjoint stationary sets
\end{proof}

\begin{lemma}[skip]
uncountable regular \(\kappa\), \(S\subseteq\kappa\) stationary, \(f:S\to\kappa\) regressive. for any \(\beta<\kappa\), define
\begin{equation*}
S_\beta=\{\alpha\in S\mid f(\alpha)=\beta\}
\end{equation*}
Let \(I=\{S_\beta\mid S_\beta\text{ stationary}\}\), then exactly one of below is true
\begin{enumerate}
\item \(\abs{I}=\kappa\)
\item \(\abs{I}<\kappa\) and there is a club \(C\), \(\im(f\uhr C\cap S)\) is bounded in \(\kappa\)
\end{enumerate}
\end{lemma}

\begin{lemma}[]
uncountable regular \(\kappa\), \(R=\{\omega<\gamma<\kappa\mid\cf(\gamma)=\gamma\}\), define
\begin{equation*}
D=\{\gamma\in R\mid R\cap\gamma\in I_{NS}(\gamma)\}
\end{equation*}
If \(R\) is stationary in \(\kappa\), then so is \(D\)
\end{lemma}

\begin{proof}
If \(D\) is not stationary, there is club \(C\) s.t. \(C\cap D=\emptyset\). Let \(C'\) be the set of limit
points of \(C\). Let \(\gamma=\min(C'\cap R)\), \(\gamma\in R-D\), thus \(R\cap\gamma\) is stationary in \(\gamma\)

Now consider \(C\cap\gamma\), since \(\gamma\) is a limit point of \(C\), this set is unbounded in \(\gamma\). By
\ref{3.3.4} (4), \(C'\cap\gamma\) is a club set in \(\gamma\), thus \(R\cap C'\cap\gamma\neq\emptyset\), which contradicts the minimality
of \(\gamma\) in \(R\cap C'\)
\end{proof}

\begin{theorem}[Soloway]
Any stationary set in uncountable regular cardinal \(\kappa\) can be partitioned into \(\kappa\) disjoint
stationary sets
\end{theorem}

\begin{proof}
stationary \(S\subseteq\kappa\), let
\begin{align*}
&S_0=\{\alpha<\kappa\mid\cf(\alpha)<\alpha\}\\
&S_1=\{\alpha<\kappa\mid\cf(\alpha)=\alpha\}
\end{align*}
Then \(S=S_0\cup S_1\), hence either \(S_0\) or \(S_1\) is stationary ?

If \(S_0\) is stationary, by \ref{3.3.26}, \(S_0\) can be partitioned into \(\kappa\) disjoint stationary
sets

Now suppose \(S_1\) is stationary, let \(D=\{\alpha\in S_1\mid S_1\cap\alpha\in I_{NS}(\alpha)\}\)
\end{proof}






\subsection{Ultrafilter and large cardinal}
\label{sec:org6f5bc34}
\subsubsection{Regular ultrafilter}
\label{sec:org7001f66}
\begin{definition}[]
limit ordinal \(\alpha\), \(\cf(\alpha)>\omega\), \(F\) is a filter on \(\alpha\). If \(F\) is closed under diagonal
intersection, then \(F\) is \textbf{regular}
\end{definition}

\begin{examplle}[]
For any limit ordinal \(\alpha\) with \(\cf(\alpha)>\omega\), \(F_{CB}(\alpha)\) is regular
\end{examplle}

Let elements in \(F\) have measure 1, otherwise has measure 0
\begin{definition}[]
limit ordinal \(\alpha\), \(\cf(\alpha)>\omega\), \(F\) is a filter on \(\alpha\), if \(\forall X\in F\), \(Y\cap X\neq\emptyset\), then \(Y\subseteq\alpha\)
has \textbf{positive measure}
\end{definition}

\begin{lemma}[]
uncountable regular cardinal \(\kappa\), \(F\) is a filter on \(\kappa\). \(F\) is regular \(\Leftrightarrow\) for
any \(f:\kappa\to\kappa\), if there is a \(X\) with positive measure s.t. \(f\uhr X\) is regressive, then
there exists \(\gamma<\kappa\) s.t. \(X_\gamma=\{\alpha\in X\mid f(\alpha)=\gamma\}\) has positive measure
\end{lemma}

\begin{proof}
\(\Rightarrow\). \(f:\kappa\to\kappa\), \(Y\) has positive measure and \(f\uhr Y\) is regressive. If for
all \(\gamma<\kappa\), \(Y_\gamma=\{\alpha\in Y\mid f(\alpha)=\gamma\}\) doesn't have positive measure, then there is \(X_\gamma\in F\)
s.t. \(Y_\gamma\cap X_\gamma=\emptyset\). Let \(X=\btu X_\gamma\in F\) since \(F\) is regular. Since \(Y\) has positive
measure, \(X\cap Y\neq\emptyset\). For any \(\gamma\in X\cap Y\), since \(\gamma\in X\), then for any \(\beta<\gamma\), \(f(\gamma)\neq\beta\),
therefore \(f(\gamma)\ge\gamma\), contradicting the fact that \(f\) is regressive on \(Y\).

\(\Leftarrow\). Suppose for any \(\beta<\kappa\), \(X_\beta\in F\) and \(X=\btu X_\beta\notin F\), then
\begin{equation*}
Y=\kappa-X=\{\alpha<\kappa\mid\exists\beta<\alpha(\alpha\notin X_\beta)\}
\end{equation*}
has positive measure. Define \(f:\kappa\to\kappa\):
\begin{equation*}
f(\alpha)=
\begin{cases}
\min\{\beta\mid\beta<\alpha\wedge\alpha\notin X_\beta\}&\alpha\in Y\\
0
\end{cases}
\end{equation*}
that is, if \(\alpha\notin X\), there is \(\beta<\alpha\) s.t. \(\alpha\notin X_\beta\)

Then \(f\uhr Y\) is regressive, hence there is \(0<\gamma<\kappa\), \(Y_\gamma=\{\alpha\in Y\mid f(\alpha)=\gamma\}\) has positive
measure. But \(\alpha\in Y_\gamma\Rightarrow\alpha\notin X_\gamma\), hence \(X_\gamma\cap Y_\gamma=\emptyset\), a contradiction
\end{proof}
\subsubsection{Measurable cardinal}
\label{sec:orgf843bd4}
\begin{lemma}[]
there is no \(\aleph_1\)-complete non-principal ultrafilter on \(2^\omega=\{f\mid f:\omega\to\{0,1\}\}\)
\end{lemma}

\begin{proof}
If \(U\) is a \(\aleph_1\)-complete non-principal ultrafilter on \(2^\omega\).
Let \(L=\{f\in 2^\omega\mid f(0)=0\}\), \(R=\{f\in 2^\omega\mid f(0)=1\}\), then \(2^\omega=L\cup R\), and only one of them
belongs to \(U\). Define \(h\) and a sequence \((X_n)_{n\in\omega}\) of subsets of \(2^\omega\) as follows:
\begin{enumerate}
\item If \(L\in U\), then \(h(0)=0\), \(X_0=R\). If \(R\in U\), then let \(h(0)=1\), \(X_0=L\)
\item Let \(h(n)\) and \(X_n\) is defined, then
\begin{equation*}
Y=\{f\in 2^\omega\mid\forall i\le n(f(i)=h(i))\}\in U
\end{equation*}
Let \(Y^L=\{f\in Y\mid f(n+1)=0\}\), \(Y^R=\{f\in Y\mid f(n+1)=1\}\), then only one of them belongs to \(U\).
If \(Y^L\in U\), let \(h(n+1)=0\), \(X_{n+1}=Y^R\); otherwise, \(h(n+1)=1\), \(X_{n+1}=Y_L\)
\end{enumerate}


For any \(f\in 2^\omega\), if \(f\neq h\), then there is a smallest \(i\in\omega\) s.t. \(f(i)\neq h(i)\), which
implies \(f\in X_i\), thus
\begin{equation*}
\{h\}\cup\bigcup_{n\in\omega}X_n=2^\omega\in U
\end{equation*}
But \(\forall n\in\omega\), \(X_n\notin U\), \(U\) is \(\aleph_1\)-complete implying \(\bigcup_{n\in\omega}X_n\notin U\).
\wu{
\(\bigcup_{n\in\omega}X_n\notin U\Leftrightarrow\ove{\bigcup_{n\in\omega}X_n}\in U\Leftrightarrow\bigcap_{n\in\omega}\ove{X_n}\in U\Leftarrow\forall n\in\omega(\ove{X_n}\in U)\)
}
And \(U\) is not principal, thus \(\{h\}\notin U\).
\end{proof}

\begin{lemma}[]
Let \(\kappa\) be the minimum cardinal with a \(\aleph_1\)-complete non-principal ultrafilter on it, then
\begin{enumerate}
\item any \(\aleph_1\)-complete non-principal ultrafilter on \(\kappa\) is \(\kappa\)-complete
\item \(\kappa\) is uncountable and regular
\end{enumerate}
\end{lemma}

\begin{proof}
\begin{enumerate}
\item \(U\) is an \(\aleph_1\)-complete non-principal ultrafilter on \(\kappa\). If \(U\) is not \(\kappa\)-complete, then
there is \(\gamma<\kappa\), \((X_\beta)_{\beta<\gamma}\) a sequence of pairwise disjoint subsets of \(\kappa\)
s.t. \(\bigcup_{\beta<\gamma}X_\beta\in U\) and \(\forall\beta<\gamma(X_\beta\notin U)\)

Now we define a filter on \(\gamma\). First, for any \(Y\subseteq\gamma\), let
\begin{equation*}
X_Y=\{\delta<\kappa\mid\exists\beta\in Y(\delta\in X_\beta)\}
\end{equation*}
that is, \(X_Y=\bigcup_{\beta\in Y}X_\beta\). Let \(F=\{Y\subseteq\gamma\mid X_Y\in U\}\), we prove that \(F\) is
an \(\aleph_1\)-complete non-principal ultrafilter on \(\gamma\), contradicting the minimality of \(\kappa\)

Since

\item Let \((X_\beta)_{\beta<\gamma}\) be a sequence of subsets of \(\kappa\), \(\gamma<\kappa\) and for
any \(\beta<\gamma\), \(\abs{X_\beta}<\kappa\), now we prove \(\abs{\bigcup_{\beta<\gamma}X_\beta}\neq\kappa\)

By 1, let \(U\) be a \(\kappa\)-complete non-principal ultrafilter on \(\kappa\). For any \(\beta<\kappa\),
since \(\abs{X_\beta}<\kappa\), \(X_\beta\notin U\). But \(U\) is \(\kappa\)-complete, so \(\bigcup_{\beta<\gamma}X_\beta\notin U\)
\end{enumerate}
\end{proof}

\section{复习}
\label{sec:org8ffbf10}
没有良序集同构于真前段

递归定理
\end{document}
