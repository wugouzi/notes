% Created 2021-08-01 Sun 16:58
% Intended LaTeX compiler: pdflatex
\documentclass[11pt]{article}
\usepackage[utf8]{inputenc}
\usepackage[T1]{fontenc}
\usepackage{graphicx}
\usepackage{grffile}
\usepackage{longtable}
\usepackage{wrapfig}
\usepackage{rotating}
\usepackage[normalem]{ulem}
\usepackage{amsmath}
\usepackage{textcomp}
\usepackage{amssymb}
\usepackage{capt-of}
\usepackage{hyperref}
\graphicspath{{../../books/}}
% TIPS
% \substack{a\\b} for multiple lines text





% pdfplots will load xolor automatically without option
\usepackage[dvipsnames]{xcolor}

\usepackage{forest}
% two-line text in node by [two \\ lines]
% \begin{forest} qtree, [..] \end{forest}
\forestset{
  qtree/.style={
    baseline,
    for tree={
      parent anchor=south,
      child anchor=north,
      align=center,
      inner sep=1pt,
    }}}
%\usepackage{flexisym}
% load order of mathtools and mathabx, otherwise conflict overbrace

\usepackage{mathtools}
%\usepackage{fourier}
\usepackage{pgfplots}
\usepackage{amsthm, mathabx,  amsmath, commath}
\usepackage{amsfonts}

\usepackage{empheq}
\usepackage{tikz}
\usetikzlibrary{arrows.meta}
\usepackage[most]{tcolorbox}

\newtheorem{theorem}{Theorem}[section]
\newtheorem{definition}{Definition}[section]
\newtheorem{corollary}{Corollary}[section]
\newtheorem{example}{Example}[section]
\newtheorem{lemma}{Lemma}[section]
\newtheorem{proposition}{Proposition}[section]

\newcommand{\bl}[1] {\boldsymbol{#1}}
\newcommand{\Wt}[1] {\stackrel{\sim}{\smash{#1}\rule{0pt}{1.1ex}}}
\newcommand{\wt}[1] {\widetilde{#1}}


%For boxed texts in align, use Aboxed{}
%otherwise use boxed{}

\DeclareMathSymbol{\widehatsym}{\mathord}{largesymbols}{"62}
\newcommand\lowerwidehatsym{%
  \text{\smash{\raisebox{-1.3ex}{%
    $\widehatsym$}}}}
\newcommand\fixwidehat[1]{%
  \mathchoice
    {\accentset{\displaystyle\lowerwidehatsym}{#1}}
    {\accentset{\textstyle\lowerwidehatsym}{#1}}
    {\accentset{\scriptstyle\lowerwidehatsym}{#1}}
    {\accentset{\scriptscriptstyle\lowerwidehatsym}{#1}}
}

\usepackage{graphicx}
    
% text on arrow for xRightarrow
\makeatletter
%\newcommand{\xRightarrow}[2][]{\ext@arrow 0359\Rightarrowfill@{#1}{#2}}
\makeatother


\def \bx {\boldsymbol{x}}
\def \ba {\boldsymbol{a}}
\def \bI {\boldsymbol{I}}
\def \bt {\boldsymbol{t}}
\def \bb {\boldsymbol{b}}
\def \bA {\boldsymbol{A}}
\def \bX {\boldsymbol{X}}
\def \bu {\boldsymbol{u}}
\def \bS {\boldsymbol{S}}
\def \bZ {\boldsymbol{Z}}
\def \bz {\boldsymbol{z}}
\def \by {\boldsymbol{y}}
\def \bw {\boldsymbol{w}}
\def \bT {\boldsymbol{T}}
\def \bS {\boldsymbol{S}}
\def \bm {\boldsymbol{m}}
\def \bW {\boldsymbol{W}}
\def \bY {\boldsymbol{Y}}
\def \bH {\boldsymbol{H}}
\def \blambda {\boldsymbol{\lambda}}
\def \bPhi {\boldsymbol{\Phi}}
\def \btheta {\boldsymbol{\theta}}
\def \bmu {\boldsymbol{\mu}}
\def \bphi {\boldsymbol{\phi}}
\def \bSigma {\boldsymbol{\Sigma}}
\def \lb {\left\{}
\def \rb {\right\}}
\def \caln {\mathcal{N}}
\def \dissum {\displaystyle\Sigma}
\def \dispro {\displaystyle\prod}
\def \E {\mathbb{E}}
\def \Q {\mathbb{Q}}
\def \V {\mathbb{V}}
\def \R {\mathbb{R}}
\def \calq {\mathcal{Q}}
\def \calg {\mathcal{G}}
\def \caln {\mathcal{N}}
\def \calr {\mathcal{R}}
\def \calm {\mathcal{M}}
\def \calc {\mathcal{C}}
\def \bcup {\bigcup}

\makeindex
\author{Wolfram Pohlers}
\date{\today}
\title{Proof Theory \\ The First Step Into Impredicativity}
\hypersetup{
 pdfauthor={Wolfram Pohlers},
 pdftitle={Proof Theory \\ The First Step Into Impredicativity},
 pdfkeywords={},
 pdfsubject={},
 pdfcreator={Emacs 27.2 (Org mode 9.5)}, 
 pdflang={English}}
\begin{document}

\maketitle
\tableofcontents

\section{Primitive Recursive Functions and Relations}
\label{sec:orgd01ae9c}
\subsection{Primitive Recursive Functions}
\label{sec:org969e495}
\subsection{Primitive Recursive Relations}
\label{sec:orga48c28f}
\section{Ordinals}
\label{sec:org7c706fd}
\subsection{Some Basic Facts about Ordinals}
\label{sec:org42080ec}
\begin{equation*}
\alpha\in On\quad:\Leftrightarrow\quad Tran(\alpha)\wedge(\alpha,\in)\text{ is well-ordered}
\end{equation*}
where
\begin{equation*}
Tran(M)\quad:\Leftrightarrow\quad(\forall x\in M)(\forall y\in x)[y\in M]
\end{equation*}
\begin{proposition}[]
\(\alpha\in On\Rightarrow Tran(\alpha)\wedge(\forall x\in\alpha)[Tran(x)]\)
\end{proposition}

\begin{proof}
If \(z\in y\in x\in \alpha\), then \(z\in x\) because \(\alpha\) is well-ordered by \(\in\).
\end{proof}
so we have
\begin{equation*}
\alpha\in On\wedge x\in\alpha\Rightarrow x\in On, \text{ i.e., }Tran(On)
\end{equation*}
and obtain
\begin{equation*}
\alpha\in On\Rightarrow\alpha=\{\beta\mid\beta<\alpha\}
\end{equation*}

We assume that an ordinal is a transitive set

\(\alpha\) is \textbf{hereditarily transitive} iff \(Tran(\alpha)\wedge(\forall x\in\alpha)[Tran(x)]\)

\begin{lemma}[]
Assume that the membership relation \(\in\) well-founded. Then \(\alpha\) is an ordinal iff \(\alpha\) is a
hereditarily transitive set
\end{lemma}

\begin{proof}
Assume \(\alpha\) is hereditarily transitive. By the foundation scheme \(\in\) is irreflexive and
well-founded on \(\alpha\). Since \(\alpha\) is hereditarily transitive, it's also transitive. Assume \(\beta\) is also
hereditarily transitive, we show

\begin{quoting}
if \(\beta\) is well-ordered by \(\in\) and \(\alpha\subseteq\beta\) then \(\alpha=\beta\vee\alpha\in\beta\)
\end{quoting}
\end{proof}
\subsection{Fundatmentals of Ordinal Arithmetic}
\label{sec:orgc3a413b}
\end{document}
