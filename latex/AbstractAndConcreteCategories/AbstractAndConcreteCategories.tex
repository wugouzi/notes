% Created 2020-04-15 三 14:46
% Intended LaTeX compiler: pdflatex
\documentclass[11pt]{article}
\usepackage[utf8]{inputenc}
\usepackage[T1]{fontenc}
\usepackage{graphicx}
\usepackage{grffile}
\usepackage{longtable}
\usepackage{wrapfig}
\usepackage{rotating}
\usepackage[normalem]{ulem}
\usepackage{amsmath}
\usepackage{textcomp}
\usepackage{amssymb}
\usepackage{capt-of}
\usepackage{imakeidx}
\usepackage{hyperref}
\usepackage{minted}
% TIPS
% \substack{a\\b} for multiple lines text





% pdfplots will load xolor automatically without option
\usepackage[dvipsnames]{xcolor}

\usepackage{forest}
% two-line text in node by [two \\ lines]
% \begin{forest} qtree, [..] \end{forest}
\forestset{
  qtree/.style={
    baseline,
    for tree={
      parent anchor=south,
      child anchor=north,
      align=center,
      inner sep=1pt,
    }}}
%\usepackage{flexisym}
% load order of mathtools and mathabx, otherwise conflict overbrace

\usepackage{mathtools}
%\usepackage{fourier}
\usepackage{pgfplots}
\usepackage{amsthm, mathabx,  amsmath, commath}
\usepackage{amsfonts}

\usepackage{empheq}
\usepackage{tikz}
\usetikzlibrary{arrows.meta}
\usepackage[most]{tcolorbox}

\newtheorem{theorem}{Theorem}[section]
\newtheorem{definition}{Definition}[section]
\newtheorem{corollary}{Corollary}[section]
\newtheorem{example}{Example}[section]
\newtheorem{lemma}{Lemma}[section]
\newtheorem{proposition}{Proposition}[section]

\newcommand{\bl}[1] {\boldsymbol{#1}}
\newcommand{\Wt}[1] {\stackrel{\sim}{\smash{#1}\rule{0pt}{1.1ex}}}
\newcommand{\wt}[1] {\widetilde{#1}}


%For boxed texts in align, use Aboxed{}
%otherwise use boxed{}

\DeclareMathSymbol{\widehatsym}{\mathord}{largesymbols}{"62}
\newcommand\lowerwidehatsym{%
  \text{\smash{\raisebox{-1.3ex}{%
    $\widehatsym$}}}}
\newcommand\fixwidehat[1]{%
  \mathchoice
    {\accentset{\displaystyle\lowerwidehatsym}{#1}}
    {\accentset{\textstyle\lowerwidehatsym}{#1}}
    {\accentset{\scriptstyle\lowerwidehatsym}{#1}}
    {\accentset{\scriptscriptstyle\lowerwidehatsym}{#1}}
}

\usepackage{graphicx}
    
% text on arrow for xRightarrow
\makeatletter
%\newcommand{\xRightarrow}[2][]{\ext@arrow 0359\Rightarrowfill@{#1}{#2}}
\makeatother


\def \bx {\boldsymbol{x}}
\def \ba {\boldsymbol{a}}
\def \bI {\boldsymbol{I}}
\def \bt {\boldsymbol{t}}
\def \bb {\boldsymbol{b}}
\def \bA {\boldsymbol{A}}
\def \bX {\boldsymbol{X}}
\def \bu {\boldsymbol{u}}
\def \bS {\boldsymbol{S}}
\def \bZ {\boldsymbol{Z}}
\def \bz {\boldsymbol{z}}
\def \by {\boldsymbol{y}}
\def \bw {\boldsymbol{w}}
\def \bT {\boldsymbol{T}}
\def \bS {\boldsymbol{S}}
\def \bm {\boldsymbol{m}}
\def \bW {\boldsymbol{W}}
\def \bY {\boldsymbol{Y}}
\def \bH {\boldsymbol{H}}
\def \blambda {\boldsymbol{\lambda}}
\def \bPhi {\boldsymbol{\Phi}}
\def \btheta {\boldsymbol{\theta}}
\def \bmu {\boldsymbol{\mu}}
\def \bphi {\boldsymbol{\phi}}
\def \bSigma {\boldsymbol{\Sigma}}
\def \lb {\left\{}
\def \rb {\right\}}
\def \caln {\mathcal{N}}
\def \dissum {\displaystyle\Sigma}
\def \dispro {\displaystyle\prod}
\def \E {\mathbb{E}}
\def \Q {\mathbb{Q}}
\def \V {\mathbb{V}}
\def \R {\mathbb{R}}
\def \calq {\mathcal{Q}}
\def \calg {\mathcal{G}}
\def \caln {\mathcal{N}}
\def \calr {\mathcal{R}}
\def \calm {\mathcal{M}}
\def \calc {\mathcal{C}}
\def \bcup {\bigcup}

\graphicspath{{../../images/CAT/}}
\author{Jiří Adámek \& Horst Herrlich \& George E. Strecker}
\date{\today}
\title{\aunclfamily\Huge Abstract and Concrete Categories \\ The Joy of Cats \\ \includegraphics[scale=1.2]{cat.png}}
\hypersetup{
 pdfauthor={Jiří Adámek \& Horst Herrlich \& George E. Strecker},
 pdftitle={\aunclfamily\Huge Abstract and Concrete Categories \\ The Joy of Cats \\ \includegraphics[scale=1.2]{cat.png}},
 pdfkeywords={},
 pdfsubject={},
 pdfcreator={Emacs 26.3 (Org mode 9.3.6)}, 
 pdflang={English}}
\begin{document}

\maketitle \clearpage
\setcounter{tocdepth}{2}
\tableofcontents \clearpage\section{Categories, Functors, and Natural Transformations}
\label{sec:org7de109f}
\subsection{Categories and Functors}
\label{sec:org28b13b6}
\subsubsection{Categories}
\label{sec:orgf070b17}
\begin{definition}[]
A \textbf{category} is a quadruple \(\bA=(\calo,\hom,id,\circ)\) consisting
\begin{enumerate}
\item a class \(\calo\), whose members are called \textbf{\(\bA\)-objects}
\item for each pair \((A,B)\) of \(\bA\)-objects, a set \(\hom(A,B)\) whose
members are called \textbf{\(\bA\)-morphisms from \(A\) to \(B\)}
\end{enumerate}
\end{definition}

\begin{examplle}[]
\begin{enumerate}
\item The following \textbf{constructs}; i.e., categories of structured sets and
structure-preserving functions between them
\begin{enumerate}
\item \(\Alg(\Omega)\) with objects all \textbf{\(\Omega\)-algebras} and morphisms all \par
\textbf{\(\Omega\)-homomorphisms} between them. Here \(\Omega=(n_i)_{i\in I}\) is a
family of natural numbers \(n_i\), indexed by a set \(I\). An
\(\Omega\)-algebra is a pair \(X,(\omega_i)_{i\in I}\) consisting of a set
\(X\) and a family of functions \(\omega_i:X^{n_i}\to X\), called \textbf{\(n_i\)-ary
operations} on \(X\). An \(\Omega\)-homomorphism \(f:(X,(\omega_i)_{i\in
         I}\to(\what{X},(\what{\omega}_i)_{i\in I})\) is a function \(f:X\to\what{X}\) for
which the diagram
\begin{center}
\begin{tikzcd}
X^{n_i}\arrow[r,"f^{n_i}"]\arrow[d,"\omega_i"']&
\what{X}^{n_i}\arrow[d,"\what{\omega}_i"]\\
X\arrow[r,"f"']&\what{X}
\end{tikzcd}
\end{center}
commutes for each \(i\in I\).
\item \textbf{\(\Sigma\)-Seq} with objects all (deterministic, sequential)
\textbf{\(\Sigma\)-acceptor}, where \(\Sigma\) is a finite set of input symbols. Objects
are quadruples \((Q,\delta,q_0,F)\) where \(Q\) is a finite set of states, 
\(\delta:\Sigma\times Q\to Q\) is a transition map, \(q_0\in Q\) is the
initial state, and \(F\subseteq Q\) is the set of final states.

A morphism \(f:(Q,\delta,q_0,F)\to(Q',\delta',q_0',F')\) (called a
\textbf{simulation}) is a function \(f:Q\to Q'\) that preserves
\begin{enumerate}
\item transitions, i.e., \(\delta'(\sigma,f(q))=f(\delta(\sigma,q))\)
\item the initial state, i.e., \(f(q_0)=q_0'\)
\item the final states, i.e., \(f[F]\subseteq F'\)
\end{enumerate}
\end{enumerate}
\item The following categories where the objects and morphisms are \emph{not}
constructed sets and structure-preserving functions:
\begin{enumerate}
\item \(\Mat\) with objects all natural numbers, and for which \(\hom(m,n)\) is
the set of all real \(m\times n\) matrices, \(id_n:n\to n\) is the unit
diagonal matrix, and composition is defined by \(A\circ B=BA\)

\item \(\Aut\) with objects all (deterministic, sequential, Moore) \textbf{automata}.
Objects are sectuples \((Q,\Sigma,Y,\delta,q_0,y)\), where \(Q\) is the set of
states, \(\Sigma\) and \(Y\) are the sets of input symbols and output symbols,
respectively, \(\delta:\Sigma\times Q\to Q\) is the transition map, 
\(q_0\in Q\) is the initial state, and \(y:Q\to Y\) is the output map.
Morphisms from an automaton \((Q,\Sigma,Y,\delta,q_0,y)\) to an automaton
\((Q',\Sigma',Y',\delta',q_0',y')\) are triples \((f_Q,f_{\Sigma},f_Y)\) of
functions satisfying the following conditions
\begin{enumerate}
\item preservation of transitions:
\(\delta'(f_{\Sigma}(\sigma),f_Q(q))=f_Q(\delta(\sigma,q))\)

\item preservation of outputs: \(f_Y(y(q))=y'(f_Q(q))\)

\item preservation of initial state: \(f_Q(q_0)=q_0'\)
\end{enumerate}
\end{enumerate}
\end{enumerate}
\end{examplle}
\subsection{The Dual Principle}
\label{sec:org9ecfb95}
\index{dual category}
\begin{definition}[]
For any category \(\bA=(\calo,\hom_{\bA},id,\circ)\) the \textbf{dual} (or \textbf{opposite})
\textbf{category of \(\bA\)} is the category
\(\bA^{\op}=(\calo,\hom_{\bA^{\op}},id,\circ^{\op})\), where
\(\hom_{\bA^{\op}}(A,B)=\hom_{\bA}(B,A)\) and \(f\circ^{\op}g=g\circ f\)
\end{definition}

Consider the property of objects \(X\) in \(\bA\):
\begin{equation*}
\calp_{\bA}(X)\equiv \textit{ For any } \bA\textit{-object } A
\textit{ there exists exactly one }
\bA\textit{-morphism } f:A\to X
\end{equation*}

Step1: In \(\calp_{\bA}(X)\) replace all occurrences of \(\bA\) by \(\bA^{\op}\),
thus obtaining the property
\begin{equation*}
\calp_{\bA^{\op}}(X)\equiv \textit{ For any } \bA^{\op}\textit{-object } A
\textit{ there exists exactly one }
\bA^{\op}\textit{-morphism } f:A\to X
\end{equation*}

Step2: Translate \(\calp_{\bA^{\op}}(X)\) into the logically equivalent
statement
\begin{equation*}
\calp_{\bA}^{\op}(X)\equiv \textit{ For any } \bA\textit{-object } A
\textit{ there exists exactly one }
\bA\textit{-morphism } f:X\to A
\end{equation*}

The \textbf{Duality Principle for Categories} states
\begin{center}
\textit{Whenever a property \(\calp\) holds for all categories,}\\
\textit{then the property \(\calp^{\op}\) holds for all categories.}
\end{center}
\section{Index}
\label{sec:orgd7312ed}
\renewcommand{\indexname}{}
\printindex
\end{document}