% Created 2020-10-02 五 19:12
% Intended LaTeX compiler: pdflatex
\documentclass[11pt]{article}
\usepackage[utf8]{inputenc}
\usepackage[T1]{fontenc}
\usepackage{graphicx}
\usepackage{grffile}
\usepackage{longtable}
\usepackage{wrapfig}
\usepackage{rotating}
\usepackage[normalem]{ulem}
\usepackage{amsmath}
\usepackage{textcomp}
\usepackage{amssymb}
\usepackage{capt-of}
\usepackage{hyperref}
\usepackage{minted}
% TIPS
% \substack{a\\b} for multiple lines text





% pdfplots will load xolor automatically without option
\usepackage[dvipsnames]{xcolor}

\usepackage{forest}
% two-line text in node by [two \\ lines]
% \begin{forest} qtree, [..] \end{forest}
\forestset{
  qtree/.style={
    baseline,
    for tree={
      parent anchor=south,
      child anchor=north,
      align=center,
      inner sep=1pt,
    }}}
%\usepackage{flexisym}
% load order of mathtools and mathabx, otherwise conflict overbrace

\usepackage{mathtools}
%\usepackage{fourier}
\usepackage{pgfplots}
\usepackage{amsthm, mathabx,  amsmath, commath}
\usepackage{amsfonts}

\usepackage{empheq}
\usepackage{tikz}
\usetikzlibrary{arrows.meta}
\usepackage[most]{tcolorbox}

\newtheorem{theorem}{Theorem}[section]
\newtheorem{definition}{Definition}[section]
\newtheorem{corollary}{Corollary}[section]
\newtheorem{example}{Example}[section]
\newtheorem{lemma}{Lemma}[section]
\newtheorem{proposition}{Proposition}[section]

\newcommand{\bl}[1] {\boldsymbol{#1}}
\newcommand{\Wt}[1] {\stackrel{\sim}{\smash{#1}\rule{0pt}{1.1ex}}}
\newcommand{\wt}[1] {\widetilde{#1}}


%For boxed texts in align, use Aboxed{}
%otherwise use boxed{}

\DeclareMathSymbol{\widehatsym}{\mathord}{largesymbols}{"62}
\newcommand\lowerwidehatsym{%
  \text{\smash{\raisebox{-1.3ex}{%
    $\widehatsym$}}}}
\newcommand\fixwidehat[1]{%
  \mathchoice
    {\accentset{\displaystyle\lowerwidehatsym}{#1}}
    {\accentset{\textstyle\lowerwidehatsym}{#1}}
    {\accentset{\scriptstyle\lowerwidehatsym}{#1}}
    {\accentset{\scriptscriptstyle\lowerwidehatsym}{#1}}
}

\usepackage{graphicx}
    
% text on arrow for xRightarrow
\makeatletter
%\newcommand{\xRightarrow}[2][]{\ext@arrow 0359\Rightarrowfill@{#1}{#2}}
\makeatother


\def \bx {\boldsymbol{x}}
\def \ba {\boldsymbol{a}}
\def \bI {\boldsymbol{I}}
\def \bt {\boldsymbol{t}}
\def \bb {\boldsymbol{b}}
\def \bA {\boldsymbol{A}}
\def \bX {\boldsymbol{X}}
\def \bu {\boldsymbol{u}}
\def \bS {\boldsymbol{S}}
\def \bZ {\boldsymbol{Z}}
\def \bz {\boldsymbol{z}}
\def \by {\boldsymbol{y}}
\def \bw {\boldsymbol{w}}
\def \bT {\boldsymbol{T}}
\def \bS {\boldsymbol{S}}
\def \bm {\boldsymbol{m}}
\def \bW {\boldsymbol{W}}
\def \bY {\boldsymbol{Y}}
\def \bH {\boldsymbol{H}}
\def \blambda {\boldsymbol{\lambda}}
\def \bPhi {\boldsymbol{\Phi}}
\def \btheta {\boldsymbol{\theta}}
\def \bmu {\boldsymbol{\mu}}
\def \bphi {\boldsymbol{\phi}}
\def \bSigma {\boldsymbol{\Sigma}}
\def \lb {\left\{}
\def \rb {\right\}}
\def \caln {\mathcal{N}}
\def \dissum {\displaystyle\Sigma}
\def \dispro {\displaystyle\prod}
\def \E {\mathbb{E}}
\def \Q {\mathbb{Q}}
\def \V {\mathbb{V}}
\def \R {\mathbb{R}}
\def \calq {\mathcal{Q}}
\def \calg {\mathcal{G}}
\def \caln {\mathcal{N}}
\def \calr {\mathcal{R}}
\def \calm {\mathcal{M}}
\def \calc {\mathcal{C}}
\def \bcup {\bigcup}

\graphicspath{{../../images/CAT/}}
\DeclareMathOperator{\Rel}{\textbf{Rel}}
\DeclareMathOperator{\Sym}{\textbf{Sym}}
\author{Jiří Adámek \& Horst Herrlich \& George E. Strecker}
\date{\today}
\title{\aunclfamily\Huge Abstract and Concrete Categories \\ The Joy of Cats \\ \includegraphics[scale=1.2]{cat.png}}
\hypersetup{
 pdfauthor={Jiří Adámek \& Horst Herrlich \& George E. Strecker},
 pdftitle={\aunclfamily\Huge Abstract and Concrete Categories \\ The Joy of Cats \\ \includegraphics[scale=1.2]{cat.png}},
 pdfkeywords={},
 pdfsubject={},
 pdfcreator={Emacs 26.3 (Org mode 9.4)}, 
 pdflang={English}}
\begin{document}

\maketitle
\setcounter{tocdepth}{2}
\tableofcontents \clearpage\section{Categories, Functors, and Natural Transformations}
\label{sec:org367830c}
\subsection{Categories and Functors}
\label{sec:org0d7482a}
\subsubsection{Categories}
\label{sec:orgeb4f845}
\begin{definition}[]
A \textbf{category} is a quadruple \(\bA=(\calo,\hom,id,\circ)\) consisting
\begin{enumerate}
\item a class \(\calo\), whose members are called \textbf{\(\bA\)-objects}
\item for each pair \((A,B)\) of \(\bA\)-objects, a set \(\hom(A,B)\) whose
members are called \textbf{\(\bA\)-morphisms from \(A\) to \(B\)}
\end{enumerate}
\end{definition}

\begin{examplle}[]
\begin{enumerate}
\item The following \textbf{constructs}; i.e., categories of structured sets and
structure-preserving functions between them
\begin{enumerate}
\item \(\Alg(\Omega)\) with objects all \textbf{\(\Omega\)-algebras} and morphisms all \par
\textbf{\(\Omega\)-homomorphisms} between them. Here \(\Omega=(n_i)_{i\in I}\) is a
family of natural numbers \(n_i\), indexed by a set \(I\). An
\(\Omega\)-algebra is a pair \(X,(\omega_i)_{i\in I}\) consisting of a set
\(X\) and a family of functions \(\omega_i:X^{n_i}\to X\), called \textbf{\(n_i\)-ary
operations} on \(X\). An \(\Omega\)-homomorphism \(f:(X,(\omega_i)_{i\in
         I}\to(\widehat{X},(\widehat{\omega}_i)_{i\in I})\) is a function \(f:X\to\widehat{X}\) for
which the diagram
\begin{center}
\begin{tikzcd}
X^{n_i}\arrow[r,"f^{n_i}"]\arrow[d,"\omega_i"']&
\widehat{X}^{n_i}\arrow[d,"\widehat{\omega}_i"]\\
X\arrow[r,"f"']&\widehat{X}
\end{tikzcd}
\end{center}
commutes for each \(i\in I\).
\item \textbf{\(\Sigma\)-Seq} with objects all (deterministic, sequential)
\textbf{\(\Sigma\)-acceptor}, where \(\Sigma\) is a finite set of input symbols. Objects
are quadruples \((Q,\delta,q_0,F)\) where \(Q\) is a finite set of states, 
\(\delta:\Sigma\times Q\to Q\) is a transition map, \(q_0\in Q\) is the
initial state, and \(F\subseteq Q\) is the set of final states.

A morphism \(f:(Q,\delta,q_0,F)\to(Q',\delta',q_0',F')\) (called a
\textbf{simulation}) is a function \(f:Q\to Q'\) that preserves
\begin{enumerate}
\item transitions, i.e., \(\delta'(\sigma,f(q))=f(\delta(\sigma,q))\)
\item the initial state, i.e., \(f(q_0)=q_0'\)
\item the final states, i.e., \(f[F]\subseteq F'\)
\end{enumerate}
\end{enumerate}
\item The following categories where the objects and morphisms are \emph{not}
constructed sets and structure-preserving functions:
\begin{enumerate}
\item \(\Mat\) with objects all natural numbers, and for which \(\hom(m,n)\) is
the set of all real \(m\times n\) matrices, \(id_n:n\to n\) is the unit
diagonal matrix, and composition is defined by \(A\circ B=BA\)

\item \(\Aut\) with objects all (deterministic, sequential, Moore) \textbf{automata}.
Objects are sectuples \((Q,\Sigma,Y,\delta,q_0,y)\), where \(Q\) is the set of
states, \(\Sigma\) and \(Y\) are the sets of input symbols and output symbols,
respectively, \(\delta:\Sigma\times Q\to Q\) is the transition map, 
\(q_0\in Q\) is the initial state, and \(y:Q\to Y\) is the output map.
Morphisms from an automaton \((Q,\Sigma,Y,\delta,q_0,y)\) to an automaton
\((Q',\Sigma',Y',\delta',q_0',y')\) are triples \((f_Q,f_{\Sigma},f_Y)\) of
functions satisfying the following conditions
\begin{enumerate}
\item preservation of transitions:
\(\delta'(f_{\Sigma}(\sigma),f_Q(q))=f_Q(\delta(\sigma,q))\)

\item preservation of outputs: \(f_Y(y(q))=y'(f_Q(q))\)

\item preservation of initial state: \(f_Q(q_0)=q_0'\)
\end{enumerate}
\end{enumerate}
\end{enumerate}
\end{examplle}
\subsubsection{The Dual Principle}
\label{sec:orgd156492}
\index{dual category}
\begin{definition}[]
For any category \(\bA=(\calo,\hom_{\bA},id,\circ)\) the \textbf{dual} (or \textbf{opposite})
\textbf{category of \(\bA\)} is the category
\(\bA^{\op}=(\calo,\hom_{\bA^{\op}},id,\circ^{\op})\), where
\(\hom_{\bA^{\op}}(A,B)=\hom_{\bA}(B,A)\) and \(f\circ^{\op}g=g\circ f\)
\end{definition}

Consider the property of objects \(X\) in \(\bA\):
\begin{equation*}
\calp_{\bA}(X)\equiv \textit{ For any } \bA\textit{-object } A
\textit{ there exists exactly one }
\bA\textit{-morphism } f:A\to X
\end{equation*}

Step1: In \(\calp_{\bA}(X)\) replace all occurrences of \(\bA\) by \(\bA^{\op}\),
thus obtaining the property
\begin{equation*}
\calp_{\bA^{\op}}(X)\equiv \textit{ For any } \bA^{\op}\textit{-object } A
\textit{ there exists exactly one }
\bA^{\op}\textit{-morphism } f:A\to X
\end{equation*}

Step2: Translate \(\calp_{\bA^{\op}}(X)\) into the logically equivalent
statement
\begin{equation*}
\calp_{\bA}^{\op}(X)\equiv \textit{ For any } \bA\textit{-object } A
\textit{ there exists exactly one }
\bA\textit{-morphism } f:X\to A
\end{equation*}

The \textbf{Duality Principle for Categories} states
\begin{center}
\textit{Whenever a property \(\calp\) holds for all categories,}\\
\textit{then the property \(\calp^{\op}\) holds for all categories.}
\end{center}
\subsubsection{Isomorphism}
\label{sec:orgd91d43e}
\begin{definition}[]
A morphism \(f:A\to B\) in a category is called an \textbf{isomorphism} provided that
there exists a morphism \(g:B\to A\) with \(g\circ f=id_A\) and \(f\circ
   g=id_B\). Such a morphism \(g\) is called an \textbf{inverse} of \(f\)
\end{definition}

\begin{proposition}[]
If \(f:A\to B,g:B\to A,h:B\to A\) are morphisms s.t. \(g\circ f=id_A\) and
\(f\circ h=id_B\), then \(g=h\)
\end{proposition}

\begin{definition}[]
Let \(F:\bA\to\bB\)  be a functor
\begin{enumerate}
\item \(F\) is called an \textbf{embedding} provided that \(F\) is injective on morphisms
\item \(F\) is called \textbf{faithful} provided that all the hom-set restrictions
\begin{equation*}
F:\hom_{\bA}(A,A')\to\hom_{\bB}(FA,FA')
\end{equation*}
are injective
\item \(F\) is called \textbf{full} provided that all hom-set restrictions are surjective
\item \(F\) is called \textbf{amnestic} provided that an \(\bA\)-isomorphism \(f\) is an
identity whenever \(Ff\) is an identity
\end{enumerate}
\end{definition}

\subsection{Subcategories}
\label{sec:org5b0e140}
\begin{definition}[]
A category \(\bA\) is said to be a \textbf{subcategory} of a category \(\bB\) provided
that the following conditions are satisfied
\begin{enumerate}
\item \(Ob(\bA)\subseteq Ob(\bB)\)
\item for each \(A,A'\in Ob(\bA)\), \(\hom_{\bA}(A,A')\subseteq\hom_{\bB}(A,A')\)
\item for each \(\bA\)-object \(A\), the \(\bB\)-identity on \(A\) is the
\(\bA\)-identity on \(A\)
\item the composition law in \(\bA\) is the restriction of the composition law
in \(\bB\) to the morphisms of \(\bA\)
\end{enumerate}


\(\bA\) is called a \textbf{full subcategory} of \(\bB\) if in addition to the above,
for each \(A,A'\in Ob(A)\), \(\hom_{\bA}(A,A')=\hom_{\bB}(A,A')\)
\end{definition}

\begin{proposition}[]
\begin{enumerate}
\item A functor \(F:\bA\to\bB\) is a (full) embedding if and only if there exists a
(full) subcategory \(\bC\) of \(\bB\) with inclusion function \(E:\bC\to\bB\)
and an isomorphism \(G:\bA\to\bC\) with \(F=E\circ G\)

\item A functor \(F:\bA\to\bB\) is faithful iff there exists embeddings
\(E_1:\bD\to\bB\) and \(E_2:\bA\to\bC\) and an equivalence \(G:\bC\to\bD\)
s.t. the diagram
\begin{center}\begin{tikzcd}
\bA\arrow[r,"F"]\arrow[d,"E_2"]&\bB\\
\bC\arrow[r,"G"]&\bD\arrow[u,"E_1"]
\end{tikzcd}\end{center}
\end{enumerate}
\end{proposition}

\begin{proof}
\begin{enumerate}
\item Let \(E_1:\bD\to\bB\) be the inclusion of the full subcategory \(\bD\) of
\(\bB\) that has as objects all images of \(\bA\)-objects. Let \(\bC\) be
the category with \(Ob(\bC)=Ob(\bA)\), with
\begin{equation*}
\hom_{\bC}(A,A')=\hom_{\bB}(FA,FA')
\end{equation*}
Now define functors \(E_2:\bA\to\bC\) and \(G:\bC\to\bD\) by
\begin{equation*}
E_2(A\xrightarrow{f}A')=A\xrightarrow{Ff}A' \quad\text{ and }\quad
G(C\xrightarrow{g}C')=FC\xrightarrow{g}FC'
\end{equation*}
Then \(E_2\) is an embedding, \(G\) is an equivalence and \(F=E_1\circ
      G\circ E_2\)
\end{enumerate}
\end{proof}

\begin{definition}[]
A category \(\bA\) is said to be \textbf{fully embeddable} into \(\bB\) provided that
there exists a full embedding \(\bA\to\bB\)
\end{definition}

\begin{definition}[]
A full subcategory \(\bA\) of a category \(\bB\) is called
\begin{enumerate}
\item \textbf{isomorphism-closed} provided that every \(\bB\)-object that is isomorphic
to some \(\bA\)-object is itself an \(\bA\)-object
\item \textbf{isomorphism-dense} provided that every \(\bB\)-object is isomorphic to
\end{enumerate}
some \(\bA\)-object
\end{definition}

\begin{remark}
If \(\bA\) is a full subcategory of \(\bB\), then the following conditions
are equivalent
\begin{enumerate}
\item \(\bA\) is an isomorphism-dense subcategory of \(\bB\)
\item the inclusion functor \(\bA\hookrightarrow\bB\) is isomorphism-dense
\item the inclusion functor \(\bA\hookrightarrow\bB\) is an equivalence
\end{enumerate}
\end{remark}

\begin{examplle}[]
The full subcategory of \(\Set\) with the single object \(\N\) is neither
isomorphism-closed nor isomorphism-dense in \(\Set\). It is equivalent to the
isomorphism-closed full subcategory of \(\Set\) consisting of all countable
infinite sets.
\end{examplle}

\begin{definition}[]
A \textbf{skeleton} of a category is a full, isomorphism-dense subcategory in which no
two distinct objects are isomorphic
\end{definition}

\begin{examplle}[]
\begin{enumerate}
\item The full subcategory of all cardinal numbers is a skeleton for \(\Set\)
\end{enumerate}
\end{examplle}

\begin{proposition}[]
\begin{enumerate}
\item Every category has a skeleton
\item Any two skeletons of a category are isomorphic
\item Any skeleton of a category \(\bC\) is equivalent to \(\bC\)
\end{enumerate}
\end{proposition}

\begin{proof}
\begin{enumerate}
\item This follows from the Axiom of Choice applied to the equivalence relation
``is isomorphic to'' on the class of objects of the category
\end{enumerate}
\end{proof}

\begin{corollary}[]
Two categories are equivalent iff they have isomorphic skeletons
\end{corollary}

\begin{definition}[]
Let \(\bA\) be a subcategory of \(\bB\), and let \(B\) be a \(\bB\)-object
\begin{enumerate}
\item An \textbf{\(\bA\)-reflection} (or \textbf{\(\bA\)-reflection arrow}) for \(B\) is a
\(\bB\)-morphism \(B\xrightarrow{r}A\) from \(B\) to an \(\bA\)-object
\(A\) with the following universal property:

for any \(\bB\)-morphism \(B\xrightarrow{f}A'\) from \(B\) into some
\(\bA\)-object \(A'\), there exists a unique \(\bA\)-morphism \(f':A\to
      A\) s.t. the triangle
\begin{center}\begin{tikzcd}
B\arrow[r,"r"]\arrow[dr,"f"]&A\arrow[d,"f'"]\\
&A'
\end{tikzcd}\end{center}
commutes
\item \(\bA\) is called a \textbf{reflective subcategory} of \(\bB\) provided that each
\(\bB\)-object has an \(\bA\)-reflection
\end{enumerate}
\end{definition}

\begin{examplle}[]
\begin{enumerate}
\item \textbf{Modifications of the Structure}
\begin{enumerate}
\item Making a relation symmetric: \(\bB=\Rel,\bA=\Sym\), the full subcategory
of symmetric relations, \((X,\rho)\xrightarrow{id_X}(X,\rho\cup\rho^{-1})\)
is an \(\bA\)-reflection for \((X,\rho)\)
\end{enumerate}
\item \textbf{Improving Objects by Forming Quotients}
\begin{enumerate}
\item Making a reachable acceptor minimal: \(\bB=\) the full subcategory of
\textbf{\(\Sigma\)-Seq} consisting of all \textbf{reachable acceptors} (i.e., those for
which each state can be reached from the initial one by an input word),
\(\bA=\) the full subcategory of \(\bB\) consisting of all \textbf{minimal
acceptors} (i.e. those reachable acceptors with the property that no two
different states are \textbf{observably equivalent}. The observability
equivalence \(\approx\) on a reachable acceptor \(B\) is given by:
\(q\approx q'\) provided that whenever the initial state of \(B\) is
changed to \(q\), the resulting acceptor recognizes the same language
as it does when the initial state is changed to \(q'\)). Then the
canonical map \(B\to B/\approx\) is an \(\bA\)-reflection for \(B\)
\end{enumerate}
\item \textbf{Completions}
\end{enumerate}
\end{examplle}

\begin{proposition}[]
Reflections are essentially unique, i.e.
\begin{enumerate}
\item if \(B\xrightarrow{r}A\) and \(B\xrightarrow{\hat{r}}\hat{A}\) are
\(\bA\)-reflections for \(B\), then there exists an \(\bA\)-isomorphism
\(k:A\to\hat{A}\) s.t. the triangle
\begin{center}\begin{tikzcd}
B\arrow[r,"r"]\arrow[rd,"\hat{r}"']&A\arrow[d,"k"]\\
&\hat{A}
\end{tikzcd}\end{center}
commutes
\item if \(B\xrightarrow{r}A\) is an \(\bA\)-reflection for \(B\) and
\(A\xrightarrow{k}\hat{A}\) is an \(\bA\)-isomorphism, then
\(B\xrightarrow{k\circ r}\hat{A}\) is an \(\bA\)-reflection for \(B\)
\end{enumerate}
\end{proposition}

\begin{proposition}[]
\label{prop4.20}
If \(\bA\) is reflective subcategory of \(\bB\), then the following
conditions are equivalent
\begin{enumerate}
\item \(\bA\) is a full subcategory of \(\bB\)
\item for each \(\bA\)-object \(A\), \(A\xrightarrow{id_A}A\) is an \(\bA\)-reflection
\item for each \(\bA\)-object \(A\), \(\bA\)-reflection arrows
\(A\xrightarrow{r_A}A^*\) are \(\bA\)-isomorphism
\item for each \(\bA\)-object \(A\), \(\bA\)-reflection arrows
\(A\xrightarrow{r_A}A^*\) are \(\bA\)-morphisms
\end{enumerate}
\end{proposition}

\begin{proof}
\(2\to3\).
\begin{center}\begin{tikzcd}
A\arrow[r,"r_A"]\arrow[ddr,"r_A"']&A^*\arrow[d,"f"]\\
&A\arrow[d,"r_A"]\\
&A^*
\end{tikzcd}\end{center}
\end{proof}

\begin{proposition}[]
Let \(\bA\) be a reflective subcategory of \(\bB\), and for each
\(\bB\)-object \(B\) let \(r_B:B\to A_B\) be an \(\bA\)-reflection arrow.
Then there exists a unique functor \(R:\bB\to\bA\) s.t. the following
conditions are satisfied
\begin{enumerate}
\item \(R(B)=A_B\) for each \(\bB\)-object \(B\)
\item for each \(\bB\)-morphism \(f:B\to B'\) the diagram
\begin{center}\begin{tikzcd}
B\arrow[r,"r_B"]\arrow[d,"f"']&R(B)\arrow[d,"R(f)"]\\
B'\arrow[r,"r_{B'}"']&R(B')
\end{tikzcd}\end{center}
commutes
\end{enumerate}
\end{proposition}

\begin{proof}
Show that functor is well-defined and preserves identities and compositions
\end{proof}

\begin{definition}[]
A functor \(R:\bB\to\bA\) constructed according to the above proposition is
called a \textbf{reflector for \(\bA\)}
\end{definition}

\begin{definition}[]
Let \(\bA\) be a subcategory of \(\bB\) and let \(B\) be a \(\bB\)-object
\begin{enumerate}
\item An \textbf{\(\bA\)-coreflection} (or \textbf{\(A\)-coreflection arrow}) for \(B\) is a
\(\bB\)-morphism \(A\xrightarrow{c}B\) form an \(\bA\)-object \(A\) to
\(B\) with the following universal property:

for any \(\bB\)-morphism \(A'\xrightarrow{f}B\) from some \(\bA\)-object
\(A'\) to \(B\) there exists a unique \(\bA\)-morphism \(f':A'\to A\) s.t.
the triangle
\begin{center}\begin{tikzcd}
A'\arrow[d,"f'"']\arrow[dr,"f"]\\
A\arrow[r,"c"']&B
\end{tikzcd}\end{center}
commutes

\item \(\bA\) is called a \textbf{coreflective subcategory} of \(\bB\) provided that each
\(\bB\)-object has an \(\bA\)-coreflection
\end{enumerate}
\end{definition}

\begin{proposition}[]
If \(\bA\) is a coreflective subcategory of \(\bB\) and for each
\(\bB\)-object \(B\), \(A_B\xrightarrow{c_B}B\) is an \(\bA\)-coreflection
arrow, then there exists a unique functor \(C:\bB\to\bA\) (called a
\textbf{coreflector for} \(\bA\)) s.t. the following conditions are satisfied
\begin{enumerate}
\item \(C(B)=A_B\) for each \(\bB\)-object \(B\)
\item for each \(\bB\)-morphism \(f:B\to B'\) the diagram
\begin{center}\begin{tikzcd}
C(B)\arrow[r,"c_B"]\arrow[d,"C(f)"']&
B\arrow[d,"f"]\\
C(B')\arrow[r,"c_{B'}"']&B'
\end{tikzcd}\end{center}
commutes
\end{enumerate}
\end{proposition}

\begin{exercise}
A subcategory \(\bA\) of a category \(\bB\) is called \textbf{isomorphism-closed}
provided that every \(\bB\)-isomorphism with domain in \(\bA\) belongs to
\(\bA\). Show that every subcategory \(\bA\) of \(\bB\) can be embedded into
a smallest isomorphism-closed subcategory \(\bA'\) of \(\bB\) that contains
\(\bA\). The inclusion functor \(\bA\hookrightarrow\bA'\) is an equivalence
iff all \(\bB\)-isomorphisms between \(\bA\)-objects belong to \(\bA\)
\end{exercise}

\begin{exercise}
\begin{enumerate}
\item Show that a category is discrete iff each of its subcategories is full
\item Show that in a poset, considered as a category
\begin{itemize}
\item every subcategory is isomorphism-closed
\item every (co)reflective subcategory is full
\end{itemize}
\end{enumerate}
\end{exercise}
\subsection{Concrete categories and concrete functors}
\label{sec:orgefbffb3}
\begin{definition}[]
Let \(\bX\) be a category. A \textbf{concrete category} over \(\bX\) is a pair
\((\bA,U)\) where \(\bA\) is the category and \(U:\bA\to\bX\) is a faithful
functors. Sometimes \(U\) is called the \textbf{forgetful} (or \textbf{underlying}) \textbf{functor} of
the concrete category and \(\bX\) is called the \textbf{base category} for \((\bA,U)\)

A concrete category over \(\Set\) is called a \textbf{construct}
\end{definition}

\begin{remark}
We adopt the following conventions:
\begin{enumerate}
\item Since faithful functors are injective on hom-sets, we usually assume that
\(\hom_{\bA}(A,B)\) is a subset of \(\hom_{\bX}(UA,UB)\) for each pair
\((A,B)\) of \(\bA\)-objects. This allows one to express the property that
``for \(bA\)-objects \(A\) and \(B\) and an \(\bX\)-morphism
\(UA\xrightarrow{f}UB\) there exists a (necessarily unique)
\(\bA\)-morphism \(A\to B\) with \(U(A\to B)=UA\xrightarrow{f}UB\)'' much
more succinctly, by stating
\begin{equation*}
UA\xrightarrow{f}UB\text{ is an $\bA$-morphism (from $A$ to $B$)}
\end{equation*}
Observe, however, that since \(U\) doesn't need to be injective on
objects, the expression
\begin{equation*}
UA\xrightarrow{id_X}UB\text{ is an $\bA$-morphism (from $A$ to $B$)}
\end{equation*}
does not imply that \(A=B\) or that \(id_X=id_A\), although it does imply
that \(UA=UB=X\). We call an \(\bA\)-morphism \(A\xrightarrow{f}B\)
\textbf{identity-carried} if \(Uf=id_X\)
\item Sometimes we will write \(\bA\) for the concrete category \((\bA,U)\) over
\(\bX\), when \(U\) is clear from the context. In these cases the
underlying object of an \(\bA\)-object \(A\) will sometimes be denoted by \(\abs{A}\)
\item If \(P\) is a property of categories (or of functors), then we will say
that a concrete category \((\bA,U)\) \textbf{has property} \(P\) provided that
\(\bA\) (or \(U\)) has property \(P\)
\end{enumerate}
\end{remark}

\begin{definition}[]
Let \((\bA,U)\) be a concrete category over \(\bX\)
\begin{enumerate}
\item The \textbf{fibre} of an \(\bX\)-object \(X\) is the preordered class consisting of
all \(\bA\)-objects \(A\) with \(U(A)=X\) ordered by
\begin{equation*}
A\le B \quad\text{ iff }\quad
id_X:UA\to UB\text{ is an $\bA$-morphism}
\end{equation*}
\item \(\bA\)-objects \(A,B\) are \textbf{equivalent} if \(A\le B\) and \(B\le A\)
\item \((\bA,U)\) is said to be \textbf{amnestic} provided that its fibres are partially
ordered classes; i.e., no two different \(\bA\)-objects are equivalent
\item \((\bA,U)\) is said to be \textbf{fibre-small} provided that each of its fibres is
small, i.e., a preordered set
\end{enumerate}
\end{definition}

\begin{remark}
A concrete category \((\bA,U)\) is amnestic iff the functor \(U\) is
amnestic. Most of the familiar concrete categories are both amnestic and
fibre-small. 
\end{remark}

\begin{definition}[]
A concrete category is called
\begin{enumerate}
\item \textbf{fibre-complete} provided that its fibres are (possibly large) complete lattices
\item \textbf{fibre-discrete} provided that its fibres are ordered by equality
\end{enumerate}
\end{definition}

\begin{proposition}[]
A concrete category \((\bA,U)\) over \(\bX\) is fibre-discrete iff \(U\)
\textbf{reflects identities} (i.e. if \(U(k)\) is an \(\bX\)-identity, then \(k\) must
be an \(\bA\)-identity)
\end{proposition}

\begin{definition}[]
If \((\bA,U)\) and \((\bB,V)\) are concrete categories over \(\bX\), then a
\textbf{concrete functor from} \((\bA,U)\) \textbf{to} \((\bB,V)\) is a functor \(F:\bA\to\bB\)
with \(U=V\circ F\). We denote such a functor by \(F:(\bA,U)\to(\bB,V)\)
\end{definition}

\begin{proposition}[]
\begin{enumerate}
\item Every concrete functor is faithful
\item Every concrete functor is completely determined by its values on objects
\item Objects that are identified by a full concrete functor are equivalent
\item Every full concrete functor with amnestic domain is an embedding
\end{enumerate}
\end{proposition}

\begin{proof}
\begin{enumerate}
\item \(U\) and \(V\) are faithful
\item Suppose that \(G:(\bA,U)\to(\bB,V)\) is a concrete functor with
\(G(A)=F(A)\) for each \(\bA\)-object \(A\). Then for any \(\bA\)-morphism
\(A\xrightarrow{f}A'\) we have the \(\bB\)-morphism
\begin{center}\begin{tikzcd}
GA=FA\arrow[r,shift left,"Ff"]\arrow[r,shift right,"Gf"']
&FA'=GA'
\end{tikzcd}\end{center}
with \(V(Ff)=U(f)=V(Gf)\). Since \(V\) is faithful, \(Ff=Gf\). Hence \(F=G\)
\item Let \(A\)  and \(A'\) be \(\bA\)-objects with \(FA=FA'\). Then
\(id_B:FA\to FA'\) can be lifted to an \(\bA\)-isomorphism \(g:A\to A'\).
Hence \(A\) and \(A'\) are equivalent
\end{enumerate}
\end{proof}

\begin{remark}
If \(F:(\bA,U)\to(\bB,V)\) is a concrete isomorphism, then its inverse
\(F^{-1}:\bB\to\bA\) is concrete from \((\bB,V)\) to \((\bA,U)\).
Unfortunately, the corresponding result does not hold for concrete
equivalences. If \(F:(\bA,U)\to(\bB,V)\) is a concrete equivalence from
\((\bB,V)\) to \((\bA,U)\) even though there are equivalences from \(\bB\) to
\(\bA\). For example, the embedding of the skeleton of cardinal numbers into
\(\Set\) is such a concrete categories over \(\bX\) that is not invertible
\end{remark}

\begin{proposition}[]
\begin{enumerate}
\item The identity functor on a concrete category is a concrete isomorphism
\item Any composite of concrete functors over \(\bX\) is a concrete functor over \(\bX\)
\end{enumerate}
\end{proposition}

\begin{definition}[]
The quansicategory that has as objects all concrete categories over \(\bX\)
and as morphisms all concrete functors between them is denoted by
\(\CAT(\bX)\). In particular, \(\CONST=\CAT(\Set)\) is the quasicategory of
all constructs.
\end{definition}

\begin{definition}[]
If \(F\) and \(G\) are both concrete functors from \((\bA,U)\) to
\((\bB,V)\), then \(F\) is \textbf{finer than} \(G\) (or \(G\) is \textbf{coaser than \(F\)}),
denoted by \(F\le G\), provided that \(F(A)\le G(A)\) for each \(\bA\)-object \(A\)
\end{definition}

\begin{examplle}[]
\begin{enumerate}
\item For order-preserving functions considered as concrete functors over
\(\bone\), \(f\le g\) iff this relation holds pointwise
\end{enumerate}
\end{examplle}

\begin{remark}
For every concrete category \((\bA,U)\) over \(\bX\), its dual
\((\bA^{\op},U^{\op})\) is a concrete category over \(\bX^{\op}\). Moreover,
for every concrete functor \(F:(\bA,U)\to(\bB,V)\) over \(\bX\) its dual
functor \(F^{\op}:(\bA^{\op},U^{\op})\to(\bB^{\op},V^{\op})\) is a concrete
functor over \(\bX^{\op}\). However, unless \(\bX=\bX^\op\) there is \textbf{no}
duality for concrete categories over a fixed base category \(\bX\). In
particular, we don't have a duality principle for constructs. However, since
\(\bone=\bone^\op\), there is a duality principle for concrete categories
over \(\bone\) (i.e., for preordered classes)
\end{remark}

If \((\bB,U)\) is a concrete category over \(\bX\) and \(\bA\) is a
subcategory of \(\bB\) with inclusion \(E:\bA\hookrightarrow\bB\), then
\(\bA\) will often be regarded (via the functor \(U\circ E\)) as a concrete
category \((\bA,U\circ E)\) over \(\bX\). In such cases we will call
\((\bA,U\circ E)\) a \textbf{concrete subcategory} of \((\bB,U)\). In the case that
the base category is \(\Set\), we will call \((\bA,U\circ E)\) a \textbf{subconstruct}
of \((\bB,U)\)

\begin{definition}[]
A concrete subcategory \((\bA,U)\) of \((\bB,V)\) is called \textbf{concretely
reflective} in \((\bB,V)\) provided that for each \((\bB)\)-object there
exists an identity-carried \(\bA\)-reflection arrow

Relectors induced by identity-carried reflection arrows are called \textbf{concrete reflectors}
\end{definition}

\begin{examplle}[]
\begin{enumerate}
\item Let \(\bX\) be a category consisting of a single object \(X\) and two
morphisms \(id_X\) and \(s\) with \(s\circ s=id_X\). Let \(\bA\) be the
concrete category over \(\bX\), consisting of two objects \(A_0\) and
\(A_1\) and the morphism sets
\begin{equation*}
\hom_{\bA}(A_i,A_j)=
\begin{cases}
\{id_X\}&i=j\\
\{s\}&i\neq j
\end{cases}
\end{equation*}
Consider \(\bA\) as a concretely reflective subcategory of itself. Then
\(id_{\bA}:\bA\to\bA\) is a concrete reflector, and the concrete functor
\(R:\bA\to\bA\), defined by \(R(A_i)=A_{1-i}\) is a reflector that is not
a concrete reflector
\end{enumerate}
\end{examplle}

\begin{proposition}[]
Every concretely reflective subcategory of an amnestic concrete category is a
full subcategory
\end{proposition}

\begin{proof}
Let \((\bA,U)\) be a concretely reflective subcategory of an amnestic
\((\bB,V)\), let \(A\) be an \(r:A\to A^*\) be an identity-carried
\(\bA\)-reflection arrow for \(A\). We wish to show that \(r=id_A\) so that
Proposition \ref{prop4.20} can be applied. By reflectivity there exists a
unique \(\bA\)-morphism \(s:A^*\to A\) s.t. the diagram
\begin{center}\begin{tikzcd}
A\arrow[r,"r"]\arrow[rd,"id_A"]&A^*\arrow[d,"s"]\\
&A
\end{tikzcd}\end{center}
commutes.

Since \(r\) is identity-carried, \(V(r)=id_{VA}\). Since also
\(V(id_A)=id_{VA}\), we conclude that \(V(s)=id_{VA}\). Faithfulness of \(V\)
gives us \(r\circ s=id_{A^*}\). Hence \(r\) is a \(\bB\)-isomorphism with
\(V(r)=id_{VA}\). Amnesticity of \((\bB,V)\) yields \(r=id_A\). 
\end{proof}

\begin{proposition}[]
For a concrete full subcategory \((\bA,U)\) of a concrete category
\((\bB,V)\) over \(\bX\), with inclusion functor
\(E:(\bA,U)\hookrightarrow(\bB,V)\), the following are equivalent
\begin{enumerate}
\item \((\bA,U)\) is concretely reflective in \((\bB,V)\)
\item there exists a concrete functor \(R:(\bB,V)\to(\bA,U)\) that is a
reflector with \(R\circ E=id_{\bA}\) and \(id_{\bB}\le E\circ R\)
\item there exists a concrete functor \(R:(\bB,V)\to(\bA,U)\) with \(R\circ E\le
      id_{\bA}\) and \(id_{\bB}\le E\circ R\)
\end{enumerate}
\end{proposition}

\begin{proof}
\(1\to2\).
\end{proof}






\section{Index}
\label{sec:org02d09cc}
\renewcommand{\indexname}{}
\printindex
\end{document}
