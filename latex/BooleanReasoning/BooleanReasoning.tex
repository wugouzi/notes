% Created 2021-05-26 三 11:05
% Intended LaTeX compiler: pdflatex
\documentclass[11pt]{article}
\usepackage[utf8]{inputenc}
\usepackage[T1]{fontenc}
\usepackage{graphicx}
\usepackage{grffile}
\usepackage{longtable}
\usepackage{wrapfig}
\usepackage{rotating}
\usepackage{amsmath}
\usepackage{textcomp}
\usepackage{amssymb}
\usepackage{capt-of}
\usepackage{hyperref}
% TIPS
% \substack{a\\b} for multiple lines text





% pdfplots will load xolor automatically without option
\usepackage[dvipsnames]{xcolor}

\usepackage{forest}
% two-line text in node by [two \\ lines]
% \begin{forest} qtree, [..] \end{forest}
\forestset{
  qtree/.style={
    baseline,
    for tree={
      parent anchor=south,
      child anchor=north,
      align=center,
      inner sep=1pt,
    }}}
%\usepackage{flexisym}
% load order of mathtools and mathabx, otherwise conflict overbrace

\usepackage{mathtools}
%\usepackage{fourier}
\usepackage{pgfplots}
\usepackage{amsthm, mathabx,  amsmath, commath}
\usepackage{amsfonts}

\usepackage{empheq}
\usepackage{tikz}
\usetikzlibrary{arrows.meta}
\usepackage[most]{tcolorbox}

\newtheorem{theorem}{Theorem}[section]
\newtheorem{definition}{Definition}[section]
\newtheorem{corollary}{Corollary}[section]
\newtheorem{example}{Example}[section]
\newtheorem{lemma}{Lemma}[section]
\newtheorem{proposition}{Proposition}[section]

\newcommand{\bl}[1] {\boldsymbol{#1}}
\newcommand{\Wt}[1] {\stackrel{\sim}{\smash{#1}\rule{0pt}{1.1ex}}}
\newcommand{\wt}[1] {\widetilde{#1}}


%For boxed texts in align, use Aboxed{}
%otherwise use boxed{}

\DeclareMathSymbol{\widehatsym}{\mathord}{largesymbols}{"62}
\newcommand\lowerwidehatsym{%
  \text{\smash{\raisebox{-1.3ex}{%
    $\widehatsym$}}}}
\newcommand\fixwidehat[1]{%
  \mathchoice
    {\accentset{\displaystyle\lowerwidehatsym}{#1}}
    {\accentset{\textstyle\lowerwidehatsym}{#1}}
    {\accentset{\scriptstyle\lowerwidehatsym}{#1}}
    {\accentset{\scriptscriptstyle\lowerwidehatsym}{#1}}
}

\usepackage{graphicx}
    
% text on arrow for xRightarrow
\makeatletter
%\newcommand{\xRightarrow}[2][]{\ext@arrow 0359\Rightarrowfill@{#1}{#2}}
\makeatother


\def \bx {\boldsymbol{x}}
\def \ba {\boldsymbol{a}}
\def \bI {\boldsymbol{I}}
\def \bt {\boldsymbol{t}}
\def \bb {\boldsymbol{b}}
\def \bA {\boldsymbol{A}}
\def \bX {\boldsymbol{X}}
\def \bu {\boldsymbol{u}}
\def \bS {\boldsymbol{S}}
\def \bZ {\boldsymbol{Z}}
\def \bz {\boldsymbol{z}}
\def \by {\boldsymbol{y}}
\def \bw {\boldsymbol{w}}
\def \bT {\boldsymbol{T}}
\def \bS {\boldsymbol{S}}
\def \bm {\boldsymbol{m}}
\def \bW {\boldsymbol{W}}
\def \bY {\boldsymbol{Y}}
\def \bH {\boldsymbol{H}}
\def \blambda {\boldsymbol{\lambda}}
\def \bPhi {\boldsymbol{\Phi}}
\def \btheta {\boldsymbol{\theta}}
\def \bmu {\boldsymbol{\mu}}
\def \bphi {\boldsymbol{\phi}}
\def \bSigma {\boldsymbol{\Sigma}}
\def \lb {\left\{}
\def \rb {\right\}}
\def \caln {\mathcal{N}}
\def \dissum {\displaystyle\Sigma}
\def \dispro {\displaystyle\prod}
\def \E {\mathbb{E}}
\def \Q {\mathbb{Q}}
\def \V {\mathbb{V}}
\def \R {\mathbb{R}}
\def \calq {\mathcal{Q}}
\def \calg {\mathcal{G}}
\def \caln {\mathcal{N}}
\def \calr {\mathcal{R}}
\def \calm {\mathcal{M}}
\def \calc {\mathcal{C}}
\def \bcup {\bigcup}

\def \ceq {\overset{\circ}{=}}
\author{Frank Markham Brown}
\date{\today}
\title{Boolean Reasoning: The Logic of Boolean Equations}
\hypersetup{
 pdfauthor={Frank Markham Brown},
 pdftitle={Boolean Reasoning: The Logic of Boolean Equations},
 pdfkeywords={},
 pdfsubject={},
 pdfcreator={Emacs 27.1 (Org mode 9.3)}, 
 pdflang={English}}
\begin{document}

\maketitle
\tableofcontents

\section{The Blake Canonical Form}
\label{sec:org0da07bf}
\subsection{Definitions and Terminology}
\label{sec:org5a2f8c3}
Two SOP formulas will be called \textbf{equivalent} (\(\equiv\)) in case they
represent the same Boolean function. We call two SOP formulas \textbf{congruent} (\(\ceq\))in
case one can be transformed into the other using only the commutative rule.
Thus congruent SOP formulas may differ only in the order of enumerations of
their terms and in the order of the literals in any term.

An \textbf{implicant} of a Boolean function \(f\) is a term \(p\) s.t. \(p\le f\).

An SOP formula \(F\) will be called \textbf{absorptive} in case no term in \(F\) is
absorbed by any other term in \(F\). If \(F\) is not absorptive, then an
equivalent absorptive formula, which we call \(ABS(F)\), may be obtained
from \(F\) by successive deletion of terms absorbed by other terms in \(F\).
\subsection{Syllogistic \& Blake Canonical Formulas}
\label{sec:org84bdfbf}
Let \(F\) and \(G\) be SOP formulas. We say that \(G\) is \textbf{formally included}
in \(F\) written \(G\ll F\) ,in case each term of \(G\) is inclused in some
term of \(F\). Formal inclusion implies inclusion,
i.e., \(G\ll F\Rightarrow G\le F\) for any \(F,G\) pair.

\begin{examplle}[]
Let
\begin{align*}
&F_1=wy'+w'z+w'x'y+wx'yz'\\
&G=w'y'z+w'x'y\\
&H=xy'z+x'yz'
\end{align*}
\(G\le F_1\) and \(H\le F_1\). Also \(G\ll F_1\) and \(H\not\ll F_1\)
\end{examplle}

   A formula \(F\) will be called \textbf{syllogistic} in case for every SOP
formula \(G\)
\begin{equation*}
G\le F\Rightarrow G\ll F
\end{equation*}
Thus \(F\) is syllogistic iff every implicant of \(F\) is included in some
term of \(F\)
\begin{examplle}[]
The formula
\begin{equation*}
F_2=wy'+w'z+w'x'y+x'yz'+y'z+wx'z'
\end{equation*}
is a syllogistic formula equivalent to \(F_1\). Every SOP in \(F_2\) is
formally included in \(F_2\).
\end{examplle}

Given SOP formulas \(F\) and \(G\), we define \(F\times G\) to be the SOP
formula produced by multiplying out the conjunction \(FG\), suing the
distributive laws. If \(F=\sum_is_i\) and \(G=\sum_jt_j\), then
\begin{equation*}
F\times G=\sum_i\sum_js_i\cdot t_j
\end{equation*}
where repeated literals are dropped in each product \(s_i\cdot t_j\) of
terms. A product is dropped if it contains a complementary pair of literals.

Let \(a\) be any letter. Two terms will be said to have an \textbf{opposition} in case
one term contains the literal \(a\) and the other the literal \(a'\). For
example \(x'yz\) and \(wy'z\) have a single opposition. Suppose two
terms \(r\) and \(s\)have exactly one opposition. Then the \textbf{consensus} of \(r\)
and \(s\), denoted by \(c(r,s)\) , is the term obtained from the
conjunction \(rs\) by deleting the two opposed literals as well as any
repeated literals.

Let \(F\) be a syllogistic formula for a Boolean function \(f\). We call the
formula \(ABS(f)\) the \textbf{Blake canonical form} for \(f\), and we denote it by \(BCF(f)\).
Blake showed that \(BCF(f)\) is minimal within the class of syllogistic
formulas for \(f\).
\subsection{Generation of \(BCF(f)\)}
\label{sec:org1bf05a5}
\begin{enumerate}
\item Find a syllogistic formula for \(f\)
\item Delete absorbed terms
\end{enumerate}
\subsection{Exhaustion of implicants}
\label{sec:orgd37b40b}
\subsection{Iterated Consensus}
\label{sec:org5befbfe}
Theorem \ref{thma-2-3} guarantees that any SOP formula is transformed into a
syllogistic formula by repeated application of the following rule
\begin{center}
If the formula contains a pair \(r,s\) of terms whose consensus \(c(r,s)\)
exists and is not included in any term of the formula, then adjoin \(c(r,s)\)
to the formula.
\end{center}

The process of iterated consensus terminates in a finite number of steps,
becuase
\begin{itemize}
\item If a pair \(r,s\) of terms meets the specified condition at a given stop,
it cannot meet that condition at any subsequent stop
\item the number of pairs of terms to be considered is finite, inasmuch as no
more than \(3^n\) terms can be produced from \(n\) letters
\end{itemize}


We say that a consensus is \textbf{applicable} to the formula from which it is derived
if the consensus is not inclused in any term of that formula


\appendix

\section{Absorptive Formulas}
\label{sec:org5cc6c6e}
\begin{lemma}[]
The formula \(ABS(F)\) is unique to within congruence
\end{lemma}

\section{Syllogistic Formulas}
\label{sec:orge0608b4}
   A formula \(F\) will be called \textbf{syllogistic} in case for every SOP
formula \(G\)
\begin{equation*}
G\le F\Rightarrow G\ll F
\end{equation*}
Thus \(F\) is syllogistic iff every implicant of \(F\) is included in some
term of \(F\).

\begin{lemma}[]
Let \(F,G\) and \(H\) be SOP formulas. If \(F\ll G+H\) and \(G\ll H\),
then \(F\ll H\).
\end{lemma}

\begin{lemma}[]
Let \(F\) be an SOP formula. \(F\) is syllogistic iff \(ABS(F)\) is syllogistic
\end{lemma}

\begin{lemma}[]
Let \(F_1\) and \(F_2\) be syllogistic. If \(F_1\equiv F_2\)
then \(ABS(F_1)\ceq ABS(F_2)\)
\end{lemma}

\begin{theorem}[]
\label{thma-2-1}
Let \(F_1,\dots,F_k\) be syllogistic formulas.
Then \(F_1\times\dots\times F_k\) is syllogistic
\end{theorem}

\begin{lemma}[]
If terms \(r\) and \(s\) have no oppositions, then \(r+s\) is syllogistic
\end{lemma}

\begin{theorem}[]
\label{thma-2-2}
Let \(r\) and \(s\)be terms. The formula \(r+s\) is non-syllogistic iff \(r\)
and \(s\) have exactly one opposition
\end{theorem}

\begin{lemma}[]
Let \(r+s\) be a non-syllogistic SOP formula. Then
\begin{enumerate}
\item \(r+s+c(r,s)\equiv r+s\)
\item \(r+s+c(r,s)\) is syllogistic
\end{enumerate}
\end{lemma}

\begin{theorem}[]
\label{thma-2-3}
If an SOP formula \(F\) is not syllogistic, it contains terms \(p\)
and \(q\), having exactly one opposition, s.t. \(c(p,q)\) is not formally
included in \(F\)
\end{theorem}

\begin{corollary}[]
If an SOP formula \(F\) is not syllogistic, then \(ABS(F)\) contains
terms \(p\)and \(q\), having  exactly one opposition, s.t. \(c(p,q)\not\ll ABS(F)\)
\end{corollary}

\section{Prime Implicants}
\label{sec:org7b95da6}
\begin{lemma}[]
An implicant \(p\) of a Boolean function \(f\) is a prime implicant of \(f\)
in case the implication
\begin{equation*}
p\le q\le f\Rightarrow p=q
\end{equation*}
holds for every term \(q\)
\end{lemma}

\begin{theorem}[]
Let \(F\) be an SOP formula for a Boolean function \(f\). Then \(F\)is
syllogistic iff every prime implicant of \(f\) is a term of \(F\)
\end{theorem}

\section{The Blake Canonical Form}
\label{sec:org091028d}
Let \(F\) be a syllogistic formula for a Boolean
function \(f\). \(ABS(F)=BCF(F)\)

\begin{theorem}[]
Let \(f\) be a Boolean function. Then \(BCF(f)\) is the disjunction of all of
the prime implicants of \(f\)
\end{theorem}
\end{document}