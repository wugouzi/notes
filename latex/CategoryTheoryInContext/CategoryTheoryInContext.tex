% Created 2021-10-25 Mon 11:17
% Intended LaTeX compiler: pdflatex
\documentclass[11pt]{article}
\usepackage[utf8]{inputenc}
\usepackage[T1]{fontenc}
\usepackage{graphicx}
\usepackage{longtable}
\usepackage{wrapfig}
\usepackage{rotating}
\usepackage[normalem]{ulem}
\usepackage{amsmath}
\usepackage{amssymb}
\usepackage{capt-of}
\usepackage{hyperref}
\graphicspath{{../../books/}}
% TIPS
% \substack{a\\b} for multiple lines text





% pdfplots will load xolor automatically without option
\usepackage[dvipsnames]{xcolor}

\usepackage{forest}
% two-line text in node by [two \\ lines]
% \begin{forest} qtree, [..] \end{forest}
\forestset{
  qtree/.style={
    baseline,
    for tree={
      parent anchor=south,
      child anchor=north,
      align=center,
      inner sep=1pt,
    }}}
%\usepackage{flexisym}
% load order of mathtools and mathabx, otherwise conflict overbrace

\usepackage{mathtools}
%\usepackage{fourier}
\usepackage{pgfplots}
\usepackage{amsthm, mathabx,  amsmath, commath}
\usepackage{amsfonts}

\usepackage{empheq}
\usepackage{tikz}
\usetikzlibrary{arrows.meta}
\usepackage[most]{tcolorbox}

\newtheorem{theorem}{Theorem}[section]
\newtheorem{definition}{Definition}[section]
\newtheorem{corollary}{Corollary}[section]
\newtheorem{example}{Example}[section]
\newtheorem{lemma}{Lemma}[section]
\newtheorem{proposition}{Proposition}[section]

\newcommand{\bl}[1] {\boldsymbol{#1}}
\newcommand{\Wt}[1] {\stackrel{\sim}{\smash{#1}\rule{0pt}{1.1ex}}}
\newcommand{\wt}[1] {\widetilde{#1}}


%For boxed texts in align, use Aboxed{}
%otherwise use boxed{}

\DeclareMathSymbol{\widehatsym}{\mathord}{largesymbols}{"62}
\newcommand\lowerwidehatsym{%
  \text{\smash{\raisebox{-1.3ex}{%
    $\widehatsym$}}}}
\newcommand\fixwidehat[1]{%
  \mathchoice
    {\accentset{\displaystyle\lowerwidehatsym}{#1}}
    {\accentset{\textstyle\lowerwidehatsym}{#1}}
    {\accentset{\scriptstyle\lowerwidehatsym}{#1}}
    {\accentset{\scriptscriptstyle\lowerwidehatsym}{#1}}
}

\usepackage{graphicx}
    
% text on arrow for xRightarrow
\makeatletter
%\newcommand{\xRightarrow}[2][]{\ext@arrow 0359\Rightarrowfill@{#1}{#2}}
\makeatother


\def \bx {\boldsymbol{x}}
\def \ba {\boldsymbol{a}}
\def \bI {\boldsymbol{I}}
\def \bt {\boldsymbol{t}}
\def \bb {\boldsymbol{b}}
\def \bA {\boldsymbol{A}}
\def \bX {\boldsymbol{X}}
\def \bu {\boldsymbol{u}}
\def \bS {\boldsymbol{S}}
\def \bZ {\boldsymbol{Z}}
\def \bz {\boldsymbol{z}}
\def \by {\boldsymbol{y}}
\def \bw {\boldsymbol{w}}
\def \bT {\boldsymbol{T}}
\def \bS {\boldsymbol{S}}
\def \bm {\boldsymbol{m}}
\def \bW {\boldsymbol{W}}
\def \bY {\boldsymbol{Y}}
\def \bH {\boldsymbol{H}}
\def \blambda {\boldsymbol{\lambda}}
\def \bPhi {\boldsymbol{\Phi}}
\def \btheta {\boldsymbol{\theta}}
\def \bmu {\boldsymbol{\mu}}
\def \bphi {\boldsymbol{\phi}}
\def \bSigma {\boldsymbol{\Sigma}}
\def \lb {\left\{}
\def \rb {\right\}}
\def \caln {\mathcal{N}}
\def \dissum {\displaystyle\Sigma}
\def \dispro {\displaystyle\prod}
\def \E {\mathbb{E}}
\def \Q {\mathbb{Q}}
\def \V {\mathbb{V}}
\def \R {\mathbb{R}}
\def \calq {\mathcal{Q}}
\def \calg {\mathcal{G}}
\def \caln {\mathcal{N}}
\def \calr {\mathcal{R}}
\def \calm {\mathcal{M}}
\def \calc {\mathcal{C}}
\def \bcup {\bigcup}

\def \MODEL {\textbf{MODEL}}
\def \Cop {\calc^{\op}}
\def \fop {f^{\op}}
\def \gop {g^{\op}}
\makeindex
\author{Emily Riehl}
\date{\today}
\title{Category Theory In Context}
\hypersetup{
 pdfauthor={Emily Riehl},
 pdftitle={Category Theory In Context},
 pdfkeywords={},
 pdfsubject={},
 pdfcreator={Emacs 27.2 (Org mode 9.5)}, 
 pdflang={English}}
\begin{document}

\maketitle
\tableofcontents


\section{Categories, Functors, Natural Transformations}
\label{sec:org545e280}
\subsection{Abstract and concrete categories}
\label{sec:org15a7cdd}
\begin{definition}[]
A \textbf{category} consists of
\begin{itemize}
\item a collection of \textbf{objects} \(X,Y,Z,\dots\)
\item a collection of \textbf{morphisms} \(f,g,h,\dots\)
\end{itemize}


so that
\begin{itemize}
\item Each morphism has specified \textbf{domain} and \textbf{codomain} objects; the notation \(f:X\to Y\)
signifies that \(f\) is a morphism with domain \(X\) and codomain \(Y\)
\item Each object has a designated \textbf{identity morphism} \(1_X:X\to X\)
\item For any pair of morphisms \(f,g\) with the codomain of \(f\) equal to the domain of \(g\),
there exists a specified \textbf{composite morphism} \(gf\) whose domain is equal to the domain
of \(f\) and whose codomain is equal to the codomain of \(g\), i.e., :
\begin{equation*}
f:X\to Y,\quad g:Y\to Z\hspace{0.7cm}\leadsto\hspace{0.7cm}gf:X\to Z
\end{equation*}
\end{itemize}


This data is subject to the following two axioms
\begin{itemize}
\item For any \(f:X\to Y\), the composites \(1_Yf\) and \(f1_X\) are both equal to \(f\)
\item For any composable triple of morphisms \(f,g,h\), the composites \(h(gf)\) and \((hg)f\) are
equal and hence denoted by \(hgf\).
\begin{equation*}
f:X\to Y,\quad g:Y\to Z,\quad h:Z\to W\hspace{0.7cm}\leadsto\hspace{0.7cm}hgf:X\to W
\end{equation*}
\end{itemize}
\end{definition}

\begin{examplle}[]
\begin{enumerate}
\item For any language \(\call\) and any theory \(T\) of \(\call\), there is a category \(\MODEL_T\) whose
objects are models of \(T\). Morphisms is just homomorphisms
\end{enumerate}
\end{examplle}

\textbf{Concrete categories} are those whose objects have underlying sets and whose morphisms are
functions between underlying sets

\begin{definition}[]
A category is \textbf{small} if it has only a set's worth of arrows

Both \(\ob(\calc)\) and \(\hom(\calc)\) are sets
\end{definition}

Thus it has only a set's worth of objects

\begin{definition}[]
A category is \textbf{locally small} if between any pair of objects there is only a set's worth of morphisms
\end{definition}

The set of arrows between a pair of fixed objects in a locally small category is typically
called a \textbf{hom-set}


\begin{definition}[]
An \textbf{isomorphism} in a category is a morphism \(f:X\to Y\) for which there exists a
morphism \(g:Y\to X\) so that \(gf=1_X\) and \(fg=1_X\), denoted by \(X\cong Y\)
\end{definition}

An \textbf{endomorphism} is a morphism whose domain equals its codomain

\begin{definition}[]
A \textbf{groupoid} is a category in which every morphism is an isomorphism
\end{definition}

\begin{lemma}[]
Any category \(\calc\) contains a \textbf{maximal groupoid}, the subcategory containing all of the objects
and only those morphisms that are isomorphisms
\end{lemma}

\begin{exercise}
\label{1.1.1}
\begin{enumerate}
\item Consider a morphism \(f:x\to y\). Show that if there exists a pair of
morphisms \(g,h:y\rightrightarrows:x\) s.t. \(gf=1_x\) and \(fh=1_y\), then \(g=h\) and \(f\)
is an isomorphism
\item Show that a morphism can have at most one inverse isomorphism
\end{enumerate}
\end{exercise}

\begin{proof}
\begin{enumerate}
\item \(g=1_xg=(hf)g=h(fg)=h1_y=h\)
\item From 1
\end{enumerate}
\end{proof}

\begin{exercise}
\label{ex1.1.3}
For any category \(\calc\) and any object \(c\in\calc\), show that
\begin{enumerate}
\item There is a category \(c/\calc\) whose objects are morphisms \(f:c\to x\) with domain \(c\) in which
a morphism from \(f:c\to x\) to \(g:c\to y\) is a map \(h:x\to y\) between the codomains so that
the triangle
\begin{center}\begin{tikzcd}[column sep=small]
&c\ar[dl,"f"']\ar[dr,"g"]\\
x\ar[rr,"h"']&&y
\end{tikzcd}\end{center}
commutes.
\item There is a category \(\calc/c\) whose objects are morphisms \(f:x\to c\) with codomain \(c\) in which
a morphism from \(f:x\to c\) to \(g:y\to c\) is a map \(h:x\to y\) between the codomains so that
the triangle
\begin{center}\begin{tikzcd}[column sep=small]
x\ar[rr,"h"]\ar[dr,"f"']&&y\ar[dl,"g"]\\
&c
\end{tikzcd}\end{center}
commutes
\end{enumerate}


The category \(c/\calc\) and \(\calc/c\) are called \textbf{slice categories} of \(\calc\) \textbf{under} and \textbf{over} \(c\), respectively
\end{exercise}
\subsection{Duality}
\label{sec:org124d7b1}
\begin{definition}[]
Let \(\calc\) be any category. The \textbf{opposite category} \(\calc^{\op}\) has
\begin{itemize}
\item the same objects as in \(\calc\)
\item a morphism \(\fop\) in \(\Cop\) for each a morphism \(f\) in \(\calc\) so that the domain
of \(\fop\) is defined to be the codomain of \(f\) and the codomain of \(\fop\) is defined to
be the domain of \(f\)
\item For each object \(X\), the arrow \(1_X^{\op}\) serves as its identity in \(\Cop\)
\item A pair of morphisms \(\fop,\gop\) in \(\Cop\) is composable precisely when the pair \(g,f\) is
composable in \(\calc\). We then define \(\gop\circ\fop\) to be \((f\circ g)^{\op}\): i.e.
\begin{equation*}
\dom(\fop)=\cod(f)=\dom(g)=\cod(\gop)
\end{equation*}
\end{itemize}
\end{definition}

\begin{lemma}[]
\label{lemma1.2.3}
T.F.A.E.
\begin{enumerate}
\item \(f:x\to y\) is an isomorphism
\item For all objects \(c\in\calc\), post-composition with \(f\) defines a bijection
\begin{equation*}
f_*:\Hom(c,x)\to\Hom(c,y)
\end{equation*}
\item For all objects \(c\in\calc\), pre-composition with \(f\) defines a bijection
\begin{equation*}
f^*:\Hom(y,c)\to\Hom(x,c)
\end{equation*}
\end{enumerate}
\end{lemma}

Lemma \ref{lemma1.2.3} asserts that isomorphisms in a locally small category are defined
\emph{representably} in terms of isomorphisms in the category of sets.

\begin{proof}
\(2\to 1\). Let \(c=y\), since \(f_*\) in an bijection, there must be an element \(g\in\Hom(y,x)\)
s.t. \(f_*(g)=1_y\). Hence \(fg=1_y\). Thus \(gf,1_x\) have common image under \(f_*\),
thus \(gf=1_x\). Whence \(f\) and \(g\) are inverse isomorphisms
\end{proof}

\begin{definition}[]
A morphism \(f:x\to y\) in a category is
\begin{enumerate}
\item a \textbf{monomorphism} if for any parallel morphisms \(h,k:w\rightrightarrows x\), \(fg=fk\) implies
that \(h=k\)
\item an \textbf{epimorphism} if for any parallel morphisms \(h,k:w\rightrightarrows x\), \(hf=kf\) implies
that \(h=k\)
\end{enumerate}
\end{definition}

Also, we can re-express it
\begin{enumerate}
\item \(f:x\to y\) is a monomorphism in \(\calc\) iff for all
objects \(c\in\calc\), \(f_*:\Hom(c,x)\to\Hom(c,y)\) is injective
\item \(f:x\to y\) is an epimorphism in \(\calc\) iff for all \(c\in\calc\), \(f^*:\Hom(y,c)\to\Hom(x,c)\)  is
injective
\end{enumerate}


\begin{examplle}[]
Suppose that \(x\xrightarrow{s}y\xrightarrow{r}x\) are morphisms s.t. \(rs=1_x\). The map \(s\) is a \textbf{section} or \textbf{right
inverse} to \(r\), while the map \(r\) defines a \textbf{retraction} or \textbf{left inverse} to \(s\). The
maps \(s\) and \(r\) express the object \(x\) as a \textbf{retract} of the object \(y\)

In this case, \(s\) is always a monomorphism and, dually, \(r\) is always an epimorphism. To
ackowledge the presence of these one-sided inverses, \(s\) is said to be a \textbf{split monomorphism}
and \(r\) is said to be a \textbf{split epimorphism}
\end{examplle}

\begin{examplle}[]
By the previous example, an isomorphism is necessarily both monic and epic, but the converse
need not hold in general. For example, the inclusion \(\Z\hookrightarrow\Q\) is both monic and epic in the
category \(\Rng\), but this map is not an isomorphism: there are no ring homomorphisms
from \(\Q\) to \(\Z\)
\end{examplle}

\begin{lemma}[]
\begin{enumerate}
\item If \(f:x\rightarrowtail y\) and \(g:y\rightarrowtail z\) are monomorphisms, then so
is \(gf:x\rightarrowtail z\)
\item If \(f:x\to y\) and \(g:y\to z\) are morphisms so that \(gf\) is monic, then \(f\) is monic
\end{enumerate}


Dually
\begin{enumerate}
\item If \(f:x\twoheadrightarrow y\) and \(g:y\twoheadrightarrow z\) are epimorphisms, then so
is \(gf:x\twoheadrightarrow z\)
\item If \(f:x\to y\) and \(g:y\to z\) are morphisms so that \(gf\) is epic, then \(g\) is epic
\end{enumerate}
\end{lemma}

\begin{exercise}
\label{ex1.2.2}
\begin{enumerate}
\item Show that a morphism \(f:x\to y\) is a split epimorphism in a category \(\calc\) iff for
all \(c\in\calc\), the post-composition function \(f_*:\Hom(c,x)\to\Hom(c,y)\) is surjective
\item Show that a morphism \(f:x\to y\) is a split monomorphism in a category \(\calc\) iff for
all \(c\in\calc\), the post-composition function \(f^*:\Hom(y,c)\to\Hom(x,c)\) is surjective
\end{enumerate}
\end{exercise}

\begin{exercise}
\label{ex1.2.6}
Prove that a morphism that is both a monomorphism and a split epimorphism is necessarily an
isomorphism. Argue by duality that a split monomorphism that is an epimorphism is also an isomorphism
\end{exercise}

\begin{proof}
Suppose \(y\xrightarrow{g}x\xrightarrow{f}y\) and \(fg=1_y\), then \(fgf=f=f\circ 1_x\). Since \(f\) is mono, \(gf=1_x\)
\end{proof}
\subsection{Functoriality}
\label{sec:orgcf39a04}
\begin{definition}[]
A \textbf{functor} \(F:\calc\to\cald\), between categories \(\calc\) and \(\cald\), consists of the following data:
\begin{itemize}
\item An object \(Fc\in\cald\), for each objects \(c\in\calc\)
\item A morphism \(Ff:Fc\to Fc'\in\cald\), for each morphism \(f:c\to c'\in\calc\)
\end{itemize}


\textbf{Functoriality axioms}
\begin{itemize}
\item For any composable pair \(f,g\in\calc\), \(Fg\circ Ff=F(g\circ f)\)
\item For each object \(c\in\calc\), \(F(1_c)=1_{Fc}\)
\end{itemize}
\end{definition}

\begin{definition}[]
A \textbf{contravariant functor} \(F\) from \(\calc\) to \(\cald\) is a functor \(F:\Cop\to\cald\)
\begin{itemize}
\item A morphism \(Ff:Fc'\to Fc\in\cald\) for each morphism \(f:c\to c'\in\calc\)
\item For any composable pair \(f,g\in\calc\), \(Ff\circ Fg=F(g\circ f)\)
\end{itemize}
\end{definition}

\begin{center}\begin{tikzcd}
\Cop\ar[r,"F"]&\cald\\
c\ar[r,mapsto]\ar[dd,"f"']&Fc\\
{}\ar[r,mapsto]&{}\\
c'\ar[r,mapsto]&Fc'\ar[uu,"Ff"']
\end{tikzcd}\end{center}

\begin{lemma}[]
Functors preserve isomorphisms
\end{lemma}

\begin{proof}
Consider a functor \(F:\calc\to\cald\) and an isomorphism \(f:x\to y\) in \(\calc\) with inverse \(g:y\to x\).
Then
\begin{equation*}
F(g)F(f)=F(gf)=F(1_x)=1_{Fx}
\end{equation*}
Thus \(Fg:Fy\to Fx\) is a left inverse to \(Ff:Fx\to Fy\)
\end{proof}

\begin{definition}[]
If \(\calc\) is locally small, then for any object \(c\in\calc\) we may define a pair of covariant and
contravariant \textbf{functors represented by} \(c\):
\begin{center}\begin{tikzcd}[column sep=tiny]
\calc\ar[rrr,"{\Hom(c,-)}"]&\quad&\quad&\Sets\\
x\ar[dd,"f"']&\ar[r,mapsto]&{}&\Hom(c,x)\ar[dd,"f_*"]\\
{}&\ar[r,mapsto]&{}&{}\\
y&\ar[r,mapsto]&{}&\Hom(c,y)
\end{tikzcd}\hspace{1cm}
\begin{tikzcd}[column sep=tiny]
\Cop\ar[rrr,"{\Hom(-,c)}"]&\quad&\quad&\Sets\\
x\ar[dd,"f"']&\ar[r,mapsto]&{}&\Hom(x,c)\\
{}&\ar[r,mapsto]&{}&{}\\
y&\ar[r,mapsto]&{}&\Hom(y,c)\ar[uu,"f^*"]
\end{tikzcd}
\end{center}
\end{definition}

Post-composition defines a \textbf{covariant} action on hom-sets

\begin{definition}[]
For any categories \(\calc\) and \(\cald\), there is a category \(\calc\times\cald\), their \textbf{product}, whose
\begin{itemize}
\item objects are ordered pairs \((c,d)\), where \(c\) is an object of \(\calc\) and \(d\) is an object
of \(\cald\)
\item morphisms are ordered pairs \((f,g):(c,d)\to(c',d')\), where \(f:c\to c'\in\calc\) and \(g:d\to d'\in\cald\) and
\item in which composition and identities are defined componentwise
\end{itemize}
\end{definition}

\begin{definition}[]
If \(\calc\) is locally small, then there is a \textbf{two-sided represented functor}
\begin{equation*}
\calc(-,-):\Cop\times\calc\to\Sets
\end{equation*}
A pair of objects \((x,y)\) is mapped to the hom-set \(\Hom(x,y)\). A pair of
morphisms \(f:w\to x\) and \(h:y\to z\) is sent to the function
\begin{center}\begin{tikzcd}
\Hom(x,y)\ar[r,"{(f^*,h_*)}"]&\Hom(w,z)\\
g\ar[r,mapsto]&hgf
\end{tikzcd}\end{center}
\end{definition}

An \textbf{isomorphism of categories} is given by a pair of inverse functors \(F:\calc\to\cald\) and \(G:\cald\to\calc\) s.t.
the composites \(Gf\) and \(FG\), respectively, equal the identity functors on \(\calc\) and \(\cald\)
\subsection{Naturality}
\label{sec:orgb117201}
\begin{definition}[]
Given categories \(\calc\) and \(\cald\) and functors \(F,G:\calc\rightrightarrows\cald\), a \textbf{natural
transformation} \(\alpha:F\Rightarrow G\) consists of
\begin{itemize}
\item an arrow \(\alpha_c:Fc\to Gc\) in \(\cald\) for each object \(c\in\calc\), the collection of which define the
\textbf{components} of the natural transformation s.t. for any morphism \(f:c\to c'\) in \(\calc\), the
following square of morphisms in \(\cald\)
\begin{center}\begin{tikzcd}
Fc\ar[r,"\alpha_c"]\ar[d,"Ff"']&Gc\ar[d,"Gf"]\\
Fc'\ar[r,"\alpha_{c'}"']&Gc'
\end{tikzcd}\end{center}
\textbf{commutes}
\end{itemize}


A \textbf{natural isomorphism} is a natural transformation \(\alpha:F\Rightarrow G\) in which every component \(\alpha_c\) is
an isomorphism. In this case, the natural isomorphism may be depicted as \(\alpha:F\cong G\)

\begin{center}\begin{tikzcd}
A\ar[r,bend left=50,""{name=U}]\ar[r,bend right=50,""{name=D}]&B
\ar[Rightarrow,from=U,to=D,yshift=-3pt,"\alpha"]
\end{tikzcd}\end{center}
\end{definition}
\end{document}
