% Created 2020-09-06 日 21:12
% Intended LaTeX compiler: pdflatex
\documentclass[11pt]{article}
\usepackage[utf8]{inputenc}
\usepackage[T1]{fontenc}
\usepackage{graphicx}
\usepackage{grffile}
\usepackage{longtable}
\usepackage{wrapfig}
\usepackage{rotating}
\usepackage[normalem]{ulem}
\usepackage{amsmath}
\usepackage{textcomp}
\usepackage{amssymb}
\usepackage{capt-of}
\usepackage{hyperref}
\usepackage{minted}
% TIPS
% \substack{a\\b} for multiple lines text





% pdfplots will load xolor automatically without option
\usepackage[dvipsnames]{xcolor}

\usepackage{forest}
% two-line text in node by [two \\ lines]
% \begin{forest} qtree, [..] \end{forest}
\forestset{
  qtree/.style={
    baseline,
    for tree={
      parent anchor=south,
      child anchor=north,
      align=center,
      inner sep=1pt,
    }}}
%\usepackage{flexisym}
% load order of mathtools and mathabx, otherwise conflict overbrace

\usepackage{mathtools}
%\usepackage{fourier}
\usepackage{pgfplots}
\usepackage{amsthm, mathabx,  amsmath, commath}
\usepackage{amsfonts}

\usepackage{empheq}
\usepackage{tikz}
\usetikzlibrary{arrows.meta}
\usepackage[most]{tcolorbox}

\newtheorem{theorem}{Theorem}[section]
\newtheorem{definition}{Definition}[section]
\newtheorem{corollary}{Corollary}[section]
\newtheorem{example}{Example}[section]
\newtheorem{lemma}{Lemma}[section]
\newtheorem{proposition}{Proposition}[section]

\newcommand{\bl}[1] {\boldsymbol{#1}}
\newcommand{\Wt}[1] {\stackrel{\sim}{\smash{#1}\rule{0pt}{1.1ex}}}
\newcommand{\wt}[1] {\widetilde{#1}}


%For boxed texts in align, use Aboxed{}
%otherwise use boxed{}

\DeclareMathSymbol{\widehatsym}{\mathord}{largesymbols}{"62}
\newcommand\lowerwidehatsym{%
  \text{\smash{\raisebox{-1.3ex}{%
    $\widehatsym$}}}}
\newcommand\fixwidehat[1]{%
  \mathchoice
    {\accentset{\displaystyle\lowerwidehatsym}{#1}}
    {\accentset{\textstyle\lowerwidehatsym}{#1}}
    {\accentset{\scriptstyle\lowerwidehatsym}{#1}}
    {\accentset{\scriptscriptstyle\lowerwidehatsym}{#1}}
}

\usepackage{graphicx}
    
% text on arrow for xRightarrow
\makeatletter
%\newcommand{\xRightarrow}[2][]{\ext@arrow 0359\Rightarrowfill@{#1}{#2}}
\makeatother


\def \bx {\boldsymbol{x}}
\def \ba {\boldsymbol{a}}
\def \bI {\boldsymbol{I}}
\def \bt {\boldsymbol{t}}
\def \bb {\boldsymbol{b}}
\def \bA {\boldsymbol{A}}
\def \bX {\boldsymbol{X}}
\def \bu {\boldsymbol{u}}
\def \bS {\boldsymbol{S}}
\def \bZ {\boldsymbol{Z}}
\def \bz {\boldsymbol{z}}
\def \by {\boldsymbol{y}}
\def \bw {\boldsymbol{w}}
\def \bT {\boldsymbol{T}}
\def \bS {\boldsymbol{S}}
\def \bm {\boldsymbol{m}}
\def \bW {\boldsymbol{W}}
\def \bY {\boldsymbol{Y}}
\def \bH {\boldsymbol{H}}
\def \blambda {\boldsymbol{\lambda}}
\def \bPhi {\boldsymbol{\Phi}}
\def \btheta {\boldsymbol{\theta}}
\def \bmu {\boldsymbol{\mu}}
\def \bphi {\boldsymbol{\phi}}
\def \bSigma {\boldsymbol{\Sigma}}
\def \lb {\left\{}
\def \rb {\right\}}
\def \caln {\mathcal{N}}
\def \dissum {\displaystyle\Sigma}
\def \dispro {\displaystyle\prod}
\def \E {\mathbb{E}}
\def \Q {\mathbb{Q}}
\def \V {\mathbb{V}}
\def \R {\mathbb{R}}
\def \calq {\mathcal{Q}}
\def \calg {\mathcal{G}}
\def \caln {\mathcal{N}}
\def \calr {\mathcal{R}}
\def \calm {\mathcal{M}}
\def \calc {\mathcal{C}}
\def \bcup {\bigcup}

\usepackage[UTF8]{ctex}
\author{wu}
\date{\today}
\title{高等代数简明教程}
\hypersetup{
 pdfauthor={wu},
 pdftitle={高等代数简明教程},
 pdfkeywords={},
 pdfsubject={},
 pdfcreator={Emacs 26.3 (Org mode 9.4)}, 
 pdflang={English}}
\begin{document}

\maketitle
\tableofcontents \clearpage\section{代数学的经典课题}
\label{sec:org20f8d76}
\subsection{线性方程组}
\label{sec:org28f2c23}
\begin{equation*}
\begin{cases}
a_{11}x_1+\dots+a_{1n}x_n=b_1,\\
a_{21}x_1+\dots+a_{2n}x_n=b_2,\\
\dots\\
a_{m1}x_1+\dots+a_{mn}x_n=b_m
\end{cases}
\end{equation*}
设在线性方程组中,未知量的系数\(a_{ij}\)和常数项\(b_1,\dots,b_m\)都属于数域
\(K\),则称它是 \textbf{数域\(K\)上的线性方程组}

\begin{definition}[]
\textbf{初等变换}
\begin{enumerate}
\item 互换两个方程的位置
\item 把某一个方程两边同乘数域\(K\)内的一个非零常数\(c\)
\item 把第\(j\)个方程加上第\(i\)个方程的\(k\)倍,这里\(k\in K\)且\(i\neq  j\)
\end{enumerate}
\end{definition}

\begin{proposition}[]
设方程组经过某一初等变换后变为另一个方程组,则新方程组与原方程组同解
\end{proposition}

\begin{definition}[]
给定数域\(K\)上\(mn\)个数\(a_{ij}(i=1,2,\dots,m;j=1,\dots,n)\),把它们按一定
次序排成一个\(m\)行\(n\)列的长方形表格
\begin{equation*}
A=
\begin{bmatrix}
a_{11}&a_{12}&\dots&a_{1n}\\
a_{21}&a_{22}&\dots&a_{2n}\\
\vdots&\vdots&&\vdots\\
a_{m1}&a_{m2}&\dots&a_{mn}
\end{bmatrix}
\end{equation*}
称为数域\(K\)上的一个 \textbf{\(m\)行\(n\)列矩阵} ,简称为 \(m\times n\) \textbf{矩阵}
\end{definition}

方程组中未知量的系数就可以排成一个矩阵,称\(A\)为方程组的 \textbf{系数矩阵} ,如果添加
常数项,则有
\begin{equation*}
\bbar{A}=
\begin{bmatrix}
a_{11}&a_{12}&\dots&a_{1n}&b_1\\
a_{21}&a_{22}&\dots&a_{2n}&b_2\\
\vdots&\vdots&&\vdots\\
a_{m1}&a_{m2}&\dots&a_{mn}&b_m
\end{bmatrix}
\end{equation*}
矩阵\(\bbar{A}\)称为方程组的 \textbf{增广矩阵}

在数域 \(K\)上的线性方程组,如果常数项 \(b_1=b_2=\dots=b_m=0\) ,则称为数域
\(K\)上的一个 \textbf{齐次线性方程组} ,这类方程的一般形式是
\begin{equation*}
\begin{cases}
a_{11}x_1+\dots+a_{1n}x_n=0\\
a_{21}x_1+\dots+a_{2n}x_n=0\\
\dots\\
a_{m1}x_1+\dots+a_{mn}x_n=0
\end{cases}
\end{equation*}
方程组显然有一组解
\begin{equation*}
x_1=0,\dots,x_n
\end{equation*}
这组解称为 \textbf{零解} 或 \textbf{平凡解} ,除此之外的解称为 \textbf{非零解} 或 \textbf{非平凡解}

\begin{proposition}[]
数域\(K\)上的齐次线性方程组中,如果方程个数\(m\)小于未知量个数\(n\),则它必有
非零解
\end{proposition}

\begin{proof}
对方程个数\(m\)作归纳

当\(m=1\),若\(a_{11}=0\),则令\(x_1=1,x_2=\dots=x_n=0\)即为一组非零解;否则,
因\(n\iffalse<\fi>m=1\),取\(x_1=-a_{12},x_2=a_{11},x_3=\dots=x_n\)。现设有\(m-1\)个方程
的齐次线性方程组

若方程组中\(x_1\)的系数全为零,则取\(x_1=1,x_2=\dots=x_n=0\)。否则调换方程的
次序,总可使第一个方程\(x_1\)的系数不为 0,因而不妨设\(a_{11}\neq0\),方程组可
化为
\begin{equation*}
\begin{cases}
a_{11}x_1+a_{12}x_2+\dots+a_{1n}x_n=0\\
\hspace{1.1cm}b_{22}x_2+\dots+b_{2n}x_n=0\\
\hspace{1.1cm}\dots\\
\hspace{1.1cm}b_{m2}x_2+\dots+b_{mn}x_n=0
\end{cases}
\end{equation*}
上述方程后面\(m-1\)个方程是有\(n-1\)个未知量\(x_2,\dots,x_n\)和\(m-1\)个方程
的齐次线性方程组,故有非零解
\end{proof}
\section{向量空间和矩阵}
\label{sec:org9407bc6}
\subsection{\(m\)维向量空间}
\label{sec:org7e86828}
\begin{definition}[]
设\(K\)是一个数域,\(K\)中\(m\)个数\(a_1,\dots,a_m\)所组成的一个\(m\)元有序数
组
\begin{equation*}
\alpha=
\begin{bmatrix}
a_1\\a_2\\\vdots\\a_m
\end{bmatrix}(a_i\in K, i=1,2,\dots,m)
\end{equation*}
称为一个 \textbf{\(m\)维向量} , \(a_i\) 称为它的第 \(i\)个 \textbf{分量} 或 \textbf{坐标} 。 \(K\)上的
全体\(m\)维向量所组成的集合记为 \(K^m\),在\(K^m\)内定义两个向量的 \textbf{加法} 如下
\begin{equation*}
\begin{bmatrix}
a_1\\a_2\\\vdots\\a_m
\end{bmatrix}+
\begin{bmatrix}
b_1\\b_2\\\vdots\\b_m
\end{bmatrix}=
\begin{bmatrix}
a_1+b_1\\a_2+b_2\\\vdots\\a_m+b_m
\end{bmatrix}\in K^m
\end{equation*}
又设 \(k\in K\),定义 \(k\) 与\(K^m\)中向量的 \textbf{数乘} 如下
\begin{equation*}
k
\begin{bmatrix}
a_1\\a_2\\\vdots\\a_m
\end{bmatrix}=
\begin{bmatrix}
ka_1\\ka_2\\\vdots\\ka_m
\end{bmatrix}\in K^m
\end{equation*}
集合 \(K^m\) 和上面定义的加法,数乘运算这一系统称为数域 \(K\)上的 \textbf{\(m\)维向
量空间}
\end{definition}

\begin{proposition}[]
\(K^m\)中向量加法、数乘满足如下八条运算性质
\begin{enumerate}
\item 加法结合律: \(\alpha+(\beta+\gamma)=(\alpha+\beta)+\gamma\)
\item 加法交换律: \(\alpha+\beta=\beta+\alpha\)
\item 称\((0,0,\dots,0)\)为\(m\)维 \textbf{零向量} ,记为 0,对任意\(m\)维向量 \(\alpha\) ,有
\(0+\alpha=\alpha=\alpha+0\)
\item 任给\(\alpha=(a_1,\dots,a_m)\),记\(-\alpha=(-a_1,\dots,-a_m)\),称其为 \(\alpha\)
的 \textbf{负向量} ,它满足 \(\alpha+(-\alpha)=(-\alpha+0)=0\)
\item 对数 1,有\(1\cdot\alpha=\alpha\)
\item 对\(K\)内任意数\(k,l\),有\((kl)\alpha=k(l\alpha)\)
\item 对\(K\)内任意数\(k,l\),有\((k+l)\alpha=k\alpha+l\alpha\)
\item 对\(K\)内任意数\(k\),有\(k(\alpha+\beta)=k\alpha+k\beta\)
\end{enumerate}
\end{proposition}

\begin{definition}[]
给定\(K^m\)内的向量组\(\alpha_1,\dots,\alpha_s\),又给定数域\(K\)内\(s\)个数
\(k_1,\dots,k_s\),称向量\(k_1\alpha_1+k_2\alpha_2+\dots+k_s\alpha_s\)为向量
组\(\alpha_1,\dots,\alpha_s\)的一个 \textbf{线性组合}
\end{definition}

\begin{definition}[]
给定 \(K^m\)内向量组\(\alpha_1,\dots,\alpha_s\),设 \(\beta\) 是 \(K^m\) 内一个向量,
如果存在数域 \(K\) 内 \(s\)个数\(k_1,\dots,k_s\)使
\begin{equation*}
\beta=k_1\alpha_1+\dots+k_s\alpha_s
\end{equation*}
则称 \(\beta\) 可被向量组 \(\alpha_1,\dots,\alpha_s\) \textbf{线性表示}
\end{definition}

给定数域\(K\)上的线性方程组
\begin{equation*}
\begin{cases}
a_{11}x_1+\dots +a_{1n}x_n=b_1\\
\dots\\
a_{m1}x_1+\dots+a_{mn}x_n=b_m
\end{cases}
\end{equation*}

考虑 \(K^m\)中的\(n+1\)个向量
\begin{equation*}
\alpha_1=
\begin{bmatrix}
a_{11}\\a_{21}\\\vdots\\a_{m1}
\end{bmatrix},\dots
\alpha_n=
\begin{bmatrix}
a_{1n}\\a_{2n}\\\vdots\\a_{mn}
\end{bmatrix},
\beta=
\begin{bmatrix}
b_{1}\\b_{2}\\\vdots\\b_{m}
\end{bmatrix}
\end{equation*}
应用\(m\)维向量 d 额加法和数乘运算,方程组可改写成
\begin{equation*}
x_1\alpha_1+\dots+x_n\alpha_n=\beta
\end{equation*}
如果方程组有一组解
\begin{equation*}
x_1=k_1,\dots,x_n=k_n(k_i\in K)
\end{equation*}
则
\begin{equation*}
\beta=k_1\alpha+\dots+k_n\alpha
\end{equation*}
即 \(\beta\) 能被向量组 \(\alpha_1,\dots,\alpha_n\) 线性表示。反之,若 \(\beta\) 能被向量组
\(\alpha_1,\dots,\alpha_n\)线性表示,则表示的系数就是方程组的一组解,于是有
\begin{enumerate}
\item 方程组有解当且仅当 \(\beta\) 能被向量组\(\alpha_1,\dots,\alpha_n\)线性表示
\item 方程组的解数等于线性表示法的组数
\end{enumerate}
\begin{definition}
给定\(K^m\)中的一个向量组
\begin{equation*}
\alpha_1=
\begin{bmatrix}
a_{11}\\a_{21}\\\vdots\\a_{m1}
\end{bmatrix}   ,\dots
\alpha_s=
\begin{bmatrix}
a_{1s}\\a_{2s}\\\vdots\\a_{ms}
\end{bmatrix}
\end{equation*}
如果齐次线性方程组
\begin{equation*}
\begin{cases}
a_{11}x_1+\dots+a_{1s}x_s=0\\
\dots\\
a_{m1}x_1+\dots+a_{ms}x_s=0
\end{cases}
\end{equation*}
有非零解,则称向量组\(\alpha_1,\dots,\alpha_s\) \textbf{线性相关} ,如果齐次线性方程组
只有零解,则称此向量组 \textbf{线性无关}
\end{definition}

\begin{proposition}[]
给定\(K^5\)内向量组
\begin{alignat*}{2}
&\alpha_1=(7,0,0,0,0)&&\alpha_2=(-1,3,4,0,0)\\
&\alpha_3=(1,0,1,1,0)\quad&&\alpha_4=(0,0,1,1,-1)
\end{alignat*}
判断它们是否线性相关
\end{proposition}

\begin{proof}
把它们竖起来排成一个\(5\times4\)矩阵
\begin{equation*}
A=
\begin{bmatrix}
7&-1&1&0\\
0&3&0&0\\
0&4&1&1\\
0&0&1&1\\
0&0&0&-1
\end{bmatrix}
\end{equation*}
用矩阵消元法把\(A\)化为阶梯形
\begin{equation*}
\begin{bmatrix}
7&-1&1&0\\
0&1&0&0\\
0&0&1&1\\
0&0&0&1\\
0&0&0&0
\end{bmatrix}
\end{equation*}
最后的阶梯形矩阵对应的齐次线性方程组显然只有零解,故以 \(A\) 为系数矩阵的齐次
线性方程组也只有零解,即\(\alpha_1,\alpha_2,\alpha_3,\alpha_4\)线性无关
\end{proof}

\begin{definition}[]
给定\(K^m\)内向量组\(\alpha_1,\dots,\alpha_s\),如果存在\(K\)内不全为零的数
\(k_1,\dots,k_s\)使
\begin{equation*}
k_1\alpha_1+\dots+k_s\alpha_s=0
\end{equation*}
则称向量组线性相关,否则称为线性无关
\end{definition}

\begin{proposition}[]
\(K^m\)内 l 向量组\(\alpha_1,\dots,\alpha_s(s\ge2)\)线性相关的 g 充分必要条件是其
中存在一个向量能被其余向量线性表示
\end{proposition}

\begin{corollary}[]
如果\(K^m\)内向量组\(\alpha_1,\dots,\alpha_s(s\ge2)\)中任一向量都不能被其余向
量线性表示,则此向量组线性无关
\end{corollary}

给定\(K^n\)中如下\(n\)个向量
\begin{align*}
&\epsilon_1=(1,0,\dots,0),\\
&\epsilon_2=(0,1,\dots,0),\\
&\dots\\
&\epsilon_n=(0,0,\dots,1)
\end{align*}
称之为数域\(K\)上\(n\)维向量空间的\(n\)个 \textbf{坐标向量}

\begin{definition}[]
给定\(K^m\)内两个向量组
\begin{align}
&   \alpha_1,\alpha_2,\dots,\alpha_r\label{eq1.1}\\
&\beta_1,\beta_2,\dots,\beta_s\label{eq1.2}\\
\end{align}
如果向量组 \eqref{eq1.2} 中每一个向量都能被向量组 \eqref{eq1.1} 线性表示,反过来
也成立,则称向量组 \eqref{eq1.1} 和向量组 \eqref{eq1.2} \textbf{线性等价}
\end{definition}

\begin{proposition}[]
给定\(K^m\)内两个向量组
\begin{align}
&\alpha_1,\alpha_2,\dots,\alpha_r\label{eq1.11}\\
&\beta_1,\beta_2,\dots,\beta_s\label{eq1.22}
\end{align}
且 \eqref{eq1.22} 中每一个向量 \(\beta_i\) 均能被向量组 \eqref{eq1.11} 线性表示,
那么当向量 \(\gamma\) 能被向量组 \eqref{eq1.22} 线性表示时,它也能被向量组 \eqref{eq1.11}
线性表示
\end{proposition}

线性等价:自反性、对称性、传递性

\begin{definition}[]
给定\(K^m\)内向量组
\begin{equation*}
\alpha_1,\dots,\alpha_s
\end{equation*}
如果它的一个部分组
\begin{equation*}
\alpha_{i_1},\dots,\alpha_{i_r}
\end{equation*}
满足
\begin{enumerate}
\item 向量组能被部分组线性表示
\item 部分组线性无关
\end{enumerate}


则称部分组是 \textbf{极大线性无关部分组}
\end{definition}

\begin{proposition}[]
给定\(K^m\)内两个向量组
\begin{align}
&\alpha_1,\dots,\alpha_r\label{eq1.1.4.1}\\
&\beta_1,\dots,\beta_s\label{eq1.1.4.2}
\end{align}
如果向量组 \eqref{eq1.1.4.1} 中每个向量都能被 \eqref{eq1.1.4.2} 线性表示,且
\(r\iffalse<\fi>s\),则向量组 \eqref{eq1.1.4.1} 线性相关
\end{proposition}

\begin{definition}[]
给定\(K^m\)内向量组
\begin{equation*}
\alpha_1,\dots,\alpha_s
\end{equation*}
设它的某一个极大线性无关部分组
\begin{equation*}
\alpha_{i_1},\dots,\alpha_{i_r}
\end{equation*}
又有另一个向量组
\begin{equation*}
\beta_1,\dots,\beta_t
\end{equation*}
设它的某一个极大线性无关部分组为
\begin{equation*}
\beta_{j_1},\dots,\beta_{j_l}
\end{equation*}
若\((\alpha)\)与\((\beta)\)线性等价,则\(r=l\)
\end{definition}

\begin{proof}
线性等价的传递性,两个部分组也等价
\end{proof}

\begin{corollary}[]
一个向量组的任意两个极大线性无关部分组中包含的向量个数相同
\end{corollary}

\begin{definition}[]
一个向量组的极大线性无关部分组中包含的向量个数称为该向量组的 \textbf{秩} ,全由零向量
组成的向量组的秩为零
\end{definition}


\begin{corollary}[]
两个线性等价的向量组的秩相等
\end{corollary}

\begin{proposition}[]
给定\(K^m\)内向量组
\begin{equation*}
\alpha_1,\dots,\alpha_n
\end{equation*}
其中\(\alpha_1\neq0\),作如下筛选:保持\(\alpha_1\)不懂,若\(\alpha_2\)可被
\(\alpha_1\)线性表示,则去掉\(\alpha_2\),否则保留,若\(\alpha_i\)可被前面保
留下来的向量线性表示,则去掉,否则保留,经\(n\)此筛选后,得到的向量组是
\begin{equation*}
\alpha_{i_1}=\alpha_1,\alpha_{i_2},\dots,\alpha_{i_r}
\end{equation*}
则\(\alpha_{i_1}\)是一个极大线性无关部分组
\end{proposition}
\subsection{矩阵的秩}
\label{sec:orgf2ae0b2}
给定数域\(K\)上一个\(m\times n\)矩阵\(A\),它的每一列可以看成一个\(m\)维向量,
它有\(n\)列,组成一个\(m\)维向量组,我们称之为矩阵\(A\)的 \textbf{列向量组} ,同样,它
的每一行可以看作一个\(n\)维向量,称为\(A\)的 \textbf{行向量组}

\begin{definition}[]
一个矩阵的行向量组的秩称为 \textbf{行秩} ,列向量组的秩称为 \textbf{列秩}
\end{definition}

设矩阵\(A\)的列向量组为\(\alpha_1,\dots,\alpha_n\),则\(A\)可写成
\begin{equation*}
A=(\alpha_1,\dots,\alpha_n)
\end{equation*}

\begin{definition}[]
对数域\(K\)上的\(m\times n\)矩阵\(A\)的行(列)作如下变换
\begin{enumerate}
\item 互换两行(列)的位置
\item 把某一行(列)乘以\(K\)内一个非零常数\(c\)
\item 把第\(j\)行(列)加上第\(i\)行(列)的\(k\)倍,这里\(k\in K\)且\(i\neq j\)
\end{enumerate}


上述三种变换的每一种都称为矩阵\(A\)的 \textbf{初等行(列)变换}
\end{definition}

\begin{proposition}[]
矩阵\(A\)的行秩在初等行变换下保持不变,列秩在初等列变化下保持不变
\end{proposition}

把矩阵\(A\)的行与列互换后得到\(A'\)称为\(A\)的转置矩阵

\begin{proposition}[]
矩阵的行秩在初等列变化下保持不变;矩阵的列秩在初等列变化下保持不变
\end{proposition}

\begin{proof}
证\(A\)的列秩在初等行变换下保持不变。设\(A\)的列向量组为
\(\alpha_1,\dots,\alpha_n\),其列秩为\(r\)。不妨设\(\alpha_1,\dots,\alpha_r\)
为列向量组的一个极大线性无关部分组,假定\(A\)经过初等行变换后所得新矩阵的列向
量组为\(\alpha_1',\dots,\alpha_n'\),只要证\(\alpha_1',\dots,\alpha_r'\)是极
大线性无关
\begin{enumerate}
\item \(\alpha_1',\dots,\alpha_r'\)线性无关。以\(\alpha_1,\dots,\alpha_r\)为列向
量排成一矩阵\(B\)。因\(\alpha_1,\dots,\alpha_r\)线性无关,故以\(B\)为系数
矩阵的齐次线性方程组只有零解;另一方面,以\(\alpha_1',\dots,\alpha_r'\)为
列向量排成矩阵\(B_1\),\(B_1\)是\(B\)经过初等行变换得到的,它们同解
\item 考察以\(\alpha_1,\dots,\alpha_r,\alpha_i\)为列向量组的矩阵\(\bbar{B}\)。因
\(\alpha_i\)可被\(\alpha_1,\dots,\alpha_r\)线性表示,故以\(\bbar{B}\)为增
广矩阵的线性方程组有解;另一方面,以
\(\alpha_1',\dots,\alpha_r',\alpha_i'\)为列向量组的矩阵\(\bbar{B_1}\)可由
\(\bbar{B}\)经初等行变换得到,说明\(\alpha_i'\)可被
\(\alpha_1',\dots,\alpha_r'\)线性表示
\end{enumerate}
\end{proof}

\begin{corollary}[]
设\(A\)是数域\(K\)上的\(m\times m\)矩阵,\(A\)经若干此初等行变换化为矩阵\(B\),
设\(A\)的列向量组是\(\alpha_1,\dots,\alpha_n\),\(B\)的列向量组是
\(\alpha_1',\dots,\alpha_n'\),我们有
\begin{enumerate}
\item 如果\(\alpha_{i_1},\dots,\alpha_{i_r}\)是\(A\)的列向量组的一个极大线性无关
部分组,则\(\alpha_{i_1}',\dots,\alpha_{i_r}'\)是\(B\)的列向量组的一个极大
线性无关部分组。而且当
\begin{equation*}
\alpha_i=k_1\alpha_{i_1}+\dots+k_r\alpha_{i_r}
\end{equation*}
时,有\(\alpha_i'=k_1\alpha_{i_1}'+\dots+k_r\alpha_{i_r}'\)
\item 如果\(\alpha_{i_1}‘,\dots,\alpha_{i_r}’\)是\(B\)的列向量组的一个极大线性无关
部分组,则\(\alpha_{i_1},\dots,\alpha_{i_r}\)是\(A\)的列向量组的一个极大
线性无关部分组。而且当
\begin{equation*}
\alpha_i'=k_1\alpha_{i_1}'+\dots+k_r\alpha_{i_r}'
\end{equation*}
时,有\(\alpha_i=k_1\alpha_{i_1}+\dots+k_r\alpha_{i_r}\)
\end{enumerate}
\end{corollary}

如果一个\(m\times n\)矩阵其所有元素都是 0,则称为 \textbf{零矩阵} ,记作 0。下面设
\(A\neq0\)
\begin{enumerate}
\item 在矩阵\(A\)中,如果\(a_{11}=0\),我们就在矩阵中找一个不为零的元素,设为
\(a_{ij}\),现对换\(1,i\)两行,再对换\(1,j\)两列
\item 若\(a_{11}\neq0\),利用初等行变换把\(A\)变成如下形状
\begin{equation*}
A\to
\begin{bmatrix}
a_{11}&a_{12}&\dots&a_{1n}\\
0&b_{22}&\dots&b_{2n}\\
\vdots&\vdots&&\vdots\\
0&b_{m2}&\dots&b_{mn}
\end{bmatrix}
\end{equation*}
再利用初等列变换把\(A\)进一步变为
\begin{equation*}
A\to
\begin{bmatrix}
1&0&\dots&0\\
0&b_{22}&\dots&b_{2n}\\
\vdots&\vdots&&\vdots\\
0&b_{m2}&\dots&b_{mn}
\end{bmatrix}
\end{equation*}
如此继续对右下角\((m-1)\times(n-1)\)矩阵重复
\end{enumerate}



如果连续施行上述初等行、列变换之后,矩阵\(A\)可化成

 \begin{tabular}{p{4.8cm} p{3.4cm} p{3cm}}
\begin{equation*}
  \begin{bmatrix}
    \begin{matrix}
      1 &&\\&\ddots&\\&&1\\
    \end{matrix}&&\bigzero\\
    \bigzero&
    \begin{matrix}
      0&&\\&\ddots&\\&&0\\
    \end{matrix}&
    \begin{matrix}&\\&\\\dots&0
    \end{matrix}
  \end{bmatrix}
\end{equation*}&
\begin{equation*}
  \begin{bmatrix}
    \begin{matrix}
      1 &&\\&\ddots&\\&&1\\
    \end{matrix}&\bigzero\\
    \bigzero&
    \begin{matrix}
      0&&\\&\ddots&\\&&0\\
    \end{matrix}
  \end{bmatrix}
\end{equation*}&
\begin{equation*}
  \begin{bmatrix}
    \begin{matrix}
      1 &&\\&\ddots&\\&&1\\
    \end{matrix}&\bigzero\\&
    \begin{matrix}
      0&&\\
      &\ddots&\\
      &&0\\
    \end{matrix}\\\bigzero&
    \begin{matrix}
      0&\ddots &\vdots\\& &0
    \end{matrix}
  \end{bmatrix}
\end{equation*}\\
  $n>m$&$n=m$&$n<m$
 \end{tabular}
\end{document}
