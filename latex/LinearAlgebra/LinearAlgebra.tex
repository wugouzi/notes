% Created 2020-09-08 二 16:41
% Intended LaTeX compiler: pdflatex
\documentclass[11pt]{article}
\usepackage[utf8]{inputenc}
\usepackage[T1]{fontenc}
\usepackage{graphicx}
\usepackage{grffile}
\usepackage{longtable}
\usepackage{wrapfig}
\usepackage{rotating}
\usepackage[normalem]{ulem}
\usepackage{amsmath}
\usepackage{textcomp}
\usepackage{amssymb}
\usepackage{capt-of}
\usepackage{hyperref}
\usepackage{minted}
% TIPS
% \substack{a\\b} for multiple lines text





% pdfplots will load xolor automatically without option
\usepackage[dvipsnames]{xcolor}

\usepackage{forest}
% two-line text in node by [two \\ lines]
% \begin{forest} qtree, [..] \end{forest}
\forestset{
  qtree/.style={
    baseline,
    for tree={
      parent anchor=south,
      child anchor=north,
      align=center,
      inner sep=1pt,
    }}}
%\usepackage{flexisym}
% load order of mathtools and mathabx, otherwise conflict overbrace

\usepackage{mathtools}
%\usepackage{fourier}
\usepackage{pgfplots}
\usepackage{amsthm, mathabx,  amsmath, commath}
\usepackage{amsfonts}

\usepackage{empheq}
\usepackage{tikz}
\usetikzlibrary{arrows.meta}
\usepackage[most]{tcolorbox}

\newtheorem{theorem}{Theorem}[section]
\newtheorem{definition}{Definition}[section]
\newtheorem{corollary}{Corollary}[section]
\newtheorem{example}{Example}[section]
\newtheorem{lemma}{Lemma}[section]
\newtheorem{proposition}{Proposition}[section]

\newcommand{\bl}[1] {\boldsymbol{#1}}
\newcommand{\Wt}[1] {\stackrel{\sim}{\smash{#1}\rule{0pt}{1.1ex}}}
\newcommand{\wt}[1] {\widetilde{#1}}


%For boxed texts in align, use Aboxed{}
%otherwise use boxed{}

\DeclareMathSymbol{\widehatsym}{\mathord}{largesymbols}{"62}
\newcommand\lowerwidehatsym{%
  \text{\smash{\raisebox{-1.3ex}{%
    $\widehatsym$}}}}
\newcommand\fixwidehat[1]{%
  \mathchoice
    {\accentset{\displaystyle\lowerwidehatsym}{#1}}
    {\accentset{\textstyle\lowerwidehatsym}{#1}}
    {\accentset{\scriptstyle\lowerwidehatsym}{#1}}
    {\accentset{\scriptscriptstyle\lowerwidehatsym}{#1}}
}

\usepackage{graphicx}
    
% text on arrow for xRightarrow
\makeatletter
%\newcommand{\xRightarrow}[2][]{\ext@arrow 0359\Rightarrowfill@{#1}{#2}}
\makeatother


\def \bx {\boldsymbol{x}}
\def \ba {\boldsymbol{a}}
\def \bI {\boldsymbol{I}}
\def \bt {\boldsymbol{t}}
\def \bb {\boldsymbol{b}}
\def \bA {\boldsymbol{A}}
\def \bX {\boldsymbol{X}}
\def \bu {\boldsymbol{u}}
\def \bS {\boldsymbol{S}}
\def \bZ {\boldsymbol{Z}}
\def \bz {\boldsymbol{z}}
\def \by {\boldsymbol{y}}
\def \bw {\boldsymbol{w}}
\def \bT {\boldsymbol{T}}
\def \bS {\boldsymbol{S}}
\def \bm {\boldsymbol{m}}
\def \bW {\boldsymbol{W}}
\def \bY {\boldsymbol{Y}}
\def \bH {\boldsymbol{H}}
\def \blambda {\boldsymbol{\lambda}}
\def \bPhi {\boldsymbol{\Phi}}
\def \btheta {\boldsymbol{\theta}}
\def \bmu {\boldsymbol{\mu}}
\def \bphi {\boldsymbol{\phi}}
\def \bSigma {\boldsymbol{\Sigma}}
\def \lb {\left\{}
\def \rb {\right\}}
\def \caln {\mathcal{N}}
\def \dissum {\displaystyle\Sigma}
\def \dispro {\displaystyle\prod}
\def \E {\mathbb{E}}
\def \Q {\mathbb{Q}}
\def \V {\mathbb{V}}
\def \R {\mathbb{R}}
\def \calq {\mathcal{Q}}
\def \calg {\mathcal{G}}
\def \caln {\mathcal{N}}
\def \calr {\mathcal{R}}
\def \calm {\mathcal{M}}
\def \calc {\mathcal{C}}
\def \bcup {\bigcup}

\usepackage[UTF8]{ctex}
\author{wu}
\date{\today}
\title{高等代数简明教程}
\hypersetup{
 pdfauthor={wu},
 pdftitle={高等代数简明教程},
 pdfkeywords={},
 pdfsubject={},
 pdfcreator={Emacs 26.3 (Org mode 9.4)}, 
 pdflang={English}}
\begin{document}

\maketitle
\tableofcontents \clearpage\section{代数学的经典课题}
\label{sec:orgfc059f3}
\subsection{线性方程组}
\label{sec:orged52cb9}
\begin{equation*}
\begin{cases}
a_{11}x_1+\dots+a_{1n}x_n=b_1,\\
a_{21}x_1+\dots+a_{2n}x_n=b_2,\\
\dots\\
a_{m1}x_1+\dots+a_{mn}x_n=b_m
\end{cases}
\end{equation*}
设在线性方程组中,未知量的系数\(a_{ij}\)和常数项\(b_1,\dots,b_m\)都属于数域
\(K\),则称它是 \textbf{数域\(K\)上的线性方程组}

\begin{definition}[]
\textbf{初等变换}
\begin{enumerate}
\item 互换两个方程的位置
\item 把某一个方程两边同乘数域\(K\)内的一个非零常数\(c\)
\item 把第\(j\)个方程加上第\(i\)个方程的\(k\)倍,这里\(k\in K\)且\(i\neq  j\)
\end{enumerate}
\end{definition}

\begin{proposition}[]
设方程组经过某一初等变换后变为另一个方程组,则新方程组与原方程组同解
\end{proposition}

\begin{definition}[]
给定数域\(K\)上\(mn\)个数\(a_{ij}(i=1,2,\dots,m;j=1,\dots,n)\),把它们按一定
次序排成一个\(m\)行\(n\)列的长方形表格
\begin{equation*}
A=
\begin{bmatrix}
a_{11}&a_{12}&\dots&a_{1n}\\
a_{21}&a_{22}&\dots&a_{2n}\\
\vdots&\vdots&&\vdots\\
a_{m1}&a_{m2}&\dots&a_{mn}
\end{bmatrix}
\end{equation*}
称为数域\(K\)上的一个 \textbf{\(m\)行\(n\)列矩阵} ,简称为 \(m\times n\) \textbf{矩阵}
\end{definition}

方程组中未知量的系数就可以排成一个矩阵,称\(A\)为方程组的 \textbf{系数矩阵} ,如果添加
常数项,则有
\begin{equation*}
\bbar{A}=
\begin{bmatrix}
a_{11}&a_{12}&\dots&a_{1n}&b_1\\
a_{21}&a_{22}&\dots&a_{2n}&b_2\\
\vdots&\vdots&&\vdots\\
a_{m1}&a_{m2}&\dots&a_{mn}&b_m
\end{bmatrix}
\end{equation*}
矩阵\(\bbar{A}\)称为方程组的 \textbf{增广矩阵}

在数域 \(K\)上的线性方程组,如果常数项 \(b_1=b_2=\dots=b_m=0\) ,则称为数域
\(K\)上的一个 \textbf{齐次线性方程组} ,这类方程的一般形式是
\begin{equation*}
\begin{cases}
a_{11}x_1+\dots+a_{1n}x_n=0\\
a_{21}x_1+\dots+a_{2n}x_n=0\\
\dots\\
a_{m1}x_1+\dots+a_{mn}x_n=0
\end{cases}
\end{equation*}
方程组显然有一组解
\begin{equation*}
x_1=0,\dots,x_n
\end{equation*}
这组解称为 \textbf{零解} 或 \textbf{平凡解} ,除此之外的解称为 \textbf{非零解} 或 \textbf{非平凡解}

\begin{proposition}[]
数域\(K\)上的齐次线性方程组中,如果方程个数\(m\)小于未知量个数\(n\),则它必有
非零解
\end{proposition}

\begin{proof}
对方程个数\(m\)作归纳

当\(m=1\),若\(a_{11}=0\),则令\(x_1=1,x_2=\dots=x_n=0\)即为一组非零解;否则,
因\(n\iffalse<\fi>m=1\),取\(x_1=-a_{12},x_2=a_{11},x_3=\dots=x_n\)。现设有\(m-1\)个方程
的齐次线性方程组

若方程组中\(x_1\)的系数全为零,则取\(x_1=1,x_2=\dots=x_n=0\)。否则调换方程的
次序,总可使第一个方程\(x_1\)的系数不为 0,因而不妨设\(a_{11}\neq0\),方程组可
化为
\begin{equation*}
\begin{cases}
a_{11}x_1+a_{12}x_2+\dots+a_{1n}x_n=0\\
\hspace{1.1cm}b_{22}x_2+\dots+b_{2n}x_n=0\\
\hspace{1.1cm}\dots\\
\hspace{1.1cm}b_{m2}x_2+\dots+b_{mn}x_n=0
\end{cases}
\end{equation*}
上述方程后面\(m-1\)个方程是有\(n-1\)个未知量\(x_2,\dots,x_n\)和\(m-1\)个方程
的齐次线性方程组,故有非零解
\end{proof}
\section{向量空间和矩阵}
\label{sec:orgbd9790b}
\subsection{\(m\)维向量空间}
\label{sec:orgb02a158}
\begin{definition}[]
设\(K\)是一个数域,\(K\)中\(m\)个数\(a_1,\dots,a_m\)所组成的一个\(m\)元有序数
组
\begin{equation*}
\alpha=
\begin{bmatrix}
a_1\\a_2\\\vdots\\a_m
\end{bmatrix}(a_i\in K, i=1,2,\dots,m)
\end{equation*}
称为一个 \textbf{\(m\)维向量} , \(a_i\) 称为它的第 \(i\)个 \textbf{分量} 或 \textbf{坐标} 。 \(K\)上的
全体\(m\)维向量所组成的集合记为 \(K^m\),在\(K^m\)内定义两个向量的 \textbf{加法} 如下
\begin{equation*}
\begin{bmatrix}
a_1\\a_2\\\vdots\\a_m
\end{bmatrix}+
\begin{bmatrix}
b_1\\b_2\\\vdots\\b_m
\end{bmatrix}=
\begin{bmatrix}
a_1+b_1\\a_2+b_2\\\vdots\\a_m+b_m
\end{bmatrix}\in K^m
\end{equation*}
又设 \(k\in K\),定义 \(k\) 与\(K^m\)中向量的 \textbf{数乘} 如下
\begin{equation*}
k
\begin{bmatrix}
a_1\\a_2\\\vdots\\a_m
\end{bmatrix}=
\begin{bmatrix}
ka_1\\ka_2\\\vdots\\ka_m
\end{bmatrix}\in K^m
\end{equation*}
集合 \(K^m\) 和上面定义的加法,数乘运算这一系统称为数域 \(K\)上的 \textbf{\(m\)维向
量空间}
\end{definition}

\begin{proposition}[]
\(K^m\)中向量加法、数乘满足如下八条运算性质
\begin{enumerate}
\item 加法结合律: \(\alpha+(\beta+\gamma)=(\alpha+\beta)+\gamma\)
\item 加法交换律: \(\alpha+\beta=\beta+\alpha\)
\item 称\((0,0,\dots,0)\)为\(m\)维 \textbf{零向量} ,记为 0,对任意\(m\)维向量 \(\alpha\) ,有
\(0+\alpha=\alpha=\alpha+0\)
\item 任给\(\alpha=(a_1,\dots,a_m)\),记\(-\alpha=(-a_1,\dots,-a_m)\),称其为 \(\alpha\)
的 \textbf{负向量} ,它满足 \(\alpha+(-\alpha)=(-\alpha+0)=0\)
\item 对数 1,有\(1\cdot\alpha=\alpha\)
\item 对\(K\)内任意数\(k,l\),有\((kl)\alpha=k(l\alpha)\)
\item 对\(K\)内任意数\(k,l\),有\((k+l)\alpha=k\alpha+l\alpha\)
\item 对\(K\)内任意数\(k\),有\(k(\alpha+\beta)=k\alpha+k\beta\)
\end{enumerate}
\end{proposition}

\begin{definition}[]
给定\(K^m\)内的向量组\(\alpha_1,\dots,\alpha_s\),又给定数域\(K\)内\(s\)个数
\(k_1,\dots,k_s\),称向量\(k_1\alpha_1+k_2\alpha_2+\dots+k_s\alpha_s\)为向量
组\(\alpha_1,\dots,\alpha_s\)的一个 \textbf{线性组合}
\end{definition}

\begin{definition}[]
给定 \(K^m\)内向量组\(\alpha_1,\dots,\alpha_s\),设 \(\beta\) 是 \(K^m\) 内一个向量,
如果存在数域 \(K\) 内 \(s\)个数\(k_1,\dots,k_s\)使
\begin{equation*}
\beta=k_1\alpha_1+\dots+k_s\alpha_s
\end{equation*}
则称 \(\beta\) 可被向量组 \(\alpha_1,\dots,\alpha_s\) \textbf{线性表示}
\end{definition}

给定数域\(K\)上的线性方程组
\begin{equation*}
\begin{cases}
a_{11}x_1+\dots +a_{1n}x_n=b_1\\
\dots\\
a_{m1}x_1+\dots+a_{mn}x_n=b_m
\end{cases}
\end{equation*}

考虑 \(K^m\)中的\(n+1\)个向量
\begin{equation*}
\alpha_1=
\begin{bmatrix}
a_{11}\\a_{21}\\\vdots\\a_{m1}
\end{bmatrix},\dots
\alpha_n=
\begin{bmatrix}
a_{1n}\\a_{2n}\\\vdots\\a_{mn}
\end{bmatrix},
\beta=
\begin{bmatrix}
b_{1}\\b_{2}\\\vdots\\b_{m}
\end{bmatrix}
\end{equation*}
应用\(m\)维向量 d 额加法和数乘运算,方程组可改写成
\begin{equation*}
x_1\alpha_1+\dots+x_n\alpha_n=\beta
\end{equation*}
如果方程组有一组解
\begin{equation*}
x_1=k_1,\dots,x_n=k_n(k_i\in K)
\end{equation*}
则
\begin{equation*}
\beta=k_1\alpha+\dots+k_n\alpha
\end{equation*}
即 \(\beta\) 能被向量组 \(\alpha_1,\dots,\alpha_n\) 线性表示。反之,若 \(\beta\) 能被向量组
\(\alpha_1,\dots,\alpha_n\)线性表示,则表示的系数就是方程组的一组解,于是有
\begin{enumerate}
\item 方程组有解当且仅当 \(\beta\) 能被向量组\(\alpha_1,\dots,\alpha_n\)线性表示
\item 方程组的解数等于线性表示法的组数
\end{enumerate}
\begin{definition}
给定\(K^m\)中的一个向量组
\begin{equation*}
\alpha_1=
\begin{bmatrix}
a_{11}\\a_{21}\\\vdots\\a_{m1}
\end{bmatrix}   ,\dots
\alpha_s=
\begin{bmatrix}
a_{1s}\\a_{2s}\\\vdots\\a_{ms}
\end{bmatrix}
\end{equation*}
如果齐次线性方程组
\begin{equation*}
\begin{cases}
a_{11}x_1+\dots+a_{1s}x_s=0\\
\dots\\
a_{m1}x_1+\dots+a_{ms}x_s=0
\end{cases}
\end{equation*}
有非零解,则称向量组\(\alpha_1,\dots,\alpha_s\) \textbf{线性相关} ,如果齐次线性方程组
只有零解,则称此向量组 \textbf{线性无关}
\end{definition}

\begin{proposition}[]
给定\(K^5\)内向量组
\begin{alignat*}{2}
&\alpha_1=(7,0,0,0,0)&&\alpha_2=(-1,3,4,0,0)\\
&\alpha_3=(1,0,1,1,0)\quad&&\alpha_4=(0,0,1,1,-1)
\end{alignat*}
判断它们是否线性相关
\end{proposition}

\begin{proof}
把它们竖起来排成一个\(5\times4\)矩阵
\begin{equation*}
A=
\begin{bmatrix}
7&-1&1&0\\
0&3&0&0\\
0&4&1&1\\
0&0&1&1\\
0&0&0&-1
\end{bmatrix}
\end{equation*}
用矩阵消元法把\(A\)化为阶梯形
\begin{equation*}
\begin{bmatrix}
7&-1&1&0\\
0&1&0&0\\
0&0&1&1\\
0&0&0&1\\
0&0&0&0
\end{bmatrix}
\end{equation*}
最后的阶梯形矩阵对应的齐次线性方程组显然只有零解,故以 \(A\) 为系数矩阵的齐次
线性方程组也只有零解,即\(\alpha_1,\alpha_2,\alpha_3,\alpha_4\)线性无关
\end{proof}

\begin{definition}[]
给定\(K^m\)内向量组\(\alpha_1,\dots,\alpha_s\),如果存在\(K\)内不全为零的数
\(k_1,\dots,k_s\)使
\begin{equation*}
k_1\alpha_1+\dots+k_s\alpha_s=0
\end{equation*}
则称向量组线性相关,否则称为线性无关
\end{definition}

\begin{proposition}[]
\(K^m\)内 l 向量组\(\alpha_1,\dots,\alpha_s(s\ge2)\)线性相关的 g 充分必要条件是其
中存在一个向量能被其余向量线性表示
\end{proposition}

\begin{corollary}[]
如果\(K^m\)内向量组\(\alpha_1,\dots,\alpha_s(s\ge2)\)中任一向量都不能被其余向
量线性表示,则此向量组线性无关
\end{corollary}

给定\(K^n\)中如下\(n\)个向量
\begin{align*}
&\epsilon_1=(1,0,\dots,0),\\
&\epsilon_2=(0,1,\dots,0),\\
&\dots\\
&\epsilon_n=(0,0,\dots,1)
\end{align*}
称之为数域\(K\)上\(n\)维向量空间的\(n\)个 \textbf{坐标向量}

\begin{definition}[]
给定\(K^m\)内两个向量组
\begin{align}
&   \alpha_1,\alpha_2,\dots,\alpha_r\label{eq1.1}\\
&\beta_1,\beta_2,\dots,\beta_s\label{eq1.2}\\
\end{align}
如果向量组 \eqref{eq1.2} 中每一个向量都能被向量组 \eqref{eq1.1} 线性表示,反过来
也成立,则称向量组 \eqref{eq1.1} 和向量组 \eqref{eq1.2} \textbf{线性等价}
\end{definition}

\begin{proposition}[]
给定\(K^m\)内两个向量组
\begin{align}
&\alpha_1,\alpha_2,\dots,\alpha_r\label{eq1.11}\\
&\beta_1,\beta_2,\dots,\beta_s\label{eq1.22}
\end{align}
且 \eqref{eq1.22} 中每一个向量 \(\beta_i\) 均能被向量组 \eqref{eq1.11} 线性表示,
那么当向量 \(\gamma\) 能被向量组 \eqref{eq1.22} 线性表示时,它也能被向量组 \eqref{eq1.11}
线性表示
\end{proposition}

线性等价:自反性、对称性、传递性

\begin{definition}[]
给定\(K^m\)内向量组
\begin{equation*}
\alpha_1,\dots,\alpha_s
\end{equation*}
如果它的一个部分组
\begin{equation*}
\alpha_{i_1},\dots,\alpha_{i_r}
\end{equation*}
满足
\begin{enumerate}
\item 向量组能被部分组线性表示
\item 部分组线性无关
\end{enumerate}


则称部分组是 \textbf{极大线性无关部分组}
\end{definition}

\begin{proposition}[]
给定\(K^m\)内两个向量组
\begin{align}
&\alpha_1,\dots,\alpha_r\label{eq1.1.4.1}\\
&\beta_1,\dots,\beta_s\label{eq1.1.4.2}
\end{align}
如果向量组 \eqref{eq1.1.4.1} 中每个向量都能被 \eqref{eq1.1.4.2} 线性表示,且
\(r\iffalse<\fi>s\),则向量组 \eqref{eq1.1.4.1} 线性相关
\end{proposition}

\begin{definition}[]
给定\(K^m\)内向量组
\begin{equation*}
\alpha_1,\dots,\alpha_s
\end{equation*}
设它的某一个极大线性无关部分组
\begin{equation*}
\alpha_{i_1},\dots,\alpha_{i_r}
\end{equation*}
又有另一个向量组
\begin{equation*}
\beta_1,\dots,\beta_t
\end{equation*}
设它的某一个极大线性无关部分组为
\begin{equation*}
\beta_{j_1},\dots,\beta_{j_l}
\end{equation*}
若\((\alpha)\)与\((\beta)\)线性等价,则\(r=l\)
\end{definition}

\begin{proof}
线性等价的传递性,两个部分组也等价
\end{proof}

\begin{corollary}[]
一个向量组的任意两个极大线性无关部分组中包含的向量个数相同
\end{corollary}

\begin{definition}[]
一个向量组的极大线性无关部分组中包含的向量个数称为该向量组的 \textbf{秩} ,全由零向量
组成的向量组的秩为零
\end{definition}


\begin{corollary}[]
两个线性等价的向量组的秩相等
\end{corollary}

\begin{proposition}[]
给定\(K^m\)内向量组
\begin{equation*}
\alpha_1,\dots,\alpha_n
\end{equation*}
其中\(\alpha_1\neq0\),作如下筛选:保持\(\alpha_1\)不懂,若\(\alpha_2\)可被
\(\alpha_1\)线性表示,则去掉\(\alpha_2\),否则保留,若\(\alpha_i\)可被前面保
留下来的向量线性表示,则去掉,否则保留,经\(n\)此筛选后,得到的向量组是
\begin{equation*}
\alpha_{i_1}=\alpha_1,\alpha_{i_2},\dots,\alpha_{i_r}
\end{equation*}
则\(\alpha_{i_1}\)是一个极大线性无关部分组
\end{proposition}
\subsection{矩阵的秩}
\label{sec:org03e621a}
给定数域\(K\)上一个\(m\times n\)矩阵\(A\),它的每一列可以看成一个\(m\)维向量,
它有\(n\)列,组成一个\(m\)维向量组,我们称之为矩阵\(A\)的 \textbf{列向量组} ,同样,它
的每一行可以看作一个\(n\)维向量,称为\(A\)的 \textbf{行向量组}

\begin{definition}[]
一个矩阵的行向量组的秩称为 \textbf{行秩} ,列向量组的秩称为 \textbf{列秩}
\end{definition}

设矩阵\(A\)的列向量组为\(\alpha_1,\dots,\alpha_n\),则\(A\)可写成
\begin{equation*}
A=(\alpha_1,\dots,\alpha_n)
\end{equation*}

\begin{definition}[]
对数域\(K\)上的\(m\times n\)矩阵\(A\)的行(列)作如下变换
\begin{enumerate}
\item 互换两行(列)的位置
\item 把某一行(列)乘以\(K\)内一个非零常数\(c\)
\item 把第\(j\)行(列)加上第\(i\)行(列)的\(k\)倍,这里\(k\in K\)且\(i\neq j\)
\end{enumerate}


上述三种变换的每一种都称为矩阵\(A\)的 \textbf{初等行(列)变换}
\end{definition}

\begin{proposition}[]
矩阵\(A\)的行秩在初等行变换下保持不变,列秩在初等列变化下保持不变
\end{proposition}

把矩阵\(A\)的行与列互换后得到\(A'\)称为\(A\)的转置矩阵

\begin{proposition}[]
矩阵的行秩在初等列变化下保持不变;矩阵的列秩在初等列变化下保持不变
\end{proposition}

\begin{proof}
证\(A\)的列秩在初等行变换下保持不变。设\(A\)的列向量组为
\(\alpha_1,\dots,\alpha_n\),其列秩为\(r\)。不妨设\(\alpha_1,\dots,\alpha_r\)
为列向量组的一个极大线性无关部分组,假定\(A\)经过初等行变换后所得新矩阵的列向
量组为\(\alpha_1',\dots,\alpha_n'\),只要证\(\alpha_1',\dots,\alpha_r'\)是极
大线性无关
\begin{enumerate}
\item \(\alpha_1',\dots,\alpha_r'\)线性无关。以\(\alpha_1,\dots,\alpha_r\)为列向
量排成一矩阵\(B\)。因\(\alpha_1,\dots,\alpha_r\)线性无关,故以\(B\)为系数
矩阵的齐次线性方程组只有零解;另一方面,以\(\alpha_1',\dots,\alpha_r'\)为
列向量排成矩阵\(B_1\),\(B_1\)是\(B\)经过初等行变换得到的,它们同解
\item 考察以\(\alpha_1,\dots,\alpha_r,\alpha_i\)为列向量组的矩阵\(\bbar{B}\)。因
\(\alpha_i\)可被\(\alpha_1,\dots,\alpha_r\)线性表示,故以\(\bbar{B}\)为增
广矩阵的线性方程组有解;另一方面,以
\(\alpha_1',\dots,\alpha_r',\alpha_i'\)为列向量组的矩阵\(\bbar{B_1}\)可由
\(\bbar{B}\)经初等行变换得到,说明\(\alpha_i'\)可被
\(\alpha_1',\dots,\alpha_r'\)线性表示
\end{enumerate}
\end{proof}

\begin{corollary}[]
设\(A\)是数域\(K\)上的\(m\times m\)矩阵,\(A\)经若干此初等行变换化为矩阵\(B\),
设\(A\)的列向量组是\(\alpha_1,\dots,\alpha_n\),\(B\)的列向量组是
\(\alpha_1',\dots,\alpha_n'\),我们有
\begin{enumerate}
\item 如果\(\alpha_{i_1},\dots,\alpha_{i_r}\)是\(A\)的列向量组的一个极大线性无关
部分组,则\(\alpha_{i_1}',\dots,\alpha_{i_r}'\)是\(B\)的列向量组的一个极大
线性无关部分组。而且当
\begin{equation*}
\alpha_i=k_1\alpha_{i_1}+\dots+k_r\alpha_{i_r}
\end{equation*}
时,有\(\alpha_i'=k_1\alpha_{i_1}'+\dots+k_r\alpha_{i_r}'\)
\item 如果\(\alpha_{i_1}‘,\dots,\alpha_{i_r}’\)是\(B\)的列向量组的一个极大线性无关
部分组,则\(\alpha_{i_1},\dots,\alpha_{i_r}\)是\(A\)的列向量组的一个极大
线性无关部分组。而且当
\begin{equation*}
\alpha_i'=k_1\alpha_{i_1}'+\dots+k_r\alpha_{i_r}'
\end{equation*}
时,有\(\alpha_i=k_1\alpha_{i_1}+\dots+k_r\alpha_{i_r}\)
\end{enumerate}
\end{corollary}

如果一个\(m\times n\)矩阵其所有元素都是 0,则称为 \textbf{零矩阵} ,记作 0。下面设
\(A\neq0\)
\begin{enumerate}
\item 在矩阵\(A\)中,如果\(a_{11}=0\),我们就在矩阵中找一个不为零的元素,设为
\(a_{ij}\),现对换\(1,i\)两行,再对换\(1,j\)两列
\item 若\(a_{11}\neq0\),利用初等行变换把\(A\)变成如下形状
\begin{equation*}
A\to
\begin{bmatrix}
a_{11}&a_{12}&\dots&a_{1n}\\
0&b_{22}&\dots&b_{2n}\\
\vdots&\vdots&&\vdots\\
0&b_{m2}&\dots&b_{mn}
\end{bmatrix}
\end{equation*}
再利用初等列变换把\(A\)进一步变为
\begin{equation*}
A\to
\begin{bmatrix}
1&0&\dots&0\\
0&b_{22}&\dots&b_{2n}\\
\vdots&\vdots&&\vdots\\
0&b_{m2}&\dots&b_{mn}
\end{bmatrix}
\end{equation*}
如此继续对右下角\((m-1)\times(n-1)\)矩阵重复
\end{enumerate}

如果连续施行上述初等行、列变换之后,矩阵\(A\)可化成

 \begin{tabular}{p{4.8cm} p{3.4cm} p{3cm}}
\begin{equation*}
  \begin{bmatrix}
    \begin{matrix}
      1 &&\\&\ddots&\\&&1\\
    \end{matrix}&&\bigzero\\
    \bigzero&
    \begin{matrix}
      0&&\\&\ddots&\\&&0\\
    \end{matrix}&
    \begin{matrix}&\\&\\\dots&0
    \end{matrix}
  \end{bmatrix}
\end{equation*}&
\begin{equation*}
  \begin{bmatrix}
    \begin{matrix}
      1 &&\\&\ddots&\\&&1\\
    \end{matrix}&\bigzero\\
    \bigzero&
    \begin{matrix}
      0&&\\&\ddots&\\&&0\\
    \end{matrix}
  \end{bmatrix}
\end{equation*}&
\begin{equation*}
  \begin{bmatrix}
    \begin{matrix}
      1 &&\\&\ddots&\\&&1\\
    \end{matrix}&\bigzero\\&
    \begin{matrix}
      0&&\\
      &\ddots&\\
      &&0\\
    \end{matrix}\\\bigzero&
    \begin{matrix}
      0&\ddots &\vdots\\& &0
    \end{matrix}
  \end{bmatrix}
\end{equation*}\\
  $n>m$&$n=m$&$n<m$
 \end{tabular}

上述三种阶梯形矩阵称为 \(A\)的 \textbf{标准形} ,设标准型中\(1\)的个数为\(r\),则标准
形的行秩和列秩都为\(r\)

\begin{proposition}[]
矩阵的行秩等于列秩
\end{proposition}

\begin{definition}[]
一个矩阵\(A\)的行秩或列秩称为该矩阵的 \textbf{秩} ,记作\(r(A)\)
\end{definition}

\begin{proposition}[]
求\(K^4\)内下面向量组的极大线性无关组
\begin{alignat*}{2}
&\alpha_1=(2,0,1,1)&&\alpha_2=(-1,-1,-1,-1)\\
&\alpha_3=(1,-1,0,0)\quad&&\alpha_4=(0,-2,-1,-1)
\end{alignat*}
\end{proposition}

\begin{proof}
把向量组作为行排成一个矩阵
\begin{equation*}
A=
\begin{bmatrix}
2&0&1&1&\alpha_1\\
-1&-1&-1&-1&\alpha_2\\
1&-1&0&0&\alpha_3\\
0&-2&-1&-1&\alpha_4
\end{bmatrix}
\end{equation*}
做初等行变换(不能列变换了)
\begin{gather*}
A\to
\begin{bmatrix}
1&-1&0&0&\alpha_3\\
2&0&1&1&\alpha_1\\
-1&-1&-1&-1&\alpha_2\\
0&-2&-1&-1&\alpha_4\\
\end{bmatrix}\to
\begin{bmatrix}
1&-1&0&0&\alpha_3\\
0&2&1&1&\alpha_1-2\alpha_3\\
0&-2&-1&-1&\alpha_2+\alpha_3\\
0&-2&-1&-1&\alpha_4\\
\end{bmatrix}\\
\to
\begin{bmatrix}
1&-1&0&0&\alpha_3\\
0&2&1&1&\alpha_1-2\alpha_3\\
0&0&0&0&\alpha_1+\alpha_2-\alpha_3\\
0&0&0&0&\alpha_1-2\alpha_3+\alpha_4
\end{bmatrix}
\end{gather*}
秩为 2,另一方面,最后两个行向量为零
\begin{equation*}
\alpha_1+\alpha_2-\alpha_3=0,\quad
\alpha_1-2\alpha_3+\alpha_4=0
\end{equation*}
因此解出
\begin{equation*}
\alpha_3=\alpha_1+\alpha_2;\quad\alpha_4=\alpha_1+2\alpha_2
\end{equation*}

因此原向量组与\(\alpha_1,\alpha_2\)线性等价,因此\(\alpha_1,\alpha_2\)的秩为 2,
因此它线性等价,因此是一个极大线性无关部分组
\end{proof}

\begin{proposition}[]
求\(K^4\)内下列向量组的一个极大线性无关部分组
\begin{alignat*}{2}
&\alpha_1=(1,1,4,2)&&\alpha_2=(1,-1,-2,4)\\
&\alpha_3=(0,2,6,-2)&&\alpha_4=(-3,-1,3,4)\\
&\alpha_5=(-1,0,-4,-7)\quad&&\alpha_6=(-2,1,7,1)
\end{alignat*}
\end{proposition}

\begin{proof}
把它们作为列向量排成\(4\times6\)矩阵\(A\),再对\(A\)作初等行(不能作列变换)变换化为阶梯形矩
阵
\begin{equation*}
A=
\begin{bmatrix}
1&1&0&-3&-1&-2\\
1&-1&2&-1&0&1\\
4&-2&6&3&-4&7\\
2&4&-2&4&7&1
\end{bmatrix}\to
\begin{bmatrix}
1&1&0&-3&-1&-2\\
0&2&-2&-2&-1&-3\\
0&0&0&3&-1&2\\
0&0&0&0&0&0
\end{bmatrix}
=B
\end{equation*}
\(B\)的列向量组的一个极大线性无关部分组是第\(1,2,4\)向量,故\(A\)的列向量组
\(\alpha_1,\dots,\alpha_6\)的一个极大线性无关部分组是\(\alpha_1,\alpha_2,\alpha_4\)
\end{proof}
\subsection{线性方程组的理论课题}
\label{sec:orgd9ed85b}
\subsubsection{齐次线性方程组的基础解系}
\label{sec:org320be7f}
考查数域\(K\)上的齐次线性方程组
\begin{equation*}
\begin{cases}
a_{11}x_1+\dots+a_{1n}x_n=0\\
\dots\\
a_{m1}x_1+\dots+a_{mn}x_n=0
\end{cases}
\end{equation*}
等价于\(K^m\)内的向量方程
\begin{equation*}
x_1\alpha_1+\dots+x_n\alpha_n=0
\end{equation*}
其中\(\alpha_1,\dots,\alpha_n\)为系数矩阵\(A\)的列向量

齐次线性方程组的解具有如下性质
\begin{enumerate}
\item 如果\(\eta_1=(k_1,\dots,k_n),\eta_2=(l_1,\dots,l_2)\)是方程组的两个解向量,
则
\begin{equation*}
\eta_1+eta_2=(k_1+l_1,\dots,k_n+l_n)
\end{equation*}
也是方程组的解向量
\item 如果\(\eta=(k_1,\dots,k_n)\)是方程组的一个解向量,则对\(K\)内任意数\(k\),
有
\begin{equation*}
k\eta=(kk_1,\dots,kk_n)
\end{equation*}
也是方程组的解向量
\end{enumerate}


\begin{definition}[]
齐次线性方程组的一组解向量\(\eta_1,\dots,\eta_s\)如果满足
\begin{enumerate}
\item \(\eta_1,\dots,\eta_s\)线性无关
\item 方程组的任意解能被\(\eta_1,\dots,\eta_s\)线性表示
\end{enumerate}


则称\(\eta_1,\dots,\eta_s\)是齐次线性方程组的一个 \textbf{基础解系}
\end{definition}

\begin{proposition}[]
如果向量组\(\alpha_1,\dots,\alpha_s\)线性无关,而向量 \(\beta\) 可被它线性表示,则表
示法唯一
\end{proposition}

\begin{theorem}[]
数域\(K\)上的齐次线性方程组的基础解系存在,且任意基础解系中解向量个数为
\(n-r\),其中\(n\)为未知量个数,而\(r\)为系数矩阵\(A\)的秩\(r(A)\)
\end{theorem}

\begin{proof}
因为方程组的任意两个基础解系是互相线性等价的,因而秩相等,它们又是线性无关的,
故两个基础解系中包含相同数量的向量

设系数矩阵\(A\)的列向量组为\(\alpha_1,\dots,\alpha_n\),如果\(r(A)=r=n\),即
\(A\)的列向量组线性无关,则方程组只有零解,其基础解系包含 0 个向量,定理成立

下面设\(r<n\),不妨设\(\alpha_1,\dots,\alpha_r\)为\(A\)的列向量组的一个极大
线性无关部分组。因为\(\alpha_{r+1},\dots,\alpha_n\)能被
\(\alpha_1,\dots,\alpha_r\)线性表示,它们的任一线性组合也能被线性表示且唯一,
因此任给\(x_{r+1},\dots,x_n\in K\)
\begin{equation*}
x_{r+1}=k_{r+1},\dots,x_n=k_n
\end{equation*}
因\(\beta=-(k_{r+1}\alpha_{r+1}+\dots+k_n\alpha_n)\)能被唯一地线性表示,所以
存在一组\(k_1,\dots,k_r\in K\)使
\begin{equation*}
k_1\alpha_1+\dots+k_r\alpha_r+k_{r+1}\alpha_{r+1}+\dots+k_n\alpha_n=0
\end{equation*}
这说明
\begin{enumerate}
\item 未知量\(x_{r+1},\dots,x_n\)任取一组数值,都有唯一确定的一组解\(x_1,\dots,x_r\)
\item 方程组的两组解\(\eta_1,\eta_2\)如它们在\(x_{r+1},\dots,x_n\)处取相同值,
则\(\eta_1=\eta_2\)
\end{enumerate}


未知量\(x_{r+1},\dots,x_n\)称为方程组的 \textbf{自由未知量} ,如果让这\(n-r\)个自由变
量中某一个取值 1,其余取值零,就得到方程组的一组解向量
\begin{align*}
&\eta_1=(*,\dots,*,1,0,\dots,0)\\
&\eta_2=(*,\dots,*,0,1,\dots,0)\\
&\dots\\
&\eta_{n-r}=(*,\dots,*,0,0,\dots,1)\\
\end{align*}

我们来证明\(\eta_1,\dots,\eta_{n-r}\)是方程组的一个基础解系
\begin{enumerate}
\item 线性无关
\item 设\(\eta=(k_1,\dots,k_n)\)是方程组的任一组解,令
\begin{equation*}
\eta'=k_{r+1}\eta_1+\dots+k_n\eta_{n-r}
\end{equation*}
\(\eta'\)也是方程组的一组解,它与 \(\eta\) 在\(x_{r+1},\dots,x_n\)处取相同值,因
而\(\eta=\eta'\)
\end{enumerate}
\end{proof}

\begin{corollary}[]
如果齐次线性方程组系数矩阵\(A\)的秩\(r\)等于未知量个数\(n\),则它只有零解;
如果\(r<n\),它必有非零解
\end{corollary}
\subsubsection{基础解系的求法}
\label{sec:org299871b}
\begin{proposition}[]
求数域\(K\)内齐次线性方程组
\begin{equation*}
\left\{
\begin{array}{ccccccccccc}
x_1&+&x_2&&&-&3x_4&-&x_5&=&0\\
x_1&-&x_2&+&2x_3&-&x_4&&&=&0\\
4x_1&-&2x_2&+&6x_3&+&3x_4&-&4x_5&=&0\\
2x_1&+&4x_2&-&2x_3&+&4x_4&-&7x_5&=&0
\end{array}
\right.
\end{equation*}
的一个基础解系
\end{proposition}

\begin{proof}
化为阶梯形
\begin{equation*}
\begin{bmatrix}
1&1&0&-3&-1\\
1&-1&2&-1&0\\
4&-2&6&3&-4\\
2&4&-2&4&-7
\end{bmatrix}\to
\begin{bmatrix}
1&1&0&-3&-1\\
0&2&-2&-2&-1\\
0&0&0&3&-1\\
0&0&0&0&0\\
\end{bmatrix}
\end{equation*}
现在\(r(A)=3\),基础解系中应有\(n-r=2\)个向量,写出阶梯形矩阵对应的方程
\begin{equation*}
\left\{
\begin{array}{ccccccccccc}
x_1&+&x_2&&&-&3x_4&-&x_5&=&0\\
&&2x_2&-&2x_3&-&2x_4&-&x_5&=&0\\
&&&&&&3x_4&-&x_5&=&0
\end{array}
\right.
\end{equation*}
移项得
\begin{equation*}
\left\{\begin{array}{ccccccccc}
x_1&+&x_2&-&3x_4&=&&&x_5\\
&&2x_2&-&2x_4&=&2x_3&+&x_5\\
&&&&3x_4&=&&&x_5
\end{array}\right.
\end{equation*}
现在\(x_3,x_5\)是自由未知量
\begin{enumerate}
\item 取\(x_3=1,x_5=0\),得一个解向量
\begin{equation*}
\eta_1=(-1,1,1,0,0)
\end{equation*}
\item 取\(x_3=0,x_5=1\),得到另一个解向量
\begin{equation*}
\eta_2=(\frac{7}{6},\frac{5}{6},0,\frac{1}{3},1)
\end{equation*}
\end{enumerate}


\(\eta_1,\eta_2\)即为方程组的一个基础解系
\end{proof}
\subsubsection{线性方程组的一般理论}
\label{sec:org81788c4}
讨论数域\(K\)上的一般线性方程组
\begin{equation*}
\begin{cases}
a_{11}x_1+\dots+a_{1n}x_n=b_1\\
\dots\\
a_{m1}x_1+\dots+a_{mn}x_n=b_m\\
\end{cases}
\end{equation*}
其系数矩阵和增广矩阵分别是
\begin{equation*}
A=
\begin{bmatrix}
a_{11}&\dots&a_{1n}\\
\vdots&\vdots&\vdots\\
a_{m1}&\dots&a_{mn}
\end{bmatrix},\quad
\bbar{A}=
\begin{bmatrix}
a_{11}&\dots&a_{1n}&b_1\\
\vdots&\vdots&\vdots&\vdots\\
a_{m1}&\dots&a_{mn}&b_{m}\\
\end{bmatrix}
\end{equation*}

\begin{proposition}[]
给定\(K^m\)中一个线性无关向量组\(\alpha_1,\dots,\alpha_n\),若添加向量 \(\beta\) 后,
向量组
\begin{equation*}
\alpha_1,\dots,\alpha_n,\beta
\end{equation*}
线性相关,则 \(\beta\) 可被 \(\alpha_1,\dots,\alpha_n\)线性表示
\end{proposition}

\begin{theorem}[判别定理]
数域\(K\)上线性方程组有解的充分必要条件是其系数矩阵与增广矩阵\(\bbar{A}\)的
秩相等,即\(r(A)=r(\bbar{A})\)
\end{theorem}

\begin{proof}
\(\beta\) 必须被\(\alpha_1,\dots,\alpha_n\)线性表示
\end{proof}

设给定两个解向量
\begin{equation*}
\gamma_1=(k_1,\dots,k_n),\quad
\gamma_2=(l_1,\dots,l_n)
\end{equation*}
\(\gamma_1-\gamma_2\)是齐次线性方程组的一个解向量,把方程组的常数项换成 0,得
到一个与之对应的齐次线性方程组,称为 \textbf{导出方程组}

设\(\gamma_0=(a_1,\dots,a_n)\)是线性方程组的解向量,而
\(\eta=(k_1,\dots,k_n)\)是导出方程组的解向量,那么\(\gamma_0+\eta\)是方程组
的解向量

因此,如果给定方程组的某个解向量\(\gamma_0\),那么对于任意解向量\(\gamma\),
\(\gamma-\gamma_0=\eta\)是导出方程的解向量,故\(\gamma\)可表示为
\begin{equation*}
\gamma=\gamma_0+\eta
\end{equation*}

\begin{theorem}[]
在\(r(A)=r(\bbar{A})\)的条件下,有
\begin{enumerate}
\item 如果\(r(A)=n\),则方程组有唯一解
\item 如果\(r(A)<n\),则方程组有无穷多组解,可由某一特殊解与导出方程组的基础解
系表示
\end{enumerate}
\end{theorem}
\subsection{矩阵的运算}
\label{sec:org2e24f82}
数域\(K\)上全体\(m\times n\)矩阵组成的集合记为
\begin{equation*}
M_{m,n}(K)=\left\{
\begin{bmatrix}
a_{11}&\dots&a_{1n}\\
\vdots&\vdots&\vdots\\
a_{m1}&\dots&a_{mn}
\end{bmatrix}\quad\rvline\quad a_{ij}\in K
\right\}
\end{equation*}

考查如下两个矩阵
\begin{equation*}
A=
\begin{bmatrix}
a_{11}&\dots&a_{1n}\\
\vdots&\vdots&\vdots\\
a_{m1}&\dots&a_{mn}
\end{bmatrix},\quad X=
\begin{bmatrix}
x_1\\
\vdots\\
x_n
\end{bmatrix}
\end{equation*}
我们规定\(A\)与\(X\)的乘法如下
\begin{align*}
AX&=
\begin{bmatrix}
a_{11}&\dots&a_{1n}\\
\vdots&\vdots&\vdots\\
a_{m1}&\dots&a_{mn}
\end{bmatrix}
\begin{bmatrix}
x_1\\\vdots\\x_n
\end{bmatrix}\\
&=
\begin{bmatrix}
a_{11}x_1+\dots+a_{1n}x_n\\
\vdots\\
a_{m1}x_1+\dots+a_{mn}x_n
\end{bmatrix}
\end{align*}
若再引入一个\(m\times1\)矩阵
\begin{equation*}
B=
\begin{bmatrix}
b_1\\\vdots\\b_m
\end{bmatrix}
\end{equation*}
则
\begin{equation*}
AX=B
\end{equation*}

\begin{definition}[]
给定数域\(K\)上的\(m\times n\)矩阵\(A\)和\(n\times s\)矩阵\(B\)
\begin{equation*}
A=
\begin{bmatrix}
a_{11}&\dots&a_{1n}\\
\vdots&\vdots&\vdots\\
a_{m1}&\dots&a_{mn}\\
\end{bmatrix},B=
\begin{bmatrix}
b_{11}&\dots&b_{1n}\\
\vdots&\vdots&\vdots\\
b_{m1}&\dots&b_{mn}\\
\end{bmatrix}
\end{equation*}

定义乘法如下
\begin{align*}
AB&=
\begin{bmatrix}
a_{11}&\dots&a_{1n}\\
\vdots&\vdots&\vdots\\
a_{m1}&\dots&a_{mn}\\
\end{bmatrix}
\begin{bmatrix}
b_{11}&\dots&b_{1n}\\
\vdots&\vdots&\vdots\\
b_{m1}&\dots&b_{mn}\\
\end{bmatrix}\\
&=
\begin{bmatrix}
\sum_{k=1}^na_{1k}b_{k1}&\dots&
\sum_{k=1}^na_{1k}b_{ks}\\
\vdots&\vdots&\vdots\\
\sum_{k=1}^na_{mk}b_{k1}&\dots&
\sum_{k=1}^na_{mk}b_{ks}
\end{bmatrix}
\end{align*}
\end{definition}

\begin{proposition}[]
设\(A\)是数域\(K\)上的\(m\times n\)矩阵,其列向量组记为
\(\alpha_1,\dots,\alpha_n\),又设\(B\)是数域\(K\)上的\(n\times s\)矩阵。令
\(C=AB\),则\(C\)的第\(j\)个列向量是以\(B\)的第\(j\)列元素为系数作\(A\)的列向
量组\(\alpha_1,\dots,\alpha_n\)的线性组和所得到的\(m\)维向量组
\begin{equation*}
A
\begin{bmatrix}
b_{1j}\\
\vdots\\
b_{nj}
\end{bmatrix}=b_{1j}\alpha_1+\dots+b_{nj}\alpha_n
\end{equation*}
\end{proposition}

给定数域\(K\)上的\(m\times n\)矩阵
\begin{equation*}
A=
\begin{bmatrix}
a_{11}&\dots&a_{1n}\\
\vdots&\ddots&\vdots\\
a_{m1}&\dots&a_{mn}\\
\end{bmatrix}
\end{equation*}
考查向量空间\(K^n\)到\(K^m\)的映射\(f_A\):对任意
\begin{equation*}
X=
\begin{bmatrix}
x_1\\\vdots\\x_n
\end{bmatrix}\in K^n
\end{equation*}
定义\(f_A(X)=AX\in K^m\)

取定\(K^n\)中坐标向量
\begin{equation*}
X_j=
\begin{bmatrix}
0\\\vdots\\0\\1\\0\\\vdots\\0
\end{bmatrix}\dots j
\end{equation*}
我们有
\begin{equation*}
f_A(X_j)=
\begin{bmatrix}
a_{11}&\dots&a_{1n}\\
\vdots&\ddots&\vdots\\
a_{m1}&\dots&a_{mn}\\
\end{bmatrix}
\begin{bmatrix}
0\\\vdots\\0\\1\\0\\\vdots\\0
\end{bmatrix}=
\begin{bmatrix}
a_{1j}\\\vdots\\a_{nj}
\end{bmatrix}
\end{equation*}
即\(f_A(X_j)\)是\(A\)的第\(j\)个列向量

如果另有\(K\)上\(m\times n\)矩阵\(A_1\)使\(f_{A_1}=f_A\),那么\(A_1\)的第
\(j\)个列向量\(=f_{A_1}X_j=f_A(X_j)=A\)的第\(j\)个列向量,因此\(f_A\)反过来决
定了\(A\)

现在给定\(K\)上\(n\times s\)矩阵
\begin{equation*}
B=
\begin{bmatrix}
b_{11}&\dots&b_{1n}\\
\vdots&\ddots&\vdots\\
b_{m1}&\dots&b_{mn}\\
\end{bmatrix}
\end{equation*}
考查向量空间的如下映射图
\begin{equation*}
K^s\xrightarrow{f_B}K^n\xrightarrow{f_A}K^m
\end{equation*}
设\(AB=C=(c_{ij})\),对任意\(X\in K^s\),我们有
\begin{align*}
(f_Af_B)(X)&=f_A(f_B(X))=f_A
\begin{bmatrix}
\sum_{l=1}^sb_{1l}x_l\\
\vdots\\
\sum_{l=1}^sb_{nl}x_l
\end{bmatrix}\\
&=
\begin{bmatrix}
\sum_{k=1}^na_{1k}\sum_{l=1}^sb_{kl}x_l\\
\vdots\\
\sum_{k=1}^na_{mk}\sum_{l=1}^sb_{kl}x_l\\
\end{bmatrix}=
\begin{bmatrix}
\sum_{l=1}^s\left(\sum_{k=1}^na_{1k}b_{kl} \right)x_l\\
\vdots\\
\sum_{l=1}^s\left(\sum_{k=1}^na_{mk}b_{kl} \right)x_l
\end{bmatrix}\\
&=
\begin{bmatrix}
\sum_{l=1}^sc_{1l}x_l\\\vdots\\
\sum_{l=1}^sc_{ml}x_l
\end{bmatrix}=f_C(X)
\end{align*}

\begin{proposition}[]
数域\(K\)上的矩阵运算满足
\begin{enumerate}
\item 乘法满足结合律 \(A(BC)=(AB)C\)
\item 分配率 \((A+B)C=AC+BC\)\\
\(A(B+C)=AB+AC\)
\item 对\(K\)内任一数\(k\),有\(k(AB)=(kA)B=A(kB)\)
\item \((A+B)'=A'+B'\), \((kA)'=kA',(AB)'=B'A'\)
\end{enumerate}
\end{proposition}

\begin{proposition}[]
在\(K^m\)中给定两个向量组
\begin{align}
&\alpha_1,\alpha_2,\dots,\alpha_n\label{eq2.4.3.1}\\
&\beta_1,\beta_2,\dots,\beta_n\label{eq2.4.3.2}
\end{align}
如果 \eqref{eq2.4.3.1} 可被 \eqref{eq2.4.3.2} 线性表示,则 \eqref{eq2.4.3.1} 的秩
\(\le\)
\eqref{eq2.4.3.2} 的秩
\end{proposition}

\begin{proposition}[]
给定\(A,B\in M_{m,n}(K)\),则有
\begin{enumerate}
\item 对任意\(k\in K,k\neq0,r(kA)=r(A)\)
\item \(r(A+B)\le r(A)+r(B)\)
\end{enumerate}
\end{proposition}

\begin{proof}
\begin{enumerate}
\setcounter{enumi}{1}
\item \(\alpha_1+\beta_1,\dots,\alpha_n+\beta_n\)显然能被
\begin{equation*}
\alpha_{i_1},\dots,\alpha_{i_r},\beta_{j_1},\dots,\beta_{j_s}
\end{equation*}
线性表示
\end{enumerate}
\end{proof}

\begin{proposition}[]
设\(A\in M_{m,n}(K),B\in M_{n,s}(K)\),则
\begin{equation*}
r(AB)\le\min\{\r(A),r(B)\}
\end{equation*}
\end{proposition}

\begin{proof}
设\(C=AB\),\(C\)的列向量能被\(A\)的列向量线性表示,因而\(r(C)\le r(A)\)

另一方面,\(r(C)=r(C')=r((AB)')=r(B'A')\le r(B')=r(B)\)
\end{proof}

\begin{proposition}[]
设\(A\in M_{m,n}(K),B\in M_{n,s}(K)\),则
\begin{equation*}
r(AB)\ge r(A)+r(B)-n
\end{equation*}
\end{proposition}

\begin{proof}
设\(C=AB\),\(B\)的列向量组为\(B_1,\dots,B_s\),\(C\)的列向量组为
\(C_1,\dots,C_s\),那么
\begin{equation*}
AB_i=C_i
\end{equation*}
设 \(C_{i_1},\dots,C_{i_r}\)为\(C\)的列向量组的一个极大线性无关部分组,于是
\(r=r(C)=r(AB)\),对任意\(C_i\),有
\begin{equation*}
C_i=k_1C_{i_1}+\dots+k_rC_{i_r}
\end{equation*}
于是
\begin{align*}
A(k_1&B_{i_1}+\dots+k_rB_{i_r})\\
&=k_1AB_{i_1}+\dots+k_rAB_{i_r}\\
&=k_1C_{i_1}+\dots+k_rC_{i_r}=C_i
\end{align*}
现在线性方程组\(AX=C_i\)有两组解
\begin{equation*}
X_1=B_i,X_2=k_1B_{i_1}+\dots+k_rB_{i_r}
\end{equation*}
如设其导出方程组\(AX=0\)的一个基础解系为\(P_1,\dots,P_t\),则\(t=n-r(A)\)
\begin{equation*}
B_i=k_1B_{i_1}+\dots+k_rB_{i_r}+l_1P_1+\dots+l_tP_t
\end{equation*}
于是\(B\)的列向量组\(B_1,\dots,B_s\)可由向量组
\begin{equation*}
S=B_{i_1},\dots,B_{i_r},P_1,\dots,P_t
\end{equation*}
线性表示,\(r(B)\le S\)的秩\(\le r+t=r(C)+n-r(A)\),因此
\begin{equation*}
r(AB)=r(C)\ge r(A)+r(B)-n
\end{equation*}
\end{proof}
\subsection{\(n\) 阶方阵}
\label{sec:orgee80af7}
\begin{definition}[]
数域\(K\)上的\(n\times n\)矩阵称为\(K\)上的 \textbf{\(n\)阶方阵} ,\(K\)上全体\(n\)阶
方阵所成的集合记作\(M_n(K)\)
\end{definition}

\begin{equation*}
Tr(A):=a_{11}+a_{22}+\dots+a_{nn}
\end{equation*}
称为\(A\)的 \textbf{迹}

   \begin{equation*}
   E_{ij}=
   \begin{blockarray}{cccccc}
  &&&&j&\\
  \begin{block}{c[ccccc]}
    &&&&\vdots&\\
    i&\dots&\dots&\dots&1&\dots\\
    &&&&\vdots&\\
    &&&&\vdots&\\
    &&&&\vdots&\\
  \end{block}
\end{blockarray}
   \end{equation*}

对\(m\)阶方阵\(E_{ij}\),我们有
\begin{equation*}
E_{ij}
\begin{bmatrix}
a_{11}&\dots&a_{1n}\\
\vdots&\ddots&\vdots\\
a_{m1}&\dots&a_{mn}\\
\end{bmatrix}=
\begin{bmatrix}
&\bigzero&\\
a_{j1}&\dots&a_{jm}\\
&\bigzero&
\end{bmatrix}i\text{th line}
\end{equation*}
也就是说,用\(m\)阶方阵\(E_{ij}\)左乘一个\(m\times n\)矩阵,其结果是把该矩阵
的第\(j\)行平移到第\(i\)行的位置,其余一律变为零。而对\(n\)阶方阵\(E_{ij}\),
我们有
\begin{equation*}
\begin{bmatrix}
a_{11}&\dots&a_{1n}\\
\vdots&\ddots&\vdots\\
a_{m1}&\dots&a_{mn}\\
\end{bmatrix}E_{ij}=
\begin{blockarray}{ccc}
&j\text{th}&\\
\begin{block}{[ccc]}
&a_{1i}&\\
\bigzero&\vdots&\bigzero\\
&a_{mi}&\\
\end{block}
\end{blockarray}
\end{equation*}

用\(n\)阶方阵\(E_{ij}\)右乘一个\(m\times n\)矩阵,其结果是把该矩阵第\(i\)列平
移到第\(j\)列,其余列变零。特别地
\begin{equation*}
E_{ij}E_{kl}=
\begin{cases}
E_{il}&j=k\\
0&j\neq k
\end{cases}
\end{equation*}

数域\(K\)上的如下\(n\)阶方阵
\begin{equation*}
D=
\begin{bmatrix}
d_1&&\bigzero\\
&\ddots&\\
\bigzero&&d_n
\end{bmatrix}=
\sum_{i=1}^nd_iE_{ii}(d_i\in K)
\end{equation*}
称为 \textbf{\(n\)阶对角矩阵}
\begin{align*}
(\sum_{i=1}^md_iE_{ii})&
\begin{bmatrix}
a_{11}&\dots&a_{1n}\\
\vdots&\ddots&\vdots\\
a_{m1}&\dots&a_{mn}\\
\end{bmatrix}=
\sum_{i=1}^m
\begin{bmatrix}
&\bigzero&\\
a_{i1}&\dots&a_{in}\\
&\bigzero&\\
\end{bmatrix}\\
&=\sum_{i=1}^m
\begin{bmatrix}
&\bigzero&\\
d_ia_{i1}&\dots&d_ia_{in}\\
&\bigzero&\\
\end{bmatrix}\\
&=
\begin{bmatrix}
d_1a_{11}&\dots&d_1a_{1n}\\
\vdots&\ddots&\vdots\\
d_ma_{m1}&\dots&d_{m}a_{mn}
\end{bmatrix}
\end{align*}
即把该对角矩阵对角线上的元素分别乘到右边矩阵的各个行上去。同样我们有
\begin{gather*}
\begin{bmatrix}
a_{11}&\dots&a_{1n}\\
\vdots&\ddots&\vdots\\
a_{m1}&\dots&a_{mn}\\
\end{bmatrix}
\begin{bmatrix}
d_1&&\bigzero\\
&\ddots&\\
\bigzero&&d_n
\end{bmatrix}\\
=
\begin{bmatrix}
d_1a_{11}&\dots&d_na_{1n}\\
\vdots&\ddots&\vdots\\
d_1a_{m1}&\dots&d_na_{mn}
\end{bmatrix}
\end{gather*}

一个\(n\)阶对角矩阵的主对角线上元素都是\(K\)上同一个数\(k\)时,称为\(n\)阶数
量矩阵。特别的
\begin{equation*}
E=
\begin{bmatrix}
1&&\bigzero\\
&\ddots&\\
\bigzero&&1
\end{bmatrix}
\end{equation*}
称为 \textbf{\(n\)阶单位矩阵}

\begin{definition}[]
\(n\)阶单位矩阵经过一次初等行变换或初等列变换所得的矩阵称为 \(n\)阶
\textbf{初等矩阵}
\begin{enumerate}
\item 互换\(E\)的\(i,j\)两行
\begin{equation*}
P_n(i,j)=
\begin{blockarray}{cccccccccccc}
\begin{block}{[ccccccccccc]c}
1&&&\vdots&&&&\vdots&&&&\\
&\ddots&&\vdots&&&&\vdots&&&&\\
&&1&\vdots&&&&\vdots&&&&\\
\cdots&\cdots&\cdots&0&\cdots&\cdots&\cdots&1
&\cdots&\cdots&\cdots&i\text{th}\\
&&&\vdots&1&&&\vdots&&&&\\
&&&\vdots&&\ddots&&\vdots&&&&\\
&&&\vdots&&&1&\vdots&&&&\\
\cdots&\cdots&\cdots&1&\cdots&\cdots&\cdots&0
&\cdots&\cdots&\cdots&j\text{th}\\
&&&\vdots&&&&\vdots&1&&&\\
&&&\vdots&&&&\vdots&&\ddots&&\\
&&&\vdots&&&&\vdots&&&1&\\
\end{block}
&&&i\text{th}&&&&j\text{th}&&&&\\
\end{blockarray}
\end{equation*}
\item 把\(E\)的第\(i\)行乘以\(c\neq0,c\in K\)
\begin{equation*}
P_n(c\cdot i)=
\begin{blockarray}{cccccccc}
\begin{block}{[ccccccc]c}
1&&&\vdots\\
&\ddots&&\vdots\\
&&1&\vdots\\
\cdots&\cdots&\cdots&c&\cdots&\cdots&\cdots&i\text{th}\\
&&&\vdots&1\\
&&&\vdots&&\ddots\\
&&&\vdots&&&1\\
\end{block}
 &&&i\text{th}&&&&\\
\end{blockarray}
\end{equation*}
\item 把\(E\)的第\(j\)行加上第\(i\)行的\(k\in K\)倍,得
\begin{equation*}
P_n(k\cdot i,j)=
\begin{blockarray}{cccccccc}
\begin{block}{[ccccccc]c}
1&&\vdots&&\vdots\\
&\ddots&\vdots&&\vdots\\
\cdots&\cdots&1&\cdots&\cdot&\cdots&\cdots&i\text{th}\\
&&\vdots&\ddots&\vdots\\
\cdots&\cdots&k&\cdots&1&\cdots&\cdots&j\text{th}\\
&&\vdots&&\vdots&\ddots\\
&&\vdots&&\vdots&&1\\
\end{block}
&&i\text{th}&&j\text{th}
\end{blockarray}
\end{equation*}
\item 把第\(j\)列加上第\(i\)列的\(k\)倍,得
\begin{equation*}
P_n'(k\cdot i,j)=
\begin{blockarray}{cccccccc}
\begin{block}{[ccccccc]c}
1&&\vdots&&\vdots\\
&\ddots&\vdots&&\vdots\\
\cdots&\cdots&1&\cdots&k&\cdots&\cdots&i\text{th}\\
&&\vdots&\ddots&\vdots\\
\cdots&\cdots&\cdot&\cdots&1&\cdots&\cdots&j\text{th}\\
&&\vdots&&\vdots&\ddots\\
&&\vdots&&\vdots&&1\\
\end{block}
&&i\text{th}&&j\text{th}
\end{blockarray}
\end{equation*}
但是\(P_n'(k\cdot i,j)\)可看作\(P_n(k\cdot j,i)\)
\end{enumerate}
\end{definition}

\begin{proposition}[]
给定数域\(K\)上\(m\times n\)矩阵\(A\),则有
\begin{enumerate}
\item \(P_m(i,j)A\)为互换\(A\)的\(i,j\)两行;\(AP_n(i,j)\)为互换\(A\)的\(i,j\)两
列
\item \(P_m(c\cdot i)A\)为把\(A\)的第\(i\)行乘以\(c\neq0\);\(AP_n(c\cdot i)\)为
把\(A\)的第\(i\)列乘以\(c\neq0\)
\item \(P_m(k\cdot i,j)A\)为把\(A\)的第\(j\)行加上第\(i\)行的\(k\)倍;
\(AP'_n(k\cdot i,j)\)为把\(A\)的第\(j\)列加上\(i\)列的\(k\)倍
\end{enumerate}
\end{proposition}

\begin{proof}
\begin{enumerate}
\item 我们有
\begin{gather*}
P_m(i,j)=E-E_{ii}-E_{jj}+E_{ij}+E_{ji}\\
P_m(i,j)A=(A-E_{ii}A-E_{jj}A)+E_{ij}A+E_{ji}A
\end{gather*}
\setcounter{enumi}{2}
\item \begin{gather*}
P_m(k\cdot i,j)=E+kE_{ji}\\
P_m(k\cdot i,j)A=A+kE_{ji}A
\end{gather*}
\end{enumerate}
\end{proof}

一个数域\(K\)上的\(n\)阶方阵\(A\),如果它的秩\(r(A)=n\),则称为一个 \textbf{满秩} 的
\(n\)阶方阵,满秩的\(n\)阶方阵在初等变换下的标准形\(D\)应为\(n\)阶单位矩阵
\(E\),故
\begin{align*}
A&=P_1\dots P_sEQ_1\dots Q_t\\
&=P_1\dots P_sQ_1\dots Q_t
\end{align*}

\begin{proposition}[]
数域\(K\)上的\(n\)阶方阵\(A\)满秩的充分必要条件是\(A\)可以表示为有限个初等矩
阵的乘积
\end{proposition}

\begin{corollary}[]
设\(A\)是数域\(K\)上满秩的\(n\)阶方阵,则\(A\)可单用初等行变换化为单位矩阵
\(E\),也可单用初等列变化化为
\end{corollary}

\begin{definition}[]
给定数域\(K\)上两个\(m\times n\)矩阵\(A,B\),若\(A\)经有限次初等行、列变化化
为\(B\),则称\(B\)与\(A\)相抵
\end{definition}

\begin{corollary}[]
给定数域\(K\)上两个\(m\times m\)矩阵\(A,B\),则下面命题等价
\begin{enumerate}
\item \(B\)与\(A\)相抵
\item \(r(A)=r(B)\)
\item 存在\(m\)阶满秩方阵\(P\)及\(n\)阶满秩方阵\(Q\),使\(B=PAQ\)
\end{enumerate}
\end{corollary}

\begin{definition}[]
设\(A\)是数域\(K\)上的一个\(n\)阶方阵,如果存在数域\(K\)上的\(n\)阶方阵\(B\),
使
\begin{equation*}
BA=AB=E
\end{equation*}
则称\(B\)是\(A\)的一个 \textbf{逆矩阵} ,此时\(A\)称为 \textbf{可逆矩阵}
\end{definition}

\begin{proposition}[]
设\(A\)是数域\(K\)上的\(n\)阶方阵,如果存在\(K\)上\(n\)阶方阵\(B,B_1\),使
\begin{equation*}
AB=B_1A=E
\end{equation*}
则\(B_1=B\)
\end{proposition}

\begin{proof}
\begin{equation*}
B_1=B_1E=B_1(AB)=(B_1A)B=EB=B
\end{equation*}
\end{proof}

\begin{corollary}[]
设\(A\in M_n(K)\),如果\(A\)可逆,则其逆矩阵唯一
\end{corollary}

\begin{proposition}[]
设\(A\)是数域\(K\)上的\(n\)阶方阵,则\(A\)可逆的充分必要条件是\(A\)是满秩的
\end{proposition}

\begin{proof}
若\(A\)可逆,有\(B\in M_n(K)\)使\(AB=E\),故
\begin{equation*}
n=r(E)=r(AB)\le r(A)\le n
\end{equation*}
于是\(r(A)=n\)

若\(A\)满秩,则存在初等矩阵\(P_1,\dots,P_s;Q_1,\dots,Q_t\)使
\begin{equation*}
P_1\dots P_sA=E,\quad AQ_1\dots Q_t=E
\end{equation*}
令\(P=P_1\cdots P_s,Q=Q_1\cdots Q_t\),则\(PA=AQ=E\)
\end{proof}

\begin{corollary}[]
设\(A\in M_n(K)\),如果存在\(B\in M_n(K)\),使\(AB=E\)或\(BA=E\)之一成立,则
\(A\)可逆
\end{corollary}

\begin{proposition}[]
\begin{enumerate}
\item \(P_n(i,j)^{-1}=P_n(i,j)\)
\item \(P_n(c\cdot i)^{-1}=P_n(\frac{1}{c}\cdot i)\)
\item \(P_n(k\cdot i,j)^{-1}=P_n(-k\cdot i,j)\)
\end{enumerate}
\end{proposition}

\begin{proposition}[]
设\(A,B\)是数域\(K\)上的可逆\(n\)阶方阵,则
\begin{enumerate}
\item \((A^{-1})^{-1}=A\)
\item \(AB\)可逆,且\((AB)^{-1}=B^{-1}A^{-1}\)
\item \(A'\)可逆,且\((A')^{-1}=(A^{-1})'\)
\end{enumerate}
\end{proposition}

逆矩阵计算如下
\begin{enumerate}
\item 把\(A\)和\(E\)并排放在一起,排成一个\(n\times 2n\)矩阵
\begin{equation*}
\begin{bmatrix}
A&\vdots&E
 \end{bmatrix}
\end{equation*}
\item 作初等行变换把左边的\(A\)化为\(E\),此时右边是\(A^{-1}\)
\end{enumerate}


\begin{proposition}[]
给定数域\(K\)上\(m\times n\)矩阵\(A\)和\(n\times s\)矩阵\(B\),证明
\begin{equation*}
r(AB)\ge r(A)+r(B)-n
\end{equation*}
\end{proposition}

\begin{proof}
若\(A\neq0\),则存在\(m\)阶初等矩阵\(P_1,\dots,P_s\)及\(n\)阶初等矩阵
\(Q_1,\dots,Q_t\)使\(P_1\dots P_sAQ_1\dots Q_t=D\)为\(A\)在初等变换下的标准形,
令\(P=P_1\dots P_s,Q=Q_1\dots Q_t\),则\(PAQ=D\),因为\(P,Q\)满秩
\begin{equation*}
r(AB)=r(PAB)=r((PAQ)q^{-1}B)=r(DB_1)
\end{equation*}
这里\(B_1=Q^{-1}B\),现在\(r(D)=r(A),r(B_1)=r(B)\)
\begin{align*}
DB_1&=
\begin{bmatrix}
1&&&&&\\
&\ddots&&&&\\
&&1&&&\\
&&&0&&\\
&&&&\ddots&\\
&&&&&0
\end{bmatrix}
\begin{bmatrix}
b_{11}&\dots&b_{1n}\\
\vdots&\ddots&\vdots\\
b_{m1}&\dots&b_{mn}\\
\end{bmatrix}\\
&=
\begin{bmatrix}
b_{11}&\dots&b_{1n}\\
\vdots&\ddots&\vdots\\
b_{r1}&\dots&b_{rn}\\
&\bigzero&
\end{bmatrix}
\end{align*}
因此\(r(B_1)\le r(DB_1)+(n-r)\),于是
\begin{equation*}
r(AB)=r(DB_1)\ge r(B_1)+r-n=r(B)+r(A)-n
\end{equation*}
\end{proof}


设\(A=(a_{ij})\)为数域\(K\)上的\(n\)阶方阵,如果\(A'=A\),则\(A\)称为\(n\)阶
\textbf{对称方阵}

\begin{exercise}
设\(A\)是数域\(K\)上的\(n\)阶方阵,证明
\begin{enumerate}
\item 若\(A^2=E\),则
\begin{equation*}
r(A+E)+r(A-E)=n
\end{equation*}
\item 若\(A^2=A\),则
\begin{equation*}
r(A)+r(A-E)=n
\end{equation*}
\end{enumerate}
\end{exercise}

\begin{proof}
\begin{enumerate}
\item \(r((A+E)(A-E))=r(A^2-E)=0\ge r(A+E)+r(A-E)-n\)

\(r(A+E)+r(A-E)\ge r(A+E+E-A)=n\)
\end{enumerate}
\end{proof}

\begin{exercise}
设\(n\)为偶数,证明存在实数域上的\(n\)阶方阵\(A\)使
\end{exercise}

\begin{proof}
\(n=2\)时,取
\begin{equation*}
A=
\begin{bmatrix}
0&1\\
-1&0
\end{bmatrix}
\end{equation*}

\(n\)为
一般偶数时由此组合出
\end{proof}
\subsection{分块矩阵}
\label{sec:org217ed4c}
设\(A\)是数域\(K\)上的\(m\times n\)矩阵,\(B\)是\(K\)上\(n\times k\)矩阵,把
它们按如下方式分割成小块

 \begin{gather*}
    A=
 \begin{bNiceArray}{c:c:cc:c}[first-row,first-col,nullify-dots]
&n_{1}&n_{2}&\Cdots&&n_{s}\\
m_{1}&&&&&\\
\hdottedline
m_{2}&&&&&\\
\hdottedline
\Vdots&&&&&\\
\hdottedline
m_{r}&&&&&\\
\end{bNiceArray},\\
    B=
 \begin{bNiceArray}{c:c:cc:c}[first-row,first-col,nullify-dots]
&k_{1}&k_{2}&\Cdots&&k_{t}\\
n_{1}&&&&&\\
\hdottedline
n_{2}&&&&&\\
\hdottedline
\Vdots&&&&&\\
\hdottedline
n_{s}&&&&&\\
\end{bNiceArray}
 \end{gather*}

即将\(A\)的行分割为\(r\)段,每段分别包含\(m_1,\dots,m_r\)行,又将\(A\)的列分
割为\(s\)段,每段分别包含\(n_1,\dots,n_s\)列,于是
\begin{equation*}
A=
\begin{bmatrix}
A_{11}&A_{12}&\cdots&A_{1s}\\
A_{21}&A_{22}&\dots&A_{2s}\\
\vdots&\vdots&\ddots&\vdots\\
A_{r1}&A_{r2}&\cdots&A_{rs}
\end{bmatrix}
\end{equation*}
其中\(A_{ij}\)为\(m_i\times n_j\)矩阵,对\(B\)作类似的分割
\begin{equation*}
B=
\begin{bmatrix}
B_{11}&B_{12}&\cdots&B_{1t}\\
B_{21}&B_{22}&\dots&B_{2t}\\
\vdots&\vdots&\ddots&\vdots\\
B_{s1}&B_{s2}&\cdots&B_{st}
\end{bmatrix}
\end{equation*}

这种分割称为矩阵
的分块,设\(AB=C\),则\(C\)有如下分块形式
\begin{equation*}
C=
\begin{bmatrix}
C_{11}&C_{12}&\cdots&C_{1t}\\
C_{21}&C_{22}&\dots&C_{2t}\\
\vdots&\vdots&\ddots&\vdots\\
C_{r1}&C_{r2}&\cdots&C_{rt}
\end{bmatrix}
\end{equation*}
其中\(C_{ij}\)是\(m_i\times k_j\)矩阵,且
\begin{equation*}
C_{ij}=\sum_{l=1}^sA_{il}B_{lj}
\end{equation*}
\subsubsection{准对角矩阵}
\label{sec:org28598e6}
给定数域\(K\)上的两个对角矩阵
\begin{equation*}
A=
\begin{bmatrix}
a_1&&\bigzero\\
&\ddots&\\
\bigzero&&a_n
\end{bmatrix},
B=
\begin{bmatrix}
b_1&&\bigzero\\
&\ddots&\\
\bigzero&&b_n
\end{bmatrix}
\end{equation*}
作矩阵乘法,显然有
\begin{equation*}
AB=
\begin{bmatrix}
a_1b_1&&\bigzero\\
&\ddots&\\
\bigzero&&a_nb_n
\end{bmatrix}
\end{equation*}
依然为对角矩阵

考虑
\begin{equation*}
A=
\begin{bmatrix}
a_{11}&a_{12}&0&0&0\\
a_{21}&a_{22}&0&0&0\\
0&0&a_{33}&a_{34}&a_{35}\\
0&0&a_{43}&a_{44}&a_{45}\\
0&0&a_{53}&a_{54}&a_{55}\\
\end{bmatrix},
B=
\begin{bmatrix}
b_{11}&b_{12}&0&0&0\\
b_{21}&b_{22}&0&0&0\\
0&0&b_{33}&b_{34}&b_{35}\\
0&0&b_{43}&b_{44}&b_{45}\\
0&0&b_{53}&b_{54}&b_{55}\\
\end{bmatrix}
\end{equation*}
如果令
\begin{alignat*}{2}
&A_1=
\begin{bmatrix}
a_{11}&a_{12}\\
a_{21}&a_{22}
\end{bmatrix}&&
B_1=
\begin{bmatrix}
b_{11}&b_{12}\\
b_{21}&b_{22}
\end{bmatrix}\\
&A_2=
\begin{bmatrix}
a_{33}&a_{34}&a_{35}\\
a_{43}&a_{44}&a_{45}\\
a_{53}&a_{54}&a_{55}\\
\end{bmatrix}
&&B_2=
\begin{bmatrix}
b_{33}&b_{34}&b_{35}\\
b_{43}&b_{44}&b_{45}\\
b_{53}&b_{54}&b_{55}\\
\end{bmatrix}
\end{alignat*}
则
\begin{equation*}
A=
\begin{bmatrix}
A_1&0\\0&A_2
\end{bmatrix},\quad
B=
\begin{bmatrix}
B_1&0\\0&B_2
\end{bmatrix}
\end{equation*}
此时有
\begin{equation*}
AB=
\begin{bmatrix}
A_1B_1&0\\0&&A_2B_2
\end{bmatrix}
\end{equation*}

\begin{definition}[]
称数域\(K\)上的分块形式的\(n\)阶方阵
\begin{equation*}
A=
\begin{bmatrix}
A_1&&&\\
&A_2&&\\
&&\ddots&\\
&&&A_s
\end{bmatrix}
\end{equation*}
为 \textbf{准对角矩阵} ,其中\(A_i(i=1,\dots,s)\)为\(n_i\)阶方阵,且\(n_1+\dots+n_s=n\)
\end{definition}

\(n\)阶准对角矩阵有如下性质
\begin{enumerate}
\item 两个同类型的\(n\)阶准对角矩阵
 \begin{equation*}
 A=
\begin{bmatrix}
A_1&&&\\
&A_2&&\\
&&\ddots&\\
&&&A_s
\end{bmatrix},\quad
B=
\begin{bmatrix}
B_1&&&\\
&B_2&&\\
&&\ddots&\\
&&&B_s
\end{bmatrix}
 \end{equation*}
有
\begin{equation*}
AB=
\end{enumerate}
\begin{bmatrix}
A_1B_1&&&\\
&A_2B_2&&\\
&&\ddots&\\
&&&A_sB_s
\end{bmatrix}
\end{equation*}
\begin{enumerate}
\setcounter{enumi}{1}
\item \(r(A)=r(A_1)+\dots+r(A_s)\)
\item \(A\)可逆\(\Leftrightarrow A_i(i=1,\dots,s)\)都可逆,且
\begin{equation*}
A^{-1}=
\begin{bmatrix}
A_1^{-1}&&\\
&\ddots&\\
&&A_s^{-1}
\end{bmatrix}
\end{equation*}
\end{enumerate}
\subsubsection{分块矩阵的秩}
\label{sec:org91fa81a}
\begin{proposition}[]
给定数域\(K\)上的分块矩阵
\begin{equation*}
M=
\begin{bmatrix}
A&C\\0&B
\end{bmatrix}
\end{equation*}
其中\(A\)为\(m\times n\)矩阵,\(B\)为\(k\times l\)矩阵,则
\begin{equation*}
r(A)+r(B)\le r(M)
\end{equation*}
\end{proposition}

\begin{corollary}[]
给定数域\(K\)上的分块矩阵
\begin{equation*}
N=
\begin{bmatrix}
A&0\\
C&B
\end{bmatrix}
\end{equation*}
则有
\begin{equation*}
r(A)+r(B)\le r(N)
\end{equation*}
\end{corollary}

\begin{corollary}[]
给定数域\(K\)上的分块矩阵
\begin{equation*}
M=
\begin{bmatrix}
A&C\\
0&B
\end{bmatrix},\quad
N=
\begin{bmatrix}
A&0\\
C&B
\end{bmatrix}
\end{equation*}
其中\(A\)为\(m\times n\)矩阵,\(B\)为\(k\times l\)矩阵

当\(r(A)=m,r(B)=k\)时,\(r(M)=r(A)+r(B)\)

当\(r(A)=n,r(B)=l\)时,\(r(N)=r(A)+r(B)\)
\end{corollary}

\begin{proposition}[]
设\(A\)是数域\(K\)上的\(m\times n\)矩阵,\(B\)是\(K\)上的\(n\times k\)矩阵,
\(C\)是\(K\)上的\(k\times s\)矩阵,则
\begin{equation*}
r(AB)+r(BC)\le r(ABC)+r(B)
\end{equation*}
\end{proposition}

\begin{proof}
令
\begin{equation*}
M=
\begin{bmatrix}
AB&0\\
B&BC
\end{bmatrix}
\end{equation*}
\end{proof}
\subsubsection{矩阵的分块矩阵}
\label{sec:orgd5874c7}
给定数域\(K\)上的\(n\)阶分块方阵
\begin{equation*}
M=
\begin{bmatrix}
A&B\\C&D
\end{bmatrix}
\end{equation*}
其中\(A\)为\(k\)阶可逆矩阵,我们有
\begin{align*}
N&=
\begin{bmatrix}
E_k&0\\-CA^{-1}&E_{n-k}
\end{bmatrix}
\begin{bmatrix}
A&B\\C&D
\end{bmatrix}
\begin{bmatrix}
E_k&-A^{-1}B\\0&E_{n-k}
\end{bmatrix}\\
&=
\begin{bmatrix}
A&0\\0&D-CA^{-1}B
\end{bmatrix}
\end{align*}
因此\(r(M)=r(N)=r(A)+r(D-CA^{-1}B)\),若\(r(M)=n\),令\(D_1=D-CA^{-1}B\),则
\(r(D_1)=n-k\),故当\(M\)可逆时,\(D_1\)可逆,而
\begin{align*}
N^{-1}&=
\begin{bmatrix}
E_k&-A^{-1}B\\0&E_{n-k}
\end{bmatrix}^{-1}
\begin{bmatrix}
A&C\\C&D
\end{bmatrix}^{-1}
\begin{bmatrix}
E_k&0\\-CA^{-1}&E_{n-k}
\end{bmatrix}^{-1}\\
&=
\begin{bmatrix}
A&0\\0&D_1
\end{bmatrix}^{-1}=
\begin{bmatrix}
A^{-1}&0\\0&D_1^{-1}\\
\end{bmatrix}
\end{align*}
于是
\begin{align*}
M^{-1}&=
\begin{bmatrix}
A&B\\C&D
\end{bmatrix}^{-1}\\
&=
\begin{bmatrix}
E_k&-A^{-1}B\\0&E_{n-k}
\end{bmatrix}
\begin{bmatrix}
A^{-1}&0\\0&D_1^{-1}
\end{bmatrix}
\begin{bmatrix}
E_k&0\\-CA^{-1}&E_{n-k}
\end{bmatrix}
\end{align*}
\section{行列式}
\label{sec:orgfe85b7d}
\subsection{平行六面体的有向体积}
\label{sec:orgb7a44c7}
给定两向量\(\ba,\bb\),它们的 \textbf{点乘} 是
\begin{equation*}
\ba\cdot\bb=\abs{\ba}\abs{\bb}\cos\la\ba,\bb\ra
\end{equation*}
如果
\begin{equation*}
\ba=(a_1,a_2,a_3),\quad\bb=(b_1,b_2,b_3)
\end{equation*}
那么
\begin{equation*}
\ba\cdot\bb=a_1b_1+a_2b_2+a_3b_3
\end{equation*}

它们的 \textbf{叉乘} 定义为一个向量\(\bc=\ba\times\bb\),当\(\ba,\bb\)共线时,\(\bc\)
为零向量,当\(\ba,\bb\)不共线时,\(\bc\)与\(\ba,\bb\)所决定的平面垂直,其指向
使\(\ba,\bb,\bc\)组成一个右手系,而\(\bc\)的长度为
\begin{equation*}
\abs{\bc}=\abs{\ba}\cdot\abs{\bb}\sin\la\ba,\bb\ra
\end{equation*}
数值恰好是以\(\ba,\bb\)为边的平行四边形的面积
\begin{equation*}
\bc=\ba\times\bb=(a_2b_3-a_3b_2,a_3b_1-a_1b_3,a_1b_2-a_2b_1)
\end{equation*}
为了把上面向量\(\ba,\bb\)的三个坐标与\(\ba,\bb\)的坐标之间的关系更清楚地表达,
引进一个记号。对数域\(K\)上的二阶方阵
\begin{equation*}
A=
\begin{bmatrix}
x_1&x_2\\y_1&y_2
\end{bmatrix}
\end{equation*}
定义
\end{document}
