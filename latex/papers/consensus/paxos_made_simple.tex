% Created 2024-04-17 Wed 20:04
% Intended LaTeX compiler: xelatex
\documentclass[11pt]{article}
\usepackage{hyperref}
% TIPS
% \substack{a\\b} for multiple lines text





% pdfplots will load xolor automatically without option
\usepackage[dvipsnames]{xcolor}

\usepackage{forest}
% two-line text in node by [two \\ lines]
% \begin{forest} qtree, [..] \end{forest}
\forestset{
  qtree/.style={
    baseline,
    for tree={
      parent anchor=south,
      child anchor=north,
      align=center,
      inner sep=1pt,
    }}}
%\usepackage{flexisym}
% load order of mathtools and mathabx, otherwise conflict overbrace

\usepackage{mathtools}
%\usepackage{fourier}
\usepackage{pgfplots}
\usepackage{amsthm, mathabx,  amsmath, commath}
\usepackage{amsfonts}

\usepackage{empheq}
\usepackage{tikz}
\usetikzlibrary{arrows.meta}
\usepackage[most]{tcolorbox}

\newtheorem{theorem}{Theorem}[section]
\newtheorem{definition}{Definition}[section]
\newtheorem{corollary}{Corollary}[section]
\newtheorem{example}{Example}[section]
\newtheorem{lemma}{Lemma}[section]
\newtheorem{proposition}{Proposition}[section]

\newcommand{\bl}[1] {\boldsymbol{#1}}
\newcommand{\Wt}[1] {\stackrel{\sim}{\smash{#1}\rule{0pt}{1.1ex}}}
\newcommand{\wt}[1] {\widetilde{#1}}


%For boxed texts in align, use Aboxed{}
%otherwise use boxed{}

\DeclareMathSymbol{\widehatsym}{\mathord}{largesymbols}{"62}
\newcommand\lowerwidehatsym{%
  \text{\smash{\raisebox{-1.3ex}{%
    $\widehatsym$}}}}
\newcommand\fixwidehat[1]{%
  \mathchoice
    {\accentset{\displaystyle\lowerwidehatsym}{#1}}
    {\accentset{\textstyle\lowerwidehatsym}{#1}}
    {\accentset{\scriptstyle\lowerwidehatsym}{#1}}
    {\accentset{\scriptscriptstyle\lowerwidehatsym}{#1}}
}

\usepackage{graphicx}
    
% text on arrow for xRightarrow
\makeatletter
%\newcommand{\xRightarrow}[2][]{\ext@arrow 0359\Rightarrowfill@{#1}{#2}}
\makeatother


\def \bx {\boldsymbol{x}}
\def \ba {\boldsymbol{a}}
\def \bI {\boldsymbol{I}}
\def \bt {\boldsymbol{t}}
\def \bb {\boldsymbol{b}}
\def \bA {\boldsymbol{A}}
\def \bX {\boldsymbol{X}}
\def \bu {\boldsymbol{u}}
\def \bS {\boldsymbol{S}}
\def \bZ {\boldsymbol{Z}}
\def \bz {\boldsymbol{z}}
\def \by {\boldsymbol{y}}
\def \bw {\boldsymbol{w}}
\def \bT {\boldsymbol{T}}
\def \bS {\boldsymbol{S}}
\def \bm {\boldsymbol{m}}
\def \bW {\boldsymbol{W}}
\def \bY {\boldsymbol{Y}}
\def \bH {\boldsymbol{H}}
\def \blambda {\boldsymbol{\lambda}}
\def \bPhi {\boldsymbol{\Phi}}
\def \btheta {\boldsymbol{\theta}}
\def \bmu {\boldsymbol{\mu}}
\def \bphi {\boldsymbol{\phi}}
\def \bSigma {\boldsymbol{\Sigma}}
\def \lb {\left\{}
\def \rb {\right\}}
\def \caln {\mathcal{N}}
\def \dissum {\displaystyle\Sigma}
\def \dispro {\displaystyle\prod}
\def \E {\mathbb{E}}
\def \Q {\mathbb{Q}}
\def \V {\mathbb{V}}
\def \R {\mathbb{R}}
\def \calq {\mathcal{Q}}
\def \calg {\mathcal{G}}
\def \caln {\mathcal{N}}
\def \calr {\mathcal{R}}
\def \calm {\mathcal{M}}
\def \calc {\mathcal{C}}
\def \bcup {\bigcup}

\graphicspath{{../../../paper/consensus/}}

%% ox-latex features:
%   !announce-start, !guess-pollyglossia, !guess-babel, !guess-inputenc,
%   underline, !announce-end.

\usepackage[normalem]{ulem}

%% end ox-latex features


\author{wu}
\date{\today}
\title{Paxos Made Simple}
\hypersetup{
 pdfauthor={wu},
 pdftitle={Paxos Made Simple},
 pdfkeywords={},
 pdfsubject={},
 pdfcreator={Emacs 29.1 (Org mode 9.7-pre)}, 
 pdflang={English}}
\begin{document}

\maketitle
\tableofcontents

\section{The Consensus Algorithm}
\label{sec:org72551da}
\subsection{The problem}
\label{sec:orgea35aa9}
Assume a collection of processes that can propose values. A consensus algorithm ensures that a single
one among the proposed values is chosen. If no value is proposed, then no value should be chosen. If a value
has been chosen, then processes should be able to learn the chosen value. The safety requirements for
consensus are:
\begin{itemize}
\item Only a value that has been proposed may be chosen
\item Only a single value is chosen
\item A process never learns that a value has been chosen unless it actually has been
\end{itemize}

The \textbf{goal} is to ensure that some proposed value is eventually chosen and, if a value has been chosen,
then a process can eventually learn the value.

We let the three roles in the consensus algorithm be performed by three classes of agents: \textbf{proposers},
\textbf{acceptors} and \textbf{learners}.

Assume that agents can communicate with one another by sending messages. We use the customary
asynchronous, non-Byzantine model, where:
\begin{itemize}
\item Agents operate at arbitrary  speed, may fail by stopping, and may restart. Since all agents may fail
after a value is chosen and then restart, a solution is impossible unless some information can be
remembered by an agent that has failed and restarted.
\item Messages can take arbitrarily long to be delivered, can be duplicated and can be lost, but they are
not corrupted.
\end{itemize}
\subsection{Choosing a Value}
\label{sec:org6ff7414}
Instead of a single acceptor, let's use multiple acceptor agents. A proposer sends a proposed value to
a set of acceptors. An acceptor may \textbf{accept} the proposed value. The value is chosen when a large enough
set of acceptors have accepted it.

In the absence of failure or message loss, we want a value to be chosen even if only one value is
proposed by a single proposer. This suggests the requirement:

\textbf{P1}. An acceptor must accept the first proposal that it receives.

But this requirement raises a problem. Several values could be proposed by different proposers at
about the same time, leading to a situation in which every acceptor has accepted a value, but no
single value is accepted by a majority of them. Even with just two proposed values, if each is
accepted by about half the acceptors, failure of a single acceptor could make it impossible to learn
which of the values was chosen.

P1 and the requirement that a value is chosen only when it is accepted by a majority of acceptors
imply that \uline{an acceptor must be allowed to accept more than one proposal}. We keep track of the
different proposals, so a proposal consists of a proposal number and a value. To prevent confusion, we
require that different proposals have different numbers. A value is chosen when a single proposal with
that value has been accepted by a majority of the acceptors. In that case, we say that the proposal
has been \textbf{chosen}.

We can allow multiple proposals to be chosen, but we must guarantee that all chosen proposals have the
same value. By induction on the proposal number, it suffices to guarantee:

\textbf{P2}. If a proposal with value \(v\) is chosen, then every higher-numbered proposal that chosen has
value \(v\).

P2 guarantees the crucial safety property that only a single value is chosen.

To be chosen, a proposal must be accepted by at least one acceptor. So, we can satisfy P2 by
satisfying

\textbf{P\(2^a\)}. If a proposal with value \(v\) is chosen, then every higher-numbered proposal accpeted by
any acceptor has value \(v\).

Because communication is asynchronous, a proposal could be chosen with some particular acceptor \(c\)
never having received any proposal. Suppose a new proposer ``wakes up'' and issues a higher-numbered
proposal with different value. P1 requires \(c\) to accept this proposal, violating P\(2^a\).
Maintaining both P1 and P\(2^a\) requires strengthening \(P2^a\) to:

\textbf{P\(2^b\)}. If a proposal with value \(v\) is chosen, then every higher-numbered proposal issued by an
proposer has value \(v\).

To discover
\end{document}
