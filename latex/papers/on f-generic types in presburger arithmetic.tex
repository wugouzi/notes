% Created 2023-03-01 Wed 20:09
% Intended LaTeX compiler: pdflatex
\documentclass[11pt]{article}
\usepackage[utf8]{inputenc}
\usepackage[T1]{fontenc}
\usepackage{graphicx}
\usepackage{longtable}
\usepackage{wrapfig}
\usepackage{rotating}
\usepackage[normalem]{ulem}
\usepackage{amsmath}
\usepackage{amssymb}
\usepackage{capt-of}
\usepackage{hyperref}
\graphicspath{{../../books/}}
% wrong resolution of image
% https://tex.stackexchange.com/questions/21627/image-from-includegraphics-showing-in-wrong-image-size?rq=1

%%%%%%%%%%%%%%%%%%%%%%%%%%%%%%%%%%%%%%
%% TIPS                                 %%
%%%%%%%%%%%%%%%%%%%%%%%%%%%%%%%%%%%%%%
% \substack{a\\b} for multiple lines text
% \usepackage{expl3}
% \expandafter\def\csname ver@l3regex.sty\endcsname{}
% \usepackage{pkgloader}
\usepackage[utf8]{inputenc}

% nfss error
% \usepackage[B1,T1]{fontenc}
\usepackage{fontspec}

% \usepackage[Emoticons]{ucharclasses}
\newfontfamily\DejaSans{DejaVu Sans}
% \setDefaultTransitions{\DejaSans}{}

% pdfplots will load xolor automatically without option
\usepackage[dvipsnames]{xcolor}

%                                                             ┳┳┓   ┓
%                                                             ┃┃┃┏┓╋┣┓
%                                                             ┛ ┗┗┻┗┛┗
% \usepackage{amsmath} mathtools loads the amsmath
\usepackage{amsmath}
\usepackage{mathtools}

\usepackage{amsthm}
\usepackage{amsbsy}

%\usepackage{commath}

\usepackage{amssymb}

\usepackage{mathrsfs}
%\usepackage{mathabx}
\usepackage{stmaryrd}
\usepackage{empheq}

\usepackage{scalerel}
\usepackage{stackengine}
\usepackage{stackrel}



\usepackage{nicematrix}
\usepackage{tensor}
\usepackage{blkarray}
\usepackage{siunitx}
\usepackage[f]{esvect}

% centering \not on a letter
\usepackage{slashed}
\usepackage[makeroom]{cancel}

%\usepackage{merriweather}
\usepackage{unicode-math}
\setmainfont{TeX Gyre Pagella}
% \setmathfont{STIX}
%\setmathfont{texgyrepagella-math.otf}
%\setmathfont{Libertinus Math}
\setmathfont{Latin Modern Math}

 % \setmathfont[range={\smwhtdiamond,\enclosediamond,\varlrtriangle}]{Latin Modern Math}
\setmathfont[range={\rightrightarrows,\twoheadrightarrow,\leftrightsquigarrow,\triangledown,\vartriangle,\precneq,\succneq,\prec,\succ,\preceq,\succeq,\tieconcat}]{XITS Math}
 \setmathfont[range={\int,\setminus}]{Libertinus Math}
 % \setmathfont[range={\mathalpha}]{TeX Gyre Pagella Math}
%\setmathfont[range={\mitA,\mitB,\mitC,\mitD,\mitE,\mitF,\mitG,\mitH,\mitI,\mitJ,\mitK,\mitL,\mitM,\mitN,\mitO,\mitP,\mitQ,\mitR,\mitS,\mitT,\mitU,\mitV,\mitW,\mitX,\mitY,\mitZ,\mita,\mitb,\mitc,\mitd,\mite,\mitf,\mitg,\miti,\mitj,\mitk,\mitl,\mitm,\mitn,\mito,\mitp,\mitq,\mitr,\mits,\mitt,\mitu,\mitv,\mitw,\mitx,\mity,\mitz}]{TeX Gyre Pagella Math}
% unicode is not good at this!
%\let\nmodels\nvDash

 \usepackage{wasysym}

 % for wide hat
 \DeclareSymbolFont{yhlargesymbols}{OMX}{yhex}{m}{n} \DeclareMathAccent{\what}{\mathord}{yhlargesymbols}{"62}

%                                                               ┏┳┓•┓
%                                                                ┃ ┓┃┏┓
%                                                                ┻ ┗┛┗┗

\usepackage{pgfplots}
\pgfplotsset{compat=1.18}
\usepackage{tikz}
\usepackage{tikz-cd}
\tikzcdset{scale cd/.style={every label/.append style={scale=#1},
    cells={nodes={scale=#1}}}}
% TODO: discard qtree and use forest
% \usepackage{tikz-qtree}
\usepackage{forest}

\usetikzlibrary{arrows,positioning,calc,fadings,decorations,matrix,decorations,shapes.misc}
%setting from geogebra
\definecolor{ccqqqq}{rgb}{0.8,0,0}

%                                                          ┳┳┓•    ┓┓
%                                                          ┃┃┃┓┏┏┏┓┃┃┏┓┏┓┏┓┏┓┓┏┏
%                                                          ┛ ┗┗┛┗┗ ┗┗┗┻┛┗┗ ┗┛┗┻┛
%\usepackage{twemojis}
\usepackage[most]{tcolorbox}
\usepackage{threeparttable}
\usepackage{tabularx}

\usepackage{enumitem}
\usepackage[indLines=false]{algpseudocodex}
\usepackage[]{algorithm2e}
% \SetKwComment{Comment}{/* }{ */}
% \algrenewcommand\algorithmicrequire{\textbf{Input:}}
% \algrenewcommand\algorithmicensure{\textbf{Output:}}
% wrong with preview
\usepackage{subcaption}
\usepackage{caption}
% {\aunclfamily\Huge}
\usepackage{auncial}

\usepackage{float}

\usepackage{fancyhdr}

\usepackage{ifthen}
\usepackage{xargs}

\definecolor{mintedbg}{rgb}{0.99,0.99,0.99}
\usepackage[cachedir=\detokenize{~/miscellaneous/trash}]{minted}
\setminted{breaklines,
  mathescape,
  bgcolor=mintedbg,
  fontsize=\footnotesize,
  frame=single,
  linenos}
\usemintedstyle{xcode}
\usepackage{tcolorbox}
\usepackage{etoolbox}



\usepackage{imakeidx}
\usepackage{hyperref}
\usepackage{soul}
\usepackage{framed}

% don't use this for preview
%\usepackage[margin=1.5in]{geometry}
% \usepackage{geometry}
% \geometry{legalpaper, landscape, margin=1in}
\usepackage[font=itshape]{quoting}

%\LoadPackagesNow
%\usepackage[xetex]{preview}
%%%%%%%%%%%%%%%%%%%%%%%%%%%%%%%%%%%%%%%
%% USEPACKAGES end                       %%
%%%%%%%%%%%%%%%%%%%%%%%%%%%%%%%%%%%%%%%

%%%%%%%%%%%%%%%%%%%%%%%%%%%%%%%%%%%%%%%
%% Algorithm environment
%%%%%%%%%%%%%%%%%%%%%%%%%%%%%%%%%%%%%%%
\SetKwIF{Recv}{}{}{upon receiving}{do}{}{}{}
\SetKwBlock{Init}{initially do}{}
\SetKwProg{Function}{Function}{:}{}

% https://github.com/chrmatt/algpseudocodex/issues/3
\algnewcommand\algorithmicswitch{\textbf{switch}}%
\algnewcommand\algorithmiccase{\textbf{case}}
\algnewcommand\algorithmicof{\textbf{of}}
\algnewcommand\algorithmicotherwise{\texttt{otherwise} $\Rightarrow$}

\makeatletter
\algdef{SE}[SWITCH]{Switch}{EndSwitch}[1]{\algpx@startIndent\algpx@startCodeCommand\algorithmicswitch\ #1\ \algorithmicdo}{\algpx@endIndent\algpx@startCodeCommand\algorithmicend\ \algorithmicswitch}%
\algdef{SE}[CASE]{Case}{EndCase}[1]{\algpx@startIndent\algpx@startCodeCommand\algorithmiccase\ #1}{\algpx@endIndent\algpx@startCodeCommand\algorithmicend\ \algorithmiccase}%
\algdef{SE}[CASEOF]{CaseOf}{EndCaseOf}[1]{\algpx@startIndent\algpx@startCodeCommand\algorithmiccase\ #1 \algorithmicof}{\algpx@endIndent\algpx@startCodeCommand\algorithmicend\ \algorithmiccase}
\algdef{SE}[OTHERWISE]{Otherwise}{EndOtherwise}[0]{\algpx@startIndent\algpx@startCodeCommand\algorithmicotherwise}{\algpx@endIndent\algpx@startCodeCommand\algorithmicend\ \algorithmicotherwise}
\ifbool{algpx@noEnd}{%
  \algtext*{EndSwitch}%
  \algtext*{EndCase}%
  \algtext*{EndCaseOf}
  \algtext*{EndOtherwise}
  %
  % end indent line after (not before), to get correct y position for multiline text in last command
  \apptocmd{\EndSwitch}{\algpx@endIndent}{}{}%
  \apptocmd{\EndCase}{\algpx@endIndent}{}{}%
  \apptocmd{\EndCaseOf}{\algpx@endIndent}{}{}
  \apptocmd{\EndOtherwise}{\algpx@endIndent}{}{}
}{}%

\pretocmd{\Switch}{\algpx@endCodeCommand}{}{}
\pretocmd{\Case}{\algpx@endCodeCommand}{}{}
\pretocmd{\CaseOf}{\algpx@endCodeCommand}{}{}
\pretocmd{\Otherwise}{\algpx@endCodeCommand}{}{}

% for end commands that may not be printed, tell endCodeCommand whether we are using noEnd
\ifbool{algpx@noEnd}{%
  \pretocmd{\EndSwitch}{\algpx@endCodeCommand[1]}{}{}%
  \pretocmd{\EndCase}{\algpx@endCodeCommand[1]}{}{}
  \pretocmd{\EndCaseOf}{\algpx@endCodeCommand[1]}{}{}%
  \pretocmd{\EndOtherwise}{\algpx@endCodeCommand[1]}{}{}
}{%
  \pretocmd{\EndSwitch}{\algpx@endCodeCommand[0]}{}{}%
  \pretocmd{\EndCase}{\algpx@endCodeCommand[0]}{}{}%
  \pretocmd{\EndCaseOf}{\algpx@endCodeCommand[0]}{}{}
  \pretocmd{\EndOtherwise}{\algpx@endCodeCommand[0]}{}{}
}%
\makeatother
% % For algpseudocode
% \algnewcommand\algorithmicswitch{\textbf{switch}}
% \algnewcommand\algorithmiccase{\textbf{case}}
% \algnewcommand\algorithmiccaseof{\textbf{case}}
% \algnewcommand\algorithmicof{\textbf{of}}
% % New "environments"
% \algdef{SE}[SWITCH]{Switch}{EndSwitch}[1]{\algorithmicswitch\ #1\ \algorithmicdo}{\algorithmicend\ \algorithmicswitch}%
% \algdef{SE}[CASE]{Case}{EndCase}[1]{\algorithmiccase\ #1}{\algorithmicend\ \algorithmiccase}%
% \algtext*{EndSwitch}%
% \algtext*{EndCase}
% \algdef{SE}[CASEOF]{CaseOf}{EndCaseOf}[1]{\algorithmiccaseof\ #1 \algorithmicof}{\algorithmicend\ \algorithmiccaseof}
% \algtext*{EndCaseOf}



%\pdfcompresslevel0

% quoting from
% https://tex.stackexchange.com/questions/391726/the-quotation-environment
\NewDocumentCommand{\bywhom}{m}{% the Bourbaki trick
  {\nobreak\hfill\penalty50\hskip1em\null\nobreak
   \hfill\mbox{\normalfont(#1)}%
   \parfillskip=0pt \finalhyphendemerits=0 \par}%
}

\NewDocumentEnvironment{pquotation}{m}
  {\begin{quoting}[
     indentfirst=true,
     leftmargin=\parindent,
     rightmargin=\parindent]\itshape}
  {\bywhom{#1}\end{quoting}}

\indexsetup{othercode=\small}
\makeindex[columns=2,options={-s /media/wu/file/stuuudy/notes/index_style.ist},intoc]
\makeatletter
\def\@idxitem{\par\hangindent 0pt}
\makeatother


% \newcounter{dummy} \numberwithin{dummy}{section}
\newtheorem{dummy}{dummy}[section]
\theoremstyle{definition}
\newtheorem{definition}[dummy]{Definition}
\theoremstyle{plain}
\newtheorem{corollary}[dummy]{Corollary}
\newtheorem{lemma}[dummy]{Lemma}
\newtheorem{proposition}[dummy]{Proposition}
\newtheorem{theorem}[dummy]{Theorem}
\newtheorem{notation}[dummy]{Notation}
\newtheorem{conjecture}[dummy]{Conjecture}
\newtheorem{fact}[dummy]{Fact}
\newtheorem{warning}[dummy]{Warning}
\theoremstyle{definition}
\newtheorem{examplle}{Example}[section]
\theoremstyle{remark}
\newtheorem*{remark}{Remark}
\newtheorem{exercise}{Exercise}[subsection]
\newtheorem{problem}{Problem}[subsection]
\newtheorem{observation}{Observation}[section]
\newenvironment{claim}[1]{\par\noindent\textbf{Claim:}\space#1}{}

\makeatletter
\DeclareFontFamily{U}{tipa}{}
\DeclareFontShape{U}{tipa}{m}{n}{<->tipa10}{}
\newcommand{\arc@char}{{\usefont{U}{tipa}{m}{n}\symbol{62}}}%

\newcommand{\arc}[1]{\mathpalette\arc@arc{#1}}

\newcommand{\arc@arc}[2]{%
  \sbox0{$\m@th#1#2$}%
  \vbox{
    \hbox{\resizebox{\wd0}{\height}{\arc@char}}
    \nointerlineskip
    \box0
  }%
}
\makeatother

\setcounter{MaxMatrixCols}{20}
%%%%%%% ABS
\DeclarePairedDelimiter\abss{\lvert}{\rvert}%
\DeclarePairedDelimiter\normm{\lVert}{\rVert}%

% Swap the definition of \abs* and \norm*, so that \abs
% and \norm resizes the size of the brackets, and the
% starred version does not.
\makeatletter
\let\oldabs\abss
%\def\abs{\@ifstar{\oldabs}{\oldabs*}}
\newcommand{\abs}{\@ifstar{\oldabs}{\oldabs*}}
\newcommand{\norm}[1]{\left\lVert#1\right\rVert}
%\let\oldnorm\normm
%\def\norm{\@ifstar{\oldnorm}{\oldnorm*}}
%\renewcommand{norm}{\@ifstar{\oldnorm}{\oldnorm*}}
\makeatother

% \stackMath
% \newcommand\what[1]{%
% \savestack{\tmpbox}{\stretchto{%
%   \scaleto{%
%     \scalerel*[\widthof{\ensuremath{#1}}]{\kern-.6pt\bigwedge\kern-.6pt}%
%     {\rule[-\textheight/2]{1ex}{\textheight}}%WIDTH-LIMITED BIG WEDGE
%   }{\textheight}%
% }{0.5ex}}%
% \stackon[1pt]{#1}{\tmpbox}%
% }

% \newcommand\what[1]{\ThisStyle{%
%     \setbox0=\hbox{$\SavedStyle#1$}%
%     \stackengine{-1.0\ht0+.5pt}{$\SavedStyle#1$}{%
%       \stretchto{\scaleto{\SavedStyle\mkern.15mu\char'136}{2.6\wd0}}{1.4\ht0}%
%     }{O}{c}{F}{T}{S}%
%   }
% }

% \newcommand\wtilde[1]{\ThisStyle{%
%     \setbox0=\hbox{$\SavedStyle#1$}%
%     \stackengine{-.1\LMpt}{$\SavedStyle#1$}{%
%       \stretchto{\scaleto{\SavedStyle\mkern.2mu\AC}{.5150\wd0}}{.6\ht0}%
%     }{O}{c}{F}{T}{S}%
%   }
% }

% \newcommand\wbar[1]{\ThisStyle{%
%     \setbox0=\hbox{$\SavedStyle#1$}%
%     \stackengine{.5pt+\LMpt}{$\SavedStyle#1$}{%
%       \rule{\wd0}{\dimexpr.3\LMpt+.3pt}%
%     }{O}{c}{F}{T}{S}%
%   }
% }

\newcommand{\bl}[1] {\boldsymbol{#1}}
\newcommand{\Wt}[1] {\stackrel{\sim}{\smash{#1}\rule{0pt}{1.1ex}}}
\newcommand{\wt}[1] {\widetilde{#1}}
\newcommand{\tf}[1] {\textbf{#1}}

\newcommand{\wu}[1]{{\color{red} #1}}

%For boxed texts in align, use Aboxed{}
%otherwise use boxed{}

\DeclareMathSymbol{\widehatsym}{\mathord}{largesymbols}{"62}
\newcommand\lowerwidehatsym{%
  \text{\smash{\raisebox{-1.3ex}{%
    $\widehatsym$}}}}
\newcommand\fixwidehat[1]{%
  \mathchoice
    {\accentset{\displaystyle\lowerwidehatsym}{#1}}
    {\accentset{\textstyle\lowerwidehatsym}{#1}}
    {\accentset{\scriptstyle\lowerwidehatsym}{#1}}
    {\accentset{\scriptscriptstyle\lowerwidehatsym}{#1}}
  }


\newcommand{\cupdot}{\mathbin{\dot{\cup}}}
\newcommand{\bigcupdot}{\mathop{\dot{\bigcup}}}

\usepackage{graphicx}

\usepackage[toc,page]{appendix}

% text on arrow for xRightarrow
\makeatletter
%\newcommand{\xRightarrow}[2][]{\ext@arrow 0359\Rightarrowfill@{#1}{#2}}
\makeatother

% Arbitrary long arrow
\newcommand{\Rarrow}[1]{%
\parbox{#1}{\tikz{\draw[->](0,0)--(#1,0);}}
}

\newcommand{\LRarrow}[1]{%
\parbox{#1}{\tikz{\draw[<->](0,0)--(#1,0);}}
}


\makeatletter
\providecommand*{\rmodels}{%
  \mathrel{%
    \mathpalette\@rmodels\models
  }%
}
\newcommand*{\@rmodels}[2]{%
  \reflectbox{$\m@th#1#2$}%
}
\makeatother

% Roman numerals
\makeatletter
\newcommand*{\rom}[1]{\expandafter\@slowromancap\romannumeral #1@}
\makeatother
% \\def \\b\([a-zA-Z]\) {\\boldsymbol{[a-zA-z]}}
% \\DeclareMathOperator{\\b\1}{\\textbf{\1}}

\DeclareMathOperator*{\argmin}{arg\,min}
\DeclareMathOperator*{\argmax}{arg\,max}

\DeclareMathOperator{\bone}{\textbf{1}}
\DeclareMathOperator{\bx}{\textbf{x}}
\DeclareMathOperator{\bz}{\textbf{z}}
\DeclareMathOperator{\bff}{\textbf{f}}
\DeclareMathOperator{\ba}{\textbf{a}}
\DeclareMathOperator{\bk}{\textbf{k}}
\DeclareMathOperator{\bs}{\textbf{s}}
\DeclareMathOperator{\bh}{\textbf{h}}
\DeclareMathOperator{\bc}{\textbf{c}}
\DeclareMathOperator{\br}{\textbf{r}}
\DeclareMathOperator{\bi}{\textbf{i}}
\DeclareMathOperator{\bj}{\textbf{j}}
\DeclareMathOperator{\bn}{\textbf{n}}
\DeclareMathOperator{\be}{\textbf{e}}
\DeclareMathOperator{\bo}{\textbf{o}}
\DeclareMathOperator{\bU}{\textbf{U}}
\DeclareMathOperator{\bL}{\textbf{L}}
\DeclareMathOperator{\bV}{\textbf{V}}
\def \bzero {\mathbf{0}}
\def \bbone {\mathbb{1}}
\def \btwo {\mathbf{2}}
\DeclareMathOperator{\bv}{\textbf{v}}
\DeclareMathOperator{\bp}{\textbf{p}}
\DeclareMathOperator{\bI}{\textbf{I}}
\def \dbI {\dot{\bI}}
\DeclareMathOperator{\bM}{\textbf{M}}
\DeclareMathOperator{\bN}{\textbf{N}}
\DeclareMathOperator{\bK}{\textbf{K}}
\DeclareMathOperator{\bt}{\textbf{t}}
\DeclareMathOperator{\bb}{\textbf{b}}
\DeclareMathOperator{\bA}{\textbf{A}}
\DeclareMathOperator{\bX}{\textbf{X}}
\DeclareMathOperator{\bu}{\textbf{u}}
\DeclareMathOperator{\bS}{\textbf{S}}
\DeclareMathOperator{\bZ}{\textbf{Z}}
\DeclareMathOperator{\bJ}{\textbf{J}}
\DeclareMathOperator{\by}{\textbf{y}}
\DeclareMathOperator{\bw}{\textbf{w}}
\DeclareMathOperator{\bT}{\textbf{T}}
\DeclareMathOperator{\bF}{\textbf{F}}
\DeclareMathOperator{\bmm}{\textbf{m}}
\DeclareMathOperator{\bW}{\textbf{W}}
\DeclareMathOperator{\bR}{\textbf{R}}
\DeclareMathOperator{\bC}{\textbf{C}}
\DeclareMathOperator{\bD}{\textbf{D}}
\DeclareMathOperator{\bE}{\textbf{E}}
\DeclareMathOperator{\bQ}{\textbf{Q}}
\DeclareMathOperator{\bP}{\textbf{P}}
\DeclareMathOperator{\bY}{\textbf{Y}}
\DeclareMathOperator{\bH}{\textbf{H}}
\DeclareMathOperator{\bB}{\textbf{B}}
\DeclareMathOperator{\bG}{\textbf{G}}
\def \blambda {\symbf{\lambda}}
\def \boldeta {\symbf{\eta}}
\def \balpha {\symbf{\alpha}}
\def \btau {\symbf{\tau}}
\def \bbeta {\symbf{\beta}}
\def \bgamma {\symbf{\gamma}}
\def \bxi {\symbf{\xi}}
\def \bLambda {\symbf{\Lambda}}
\def \bGamma {\symbf{\Gamma}}

\newcommand{\bto}{{\boldsymbol{\to}}}
\newcommand{\Ra}{\Rightarrow}
\newcommand{\xrsa}[1]{\overset{#1}{\rightsquigarrow}}
\newcommand{\xlsa}[1]{\overset{#1}{\leftsquigarrow}}
\newcommand\und[1]{\underline{#1}}
\newcommand\ove[1]{\overline{#1}}
%\def \concat {\verb|^|}
\def \bPhi {\mbfPhi}
\def \btheta {\mbftheta}
\def \bTheta {\mbfTheta}
\def \bmu {\mbfmu}
\def \bphi {\mbfphi}
\def \bSigma {\mbfSigma}
\def \la {\langle}
\def \ra {\rangle}

\def \caln {\mathcal{N}}
\def \dissum {\displaystyle\Sigma}
\def \dispro {\displaystyle\prod}

\def \caret {\verb!^!}

\def \A {\mathbb{A}}
\def \B {\mathbb{B}}
\def \C {\mathbb{C}}
\def \D {\mathbb{D}}
\def \E {\mathbb{E}}
\def \F {\mathbb{F}}
\def \G {\mathbb{G}}
\def \H {\mathbb{H}}
\def \I {\mathbb{I}}
\def \J {\mathbb{J}}
\def \K {\mathbb{K}}
\def \L {\mathbb{L}}
\def \M {\mathbb{M}}
\def \N {\mathbb{N}}
\def \O {\mathbb{O}}
\def \P {\mathbb{P}}
\def \Q {\mathbb{Q}}
\def \R {\mathbb{R}}
\def \S {\mathbb{S}}
\def \T {\mathbb{T}}
\def \U {\mathbb{U}}
\def \V {\mathbb{V}}
\def \W {\mathbb{W}}
\def \X {\mathbb{X}}
\def \Y {\mathbb{Y}}
\def \Z {\mathbb{Z}}

\def \cala {\mathcal{A}}
\def \cale {\mathcal{E}}
\def \calb {\mathcal{B}}
\def \calq {\mathcal{Q}}
\def \calp {\mathcal{P}}
\def \cals {\mathcal{S}}
\def \calx {\mathcal{X}}
\def \caly {\mathcal{Y}}
\def \calg {\mathcal{G}}
\def \cald {\mathcal{D}}
\def \caln {\mathcal{N}}
\def \calr {\mathcal{R}}
\def \calt {\mathcal{T}}
\def \calm {\mathcal{M}}
\def \calw {\mathcal{W}}
\def \calc {\mathcal{C}}
\def \calv {\mathcal{V}}
\def \calf {\mathcal{F}}
\def \calk {\mathcal{K}}
\def \call {\mathcal{L}}
\def \calu {\mathcal{U}}
\def \calo {\mathcal{O}}
\def \calh {\mathcal{H}}
\def \cali {\mathcal{I}}
\def \calj {\mathcal{J}}

\def \bcup {\bigcup}

% set theory

\def \zfcc {\textbf{ZFC}^-}
\def \BGC {\textbf{BGC}}
\def \BG {\textbf{BG}}
\def \ac  {\textbf{AC}}
\def \gl  {\textbf{L }}
\def \gll {\textbf{L}}
\newcommand{\zfm}{$\textbf{ZF}^-$}

\def \ZFm {\text{ZF}^-}
\def \ZFCm {\text{ZFC}^-}
\DeclareMathOperator{\WF}{WF}
\DeclareMathOperator{\On}{On}
\def \on {\textbf{On }}
\def \cm {\textbf{M }}
\def \cn {\textbf{N }}
\def \cv {\textbf{V }}
\def \zc {\textbf{ZC }}
\def \zcm {\textbf{ZC}}
\def \zff {\textbf{ZF}}
\def \wfm {\textbf{WF}}
\def \onm {\textbf{On}}
\def \cmm {\textbf{M}}
\def \cnm {\textbf{N}}
\def \cvm {\textbf{V}}

\renewcommand{\restriction}{\mathord{\upharpoonright}}
%% another restriction
\newcommand\restr[2]{{% we make the whole thing an ordinary symbol
  \left.\kern-\nulldelimiterspace % automatically resize the bar with \right
  #1 % the function
  \vphantom{\big|} % pretend it's a little taller at normal size
  \right|_{#2} % this is the delimiter
  }}

\def \pred {\text{pred}}

\def \rank {\text{rank}}
\def \Con {\text{Con}}
\def \deff {\text{Def}}


\def \uin {\underline{\in}}
\def \oin {\overline{\in}}
\def \uR {\underline{R}}
\def \oR {\overline{R}}
\def \uP {\underline{P}}
\def \oP {\overline{P}}

\def \dsum {\displaystyle\sum}

\def \Ra {\Rightarrow}

\def \e {\enspace}

\def \sgn {\operatorname{sgn}}
\def \gen {\operatorname{gen}}
\def \Hom {\operatorname{Hom}}
\def \hom {\operatorname{hom}}
\def \Sub {\operatorname{Sub}}

\def \supp {\operatorname{supp}}

\def \epiarrow {\twoheadarrow}
\def \monoarrow {\rightarrowtail}
\def \rrarrow {\rightrightarrows}

% \def \minus {\text{-}}
% \newcommand{\minus}{\scalebox{0.75}[1.0]{$-$}}
% \DeclareUnicodeCharacter{002D}{\minus}


\def \tril {\triangleleft}

\def \ISigma {\text{I}\Sigma}
\def \IDelta {\text{I}\Delta}
\def \IPi {\text{I}\Pi}
\def \ACF {\textsf{ACF}}
\def \pCF {\textit{p}\text{CF}}
\def \ACVF {\textsf{ACVF}}
\def \HLR {\textsf{HLR}}
\def \OAG {\textsf{OAG}}
\def \RCF {\textsf{RCF}}
\DeclareMathOperator{\GL}{GL}
\DeclareMathOperator{\PGL}{PGL}
\DeclareMathOperator{\SL}{SL}
\DeclareMathOperator{\Inv}{Inv}
\DeclareMathOperator{\res}{res}
\DeclareMathOperator{\Sym}{Sym}
%\DeclareMathOperator{\char}{char}
\def \equal {=}

\def \degree {\text{degree}}
\def \app {\text{App}}
\def \FV {\text{FV}}
\def \conv {\text{conv}}
\def \cont {\text{cont}}
\DeclareMathOperator{\cl}{\text{cl}}
\DeclareMathOperator{\trcl}{\text{trcl}}
\DeclareMathOperator{\sg}{sg}
\DeclareMathOperator{\trdeg}{trdeg}
\def \Ord {\text{Ord}}

\DeclareMathOperator{\cf}{cf}
\DeclareMathOperator{\zfc}{ZFC}

%\DeclareMathOperator{\Th}{Th}
%\def \th {\text{Th}}
% \newcommand{\th}{\text{Th}}
\DeclareMathOperator{\type}{type}
\DeclareMathOperator{\zf}{\textbf{ZF}}
\def \fa {\mathfrak{a}}
\def \fb {\mathfrak{b}}
\def \fc {\mathfrak{c}}
\def \fd {\mathfrak{d}}
\def \fe {\mathfrak{e}}
\def \ff {\mathfrak{f}}
\def \fg {\mathfrak{g}}
\def \fh {\mathfrak{h}}
%\def \fi {\mathfrak{i}}
\def \fj {\mathfrak{j}}
\def \fk {\mathfrak{k}}
\def \fl {\mathfrak{l}}
\def \fm {\mathfrak{m}}
\def \fn {\mathfrak{n}}
\def \fo {\mathfrak{o}}
\def \fp {\mathfrak{p}}
\def \fq {\mathfrak{q}}
\def \fr {\mathfrak{r}}
\def \fs {\mathfrak{s}}
\def \ft {\mathfrak{t}}
\def \fu {\mathfrak{u}}
\def \fv {\mathfrak{v}}
\def \fw {\mathfrak{w}}
\def \fx {\mathfrak{x}}
\def \fy {\mathfrak{y}}
\def \fz {\mathfrak{z}}
\def \fA {\mathfrak{A}}
\def \fB {\mathfrak{B}}
\def \fC {\mathfrak{C}}
\def \fD {\mathfrak{D}}
\def \fE {\mathfrak{E}}
\def \fF {\mathfrak{F}}
\def \fG {\mathfrak{G}}
\def \fH {\mathfrak{H}}
\def \fI {\mathfrak{I}}
\def \fJ {\mathfrak{J}}
\def \fK {\mathfrak{K}}
\def \fL {\mathfrak{L}}
\def \fM {\mathfrak{M}}
\def \fN {\mathfrak{N}}
\def \fO {\mathfrak{O}}
\def \fP {\mathfrak{P}}
\def \fQ {\mathfrak{Q}}
\def \fR {\mathfrak{R}}
\def \fS {\mathfrak{S}}
\def \fT {\mathfrak{T}}
\def \fU {\mathfrak{U}}
\def \fV {\mathfrak{V}}
\def \fW {\mathfrak{W}}
\def \fX {\mathfrak{X}}
\def \fY {\mathfrak{Y}}
\def \fZ {\mathfrak{Z}}

\def \sfA {\textsf{A}}
\def \sfB {\textsf{B}}
\def \sfC {\textsf{C}}
\def \sfD {\textsf{D}}
\def \sfE {\textsf{E}}
\def \sfF {\textsf{F}}
\def \sfG {\textsf{G}}
\def \sfH {\textsf{H}}
\def \sfI {\textsf{I}}
\def \sfJ {\textsf{J}}
\def \sfK {\textsf{K}}
\def \sfL {\textsf{L}}
\def \sfM {\textsf{M}}
\def \sfN {\textsf{N}}
\def \sfO {\textsf{O}}
\def \sfP {\textsf{P}}
\def \sfQ {\textsf{Q}}
\def \sfR {\textsf{R}}
\def \sfS {\textsf{S}}
\def \sfT {\textsf{T}}
\def \sfU {\textsf{U}}
\def \sfV {\textsf{V}}
\def \sfW {\textsf{W}}
\def \sfX {\textsf{X}}
\def \sfY {\textsf{Y}}
\def \sfZ {\textsf{Z}}
\def \sfa {\textsf{a}}
\def \sfb {\textsf{b}}
\def \sfc {\textsf{c}}
\def \sfd {\textsf{d}}
\def \sfe {\textsf{e}}
\def \sff {\textsf{f}}
\def \sfg {\textsf{g}}
\def \sfh {\textsf{h}}
\def \sfi {\textsf{i}}
\def \sfj {\textsf{j}}
\def \sfk {\textsf{k}}
\def \sfl {\textsf{l}}
\def \sfm {\textsf{m}}
\def \sfn {\textsf{n}}
\def \sfo {\textsf{o}}
\def \sfp {\textsf{p}}
\def \sfq {\textsf{q}}
\def \sfr {\textsf{r}}
\def \sfs {\textsf{s}}
\def \sft {\textsf{t}}
\def \sfu {\textsf{u}}
\def \sfv {\textsf{v}}
\def \sfw {\textsf{w}}
\def \sfx {\textsf{x}}
\def \sfy {\textsf{y}}
\def \sfz {\textsf{z}}

\def \ttA {\texttt{A}}
\def \ttB {\texttt{B}}
\def \ttC {\texttt{C}}
\def \ttD {\texttt{D}}
\def \ttE {\texttt{E}}
\def \ttF {\texttt{F}}
\def \ttG {\texttt{G}}
\def \ttH {\texttt{H}}
\def \ttI {\texttt{I}}
\def \ttJ {\texttt{J}}
\def \ttK {\texttt{K}}
\def \ttL {\texttt{L}}
\def \ttM {\texttt{M}}
\def \ttN {\texttt{N}}
\def \ttO {\texttt{O}}
\def \ttP {\texttt{P}}
\def \ttQ {\texttt{Q}}
\def \ttR {\texttt{R}}
\def \ttS {\texttt{S}}
\def \ttT {\texttt{T}}
\def \ttU {\texttt{U}}
\def \ttV {\texttt{V}}
\def \ttW {\texttt{W}}
\def \ttX {\texttt{X}}
\def \ttY {\texttt{Y}}
\def \ttZ {\texttt{Z}}
\def \tta {\texttt{a}}
\def \ttb {\texttt{b}}
\def \ttc {\texttt{c}}
\def \ttd {\texttt{d}}
\def \tte {\texttt{e}}
\def \ttf {\texttt{f}}
\def \ttg {\texttt{g}}
\def \tth {\texttt{h}}
\def \tti {\texttt{i}}
\def \ttj {\texttt{j}}
\def \ttk {\texttt{k}}
\def \ttl {\texttt{l}}
\def \ttm {\texttt{m}}
\def \ttn {\texttt{n}}
\def \tto {\texttt{o}}
\def \ttp {\texttt{p}}
\def \ttq {\texttt{q}}
\def \ttr {\texttt{r}}
\def \tts {\texttt{s}}
\def \ttt {\texttt{t}}
\def \ttu {\texttt{u}}
\def \ttv {\texttt{v}}
\def \ttw {\texttt{w}}
\def \ttx {\texttt{x}}
\def \tty {\texttt{y}}
\def \ttz {\texttt{z}}

\def \bara {\bbar{a}}
\def \barb {\bbar{b}}
\def \barc {\bbar{c}}
\def \bard {\bbar{d}}
\def \bare {\bbar{e}}
\def \barf {\bbar{f}}
\def \barg {\bbar{g}}
\def \barh {\bbar{h}}
\def \bari {\bbar{i}}
\def \barj {\bbar{j}}
\def \bark {\bbar{k}}
\def \barl {\bbar{l}}
\def \barm {\bbar{m}}
\def \barn {\bbar{n}}
\def \baro {\bbar{o}}
\def \barp {\bbar{p}}
\def \barq {\bbar{q}}
\def \barr {\bbar{r}}
\def \bars {\bbar{s}}
\def \bart {\bbar{t}}
\def \baru {\bbar{u}}
\def \barv {\bbar{v}}
\def \barw {\bbar{w}}
\def \barx {\bbar{x}}
\def \bary {\bbar{y}}
\def \barz {\bbar{z}}
\def \barA {\bbar{A}}
\def \barB {\bbar{B}}
\def \barC {\bbar{C}}
\def \barD {\bbar{D}}
\def \barE {\bbar{E}}
\def \barF {\bbar{F}}
\def \barG {\bbar{G}}
\def \barH {\bbar{H}}
\def \barI {\bbar{I}}
\def \barJ {\bbar{J}}
\def \barK {\bbar{K}}
\def \barL {\bbar{L}}
\def \barM {\bbar{M}}
\def \barN {\bbar{N}}
\def \barO {\bbar{O}}
\def \barP {\bbar{P}}
\def \barQ {\bbar{Q}}
\def \barR {\bbar{R}}
\def \barS {\bbar{S}}
\def \barT {\bbar{T}}
\def \barU {\bbar{U}}
\def \barVV {\bbar{V}}
\def \barW {\bbar{W}}
\def \barX {\bbar{X}}
\def \barY {\bbar{Y}}
\def \barZ {\bbar{Z}}

\def \baralpha {\bbar{\alpha}}
\def \bartau {\bbar{\tau}}
\def \barsigma {\bbar{\sigma}}
\def \barzeta {\bbar{\zeta}}

\def \hata {\hat{a}}
\def \hatb {\hat{b}}
\def \hatc {\hat{c}}
\def \hatd {\hat{d}}
\def \hate {\hat{e}}
\def \hatf {\hat{f}}
\def \hatg {\hat{g}}
\def \hath {\hat{h}}
\def \hati {\hat{i}}
\def \hatj {\hat{j}}
\def \hatk {\hat{k}}
\def \hatl {\hat{l}}
\def \hatm {\hat{m}}
\def \hatn {\hat{n}}
\def \hato {\hat{o}}
\def \hatp {\hat{p}}
\def \hatq {\hat{q}}
\def \hatr {\hat{r}}
\def \hats {\hat{s}}
\def \hatt {\hat{t}}
\def \hatu {\hat{u}}
\def \hatv {\hat{v}}
\def \hatw {\hat{w}}
\def \hatx {\hat{x}}
\def \haty {\hat{y}}
\def \hatz {\hat{z}}
\def \hatA {\hat{A}}
\def \hatB {\hat{B}}
\def \hatC {\hat{C}}
\def \hatD {\hat{D}}
\def \hatE {\hat{E}}
\def \hatF {\hat{F}}
\def \hatG {\hat{G}}
\def \hatH {\hat{H}}
\def \hatI {\hat{I}}
\def \hatJ {\hat{J}}
\def \hatK {\hat{K}}
\def \hatL {\hat{L}}
\def \hatM {\hat{M}}
\def \hatN {\hat{N}}
\def \hatO {\hat{O}}
\def \hatP {\hat{P}}
\def \hatQ {\hat{Q}}
\def \hatR {\hat{R}}
\def \hatS {\hat{S}}
\def \hatT {\hat{T}}
\def \hatU {\hat{U}}
\def \hatVV {\hat{V}}
\def \hatW {\hat{W}}
\def \hatX {\hat{X}}
\def \hatY {\hat{Y}}
\def \hatZ {\hat{Z}}

\def \hatphi {\hat{\phi}}

\def \barfM {\bbar{\fM}}
\def \barfN {\bbar{\fN}}

\def \tila {\tilde{a}}
\def \tilb {\tilde{b}}
\def \tilc {\tilde{c}}
\def \tild {\tilde{d}}
\def \tile {\tilde{e}}
\def \tilf {\tilde{f}}
\def \tilg {\tilde{g}}
\def \tilh {\tilde{h}}
\def \tili {\tilde{i}}
\def \tilj {\tilde{j}}
\def \tilk {\tilde{k}}
\def \till {\tilde{l}}
\def \tilm {\tilde{m}}
\def \tiln {\tilde{n}}
\def \tilo {\tilde{o}}
\def \tilp {\tilde{p}}
\def \tilq {\tilde{q}}
\def \tilr {\tilde{r}}
\def \tils {\tilde{s}}
\def \tilt {\tilde{t}}
\def \tilu {\tilde{u}}
\def \tilv {\tilde{v}}
\def \tilw {\tilde{w}}
\def \tilx {\tilde{x}}
\def \tily {\tilde{y}}
\def \tilz {\tilde{z}}
\def \tilA {\tilde{A}}
\def \tilB {\tilde{B}}
\def \tilC {\tilde{C}}
\def \tilD {\tilde{D}}
\def \tilE {\tilde{E}}
\def \tilF {\tilde{F}}
\def \tilG {\tilde{G}}
\def \tilH {\tilde{H}}
\def \tilI {\tilde{I}}
\def \tilJ {\tilde{J}}
\def \tilK {\tilde{K}}
\def \tilL {\tilde{L}}
\def \tilM {\tilde{M}}
\def \tilN {\tilde{N}}
\def \tilO {\tilde{O}}
\def \tilP {\tilde{P}}
\def \tilQ {\tilde{Q}}
\def \tilR {\tilde{R}}
\def \tilS {\tilde{S}}
\def \tilT {\tilde{T}}
\def \tilU {\tilde{U}}
\def \tilVV {\tilde{V}}
\def \tilW {\tilde{W}}
\def \tilX {\tilde{X}}
\def \tilY {\tilde{Y}}
\def \tilZ {\tilde{Z}}

\def \tilalpha {\tilde{\alpha}}
\def \tilPhi {\tilde{\Phi}}

\def \barnu {\bar{\nu}}
\def \barrho {\bar{\rho}}
%\DeclareMathOperator{\ker}{ker}
\DeclareMathOperator{\im}{im}

\DeclareMathOperator{\Inn}{Inn}
\DeclareMathOperator{\rel}{rel}
\def \dote {\stackrel{\cdot}=}
%\DeclareMathOperator{\AC}{\textbf{AC}}
\DeclareMathOperator{\cod}{cod}
\DeclareMathOperator{\dom}{dom}
\DeclareMathOperator{\card}{card}
\DeclareMathOperator{\ran}{ran}
\DeclareMathOperator{\textd}{d}
\DeclareMathOperator{\td}{d}
\DeclareMathOperator{\id}{id}
\DeclareMathOperator{\LT}{LT}
\DeclareMathOperator{\Mat}{Mat}
\DeclareMathOperator{\Eq}{Eq}
\DeclareMathOperator{\irr}{irr}
\DeclareMathOperator{\Fr}{Fr}
\DeclareMathOperator{\Gal}{Gal}
\DeclareMathOperator{\lcm}{lcm}
\DeclareMathOperator{\alg}{\text{alg}}
\DeclareMathOperator{\Th}{Th}
%\DeclareMathOperator{\deg}{deg}


% \varprod
\DeclareSymbolFont{largesymbolsA}{U}{txexa}{m}{n}
\DeclareMathSymbol{\varprod}{\mathop}{largesymbolsA}{16}
% \DeclareMathSymbol{\tonm}{\boldsymbol{\to}\textbf{Nm}}
\def \tonm {\bto\textbf{Nm}}
\def \tohm {\bto\textbf{Hm}}

% Category theory
\DeclareMathOperator{\ob}{ob}
\DeclareMathOperator{\Ab}{\textbf{Ab}}
\DeclareMathOperator{\Alg}{\textbf{Alg}}
\DeclareMathOperator{\Rng}{\textbf{Rng}}
\DeclareMathOperator{\Sets}{\textbf{Sets}}
\DeclareMathOperator{\Set}{\textbf{Set}}
\DeclareMathOperator{\Grp}{\textbf{Grp}}
\DeclareMathOperator{\Met}{\textbf{Met}}
\DeclareMathOperator{\BA}{\textbf{BA}}
\DeclareMathOperator{\Mon}{\textbf{Mon}}
\DeclareMathOperator{\Top}{\textbf{Top}}
\DeclareMathOperator{\hTop}{\textbf{hTop}}
\DeclareMathOperator{\HTop}{\textbf{HTop}}
\DeclareMathOperator{\Aut}{\text{Aut}}
\DeclareMathOperator{\RMod}{R-\textbf{Mod}}
\DeclareMathOperator{\RAlg}{R-\textbf{Alg}}
\DeclareMathOperator{\LF}{LF}
\DeclareMathOperator{\op}{op}
\DeclareMathOperator{\Rings}{\textbf{Rings}}
\DeclareMathOperator{\Ring}{\textbf{Ring}}
\DeclareMathOperator{\Groups}{\textbf{Groups}}
\DeclareMathOperator{\Group}{\textbf{Group}}
\DeclareMathOperator{\ev}{ev}
% Algebraic Topology
\DeclareMathOperator{\obj}{obj}
\DeclareMathOperator{\Spec}{Spec}
\DeclareMathOperator{\spec}{spec}
% Model theory
\DeclareMathOperator*{\ind}{\raise0.2ex\hbox{\ooalign{\hidewidth$\vert$\hidewidth\cr\raise-0.9ex\hbox{$\smile$}}}}
\def\nind{\cancel{\ind}}
\DeclareMathOperator{\acl}{acl}
\DeclareMathOperator{\tspan}{span}
\DeclareMathOperator{\acleq}{acl^{\eq}}
\DeclareMathOperator{\Av}{Av}
\DeclareMathOperator{\ded}{ded}
\DeclareMathOperator{\EM}{EM}
\DeclareMathOperator{\dcl}{dcl}
\DeclareMathOperator{\Ext}{Ext}
\DeclareMathOperator{\eq}{eq}
\DeclareMathOperator{\ER}{ER}
\DeclareMathOperator{\tp}{tp}
\DeclareMathOperator{\stp}{stp}
\DeclareMathOperator{\qftp}{qftp}
\DeclareMathOperator{\Diag}{Diag}
\DeclareMathOperator{\MD}{MD}
\DeclareMathOperator{\MR}{MR}
\DeclareMathOperator{\RM}{RM}
\DeclareMathOperator{\el}{el}
\DeclareMathOperator{\depth}{depth}
\DeclareMathOperator{\ZFC}{ZFC}
\DeclareMathOperator{\GCH}{GCH}
\DeclareMathOperator{\Inf}{Inf}
\DeclareMathOperator{\Pow}{Pow}
\DeclareMathOperator{\ZF}{ZF}
\DeclareMathOperator{\CH}{CH}
\def \FO {\text{FO}}
\DeclareMathOperator{\fin}{fin}
\DeclareMathOperator{\qr}{qr}
\DeclareMathOperator{\Mod}{Mod}
\DeclareMathOperator{\Def}{Def}
\DeclareMathOperator{\TC}{TC}
\DeclareMathOperator{\KH}{KH}
\DeclareMathOperator{\Part}{Part}
\DeclareMathOperator{\Infset}{\textsf{Infset}}
\DeclareMathOperator{\DLO}{\textsf{DLO}}
\DeclareMathOperator{\PA}{\textsf{PA}}
\DeclareMathOperator{\DAG}{\textsf{DAG}}
\DeclareMathOperator{\ODAG}{\textsf{ODAG}}
\DeclareMathOperator{\sfMod}{\textsf{Mod}}
\DeclareMathOperator{\AbG}{\textsf{AbG}}
\DeclareMathOperator{\sfACF}{\textsf{ACF}}
\DeclareMathOperator{\DCF}{\textsf{DCF}}
% Computability Theorem
\DeclareMathOperator{\Tot}{Tot}
\DeclareMathOperator{\graph}{graph}
\DeclareMathOperator{\Fin}{Fin}
\DeclareMathOperator{\Cof}{Cof}
\DeclareMathOperator{\lh}{lh}
% Commutative Algebra
\DeclareMathOperator{\ord}{ord}
\DeclareMathOperator{\Idem}{Idem}
\DeclareMathOperator{\zdiv}{z.div}
\DeclareMathOperator{\Frac}{Frac}
\DeclareMathOperator{\rad}{rad}
\DeclareMathOperator{\nil}{nil}
\DeclareMathOperator{\Ann}{Ann}
\DeclareMathOperator{\End}{End}
\DeclareMathOperator{\coim}{coim}
\DeclareMathOperator{\coker}{coker}
\DeclareMathOperator{\Bil}{Bil}
\DeclareMathOperator{\Tril}{Tril}
\DeclareMathOperator{\tchar}{char}
\DeclareMathOperator{\tbd}{bd}

% Topology
\DeclareMathOperator{\diam}{diam}
\newcommand{\interior}[1]{%
  {\kern0pt#1}^{\mathrm{o}}%
}

\DeclareMathOperator*{\bigdoublewedge}{\bigwedge\mkern-15mu\bigwedge}
\DeclareMathOperator*{\bigdoublevee}{\bigvee\mkern-15mu\bigvee}

% \makeatletter
% \newcommand{\vect}[1]{%
%   \vbox{\m@th \ialign {##\crcr
%   \vectfill\crcr\noalign{\kern-\p@ \nointerlineskip}
%   $\hfil\displaystyle{#1}\hfil$\crcr}}}
% \def\vectfill{%
%   $\m@th\smash-\mkern-7mu%
%   \cleaders\hbox{$\mkern-2mu\smash-\mkern-2mu$}\hfill
%   \mkern-7mu\raisebox{-3.81pt}[\p@][\p@]{$\mathord\mathchar"017E$}$}

% \newcommand{\amsvect}{%
%   \mathpalette {\overarrow@\vectfill@}}
% \def\vectfill@{\arrowfill@\relbar\relbar{\raisebox{-3.81pt}[\p@][\p@]{$\mathord\mathchar"017E$}}}

% \newcommand{\amsvectb}{%
% \newcommand{\vect}{%
%   \mathpalette {\overarrow@\vectfillb@}}
% \newcommand{\vecbar}{%
%   \scalebox{0.8}{$\relbar$}}
% \def\vectfillb@{\arrowfill@\vecbar\vecbar{\raisebox{-4.35pt}[\p@][\p@]{$\mathord\mathchar"017E$}}}
% \makeatother
% \bigtimes

\DeclareFontFamily{U}{mathx}{\hyphenchar\font45}
\DeclareFontShape{U}{mathx}{m}{n}{
      <5> <6> <7> <8> <9> <10>
      <10.95> <12> <14.4> <17.28> <20.74> <24.88>
      mathx10
      }{}
\DeclareSymbolFont{mathx}{U}{mathx}{m}{n}
\DeclareMathSymbol{\bigtimes}{1}{mathx}{"91}
% \odiv
\DeclareFontFamily{U}{matha}{\hyphenchar\font45}
\DeclareFontShape{U}{matha}{m}{n}{
      <5> <6> <7> <8> <9> <10> gen * matha
      <10.95> matha10 <12> <14.4> <17.28> <20.74> <24.88> matha12
      }{}
\DeclareSymbolFont{matha}{U}{matha}{m}{n}
\DeclareMathSymbol{\odiv}         {2}{matha}{"63}


\newcommand\subsetsim{\mathrel{%
  \ooalign{\raise0.2ex\hbox{\scalebox{0.9}{$\subset$}}\cr\hidewidth\raise-0.85ex\hbox{\scalebox{0.9}{$\sim$}}\hidewidth\cr}}}
\newcommand\simsubset{\mathrel{%
  \ooalign{\raise-0.2ex\hbox{\scalebox{0.9}{$\subset$}}\cr\hidewidth\raise0.75ex\hbox{\scalebox{0.9}{$\sim$}}\hidewidth\cr}}}

\newcommand\simsubsetsim{\mathrel{%
  \ooalign{\raise0ex\hbox{\scalebox{0.8}{$\subset$}}\cr\hidewidth\raise1ex\hbox{\scalebox{0.75}{$\sim$}}\hidewidth\cr\raise-0.95ex\hbox{\scalebox{0.8}{$\sim$}}\cr\hidewidth}}}
\newcommand{\stcomp}[1]{{#1}^{\mathsf{c}}}

\setlength{\baselineskip}{0.5in}

\stackMath
\newcommand\yrightarrow[2][]{\mathrel{%
  \setbox2=\hbox{\stackon{\scriptstyle#1}{\scriptstyle#2}}%
  \stackunder[0pt]{%
    \xrightarrow{\makebox[\dimexpr\wd2\relax]{$\scriptstyle#2$}}%
  }{%
   \scriptstyle#1\,%
  }%
}}
\newcommand\yleftarrow[2][]{\mathrel{%
  \setbox2=\hbox{\stackon{\scriptstyle#1}{\scriptstyle#2}}%
  \stackunder[0pt]{%
    \xleftarrow{\makebox[\dimexpr\wd2\relax]{$\scriptstyle#2$}}%
  }{%
   \scriptstyle#1\,%
  }%
}}
\newcommand\yRightarrow[2][]{\mathrel{%
  \setbox2=\hbox{\stackon{\scriptstyle#1}{\scriptstyle#2}}%
  \stackunder[0pt]{%
    \xRightarrow{\makebox[\dimexpr\wd2\relax]{$\scriptstyle#2$}}%
  }{%
   \scriptstyle#1\,%
  }%
}}
\newcommand\yLeftarrow[2][]{\mathrel{%
  \setbox2=\hbox{\stackon{\scriptstyle#1}{\scriptstyle#2}}%
  \stackunder[0pt]{%
    \xLeftarrow{\makebox[\dimexpr\wd2\relax]{$\scriptstyle#2$}}%
  }{%
   \scriptstyle#1\,%
  }%
}}

\newcommand\altxrightarrow[2][0pt]{\mathrel{\ensurestackMath{\stackengine%
  {\dimexpr#1-7.5pt}{\xrightarrow{\phantom{#2}}}{\scriptstyle\!#2\,}%
  {O}{c}{F}{F}{S}}}}
\newcommand\altxleftarrow[2][0pt]{\mathrel{\ensurestackMath{\stackengine%
  {\dimexpr#1-7.5pt}{\xleftarrow{\phantom{#2}}}{\scriptstyle\!#2\,}%
  {O}{c}{F}{F}{S}}}}

\newenvironment{bsm}{% % short for 'bracketed small matrix'
  \left[ \begin{smallmatrix} }{%
  \end{smallmatrix} \right]}

\newenvironment{psm}{% % short for ' small matrix'
  \left( \begin{smallmatrix} }{%
  \end{smallmatrix} \right)}

\newcommand{\bbar}[1]{\mkern 1.5mu\overline{\mkern-1.5mu#1\mkern-1.5mu}\mkern 1.5mu}

\newcommand{\bigzero}{\mbox{\normalfont\Large\bfseries 0}}
\newcommand{\rvline}{\hspace*{-\arraycolsep}\vline\hspace*{-\arraycolsep}}

\font\zallman=Zallman at 40pt
\font\elzevier=Elzevier at 40pt

\newcommand\isoto{\stackrel{\textstyle\sim}{\smash{\longrightarrow}\rule{0pt}{0.4ex}}}
\newcommand\embto{\stackrel{\textstyle\prec}{\smash{\longrightarrow}\rule{0pt}{0.4ex}}}

% from http://www.actual.world/resources/tex/doc/TikZ.pdf

\tikzset{
modal/.style={>=stealth’,shorten >=1pt,shorten <=1pt,auto,node distance=1.5cm,
semithick},
world/.style={circle,draw,minimum size=0.5cm,fill=gray!15},
point/.style={circle,draw,inner sep=0.5mm,fill=black},
reflexive above/.style={->,loop,looseness=7,in=120,out=60},
reflexive below/.style={->,loop,looseness=7,in=240,out=300},
reflexive left/.style={->,loop,looseness=7,in=150,out=210},
reflexive right/.style={->,loop,looseness=7,in=30,out=330}
}


\makeatletter
\newcommand*{\doublerightarrow}[2]{\mathrel{
  \settowidth{\@tempdima}{$\scriptstyle#1$}
  \settowidth{\@tempdimb}{$\scriptstyle#2$}
  \ifdim\@tempdimb>\@tempdima \@tempdima=\@tempdimb\fi
  \mathop{\vcenter{
    \offinterlineskip\ialign{\hbox to\dimexpr\@tempdima+1em{##}\cr
    \rightarrowfill\cr\noalign{\kern.5ex}
    \rightarrowfill\cr}}}\limits^{\!#1}_{\!#2}}}
\newcommand*{\triplerightarrow}[1]{\mathrel{
  \settowidth{\@tempdima}{$\scriptstyle#1$}
  \mathop{\vcenter{
    \offinterlineskip\ialign{\hbox to\dimexpr\@tempdima+1em{##}\cr
    \rightarrowfill\cr\noalign{\kern.5ex}
    \rightarrowfill\cr\noalign{\kern.5ex}
    \rightarrowfill\cr}}}\limits^{\!#1}}}
\makeatother

% $A\doublerightarrow{a}{bcdefgh}B$

% $A\triplerightarrow{d_0,d_1,d_2}B$

\def \uhr {\upharpoonright}
\def \rhu {\rightharpoonup}
\def \uhl {\upharpoonleft}


\newcommand{\floor}[1]{\lfloor #1 \rfloor}
\newcommand{\ceil}[1]{\lceil #1 \rceil}
\newcommand{\lcorner}[1]{\llcorner #1 \lrcorner}
\newcommand{\llb}[1]{\llbracket #1 \rrbracket}
\newcommand{\ucorner}[1]{\ulcorner #1 \urcorner}
\newcommand{\emoji}[1]{{\DejaSans #1}}
\newcommand{\vprec}{\rotatebox[origin=c]{-90}{$\prec$}}

\newcommand{\nat}[6][large]{%
  \begin{tikzcd}[ampersand replacement = \&, column sep=#1]
    #2\ar[bend left=40,""{name=U}]{r}{#4}\ar[bend right=40,',""{name=D}]{r}{#5}\& #3
          \ar[shorten <=10pt,shorten >=10pt,Rightarrow,from=U,to=D]{d}{~#6}
    \end{tikzcd}
}


\providecommand\rightarrowRHD{\relbar\joinrel\mathrel\RHD}
\providecommand\rightarrowrhd{\relbar\joinrel\mathrel\rhd}
\providecommand\longrightarrowRHD{\relbar\joinrel\relbar\joinrel\mathrel\RHD}
\providecommand\longrightarrowrhd{\relbar\joinrel\relbar\joinrel\mathrel\rhd}
\def \lrarhd {\longrightarrowrhd}


\makeatletter
\providecommand*\xrightarrowRHD[2][]{\ext@arrow 0055{\arrowfill@\relbar\relbar\longrightarrowRHD}{#1}{#2}}
\providecommand*\xrightarrowrhd[2][]{\ext@arrow 0055{\arrowfill@\relbar\relbar\longrightarrowrhd}{#1}{#2}}
\makeatother

\newcommand{\metalambda}{%
  \mathop{%
    \rlap{$\lambda$}%
    \mkern3mu
    \raisebox{0ex}{$\lambda$}%
  }%
}

%% https://tex.stackexchange.com/questions/15119/draw-horizontal-line-left-and-right-of-some-text-a-single-line
\newcommand*\ruleline[1]{\par\noindent\raisebox{.8ex}{\makebox[\linewidth]{\hrulefill\hspace{1ex}\raisebox{-.8ex}{#1}\hspace{1ex}\hrulefill}}}

% https://www.dickimaw-books.com/latex/novices/html/newenv.html
\newenvironment{Block}[1]% environment name
{% begin code
  % https://tex.stackexchange.com/questions/19579/horizontal-line-spanning-the-entire-document-in-latex
  \noindent\textcolor[RGB]{128,128,128}{\rule{\linewidth}{1pt}}
  \par\noindent
  {\Large\textbf{#1}}%
  \bigskip\par\noindent\ignorespaces
}%
{% end code
  \par\noindent
  \textcolor[RGB]{128,128,128}{\rule{\linewidth}{1pt}}
  \ignorespacesafterend
}

\mathchardef\mhyphen="2D % Define a "math hyphen"

\def \QQ {\quad}
\def \QW {​\quad}

\makeindex
\author{wu}
\date{\today}
\title{On \(f\)-Generic Types in Presburger Arithmetic}
\hypersetup{
 pdfauthor={wu},
 pdftitle={On \(f\)-Generic Types in Presburger Arithmetic},
 pdfkeywords={},
 pdfsubject={},
 pdfcreator={Emacs 28.0.92 (Org mode 9.6)}, 
 pdflang={English}}
\begin{document}

\maketitle
\tableofcontents

\section{Definable types and f-generics in presburger arithmetic}
\label{sec:orgb16624a}
\href{https://people.math.osu.edu/conant.38/Math/presburger\_note.pdf}{Link}
\subsection{Definable groups and \texorpdfstring{\(f\)}{f}-generics}
\label{sec:org093ff66}
Presburger arithmetic: the complete first-order theory of the ordered group of
integers \((\Z,+,<,0)\).

Let \(T\) be a complete theory, with a monster model \(M\). We also work with a larger monster
model \(M^*\) in which we can take realizations of global types over \(M\).

Suppose \(G=G(M)\) is a definbale group in \(T\), let \(S_G(M)\) denote the space of global
types containing the formula defining \(G\). Given \(p\in S_G(M)\) and \(g\in G\), we let \(gp\)
denote the translate \(\{\varphi(g^{-1}x):\varphi(x)\in p\}\) of \(p\).

\begin{definition}[]
Let \(p\in S_G(M)\) be a global \(G\)-type.
\begin{enumerate}
\item \(p\) is \textbf{definable (over \(G\))} if, for any formula \(\varphi(\barx,\bary)\) there is a
formula \(d_p[\varphi](\bary)\) over \(G\) s.t., for any \(\barb\in G\), \(\varphi(\barx,\barb)\in p\)
iff \(G\vDash d_p[\varphi](\barb)\)
\item \(p\) is \textbf{\(f\)-generic} if, for every formula \(\phi(x)\in p\) there is a small
model \(M_0\) s.t. no translate \(\phi(gx)\) of \(\phi(x)\) forks over \(M_0\)
\item \(p\) is \textbf{strongly \(f\)-generic} if there is a small model \(M_0\) s.t. no translate \(gp\)
of \(p\) forks over \(M_0\)
\item \(p\) is \textbf{definably \(f\)-generic} if there is a small model \(M_0\) s.t. every translate
\(gp\) is definable over \(M_0\)
\end{enumerate}
\end{definition}

\subsection{End extensions of discrete orders}
\label{sec:org0438f2b}
Assume \(\call\) contains a symbol < and \(T\) extends the theory of linear orders. We say
that \(T\) is \textbf{definably complete} if any nonempty definable subset
of \(M\), with an upper bound in \(M\), has a least upper bound in \(M\), and similarly for
lower bounds. Note that this does not depend on the model \(M\).

If \(T\) is definably complete, and we further assume that \(M\) is discretely ordered by <,
then it follows that definable subsets of \(M\) contain their least upper bound and greatest
lower bound. We will say \(T\) is \textbf{discretely ordered} to indicate that the ordering < on \(M\) is
discrete.

In a totally ordered structure, algebraic closure and definable closure coincide.

Given a tuple \(\bara\in(M^*)^n\), we let \(M(\bara)=\dcl(M\bara)\).

\begin{definition}[]
Given subsets \(A\subseteq B\) of \(M^*\), we say \(B\) is an \textbf{end extension} of \(A\) if, for
all \(b\in B\setminus A\), either \(b<a\) for all \(a\in A\) or \(b>a\) for all \(a\in A\).
\end{definition}

\begin{lemma}[]
\label{0.3}
Suppose \(T\) is discretely ordered and definably complete. Fix a non-isolated
type \(p\in S_n(M)\) and a realization \(\bara\) in \(M^*\). If \(M(\bara)\) is not an end
extension of \(M\) then
\begin{enumerate}
\item \(p\) is not definable
\item \(p\) has at least two distinct coheirs to \(M^*\)
\end{enumerate}
\end{lemma}

\begin{proof}
Since \(M(\bara)\) is not an end extension of \(M\), we may fix an \(M\)-definable
function \(f:(M^*)^n\to M^*\), and \(m_1,m_2\in M\) s.t. \(f(\bara)\notin M\) and \(m_1<f(\bara)<m_2\). Define
the upwards closed set
\begin{equation*}
X=\{m\in M:p\vDash f(\bara)<m\}
\end{equation*}
Then \(m_1\) and \(m_2\) witness that \(X\) is nonempty and not all of \(M\). If \(X\) has a
minimal element \(m_0\) and \(m_0^-\) is the immediate predecessor of \(m_0\) in \(M\), then we
must have \(m_0^-\le f(\bara)<m_0\) and so \(f(\bara)=m_0^-\in M\), which is a contradiction. So \(X\)
has no minimal element, and therefore cannot be \(M\)-definable. This proves part 1.

Now define
\begin{equation*}
C=\{c\in M^*:m<c<m'\text{ for all }m\in M\setminus X\text{ and }m'\in X\}
\end{equation*}
Then \(f(\bara)\in C\), and so \(C\neq\emptyset\). We define the following partial types over \(M^*\):
\begin{align*}
q_1&=p\cup\{m<f(\barx)<c:m\in M\setminus X,c\in C\}\\
q_2&=p\cup\{c<f(\barx)<m:c\in C,m\in X\}
\end{align*}
Note that \(q_1\) and \(q_2\) are distinct since \(C\neq\emptyset\). If we can show that they are each
finitely satisfiable in \(M\), then they will extend to distinct coheirs of \(p\), which proves
part 2. So we show \(q_1\) is finitely satisfiable in \(M\).

Fix a formula \(\varphi(\barx)\in p\) and some \(m\in M\setminus X\) (which exists since \(X\) is not all
of \(M\)) . Set
\begin{equation*}
A=\{m'\in f(\varphi(M^n)):m<m'\}
\end{equation*}
Then \(A\) is an \(M\)-definable subset of \(M\), which is nonempty since \(\bara\in A(M^*)\).
Since \(A\) is bounded below by \(m\), we may fix a minimal element \(m_0\in A\). By
elementarity, \(m_0\) is the minimal element of \(A(M^*)\). In particular, \(m_0<f(\bara)\), and
so \(m_0\in M\setminus X\). In particular, \(m_0<f(\bara)\), and so \(m_0\in M\setminus X\). By definition
of \(A\), \(m_0=f(\bara')\) for some \(\bara'\in M^n\) s.t. \(M\vDash\varphi(\bara')\). Altogether, we
have \(M\vDash\varphi(\bara')\) and \(m<f(\bara')<c\) for any \(c\in C\).
\end{proof}

Suppose \(T\) is discretely ordered and definably complete. If, moreover, \(\dcl(\emptyset)\) is
nonempty, then \(T\) has definable Skolem functions by picking out either the maximal element of
a definable set or the least element greater than some \(\emptyset\)-definable constant. It follows
that \(M(\bara)\) is the unique prime model over \(M\bara\). \label{Problem3}
\subsection{Presburger arithmetic}
\label{sec:org4923376}
Let \(T=\Th(\Z,+,<,0)\). Let \(G\) denote a sufficiently saturated model of \(T\), and
let \(G^*\) denote a larger elementary extension of \(G\), which is sufficiently saturated
w.r.t. \(G\). We treat types over \(G\) as \emph{global types}, but use \(G^*\) as an even larger
monster model in which we can realize such types.

Note that \(T\) satisfies the properties discussed above: it is discretely ordered and definably
complete, with \(\dcl(\emptyset)\) nonempty. Therefore, for \(\bara\in G^*\), \(G(\bara)\) is the prime
model over \(G\bara\). Recall that \(T\) has quantifier elimination in the expanded language
\(\call^*=\{+,<,0,1,(D_n)_{n<\omega}\}\) where \(D_n\) is a unary predicate interpreted as \(n\Z\).
Consequently, given \(\bara\in G^*\), \(G(\bara)\) is the divisible hull of the subgroup
of \(G^*\) generated by \(G\bara\).  \label{Problem4}

Given \(a\in G^*\) and \(n>0\), let \([a]_n\in\{0,1,\dots,n-1\}\) be the unique remainder of \(a\)
modulo \(n\). Given \(\bark\in\Z^n\), we let \(s_{\bark}(\barx)\) denote the definable function
\(\barx\mapsto k_1x_1+\dots+k_nx_n\)

\begin{proposition}[]
\label{0.4}
\begin{enumerate}
\item Let \(G_0\prec G\) be a small model, and fix \(a,b\in G\)
\begin{enumerate}
\item If \(G_0<a<b\) then there is some \(c\in G\) s.t. \(b<c\) and \(a\equiv_{G_0}c\)
\item If \(a<b<G_0\) then there is some \(c\in G\) s.t. \(c<a\) and \(b\equiv_{G_0}c\)
\end{enumerate}
\item For any \(p\in S_n(G)\) and \(\bara\vDash p\), if \(G(\bara)\) is not an end extension of \(G\) then
there are \(h_1,h_2\in G\) and \(\bark\in\Z^n\) s.t. \(h_1<s_{\bark}(\bara)<h_2\)
and \(s_{\bark}(\bara)\notin G\).
\end{enumerate}
\end{proposition}

\begin{proof}
\begin{enumerate}
\item By quantifier elimination and saturation of \(G\) it is enough to fix an integer \(N>0\) and
find \(c\in G\) s.t. \(b<c\) and \([c]_n=[a]_n\) for all \(0<n\le N\). To find such an element,
simply note that \(\bigcap_{0<n\le N}nG+[a]_n\) is nonempty as it contains \(a\) and is therefore a
single coset \(mG+r\) for some \(m,r\in\Z\)(chinese remainder theorem). So we may
choose \(c=b-[b]_m+m+r\)
\item By assumption, there is \(b\in\dcl(G\bara)\setminus G\) and \(h_1',h_2'\in G\) s.t. \(h_1'<b<h_2'\). By the
description of definable closure in Presburger arithmetic, there are
integers \(r\in\Z^+\), \(\bark\in\Z^n\) and some \(h_0\in G\) s.t. \(rb=s_{\bark}(\bara)+h_0\). Now
let \(h_i=rh_i'-h_0\).
\end{enumerate}
\end{proof}
\subsection{Definable types in Presburger arithmetic}
\label{sec:org014669d}
Consider the situation where \(G\) is the monster model \(M\), and the definable group
is \(G^n=\Z^n(G)\), for a fixed \(n>0\), under coordinate addition. In particular.

\begin{definition}[]
A type \(p\in S_n(G)\) is \textbf{algebraically independent} if for all
(some) \(\bara\vDash p\), \(a_i\notin G(\bara_{\neq i})\) for all \(1\le i\le n\).
\end{definition}

\begin{lemma}[]
\label{0.6}
Suppose \(p\in S_n(G)\) is algebraically independent and for all (some) \(\bara\vDash p\), \(G(\bara)\)
is an end extension of \(G\). Then \(p\) is definable over \(\emptyset\).
\end{lemma}

\begin{proof}
Let \(\Z^n_*\) denote \(\Z^n\setminus\{0\}\). By quantifier elimination, it suffices to give definitions  for
atomic formulas of the following forms:
\begin{itemize}
\item \(\varphi_1(\barx,\bary):=(s_{\bark}(\barx)=t(\bary))\), where \(\bark\in\Z^n_*\) and \(t(\bary)\) is a
term in variables \(\bary\).
\item \(\varphi_2(\barx,\bary):=(s_{\bark}(\barx)>t(\bary))\), where \(\bark\in\Z^n_*\) and \(t(\bary)\) is a
term in variables \(\bary\)
\item \(\varphi_3(\barx,\bary):=([s_{\bark}(\barx)+t(\bary)]_m=0)\), where \(\bark\in\Z^n_*\), \(m\in\Z^+\),
and \(t(\bary)\) is a term in variables \(\bary\).
\end{itemize}

Fix \(\bara\vDash p\) and fix \(\bark\in\Z_*^n\). Since \(p\) is algebraically independent, it follows
that \(s_{\bark}(\bara)\notin G\). Since \(G(\bara)\) is an end extension of \(G\), we may
partition \(\Z^n_*=S^+\cup S^-\) where
\begin{equation*}
S^+=\{\bark:s_{\bark}(\bara)>G\} \quad\text{ and }\quad S^-=\{\bark:s_{\bark}(\bara)<G\}
\end{equation*}
Note that \(S^+\) and \(S^-\) depends only on \(p\), and not choice of realization \(\bara\).
Moreover, for any \(\bark\in\Z^n\) and \(m>0\), the integer \([s_{\bark}(\bara)]_m\in\{0,\dots,m-1\}\)
depends only on \(p\). We now give the following definitions for \(p\) (note that they are
formulas over \(\emptyset\)):
\begin{align*}
&d_p[\varphi_1](\bary):=(y_1\neq y_1)\\
&d_p[\varphi_2](\bary):=
\begin{cases}
y_1=y_1&\bark\in S^+\\
y_1\neq y_1&\bark\in S^-
\end{cases}\\
&d_p[\varphi_3](\bary):=([t(\bary)+[s_{\bark}(\bara)]_m]_m=0)
\end{align*}
\end{proof}

\begin{theorem}[]
\label{0.7}
Given \(p\in S_n(G)\), TFAE
\begin{enumerate}
\item \(p\) is definable over \(G\)
\item \(p\) has a unique coheir to \(G^*\)
\item For any (some) \(\bara\vDash p\), \(G(\bara)\) is an end extension of \(G\)
\end{enumerate}
\end{theorem}

\begin{proof}
\(1\Rightarrow 2\): True for any NIP theory

\(2\Rightarrow 3\): \ref{0.3}

\(3\Rightarrow 1\): We may assume \(p\) is non-isolated. We proceed by induction on \(n\). If \(n=1\)
then \(p\) is algebraically independent since it is non-isolated, and so we apply Lemma \ref{0.6}.
Assume the result for \(n'<n\) and fix \(p\in S_n(G)\). If \(p\) is algebraically independent then
we apply Lemma \ref{0.6}. So assume, W.L.O.G., that we have \(\bara\vDash p\)
with \(a_n\in G(\bara_{<n})\). Let \(q=\tp(\bara_{<n}/G)\in S_{n-1}(G)\). By
assumption, \(G(\bara_{<n})=G(\bara)\) is an end extension of \(G\), and so \(q\) is definable
by induction. Fix a \(G\)-definable function \(f:(G^*)^{n-1}\to G^*\) s.t. \(f(\bara{<n})=a_n\).
Fix a formula \(\varphi(\barx,\bary)\) and define
\begin{equation*}
\psi(\barx_{<n},\bary):=\varphi(\barx_{<n},f(\barx_{<n}),\bary)
\end{equation*}
Let \(d_q[\psi](\bary)\) be an \(\call_G\)-formula s.t., for
any \(\barb\in G\), \(\psi(\barx_{<n},\barb)\in q\) iff \(G\vDash d_q[\psi](\barb)\). Then for
any \(\barb\in G\), we have
\begin{equation*}
\varphi(\barx,barb)\in p\Leftrightarrow G^*\vDash\varphi(\bara,\barb)\Leftrightarrow G^*\vDash\psi(\bara_{<n},\barb)\Leftrightarrow
G\vDash d_q[\psi](\barb)
\end{equation*}
\end{proof}
\subsection{\texorpdfstring{\(f\)}{f}-generics in Presburger arithmetic}
\label{sec:org12c6755}
\begin{proposition}[]
Any \(f\)-generic \(p\in S_n(G)\) is algebraically independent
\end{proposition}

\begin{proof}
Suppose \(p\) is not algebraically independent. W.L.O.G., fix \(\bara\vDash p\)
with \(a_n\in G(\bara_{<n})\). Then there are \(r,k_1,\dots,k_{n-1}\in\Z\) and \(b\in G\)
s.t. \(ra_n=b+k_1a_1+\dots+k_{n-1}a_{n-1}\). Consider the
formula \(\phi(\barx;b):=rx_n=b+k_1x_1+\dots+k_{n-1}x_{n-1}\), and note that \(\phi(\barx;b)\in p\). We fix a
small model \(G_0\prec G\), and find a translate of \(\phi(\barx;b)\) that forks over \(G_0\).

Pick \(c\in rG\) s.t. \(b-c\notin G_0\), and set \(g=\frac{c}{r}\). Let \(\barg=(0,\dots,0,g)\) and
set \(\psi(\barx;b,\barg):=\phi(\barx+\barg;b)\). By construction, we may find
automorphism \(\sigma_i\in\Aut(G/G_0)\) s.t. \(\sigma_i(b-c)\neq\sigma_j(b-c)\) for all \(i\neq j\). (\(b-c\) is
not almost \(G_0\)-definable, therefore it has infinite orbits)
Setting \(b_i=\sigma_i(b)\) and \(\barg_i=\sigma_i(\barg)\), we have that \(\{\psi(\barx;b_i,\barg_i):i<\omega\}\) is
2-inconsistent. So \(\psi(\barx;b,\barg)\) forks over \(G_0\)
\end{proof}

\begin{theorem}[]
\label{0.10}
If \(p\in S_n(G)\) is algebraically independent, TFAE
\begin{enumerate}
\item \(p\) is \(f\)-generic
\item \(p\) is strongly \(f\)-generic
\item \(p\) is definable \(f\)-generic
\item \(p\) is definable over \(G\)
\item \(p\) is definable over \(\emptyset\)
\item For any (some) \(\bara\vDash p\), \(G(\bara)\) is an end extension of \(G\)
\end{enumerate}
\end{theorem}

\begin{proof}
\(4\Leftrightarrow 6\): \ref{0.7}

\(6\Rightarrow 5\): \ref{0.6}

\(5\Rightarrow 4\): trivial

\(1\Rightarrow 6\): Suppose \(G(\bara)\) is not an end extension of \(G\), and fix \(\bark\in\Z^n\)
and \(h_1,h_2\in G\) s.t. \(s_{\bark}(\bara)\notin G\) and \(h_1<s_{\bark}(\bara)<h_2\). Consider the
formula \(\phi(\barx;h_1,h_2):=h_1<s_{\bark}(\barx)<h_2\), and note that \(\phi(\barx;h_1,h_2)\in p\). We fix
a small model \(G_0\prec G\), and find a translate of \(\phi(\barx;h_1,h_2)\) that forks over \(G_0\).
W.L.O.G., assume \(b>0\) and also \(h_1>0\). Let \(k_i\) be a nonzero element of the
tuple \(\bark\). By saturation of \(G\), we may find \(g\in G\) s.t. \(k_ig>c\) for all \(c\in G_0\).
Let \(\barg\in G^n\) be s.t. \(g_j=0\) for all \(j\neq i\) and \(g_i=g\). For \(t\in\{1,2\}\),
set \(c_t=h_t+k_ig\in G\) . Then \(\phi(\barx-\barg;h_1,h_2)\) is equivalent
to \(c_1<s_{\bark}(\barx)<c_2\). Since \(c<c_1\) for all \(c\in G_0\), by Proposition \ref{0.4},
that \(\phi(\barx-\barg;h_1,h_2)\) forks over \(G_0\), as desired. (By increase \(g\), we can show
that \(\phi(\barx;h_1,h_2;g_i)\)) is 2-inconsistent or something.
So \(p\) is not \(f\)-generic.

\(6\Rightarrow 3\): Suppose \(G(\bara)\) is an end extension of \(G\). For any \(\barg\in G^n\), we
have \(G(\bara)=G(\barg+\bara)\), and \(\barg p\) is still algebraically independent. Therefore,
for any \(\barg\in G^n\), we use Lemma \ref{0.6} to conclude that \(\barg p\) is definable over \(\emptyset\).
\end{proof}

\section{Introduction and Preliminaries}
\label{sec:org278c7a9}
\subsection{Introduction}
\label{sec:orgee248e1}
Marcin Petrykowski gave a nice description of \(f\)-generic types in groups \((R,+)\times(R,+)\)
with \((R,<,+,\cdot)\) with \((R,<,+,\cdot)\) an o-minimal expansion of real closed field. An analogs
question is: What are the \(f\)-generic types of \(G^n\), the product of \(n\) copies of ordered
additive groups \((\Z,+,<)\) of integers.

Let \(M\) be an elementary extension of \((\Z,+,<,0)\), \(\M\succ M\) a monster model. \(G\) denotes
the additive group \((\M,+)\), \(S_G(M)\) the space of complete types over \(M\) extending the
formula '\(x\in G\)'. \(G^0\) is the definable connected component of \(G\). Namely, \(G^0\) is the
intersection of all definable subgroups of \(G\) with finite index.

Let \(L_n\) denote the space of homogeneous \(n\)-ary \(\Q\)-linear functions. For \(f,g\in L_n\)
and \(\alpha,\beta\in\M^n\) s.t. \(\alpha\in\dom(f)\) and \(\beta\in\dom(g)\), by \(f(\alpha)\ll_Mg(\beta)\) we mean that for
all \(a,b\in M\) and \(k,l\in\N^+\), \(kf(\alpha)+a<lg(\beta)+b\). By \(f(\alpha)\sim_Mg(\beta)\) we mean that
neither \(f(\alpha)\ll_Mg(\beta)\) nor \(g(\beta)\ll_Mf(\alpha)\). Let \(f_0,\dots,f_m\in L_n\), we
say \(0\ll_Mf_1(\alpha)\ll_M\dots\ll_Mf_m(\alpha)\) is a maximal positive chain of \(\alpha\) over \(M\) if for any \(g\in L_n\)
with \(g(\alpha)>0\), neither \(f_m(\alpha)\ll_Mg(\alpha)\) nor \(g(\alpha)\ll_Mf_1(\alpha)\)

\begin{theorem}[]
Let \(M\succ\Z\), \(\alpha=(\alpha_1,\dots,\alpha_n)\in(G^n)^0\). Then there exists a finite subset \(\{f_0,\dots,f_m\}\subset L_n\)
s.t. \(f_0(\alpha)=0\ll_Mf_1(\alpha)\ll_M\dots\ll_Mf_m(\alpha)\) is the maximal positive chain of \(\alpha\) over \(M\). If \(\alpha\)
realizes an \(f\)-generic type \(p\in S_{G^n}(M)\) then for
every \(\beta\in G^0\), \(p=\tp(\alpha,\beta/M)\in S_{G^{n+1}}(M)\) is an \(f\)-generic type iff one of the
following holds:
\begin{enumerate}
\item \(f_m(\alpha)\ll_M\beta\) or \(\beta\ll_M-f_m(\beta)\)
\item there is \(i\) with \(0\le i<m\) and \(g\in L_n\) s.t. \(f_i(\alpha)\ll_M\epsilon(\beta-g(\alpha))\ll_Mf_{i+1}(\alpha)\)
where \(\epsilon=\pm 1\)
\item there is \(i\) with \(1\le i\le m\) and \(g\in L_n\) s.t. for all \(h\in L_n\) with \(h(\alpha)\sim_Mf_i(\alpha)\)
there is an irrational number \(r_h\in\R\setminus\Q\) s.t. \(q_1h(\alpha)<\beta-g(\alpha)<q_2h(\alpha)\) for all \(q_1,q_2\in\Q\)
with \(q_1<r_h<q_2\)
\end{enumerate}
\end{theorem}
\subsection{Preliminaries}
\label{sec:org010ea45}
\begin{definition}[]
\begin{enumerate}
\item A definable subset \(X\subseteq G\) is  \textbf{\(f\)-generic} if for some/any model \(M\) over which \(X\) is
defined and any \(g\in G\), \(gX\) does not divide over \(M\). Namely, for
any \(M\)-indiscernible sequence \((g_i:i<\omega)\) with \(g=g_0\), \(\{g_iX:i<\omega\}\)  is consistent.
\end{enumerate}
\end{definition}

\begin{remark}
The class of all non-weakly generic formulas forms an ideal. So any weakly generic
type \(p\in S_G(M)\) has a global extension \(\barp\in S_G(\M)\) which is weakly generic.
\end{remark}

\(T\) is said to be (or have) NIP if for any indiscernible sequence \((b_i:i<\omega)\)
formula \(\psi(x,y)\) and \(a\in\M\), there is an eventual truth value of \(\psi(a,b_i)\) as \(i\to\infty\).

A type definable over \(A\) subgroup \(H\le G\) has bounded index
if \(\abs{G/H}<2^{\abs{T}+\abs{A}}\). For groups definbale in NIP structures, the smallest
type-definable subgroup \(G^{00}\) exists. Namely, the intersection of all type-definable
subgroup of bounded index still has bounded index. We call \(G^{00}\) the \textbf{type-definable
connected component} of \(G\). Another model theoretic invariant is \(G^0\), called the
definably-connected component of \(G\), which is the intersection of all definable subgroup
of \(G\) of finite index.

The Keisler measure over \(M\) on \(X\), with \(X\) a definable set over \(M\), is a finitely
additive measure on the Boolean algebra of definable subsets of \(X\) over \(M\).

A definable group \(G\) is \textbf{definably amenable} if it admits a global (left) \(G\)-invariant
probability Keisler measure

\begin{fact}[]
\label{fact1.3}
Assuming NIP, a nip group \(G\) is definably amenable iff it admits a global type \(p\in S_G(\M)\)
with bounded \(G\)-orbit.
\end{fact}

\begin{fact}[]
\label{fact1.4}
For a definable amenable NIP group \(G\), we have
\begin{itemize}
\item weakly generic definable subsets, formulas and types coincide with \(f\)-generic definable
subsets, formulas, and types, respectively
\item \(p\in S_G(\M)\) is \(f\)-generic iff it has bounded \(G\)-orbit
\item \(p\in S_G(\M)\) is \(f\)-generic iff it is \(G^{00}\)-invariant
\item A type-definable subgroup \(H\) fixing a global \(f\)-generic type is exactly \(G^{00}\)
\end{itemize}
\end{fact}

\begin{remark}
Assuming that \(G\) is definable amenable NIP group
\end{remark}

Assume that \(T=\Th(\Z,+,\{D_n\}_{n\in\N^+},<,0)\) is the first order theory of integers in Presburger
language \(L_{Pres}=(+,\{D_n\}_{n\in\N^+},<,0)\) where each \(D_n\) is a unary predicate symbol
for the set of elements divisible by \(n\). \(\M\) is the monster model of \(T\).

\(T\) has quantifier elimination and cell decomposition.

\begin{definition}[]
We call a function \(f:X\subseteq M^m\to M\) \textbf{linear} if there is a constant \(\gamma\in M\) and
integers \(a_i\), \(0\le c_i<n_i\) for \(i=1,\dots,m\) s.t. \(D_{n_i}(x_i-c_i)\) and
\begin{equation*}
f(x)=\sum_{1\le i\le m}a_i(\frac{x_i-c_i}{n_i})+\gamma
\end{equation*}
for all \(x=(x_1,\dots,x_m)\in X\). We call \(f\) \textbf{piecewise linear} if there is a finite partition \(\calp\)
of \(X\) s.t. all restrictions \(f|_A\), \(A\in\calp\) are linear.
\end{definition}

Note that \(x\in\dom(f)\) iff \(D_{n_i}(x_i-c_i)\) for each \(i\).

\begin{definition}[]
\begin{itemize}
\item A (0)-cell is a point \(\{a\}\subset M\).
\item An (1)-cell is a set with infinite cardinality of the form
\begin{equation*}
\{x\in M|a\Box_1x\Box_2b,D_n(x-c)\}
\end{equation*}
with \(a,b\in M\), integers \(0\le c<n\) and \(\Box_i\) either \(\le\) or no condition.
\item Let \(i_j\in\{0,1\}\) for \(j=1,\dots,m\) and \(x=(x_1,\dots,x_m)\). A \((i_1,\dots,i_m,1)\)-cell is a set \(A\)
of the form
\begin{equation*}
\{(x,t)\in M^{m+1}\mid x\in D,f(x)\Box_1t\Box_2g(x), D_n(t-c)\}
\end{equation*}
with \(D=\pi_m(A)\) an \((i_1,\dots,i_m)\)-cell. \(f,g:D\to M\) linear functions, \(\Box_i\) either \(\le\)
or no condition and integers \(0\le c<n\) s.t. the cardinality of the
fibers \(A_x=\{t\in M\mid (x,t)\in A\}\) can not be bounded uniformly in \(x\in D\) by an integers.
\item An \((i_1,\dots,i_m,0)\)-cell is a set \(A\) of the form
\begin{equation*}
\{(x,t)\in M^{m+1}\mid x\in D,t=g(x)\}
\end{equation*}
with \(g:D\to M\) a linear function and \(D\in M^m\) an \((i_1,\dots,i_m)\)-cell
\end{itemize}
\end{definition}

\begin{fact}[\cite{10.2307/4147737}Cell Decomposition Theorem]
\label{1.8}
Let \(X\subset M^m\) and \(f:X\to G\) be definable. Then there exists a finite partition \(\calp\) of \(X\)
into cells, s.t. the restriction \(f|_A:A\to M\) is linear for each cell \(A\in\calp\). Moreover,
if \(X\) and \(f\) are \(S\)-definable, then the parts \(A\) can be taken \(S\)-definable.
\end{fact}

By the Cell Decomposition Theorem, we conclude that every definable subset of \(M^n\) is a finite
union of cells. So every definable subset \(X\subseteq M\) is a finite union of points and intervals mod
some \(n\in\N\). This implies that \(T\) has NIP. \label{Problem2}

From now on, we assume that \(G=(\M,+)\) is the additive group of the Presburger arithmetic.
Namely, \(G\) is defined by the formula ``\(x=x\)'', \(G=\M\) as a set, and \(G(M)=M\) for
any \(M\prec\M\). For any \(n\)-tuple \(x=(x_1,\dots,x_n)\), by \(D_m(x)\) we mean \(\bigwedge_{1\le i\le n}D_m(x_i)\).
For any \(\alpha\in\M\), and \(A\subseteq\M\), by \(\alpha>A\) we mean \(\alpha>a\) for all \(a\in\acl(A)\).

\(\dcl(A)=\acl(A)\) since \(\M\) is a linear order  \label{Problem1}
\wu{
If \(a\in \acl(A)\), then suppose \(\varphi(\M)\) is finite, then \(\varphi(\M)\) lies in some finite interval
in \(A\)
}

\begin{fact}[]
For every \(n\in\N\)
\begin{itemize}
\item \(G^n\) is definably amenable;
\item the type-definable connected component of \(G^n\) is \(\bigcap_{m\in\N^+}D_m(\M^n)\)
\end{itemize}
\end{fact}

\begin{proof}
Let \(x=(x_1,\dots,x_n)\) be an \(n\)-tuple. Let \(\Pi(x)\) be the partial type of form
\begin{align*}
\{x_1>\M\}&\wedge\{x_2>\dcl(\M,x_1)\}\wedge\dots\\
&\wedge\{x_n>\dcl(\M,x_1,\dots,x_{n-1})\}\wedge\{D_m(x):m\in\N^+\}
\end{align*}
By the cell decomposition theorem, and induction on \(n\), it is easy to see that \(\Pi\) determines a
unique type \(p\in S_{G^n}(\M)\). Moreover, \(\Pi\) is invariant under \(\bigcap_{m\in\N^+}D_m(\M^n)\).

Since \(D_m(\M^n)\) is a definable subgroup of \(G^n\) of finite index, \(G^{00}\le\bigcap_{m\in\N^+}D_m(\M^n)\).
Thus \(p\) is \(G^{00}\)-invariant and hence has a bounded orbit.

By Fact \ref{fact1.3} \(G^n\) is definably amenable and \(G^{n00}=\bigcap_{m\in\N^+}D_m(\M^n)\)
\end{proof}

\begin{corollary}[]
\(G^{n0}=G^{n00}\) for all \(n\in\N^+\).
\end{corollary}

\begin{remark}
\label{1.11}
\begin{itemize}
\item \(G^0\) is a densely linear ordered divisible abelian group, hence is isomorphic to an ordered
vector space over \(\Q\).
\item For every \(n\in\N^+\), \((G^0)^n=(G^n)^0\)
\end{itemize}
\end{remark}

\begin{proof}
divisibility and abelian is trivial. For any \(a,b\in G^0\), \(\frac{a+b}{2}\in G^0\).
\end{proof}

\begin{fact}[]
\label{1.12}
Suppose that \(f\) is an \(M\)-definable function from \(X\subseteq\M^n\) to \(Y\subseteq\M\). Then for
any \(\alpha\in(G^0)^n\) there are \(q_1,\dots,q_n\in\Q\) and \(a\in M\) s.t. \(f(\alpha)=q_1\alpha_1+\dots+q_n\alpha_n+a\)
\end{fact}

\begin{proof}
By Cell Decomposition we may assume \(f\) is linear. Then apply remark \ref{1.11},
\(\alpha\in(G^n)^0\), therefore \(\alpha_i\in G^0\) and we don't need the \(c_i\).
\end{proof}

\begin{definition}[]
We call the function \(f\) of the form \(q_1x_1+\dots+q_nx_n+a\) with \(q_1,\dots,q_n\in\Q\) and \(a\in M\) an
\textbf{\(n\)-ary \(\Q\)-linear function} over \(M\). If \(a=0\), we call \(f\) a
\textbf{homogeneous} \(n\)-ary \(\Q\)-linear function. By \(L_n(M)\) we mean the space of
all \(n\)-ary \(\Q\)-linear functions over \(M\), and \(L_n\) the space of all
homogeneous \(n\)-ary \(\Q\)-linear functions.
\end{definition}

It is easy to see that any \(f\in L_n(M)\) is \(M\)-definable, and there is a natural number \(m\)
s.t. \(D_m(\M^n)\subseteq\dom(f)\) (common factor). In particular, \((G^0)^n\subseteq\dom(f)\). By Fact \ref{1.8} and Fact \ref{1.12} we
conclude that:
\begin{corollary}[]
If \(\alpha=(\alpha_1,\dots,\alpha_n)\in(G^0)^n\), then for any \(\phi(x_1,\dots,x_n)\in\tp(\alpha/M)\) there is a
formula \(\psi(x_1,\dots,x_n)\in\tp(\alpha/M)\) of the form
\begin{equation*}
\theta(x_1,\dots,x_{n-1})\wedge D_m(x_n)\wedge(f_1(x_1,\dots,x_{n-1})\Box_1x_n\Box_2f_2(x_1,\dots,x_{n-1}))
\end{equation*}
with \(m\in\N\), \(\theta(M)\) a cell, \(f_i\in L_{n-1}(M)\), and \(\Box_i\) either \(\le\) or no condition,
s.t. \(M\vDash\forall x(\psi(x)\to\phi(x))\).
\end{corollary}

\begin{remark}
There are only 2 \(f\)-generic types contained in every coset of \(G^0\). More precisely, for any
model \(M\),
\begin{align*}
p^+(x)&=\{D_n(x)\mid n\in\N^+\}\cup\{x>a\mid a\in M\}\\
p^-(x)&=\{D_n(x)\mid n\in\N^-\}\cup\{x<a\mid a\in M\}
\end{align*}
Then every \(f\)-generic type over \(M\) is one of \(G(M)\)-translates of \(p^+\) or \(p^-\).
\end{remark}
\section{Main results}
\label{sec:orge23d08b}
\subsection{The \texorpdfstring{\(f\)}{f}-generics of \(G^2\)}
\label{sec:orgec2c047}
Let \(\M\) be the saturated model of \(\Th(\Z,+,D_n,<,0,1)_{n\in\N+}\), \(T\) the theory of
Presburger Arithmetic.

\begin{proposition}[]
For any \(M\succ\Z\), the \(f\)-generic type \(\tp(\alpha,\beta/M)\in S_{G^2}(M)\), with \(\alpha,\beta\in G^0\), has one of
the following forms:
\begin{itemize}
\item \(\beta>\dcl(M,\alpha)\) (\(+\infty\)-type)
\item \(\beta<\dcl(M,\alpha)\) (\(-\infty\)-type)
\item there is some \(q\in\Q\) s.t. \(q\alpha+m<\beta<(q+\frac{1}{n})\alpha\) for all \(m\in M\) and \(n\in\N\)
(\(q^+\)-type)
\item there is some \(q\in\Q\) s.t. \((q-\frac{1}{n})\alpha<\beta<q\alpha+m\) for all \(m\in M\) and \(n\in\N\)
(\(q^-\)-type)
\item there is some \(r\in\R\) s.t. \(q_1\alpha<\beta<q_2\alpha\) for all \(q_1,q_2\in\Q\) with \(q_1<r<q_2\) (\(r^0\)-type)
\end{itemize}
\end{proposition}

\begin{proof}
Let \(p=\tp(\alpha,\beta/M)\) be a \(f\)-generic type which contained in \((G^2)^0\). By the cell
decomposition, we may assume that every formula \(\phi(x,y)\) in \(p\) is of the form
\begin{equation*}
D_n(x)\wedge(a<x)\wedge D_n(y)\wedge(f_1(x)\Box_1y\Box_2f_2(x))
\end{equation*}
with \(n\in\N\), \(a\in M\), \(f_i:D_n(M)\to M\) linear, and \(\Box_i\) either \(\le\) or no condition.

If every formula in \(p\) contains a cell of the form \(D_n(x)\wedge D_n(y)\wedge f_1(x)\le y\), it's then
a \(+\infty\)-type

Similar for \(-\infty\)-type.

Otherwise there are linear functions \(f_1(x)=q_1x+b_1\) and \(f_2(x)=q_2x+b_2\), with \(q_1,q_2\in\Q\)
and \(b_1,b_2\in M\) s.t. the cell
\begin{equation*}
D_n(x)\wedge(a<x)\wedge D_n(y)\wedge(f_1(x)\le y\le f_2(x))
\end{equation*}
is contained in \(p\), where both \(nq_1\) and \(nq_2\) are some integers. We call the above cell
a \((n,a,q_1,q_2)\)-cell.

Let
\begin{align*}
&Q_1=\{t\in\Q:\text{there is an \((n,a,t,q_2)\)-cell which is contained in \(p(x,y)\)}\}\\
&Q_2=\{t\in\Q:\text{there is an \((n,a,q_1,t)\)-cell which is contained in }p(x,y)\}
\end{align*}
Then both \(Q_1\) and \(Q_2\) are nonempty.

\begin{claim}
\((Q_1,Q_2)\) is a cut of \(\Q\)
\end{claim}

\#+BEGIN\textsubscript{proof}
\end{proof}
\#+END\textsubscript{proof}

\section{Problem}
\label{sec:orgb674c60}
\begin{center}
\begin{tabular}{lll}
\ref{Problem1} &  & \\\empty
\ref{Problem2} &  & \\\empty
\ref{Problem3} &  & \\\empty
\ref{Problem4} &  & \\\empty
\end{tabular}
\end{center}
\end{document}