% Created 2022-10-04 Tue 15:20
% Intended LaTeX compiler: pdflatex
\documentclass[11pt]{article}
\usepackage[utf8]{inputenc}
\usepackage[T1]{fontenc}
\usepackage{graphicx}
\usepackage{longtable}
\usepackage{wrapfig}
\usepackage{rotating}
\usepackage[normalem]{ulem}
\usepackage{amsmath}
\usepackage{amssymb}
\usepackage{capt-of}
\usepackage{hyperref}
\graphicspath{{../../books/}}
% TIPS
% \substack{a\\b} for multiple lines text





% pdfplots will load xolor automatically without option
\usepackage[dvipsnames]{xcolor}

\usepackage{forest}
% two-line text in node by [two \\ lines]
% \begin{forest} qtree, [..] \end{forest}
\forestset{
  qtree/.style={
    baseline,
    for tree={
      parent anchor=south,
      child anchor=north,
      align=center,
      inner sep=1pt,
    }}}
%\usepackage{flexisym}
% load order of mathtools and mathabx, otherwise conflict overbrace

\usepackage{mathtools}
%\usepackage{fourier}
\usepackage{pgfplots}
\usepackage{amsthm, mathabx,  amsmath, commath}
\usepackage{amsfonts}

\usepackage{empheq}
\usepackage{tikz}
\usetikzlibrary{arrows.meta}
\usepackage[most]{tcolorbox}

\newtheorem{theorem}{Theorem}[section]
\newtheorem{definition}{Definition}[section]
\newtheorem{corollary}{Corollary}[section]
\newtheorem{example}{Example}[section]
\newtheorem{lemma}{Lemma}[section]
\newtheorem{proposition}{Proposition}[section]

\newcommand{\bl}[1] {\boldsymbol{#1}}
\newcommand{\Wt}[1] {\stackrel{\sim}{\smash{#1}\rule{0pt}{1.1ex}}}
\newcommand{\wt}[1] {\widetilde{#1}}


%For boxed texts in align, use Aboxed{}
%otherwise use boxed{}

\DeclareMathSymbol{\widehatsym}{\mathord}{largesymbols}{"62}
\newcommand\lowerwidehatsym{%
  \text{\smash{\raisebox{-1.3ex}{%
    $\widehatsym$}}}}
\newcommand\fixwidehat[1]{%
  \mathchoice
    {\accentset{\displaystyle\lowerwidehatsym}{#1}}
    {\accentset{\textstyle\lowerwidehatsym}{#1}}
    {\accentset{\scriptstyle\lowerwidehatsym}{#1}}
    {\accentset{\scriptscriptstyle\lowerwidehatsym}{#1}}
}

\usepackage{graphicx}
    
% text on arrow for xRightarrow
\makeatletter
%\newcommand{\xRightarrow}[2][]{\ext@arrow 0359\Rightarrowfill@{#1}{#2}}
\makeatother


\def \bx {\boldsymbol{x}}
\def \ba {\boldsymbol{a}}
\def \bI {\boldsymbol{I}}
\def \bt {\boldsymbol{t}}
\def \bb {\boldsymbol{b}}
\def \bA {\boldsymbol{A}}
\def \bX {\boldsymbol{X}}
\def \bu {\boldsymbol{u}}
\def \bS {\boldsymbol{S}}
\def \bZ {\boldsymbol{Z}}
\def \bz {\boldsymbol{z}}
\def \by {\boldsymbol{y}}
\def \bw {\boldsymbol{w}}
\def \bT {\boldsymbol{T}}
\def \bS {\boldsymbol{S}}
\def \bm {\boldsymbol{m}}
\def \bW {\boldsymbol{W}}
\def \bY {\boldsymbol{Y}}
\def \bH {\boldsymbol{H}}
\def \blambda {\boldsymbol{\lambda}}
\def \bPhi {\boldsymbol{\Phi}}
\def \btheta {\boldsymbol{\theta}}
\def \bmu {\boldsymbol{\mu}}
\def \bphi {\boldsymbol{\phi}}
\def \bSigma {\boldsymbol{\Sigma}}
\def \lb {\left\{}
\def \rb {\right\}}
\def \caln {\mathcal{N}}
\def \dissum {\displaystyle\Sigma}
\def \dispro {\displaystyle\prod}
\def \E {\mathbb{E}}
\def \Q {\mathbb{Q}}
\def \V {\mathbb{V}}
\def \R {\mathbb{R}}
\def \calq {\mathcal{Q}}
\def \calg {\mathcal{G}}
\def \caln {\mathcal{N}}
\def \calr {\mathcal{R}}
\def \calm {\mathcal{M}}
\def \calc {\mathcal{C}}
\def \bcup {\bigcup}

\makeindex
\author{Marcin Petrykowski}
\date{\today}
\title{Generic Properties Of Groups}
\hypersetup{
 pdfauthor={Marcin Petrykowski},
 pdftitle={Generic Properties Of Groups},
 pdfkeywords={},
 pdfsubject={},
 pdfcreator={Emacs 28.0.92 (Org mode 9.6)}, 
 pdflang={English}}
\begin{document}

\maketitle
\tableofcontents


\section{Preliminaries}
\label{sec:orgf78afc1}
If \(p(\barx)\) is a type over \(A\), then we call the set of realizations of \(p\) in \(M\)
\begin{equation*}
p(M^n)=\{\bara\in M^n:(\forall\varphi(\barx)\in p(\barx))M\vDash\varphi(\bara)\}\vDash\bigcap_{\varphi(\barx)\in p(\barx)}\varphi(M^n)
\end{equation*}
\textbf{type definable over} \(A\). If \(V\) is a 0-type-definable subset of \(M^n\), then we sometimes
 identify \(V\) with the set
 \begin{equation*}
[V]=\{\tp(\bara):\bara\in V\}\subseteq S_n(\emptyset)
 \end{equation*}

A first order structure \(M\) is \(\kappa\)-saturated if for any \(A\subseteq M\) with \(\abs{A}<\kappa\), \(n<\omega\)
and \(p\in S_n(A)\), \(p\) has a realization in \(M\).

A group \((G,\cdot)\) is definable in a structure \(M\) if \(G\) is a definable subset of \(M^n\)
for some \(n<\omega\) and the group action \(\cdot:G\times G\to G\) is a definable function in \(M\).
If \(p(x)\) is a type over \(G\) and \(g\in G\), then

We call a first order structure \((M,\cdot,\dots)\) a group if \((M,\cdot)\) satisfies group axioms. We
usually denote it by \((G,\cdot,\dots)\). A structure of the form \((G,\cdot)\) is called a \textbf{pure} group.
 \begin{equation*}
g\cdot p(x)=\{g\cdot\varphi(x):\varphi(x)\in p(x)\}=\{\varphi(g^{-1}\cdot x):\varphi(x)\in p(x)\}
 \end{equation*}
A group \((G,\cdot)\) is definable in a structure \(M\) if \(G\) is a definable

An infinite totally ordered first order structure \((M,<,\dots)\) is \textbf{o-minimal} if every definable
subset of \(M\) is a union of finitely many intervals and points.

Let \((M,<,\dots)\) be an o-minimal structure. We usually say ``ultimately'' instead of ``for all
sufficiently large \(a\in M\)''. We denote an open interval with endpoints \(a\) and \(b\)
by \((a,b)\) and a closed one by \([a,b]\). In contrast, \(\la a,b\ra\) denotes the pair of
elements \(a\) and \(b\).

If \(a\in M\cup\{-\infty\}\), \(b\in M\cup\{+\infty\}\), \(a<b\) and \(f:(a,b)\to M\) is a definable function, then there
are \(a=a_1<\dots<a_n=b\) s.t. each interval \((a_i,a_{i+1})\) of \(f\) is either constant or
strictly monotone and continuous in the order topology. In particular, every definable
function \(f:M\to M\) is ultimately continuous and monotone

\section{Weak generic types}
\label{sec:org68491fb}
\subsection{Introduction}
\label{sec:org8aea476}
\begin{definition}[]
A set \(X\subseteq G\) is \textbf{(left) generic} if some finitely many left \(G\)-translates of \(X\)
cover \(G\). We say that a formula \(\varphi(x)\) is \textbf{(left) generic} if the set \(\varphi(G)\) of elements
of \(G\) realizing \(\varphi\) is \textbf{(left) generic}. Finally, we say that a type \(p(x)\) of elements
of \(G\) is \textbf{(left) generic} if every formula \(\varphi(x)\) with \(p(x)\vdash\varphi(x)\) is (left) generic
\end{definition}

In the stable case left generic = right generic \label{Problem1}

and each partial generic type extends to a complete generic type (since type is definable)

\begin{definition}[]
A set \(A\subseteq G\) is \textbf{weak generic}, if for some non-generic \(B\subseteq G\) we have that \(A\cup B\) is
generic. A formula \(\varphi(x)\) is \textbf{weak generic} if the set \(\varphi(G)\) is weak generic. A type \(p(x)\)
of elements of \(G\) is weak generic if every formula \(\varphi(x)\) with \(p(x)\vdash\varphi(x)\) is weak generic
\end{definition}

\subsection{Basic properties of weak generic sets and types}
\label{sec:orgdca1d9c}
\begin{lemma}[]
\label{3.2.1}
Assume that \(G\) is a group and \(X\) is a definable subset of \(G\). TFAE
\begin{enumerate}
\item the set \(X\) is weak generic
\item for some finitely many elements \(a_1,\dots,a_n\in G\) the set \(G\setminus\bigcup_{i=1}^na_i\cdot X\) is not generic
\item for some definable non-generic set \(Y\subseteq G\) the set \(X\cup Y\) is generic
\end{enumerate}
\end{lemma}

\begin{proof}
\(1\Rightarrow 2\): Assume \(X\) is weak generic, then there is non-generic set \(Y\subseteq G\) s.t. \(X\cup Y\) is
generic. Then there are \(a_1,\dots,a_n\in G\) s.t.
\begin{equation*}
\bigcup_{i=1}^na_i\cdot(X\cup Y)=\bigcup_{i=1}^na_i\cdot X\cup\bigcup_{i=1}^na_i\cdot Y=G
\end{equation*}
This means that
\begin{equation*}
G\setminus\bigcup_{i=1}^na_i\cdot X\subseteq\bigcup_{i=1}^na_i\cdot Y
\end{equation*}

\(2\Rightarrow 3\): Let \(Y=G\setminus\bigcup_{i=1}^na_i\cdot X\). Then \(Y\) is definable and not generic so
putting \(a_{n+1}=e\). Then \(G=\bigcup_{i=1}^{n+1}a_i\cdot(X\cup Y)\)
\end{proof}

\begin{lemma}[]
\label{3.2.2}
\begin{enumerate}
\item If \(X,Y\subseteq G\) are not weak generic, then \(X\cup Y\) is not weak generic
\item If \(p(x)\) is a (partial) weak generic type over \(A\subseteq G\), then \(p(x)\) may be extended to
a complete weak generic type over \(A\)
\end{enumerate}
\end{lemma}

\begin{proof}
\begin{enumerate}
\item Let \(Z\subseteq G\)  be non-generic. \(Y\) is not weak generic so \(Y\cup Z\) is not generic,
so \(Y\cup Z\cup X\) is not generic
\item non weak generics form an ideal

Let \(q(x)=\{\varphi(x)\in L(A):p(x)\cup\{\neg\varphi(x)\}\text{ is not weak generic}\}\). Then \(p\subseteq q\). We shall
show that \(q\) is a consistent partial type over \(A\). If not, then
\begin{equation*}
G\vDash\neg\exists x\bigwedge_{k=1}^n\varphi_k(x)
\end{equation*}
for some \(n<\omega\) and \(\varphi_1,\dots,\varphi_n\in q\). By compactness, for each \(k\in\{1,\dots,n\}\) we can find a
finite set of formulas \(p_k\subseteq p\) s.t. the type \(p_k(x)\cup\{\neg\varphi_k(x)\}\) is not weak generic.
Let \(\psi(x)=\bigwedge\{p_k(x):1\le k\le n\}\) and note that for every \(k\in\{1,\dots,n\}\) the set \(\psi(G)\cap\neg\varphi_k(G)\)
is not weak generic. By 1, neither is the union
\begin{equation*}
\bigcup_{k=1}^n(\psi(G)\cap\neg\varphi_k(G))=\psi(G)\cap\bigcup_{k=1}^n\neg\varphi_k(G)=\psi(G)\cap G=\psi(G)
\end{equation*}
contradicting the fact that \(p(x)\vdash\psi(x)\). Finally we take any \(r(x)\in S(A)\)
with \(r\supseteq q\) and the proof is complete
\end{enumerate}
\end{proof}

We see that (complete) weak generic types exist. By Lemma \ref{3.2.2}, the set
\begin{equation*}
WGEN(A)=\{p\in S(A):p\text{ is weak generic}\}
\end{equation*}
is closed and non-empty in \(S(A)\)

\begin{lemma}[]
\label{3.2.3}
Assume \(G\) is a group and \(A\subseteq G\)
\begin{enumerate}
\item If some weak generic type \(p(x)\in S(G)\) is generic, then all weak generic
types \(q(x)\in S(A)\) are generic
\item If for every \(p,q\in WGEN(G)\) there is \(g\in G\) s.t. \(g\cdot p=q\), then all weak generic
types \(q(x)\in S(A)\) are generic
\item If there is just one weak generic type in \(S(A)\), then it is generic
\end{enumerate}
\end{lemma}

\begin{proof}
\begin{enumerate}
\item Suppose that some weak generic type \(q(x)\in S(A)\) is not generic. Then some definable
generic set \(X\subseteq G\) may be divided into two non-generic definable sets \(X_1,X_2\).
Since \(X\) is generic, some left \(G\)-translates \(X'\) of \(X\) belongs to \(p(x)\). Then
the corresponding translates \(X_1',X_2'\) of \(X_1,X_2\) are also non-generic and one of them
belongs to \(p(x)\). Hence \(p(x)\) is not generic, a contradiction
\item If not, then we can find a formula \(\varphi(x)\in L(A)\) which is weak generic but not generic. Note
that \(\{\neg g\cdot\varphi(x):g\in G\}\) is a partial weak generic type over \(G\): for each \(m<\omega\)
and \(g_1,\dots,g_m\in G\), the set \(\bigcup_{i=1}^mg_i\cdot\varphi(G)\) is not generic, which implies that the
set \(\bigcap_{i=1}^m(G\setminus g_i\cdot\varphi(G))\) is weak generic. Extend the type \(\{\neg g\cdot\varphi(x):g\in G\}\) to
some \(q(x)\in WGEN(G)\). Next extend \(\varphi(x)\) to \(p(x)\in WGEN(G)\). Then \(\forall g\in G\;g\cdot p\neq q\),
a contradiction
\item by 2, immediately
\end{enumerate}
\end{proof}

By Lemma \ref{3.2.3} (1), in the stable case weak generic = generic

As an example note that if \(G=(G,<,+,\dots)\) is o-minimal, then there are exactly two complete weak
generic types, corresponding to \(-\infty\) and \(+\infty\), and they are not generic

\begin{lemma}[]
Assume that \(G\prec H\) and \(\varphi(x)\in L(G)\)
\begin{enumerate}
\item If \(\varphi(G)\) is weak generic in \(G\), then \(\varphi(H)\) is weak generic in \(H\)
\item If \(G\) is \(\aleph_0\)-saturated and \(\varphi(H)\) is weak generic in \(H\), then \(\varphi(G)\) is weak
generic in \(G\)
\end{enumerate}
\end{lemma}

\begin{proof}
\begin{enumerate}
\item There is a non-generic formula \(\psi(x)\in L(G)\) s.t. \(\varphi(G)\cup\psi(G)\) is generic in \(G\),
therefore \(\psi(H)\) is not generic in \(H\) and \(\varphi(H)\cup\psi(H)\) is generic in \(H\).
Thus \(\varphi(H)\) is weak generic in \(H\)
\item There is a formula \(\psi(x)\in L(H)\) s.t. \(\psi(H)\) is not generic in \(H\) and \(\varphi(H)\cup\psi(H)\) is
generic in \(H\). We have that \(\psi(x)=\psi(x,b)\) where \(b\subset H\). Let \(A\subseteq G\)  be a finite set
containing all parameters of \(\varphi(x)\). By \(\aleph_0\)-saturation of \(G\), we are able to find
in \(G\) a tuple \(a\subset G\) s.t. \(\tp(a/A)=\tp(b/A)\). Then \(\psi(x,a)\in L(G)\) has properties
needed to deduce the weak genericity of the set \(\varphi(G)\) in \(G\). Namely \(\psi(G,a)\) is not
generic in \(G\) and \(\varphi(G)\cup\psi(G,a)\) is generic in \(G\). If \(\psi(G,a)\) is generic in \(G\),
then for some \(0<n<\omega\) we have that
\begin{equation*}
G\vDash\exists x_1,\dots,x_n\forall y\exists z(\psi(z,a)\wedge\bigvee_{k=1}^ny=x_k\cdot z)
\end{equation*}
and the same holds in \(H\) since \(G\prec H\), which would lead to a contradiction
\end{enumerate}
\end{proof}

All lemmas in this section remain true if we consider a group \((G,\cdot)\) definable in a first
order structure \(M\). Then \(G\) is a definable subset of \(M^n\) for some \(n<\omega\) and for
every \(A\subseteq M\) we define the set \(WGEN(A)\) of complete weak generic types over \(A\) as the
set
\begin{equation*}
\{p\in S_n(A):\forall\varphi(x_1,\dots,x_n)\in p,G\cap\varphi(M^n)\text{ is weak generic in }G\}
\end{equation*}

\subsection{Characterizations of weak genericity}
\label{sec:orge0674fd}
\begin{proposition}[]
\label{3.3.1}
Assume \(G\) is a definable group in an o-minimal structure \(M\) and \(X\) is a definable weak
generic subset of \(G\). Then \(\dim(X)=\dim(G)\)
\end{proposition}

\begin{proof}
Suppose \(\dim(X)<\dim(G)\). Take a generic set \(A\) and a non-generic set \(B\)
s.t. \(A=B\cup X\) (where \(A\) and \(B\) are definable subsets of \(G\), apply Lemma \ref{3.2.1})
Choose a finite \(S\subseteq G\) with \(S\cdot A=G\). Then \(G\setminus(S\cdot B)\subseteq S\cdot X\) and
\begin{equation*}
\dim(G\setminus(S\cdot B))\le\dim(S\cdot X)=\dim(X)<\dim(G)
\end{equation*}
Hence the set \(S\cdot B\) is large in the sense
\end{proof}

Assume \(G\) is a group and \(X,Y\subseteq G\). We say that the set \(X\) is \textbf{translation disjoint} from
the set \(Y\) if for some \(a\in G\), \(a\cdot X\cap Y=\emptyset\)

\begin{lemma}[]
\label{3.3.2}
Assume \(G\) is a group and \(X\) is a weak generic subset of \(G\). Then for some
finite \(A\subseteq G\) there is no finite covering of \(X\) by sets that are translation disjoint
from \(A\cdot X\)
\end{lemma}

\begin{proof}
take \(Y\supseteq X\) generic and \(Y\setminus X\) not generic. We have that \(G=A\cdot Y\) for some
finite \(A\subseteq G\). We shall prove that \(A\) meets conditions of the lemma.

Suppose for some \(X_0,\dots,X_{n-1}\subseteq G\) and \(a_0,\dots,a_{n-1}\in G\) we have that
\begin{equation*}
X=\bigcup_{i<n}X_i\text{ and }\bigwedge_{i<n}(a_i\cdot X_i)\cap(A\cdot X)=\emptyset
\end{equation*}
Then for each \(i<n\), \(a_i\cdot X_i\subseteq G\setminus A\cdot X\subseteq A\cdot(Y\setminus X)\). So for
each \(i<n\), \(X_i\subseteq a_i^{-1}\cdot A\cdot(Y\setminus X)\), which implies
that \(X\subseteq\{a_0^{-1},\dots,a_{n-1}^{-1}\}\cdot A\cdot(Y\setminus X)\) and finally
\begin{equation*}
G=A\cdot Y=A\cdot(Y\setminus X)\cup A\cdot X\subseteq(A\cup(A\cdot\{a_0^{-1},\dots,a_{n-1}^{-1}\}\cdot A))\cdot(Y\setminus X)
\end{equation*}
Then \(G\) is covered by finitely many things
\end{proof}

\begin{corollary}[]
Assume \(G\) is a group and \(X\) is a weak generic subset of \(G\). Then the set \(X\cdot X^{-1}\)
is generic in \(G\)
\end{corollary}

\begin{proof}
Take a finite \(A\subseteq G\) as in Lemma \ref{3.3.2}. Then for each \(a\in G\), \(a\cdot X\cap A\cdot X\neq\emptyset\), which
implies that \(a\in A\cdot X\cdot X^{-1}\). So \(G=A\cdot X\cdot X^{-1}\)
\end{proof}

From now on, let \((G,<,+,\dots)\) be an o-minimal expansion of an ordered group \((G,<,+)\). Then
the group \(G\) is commutative, divisible and torsion-free. By \((G^n,+)\) we mean the product of
groups \((G,+)\times\dots\times(G,+)\) (\(n\) times). The ordering of \(G\) is dense since for
every \(a,b\in G\) with \(a<b\) we have that \(a<\frac{a+b}{2}<b\)

\begin{theorem}[]
\label{3.3.4}
Assume that \((G,<,+,\dots)\) is an o-minimal expansion of an ordered group \((G,<,+)\), \(0<n<\omega\)
and \(\varphi(x_1,\dots,x_n)\in L(G)\). TFAE
\begin{enumerate}
\item \(\varphi(x_1,\dots,x_n)\) is weak generic in \((G^n,+)\)
\item \(\neg\varphi(x_1,\dots,x_n)\) is not generic in \((G^n,+)\)
\item the set \(\varphi(G^n)\) contains arbitrarily large \(n\)-dimensional boxes
\begin{equation*}
(\forall R>0)(\exists a_1,\dots,a_n\in G)[a_1,a_1+R]\times\dots\times[a_n,a_n+R]\subseteq\varphi(G^n)
\end{equation*}
\end{enumerate}
\end{theorem}

\begin{proof}
\(3\Rightarrow 2\): suppose there is \(k<\omega\) and \(\la g_1^1,\dots,g_n^1\ra,\dots,\la g_1^k,\dots,g_n^k\ra\in G^n\) we have that
\begin{equation*}
G^n=\bigcup_{j=1}^k(\la g_1^j,\dots,g_n^j)+(G^n\setminus\varphi(G^n))
\end{equation*}
Put \(M=\max\{\abs{g_i^j}:1\le i\le n,1\le j\le k\}\). Using 3 we are able to find \(a_1,\dots,a_n\in G\) s.t.
\begin{equation*}
[a_1-M,a_1+M]\times\dots\times[a_n-M,a_n+M]\subseteq\varphi(G^m)
\end{equation*}
Then
\begin{equation*}
\la a_1,\dots,a_n\ra\notin\bigcup_{j=1}^k(\la g_1^j,\dots,g^j_n\ra+(G^n\setminus\varphi(G^n)))
\end{equation*}
a contradiction

\(2\Rightarrow 1\): since the set \(G^n=\varphi(G^n)\cup(G^n\setminus\varphi(G^n))\) is generic in \((G^n,+)\) and the
set \(G^n\setminus\varphi(G^n)\) is not generic

\(1\Rightarrow 3\): W.L.O.G., \(n\ge 2\). Using Lemma \ref{3.2.2} (2) find \(p(x_1,\dots,x_n)\in S_n(G)\) s.t. \(p\)
is a weak generic type in \((G^n,+)\) and \(\varphi\in p\). Extend \(G\) to a \(\abs{G}^+\)-saturated
group \(H\succ G\). Take \(\la a_1,\dots,a_n\ra\in H^n\) realizing \(p\) and fix a positive \(R\in G\). We shall
show that the follwing condition holds
\begin{equation*}
(\forall a\in H)(a_n\le a\le a_n+R\Rightarrow\tp(a/Ga_{<n})=\tp(a_n/Ga_{<n}))\tag{\star}
\end{equation*}

For the sake of contradiction assume that for some \(a\in[a_n,a_n+R]_H\) the
types \(\tp(a/Ga_{<n})\) and \(\tp(a_n/Ga_{<n})\) are distinct. By the o-minimality of \(H\), we
can find \(b\in[a_n,a_n+R]_H\) with \(b\in\dcl(Ga_{<n})\) (dense). Let \(\psi(x_1,\dots,x_{n-1},y)\in L(G)\)
be s.t. \(H\vDash\psi(a_{<n},b)\wedge\exists!y \psi(a_{<n},y)\). As \(b-R\le a_n\le b\), we have that \(\chi\in p\) where
\begin{equation*}
\chi(x_1,\dots,x_n)=\exists!y\psi(x_{<n},y)\wedge\forall y(\psi(y_{<n},y)\to(y-R\le x_n\le y))
\end{equation*}
Since \(\chi\in p\), the set \(\chi(G^n)\) is weak generic in \((G^n,+)\)

We define \(f:G^{n-1}\to G\) as:
\begin{equation*}
f(c_{<n})=
\begin{cases}
c_n-R&G\vDash\chi(\barc)\\
0&\text{otherwise}
\end{cases}
\end{equation*}
Take \(\la c_1,\dots,c_{n-1}\ra\in G^{n-1}\). If there is \(c_n\in G\)
s.t. \(G\vDash\chi(c_1,\dots,c_n)\), then there exists just one \(d\in G\) with \(G\vDash\psi(c_1,\dots,c_{n-1},d)\) and we
put \(f(c_1,\dots,c_{n-1})=d-R\). Otherwise we put \(f(c_1,\dots,c_{n-1})=0\). Then the function \(f\) is
definable over \(G\) and we consider the following formula over \(G\):
\begin{equation*}
\delta(x_1,\dots,x_n)=f(x_1,\dots,x_{n-1})\le x_n\le f(x_1,\dots,x_{n-1})+R
\end{equation*}
Since \(\chi(G^n)\subseteq\delta(G^n)\subseteq G^n\), the set \(\delta(G^n)\) is weak generic in \((G^n,+)\). Let \(A\subseteq G^n\) be a
finite set chosen for \(\delta(G^n)\) as in Lemma \ref{3.3.2}. Consider an
arbitrary \(\la h_1,\dots,h_{n-1}\ra\in H^{n-1}\). Choose \(M_{h_{<n}}\in G\) s.t.
\begin{equation*}
\{\la h_1,\dots,h_n\ra:f(h_{<n})+M_{h_{<n}}\le h_n\le f(h_{<n})+M_{h_{<n}}+R\}
\cap(A+\delta(H^n))=\emptyset
\end{equation*}
(exists since is bounded and \(A\) is finite)
If \(\tp(h_{<n}/G)=\tp(h'_{<n}/G)\), then \(M_{h_{<n}}\) is good also for \(h'_{<n}\). By
compactness, for each \(q(x_1,\dots,x_{n-1})\in S_{n-1}(G)\) we can find a
formula \(\varphi_q(x_1,\dots,x_{n-1})\in L(G)\) and \(M_q\in G\) s.t. for every \(h_{<n}\in H^{n-1}\) with
\(H\vDash\varphi_q(h_{<n})\) we have
\begin{equation*}
\{\la h_1,\dots,h_n\ra:f(h_{<n})+M_q\le h_n\le f(h_{<n})+M_q+R\}\cap(A+\delta(H^n))=\emptyset
\end{equation*}
Again by compactness, \(S_{n-1}(G)=[\varphi_{q_1}]\cup\dots\cup[\varphi_{q_k}]\) for some \(k<\omega\)
and \(q_1,\dots,q_k\in S_{n-1}(G)\).
\wu{
If not, then \(\forall n\in\omega\), \(G\vDash\bigwedge_{i=1}^n\neg\varphi_q{i}\), that is, \(\{\neg\varphi_{q_i}:i\in\omega\}\) is consistent
with \(G\), then realized by \(H\), which leads to a contradiction.
}
For \(i\in\{1,\dots,k\}\) put \(X_i=(\varphi_{q_i}(G^{n-1})\times G)\cap\delta(G^n)\)
and \(e_i=\la 0,\dots,0,M_{q_i}\ra\in G^n\). Then \(\delta(G^n)=X_1\cup\dots\cup X_k\) and for every \(i\in\{1,\dots,k\}\) we have
that \((e_i+X_i)\cap(A+\delta(G^n))=\emptyset\). This contradicts the choice of \(A\) and finishes the proof of
(\(\star\))

By (\(\star\)), we have that
\begin{equation*}
H\vDash\forall y((a_n\le y\wedge y\le a_n+R)\to\varphi(a_1,\dots,a_{n-1},y))
\end{equation*}
Therefore the formula \(\forall y((x_n\le y\le x_n+R\to\varphi(x_1,\dots,x_{n-1},y)))\) belongs to \(p\). In general,
for each formula \(\psi(x_1,\dots,x_n)\in p(x_1,\dots,x_n)\), \(k\in\{1,\dots,n\}\) and positive \(R\in G\) the formula
\begin{equation*}
\forall y((x_k\le y\le x_k+R)\to\psi(x_1,\dots,x_{k-1},y,x_{k+1},\dots,x_n))
\end{equation*}
belongs to \(p\). We inductively create formulas \(\varphi_k(x_1,\dots,x_n)\in p(x_1,\dots,x_n)\), \(k=\{1,\dots,n\}\),.
Namely, provided that \(\varphi_1(x_1,\dots,x_n),\dots,\varphi_{k-1}(x_1,\dots,x_n)\) have already been defined,
let \(\varphi_k(x_1,\dots,x_n)\) be the formula
\begin{equation*}
\forall y((x_k\le y\le x_k+R)\to(\varphi\wedge\varphi_1\wedge\dots\wedge\varphi_{k-1}(x_1,\dots,x_{k-1},y,x_{k+1},\dots,x_n)))
\end{equation*}
Finally, we take any \(\barg\in(\varphi\wedge\varphi_1\wedge\dots\wedge\varphi_n)(G^n)\) and see that
\begin{equation*}
[g_1,g_1+R]\times\dots\times[g_n,g_n+R]\subseteq\varphi(G^n)
\end{equation*}
\end{proof}

\begin{corollary}[]
Assume that \((G,<,+,\dots)\) is an o-minimal expansion of an ordered group \((G,<,+)\), \(0<n,k<\omega\)
and \(\varphi(x_1,\dots,x_n,y_1,\dots,y_k)\in L\)
\begin{enumerate}
\item there is \(\psi_1(y_1,\dots,y_k)\) s.t. for every \(\la a_1,\dots,a_k\ra\in G^k\) we have that \(G\vDash\psi_1(a)\)
iff \(\varphi(G^n,a)\) is weak generic in \((G^n,+)\)
\item There is \(\psi_2(y_1,\dots,y_k)\) s.t. for every \(\la a_1,\dots,a_k\ra\in G^k\) we have that \(G\vDash\psi_2(a)\)
iff \(\varphi(G^n,a)\) is generic in \((G^n,+)\)
\item there is a natural number \(N\) s.t. for every \(\varphi\)-definable \(X\subseteq G^n\) the set \(X\) is generic
in \((G^n,+)\) iff \(G^n\) may be covered by at most \(N\) left translates of \(X\)
\end{enumerate}
\end{corollary}

\begin{proof}
\begin{enumerate}
\item let \(\psi_1(y_1,\dots,y_k)\) be
\begin{equation*}
\forall r\exists z_1,\dots,z_n\forall x_1,\dots,x_n((\bigwedge_{i=1}^nz_i\le x_i\wedge x_i\le z_i+r)\to\varphi(x_1,\dots,x_n,y_1,\dots,y_k))
\end{equation*}
\setcounter{enumi}{2}
\item Assume that \(n=1\). Let \(\psi_2(y_1,\dots,y_k)\) be such as 2. Suppose for every \(N<\omega\) we can
find \(\la a_1,\dots,a_k\ra\in G^k\) s.t. the set \(\varphi(G,a_1,\dots,a_k)\) is generic in \(G\) but
not \(N\)-generic. Then the set of formulas
\begin{equation*}
\bigcup_{N<\omega}\{\psi_2(y_1,\dots,y_k)\wedge\forall z_1,\dots,z_N\exists t\forall x(\varphi(x,y_1,\dots,y_k)\to\bigwedge_{i=1}^Nt\neq z_i\cdot x)\}
\end{equation*}
is a type in variables \(y_1,\dots,y_k\) and has a realization \(\la b_1,\dots,b_k\ra\in H^k\) in
some \(\aleph_0\)-saturated elementary extension \(H\) of \(G\). Then we reach a contradiction as
the set \(\varphi(H,b_1,\dots,b_k)\) is simultaneously generic and not generic in \(H\)
\end{enumerate}
\end{proof}

\begin{corollary}[]
Assume that \((G,<,+,\dots)\) is an o-minimal expansion of an ordered group \((G,<,+)\), \(0<n<\omega\),
and \(p(x_1,\dots,x_n)\in S_n(G)\). TFAE
\begin{enumerate}
\item \(p(x_1,\dots,x_n)\) is weak generic in \((G^n,+)\)
\item \(\la g_1,\dots,g_n\ra+p(x_1,\dots,x_n)=p(x_1,\dots,x_n)\) for every \(\la g_1,\dots,g_n\ra\in  G^n\)
\end{enumerate}
\end{corollary}

\begin{proof}
\(1\Rightarrow 2\): suppose
\begin{equation*}
\la g_1,\dots,g_n\ra+p(x_1,\dots,x_n)\neq p(x_1,\dots,x_n)
\end{equation*}
for some \(\la g_1,\dots,g_n\ra\in G^n\). Then for some \(\varphi(x_1,\dots,x_n)\in p(x_1,\dots,x_n)\) we have that
\((\la g_1,\dots,g_n\ra+\varphi(G^n))\cap\varphi(G^n)=\emptyset\). \(\varphi(G^n)\) is weak generic in \((G^n,+)\) and hence contains
arbitrarily large boxes. Take any \(R>\max(\abs{g_1},\dots,\abs{g_n})\) and choose \(a_1,\dots,a_n\in G\)
s.t.
\begin{equation*}
B=[a_1,a_1+R]\times\dots\times[a_n,a_n+R]\subseteq\varphi(G^n)
\end{equation*}
we obtain
\begin{equation*}
\emptyset\neq(\la g_1,\dots,g_n\ra+B)\cap B\subseteq(\la g_1,\dots,g_n\ra+\varphi(G^n))\cap\varphi(G^n)=\emptyset
\end{equation*}
a contradiction

\(2\Rightarrow 1\): we shall prove a more general fact. Namely if \(G\) is a group and \(p(x)\in S(G)\) is
s.t. for every \(g\in G\) we have that \(g\cdot p=p\), then \(p\) is weak generic in \(G\)

If not, then we can find a formula \(\varphi(x)\in p(x)\) which is not weak generic in \(G\).
Then \(\neg\varphi(x)\) is generic in \(G\) so there are \(m<\omega\) and \(g_1,\dots,g_m\in G\)
s.t \(G=\bigcup_{i=1}^mg_i(G\setminus\varphi(G))\). Thus \(\bigcap_{i=1}^mg_i\cdot\varphi(G)=\emptyset\), which contradicts the fact that the
formulas \(g_1\cdot\varphi,\dots,g_m\cdot\varphi\) belong to the consistent type \(p(x)\)
\end{proof}

\subsection{Stationary}
\label{sec:org0ff691e}
In this section we assume that \((G,<,+,\dots)\) is an o-minimal expansion of an ordered
group \((G,<,+)\)

Recall that in stable group all weak generic types are generic. \label{Problem2} Moreover, all of
them are stationary over any model \(M\). This means that every (weak) generic type \(p\in S(M)\)
has a unique extension to a (weak) generic type \(q\in S(A)\) for each \(A\supseteq M\)

\begin{definition}[]
We call a weak generic type \(p\) over a set \(A\) \textbf{stationary} if for every \(B\supseteq A\) the
type \(p\) has just one extension to a complete weak generic type over \(B\)
\end{definition}

In general weak generic types do not need to be stationary

\begin{examplle}[]
we shall prove that the types \(p_1(x)=\{x<a:a\in G\}\) and \(p_2(x)=\{x>a:a\in G\}\) are the only two
weak generic types in \((G,+)\) complete over \(G\) and that both of them are stationary

By the o-minimality of \((G,<,+,\dots)\), every definable subset of \(G\) is a union of finitely
many points and intervals. For every \(a,b\in G\) the interval \((a,b)\) is not weak generic
in \((G,+)\) by Lemma \ref{3.2.1} (2). Thus no type in \(S_1(G)\) but \(p_1\) and \(p_2\) is weak
generic in \((G,+)\)

On the other hand, all intervals of the form \((-\infty,a)\) or \((b,+\infty)\) are weak generic
in \((G,+)\) since their complements in \(G\) are not generic in \((G,+)\).  This gives us the
weak genericity of the types \(p_1\) and \(p_2\)

If \(H\) is any elementary extension of \(G\), then there are also two complete (over \(H\))
weak generic types in \((H,+)\). This means that \(p_1\) and \(p_2\) are stationary
\end{examplle}

\begin{definition}[]
We call an o-minimal structure \((M,<,\dots)\) \textbf{stationary} if for every elementary extension \(N\)
of \(M\) and \(N\)-definable function \(g:N\to N\) there exists an \(M\)-definable
function \(f:N\to N\) s.t. \(g(x)\le f(x)\) for all sufficiently large \(x\in N\)
\end{definition}

\begin{theorem}[]
\label{3.4.4}
Assume \((M,<,\dots)\) is a stationary o-minimal structure and \(N\succ M\). For every \(N\)-definable
map \(g:N\to N\) with \(\lim_{x\to+\infty}g(x)=+\infty\)  we can find an \(M\)-definable map \(f:N\to N\)
s.t. \(\lim_{x\to+\infty}f(x)=+\infty\) and \(f(x)\le g(x)\) for all sufficiently large \(x\in N\)
\end{theorem}

\begin{proof}
First of all, assume that \(g\) is a bijection. Then \(g^{-1}\) exists and by the stationary
of \((M,<,\dots)\) we can find an \(M\)-definable function \(f:N\to N\) s.t. ultimately \(g^{-1}\le f\).
We have that \(\lim_{x\to+\infty}g^{-1}(x)=+\infty\), which implies that \(\lim_{x\to+\infty}f(x)=+\infty\). Since \(f\)
is \(M\)-definable, we can choose \(a\in M\) s.t. \(f\) is strictly increasing on \((a,+\infty)\)
(monotonicity theorem). We define a function \(f_1:N\to N\) as follows
\begin{equation*}
f_1(x)=
\begin{cases}
f(x)&x>a\\
f(a)+x-a&x\le a
\end{cases}
\end{equation*}
Then \(f_1\) is an \(M\)-definable bijection, hence \(f_1^{-1}\) exists and also
is \(M\)-definable. Moreover, \(\lim_{x\to+\infty}f_1^{-1}(x)=+\infty\) and ultimately \(f^{-1}_1\le g\)
so \(f_1^{-1}\) has the desired properties

If \(g\) is not a bijection, then proceeding as above we can find an \(N\)-definable
bijection \(g_1:N\to N\) s.t. ultimately \(g_1=g\)
\end{proof}

By the o-minimality of \((G,<,+,\dots)\), every definable subset of the set \(G\times G\) is a union of
finitely many cells of dimension 0,1,2. By Proposition \ref{3.3.1}, we are interested only in
cells of dimension 2 (we are interested in weak generic subsets). They are of the form
\begin{equation*}
C_{a,b}^{f,g}=\{\la x,y\ra\in G\times G:a<x<b\wedge f(x)<y<g(x)\}
\end{equation*}
where \(\{-\infty\}\cup G\ni a<b\in G\cup\{\infty\}\) and \(f,g:(a,b)\to G\cup\{-\infty,\infty\}\) are definable maps s.t. \(f(x)<g(x)\)
for each \(x\in(a,b)\). If \(a,b\in G\), then the cell \(C_{a,b}^{f,g}\) is not weak generic
in \((G,+)\times(G,+)\) by Theorem \ref{3.3.4}. Since we shall consider only weak generic
types \(p(x,y)\) in \((G,+)\times (G,+)\) s.t. \(\{x>a:a\in G\}\subseteq p(x,y)\) \label{Problem3}, we shall be interested only in
weak generic cells of the form \(C_{a,b}^{f,g}\) where \(a\in G\) and \(b=+\infty\)

\begin{definition}[]
Assume that functions \(f,g:G\to G\) are definable
\begin{enumerate}
\item \(f\ll g\) if \(f(x)<g(x)\) for all sufficiently large \(x\in G\) and the set
\begin{equation*}
\{\la x,y\ra\in G\times G:x>0\wedge f(x)<y\wedge y<g(x)\}
\end{equation*}
is weak generic in \((G,+)\times (G,+)\) (\(C_{0,+\infty}^{f,g}\))
\item \(f\sim g\) if
\begin{equation*}
\{\la x,y\ra\in G\times G:x>0\wedge f(x)<y\wedge y<g(x)\}
\end{equation*}
is not weak generic in \((G,+)\times (G,+)\)
\end{enumerate}
\end{definition}

\(\sim\) is an equivalence relation on the set of all definable functions from \(G\) to \(G\) and
that equivalence classes of \(\sim\) are convex (i.e., if \(f,g,h:G\to G\) are definable, \(f\sim h\)
and ultimately \(f(x)\le g(x)\le h(x)\), then \(f\sim g\) and \(g\sim h\))

\begin{definition}[]
Let \(f:G\to G\) be a definable function
\begin{enumerate}
\item Let \(p_f^+(x,y)\) denote the only extension of the type
\begin{equation*}
\{x>a:a\in G\}\cup\{y>f(x)\}\cup\{y<g(x):g\gg f\}
\end{equation*}
to a type which is complete over \(G\) and weak generic in \((G,+)\times(G,+)\)
\item Let \(p_f^-(x,y)\) denote the only extension of the type
\begin{equation*}
\{x>a:a\in G\}\cup\{y<f(x)\}\cup\{y>g(x):g\ll f\}
\end{equation*}
to a type which is complete over \(G\) and weak generic in \((G,+)\times(G,+)\)
\item Let \(p_{+\infty}(x,y)\) denote the weak generic type
\begin{equation*}
\{x>a:a\in G\}\cup\{y>g(x):g:G\to G\text{ definable}\}
\end{equation*}
\item Let \(p_{-\infty}(x,y)\) denote the weak generic type
\begin{equation*}
\{x>a:a\in G\}\cup\{y<g(x):g:G\to G\text{ definable}\}
\end{equation*}
\end{enumerate}
\end{definition}

\begin{theorem}[]
\label{3.4.7}
Assume that \((G,<,+,\dots)\) is an o-minimal expansion of an ordered group \((G,<,+)\). TFAE
\begin{enumerate}
\item \(p_f^+(x,y)\) and \(p_f^-(x,y)\) are stationary for each definable function \(f:G\to G\)
\item \(p_{+\infty}(x,y)\) and \(p_{-\infty}(x,y)\) are stationary
\item \((G,<,+,\dots)\) are stationary
\end{enumerate}
\end{theorem}

\begin{proof}
\(1\Rightarrow 2\): Let \(f:G\to G\) be a map constantly equal to 0. Then \(p_{+\infty}(x,y)=p_f^+(y,x)\) and
therefore \(p_{​+\infty}\) is stationary \label{Problem4}

\(2\Rightarrow 3\): Suppose the structure \((G,<,+,\dots)\) is not stationary. Then there exist an \(H\succ G\)
and a \(H\)-definable function \(g:H\to H\) s.t. no \(G\)-definable map \(f:H\to H\) dominates \(g\)

Consider the following partial types over \(H\):
\begin{gather*}
p_1(x,y)=p_{+\infty}(x,y)\cup\{y<g(x)\}\\
p_2(x,y)=p_{+\infty}(x,y)\cup\{y>g(x)\}
\end{gather*}
To reach a contradiction, it is enough to prove that both of them are weak generic
in \((H,+)\times(H,+)\), and therefore \(p_+(x,y)\) is not stationary. We begin with \(p_1\).

Goal:
\begin{equation*}
(\bigwedge_{i=1}^mx>a_i)\wedge(\bigwedge_{i=1}^ny>f_i(x))\wedge y<g(x)
\end{equation*}
is weak generic in \((H,+)\times(H,+)\) where \(a_1,\dots,a_m\in G\) and \(f_1,\dots,f_n\) are functions
from \(H\) to \(H\) definable over \(G\).

Take \(a=\max(a_1,\dots,a_n)\) and \(f=\max(f_1,\dots,f_n)\) we can confine our attention to the
sets \(X\) of the form
\begin{equation*}
X=\{\la x,y\ra\in H\times H:x>a\wedge y>f(x)\wedge y<g(x)\}
\end{equation*}
where \(a\in G\) and \(f:H\to H\) is definable over \(G\). W.L.O.G., we can assume that \(f\) is
ultimately non-decreasing

Consider a map \(h:H\to H\) defined as follows: \(h(a)=f(2a)+a\) for each \(a\in H\). Since \(h\)
is \(G\)-definable, \(g\) dominates \(h\) \label{Problem5}.
\wu{
Actually, \(\forall x\in N\exists x<y\in N\) s.t. \(g(y)>h(y)\). Therefore we can define \(g'\) to be
\begin{equation*}
g'(x)=\min\{g(y):x<y\wedge g(y)>h(y)\}
\end{equation*}
Since \(h\) is non-decreasing, \(g'\) dominates \(h\).
}
Note that for each large enough \(M\in H\) the area
between the graphs of \(f\) and \(g\) in \(H\times H\) contains the square whose vertices are
\begin{equation*}
\la M,f(2M)\ra,\la M,f(2M)+M\ra, \la 2M,f(2M)\ra, \la 2M,f(2M)+M\ra
\end{equation*}
By Theorem \ref{3.3.4}, \(X\) is weak generic in \((H,+)\times(H,+)\). As a result, the type \(p_1\) is
weak generic in \((H,+)\times(H,+)\)

\(3\Rightarrow 1\): Take any definable \(f:G\to G\). We shall show that both \(p_f^+\) and \(p_f^-\) are
stationary weak generic types

By the o-minimality of \(G\), \(f\) is ultimately non-negative or ultimately non-positive. It is
easy to see that \(p_f^+\) is stationary iff \(p_{-f}^-\) is stationary and \(p_f^-\) is stationary
iff \(p_{-f}^+\) is stationary. Therefore, W.L.O.G, we can assume that \(f\) is ultimately
non-negative. Moreover, \(f\) is ultimately non-increasing or ultimately non-decreasing.
If \(f\) is ultimately non-increasing, then \(p_f^+=p_z^+\) and \(p_f^-=p_z^-\) where \(z:G\to G\) is
constantly equal to 0.
So we can assume that \(f\) is ultimately non-decreasing (this includes the constant case)

Consider definable sets:
\begin{align*}
A&=\{a\in G:(\exists b>a)(\forall c\in(a,b))f(c)-f(a)\le c-a\}\\
B&=\{a\in G:(\exists b>a)(\forall c\in(a,b))f(c)-f(a)>c-a\}
\end{align*}
Note that by the o-minimality of \(G\), we have that \(G=A\cup B\) and for some \(M\in G\)
either \((M,+\infty)\subseteq A\) or \((M,+\infty)\subseteq B\). Enlarge \(M\) in order to ensure that \(f\) is continuous
on \((M,+\infty)\)

\textbf{Case 1}: \((M,+\infty)\subseteq A\). Then \(f\) grows ``slowly'' on \((M,+\infty)\):
\begin{equation*}
(\forall a>M)(\exists b>0)(\forall c\in(0,b))f(a+c)\le f(a)+c\tag{\star}
\end{equation*}
By (\(\star\)) and the continuity of \(f\)
\begin{equation*}
(\forall a>M)(\forall c>0)f(a+c)\le f(a)+c\tag{\star\star}
\end{equation*}
Because if not, then the opposite holds: \((\exists a>M)(\exists c>0)f(a+c)>f(a)+c\).
Let \(C=\{c>0:f(a+c)>f(a)+c\}\) and \(c_0=\inf(C)\).
Assertion (\(\star\)) implies that \(c_0>0\). Since \(f\) is continuous at \(c_0\), \(c_0\notin C\).
Choose \(d>c_0\) s.t. \((c_0,d)\subseteq C\). Since \(c_0\notin C\), \(f(a+c_0)\le f(a)+c_0\). On the other
hand, by the continuity of \(f\) at \(a+c_0\), we have that \(f(a+c_0)\ge f(a)+c_0\).
Thus \(f(a+c_0)=f(a)+c_0\) and for every \(e\in(0,d-c_0)\) we have that
\begin{equation*}
f(a+c_0+e)>f(a)+c_0+e=f(a+c_0)+e
\end{equation*}
which implies that \(a+c_0\notin A\). But \(a+c_0\in(M,+\infty)\subseteq A\), a contradiction. So (\(\star\star\) holds)

For the sake of contradiction assume that \(p_f^+\) is not stationary. Then for some \(H\succ G\) and
\(H\)-definable \(g:H\to H\) we have that \(f\ll g\) and \(g\ll h\) for each \(G\)-definable \(h:H\to H\)
with \(f\ll h\).
\wu{
For any \(q\supseteq p_f^+\), \(\{x>a:a\in H\}\subseteq q\).
}
Since \(\lim_{x\to+\infty}(g(x)-f(x))=+\infty\), there exists an increasing
to \(+\infty\) \(G\)-definable function \(h:H\to H\) s.t. ultimately \(h\le g-f\) by \ref{3.4.4}.
Enlarging \(M\) we can assume that \(h\) is increasing on \((M,​+\infty)\).

Now fix any positive \(R\in H\) and find \(a>M\) with \(h(a)\ge 2R\). By (\(\star\star\)), we have
that \(f(a+R)\le f(a)+R\). So the area between the graphs of \(f\) and \(f+h\) contains the square
whose vertices are
\begin{equation*}
\la a,f(a)+R\ra,\la a,f(a)+2R\ra,\la a+R,f(a)+R\ra,\la a+R,f(a)+2R\ra
\end{equation*}
As \(R\) was arbitrary, we can use Theorem \ref{3.3.4} to conclude that the area between the
graphs of \(f\) and \(f+h\) is weak generic in \((H,+)\times(H,+)\). So \(f\ll f+h\) and
therefore \(g\ll f+h\), which contradicts the fact that ultimately \(g\ge f+h\). So the
type \(p_f^+\) is stationary

\textbf{Case 2}: \((M,+\infty)\subseteq B\).  Then \(f\) grows ``quickly'' on \((M,+\infty)\), which implies
that \(\lim_{x\to+\infty}f(x)=+\infty\). As in \ref{3.4.4} find a definable bijection \(f_1:G\to G\)
s.t. \(f_1(af=f(a))\) for each \(a\in(M,+\infty)\). If \(g=f_1^{-1}\), then \(g\) grows ``slowly''
on \((M,+\infty)\) and from the previous case we know that the types \(p_g^+\) and \(p_g^-\) are
stationary. The proof is complete since
\(p_f^+(x,y)=p_{f_1}^+(x,y)=p_g^-(y,x)\) and \(p_f^-(x,y)=p_{f_1}^-(x,y)=p_g^+(x,y)\)
\end{proof}

\begin{examplle}[]
If \((G,<,+)\) is an o-minimal ordered group, then every definable function \(f:G\to G\) is
ultimately equal to \(f_q(x)+a\) for some \(a\in G\) and \(q\in\Q\) where \(f_q(x)=q\cdot x\)
(\cite{van1998tame}, Corollary 1.7.6) by considering \(G\) as a \(\Q\)-vector space. Below we list
all weak generic types in \((G,+)\times(G,+)\)
that are complete over \(G\) and contain the formula \(x>0\)
\begin{enumerate}
\item \(p_{-\infty}(x,y)\) and \(p_{+\infty}(x,y)\)
\item \(p_{f_q}^-(x,y)\) and \(p_{f_q}^+(x,y)\), \(q\in\Q\)
\item \(\{x>a:a\in G\}\cup\{y>q\cdot x:q\in\Q\wedge q<r\}\cup\{y<q\cdot x:q\in\Q\wedge q>r\}\), \(r\in\R\setminus\Q\)
\end{enumerate}
The structure \((G,<,+)\) is stationary since its elementary extensions are all linearly
bounded. Thus by Theorem \ref{3.4.7}, weak generic types of the form (1) and (2) are stationary.
\end{examplle}

\subsection{Expansions of real closed fields}
\label{sec:org003440f}
In this section, \((R,<,+,\cdot,0,1,\dots)\) is an o-minimal expansion of an ordered
ring \((R,<,+,\cdot,-,1)\). Such a ring must be a real closed field. Since \((R,<,+,\cdot,0,1,\dots)\) is an
o-minimal expansion of the ordered group \((R,<,+)\), all results obtained in the previous
section apply

\begin{definition}[]
We call a structure \((R,<,+,\cdot,\dots)\) \textbf{polynomially bounded} if for every definable
function \(f:R\to R\) there is \(n\in\N^+\) s.t. \(\abs{f(x)}\le x^n\) for all sufficiently large \(x\in R\)
\end{definition}

\begin{remark}
\label{3.5.2}
If a real closed field \((R,<,+,\cdot,\dots)\) is polynomially bounded and o-minimal, then for every
definable \(f:R\to R\) with \(\lim_{x\to+\infty}f(x)=+\infty\) we can find \(n\in\N_+\) s.t. \(f(x)\ge\sqrt[n]{x}\)
for all sufficiently large \(x\in R\)
\end{remark}

\begin{proof}
We proceed as in the proof of \ref{3.4.4}. Since \(f\) is ultimately increasing, we are able to
find a definable bijection \(g:R\to R\) s.t. \(f(x)=g(x)\) for all sufficiently large \(x\in R\). We
know that the inverse map \(g^{-1}\) is ultimately dominated by the polynomial
function \(x\mapsto x^n\) for some \(n\in\N_+\). And this implies \(f(x)=g(x)\ge\sqrt[n]{x}\) for
sufficiently large \(x\)
\end{proof}

Assume \((R,<,+,\cdot)\) is a pure real closed field. Since every definable map \(f:R\to R\) is
semi-algebraic, it follows from Proposition 2.6.1 in \cite{bochnak2013real} that the
structure \((R,<,+,\cdot)\) is polynomially bounded

\begin{corollary}[]
Every pure real closed field \((R,<,+,\cdot)\) is stationary and so are the weak generic
types \(p_f^-(x,y)\) and \(p_f^+(x,y)\) for each definable \(f:R\to R\)
\end{corollary}

\begin{proof}
Consider an arbitrary elementary extension \(S\) of \(R\) and any definable map \(f:S\to S\).
Since the real closed field \((S,<,+,\cdot)\) is polynomially bounded, there exists \(n\in\N_+\) s.t.
ultimately \(\abs{f(x)}\le x^n\). This gives us the stationary of the structure \((R,<,+,\cdot)\)
\end{proof}

\begin{definition}[]
Assume \((R,+,\cdot,0,1)\) is a field, \(f,g:R\to R\) and \(g(x)\neq 0\) for all sufficiently
large \(x\in R\). We write \(f\approx g\) iff
\begin{equation*}
\lim_{x\to+\infty}\frac{f(x)}{g(x)}=1
\end{equation*}
\end{definition}

\begin{lemma}[]
\label{3.5.5}
Assume \((R,<,+,\cdot)\) is a pure real closed field. If a function \(f:R\to R\) is definable and
ultimately non-zero, then for some \(q\in\Q\) and \(c\in R\setminus\{0\}\) we have that \(f(x)\approx c\cdot x^q\)
\end{lemma}

\begin{proof}
Let \(S\) be an arbitrary \(\abs{R}^+\)-saturated elementary extension of \(R\). We can
find \(a\in S\) s.t. \(a>r\) for every \(r\in R\). Let
\begin{equation*}
T=\{s\in S:\abs{s}<r\text{ for some }r\in R\}
\end{equation*}
Then \(T\) is a convex subring of \(S\),
\begin{equation*}
T^*=\{s\in S:\frac{1}{r}<\abs{s}<r\text{ for some }r\in R\}
\end{equation*}
and \((T^*,\cdot)\) is a subgroup of the multiplicative group \((S^*,\cdot)\).
\wu{
This definition is stronger than \(\abs{s}>0\) since there may be infinitesimal
}
The quotient group \((S^*/T^*,*,1)\) may be ordered in the following way:
\begin{equation*}
s_1/T^*\le s_2/T^*\Leftrightarrow\frac{s_1}{s_2}\in T
\end{equation*}
We define a function \(\nu:S\to S^*/T^*\cup\{-\infty\}\) (where for every \(s\in S^*\), \(-\infty<s/T^*\)
and \((-\infty)* s/T^*=-\infty\)) as follows:
\begin{equation*}
\nu(s)=
\begin{cases}
-\infty&s=0\\
s/T^*&\text{otherwise}
\end{cases}
\end{equation*}
Then \(\nu\) is a valuation of the field \(S\), i.e., \(\forall x,y\in S\),
\begin{enumerate}
\item \(\nu(x\cdot y)=\nu(x)*\nu(y)\)
\item \(\nu(x+y)\ge\min(\nu(x),\nu(y))\)
\item \(\nu(x)\neq\nu(y)\Rightarrow\nu(x+y)=\min(\nu(x),\nu(y))\)

\setcounter{enumi}{0}
\item \(\nu(x\cdot y)=\nu(x)*\nu(y)\)
\item \(\nu(x+y)\le\max(\nu(x),\nu(y))\)
\item \(\nu(x)\neq\nu(y)\Rightarrow\nu(x+y)=\max(\nu(x),\nu(y))\)
\end{enumerate}
Since \(f\) is semi-algebraic, by Lemma 2.5.2 in \cite{bochnak2013real}, there exists a
non-zero polynomial \(P(X,Y)\in R[X,Y]\) s.t. \(R\vDash\forall x(P(x,f(x))=0)\). So \(S\vDash\forall x(P(x,f(x))=0)\)
and, in particular, \(P(a,f(a))=0\). The polynomial \(P(X,Y)\) is of the form
\begin{equation*}
P(X,Y)=\sum_{i=1}^nr_i\cdot X^{k_i}\cdot Y^{l_i}
\end{equation*}
for some \(n\in\N_+\), \(r_i\in R\setminus\{0\}\) and \(k_i,l_i<\omega\) s.t. \(\la k_i,l_i\ra\neq\la k_j,l_j\ra\) for
every \(i\neq j\in\{1,\dots,n\}\). Thus
\begin{equation*}
0=\sum_{i=1}^nr_i\cdot a^{k_i}\cdot f(a)^{l_i}
\end{equation*}
and for some \(i\neq j\in\{1,\dots,n\}\) we have that
\begin{equation*}
\nu(r_i\cdot a^{k_i}\cdot f(a)^{l_i})=\nu(r_j\cdot a^{k_j}\cdot f(a)^{l_j})\neq-\infty
\end{equation*}
since \(f(a)\neq 0\) (if \(f(a)=0\), then \(f:R\to R\) would be ultimately equal to 0)

This implies that \(\nu(\frac{r_i}{f_j}\cdot a^{k_i-k_j}\cdot f(a)^{l_i-l_j})=1\)
and \(\nu(a^{k_i-k_j}\cdot f(a)^{l_i-l_j})=1\). So \(a^{k_i-k_j}\cdot f(a)^{l_i-l_j}\in T^*\). If \(l_i=l_j\),
then \(k_i\neq k_j\) and \(a^{k_i-k_j}\in T^*\), which implies that \(a\in T^*\)(\((T^*,\cdot,1)\) is
divisible), a contradiction.

So \(l_i\neq l_j\). Let \(q=-\frac{k_i-k_j}{l_i-l_j}\in\Q\) we obtain \(\frac{f(a)}{a^q}\in T^*\).
Therefore \(\frac{1}{r}<\abs{\frac{f(a)}{a^q}}<r\) for some \(r\in R\). If \(b\in S\) and \(b>a\),
then \(\tp(a/R)=\tp(b/R)\). Hence for every \(b>a\) we have
that \(\frac{1}{r}<\abs{\frac{f(b)}{b^q}}<r\) and consequently
\begin{equation*}
S\vDash\exists y\forall x(x>y\to\frac{1}{r}<\abs{\frac{f(x)}{x^q}}<r)
\end{equation*}
As \(R\prec S\), this implies that \(\frac{1}{r}<\abs{\frac{f(x)}{x^q}}<r\) for all sufficiently
large \(x\in R\). By the o-minimality of \(R\), for some \(c\in R\) with \(\frac{1}{r}\le\abs{c}\le r\)
we have that \(\lim_{x\to+\infty}\frac{f(x)}{x^q}=c\), which finishes the proof
\end{proof}

\begin{theorem}[]
Assume \((R,<,+,\cdot)\) is a pure real closed field. Let
\begin{equation*}
f(x)=\sum_{i=1}^ma_i\cdot x^{p_i}\quad\text{ and }\quad g(x)=\sum_{j=1}^nb_j\cdot x^{q_j}
\end{equation*}
where \(m,n\in\N_+\), \(a_1,\dots,a_m,b_1,\dots,b_n\in R\), \(a_1,b_1>0\), \(p_1>\dots>p_m\in\Q\) and \(q_1>\dots>q_n\in\Q\). TFAE
\begin{enumerate}
\item \(f\ll f+g\)
\item \(q_1>\max(0,p_1-1)\)
\end{enumerate}
\end{theorem}

\begin{proof}
We define a rate of growth \(gr(f)\) of a definable map \(f:R\to R\) as follows:
if \(f(x)\approx c\cdot x^q\) for some \(c\in R\setminus\{0\}\) and \(q\in\Q\), then \(gr(f)=q\) (Lemma \ref{3.5.5} implies
that \(gr(f)\) is well defined for each ultimately non-zero definable function \(f:R\to R\))
\end{proof}


\section{Problems}
\label{sec:org4a347eb}
\begin{center}
\begin{tabular}{lllll}
\ref{Problem1} & \ref{Problem2} & \ref{Problem3} :done & \ref{Problem4} & \ref{Problem5}\\
 &  &  &  & \\
\end{tabular}
\end{center}
\end{document}
