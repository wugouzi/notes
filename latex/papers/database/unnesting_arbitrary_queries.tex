% Created 2025-02-11 Tue 19:51
% Intended LaTeX compiler: xelatex
\documentclass[11pt]{article}
\usepackage{capt-of}
\usepackage{hyperref}
% TIPS
% \substack{a\\b} for multiple lines text





% pdfplots will load xolor automatically without option
\usepackage[dvipsnames]{xcolor}

\usepackage{forest}
% two-line text in node by [two \\ lines]
% \begin{forest} qtree, [..] \end{forest}
\forestset{
  qtree/.style={
    baseline,
    for tree={
      parent anchor=south,
      child anchor=north,
      align=center,
      inner sep=1pt,
    }}}
%\usepackage{flexisym}
% load order of mathtools and mathabx, otherwise conflict overbrace

\usepackage{mathtools}
%\usepackage{fourier}
\usepackage{pgfplots}
\usepackage{amsthm, mathabx,  amsmath, commath}
\usepackage{amsfonts}

\usepackage{empheq}
\usepackage{tikz}
\usetikzlibrary{arrows.meta}
\usepackage[most]{tcolorbox}

\newtheorem{theorem}{Theorem}[section]
\newtheorem{definition}{Definition}[section]
\newtheorem{corollary}{Corollary}[section]
\newtheorem{example}{Example}[section]
\newtheorem{lemma}{Lemma}[section]
\newtheorem{proposition}{Proposition}[section]

\newcommand{\bl}[1] {\boldsymbol{#1}}
\newcommand{\Wt}[1] {\stackrel{\sim}{\smash{#1}\rule{0pt}{1.1ex}}}
\newcommand{\wt}[1] {\widetilde{#1}}


%For boxed texts in align, use Aboxed{}
%otherwise use boxed{}

\DeclareMathSymbol{\widehatsym}{\mathord}{largesymbols}{"62}
\newcommand\lowerwidehatsym{%
  \text{\smash{\raisebox{-1.3ex}{%
    $\widehatsym$}}}}
\newcommand\fixwidehat[1]{%
  \mathchoice
    {\accentset{\displaystyle\lowerwidehatsym}{#1}}
    {\accentset{\textstyle\lowerwidehatsym}{#1}}
    {\accentset{\scriptstyle\lowerwidehatsym}{#1}}
    {\accentset{\scriptscriptstyle\lowerwidehatsym}{#1}}
}

\usepackage{graphicx}
    
% text on arrow for xRightarrow
\makeatletter
%\newcommand{\xRightarrow}[2][]{\ext@arrow 0359\Rightarrowfill@{#1}{#2}}
\makeatother


\def \bx {\boldsymbol{x}}
\def \ba {\boldsymbol{a}}
\def \bI {\boldsymbol{I}}
\def \bt {\boldsymbol{t}}
\def \bb {\boldsymbol{b}}
\def \bA {\boldsymbol{A}}
\def \bX {\boldsymbol{X}}
\def \bu {\boldsymbol{u}}
\def \bS {\boldsymbol{S}}
\def \bZ {\boldsymbol{Z}}
\def \bz {\boldsymbol{z}}
\def \by {\boldsymbol{y}}
\def \bw {\boldsymbol{w}}
\def \bT {\boldsymbol{T}}
\def \bS {\boldsymbol{S}}
\def \bm {\boldsymbol{m}}
\def \bW {\boldsymbol{W}}
\def \bY {\boldsymbol{Y}}
\def \bH {\boldsymbol{H}}
\def \blambda {\boldsymbol{\lambda}}
\def \bPhi {\boldsymbol{\Phi}}
\def \btheta {\boldsymbol{\theta}}
\def \bmu {\boldsymbol{\mu}}
\def \bphi {\boldsymbol{\phi}}
\def \bSigma {\boldsymbol{\Sigma}}
\def \lb {\left\{}
\def \rb {\right\}}
\def \caln {\mathcal{N}}
\def \dissum {\displaystyle\Sigma}
\def \dispro {\displaystyle\prod}
\def \E {\mathbb{E}}
\def \Q {\mathbb{Q}}
\def \V {\mathbb{V}}
\def \R {\mathbb{R}}
\def \calq {\mathcal{Q}}
\def \calg {\mathcal{G}}
\def \caln {\mathcal{N}}
\def \calr {\mathcal{R}}
\def \calm {\mathcal{M}}
\def \calc {\mathcal{C}}
\def \bcup {\bigcup}

\graphicspath{{../../../paper/database/}}
\definecolor{mintedbg}{rgb}{0.99,0.99,0.99}
\usepackage[cachedir=\detokenize{~/miscellaneous/trash}]{minted}
\setminted{breaklines,
mathescape,
bgcolor=mintedbg,
fontsize=\footnotesize,
frame=single,
linenos}

%% ox-latex features:
%   !announce-start, !guess-pollyglossia, !guess-babel, !guess-inputenc, caption,
%   image, !announce-end.

\usepackage{capt-of}

\usepackage{graphicx}

%% end ox-latex features


\author{Thomas Neumann \& Alfons Kemper}
\date{\today}
\title{Unnesting Arbitrary Queries}
\hypersetup{
 pdfauthor={Thomas Neumann \& Alfons Kemper},
 pdftitle={Unnesting Arbitrary Queries},
 pdfkeywords={},
 pdfsubject={},
 pdfcreator={Emacs 31.0.50 (Org mode 9.8-pre)},
 pdflang={English}}
\begin{document}

\maketitle
\section{Introduction}
\label{sec:orgb7f8077}
\begin{listing}[htbp]
\begin{minted}[]{sql}
select s.name, e.course
from   students s, exams e
where  s.id = e.sid and
       e.grade = (select min(e2.grade)
                  from exams e2
                  where s.id = e2.sid)
\end{minted}
\caption{Q1}
\end{listing}
The query contains a \textbf{dependent join}, i.e., a nested loop join where the evaluation of the right hand
side depends on the current value of the left-hand side. These joins are highly inefficient, and lead
to quadratic execution time.

A rewrite would look like this
\begin{listing}[htbp]
\begin{minted}[]{sql}
select s.name, e.course
from   students s, exams e,
       (select e2.sid as id, min(e2.grade) as best
        from exams e2
        group by e2.sid) m
where  s.id = e.sid and m.id = s.id and
       e.grade = m.best
\end{minted}
\caption{Q1'}
\end{listing}
Here, the evaluation of the subquery no longer depends on the values of s, and thus regular joins can
be used.

Now consider another query, by 2015, no system can unnests it
\begin{listing}[htbp]
\begin{minted}[]{sql}
select s.name, e.course
from   students s, exams e
where  s.id = e.sid and
  (s.major = 'CS' or s.major = 'Games Eng') and
   e.grade >= (select avg(e2.grade) + 1
               from exams e2
               where s.id = e2.sid or
               (e2.curriculums = s.major and
                s.year > e2.date))
\end{minted}
\caption{Q2}
\end{listing}
\section{Preliminaries}
\label{sec:org96ca1f9}
\textbf{(inner) join}:
\begin{equation*}
T_1\bowtie_p T_2:=\sigma_p(T_1\times T_2)
\end{equation*}
\textbf{dependent join}:
\begin{equation*}
T_1\lfbowtie_p T_2:=\{t_1\circ t_2\mid t_1\in T_1\wedge t_2\in T_2(t_1)\wedge p(t_1\circ t_2)\}
\end{equation*}
Here the right hand side is evaluated for every tuple of the left hand side. We denote the attributes
produced by an expression \(T\) by \(\cala(T)\), and free variables occurring in an expression \(T\)
by \(\calf(T)\). To evaluate dependent join, \(\calf(T_2)\subseteq \cala(T_1)\) must hold.

Take Q1 as example (sort of). \(T_1\) is
\begin{minted}[]{sql}
select s.name, s.id from students
\end{minted}
\(T_2\) is
\begin{minted}[]{sql}
select e.id, e.course
from exams e
where e.grade = (select min(e2.grade)
                 from exams e2
                 where X = e2.sid)
\end{minted}
So \(T_1, T_2\) here are actually all functions


We use \textbf{natural join} in the join predicate to simplify the notation. We assume that all relations
occuring in a query will have unique attribute names, even if they reference the same physical table,
thus \(A\bowtie B\equiv A\times B\). However, if we explicitly reference the same relation name twice,
and call for the natural join, then the attribute columns with the same name are compared, and the
duplicate columns are projected out. Consider, for example:
\begin{equation*}
(A\bowtie C)\bowtie_{p\wedge\text{natural join }C}(B\bowtie C)
\end{equation*}

Here, the top-most join checks both the predicate \(p\) and compares the columns of \(C\) that come
from both sides (and eliminates one of the two copies of \(C\)'s columns)

\begin{itemize}
\item \textbf{semi join}:
\begin{equation*}
T_1\ltimes T_2:=\{t_1\mid t_1\in T_1\wedge \exists t_2\in T_2:p(t_1\circ t_2)\}
\end{equation*}
\item \textbf{anti semi join}:
\begin{equation*}
T_1\rhd_pT_2:=\{t_1\mid t_1\in T_1\wedge\not\exists t_2\in T_2:p(t_1\circ t_2)\}
\end{equation*}
\item \textbf{left outer join}:
\begin{equation*}
T_1\leftouterjoin_pT_2:=(T_1\bowtie_pT_2)\cup\{t_1\circ_{a\in\cala(T_2)}(a:null)\mid
t_1\in(T_1\rhd_pT_2)\}
\end{equation*}
\item \textbf{full outer join}:
\begin{equation*}
T_1\fullouterjoin_p T_2:=(T_1\leftouterjoin_p T_2)\cup\{t_2\circ_{a\in\cala(T_1)}(a:null)\mid t_2\in(T_2\rhd_pT_1)\}
\end{equation*}
\end{itemize}

We define the dependent joins accordingly as \(\lftimes\), \(\rhd'\), \(\leftouterjoin'\), \(\fullouterjoin'\)

\textbf{group by}:
\begin{equation*}
\Gamma_{A;a:f}(e):=\{x\circ(a:f(y))\mid x\in\Pi_A(e)\wedge y=\{z\mid z\in e\wedge\forall a\in A:x.a=z.a\}\}
\end{equation*}
It groups its input \(e\) by \(A\) and evaluates one aggregation function.
\wu{I guess \(a\) is the attribute of \(\im f\)}

\textbf{map}:
\begin{equation*}
\chi_{a:f}:=\{x\circ(a:f(x))\mid x\in e\}
\end{equation*}

We define the attribute comparison operator \(=_A\) as
\begin{equation*}
t_1=_At_2:=\forall_{a\in A}:t_1.a=t_2.a
\end{equation*}
It compares NULL values as equal
\section{Unnesting}
\label{sec:org471566d}
The algebraic representation of a query with correlated subqueries results in a dependent join
\subsection{Simple Unnesting}
\label{sec:org042c665}
Consider
\begin{minted}[]{sql}
select ...
from lineitem l1 ...
where exists (select *
              from lineitem l2
              where l2.l_orderkey = l1.l_orderkey)
...
\end{minted}
This is translated into an algebra expression of the form
\begin{equation*}
l_1\lftimes(\sigma_{l_1.okey=l_2.okey}(l_2))
\end{equation*}
which is equivalent to
\begin{equation*}
l_1\ltimes_{l_1.okey=l_2.okey}(l_2)
\end{equation*}
\subsection{General Unnesting}
\label{sec:orgc81aa7c}
First, we translate the dependent join into a “nicer” dependent join (i.e., one that is easier to
manipulate), and second, we will push the new dependent join down into the query until we can
transform it into a regular join.

First,
\begin{equation*}
T_1\lfbowtie_pT_2\equiv T_1\bowtie_{p\wedge T_1=_{\cala(D)}D}(D\lfbowtie T_2)
\end{equation*}
where
\begin{equation*}
D:=\Pi_{\calf(T_2)\cap\cala(T_1)}(T_1)
\end{equation*}

In the original expression, we had to evaluate \(T_2\) for every tuple of \(T_1\). In the second
expression, we first compute the domain \(D\) of all variables bindings, evaluate \(T_2\) only once
for every distinct variable binding \wu{the result of \(\Pi\) is a set}, and then use a regular join to match the results to the original
\(T_1\) value. If there are a lot of duplicates, this already greatly reduces the number of
invocations of \(T_2\).

Consider the query for determining the worst exam for every student:
\begin{gather*}
\sigma_{e.grade=m}((\text{student }s\bowtie_{s.id=e.id}\text{exams }e)\lfbowtie\\
(\Gamma_{\emptyset;m:min(e2.grade)}(\sigma_{s.id=e2.sid}\text{exams }e2)))
\end{gather*}
\wu{
\((\text{student }s\bowtie_{s.id=e.id}\text{exams
        }e)\lfbowtie(\Gamma_{\emptyset;m:min(e2.grade)}(\sigma_{s.id=e2.sid}\text{exams }e2))\)
In left side, each student has many exams and grades. Then in the right side, for each student, its
worst grade is taken. Therefore it generates students + students grades + student worst grade. Then
we need \(\sigma\) to filter out those unrelated grades.
}

The equivalence rule allows to restrict the computation of the best grades to each student
\begin{gather*}
\dots(\Pi_{d.id,s.id}(\text{students }s\bowtie_{s.id=e.sid}\text{exams }e)\lfbowtie\\
(\Gamma_{\emptyset;m:min(e2.grade)}(\sigma_{d.id=e2.sid}\text{exams }e2)))
\end{gather*}
\wu{
Here \(d.id:s.id\) means we replace \(s.id\) with \(d.id\).
}

\begin{center}
\includegraphics[width=.8\textwidth]{../../images/papers/53.png}
\label{}
\end{center}

Knowing that \(D\) contains no duplicates helps in moving the dependent join further down into the
query. In the following, we will assume that any relation named \(D\) is duplicate free, and in the
following equivalences we only consider dependent joins where the left hand side is a set.

The untimate goal of our dependent join push-down is to reach a state where the right hand side no
longer dependents on the left hand side, i.e.,
\begin{equation*}
D\lfbowtie T\equiv D\bowtie T\text{ if }\calf(T)\cap\cala(D)=\emptyset
\end{equation*}

For selections, a push-down is very simple:
\begin{equation*}
D\lfbowtie \sigma_p(T_2)\equiv\sigma_p(D\lfbowtie T_2)
\end{equation*}
We first push the dependent join down as far as possible, until it can either be eliminated completely
due to substitution, or until it can be transformed into a regular join. Once all dependent joins have
been eliminated we can use the regular techniques like selection push-down and join reordering to
re-optimize the transformed query.

\begin{equation*}
D\lfbowtie(T_1\bowtie_pT_2)=
\begin{cases}
(D\lfbowtie T_1)\bowtie_pT_2&\calf(T_2)\cap\cala(D)=\emptyset\\
T_1\bowtie_p(D\lfbowtie T_2)&\calf(T_1)\cap\cala(D)=\emptyset\\
(D\lfbowtie T_1)\bowtie_{p\wedge\text{natural join }D}(D\lfbowtie T_2)&\text{otherwise}
\end{cases}
\end{equation*}
If we pushed the dependent join to both sides we have to augment the join predicate s.t. both sides
are matched on the \(D\) values.

For \emph{outer joins} we always have to replicate the dependent join if the inner side depends on it, as
otherwise we cannot keep track of unmatched tuples from the outer side.
\begin{align*}
&D\lfbowtie(T_1\scriptstyle\leftouterjoin_pT_2)\equiv
\begin{cases}
(D\lfbowtie T_1)\scriptstyle\leftouterjoin_pT_2&\calf(T_2)\cap\cala(D)=\emptyset\\
(D\lfbowtie T_1)\scriptstyle\leftouterjoin_{p\wedge\text{natural join }D}(D\lfbowtie T_2)&\text{otherwise}\\
\end{cases}\\
&D\lfbowtie(T_1\scriptstyle\fullouterjoin_pT_2)\equiv(D\lfbowtie T_1)\scriptstyle\fullouterjoin_{p\wedge\text{natural join }D}(D\lfbowtie T_2)
\end{align*}

Similar for \emph{semi join} and \emph{anti join}:
\begin{align*}
&D\lfbowtie(T_1\ltimes_pT_2)\equiv
\begin{cases}
(D\lfbowtie T_1)\ltimes_pT_2&\calf(T_2)\cap\cala(D)=\emptyset\\
(D\lfbowtie T_1)\ltimes_{p\wedge\text{natural join }D}(D\lfbowtie T_2)&\text{otherwise}
\end{cases}\\
&D\lfbowtie(T_1\rhd_pT_2)\equiv
\begin{cases}
(D\lfbowtie T_1)\rhd_pT_2&\calf(T_2)\cap\cala(D)=\emptyset\\
(D\lfbowtie T_1)\rhd_{p\wedge\text{natural join }D}(D\lfbowtie T_2)&\text{otherwise}
\end{cases}
\end{align*}
\emph{group by}
\begin{equation*}
D\lfbowtie(\Gamma_{A;a:f}(T))\equiv\Gamma_{A\cup\cala(D);a:f}(D\lfbowtie T)
\end{equation*}
\emph{projection}
\begin{equation*}
D\lfbowtie(\Pi_A(T))\equiv\Pi_{A\cup\cala(D)}(D\lfbowtie T)
\end{equation*}
\emph{set operation}
\begin{align*}
&D\lfbowtie(T_1\cup T_2)\equiv(D\lfbowtie T_1)\cup(D\lfbowtie T_2)\\
&D\lfbowtie(T_1\cap T_2)\equiv(D\lfbowtie T_1)\cap(D\lfbowtie T_2)\\
&D\lfbowtie(T_1\setminus T_2)\equiv(D\lfbowtie T_1)\setminus(D\lfbowtie T_2)
\end{align*}
\section{Problems}
\label{sec:org273725f}


\section{References}
\label{sec:org5a66ca9}
\label{bibliographystyle link}
\bibliographystyle{alpha}

\label{bibliography link}
\bibliography{../../../references}
\end{document}
