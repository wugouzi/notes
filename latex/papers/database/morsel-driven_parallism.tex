% Created 2025-04-21 Mon 13:39
% Intended LaTeX compiler: xelatex
\documentclass[11pt]{article}
\usepackage{capt-of}
\usepackage{hyperref}
% TIPS
% \substack{a\\b} for multiple lines text





% pdfplots will load xolor automatically without option
\usepackage[dvipsnames]{xcolor}

\usepackage{forest}
% two-line text in node by [two \\ lines]
% \begin{forest} qtree, [..] \end{forest}
\forestset{
  qtree/.style={
    baseline,
    for tree={
      parent anchor=south,
      child anchor=north,
      align=center,
      inner sep=1pt,
    }}}
%\usepackage{flexisym}
% load order of mathtools and mathabx, otherwise conflict overbrace

\usepackage{mathtools}
%\usepackage{fourier}
\usepackage{pgfplots}
\usepackage{amsthm, mathabx,  amsmath, commath}
\usepackage{amsfonts}

\usepackage{empheq}
\usepackage{tikz}
\usetikzlibrary{arrows.meta}
\usepackage[most]{tcolorbox}

\newtheorem{theorem}{Theorem}[section]
\newtheorem{definition}{Definition}[section]
\newtheorem{corollary}{Corollary}[section]
\newtheorem{example}{Example}[section]
\newtheorem{lemma}{Lemma}[section]
\newtheorem{proposition}{Proposition}[section]

\newcommand{\bl}[1] {\boldsymbol{#1}}
\newcommand{\Wt}[1] {\stackrel{\sim}{\smash{#1}\rule{0pt}{1.1ex}}}
\newcommand{\wt}[1] {\widetilde{#1}}


%For boxed texts in align, use Aboxed{}
%otherwise use boxed{}

\DeclareMathSymbol{\widehatsym}{\mathord}{largesymbols}{"62}
\newcommand\lowerwidehatsym{%
  \text{\smash{\raisebox{-1.3ex}{%
    $\widehatsym$}}}}
\newcommand\fixwidehat[1]{%
  \mathchoice
    {\accentset{\displaystyle\lowerwidehatsym}{#1}}
    {\accentset{\textstyle\lowerwidehatsym}{#1}}
    {\accentset{\scriptstyle\lowerwidehatsym}{#1}}
    {\accentset{\scriptscriptstyle\lowerwidehatsym}{#1}}
}

\usepackage{graphicx}
    
% text on arrow for xRightarrow
\makeatletter
%\newcommand{\xRightarrow}[2][]{\ext@arrow 0359\Rightarrowfill@{#1}{#2}}
\makeatother


\def \bx {\boldsymbol{x}}
\def \ba {\boldsymbol{a}}
\def \bI {\boldsymbol{I}}
\def \bt {\boldsymbol{t}}
\def \bb {\boldsymbol{b}}
\def \bA {\boldsymbol{A}}
\def \bX {\boldsymbol{X}}
\def \bu {\boldsymbol{u}}
\def \bS {\boldsymbol{S}}
\def \bZ {\boldsymbol{Z}}
\def \bz {\boldsymbol{z}}
\def \by {\boldsymbol{y}}
\def \bw {\boldsymbol{w}}
\def \bT {\boldsymbol{T}}
\def \bS {\boldsymbol{S}}
\def \bm {\boldsymbol{m}}
\def \bW {\boldsymbol{W}}
\def \bY {\boldsymbol{Y}}
\def \bH {\boldsymbol{H}}
\def \blambda {\boldsymbol{\lambda}}
\def \bPhi {\boldsymbol{\Phi}}
\def \btheta {\boldsymbol{\theta}}
\def \bmu {\boldsymbol{\mu}}
\def \bphi {\boldsymbol{\phi}}
\def \bSigma {\boldsymbol{\Sigma}}
\def \lb {\left\{}
\def \rb {\right\}}
\def \caln {\mathcal{N}}
\def \dissum {\displaystyle\Sigma}
\def \dispro {\displaystyle\prod}
\def \E {\mathbb{E}}
\def \Q {\mathbb{Q}}
\def \V {\mathbb{V}}
\def \R {\mathbb{R}}
\def \calq {\mathcal{Q}}
\def \calg {\mathcal{G}}
\def \caln {\mathcal{N}}
\def \calr {\mathcal{R}}
\def \calm {\mathcal{M}}
\def \calc {\mathcal{C}}
\def \bcup {\bigcup}

\graphicspath{{../../../paper/database/}}

%% ox-latex features:
%   !announce-start, !guess-pollyglossia, !guess-babel, !guess-inputenc, caption,
%   image, !announce-end.

\usepackage{capt-of}

\usepackage{graphicx}

%% end ox-latex features


\date{\today}
\title{Morsel-Driven Parallism: A NUMA-Aware Query Evaluation Framework for the Many-Core Age}
\hypersetup{
 pdfauthor={},
 pdftitle={Morsel-Driven Parallism: A NUMA-Aware Query Evaluation Framework for the Many-Core Age},
 pdfkeywords={},
 pdfsubject={},
 pdfcreator={},
 pdflang={English}}
\begin{document}

\maketitle
\section{Introduction}
\label{sec:org3038900}
\begin{center}
\includegraphics[width=.8\textwidth]{../../images/papers/63.png}
\captionof{figure}{\label{}Idea of morsel-driven parallism: \(R\bowtie_AS\bowtie_BT\)}
\end{center}

Parallism is achieved by processing each pipeline on different cores in parallel, as indicated by the
two (upper/red and lower/blue) pipelines in the figure. The core idea is a \textbf{scheduling} mechanism (the
“dispatcher”) that allows flexible parallel execution of an operator pipeline, that can change the
parallelism degree even during query execution.

A query is divided into \textbf{segments}, and each executing segment takes a morsel (e.g, 100,000) of input
tuples and executes these, materializing results in the next pipeline breaker.

The morsel framework enables NUMA local processing as indicated by the color coding in the figure: A
thread operates on NUMA-local input and writes its result into a NUMA-local storage area. Our
dispatcher runs a fixed, machine-dependent number of threads, such that even if new queries arrive
there is no resource over-subscription, and these threads are pinned to the cores, such that no
unexpected loss of NUMA locality can occur due to the OS moving a thread to a different core.
\section{Morsel-Driven Execution}
\label{sec:orgd9eeb79}
Consider
\begin{equation*}
\sigma_{\dots}(R)\bowtie_A\sigma_{\dots}(S)\bowtie_B\sigma_{\dots}(T)
\end{equation*}
Assuming that \(R\) is the largest table (after filtering) the optimizer would choose \(R\) as probe
input and build hash tables of the other two \(S\) and \(T\).

\begin{center}
\includegraphics[width=.8\textwidth]{../../images/papers/64.png}
\label{}
\end{center}

The plan consists of the three pipelines:
\begin{enumerate}
\item Scanning, filtering and building the hash table \(HT(T)\) of base relation \(T\),
\item Scanning, filtering and building the hash table \(HT(S)\) of argument \(S\),
\item Scanning, filtering \(R\) and probing the hash table \(HT(S)\) of \(S\) and probing the hash table
\(HT(T)\) of \(T\) and storing the result tuples.
\end{enumerate}

HyPer uses Just-In-Time (JIT) compilation to generate highly efficient machine code. Each pipeline
segment, including all operators, is compiled into one code fragment.

The morsel-driven execution of the algebraic plan is controlled by a so called \texttt{QEPobject} which
transfers executable pipelines to a dispatcher. It is the \texttt{QEPobject}'s responsibility
to observe data dependencies.

In our example query, the third (probe) pipeline can only be executed after the two hash tables have
been built, i.e., after the first two pipelines have been fully executed. For each pipeline the
\texttt{QEPobject} allocates the temporary storage areas into which the parallel threads executing the pipeline
write their results. After completion of the entire pipeline the temporary storage areas are logically
re-fragmented into equally sized morsels; this way the succeeding pipelines start with new
homogeneously sized morsels instead of retaining morsel boundaries across pipelines which could easily
result in skewed morsel sizes.

In order to write NUMA-locally and to avoid synchronization while writing intermediate results the
\texttt{QEPobject} allocates a storage area for each such thread/core for each executable pipeline.

\begin{center}
\includegraphics[width=.7\textwidth]{../../images/papers/65.png}
\label{3}
\end{center}
The \textbf{parallel} processing of the pipeline for filtering \(T\) and building the hash table \(HT(T)\) is
shown in Figure \ref{3}. In our figure three parallel threads are shown, each of which operates on one
morsel at a time. As our base relation \(T\) is stored “morsel-wise” across a NUMA-organized memory,
the scheduler assigns, whenever possible, a morsel located on the same socket where the thread is
executed. This is indicated by the coloring in the figure: The red thread that runs on a core of the
red socket is assigned the task to process a red-colored morsel, i.e., a small fragment of the base
relation \(T\) that is located on the red socket. Once, the thread has finished processing the
assigned morsel it can either be delegated (dispatched) to a different task or it obtains another
morsel (of the same color) as its next task. As the threads process one morsel at a time the system is
fully elastic. The degree of  parallelism (MPL) can be reduced or increased at any point (more
precisely, at morsel boundaries) while processing a query.

The logical algebraic pipeline of
\begin{enumerate}
\item scanning/filtering the input \(T\)
\item building the hash table
\end{enumerate}
is actually broken up into two physical processing pipelines marked as phases on the left-hand side of
the figure.

In the first phase the filtered tuples are inserted into NUMA-local storage areas, i.e., for each core
there is a separate storage area in order to avoid synchronization. To preserve NUMA-locality in
further processing stages, the storage area of a particular core is locally allocated on the same
socket.
\wu{
What if the data is skewed?
}

After all base table morsels have been scanned and filtered, in the second phase these storage areas
are scanned – again by threads located on the corresponding cores – and pointers are inserted into the
hash table. Segmenting the logical hash table building pipeline into two phases enables perfect sizing
of the global hash table because after the first phase is complete, the exact number of “surviving”
objects is known. This (perfectly sized) global hash table will  be probed by threads located on
various sockets of a NUMA system; thus, to avoid contention, it should not reside in a particular
NUMA-area and is therefore is interleaved (spread) across all sockets. As many parallel threads
compete to insert data into this hash table, a lock-free implementation is essential.

\begin{center}
\includegraphics[width=.7\textwidth]{../../images/papers/66.png}
\label{f4}
\end{center}

After both hash tables have been constructed, the probing pipeline can be scheduled as in Figure
\ref{f4}. Again, a thread requests work from the dispatcher which assigns a morsel in the corresponding
NUMA partition.
\section{Dispatcher: Scheduling Parallel Pipeline Tasks}
\label{sec:orgb8556bf}
\begin{center}
\includegraphics[width=.7\textwidth]{../../images/papers/155.png}
\label{f5}
\end{center}

The \textbf{dispatcher} is controlling and assigning the compute resources to the parallel pipelines. We
(pre-)create one worker thread for each hardware thread that the machine provides and permanently
bind each worker to it. Preemption of a task occurs at morsel boundaries.
\subsection{Elasticity}
\label{sec:org2b35261}
\subsection{Implementation Overview}
\label{sec:org1b88993}
In Figure \ref{f5} the Dispatcher appears like a separate thread. This, however, would incur two
disadvantages:
\begin{enumerate}
\item the dispatcher itself would need a core to run on or might preempt query evaluation threads
\item it could become a source of contention, in particular if the morsel size was configured quite small
\end{enumerate}

Therefore, the dispatcher is implemented as a lock-free data structure only. The dispatcher’s code is
then executed by the work-requesting query evaluation thread itself. Thus, the dispatcher is
automatically executed on the (otherwise unused) core of this worker thread. Relying on lock-free data
structures (i.e., the pipeline job queue as well as the associated morsel queues) reduces contention
even if multiple query evaluation threads request new tasks at the same time. Analogously, the
\texttt{QEPobject} that triggers the progress of a particular query by observing data dependencies (e.g.,
building hash tables before executing the probe pipeline) is implemented as a passive state machine.
The code is invoked by the dispatcher whenever a pipeline job is fully executed as observed by not
being able to find a new morsel upon a work request. Again, this state machine is executed on the
otherwise unused core of the worker thread that originally requested a new task from the dispatcher.

If, for some reason, a core finishes processing all morsels on its particular socket, the dispatcher
will “steal work” from another core, i.e., it will assign morsels on a different socket.

Therefore, we currently avoid to execute multiple pipelines from one query in parallel; in our example,
\subsection{Morsel Size}
\label{sec:org0c16f09}
it only needs to be large enough to amortize scheduling overhead while providing good response times.
\section{Parallel Operator Details}
\label{sec:orga54e071}
\subsection{Hash Join}
\label{sec:org40f28b7}
\subsection{Lock-Free Tagged Hash Table}
\label{sec:org136bc8f}
The hash table that we use for the hash join operator has an early-filtering optimization, which
improves performance of selective joins, which are quite common. The key idea is to tag a hash bucket
list with a small filter into which all elements of that particular list are “hashed” to set their
1-bit.

\begin{center}
\includegraphics[width=.8\textwidth]{../../images/papers/156.png}
\label{f7}
\end{center}
\subsection{NUMA-Aware Table Partitioning}
\label{sec:org2863e58}
Goal: NUMA-local tables scan

Prerequisite: relations have to be distributed over the memory nodes.

How: partition relations using the hash value of some “important” attribute.
\subsection{Grouping/Aggregation}
\label{sec:org53a1864}
\begin{center}
\includegraphics[width=.7\textwidth]{../../images/papers/157.png}
\label{f8}
\end{center}

In the first phase, thread-local pre-aggregation efficiently aggregates heavy hitters using a
thread-local, fixed-sized hash table. When this small pre-aggregation table becomes full, it is
flushed to overflow partitions. After all input data has been partitioned, the partitions are
exchanged between the threads.

The second phase consists of each thread scanning a partition and aggregating it into a thread-local
hash table. As there are more partitions than worker threads, this process is repeated until all
partitions are finished. Whenever a partition has been fully aggregated, its tuples are immediately
pushed into the following operator before processing any other partitions. As a result, the aggregated
tuples are likely still in cache and can be processed more efficiently.
\subsection{Sorting}
\label{sec:org4a44ffd}
\begin{center}
\includegraphics[width=.7\textwidth]{../../images/papers/158.png}
\label{f9}
\end{center}

Each thread first computes local separators by picking equidistant keys from its sorted run. Then, to
handle skewed distribution and similar to the median-of-medians algorithm, the local separators of all
threads are combined, sorted, and the eventual, global separator keys are computed. After determining
the global separator keys, binary (or interpolation) search finds the indexes of them in the data
arrays. Using these indexes, the exact layout of the output array can be computed. Finally, the runs
can be merged into the output array without any synchronization. 
\section{Problems}
\label{sec:orgcd3a22d}


\section{References}
\label{sec:org03ee25a}
\label{bibliographystyle link}
\bibliographystyle{alpha}

\label{bibliography link}
\bibliography{../../../references}
\end{document}
