% Created 2024-07-16 Tue 16:48
% Intended LaTeX compiler: xelatex
\documentclass[11pt]{article}
\usepackage{hyperref}
% TIPS
% \substack{a\\b} for multiple lines text





% pdfplots will load xolor automatically without option
\usepackage[dvipsnames]{xcolor}

\usepackage{forest}
% two-line text in node by [two \\ lines]
% \begin{forest} qtree, [..] \end{forest}
\forestset{
  qtree/.style={
    baseline,
    for tree={
      parent anchor=south,
      child anchor=north,
      align=center,
      inner sep=1pt,
    }}}
%\usepackage{flexisym}
% load order of mathtools and mathabx, otherwise conflict overbrace

\usepackage{mathtools}
%\usepackage{fourier}
\usepackage{pgfplots}
\usepackage{amsthm, mathabx,  amsmath, commath}
\usepackage{amsfonts}

\usepackage{empheq}
\usepackage{tikz}
\usetikzlibrary{arrows.meta}
\usepackage[most]{tcolorbox}

\newtheorem{theorem}{Theorem}[section]
\newtheorem{definition}{Definition}[section]
\newtheorem{corollary}{Corollary}[section]
\newtheorem{example}{Example}[section]
\newtheorem{lemma}{Lemma}[section]
\newtheorem{proposition}{Proposition}[section]

\newcommand{\bl}[1] {\boldsymbol{#1}}
\newcommand{\Wt}[1] {\stackrel{\sim}{\smash{#1}\rule{0pt}{1.1ex}}}
\newcommand{\wt}[1] {\widetilde{#1}}


%For boxed texts in align, use Aboxed{}
%otherwise use boxed{}

\DeclareMathSymbol{\widehatsym}{\mathord}{largesymbols}{"62}
\newcommand\lowerwidehatsym{%
  \text{\smash{\raisebox{-1.3ex}{%
    $\widehatsym$}}}}
\newcommand\fixwidehat[1]{%
  \mathchoice
    {\accentset{\displaystyle\lowerwidehatsym}{#1}}
    {\accentset{\textstyle\lowerwidehatsym}{#1}}
    {\accentset{\scriptstyle\lowerwidehatsym}{#1}}
    {\accentset{\scriptscriptstyle\lowerwidehatsym}{#1}}
}

\usepackage{graphicx}
    
% text on arrow for xRightarrow
\makeatletter
%\newcommand{\xRightarrow}[2][]{\ext@arrow 0359\Rightarrowfill@{#1}{#2}}
\makeatother


\def \bx {\boldsymbol{x}}
\def \ba {\boldsymbol{a}}
\def \bI {\boldsymbol{I}}
\def \bt {\boldsymbol{t}}
\def \bb {\boldsymbol{b}}
\def \bA {\boldsymbol{A}}
\def \bX {\boldsymbol{X}}
\def \bu {\boldsymbol{u}}
\def \bS {\boldsymbol{S}}
\def \bZ {\boldsymbol{Z}}
\def \bz {\boldsymbol{z}}
\def \by {\boldsymbol{y}}
\def \bw {\boldsymbol{w}}
\def \bT {\boldsymbol{T}}
\def \bS {\boldsymbol{S}}
\def \bm {\boldsymbol{m}}
\def \bW {\boldsymbol{W}}
\def \bY {\boldsymbol{Y}}
\def \bH {\boldsymbol{H}}
\def \blambda {\boldsymbol{\lambda}}
\def \bPhi {\boldsymbol{\Phi}}
\def \btheta {\boldsymbol{\theta}}
\def \bmu {\boldsymbol{\mu}}
\def \bphi {\boldsymbol{\phi}}
\def \bSigma {\boldsymbol{\Sigma}}
\def \lb {\left\{}
\def \rb {\right\}}
\def \caln {\mathcal{N}}
\def \dissum {\displaystyle\Sigma}
\def \dispro {\displaystyle\prod}
\def \E {\mathbb{E}}
\def \Q {\mathbb{Q}}
\def \V {\mathbb{V}}
\def \R {\mathbb{R}}
\def \calq {\mathcal{Q}}
\def \calg {\mathcal{G}}
\def \caln {\mathcal{N}}
\def \calr {\mathcal{R}}
\def \calm {\mathcal{M}}
\def \calc {\mathcal{C}}
\def \bcup {\bigcup}

\graphicspath{{../../../paper/transaction/}}

%% ox-latex features:
%   !announce-start, !guess-pollyglossia, !guess-babel, !guess-inputenc, caption,
%   underline, image, !announce-end.

\usepackage{capt-of}

\usepackage[normalem]{ulem}

\usepackage{graphicx}

%% end ox-latex features


\author{Google}
\date{\today}
\title{Online, Asynchronous Schema Change in F1}
\hypersetup{
 pdfauthor={Google},
 pdftitle={Online, Asynchronous Schema Change in F1},
 pdfkeywords={},
 pdfsubject={},
 pdfcreator={Emacs 31.0.50 (Org mode 9.8-pre)}, 
 pdflang={English}}
\begin{document}

\maketitle
\section{Introduction}
\label{sec:org912ad9a}
We introduce a protocol for schema evolution in a globally distributed database management system with
shared data, stateless servers, and no global membership. Our protocol is
\begin{itemize}
\item \emph{asynchronous}: it allows different servers in the database system to transition to a new schema at
different times
\item \emph{online}: all servers can access and update all data during a schema change.
\end{itemize}

We provide a formal model for determining the correctness of schema changes under these conditions,
and we demonstrate that many common schema changes can cause anomalies and database corruption. We
avoid these problems by replacing corruption-causing schema changes with a sequence of schema changes
that is guaranteed to avoid corrupting the database so long as all servers are no more than one schema
version behind at any time.


F1 is built on top of Spanner. Features:
\begin{itemize}
\item \textbf{Massively distributed}
\item \textbf{Relational schema}
\item \textbf{Shared data storage}: All F1 servers in all datacenters have access to all data stored in Spanner.
\item \textbf{Stateless servers}
\item \textbf{No global membership}
\end{itemize}


\begin{center}
\includegraphics[width=.88\textwidth]{../../images/papers/27.png}
\label{}
\end{center}

Because each server has shared access to all data, servers using different schema versions may corrupt
the database. Consider a schema change from schema \(S_1\) to schema \(S_2\) that adds index \(I\) on
table \(R\). Assume two different servers, \(M_1\) and \(M_2\) , execute the following sequence of
operations:
\begin{enumerate}
\item Server \(M_2\), using schema \(S_2\), inserts a new row \(r\) to table \(R\). Because \(S_2\)
contains index \(I\), server \(M_2\) also adds a new index entry corresponding to \(r\) to the
key-value store.
\item Server \(M_1\), using schema \(S_1\), deletes \(r\). Because \(S_1\) does not contain \(I\),
\(M_1\) removes \(r\) from the key-value store but fails to remvoe the corresponding index entry in
\(I\).
\end{enumerate}
\section{Background}
\label{sec:orgc851acb}
\subsection{Key-value Store}
\label{sec:orgd740840}
F1 assumes the key-value store supports three operations: \(put\), \(del\) and \(get\). Additionally,
F1's \uline{optimistic} concurrency control adds two more requirements on the key-value store:
\begin{enumerate}
\item \textbf{Commit timestamps}: Every key–value pair has a last-modified timestamp which is updated atomically by the key–value store.
\item \textbf{Atomic test-and-set support}: Multiple get and put operations can be executed atomically.
\end{enumerate}
\subsection{Relational Schema}
\label{sec:org0795fcc}
An F1 \textbf{schema} is a set of table definitions that enable F1 to interpret the database located in the
key–value store. Each table definition has
\begin{itemize}
\item a list of columns (along with their types)
\item a list of secondary indexes
\item a list of integrity constraints (foreign key or index uniqueness constraints)
\item a list of optimistic locks.
\end{itemize}

\textbf{Optimistic locks} are required columns that cannot be read directly by client transactions; A subset of
columns in a table forms the primary key of the table.

Column values can be either primitive types or complex types (specified in F1 as protocol buffers).
Primary key values are restricted to only primitive types.

We call a column \textbf{required} if its value must be present in every row. All primary-key columns are
implicitly required, while non-key columns may be either required or optional.
\subsection{Row Representation}
\label{sec:orge034392}
Each key logically includes the name of the table, the primary key values of the containing row, and
the name of the column whose value is stored in the pair. Although this appears to needlessly repeat
all primary key values in the key for each column value, in practice, F1’s physical storage format
eliminates this redundancy. We denote the key for the value of column \(C\) in row r as \(k_r(C)\).

In addition to the column values, there is also a reserved key–value pair with the special column
\(exists\). This key–value pair indicates the existence of row \(r\) in the table, and it has no
associated value, which we denote as \(\la key,null\ra\).

\begin{center}
\includegraphics[width=.99\textwidth]{../../images/papers/28.png}
\label{}
\end{center}

F1 also supports \textbf{secondary indexes}. A secondary index in F1 covers a non-empty subset of columns on a
table and is itself represented by a set of key–value pairs in the key–value store. Each row in the
indexed table has an associated index key–value pair. The key for this pair is formed by concatenating
the table name, the index name, the row’s indexed column values, and the row’s primary key values. We
denote the index key for row \(r\) in index \(I\) as \(k_r(I)\), and as in the case of the special
exists column, there is no associated value.
\subsection{Relational Operations}
\label{sec:org0ca3912}
\begin{itemize}
\item \(insert(R, vk_r, vc_r)\) inserts row \(r\) to table \(R\) with primary key values \(vk_r\) and
non-key column values \(vc_r\). Insert fails if a row with the same primary key values already
exists in \(R\).
\item \(delete(R, vk_r)\) deletes row \(r\) with primary key values \(vk_r\) from table \(R\).
\item \(update(R, vk_r , vc_r)\) updates row \(r\) with primary key values \(vk_r\) in table \(R\) by
replacing the values of a subset of non-key columns with those in \(vc_r\) . \(update\) cannot
modify values of primary keys. Such updates are modeled by a \(delete\) followed by an \(insert\).
\item \(query(\vec{R}, \vec{C}, P)\)returns a projection \(\vec{C}\) of rows from tables in \(\vec{R}\)
that satisfy predicate \(P\).
\end{itemize}

We use the notation \(write(R, vk_r, vc_r)\) to mean any of \(insert\), \(delete\), or \(update\) when
we wish to model the fact that some data has changed, but we do not care about the specific type of
operation that changed it.

These relational operations are translated into changes to the key–value store based on the schema. We
subscript all operations with their related schema, such as \(delete_S(R, vk_r)\).

Whenever we need to distinguish the transaction that issued a particular operation, we superscript the
operation with the transaction identifier. We introduce a shorthand notation \(query(R, C, vk_r)\) for
a query reading a single value of column \(C\) in row \(r\) with primary key \(vk_r\) in table \(R\).
\subsection{Concurrency Control}
\label{sec:orgf1b0b64}
F1’s concurrency control is relevant to schema evolution because F1's schema contains an additional
element on each table: \textbf{optimistic locks}.

Each column in the table is associated with exactly one optimistic lock. Each row has its own instance
of each of the optimistic locks defined in the schema, and these instances control concurrent access
to that row’s column values by multiple transactions.

When clients read column values as part of a transaction, they accumulate last-modified timestamps
from the locks covering those columns; at commit time, these timestamps are submitted to the server
and validated to ensure that they have not changed. If validation succeeds, the last-modified
timestamps of all locks associated with columns modified by the transaction are updated to the current
timestamp. This form of concurrency control can be shown to be conflict serializable
\section{Schema Changes}
\label{sec:org9d625be}
All servers in an F1 instance share a set of key–value pairs, called a \textbf{database representation}, that
are located in a key–value store. To interpret these key–value pairs as rows, every F1 server
maintains a copy of its instance’s schema in its memory, and it uses this schema to translate
relational operators into the operations supported by the key–value store. Accordingly, when a client
submits an operation, the schema used for that operation is determined by the schema currently in the
memory of the F1 server the client is connected to.

The canonical copy of the schema is stored within the database representation as a special key–value
pair known to all F1 servers in the instance. When the canonical copy of the schema is replaced with a
new version, it begins a \textbf{schema change}, which is a process that propagates the new schema to all
servers in an F1 instance.

The fundamental cause of this corruption is that the change made to the schema is, in some sense, too
abrupt. Servers on the old schema have no knowledge of the index, while servers on the new schema use
it for all operations as if it were fully maintained. Additionally, although we used adding an index
as an example, this problem occurs for all fundamental schema change operations in our system.

To simplify reasoning about the correctness of our implementation, we restrict servers in an F1
instance from using more than two distinct schema versions. In particular, our protocol expects that
all servers use either the most recent schema version or a schema that is at most one version old.
\subsection{Schema Elements and States}
\label{sec:orgca0eade}
\begin{definition}[]
A \textbf{delete-only table}, \textbf{column}, or \textbf{index} cannot have their key-value pairs read by user transactions and
\begin{enumerate}
\item if \(E\) is a table or a column, it can be modified only by \(delete\) operations
\item if \(E\) is an index, it is modified only by \(delete\) and \(update\) operations. Moreover,
\(update\) operations can delete key-value pairs corresponding to updated index keys, but they
cannot create any new ones.
\end{enumerate}
\end{definition}

\begin{definition}[]
A \textbf{write-only column} or \textbf{index} can have their key-value pairs modified by \(insert\), \(delete\), and
\(update\) operations, but none of their pairs can be read by user transactions.
\end{definition}

\begin{definition}[]
A \textbf{write-only constraint} is applied for all new \(insert\), \(delete\), and \(update\) operations, but
it is not guaranteed to hold over all existing data.
\end{definition}
\subsection{Database Consistency}
\label{sec:org3b4806a}
\begin{definition}[]
A database representation \(d\) is \textbf{consistent w.r.t. schema \(S\)} iff
\begin{enumerate}
\item \textbf{No column values exist without containing  row and table}. For every column key-value pair
\(\la k_r(C),v_r(C)\ra\in d\), there exists \(\la k_r(exists),null\ra\in d\) and there exists table
\(R\in S\) containing column \(C\).
\end{enumerate}
\end{definition}
\section{Problems}
\label{sec:org86ffa06}


\section{References}
\label{sec:org95be0a3}
\label{bibliographystyle link}
\bibliographystyle{alpha}

\label{bibliography link}
\bibliography{../../../references}
\end{document}
