% Created 2024-05-13 Mon 15:19
% Intended LaTeX compiler: xelatex
\documentclass[11pt]{article}
\usepackage{hyperref}
% TIPS
% \substack{a\\b} for multiple lines text





% pdfplots will load xolor automatically without option
\usepackage[dvipsnames]{xcolor}

\usepackage{forest}
% two-line text in node by [two \\ lines]
% \begin{forest} qtree, [..] \end{forest}
\forestset{
  qtree/.style={
    baseline,
    for tree={
      parent anchor=south,
      child anchor=north,
      align=center,
      inner sep=1pt,
    }}}
%\usepackage{flexisym}
% load order of mathtools and mathabx, otherwise conflict overbrace

\usepackage{mathtools}
%\usepackage{fourier}
\usepackage{pgfplots}
\usepackage{amsthm, mathabx,  amsmath, commath}
\usepackage{amsfonts}

\usepackage{empheq}
\usepackage{tikz}
\usetikzlibrary{arrows.meta}
\usepackage[most]{tcolorbox}

\newtheorem{theorem}{Theorem}[section]
\newtheorem{definition}{Definition}[section]
\newtheorem{corollary}{Corollary}[section]
\newtheorem{example}{Example}[section]
\newtheorem{lemma}{Lemma}[section]
\newtheorem{proposition}{Proposition}[section]

\newcommand{\bl}[1] {\boldsymbol{#1}}
\newcommand{\Wt}[1] {\stackrel{\sim}{\smash{#1}\rule{0pt}{1.1ex}}}
\newcommand{\wt}[1] {\widetilde{#1}}


%For boxed texts in align, use Aboxed{}
%otherwise use boxed{}

\DeclareMathSymbol{\widehatsym}{\mathord}{largesymbols}{"62}
\newcommand\lowerwidehatsym{%
  \text{\smash{\raisebox{-1.3ex}{%
    $\widehatsym$}}}}
\newcommand\fixwidehat[1]{%
  \mathchoice
    {\accentset{\displaystyle\lowerwidehatsym}{#1}}
    {\accentset{\textstyle\lowerwidehatsym}{#1}}
    {\accentset{\scriptstyle\lowerwidehatsym}{#1}}
    {\accentset{\scriptscriptstyle\lowerwidehatsym}{#1}}
}

\usepackage{graphicx}
    
% text on arrow for xRightarrow
\makeatletter
%\newcommand{\xRightarrow}[2][]{\ext@arrow 0359\Rightarrowfill@{#1}{#2}}
\makeatother


\def \bx {\boldsymbol{x}}
\def \ba {\boldsymbol{a}}
\def \bI {\boldsymbol{I}}
\def \bt {\boldsymbol{t}}
\def \bb {\boldsymbol{b}}
\def \bA {\boldsymbol{A}}
\def \bX {\boldsymbol{X}}
\def \bu {\boldsymbol{u}}
\def \bS {\boldsymbol{S}}
\def \bZ {\boldsymbol{Z}}
\def \bz {\boldsymbol{z}}
\def \by {\boldsymbol{y}}
\def \bw {\boldsymbol{w}}
\def \bT {\boldsymbol{T}}
\def \bS {\boldsymbol{S}}
\def \bm {\boldsymbol{m}}
\def \bW {\boldsymbol{W}}
\def \bY {\boldsymbol{Y}}
\def \bH {\boldsymbol{H}}
\def \blambda {\boldsymbol{\lambda}}
\def \bPhi {\boldsymbol{\Phi}}
\def \btheta {\boldsymbol{\theta}}
\def \bmu {\boldsymbol{\mu}}
\def \bphi {\boldsymbol{\phi}}
\def \bSigma {\boldsymbol{\Sigma}}
\def \lb {\left\{}
\def \rb {\right\}}
\def \caln {\mathcal{N}}
\def \dissum {\displaystyle\Sigma}
\def \dispro {\displaystyle\prod}
\def \E {\mathbb{E}}
\def \Q {\mathbb{Q}}
\def \V {\mathbb{V}}
\def \R {\mathbb{R}}
\def \calq {\mathcal{Q}}
\def \calg {\mathcal{G}}
\def \caln {\mathcal{N}}
\def \calr {\mathcal{R}}
\def \calm {\mathcal{M}}
\def \calc {\mathcal{C}}
\def \bcup {\bigcup}

\graphicspath{{../../../paper/transaction/}}
\usepackage{minted}

%% ox-latex features:
%   !announce-start, !guess-pollyglossia, !guess-babel, !guess-inputenc, caption,
%   image, !announce-end.

\usepackage{capt-of}

\usepackage{graphicx}

%% end ox-latex features


\author{wu}
\date{\today}
\title{Large-scale Incremental Processing Using Distributed Transactions and Notifications}
\hypersetup{
 pdfauthor={wu},
 pdftitle={Large-scale Incremental Processing Using Distributed Transactions and Notifications},
 pdfkeywords={},
 pdfsubject={},
 pdfcreator={Emacs 29.1 (Org mode 9.7-pre)}, 
 pdflang={English}}
\begin{document}

\maketitle
\section{Abstract}
\label{sec:org90454ea}
Target: a class of data processing tasks that transform a large repository of data via small,
independent mutations.

We have built Percolator, a system for \textbf{incrementally processing updates to a large data set}, and
deployed it to create the Google web search index.
\section{Introduction}
\label{sec:orgc28287e}
The indexing system could store the repository in a DBMS and update individual documents while using
transactions to maintain invariants.

An ideal data processing system for the task of maintaining the web search index would be optimized
for \textbf{incremental processing}; that is, it would allow us to maintain a very large repository of
documents and update it efficiently as each new document was crawled. Given that the system will be
processing many small updates concurrently, an ideal system would also provide mechanisms for
maintaining invariants despite concurrent updates and for keeping track of which updates have been
processed.


The remainder of this paper describes a particular incremental processing system: Percolator:
\begin{itemize}
\item Percolator provides the user with random access to a multi-PB repository. To achieve high
throughput, many threads on many machines need to transform the repository concurrently, so
Percolator provides ACID-compliant transactions to make it easier for programmers to reason about the state of the repository;
\item programmers of an incremental system need to keep track of the state of the incremental computation.
To assist them in this task, Percolator provides observers: pieces of code that are invoked by the
system whenever a user-specified column changes.

Percolator applications are structured as a series of observers; each observer completes a task and
creates more work for ``downstream'' observers by writing to the table. An external process triggers
the first observer in the chain by writing initial data into the table.
\item Requirements:
\begin{enumerate}
\item Computations where the result can’t be broken down into small updates (sorting a file, for example) are better handled by MapReduce.
\item the computation should have strong consistency requirements; otherwise, Bigtable is sufficient.
\item the computation should be very large in some dimension (total data size, CPU required for
transformation, etc.);
\end{enumerate}
\end{itemize}
\section{Design}
\label{sec:orga8a5c6c}
Percolator provides two main abstractions for performing incremental processing at large scale:
\begin{enumerate}
\item ACID transactions over a random-access repository
\item observers, a way to organize an incremental computation.
\end{enumerate}

A Percolator system consists of three binaries that run on every machine in the cluster: a Percolator
worker, a Bigtable tablet server, and a GFS chunkserver. All observers are linked into the Percolator
worker, which scans the Bigtable for changed columns (“notifications”) and invokes the corresponding
observers as a function call in the worker process.

The observers perform transactions by sending read/write RPCs to Bigtable tablet servers, which in
turn send read/write RPCs to GFS chunkservers. The system also depends on two small services: the
timestamp oracle and the lightweight lock service. The timestamp oracle provides strictly increasing
timestamps: a property required for correct operation of the snapshot isolation protocol. Workers use
the lightweight lock service to make the search for dirty notifications more efficient.

The design of Percolator was influenced by the requirement to run at massive scales and the lack of a
requirement for extremely low latency.  Percolator has no central location for transaction management;
in particular, it lacks a global deadlock detector.
\subsection{Bigtable overview}
\label{sec:org12d3640}
\subsection{Transactions}
\label{sec:org3b5eeb4}
Percolator provides cross-row, cross-table transactions with ACID snapshot-isolation semantics.

\begin{center}
\includegraphics[width=.8\textwidth]{../../images/papers/3.png}
\label{}
\end{center}

If \texttt{Commit()} return false, the transaction has conflicted (in this case, because two URLs with the same
content hash were processed simultaneously) and should be retried after a backoff. Calls to \texttt{Get()} and
\texttt{Commit()} are blocking; parallelism is achieved by running many transactions simultaneously in a thread
pool.

transactions make it more tractable for the user to reason about the state of the system and to avoid
the introduction of errors into a long-lived repository: For example, in a transactional web-indexing
system the programmer can make assumptions like: the hash of the contents of a document is always
consistent with the table that indexes duplicates. \label{Problem1}
Without transactions, an ill-timed crash could result in a permanent error: an entry in the document
table that corresponds to no URL in the duplicates table.
Note that both of these examples require transactions that \textbf{span rows}, rather than the single-row
transactions that Bigtable already provides.

Percolator stores multiple versions of each data item using Bigtable’s timestamp dimension.
\begin{center}
\includegraphics[width=.8\textwidth]{../../images/papers/4.png}
\label{}
\end{center}

Any node in Percolator can (and does) issue requests to directly modify state in Bigtable: there is no
convenient place to intercept traffic and assign locks. As a result, Percolator must explicitly
maintain locks.

Locks should:
\begin{enumerate}
\item Locks must persist in the face of machine failure; if a lock could disappear between the two phases
of commit, the system could mistakenly commit two transactions that should have conflicted.
\item provide high throughput; thousands of machines will be requesting locks simultaneously.
\item be low-latency; each \texttt{Get()} operation requires reading locks in addition to data, and we prefer to
minimize this latency.
\end{enumerate}

Given these requirements, the lock server will need to be replicated (to survive failure), distributed
and balanced (to handle load), and write to a persistent data store.

\begin{minted}[]{c++}
class Transaction {
    struct Write { Row row; Column col; string value; };
    vector<Write> writes_;
    int start_ts_;
    Transaction() : start_ts_(oracle.GetTimestamp()) {}
    void Set(Write w) { writes_.push back(w); }
    bool Get(Row row, Column c, string* value) {
        while (true) {
            bigtable::Txn T = bigtable::StartRowTransaction(row);
            // Check for locks that signal concurrent writes.
            if (T.Read(row, c+"lock", [0, start_ts_])) {
                // There is a pending lock; try to clean it and wait
                BackoffAndMaybeCleanupLock(row, c);
                continue;
            }
            // Find the latest write below our start_timestamp.
            latest write = T.Read(row, c+"write", [0, start_ts_]);
            if (!latest write.found()) return false; // no data
            int data_ts = latest write.start timestamp();
            *value = T.Read(row, c+"data", [data_ts, data_ts]);
            return true;
        }
    }
    // Prewrite tries to lock cell w, returning false in case of conflict.
    bool Prewrite(Write w, Write primary) {
        Column c = w.col;
        bigtable::Txn T = bigtable::StartRowTransaction(w.row);

        // Abort on writes after our start timestamp . . .
        if (T.Read(w.row, c+"write", [start_ts_, ∞])) return false;
        // ... or locks at any timestamp.
        if (T.Read(w.row, c+"lock", [0, ∞])) return false;
        T.Write(w.row, c+"data", start_ts_, w.value);
        T.Write(w.row, c+"lock", start_ts_,
                {primary.row, primary.col});    // The primary’s location.
        return T.Commit();
    }
    bool Commit() {
        Write primary = writes_[0];
        vector<Write> secondaries(writes_.begin()+1, writes_.end());
        if (!Prewrite(primary, primary)) return false;
        for (Write w : secondaries)
            if (!Prewrite(w, primary)) return false;
        int commit_ts = oracle .GetTimestamp();
        // Commit primary first.
        Write p = primary;
        bigtable::Txn T = bigtable::StartRowTransaction(p.row);
        if (!T.Read(p.row, p.col+"lock", [start_ts_, start_ts_]))
            return false;
        // aborted while working
        T.Write(p.row, p.col+"write", commit_ts,
                start_ts_); // Pointer to data written at start_ts_.
        T.Erase(p.row, p.col+"lock", commit_ts);
        if (!T.Commit()) return false; // commit point
        // Second phase: write out write records for secondary cells.
        for (Write w : secondaries) {
            bigtable::Write(w.row, w.col+"write", commit_ts, start_ts_);
            bigtable::Erase(w.row, w.col+"lock", commit_ts);
        }
        return true;
    }
} // class Transaction
\end{minted}

\begin{center}
\includegraphics[height=\textheight]{../../images/papers/6.png}
\label{}
\end{center}

\begin{table}[htbp]
\caption{The columns in the Bigtable representation of a Percolator column named ``c''}
\centering
\begin{tabular}{l|l}
Column & Use\\
\hline
\texttt{c:lock} & An uncommitted transaction is writing this cell; the location of primary lock\\
\texttt{c:write} & committed data present; stores the Bigtable timestamp of the data\\
\texttt{c:data} & stores the data itself\\
\texttt{c:tabnotify} & Hint: observers may need to run\\
\texttt{c:ack\_()} & Observer ``O'' has run; stores start timestamp of successful last run\\
\end{tabular}
\end{table}

The transaction's constructor asks the timestamp oracle for a start timestamp, which determines the
consistent snapshot seen by \texttt{Get()}. Calls to \texttt{Set()} are buffered until commit time. The basic approach
for committing buffered writes is two-phase commit, which is coordinated by the client. Transactions
on different machines interact through row transactions on Bigtable tablet servers.

In the first phase of commit (``prewrite'')\label{Problem2}, we try to lock all the cells being written. The
transaction reads metadata to check for conflicts in each cell being written. There are two kinds of
conflicting metadata:
\begin{enumerate}
\item if the transaction sees another write record after its start timestamp, it aborts; this is the
write-write conflict that snapshot isolation guards against.
\item If the transaction sees another lock at any timestamp, it also aborts.

It’s possible that the other transaction is just being slow to release its lock after having
already committed below our start timestamp, but we consider this unlikely, so we abort.
\end{enumerate}

If no cells conflict, the transaction may commit and proceeds to the second phase. At the beginning of
the second phase, the client obtains the commit timestamp from the timestamp oracle. Then at each
cell (starting with the primary), the client releases its lock and make its write visible to readers
by replacing the lock with a write record. The write record indicates to readers that committed data
exists in this cell; it contains a pointer to the start timestamp where readers can find the actually
data.
Once the primary's write is visible (commit point), the transaction must commit since it has made a
write visible to readers.

If a client fails while a transaction is being committed, locks will be left behind. Percolator must
clean up those locks or they will cause future transactions to hang indefinitely. Percolator takes a
lazy approach to cleanup: when a transaction A encounters a conflicting lock left behind by
transaction B, A may determine that B has failed and erase its locks.

It is very difficult for A to be perfectly confident in its judgment that B is failed; as a result we
must avoid a race between A cleaning up B’s transaction and a not-actually-failed B committing the same
transaction. Percolator handles this by designating one cell in every transaction as a synchronizing
point for any commit or cleanup operations. This cell’s lock is called the \textbf{primary lock}. Both A and B
agree on which lock is primary (the location of the primary is written into the locks at all other cells).
\section{Problems}
\label{sec:orga35f004}
\begin{enumerate}
\item what if two concurrent writes in \texttt{prewrite}, e.g. two \texttt{T.Write(w.row, c+"data", start\_ts\_, w.value);},
the lock check does not do anything actually.
\end{enumerate}
\begin{center}
\begin{tabular}{ll}
Problems & Desc\\
\hline
\ref{Problem1} & what's duplicates\\
\ref{Problem2} & whats primary for\\
what if two writes & \\
\end{tabular}
\end{center}
\section{References}
\label{sec:org7bfa680}
\label{bibliographystyle link}
\bibliographystyle{alpha}

\label{bibliography link}
\bibliography{../../references}
\end{document}
