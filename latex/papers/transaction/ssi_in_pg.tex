% Created 2025-01-07 Tue 19:41
% Intended LaTeX compiler: xelatex
\documentclass[11pt]{article}
\usepackage{capt-of}
\usepackage{hyperref}
% TIPS
% \substack{a\\b} for multiple lines text





% pdfplots will load xolor automatically without option
\usepackage[dvipsnames]{xcolor}

\usepackage{forest}
% two-line text in node by [two \\ lines]
% \begin{forest} qtree, [..] \end{forest}
\forestset{
  qtree/.style={
    baseline,
    for tree={
      parent anchor=south,
      child anchor=north,
      align=center,
      inner sep=1pt,
    }}}
%\usepackage{flexisym}
% load order of mathtools and mathabx, otherwise conflict overbrace

\usepackage{mathtools}
%\usepackage{fourier}
\usepackage{pgfplots}
\usepackage{amsthm, mathabx,  amsmath, commath}
\usepackage{amsfonts}

\usepackage{empheq}
\usepackage{tikz}
\usetikzlibrary{arrows.meta}
\usepackage[most]{tcolorbox}

\newtheorem{theorem}{Theorem}[section]
\newtheorem{definition}{Definition}[section]
\newtheorem{corollary}{Corollary}[section]
\newtheorem{example}{Example}[section]
\newtheorem{lemma}{Lemma}[section]
\newtheorem{proposition}{Proposition}[section]

\newcommand{\bl}[1] {\boldsymbol{#1}}
\newcommand{\Wt}[1] {\stackrel{\sim}{\smash{#1}\rule{0pt}{1.1ex}}}
\newcommand{\wt}[1] {\widetilde{#1}}


%For boxed texts in align, use Aboxed{}
%otherwise use boxed{}

\DeclareMathSymbol{\widehatsym}{\mathord}{largesymbols}{"62}
\newcommand\lowerwidehatsym{%
  \text{\smash{\raisebox{-1.3ex}{%
    $\widehatsym$}}}}
\newcommand\fixwidehat[1]{%
  \mathchoice
    {\accentset{\displaystyle\lowerwidehatsym}{#1}}
    {\accentset{\textstyle\lowerwidehatsym}{#1}}
    {\accentset{\scriptstyle\lowerwidehatsym}{#1}}
    {\accentset{\scriptscriptstyle\lowerwidehatsym}{#1}}
}

\usepackage{graphicx}
    
% text on arrow for xRightarrow
\makeatletter
%\newcommand{\xRightarrow}[2][]{\ext@arrow 0359\Rightarrowfill@{#1}{#2}}
\makeatother


\def \bx {\boldsymbol{x}}
\def \ba {\boldsymbol{a}}
\def \bI {\boldsymbol{I}}
\def \bt {\boldsymbol{t}}
\def \bb {\boldsymbol{b}}
\def \bA {\boldsymbol{A}}
\def \bX {\boldsymbol{X}}
\def \bu {\boldsymbol{u}}
\def \bS {\boldsymbol{S}}
\def \bZ {\boldsymbol{Z}}
\def \bz {\boldsymbol{z}}
\def \by {\boldsymbol{y}}
\def \bw {\boldsymbol{w}}
\def \bT {\boldsymbol{T}}
\def \bS {\boldsymbol{S}}
\def \bm {\boldsymbol{m}}
\def \bW {\boldsymbol{W}}
\def \bY {\boldsymbol{Y}}
\def \bH {\boldsymbol{H}}
\def \blambda {\boldsymbol{\lambda}}
\def \bPhi {\boldsymbol{\Phi}}
\def \btheta {\boldsymbol{\theta}}
\def \bmu {\boldsymbol{\mu}}
\def \bphi {\boldsymbol{\phi}}
\def \bSigma {\boldsymbol{\Sigma}}
\def \lb {\left\{}
\def \rb {\right\}}
\def \caln {\mathcal{N}}
\def \dissum {\displaystyle\Sigma}
\def \dispro {\displaystyle\prod}
\def \E {\mathbb{E}}
\def \Q {\mathbb{Q}}
\def \V {\mathbb{V}}
\def \R {\mathbb{R}}
\def \calq {\mathcal{Q}}
\def \calg {\mathcal{G}}
\def \caln {\mathcal{N}}
\def \calr {\mathcal{R}}
\def \calm {\mathcal{M}}
\def \calc {\mathcal{C}}
\def \bcup {\bigcup}

\graphicspath{{../../../paper/transaction/}}

%% ox-latex features:
%   !announce-start, !guess-pollyglossia, !guess-babel, !guess-inputenc, caption,
%   image, !announce-end.

\usepackage{capt-of}

\usepackage{graphicx}

%% end ox-latex features


\date{\today}
\title{Serializable Snapshot Isolation in PostgreSQL}
\hypersetup{
 pdfauthor={},
 pdftitle={Serializable Snapshot Isolation in PostgreSQL},
 pdfkeywords={},
 pdfsubject={},
 pdfcreator={Emacs 31.0.50 (Org mode 9.8-pre)},
 pdflang={English}}
\begin{document}

\maketitle
The \href{https://github.com/postgres/postgres/commit/dafaa3efb75ce1aae2e6dbefaf6f3a889dea0d21}{pr}. The \href{https://www.postgresql.org/message-id/flat/4A0019EE.EE98.0025.0\%40wicourts.gov}{discussion}. \href{https://www.postgresql.org/message-id/flat/4A1D5D8C.EE98.0025.1\%40wicourts.gov}{Discussion 2}.
The problem in mind
\begin{enumerate}
\item There is abstract, conceptual agreement that support for serializable transactions would be A Good
Thing.
\item There is doubt that an acceptably performant implementation is possible in PostgreSQL.
\item Some, but not all, don't want to see an implementation which produces false positive serialization
faults with some causes, but will accept them for other causes.
\item Nobody believes that an implementation with acceptable performance is possible without the disputed
false positives mentioned in (3).
\item There is particular concern about how to handle repeated rollbacks gracefully if we use the non-blocking technique.
\item There is particular concern about how to protect long-running transactions from rollback.  (I'm not
sure those concerns are confined to the new technique.)
\item Some, but not all, feel that it would be beneficial to have a correct implementation (no false
negatives) even if it had significant false positives, as it would allow iterative refinement of
the locking techniques.
\item One or two people feel that there would be benefit to an implementation which reduces the false
negatives, even if it doesn't eliminate them entirely.  (Especially if this could be a step toward
a full implementation.)
\end{enumerate}
\section{Snapshot Isolation}
\label{sec:org47d8c59}

\subsection{Example 1: Simple Write Skew}
\label{sec:org3f7d755}

\subsection{Example 2: Batch Processing}
\label{sec:org713261f}
Consider a transaction-processing system that maintains two tables. A \emph{receipts} table tracks the day’s
receipts, with each row tagged with the associated batch number. A separate \emph{control} table simply holds
the current batch number. There are three transaction types:
\begin{itemize}
\item NEW-RECEIPT: reads the current batch number from the control table, then inserts a new entry in the
receipts table tagged with that batch number
\item CLOSE-BATCH: increments the current batch number in the control table
\item REPORT: reads the current batch number from the control table, then reads all entries from the
receipts table with the previous batch number (i.e. to display a total of the previous day’s
receipts)
\end{itemize}


The following useful invariant holds under serializable executions: after a REPORT transaction has
shown the total for a particular batch, subsequent transactions cannot change that total.
\begin{center}
\includegraphics[width=.5\textwidth]{../../images/papers/74.png}
\label{2}
\end{center}
\section{Serializable Snapshot Isolation}
\label{sec:org911e8e3}

\subsection{Snapshot Isolation Anomalies}
\label{sec:org0743ec4}
Adya proposed representing an execution with a multi-version serialization history graph. This graph
contains a node per transaction, and an edge from transaction \(T_1\) to transaction \(T_2\) if
\(T_1\) must have preceded \(T_2\) in the apparent serial order of execution. Three types of
dependencies can create these edges:
\begin{itemize}
\item \textbf{wr-dependencies}:
\item \textbf{ww-dependencies}:
\item \textbf{rw-dependencies}: if \(T_1\) writes a version of some object, and \(T_2\) reads the previous version
of that object, then \(T_1\) appears to have executed after \(T_2\) , because \(T_2\) did not see its update.
\end{itemize}


\begin{center}
\includegraphics[width=.5\textwidth]{../../images/papers/75.png}
\label{3}
\end{center}
\subsection{Serializability Theory}
\label{sec:orge877114}
Note that a wr-dependency from \(A\) to \(B\) means that \(A\) must have committed before  \(B\)
began, as this is required for \(A\)’s changes to be visible to \(B\)’s snapshot. The same is true of
ww-dependencies because of write locking. However, rw-antidependencies occur between concurrent
transactions: one must start while the other was active. Therefore, they play an important role in SI
anomalies.

\begin{theorem}[]
\label{thm1}
Every cyle in the serialization history graph contains a sequence of edges
\(T_1\xrightarrow{rw}T_2\xrightarrow{rw}T_3\) where each edge is a rw-dependency. Furthermore, \(T_3\)
must be the first transaction in the cycle to commit.
\end{theorem}

\begin{corollary}[]
Transaction \(T_1\) is concurrent with \(T_2\), and \(T_2\) is concurrent with \(T_3\), because
rw-antidependencies occur only between concurrent transactions.
\end{corollary}
\subsection{SSI}
\label{sec:org5967c8b}
The SSI paper describes a method for identifying these dependencies by having transactions acquire
locks in a special “SIREAD” mode on the data they read. These locks do not block conflicting writes
(thus, “lock” is somewhat of a misnomer). Rather, a conflict between a SIREAD lock and a write lock
flags an rw-antidependency, which might cause a transaction to be aborted. Furthermore, SIREAD locks
must persist after a transaction commits, because conflicts can occur even after the reader has
committed (e.g. the \(T_1\xrightarrow{rw}T_2\) conflict in Example 2). Corollary 2 implies that the
locks must be retained until all concurrent transactions commit. Our PostgreSQL implementation uses
SIREAD locks, but their implementation differs significantly because PostgreSQL was purely
snapshot-based, as we describe in Section 5.2.
\subsubsection{Varaints on SSI}
\label{sec:org3b2ce0b}
Subsequent work has suggested refinements to the basic SSI approach. Cahill’s thesis suggests a commit
ordering optimization that can reduce false positives. Theorem 1 actually shows that every cycle
contains a dangerous structure \(T_1\xrightarrow{rw}T_2\xrightarrow{rw}T_3\) , where \(T_3\) is the
first to commit. Thus, even if a dangerous structure is found, no aborts are necessary if either
\(T_1\) or \(T_2\) commits before \(T_3\) . Verifying this condition requires tracking some additional
state, but avoids some false positive aborts. We use an extension of this optimization in PostgreSQL.
It does not, however, eliminate all \(T_1\) that closes the false positives: there may not be a path
\(T_3\) cycle. For example, in Example 2, if \(T_1\)'s REPORT accessed only the receipts table (not
the current batch number), there would be no wr-dependency from T3 to T1 , and the execution would be
serializable with order \(\la T_1,T_2,T_3\ra\). However, the dangerous structure of
rw-antidependencies \(T_1\xrightarrow{rw}T_2\xrightarrow{rw}T_3\) would force some transaction to be
spuriously aborted.
\section{Read-only Optimizations}
\label{sec:org31e5203}
We improve performance for read-only transactions in two ways. Both derive from a new serializability
theory result that characterizes when read-only transactions can be involved in SI  anomalies.
\begin{enumerate}
\item the theory enables a \emph{read-only snapshot ordering optimization} to reduce the false-positive abort rate
\item we also identify certain safe snapshots on which read-only transactions can execute safely without
any SSI overhead or abort risk, and introduce deferrable transactions, which delay their execution
to ensure they run on safe snapshots.
\end{enumerate}
\subsection{Theory}
\label{sec:orgc969404}
\begin{theorem}[]
Every serialization anomaly contains a dangerous structure
\(T_1\xrightarrow{rw}T_2\xrightarrow{rw}T_3\), where if \(T_1\) is read-only, \(T_3\) must have
committed before \(T_1\) took its snapshot.
\end{theorem}

\begin{proof}
Consider a cycle in the serialization history graph. From Theorem \ref{thm1}, we know it must have a
dangerous structure \(T_1\xrightarrow{rw}T_2\xrightarrow{rw}T_3\) where \(T_3\)is the first
transaction in the cycle to commit. Consider the case where \(T_1\) is read-only.

Because there is a cycle, there must be some transaction \(T_0\) that precedes \(T_1\) in the cycle.
The edge \(T_0\to T_1\) can't be a rw-antidependency or a ww-dependency, because \(T_1\) was
read-only, so it must be a wr-dependency. A wr-dependency means that \(T_0\)'s change were visible to
\(T_1\), so \(T_0\) must have committed before \(T_1\) took its snapshot. Because \(T_3\) is the first
transaction in the cycle to commit, it must commit before \(T_0\) commits, and therefore before
\(T_1\) takes its snapshot.
\end{proof}

Therefore if a dangerous structure is detected where \(T_1\) is read-only, it can be disregarded as a
false positive unless \(T_3\) committed before \(T_1\)'s snapshot.
\subsection{Safe Snapshots}
\label{sec:orga7a90e5}
A read-only transaction \(T_1\) cannot have a rw-conflict pointing in, as it did not perform any
writes. The only way it can be part of a dangerous structure, therefore, is if it has a conflict out
to a concurrent read/write transaction \(T_2\) , and \(T_2\) has a conflict out to a third transaction
\(T_3\) that committed before \(T_1\)’s snapshot. If no such \(T_2\) exists, then \(T_1\) will never
cause a serialization failure. This depends only on the concurrent transactions, not on \(T_1\)’s
behavior; therefore, we describe it as a property of the snapshot:

\textbf{Safe snapshots}: A read-only transaction \(T\) has a \textbf{safe snapshot} if no concurrent read/write
transaction has committed with a rw-antidependency out to a transaction that committed before \(T\)'s
snapshot, or has the possibility to do so.

An unusual property of this definition is that we cannot determine whether a snapshot is safe at the
time it is taken, only once all concurrent read/write transactions complete, as those transactions
might subsequently develop conflicts. Therefore, when a \texttt{READ ONLY} transaction is started, PostgreSQL
makes a list of concurrent transactions. The read-only transaction executes as normal, maintaining
SIREAD locks and other SSI state, until those transactions commit. After they have committed, if the
snapshot is deemed safe, the read-only transaction can drop its SIREAD locks, essentially becoming a
REPEATABLE READ (snapshot isolation) transaction.
\subsection{Deferrable Transactions}
\label{sec:org5907589}
Some workloads contain long-running read-only transactions and take more SIREAD locks.

These transactions would especially benefit from running on safe snapshots: they could avoid taking
SIREAD locks, they would be guaranteed not to abort, and they would not prevent concurrent
transactions from releasing their locks. \textbf{Deferrable transactions}, a new feature, provide a way to
ensure that complex read-only transactions will always run on a safe snapshot. Read-only serializable
transactions can be marked as deferrable with a new keyword, e.g. BEGIN TRANSACTION READ ONLY,
DEFERRABLE. Deferrable transactions always run on a safe snapshot, but may block before their first
query.
\section{Implementing SSI in PostgreSQL}
\label{sec:org08612a1}
\subsection{PostgreSQL Background}
\label{sec:orgfda58f0}
        All queries in PostgreSQL are performed with respect to a snapshot, which is represented as the set of transactions whose effects
are visible in the snapshot. Each tuple is tagged with the transaction
ID of the transaction that created it (xmin), and, if it has been deleted
or replaced with a new version, the transaction that did so (xmax).
Checking which of these transactions are included in a snapshot
determines whether the tuple should be visible. Updating a tuple
is, in most respects, identical to deleting the existing version and
creating a new tuple. The new tuple has a separate location in the
heap, and may have separate index entries.2 Here, PostgreSQL dif-
fers from other MVCC implementations (e.g. Oracle’s) that update
tuples in-place and keep a separate rollback log.
\section{Problems}
\label{sec:org4d16092}


\section{References}
\label{sec:org9f9842a}
\label{bibliographystyle link}
\bibliographystyle{alpha}

\label{bibliography link}
\bibliography{../../../references}
\end{document}
