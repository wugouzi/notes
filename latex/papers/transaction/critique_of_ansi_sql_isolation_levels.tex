% Created 2025-03-18 Tue 20:06
% Intended LaTeX compiler: xelatex
\documentclass[11pt]{article}
\usepackage{capt-of}
\usepackage{hyperref}
% TIPS
% \substack{a\\b} for multiple lines text





% pdfplots will load xolor automatically without option
\usepackage[dvipsnames]{xcolor}

\usepackage{forest}
% two-line text in node by [two \\ lines]
% \begin{forest} qtree, [..] \end{forest}
\forestset{
  qtree/.style={
    baseline,
    for tree={
      parent anchor=south,
      child anchor=north,
      align=center,
      inner sep=1pt,
    }}}
%\usepackage{flexisym}
% load order of mathtools and mathabx, otherwise conflict overbrace

\usepackage{mathtools}
%\usepackage{fourier}
\usepackage{pgfplots}
\usepackage{amsthm, mathabx,  amsmath, commath}
\usepackage{amsfonts}

\usepackage{empheq}
\usepackage{tikz}
\usetikzlibrary{arrows.meta}
\usepackage[most]{tcolorbox}

\newtheorem{theorem}{Theorem}[section]
\newtheorem{definition}{Definition}[section]
\newtheorem{corollary}{Corollary}[section]
\newtheorem{example}{Example}[section]
\newtheorem{lemma}{Lemma}[section]
\newtheorem{proposition}{Proposition}[section]

\newcommand{\bl}[1] {\boldsymbol{#1}}
\newcommand{\Wt}[1] {\stackrel{\sim}{\smash{#1}\rule{0pt}{1.1ex}}}
\newcommand{\wt}[1] {\widetilde{#1}}


%For boxed texts in align, use Aboxed{}
%otherwise use boxed{}

\DeclareMathSymbol{\widehatsym}{\mathord}{largesymbols}{"62}
\newcommand\lowerwidehatsym{%
  \text{\smash{\raisebox{-1.3ex}{%
    $\widehatsym$}}}}
\newcommand\fixwidehat[1]{%
  \mathchoice
    {\accentset{\displaystyle\lowerwidehatsym}{#1}}
    {\accentset{\textstyle\lowerwidehatsym}{#1}}
    {\accentset{\scriptstyle\lowerwidehatsym}{#1}}
    {\accentset{\scriptscriptstyle\lowerwidehatsym}{#1}}
}

\usepackage{graphicx}
    
% text on arrow for xRightarrow
\makeatletter
%\newcommand{\xRightarrow}[2][]{\ext@arrow 0359\Rightarrowfill@{#1}{#2}}
\makeatother


\def \bx {\boldsymbol{x}}
\def \ba {\boldsymbol{a}}
\def \bI {\boldsymbol{I}}
\def \bt {\boldsymbol{t}}
\def \bb {\boldsymbol{b}}
\def \bA {\boldsymbol{A}}
\def \bX {\boldsymbol{X}}
\def \bu {\boldsymbol{u}}
\def \bS {\boldsymbol{S}}
\def \bZ {\boldsymbol{Z}}
\def \bz {\boldsymbol{z}}
\def \by {\boldsymbol{y}}
\def \bw {\boldsymbol{w}}
\def \bT {\boldsymbol{T}}
\def \bS {\boldsymbol{S}}
\def \bm {\boldsymbol{m}}
\def \bW {\boldsymbol{W}}
\def \bY {\boldsymbol{Y}}
\def \bH {\boldsymbol{H}}
\def \blambda {\boldsymbol{\lambda}}
\def \bPhi {\boldsymbol{\Phi}}
\def \btheta {\boldsymbol{\theta}}
\def \bmu {\boldsymbol{\mu}}
\def \bphi {\boldsymbol{\phi}}
\def \bSigma {\boldsymbol{\Sigma}}
\def \lb {\left\{}
\def \rb {\right\}}
\def \caln {\mathcal{N}}
\def \dissum {\displaystyle\Sigma}
\def \dispro {\displaystyle\prod}
\def \E {\mathbb{E}}
\def \Q {\mathbb{Q}}
\def \V {\mathbb{V}}
\def \R {\mathbb{R}}
\def \calq {\mathcal{Q}}
\def \calg {\mathcal{G}}
\def \caln {\mathcal{N}}
\def \calr {\mathcal{R}}
\def \calm {\mathcal{M}}
\def \calc {\mathcal{C}}
\def \bcup {\bigcup}

\graphicspath{{../../../paper/transaction/}}

%% ox-latex features:
%   !announce-start, !guess-pollyglossia, !guess-babel, !guess-inputenc, caption,
%   image, !announce-end.

\usepackage{capt-of}

\usepackage{graphicx}

%% end ox-latex features


\date{\today}
\title{A Critique Of Ansi SQL Isolation Levels}
\hypersetup{
 pdfauthor={},
 pdftitle={A Critique Of Ansi SQL Isolation Levels},
 pdfkeywords={},
 pdfsubject={},
 pdfcreator={Emacs 31.0.50 (Org mode 9.8-pre)},
 pdflang={English}}
\begin{document}

\maketitle
\section{Introduction}
\label{sec:orge8c2f78}
The ANSI/ISO SQL-92 specifications define four isolation levels:
\begin{enumerate}
\item READ UNCOMMITTED
\item READ COMMITTED
\item REPEATABLE READ
\item SERIALIZABLE.
\end{enumerate}

These levels are defined with the classical serializability definition, plus three prohibited action
subsequences, called phenomena: Dirty Read, Non-repeatable Read, and Phantom.

This paper shows a number of weaknesses in the anomaly approach to defining isolation levels. The
three ANSI phenomena are ambiguous.
\section{Isolation Definitions}
\label{sec:org60bb3a2}
\subsection{Serializability Concepts}
\label{sec:org302ad26}
A \textbf{transaction} groups a set of actions that transform the database from one consistent state to
another. A \textbf{history} models the interleaved execution of a set of transactions as a linear ordering of
their actions, such as Reads and Writes (i.e., inserts, updates, and deletes) of specific data items.
Two actions in a history are said to \textbf{conflict} if they are performed by distinct transactions on the
same data item and at least one of is a Write action.

A particular history gives rise to a \textbf{dependency graph} defining the temporal data flow among
transactions. The actions of committed transactions in the history are represented as graph nodes. If
action \(op_1\) of transaction \(T_1\) conflicts with and precedes action \(op_2\) of transaction
\(T_2\) in the history, then the pair \(\la op_1,op_2\ra\) becomes an edge in the dependency graph.
Two histories are \textbf{equivalent} if they have the same committed transactions and the same dependency
graph. A history is \textbf{serializable} if it is equivalent to a serial history — that is, if it has the same
dependency graph (inter-transaction temporal data flow) as some history that executes transactions one
at a time in sequence.
\subsection{ANSI SQL Isolation Levels}
\label{sec:org8940c45}
\begin{itemize}
\item P1 (Dirty Read): Transaction \(T_1\) modifies a data item. Another transaction \(T_2\) then reads
that data item before \(T_1\) performs a \texttt{COMMIT} or \texttt{ROLLBACK}. If \(T_1\) then performs a \texttt{ROLLBACK},
\(T_2\) has read a data item that was never committed and so never really existed.
\item P2 (Non-repeatable or Fuzzy Read): Transaction \(T_1\) reads a data item. Another transaction
\(T_2\) then modifies or deletes that data item and commits. If \(T_1\) then attempts to reread the
data item, it receives a modified value or discovers that the data item has been deleted.
\item P3 (Phantom): Transaction \(T_1\) reads a set of data items satisfying some \texttt{<search condition>}.
Transaction \(T_2\) then creates data items that satisfy \(T_1\)’s \texttt{<search condition>} and commits.
If \(T_1\) then repeats its read with the same \texttt{<search condition>}, it gets a set of data items
different from the first read.
\end{itemize}

Histories consisting of reads, writes, commits, and aborts can be written in a shorthand notation:
\(w_1[x]\) means a write by transaction 1 on data item \(x\) , and \(r_2[x]\) represents a read of
\(x\) by transaction 2. Transaction 1 reading and writing a set of records satisfying predicate \(P\)
is denoted by \(r_1[P]\) and \(w_1[P]\) respectively. Transaction 1’s commit and abort (\texttt{ROLLBACK}) are
written \(c_1\) and \(a_1\), respectively.

P1 might be restated as disallowing the following scenario:
\begin{equation}
\label{2.1}
w_1[x]\dots r_2[x]\dots(a_1\text{ and }c_2\text{ in any order})
\end{equation}
But P1 does not insist that \(T_1\); it simply states that if this happens something might occur.
Therefoe we can also interpret P1 as:
\begin{equation}
\label{2.2}
w_1[x]\dots r_2[x]\dots((c_1\text{ or }a_1)\text{ and }(c_2\text{ or }a_2)\text{ in any order})
\end{equation}
Forbidding \eqref{2.2} variant of P1 disallows any history where \(T_1\) modifes a data item \(x\), then
\(T_2\) reads the data item before \(T_1\) commits or aborts. It does not insist that \(T_1\) aborts
or that \(T_2\) commits.

We call \eqref{2.1} the \textbf{strict interpretation} of P1 and \eqref{2.2} the \textbf{broad interpretation} of P1.
Interpretation \eqref{2.2} specifies a phenomenon that might lead to an anomaly, while \eqref{2.1}
specifies an actual anomaly. Denote them as P1 and A1 respectively.

Similarly, the English language phenomena P2 and P3 have strict and broad interpretations, and are
denoeted as:
\begin{itemize}
\item P2: \(r_1[x]\dots w_2[x]\dots((c_1\text{ or }a_1)\text{ and }(c_2\text{ or }a_2)\text{ in any
         order})\)
\item A2: \(r_1[x]\dots w_2[x]\dots c_2\dots r_1[x]\dots c_1\)
\item P3: \(r_1[P]\dots w_2[y\in P]\dots((c_1\text{ or }a_1)\text{ and }(c_2\text{ or }a_2)\text{ in any
          order})\)
\item A3: \(r_1[P]\dots w_2[y\in P]\dots c_2\dots r_1[P]\dots c_1\)
\end{itemize}


ANSI SQL defines four levels of isolation by the matrix of Table 1
\begin{table}[htbp]
\label{1}
\centering
\begin{tabular}{llll}
\hline
Isolation Level & P1 & P2 & P3\\
\hline
ANSI READ UNCOMMITTED & Possible & Possible & Possible\\
ANSI READ COMMITTED & Not Possible & Possible & Possible\\
ANSI REPEATABLE READ & Not Possible & Not Possible & Possible\\
ANOMALY SERIALIZABLE & Not Possible & Not Possible & Not Possible\\
\hline
\end{tabular}
\end{table}
\subsection{Locking}
\label{sec:org94bd97e}
Transactions executing under a locking scheduler request Read (Share) and Write (Exclusive) locks on
data items or sets of data items they read and write. Two locks by different transactions on the same
item \textbf{conflict} if at least one of them is a Write lock.

A Read (resp. Write) predicate lock on a given \texttt{<search condition>} is effectively a lock on all data
items satisfying the \texttt{<search condition>}. This may be an infinite set. It includes data present in the
database and also any phantom data items not currently in the database but that would satisfy the
predicate if they were inserted or if current data items were updated to satisfy the
\texttt{<search condition>}. In SQL terms, a predicate lock covers alltuples that satisfy the predicate and any
that an \texttt{INSERT}, \texttt{UPDATE}, or \texttt{DELETE} statement would cause to satisfy the predicate. Two predicate locks
by different transactions conflict if one is a Write lock and if there is a (possibly phantom) data
item covered by both locks. An item lock (record lock) is a predicate lock where the predicate names
the specific record.

A transaction has \textbf{well-formed writes} (reads) if it requests a Write (Read) lock on each data item or
predicate before writing (reading) that data item, or set of data items defined by a predicate. The
transaction is \textbf{well-formed} if it has well-formed writes and reads. A transaction has \textbf{two-phase writes}
(reads) if it does not set a new Write (Read) lock on a data item after releasing a Write (Read) lock.
A transaction exhibits \textbf{two-phase locking} if it does not request any new locks after releasing some
lock.

The locks requested by a transaction are of \textbf{long duration} if they are held until after the transaction
commits or aborts. Otherwise, they are of \textbf{short duration}. Typically, short locks are released
immediately after the action completes.

The fundamental serialization theorem is that well-formed two-phase locking guarantees serializability
— each history arising under two-phase locking is equivalent to some serial history. Conversely, if a
transaction is not well-formed or two-phased then, except in degenerate cases, non-serializable
execution histories are possible

Table \ref{2} defines a number of isolation types in terms of lock scopes (items or predicates), modes
(read or write), and their durations (short or long). We believe the isolation levels called Locking
READ UNCOMMITTED, Locking READ COMMITTED, Locking REPEATABLE READ, and Locking SERIALIZABLE are the
locking definitions intended by ANSI SQL Isolation levels

\begin{center}
\includegraphics[width=.7\textwidth]{../../images/papers/118.png}
\label{2}
\end{center}

\begin{definition}[]
Isolation level \(L_1\) is \textbf{weaker} than isolation level \(L_2\) (or \(L_2\) is stronger than \(L_1\)),
denoted \(L_1<L_2\), if all non-serializable histories that obey the criteria of L2 also satisfy L1
and there is at least one non-serializable history that can occur at level L1 but not at level L2. Two
isolation levels \(L_1\) and \(L_2\) are \textbf{equivalent}, denoted \(L_1=L_2\), if the sets of
non-serializable histories satisfying \(L_1\) and \(L_2\) are identical. \(L_1\) is \textbf{no stronger} than
\(L_2\), denoted \(L_1\le L_2\) if either \(L_1<L_2\) or \(L_1=L_2\). Two isolation levels are
\textbf{incomparable}, denoted \(L_1><L_2\), when each isolation level allows a non-serializable history that
is disallowed by the other.
\end{definition}

In comparing isolation levels we differentiate them only in terms of the non-serializable histories
that can occur  one but not the other. Two isolation levels can also differ in terms of the
serializable histories they permit, but we say  \(\text{Locking SERIALIZABLE}=\text{Serializable}\)
even though it is well known that a locking scheduler does not admit all possible Serializable
histories.
\begin{align*}
\text{Locking READ COMMITTED}&<\text{Locking READ COMMITTED}\\
&<\text{Locking REPEATABLE READ}\\
&<\text{Locking SERIALIZABLE}
\end{align*}
\section{Analyzing ANSI SQL Isolation Levels}
\label{sec:orgfc933a1}
Locking READ UNCOMMITTED provides long duration write locking to avoid a phenomenon called ``Dirty
Writes,'' but ANSI SQL does not exclude this anomalous behavior other than ANSI SERIALIZABLE. Dirty
writes are defined as follows:

\begin{itemize}
\item P0 (Dirty Write): Transaction \(T_1\) modifies a data item. Another transaction \(T_2\) then further
modifies that data item before \(T_1\) performs a \texttt{COMMIT} or \texttt{ROLLBACK}. If \(T_1\) or \(T_2\) then
performs a \texttt{ROLLBACK}, it is unclear whtat the correct data value should be
\end{itemize}


\begin{center}
P0: \(w_1[x]\dots w_2[x]\dots((c_1\text{ or }a_1)\text{ and }(c_2\text{ or }a_2)\text{ in any order})\)
\end{center}

Without protection from P0, the system can’t undo updates by restoring before images. Consider the
history: \(w_1[x]w_2[x]a_1\). You don’t want to undo \(w_1[x]\) by restoring its before-image of \(x\), because
that would wipe out \(w_2\)’s update. But if you don’t restore its before-image, and transaction \(T_2\) later aborts,
you can’t undo \(w_2[x]\) by restoring its before-image either!

So we conclude:
\begin{remark}[]
ANSI SQL isolation should be modified to require P0 for all isolation levels.
\end{remark}

Consider history \(H_1\):
\begin{equation*}
H_1:r_1[x=50]w_1[x=10]r_2[x=10]r_2[y=50]c_2r_1[y=50]w_1[y=90]c_1
\end{equation*}

\(H_1\) is non-serializable, the classical inconsistent analysis problem where transaction \(T_1\) is
transferring a quantity 40 from \(x\) to \(y\), maintaining a total balance of 100, but \(T_2\) reads
an inconsistent state where the total balance is 60.

But \(H_1\) does not violate any of the anomalies \(A_1\), \(A_2\) or \(A_3\). But consider instead
taking the broad interpretation of \(A_1\), the phenomenon \(P_1\):
\begin{equation*}
w_1[x]\dots r_2[x]\dots((c_1\text{ or }a_1)\text{ and }(c_2\text{ or }a_2)\text{ in any order})
\end{equation*}
\(H_1\)indeed violates \(P_1\).

Similar arguments show that \(P_2\) should be taken as the ANSI intention rather than \(A_2\). A
history that discriminates these two interpretations is:
\begin{equation*}
H_2:r_1[x=50]r_2[x=50]w_2[x=10]r_2[y=50]w_2[y=90]c_2r_1[y=90]c_1
\end{equation*}
\(H_2\) is non-serializable, where \(T_1\) sees a total balance of 140. This time neither transaction
reads dirty data. Thus \(P_1\) is satisfied. Once again, no data item is read twice nor is any
relevant predicate evaluation changed. The problem with \(H_2\) is that by the time \(T_1\) reads
\(y\), the value for \(x\) is out of date. If \(T_1\) were to read \(x\) again, it would have been
changed; but since \(T_1\) doesn't do that, \(A_2\) doesn't apply.

\wu{
In essence, \(T_1\) doesn't aware of \(T_2\)'s existence and can't determine if its read value is the newest.
}

Finally, consider
\begin{equation*}
H_3:r_1[P]w_2[\text{insert }y\in  P]r_2[z]w_2[z]c_2r_1[z]c_1
\end{equation*}
Here \(T_1\) performs a \texttt{<search condition>} to find the list of active employees. Then \(T_2\) performs an
insert of a new active employee and then updates \(z\), the count of employees in the company.
Following this, \(T_1\) reads the count of active employees as a check and sees a discrepancy. This
history is clearly not serializable, but is allowed by \(A_3\) since no predicate is evaluated twice.

\begin{remark}[]
Strict interpretations \(A_1\), \(A_2\) and \(A_3\) have unintended weakness. The correct
interpretations are the Broad ones. We assume in what follows that ANSI meant to define \(P_1\),
\(P_2\) and \(P_3\)
\end{remark}

\begin{remark}[]
\begin{itemize}
\item P0 (Dirty Write): \(w_1[x]\dots w_2[x]\dots(c_1\text{ or }a_1)\)
\item P1 (Dirty Read): \(w_1[x]\dots r_2[x]\dots(c_1\text{ or }a_1)\)
\item P2 (Fuzzy or Non-repeatable Read): \(r_1[x]\dots w_2[x]\dots(c_1\text{ or }a_1)\)
\item P3 (Phantom): \(r_1[P]\dots w_2[y\in P]\dots(c_1\text{ or }a_1)\)
\end{itemize}

The definition of proposed ANSI isolation levels in terms of these phenomena is given in Table \ref{3}.

For single version histories, it turns out that the P0, P1, P2, P3 phenomena are disguised versions of
locking.

\begin{itemize}
\item prohibiting P0 precludes a second transaction writing an item after the first transaction has
written it, equivalent to saying that long-term Write locks are held on data items (and predicates). Thus Dirty Writes are impossible at all levels.
\item Similarly, prohibiting P1 is equivalent to having well-formed reads on data items.
\item Prohibiting P2 means long-term Read locks on data items.
\item Finally, Prohibiting P3 means long-term Read predicate locks.
\end{itemize}

Thus the isolation levels of Table \ref{3} defined by these phenomena provide the same behavior as the Locking isolation levels of Table \ref{2}.

\begin{center}
\includegraphics[width=.9\textwidth]{../../images/papers/119.png}
\label{3}
\end{center}
\end{remark}

\begin{remark}[]
The locking isolation levels of Table \ref{2} and the phenomenological definitions of Table \ref{3} are
equivalent. Put another way, P0, P1, P2, and P3 are disguised redefinition’s of locking behavior.
\end{remark}
\section{Other Isolation Types}
\label{sec:org905a0f8}
\subsection{Cursor Stability}
\label{sec:org15cd7d7}
P4 (Lost Update): The lost update anomaly occurs when transaction \(T_1\) reads a data item and then
\(T_2\) updates the data item (possibly based on a previous read), then \(T_1\) (based on its earlier
read value) updates the data item and commits.

\begin{center}
P4 (Lost Update): \(r_1[x]\dots w_2[x]\dots w_1[x]\dots c_1\)
\end{center}

\begin{remark}[]
\(\text{READ COMMITTED} < \text{Cursor Stability}<\text{REPEATABLE READ}\)
\end{remark}
\subsection{Snapshot Isolation}
\label{sec:orge725d60}
Snapshot Isolation: each transaction read reads data from a snapshot of the (committed) data as of the
time the transaction started, called its Start-Timestamp.

When the transaction \(T_1\) is ready to commit, it gets a Commit-Timestamp, which is larger than any
existing Start-Timestamp or Commit-Timestamp. The transaction successfully commits only if no other
transaction \(T_2\) with a Commit-Timestamp in \(T_1\)’s execution interval [Start-Timestamp,
Commit-Timestamp] wrote data that T1 also wrote. Otherwise, T1 will abort. This feature, called
\textbf{First-committer-wins} prevents lost updates. When \(T_1\) commits, its changes become visible to all
transactions whose Start-Timestamps are larger than T1‘s Commit-Timestamp.

Snapshot Isolation is non-serializable because a transaction's Reads come at one instant and the
Writes at another. For example, consider the single-value history
\begin{equation*}
\text{H5:} r_1[x=50]r_1[y=50]r_2[x=50]r_2[y=50]w_1[y=-40]w_2[x=-40]c_1c_2
\end{equation*}
Here we assume that each transaction that writes a new value for x and y is expected to maintain the
constraint that x + y should be positive, and while T1 and T2 both act properly in isolation, the
constraint fails to hold in H5.

A5 (Data Item Constraint Violation). Suppose \(C()\) is a database constraint between two data items
\(x\) and \(y\) in the database. Here are two anomalies arising from constraint violation.

A5A Read Skew: Suppose transaction \(T_1\) reads \(x\), and then a second transaction \(T_2\) updates
\(x\) and \(y\) to new values and commits. If now \(T_1\) reads \(y\), it may see an inconsistent
state, and therefore produce an inconsistent state as output.
\begin{center}
A5A (Read Skew): \(r_1[x]\dots w_2[x]\dots w_2[y]\dots c_2\dots r_1[y](c_1\text{ or }a_1)\)
\end{center}

A5B Write Skew: Suppose \(T_1\) reads \(x\) and \(y\), which are consistent with \(C()\), and then a
\(T_2\) reads \(x\) and \(y\), writes \(x\), and commits. Then \(T_1\) writes \(y\). If there were a
constraint between \(x\) and \(y\), it might be violated.
\begin{center}
A5B (Write Skew): \(r_1[x]\dots r_2[y]\dots w_1[y]\dots w_2[x]\dots(c_1\text{ and }c_2\text{ occur})\)
\end{center}

Clearly neither A5A nor A5B could arise in histories where P2 is precluded, since both A5A and A5B
have T2 write a data item that has been previously read by an uncommitted T1. Thus, phenomena A5A and
A5B are only useful for distinguishing isolation levels that are below REPEATABLE READ in strength.

\begin{proposition}[]
\(\text{READ COMMITTED}<\text{Snapshot Isolation}\)
\end{proposition}

\begin{proof}
In Snapshot Isolation, first-committer-wins precludes P0 (dirty writes), and the timestamp mechanism
prevents P1 (dirty reads), so Snapshot Isolation is no weaker than READ COMMITTED. In addition, A5A is
possible under READ COMMITTED, but not under the Snapshot Isolation timestamp mechanism. Therefore
\(\text{READ COMMITTED}<\text{Snapshot Isolation}\)
\end{proof}

\begin{proposition}[]
\(\text{REPEATABLE READ}><\text{Snapshot Isolation}\)
\end{proposition}

\begin{proof}
Snapshot Isolation histories prohibit histories with anomaly A3, but allow A5B, while REPEATABLE READ does the opposite
\end{proof}

\begin{remark}[]
Snapshot Isolation histories preclude anomalies A1, A2 and A3. Therefore
\begin{equation*}
\text{ANOMALY SERIALIZABLE}<\text{Snapshot Isolation}
\end{equation*}
\end{remark}
\section{Summary}
\label{sec:orgc0a343f}
\begin{center}
\includegraphics[width=.8\textwidth]{../../images/papers/120.png}
\label{}
\end{center}
\section{Problems}
\label{sec:org22543d9}


\section{References}
\label{sec:org85e04c0}
\label{bibliographystyle link}
\bibliographystyle{alpha}

\bibliography{/Users/wu/notes/notes/references.bib}
\end{document}
