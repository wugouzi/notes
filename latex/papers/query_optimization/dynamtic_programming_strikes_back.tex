% Created 2025-02-18 Tue 13:28
% Intended LaTeX compiler: xelatex
\documentclass[11pt]{article}
\usepackage{capt-of}
\usepackage{hyperref}
% TIPS
% \substack{a\\b} for multiple lines text





% pdfplots will load xolor automatically without option
\usepackage[dvipsnames]{xcolor}

\usepackage{forest}
% two-line text in node by [two \\ lines]
% \begin{forest} qtree, [..] \end{forest}
\forestset{
  qtree/.style={
    baseline,
    for tree={
      parent anchor=south,
      child anchor=north,
      align=center,
      inner sep=1pt,
    }}}
%\usepackage{flexisym}
% load order of mathtools and mathabx, otherwise conflict overbrace

\usepackage{mathtools}
%\usepackage{fourier}
\usepackage{pgfplots}
\usepackage{amsthm, mathabx,  amsmath, commath}
\usepackage{amsfonts}

\usepackage{empheq}
\usepackage{tikz}
\usetikzlibrary{arrows.meta}
\usepackage[most]{tcolorbox}

\newtheorem{theorem}{Theorem}[section]
\newtheorem{definition}{Definition}[section]
\newtheorem{corollary}{Corollary}[section]
\newtheorem{example}{Example}[section]
\newtheorem{lemma}{Lemma}[section]
\newtheorem{proposition}{Proposition}[section]

\newcommand{\bl}[1] {\boldsymbol{#1}}
\newcommand{\Wt}[1] {\stackrel{\sim}{\smash{#1}\rule{0pt}{1.1ex}}}
\newcommand{\wt}[1] {\widetilde{#1}}


%For boxed texts in align, use Aboxed{}
%otherwise use boxed{}

\DeclareMathSymbol{\widehatsym}{\mathord}{largesymbols}{"62}
\newcommand\lowerwidehatsym{%
  \text{\smash{\raisebox{-1.3ex}{%
    $\widehatsym$}}}}
\newcommand\fixwidehat[1]{%
  \mathchoice
    {\accentset{\displaystyle\lowerwidehatsym}{#1}}
    {\accentset{\textstyle\lowerwidehatsym}{#1}}
    {\accentset{\scriptstyle\lowerwidehatsym}{#1}}
    {\accentset{\scriptscriptstyle\lowerwidehatsym}{#1}}
}

\usepackage{graphicx}
    
% text on arrow for xRightarrow
\makeatletter
%\newcommand{\xRightarrow}[2][]{\ext@arrow 0359\Rightarrowfill@{#1}{#2}}
\makeatother


\def \bx {\boldsymbol{x}}
\def \ba {\boldsymbol{a}}
\def \bI {\boldsymbol{I}}
\def \bt {\boldsymbol{t}}
\def \bb {\boldsymbol{b}}
\def \bA {\boldsymbol{A}}
\def \bX {\boldsymbol{X}}
\def \bu {\boldsymbol{u}}
\def \bS {\boldsymbol{S}}
\def \bZ {\boldsymbol{Z}}
\def \bz {\boldsymbol{z}}
\def \by {\boldsymbol{y}}
\def \bw {\boldsymbol{w}}
\def \bT {\boldsymbol{T}}
\def \bS {\boldsymbol{S}}
\def \bm {\boldsymbol{m}}
\def \bW {\boldsymbol{W}}
\def \bY {\boldsymbol{Y}}
\def \bH {\boldsymbol{H}}
\def \blambda {\boldsymbol{\lambda}}
\def \bPhi {\boldsymbol{\Phi}}
\def \btheta {\boldsymbol{\theta}}
\def \bmu {\boldsymbol{\mu}}
\def \bphi {\boldsymbol{\phi}}
\def \bSigma {\boldsymbol{\Sigma}}
\def \lb {\left\{}
\def \rb {\right\}}
\def \caln {\mathcal{N}}
\def \dissum {\displaystyle\Sigma}
\def \dispro {\displaystyle\prod}
\def \E {\mathbb{E}}
\def \Q {\mathbb{Q}}
\def \V {\mathbb{V}}
\def \R {\mathbb{R}}
\def \calq {\mathcal{Q}}
\def \calg {\mathcal{G}}
\def \caln {\mathcal{N}}
\def \calr {\mathcal{R}}
\def \calm {\mathcal{M}}
\def \calc {\mathcal{C}}
\def \bcup {\bigcup}

\graphicspath{{../../../paper/query_optimization/}}

%% ox-latex features:
%   !announce-start, !guess-pollyglossia, !guess-babel, !guess-inputenc, caption,
%   underline, image, !announce-end.

\usepackage{capt-of}

\usepackage[normalem]{ulem}

\usepackage{graphicx}

%% end ox-latex features


\date{\today}
\title{Dynamic Programming Strikes Back}
\hypersetup{
 pdfauthor={},
 pdftitle={Dynamic Programming Strikes Back},
 pdfkeywords={},
 pdfsubject={},
 pdfcreator={Emacs 31.0.50 (Org mode 9.8-pre)},
 pdflang={English}}
\begin{document}

\maketitle
\section{Introduction}
\label{sec:org7a2460f}
In this paper, we introduct \texttt{DPhyp}. Experiments will shwo that it is highly superior to existing
approaches.
\section{Hypergraphs}
\label{sec:org450b895}
\subsection{Definitions}
\label{sec:orgefef46d}
\begin{definition}[Hypergraph]
A \textbf{hypergraph} is a pair \(H=(V,E)\) s.t.
\begin{enumerate}
\item \(V\) is a non-empty set of nodes
\item \(E\) is a set of hyperedges, where a \textbf{hyperedge} is an unordered pair \((u,v)\) of non-empty subsets
of \(V\) (\(u\subset V\) and \(v\subset V\)) with the additional condition that \(u\cap
           v=\emptyset\)
\end{enumerate}

We call any non-empty subset of \(V\) a \textbf{hypernode}. We assume that the nodes in \(V\) are totally
ordered via an (arbitrary) relation \(\prec\). The ordering on nodes is important for our algorithm.

A hyperedge \((u,v)\) is \textbf{simple} if \(\abs{u}=\abs{v}=1\). A hypergraph is \textbf{simple} if all its hyperedges
are simple.
\end{definition}

In our context, the nodes of hypergraphs are relations and the edges are abstractions of join
predicates. Consider a join predicate of the form
\begin{equation*}
R_1.a+R_2.b+R_3.c=R_4.d+R_5.e+R_6.f
\end{equation*}
This predicate will result in a hyperedge \((\{R_1,R_2,R_3\}, \{R_4,R_5,R_6\})\).


\begin{center}
\includegraphics[width=.5\textwidth]{../../images/papers/94.png}
\captionof{figure}{\label{2}Sample hypergraph}
\end{center}

Fig \ref{2} contains an example of a hypergraph. The set \(V\) of nodes is \(V=\{R_1,\dots,R_6\}\). Concerning the node ordering, we assume that
\(R_i\prec R_j\Leftrightarrow i<j\). There are simple edges \((\{R_1\},\{R_2\})\),
\((\{R_2\},\{R_3\})\), \((\{R_4\},\{R_5\})\) nad \((\{R_5\},\{R_6\})\).

Note that is possible to rewrite the above complex join predicate. For example, it is equivalent to
\begin{equation*}
R_1.a+R_2.b=R_4.d+R_5.e+R_6.f-R_3.c
\end{equation*}
This leads to a hyperedge \((\{R_1,R_2\},\{R_3,R_4,R_5,R_6\})\).

\begin{definition}[Subgraph]
Let \(H=(V,E)\) be a hypergraph and \(V'\subseteq V\) a subset of nodes. The \textbf{node induced subgraph}
\(G|_{V'}\) of \(G\) is defined as \(G|_{V'}=(V',E')\) with \(E'=\{(u,v)\mid (u,v)\in E,u\subseteq
        V',v\subseteq V'\}\).
The node ordering on \(V'\) is the restriction of the node ordering of \(V\).
\end{definition}

\begin{definition}[Connected]
Let \(H=(V,E)\) be a hypergraph. \(H\) is connected if \(\abs{V}=1\) or if there exists a partitioning
\(V'\), \(V''\) of \(V\) and a hyperedge \((u,v)\in E\) s.t. \(u\subseteq V'\), \(v\subseteq V''\) and
both \(G|_{V'}\) and \(G|_{V''}\) are connected.
\end{definition}

Let \(H=(V,E)\) is a hypergraph and \(V'\subseteq V\) is a subset of the nodes s.t. the node-induced
subgraph \(G|_{V'}\) is connected, then we call \(V'\) a \textbf{connected subgraph} or \textbf{csg} for short. The
number of connected subgraphs is important for dynamic programming: it directly corresponds to the
number of entries in the dynamic programming table. If a node set \(V''\subseteq(V\setminus V')\)
induces a connected subgraph \(G|_{V''}\), we call \(V''\) a \textbf{connected complement} of \(V'\) or \textbf{cmp} for
short.

Within this paper, \uline{we will assume that all hypergraphs are connected}. This way, we can make sure that
no cross products are needed. However, when dealing with hypergraphs, this condition can easily be
assured by adding according hyperedges: for every pair of connected components, we can add a hyperedge
whose hypernodes contain exactly the relations of the connected components. By considering these
hyperedges as \(\bowtie\) operators with selectivity 1, we get an equivalent connected hypergraph
\subsection{Csg-cmp-pair}
\label{sec:orgeeff945}
\begin{definition}[Csg-cmp-pair]
Let \(H=(V,E)\) be a hypergraph and \(S_1\), \(S_2\) two subsets of \(V\) s.t. \(S_1\subseteq V\) and
\(S_2\subseteq(V\setminus S_1)\) are a connected subgraph and a connected complement. If there further
exists a hyperedge \((u,v)\in E\) s.t. \(u\subseteq S_1\) and \(v\subseteq S_2\), we call
\((S_1,S_2)\) a \textbf{csg-cmp-pair}.
\end{definition}

We will restrict the enumeration of csg-cmp-pairs to those \((S_1,S_2)\) which satisfy the condition
that \(\min(S_1)\prec\min(S_2)\).

Obviously, in order to be correct, any dynamic programming algorithm has to consider all
csg-cmp-pairs. Further, only these have to be considered. Thus, the minimal number of cost function
calls of any dynamic programming algorithm is exactly the number of csg-cmp-pairs for a given
hypergraph. Note that the number of connected subgraphs is far smaller than the number of
csg-cmp-pairs.

The problem now is to enumerate the csg-cmp-pairs efficiently and in an order
acceptable for dynamic programming. The latter can be expressed more specifically. Before enumerating
a csg-cmp-pair \((S_1,S_2)\), all csg-cmp-pairs \(S_1',S_2'\) with \(S_1'\subseteq S\) and
\(S_2'\subseteq S_2\) have to be enumerated.
\subsection{Neighborhood}
\label{sec:org95aaf8b}
The main idea to generate csg-cmp-pairs is to incrementally expand connected subgraphs by considering
new nodes in the \textbf{neighborhood} of a subgraph. Informally, the neighborhood \(N(S)\) under an exclusion
set \(X\) consists of all nodes reachable from \(S\) that are not in \(X\). We derive an exact
definition below

When choosing subsets of the neighborhood for inclusion, we have to treat a hypernode as a single
instance: either all of its nodes are inside an enumerated subset or none of them. Since we want to
use the fast subset enumeration procedure introduced by Vance and Maier, we must have a single bit
representing a hypernode and also single bits for relations occurring in simple edges. Since these may
overlap, we are constained to choose one unique representative of every hypernode occuring in a
hyperedge. We choose the node \(\min(S)\).

Let \(S\) be a current set, which we want to expand by adding further relations. Consider a hyperedge
\((u,v)\) with \(u\subseteq S\). Then we will add \(\min(v)\) to the neighborhood of \(S\). However,
we have to make sure that the missing elements of \(v\), i.e., \(v\setminus\min(v)\), are also
contained in any set emitted. We thus define
\begin{equation*}
\bbar{\min}(S)=S\setminus \min(S)
\end{equation*}

We define the set of non-subsumed hyperedges as the minimal subset \(E\downarrow\) of \(E\) s.t. for
all \((u,v)\in E\) there exists a hyperedge \((u',v')\in E\downarrow\) with \(u'\subseteq u\) and
\(v'\subseteq v\). Additionally, we make sure that none of the nodes of a hypernode are contained in a
set \(X\), which is to be excluded from neighborhood considerations. We thus define a set containing
the \textbf{interesting hypernodes} for given sets \(S\) and \(X\). We do so in two steps.
\begin{enumerate}
\item Collect the potentially interesting hypernodes into a set \(E\downarrow'(S,X)\) and then minimize
this set to eliminate subsumed hypernodes:
\begin{equation*}
E\downarrow'(S,X)=\{v\mid (u,v)\in E,u\subseteq S,v\cap S=\emptyset,v\cap X=\emptyset\}
\end{equation*}
\item Define \(E\downarrow(S,X)\) to be the minimal set of hypernodes s.t. for all
\(v\in E\downarrow'(S,X)\) there exists a hypernode \(v'\) in \(E\downarrow(S,X)\) s.t.
\(v'\subseteq v\).
\end{enumerate}



We now define the \textbf{neighborhood} of a hypernode \(S\), given a set of excluded nodes \(X\), to be:
\begin{equation*}
\caln(S,X)=\bigcup_{v\in E\downarrow(S,X)}\min(v)
\end{equation*}
For hypergraph in Fig \ref{2} and with \(X=S=\{R_1,R_2,R_3\}\), we have
\begin{gather*}
E\downarrow'(S,X)=\{\{R_4,R_5,R_6\}\}\\
E\downarrow(S,X)=\{\{R_4,R_5,R_6\}\}\\
\caln(S,X)=\{R_4\}
\end{gather*}
\section{The algorithm}
\label{sec:org017a573}
High level:
\begin{enumerate}
\item Constructs ccps (csg-cmp-pair) by enumerating connected subgraphs from an increasing part of the query graph
\item both the primary connected subgraphs and its connected complement are created by recursive graph traversals
\item during traversal, some nodes are \textbf{forbidden} to avoid creating duplicates
\item connected subgraphs are increased by following edges to neighboring nodes. For this purpose
hyperedges are interpretaed as \(n:1\) edges, leading from \(n\) of one side to one canonical nodes
of the other side.
\end{enumerate}

Summarizing the above, the algorithm traverses the graph in a fixed order and recursively produces
larger connected sub- graphs.

We give the implementation of our join ordering algorithm for hypergraphs by means of pseudocode for
member functions of a class \texttt{DPhyp}.

The whole algorithm is distributed over five subroutines.
\begin{itemize}
\item The top-level routine \texttt{Solve} initializes the dynamic programming table with access plans for single
relations and then calls \texttt{EmitCsg} and \texttt{EnumerateCsgRec} for each set containing exactly one relation.
\item The member function \texttt{EnumerateCsgRec} is responsible for enumerating connected subgraphs. It does so
by calculating the neighborhood and iterating over each of its subset.
\item For each such subset \(S_1\) , \texttt{EnumerateCsgRec} calls \texttt{EmitCsg}. This member function is responsible
for finding suitable complements.
\item \texttt{EmitCsg} does so by calling \texttt{EnumerateCmpRec}, which recursively enumerates the complements \(S_2\) for
the connected subgraph \(S_1\) found before.
\item The pair \((S_1,S_2)\) is a csg-cmp-pair. For every such pair, \texttt{EmitCsgCmp} is called. Its main
responsibility is to consider a plan built up from the plans for \(S_1\) and \(S_2\) .
\end{itemize}


\begin{center}
\includegraphics[width=.99\textwidth]{../../images/papers/97.png}
\captionof{figure}{\label{3}Trace of algorithm For Figure 2}
\end{center}
\subsection{Solve}
\label{sec:orgae8c5da}
\begin{align*}
&\texttt{Solve}()\\
&\textbf{for each }v\in V\quad\text{// initialize dpTable}\\
&\quad\text{dpTable}[\{v\}]=\text{plan for }v\\
&\textbf{for each }v\in V\textbf{ descending }\text{according to }\prec\\
&\quad\texttt{EmitCsg}(\{v\})\quad\text{// process singleton sets}\\
&\quad\texttt{EnumerateCsgRec}(\{v\},\calb_v)\quad\text{// expand singleton sets}\\
&\textbf{return }\text{dpTable}[V]
\end{align*}
\subsection{EnumerateCsgRec}
\label{sec:org8bea9ec}
\begin{align*}
&\texttt{EnumerateCsgRec}(S_1,X)\\
&\textbf{for each }N\subseteq\caln(S_1,X):N\neq\emptyset\\
&\quad\textbf{if }\text{dpTable}[S_1\cup N]\neq\emptyset\\
&\quad\quad\texttt{EmitCsg(S_1\cup N)}
\end{align*}
\subsection{EmitCsg}
\label{sec:org32c6212}
\begin{align*}
&\texttt{EmitCsg}(S_1)\\
&X=S_1\cup\calb_{\min(S_1)}\\
&N=\caln(S_1,X)\\
&\textbf{for each }v\in N\textbf{ descending }\text{according to }\prec\\
&\quad S_2=\{v\}\\
&\quad\textbf{if }\exists(u,v)\in E:u\subseteq S_1\wedge v\subseteq S_2\\
&\quad\quad\texttt{EmitCsgCmp}(S_1,S_2)\\
&\quad\texttt{EnumerateCmpRec}(S_1,S_2,X)
\end{align*}
\section{Problems}
\label{sec:org15f8b18}


\section{References}
\label{sec:org2653cac}
\label{bibliographystyle link}
\bibliographystyle{alpha}

\label{bibliography link}
\bibliography{../../../references}
\end{document}
