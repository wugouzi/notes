% Created 2025-02-25 Tue 19:54
% Intended LaTeX compiler: xelatex
\documentclass[11pt]{article}
\usepackage{capt-of}
\usepackage{hyperref}
% TIPS
% \substack{a\\b} for multiple lines text





% pdfplots will load xolor automatically without option
\usepackage[dvipsnames]{xcolor}

\usepackage{forest}
% two-line text in node by [two \\ lines]
% \begin{forest} qtree, [..] \end{forest}
\forestset{
  qtree/.style={
    baseline,
    for tree={
      parent anchor=south,
      child anchor=north,
      align=center,
      inner sep=1pt,
    }}}
%\usepackage{flexisym}
% load order of mathtools and mathabx, otherwise conflict overbrace

\usepackage{mathtools}
%\usepackage{fourier}
\usepackage{pgfplots}
\usepackage{amsthm, mathabx,  amsmath, commath}
\usepackage{amsfonts}

\usepackage{empheq}
\usepackage{tikz}
\usetikzlibrary{arrows.meta}
\usepackage[most]{tcolorbox}

\newtheorem{theorem}{Theorem}[section]
\newtheorem{definition}{Definition}[section]
\newtheorem{corollary}{Corollary}[section]
\newtheorem{example}{Example}[section]
\newtheorem{lemma}{Lemma}[section]
\newtheorem{proposition}{Proposition}[section]

\newcommand{\bl}[1] {\boldsymbol{#1}}
\newcommand{\Wt}[1] {\stackrel{\sim}{\smash{#1}\rule{0pt}{1.1ex}}}
\newcommand{\wt}[1] {\widetilde{#1}}


%For boxed texts in align, use Aboxed{}
%otherwise use boxed{}

\DeclareMathSymbol{\widehatsym}{\mathord}{largesymbols}{"62}
\newcommand\lowerwidehatsym{%
  \text{\smash{\raisebox{-1.3ex}{%
    $\widehatsym$}}}}
\newcommand\fixwidehat[1]{%
  \mathchoice
    {\accentset{\displaystyle\lowerwidehatsym}{#1}}
    {\accentset{\textstyle\lowerwidehatsym}{#1}}
    {\accentset{\scriptstyle\lowerwidehatsym}{#1}}
    {\accentset{\scriptscriptstyle\lowerwidehatsym}{#1}}
}

\usepackage{graphicx}
    
% text on arrow for xRightarrow
\makeatletter
%\newcommand{\xRightarrow}[2][]{\ext@arrow 0359\Rightarrowfill@{#1}{#2}}
\makeatother


\def \bx {\boldsymbol{x}}
\def \ba {\boldsymbol{a}}
\def \bI {\boldsymbol{I}}
\def \bt {\boldsymbol{t}}
\def \bb {\boldsymbol{b}}
\def \bA {\boldsymbol{A}}
\def \bX {\boldsymbol{X}}
\def \bu {\boldsymbol{u}}
\def \bS {\boldsymbol{S}}
\def \bZ {\boldsymbol{Z}}
\def \bz {\boldsymbol{z}}
\def \by {\boldsymbol{y}}
\def \bw {\boldsymbol{w}}
\def \bT {\boldsymbol{T}}
\def \bS {\boldsymbol{S}}
\def \bm {\boldsymbol{m}}
\def \bW {\boldsymbol{W}}
\def \bY {\boldsymbol{Y}}
\def \bH {\boldsymbol{H}}
\def \blambda {\boldsymbol{\lambda}}
\def \bPhi {\boldsymbol{\Phi}}
\def \btheta {\boldsymbol{\theta}}
\def \bmu {\boldsymbol{\mu}}
\def \bphi {\boldsymbol{\phi}}
\def \bSigma {\boldsymbol{\Sigma}}
\def \lb {\left\{}
\def \rb {\right\}}
\def \caln {\mathcal{N}}
\def \dissum {\displaystyle\Sigma}
\def \dispro {\displaystyle\prod}
\def \E {\mathbb{E}}
\def \Q {\mathbb{Q}}
\def \V {\mathbb{V}}
\def \R {\mathbb{R}}
\def \calq {\mathcal{Q}}
\def \calg {\mathcal{G}}
\def \caln {\mathcal{N}}
\def \calr {\mathcal{R}}
\def \calm {\mathcal{M}}
\def \calc {\mathcal{C}}
\def \bcup {\bigcup}

\graphicspath{{../../../paper/query_optimization/}}
\DeclareMathOperator{\csg}{\texttt{csg}}
\DeclareMathOperator{\ccp}{\texttt{ccp}}
\DeclareMathOperator{\ncsg}{\#\texttt{csg}}
\DeclareMathOperator{\nccp}{\#\texttt{ccp}}

%% ox-latex features:
%   !announce-start, !guess-pollyglossia, !guess-babel, !guess-inputenc, caption,
%   image, !announce-end.

\usepackage{capt-of}

\usepackage{graphicx}

%% end ox-latex features


\date{\today}
\title{Analysis of Two Existing and One New Dynamic Programming Algorithm for the Generation of Optimal Bushy Join Trees without Cross Products}
\hypersetup{
 pdfauthor={},
 pdftitle={Analysis of Two Existing and One New Dynamic Programming Algorithm for the Generation of Optimal Bushy Join Trees without Cross Products},
 pdfkeywords={},
 pdfsubject={},
 pdfcreator={Emacs 31.0.50 (Org mode 9.8-pre)},
 pdflang={English}}
\begin{document}

\maketitle
\section{Introduction}
\label{sec:org4a98df8}
Problem: Find the optimal join order
\begin{center}
\includegraphics[width=.8\textwidth]{../../images/papers/98.png}
\captionof{figure}{\label{1}Algorithm \texttt{DPsize}}
\end{center}
\section{Algorithms and Analysis}
\label{sec:orgf471c8c}
\subsection{Size-Driven Enumeration}
\label{sec:org99448b4}
We can construct optimal plans of size \(n\) by joining plans \(P_1\) and \(P_2\) of size \(k\) and
\(n-k\). We just have to take:
\begin{enumerate}
\item the sets of relations contained in \(P_1\) and \(P_2\) do not overlap
\item there is a join predicate connecting a relation in \(P_1\) with a relation in \(P_2\)
\end{enumerate}


The algorithm \texttt{DPsize} can be made more efficient in case of \(s_1=s_2\). Assume that plans of equal
size are repenseted as a linked list. If \(s_1=s_2\), then it is possible to iterate through the list
for retriving all plans \(p_1\). For \(p_2\) we consider the plans succeeding \(p_1\)in the list.
Thus, the complexity can be decreased to \(s_1\times s_2/2\).
\subsection{Subset-Driven Enumeration}
\label{sec:org5610336}
\begin{center}
\includegraphics[width=.8\textwidth]{../../images/papers/99.png}
\captionof{figure}{\label{2}Algorithm \texttt{DPsub}}
\end{center}
\subsection{Algorithm-Independent Results}
\label{sec:orgc876a0f}
\subsubsection{Definition of \texttt{csg} and \texttt{ccp}}
\label{sec:orgae5c0aa}
Consider a join ordering problem with \(n\) relations \(R_0,\dots,R_{n-1}\). We assume the query graph
to be connected. Any subset \(S\) of \(\{R_0,\dots,R_{n-1}\}\) induces a subgraph of the query. If the
subgraph induced by \(S\) is connected, we call \(S\) a \textbf{connected subset} or simply \textbf{connected}. For a
given query graph \(G\) in \(n\) relations, we denote by \(\#\csg_G\) the number of non-empty
connected subgraphs/subsets. For a given kind of query graph, every \(n\) uniquely determines a query
graph. Since the kind of query graph will always be clear from the context, we write \(\ncsg(n)\).

Let \(S_1\) and \(S_2\) be two subsets of \(\{R_0,\dots,R_{n-1}\}\). If there is a join predicate
between a relation in \(S_1\) and another relation in \(S_2\), we call \(S_1\) and \(S_2\) \textbf{connected}.
Since we want to enumerate only bushy trees without cross products, we are only interested in
connected sets \(S_1\) and \(S_2\) which are connected. Moreover, in order to form a valid join tree
for relations in \(S:=S_1\cup S_2\), \(S_1\) and \(S_2\) may not overlap

Summarizing, during plan generation we are interested in pairs \((S_1,S_2)\) where
\begin{itemize}
\item \(S_1\) is a non-empty subset of \(\{R_0,\dots,R_{n-1}\}\)
\item \(S_2\) is a non-empty subset of \(\{R_0,\dots,R_{n-1}\}\)
\end{itemize}
s.t.
\begin{enumerate}
\item \(S_1\) is connected
\item \(S_2\) is connected
\item \(S_1\cap S_2\)
\item there exists nodes \(v_1\in S_1\) and \(v_2\in S_2\) s.t. there is an edge between \(v_1\) and
\(v_2\) in the query graph
\end{enumerate}

Let's call such a pair \textbf{csg-cmp-pair}. Here csg is the abbreviation of connected subgraph and cmp is the
abbreviation of complement.

In the following, we are interested in
\begin{enumerate}
\item the number of connected, non-empty subsets
\item the number of csg-cmp-pairs
\end{enumerate}

We denote the total number of csg-cmp-pairs including symmetric pairs by \(\nccp\).
Ono and Lohman counted the number of csg-cmp-pairs by excluding symmetric pairs

For any correct dynamic programming algorithm \(\nccp\) provides a lower bound on the number of calls
to \texttt{CreateJoinTree}
\section{The New Algorithm \texttt{DPccp}}
\label{sec:org25399e6}
\subsection{Problem Statement}
\label{sec:orgdac8070}
If the search space is sparse, the \texttt{DPsub} algorithm considers many subproblems which are not connected
and, therefore, are not relevant for the solution.

The main idea of \texttt{DPccp} is that it only considers pairs of connected subproblems.

Thus, our goal is to efficiently enumerate all csg-cmp-pairs \((S_1,S_2)\). Requirements:
\begin{enumerate}
\item Enumerate every pair once and only once.
\item whenever a pair \((S_1,S_2)\) is generated, all non-empty subsets of \(S_1\) and \(S_2\) must have
been generated before as a component of a pair.
\item overhead for generating a single csg-cmp-pair must be constant or at most linear
\end{enumerate}

\begin{center}
\includegraphics[width=.8\textwidth]{../../images/papers/100.png}
\captionof{figure}{\label{i4}Algorithm \texttt{DPccp}}
\end{center}
\subsection{Enumerating Connected Subsets}
\label{sec:orgcb4c3da}
Let \(G=(V,E)\) be an undirected graph. For a node \(v\in V\) define the \textbf{neighborhood} \(\caln(v)\) of
\(v\) as \(\caln(v):=\{v'\mid (v,v')\in E\}\). For a subset \(S\subseteq V\) of \(V\) we define the
\textbf{neighborhood} of \(S\) as \(\caln(S):=\bigcup_{v\in S}\caln(v)\setminus S\). Note that for all
\(S,S'\subset V\) we have \(\caln(S\cup S')=(\caln(S)\cup\caln(S'))\setminus(S\cup S')\). This allows
for an efficient bottom-up calculation.

Let \(S\) be a connected subset of an undirected graph \(G\) and \(S'\) be any subset of \(\caln(S)\).
Then \(S\cup S'\) is connected. As a consequence, a connected subset can be enlarged by adding any
subset of its neighborhood.

We could generate all connected subsets as follows. For every node \(v_i\in V\) we perform the
following enumeration steps:
\begin{enumerate}
\item Emit \(\{v_i\}\) as a connected subset
\item Expand \(\{v_i\}\) by calling a routine that extends a given connected set to bigger connected sets
\item let the routine be called with some connected set \(S\). It then calculates the neighborhood
\(\caln(S)\)
\item For every non-empty subset \(N\subseteq\caln(S)\), it emits \(S'=S\cup N\) as a further connected
subset and resurcively calls itself with \(S'\)
\end{enumerate}
The problem with this routine is that it produces duplicates.

This is the point where the breadth-first numbering comes into play. Let \(V=\{v_0,\dots,v_{n-1}\}\),
where the indices are consistent with a breadth-first numbering produced by a breadth-first search
starting at node \(v_0\). The idea is to use the numbering to define an enumeration order: In order to
avoid duplicates, the algorithm enumerates connected subgraphs for every node \(v_i\), but restricts
them to contain no \(v_j\) with \(j<i\). Using the definition \(\calb_i=\{v_j\mid j\le i\}\), the
pseudocode looks as follows:


\begin{center}
\includegraphics[width=.8\textwidth]{../../images/papers/101.png}
\captionof{figure}{\label{4.1}Algorithm \texttt{EnumerateCsg}}
\end{center}

Lets's consider an example. Figure \ref{6} contains a query graph. The calls to \texttt{EnumerateCsgRec} are
contained in the table \ref{7}. In this table, \(S\) and \(X\) are the arguments of \texttt{EnumerateCsgRec}.
\(N\) is the local variable after its initialization. The column \(\text{emit}/S\) contains the
connected subset emitted, which then becomes the argument of the recursive call to \texttt{EnumerateCsgRec}
(labelled by \(\to\))

\begin{center}
\includegraphics[width=.5\textwidth]{../../images/papers/103.png}
\captionof{figure}{\label{i6}Sample graph to illustrate \texttt{EnumerateCsgRec}}
\end{center}


\begin{center}
\includegraphics[width=.8\textwidth]{../../images/papers/102.png}
\captionof{figure}{\label{i7}Call sequence for Figure \ref{6}}
\end{center}
\subsection{Enumerating Complements of Connected Subgraphs}
\label{sec:org227b54a}
We have to generate all csg-cmp-pairs. The basic idea to do so is as follows. Algorithm \texttt{EnumerateCsg}
is used to create the first component \(S_1\) of every csg-cmp-pair. Then, for each such \(S_1\), we
generate all its complement components \(S_2\). This can be done by calling \texttt{EnumerateCsgRec} with the
correct parameters.

We need some definitions to state the actual algorithm., Let \(S_1\subseteq V\) be a non-empty subset
of \(V\). Then, we need to define \(\min(S_1):=\min(\{i\mid v_i\in S_1\})\). This is used to extract
the starting node from which \(S_1\) was constructed. Let \(W\subset V\) be a non-empty subset of
\(V\). Then we define \(\calb_i(W):=\{v_j\mid v_j\in W,j\le i\}\).

\begin{center}
\includegraphics[width=.8\textwidth]{../../images/papers/104.png}
\captionof{figure}{\label{233}Algorithm \texttt{EnumerateCmp}}
\end{center}

Consider the graph \ref{6}. Assume \texttt{EnumerateCmp} is called with \(S_1=\{R_1\}\). Then \(X=\{R_0,R_1\}\).
\(N=\{R_0,R_4\}\setminus\{R_0,R_1\}=\{R_4\}\). Now we get a pair \((\{R_1\},\{R_4\})\).
Then the recursive call to \texttt{EnumerateCsgRec} follows with arguments \(G\), \(\{R_4\}\) and
\(\{R_0,R_1,R_4\}\). Subsequent \texttt{EnumerateCsgRec} generates the connected sets \(\{R_2,R_4\}\),
\(\{R_3,R_4\}\) and \(\{R_2,R_3,R_4\}\), giving three more csg-cmp-pairs.
\subsection{Correctness Proof}
\label{sec:org6580ae3}
\subsubsection{Preliminaries}
\label{sec:orga0774e3}
The correctness of \texttt{DPccp} follows if the csg-cmp-pairs are enumerated correctly, as it simply
enumerates all possible pairs and fills the DP table accordingly. Therefore, we only have to prove the
correctness of the functions \texttt{EnumerateCsg}, \texttt{EnumerateCsgRec} and \texttt{EnumerateCmp}. The rest of this section
is independent of the join ordering problem. Thus, we concentrate on undirected graphs.

Given a connected undirected graph \(G=(V,E)\), we want to enumerate all vertices \(V'\subseteq V\),
s.t. \(G'=(V',E|_{V'})\) is a connected subgraph of \(G\), where
\(E|_{V'}=\{(v,v')\in E\mid v,v'\in V'\}\). We denote the direct neighbors of a node \(v\) by
\begin{equation*}
\caln(v)=\{v'\in V\mid(v,v')\in E\}
\end{equation*}
Indirect neighbors are collected into sets \(\caln_i(v)\):
\begin{align*}
\caln_0(v)&=\{v\}\\
\caln_1(v)&=\caln(v)\\
\caln_{i+1}(v)&=\left(\bigcup_{v'\in\caln_i(v)}\caln(v')\right)\setminus\left(\bigcup_{j=0}^i\caln_j(v)\right)
\end{align*}
If a vertex \(v\in V\) has a label, the label is determined by \(L(v)\). The labels will be unique,
therefore we can identify a vertex by its label: \(v=v_{L(v)}\)

We assume that the graph \(G\) contains no self-cycles, i.e., \(\exists v\in V:(v,v)\in E\).
Furthermore, we assume that the vertices in the graph are labeled in a breadth-first manner. That is,
we demand that
\begin{itemize}
\item there exists one vertex \(v_0\in V\) that has the label 0
\item the vertices in \(\caln_1(v_0)\) have labels in \([1,\abs{\caln_1(v_0)}]\)
\item the vertices in \(\caln_k(v_0)\) have labels in
\(\left[\sum_{i=0}^{k-1}\abs{\caln_i(v_0)},\sum_{i=0}^k\abs{\caln_i(v_0)}\right]\)
\end{itemize}
\subsubsection{Correctness of \texttt{EnumerateCsg}}
\label{sec:org7b55145}
\begin{lemma}[]
\label{1}
Algorithm \texttt{EnumerateCsg} terminates if \(G\) is a finite graph
\end{lemma}

\begin{lemma}[]
\label{2}
Algorithm \texttt{EnumerateCsg} enumerates only connected components
\end{lemma}

\begin{proof}
Induction on the recursion depth \(n\).

\(n=0\): Singleton is a connected component.

Induction: \texttt{EnumerateCsgRec} at recursion level \(n+1\) is called with a connected component \(S\) (IH)
and considers only vertices that are connected to vertices in \(S\). Any subset of \(N\) can be added
to \(S\) to form a connected component
\end{proof}

\begin{lemma}[]
\label{5}
Given a connected undirected graph \(G=(V,E)\), a vertex \(v\in V\), a natural number \(n\ge 0\), and
\(V_n'=\bigcup_{i=0}^n\caln_i(v)\). Then \((V_n',E|_{V_n'})\) is a connected component.
\end{lemma}

\begin{lemma}[]
Given a connected, undirected graph \(G=(V,E)\) and a vertex \(v\in V\). Then \(\exists n\ge 0\) s.t.
\(\forall_{0\le i\le n}\caln_i(v)\neq\emptyset\) and \(\forall_{i>n}\caln_i(v)=\emptyset\)
\end{lemma}

\begin{lemma}[]
Given a connected, undirect graph \(G=(V,E)\), \(\abs{V}>1\) and a set of vertices \(V'\subseteq V\)
s.t. \((V',E|_{V'})\) is a connected component. Then \(\exists v\in V'\) s.t.
\((V'\setminus\{v\},E|_{V'\setminus\{v\}})\) is a connected component.
\end{lemma}

\begin{proof}
Let \(G'=(V',E|_{V'})\) be a connected undirected graph and the base to compute \(\caln(v)\) and
\(\caln_i(v)\). Choose arbitrary \(v_0\in V'\) and a natural number \(n\) s.t.
\(\caln_n(v_0)\neq\emptyset\wedge\caln_{n+1}(v_0)=\emptyset\). Note that \(n>0\) as \(\abs{V'}>1\) and
that \(\bigcup_{0\le i\le n}\caln_i(v_0)=V'\). Now any \(v\in\caln_n(v_0)\) can be removed.
\end{proof}

\begin{lemma}[]
When \texttt{EnumerateCsgRec} is called with additional vertices, it enumerates at least the same components as
without the vertices. More formally:
\begin{gather*}
\{V\cup A\mid(V,E)\text{ enumerated by }\texttt{EnumerateCsgRec}(G,S,X)\}\subseteq\\
\{V\mid(V,E)\text{ enumerated by }\texttt{EnumerateCsgRec}(G,S\cup A,X)\}
\end{gather*}
\end{lemma}

\begin{lemma}[]
\label{7}
Algorithm \texttt{EnumerateCsg} enumerates all connected components consisting of a single vertex
\end{lemma}

\begin{lemma}[]
Algorithm \texttt{EnumerateCsg} enumerates all connected components
\end{lemma}

\begin{proof}
By contradiction. We assume that not all connected components are enumerated. Thus
\(\exists V'\subseteq V\wedge V\neq\emptyset\) s.t. \((V',E|_{V'})\) is a connected component and
\(V'\) is not enumerated. If several such \(V'\) exists, we choose \(V'\) s.t. \(\abs{V'}\) is
minimal. Then \(\abs{V'}>1\) by Lemma \ref{7} and we can delete a vertice \(v'\) from \(V'\) by Lemma \ref{5} and
get a new connected component which is enumerated.

Case 1: \(v'\) appeared in \(N\) during the enumeration of \(V'\setminus\{v\}\), then \(V'\) would be
enumerated by Lemma \ref{6}

Case 2: \(v'\) did not appear in \(N\) during the enumeration of \(V'\setminus\{v'\}\). Since \(v'\)
is connected to \(V'\setminus\{v'\}\), it must have been excluded, i.e.,
\(L(v')<\min(\{L(v)\mid v\in V'\setminus\{v'\}\})\). Then \texttt{EnumerateCsg} will enumerate \(V'\) when
selecting \(v'\) as the start vertex
\end{proof}

\begin{lemma}[]
\label{9}
If \(V'\) and \(V''\) are both enumerated and
\(\min(\{L(v)\mid v\in V'\})=\min(\{L(v)\mid v\in V''\})\), \(V'\) and \(V''\) are enumerated using
the same start vertex
\end{lemma}

\begin{lemma}[]
\label{10}
Algorithm \texttt{EnumerateCsg} enumerates all connected components only once
\end{lemma}

\begin{proof}
By contradiction. Choose  \(V'\subseteq V\) that is enumerated at least twice and is of minimal
cardinality.

Case 1: \(\abs{V'}=1\)

Case 2: \(\abs{V'}>1\). By Lemma \ref{9}, all enumerations of \(V'\) started with the same vertex and
\(X\).

A single invocation of \texttt{EnumerateCsgRec} (without the recursive call) does not produce duplicates.
\(V'\) cannot be enumerated by two different calls to \texttt{EnumerateCsgRec} with the same parameters, as
\(\abs{V'}\) is minimal. Thus there exists \(S_1,S_2,X_1,X_2\subseteq V\) s.t. \(S_1\neq S_2\),
\(S_1,S_2,X_1,X_2\) are constructed by \texttt{EnumerateCsgRec} starting from the same start vertex and both
\(\texttt{EnumerateCsgRec}(G,S_1,X_1)\) and \(\texttt{EnumerateCsgRec}(G,S_2,X_2)\) enumerate \(V'\).
Hence
\begin{equation*}
(V'\setminus S_1)\cap X_1=\emptyset\wedge
(V'\setminus S_2)\cap X_2=\emptyset
\end{equation*}

As \(S_1\neq S_2\), there exists a invocation of \texttt{EnumerateCsgRec}, that recursively calls
\texttt{EnumerateCsgRec} with \(S_1'\) and \(S_2'\), which finally lead to \(S_1\) and \(S_2\). Let \(Y\) be
the corresponding exclusion filter in line 6. Then
\begin{gather*}
\exists v\in (S_1'\cup S_2'):v\notin(S_1'\cap S_2')\wedge v\in Y\\
((v\in S_1\wedge v\notin S_2)\vee(v\notin S_1\wedge v\in S_2))\wedge(v\in Y)\\
v\in V'\wedge v\notin V'
\end{gather*}

\wu{
Essentially, \(S_1\neq S_2\) but \(X_1=X_2\). Now \(v\in X_2=X_1\), \(v\notin S_2\) and therefore \(v\notin V\). But
\(v\in S_1\subseteq V'\).
}
\end{proof}

\begin{lemma}[]
If \(V'\subset V''\), \(n=\abs{V''}-\abs{V'}-1\) and both \((V',E|_{V'})\) and
\((V'',E|_{V''})\) are connected components, then \(\exists V_1\dots V_n\) s.t.
\(V'\subset V_1\), \(V_i\subset V_{i+1}\), \(V_n\subset V''\) and \((V_i,E|_{V_i})\) is a connected
component for all \(1\le i\le n\)
\end{lemma}

\begin{lemma}[]
\label{12}
If \(V'\subset V''\) and both \((V',E|_{V'})\) and \((V'',E|_{V''})\) are connected components,
\texttt{EnumerateCsg} enumerates \((V',E|_{V'})\) before \((V'',E|_{V''})\)
\end{lemma}

\begin{theorem}[]
Algorithm \texttt{EnumerateCsg} is correct
\end{theorem}

\begin{proof}
\ref{1}, \ref{2}, \ref{8}, \ref{10}, \ref{12}
\begin{itemize}
\item Terminate
\item Only list connected components
\item enumerates all connected components
\item enumerates all connected components only once
\item enumerates in a order that is suitable for dynamic programming
\end{itemize}
\end{proof}
\subsubsection{Correctness of \texttt{EnumerateCmp}}
\label{sec:org73d7281}
Besides enumerating the connected components themselves, the \texttt{DPccp} algorithm requires enumerating all
connected components in the adjacent complement of the graph. More formally, given a connected graph
\(G=(V,E)\) and \((V'\subseteq V)\) s.t. \((V',E|_{V'})\) is a connected component, enumerate all
\(V''\subseteq V\setminus V'\) s.t. \((V'',E|_{V''})\) and \((V'\cup V'',E|_{V'\cup V''})\) are
connected components.

The algorithm presented suppress duplicates. This means that if \(V''\) is enumerated for a given
\(V'\), \(V'\) will not be enumerated if \(V''\) is given as a \textbf{primary} connected component (i.e., as a
first component in a csg-cmp-pair). Furthermore, a \(V''\) is only enumerated if it was already
enumerated as a primary connected component. This allows us to define a total ordering between disjoin
connected components that matches the enumeration order used in \texttt{EnumeratedCsg}:
\begin{equation*}
V''<V'\Leftrightarrow\min(\{L(v)\mid v\in V'\})<\min(\{L(v)\mid v\in V''\})
\end{equation*}
Using this ordering, we only enumerate \(V''\) for \(V'\) if \(V'<V''\).
\subsubsection{Proofs}
\label{sec:orgfe76cbc}
\begin{lemma}[]
Algorithm \texttt{EnumerateCmp} terminates if \(G\) is a finite graph
\end{lemma}

\begin{lemma}[]
Algorithm \texttt{EnumerateCmp} enumerates all connected components consisting of a single vertex
\end{lemma}

\begin{lemma}[]
Algorithm \texttt{EnumerateCmp} enumerates all adjacent connected components in the complement (that satisfy
the ordering)
\end{lemma}

\begin{lemma}[]
Algorithm \texttt{EnumerateCmp} enumerates connected components only once
\end{lemma}

\begin{theorem}[]
Algorithm \texttt{EnumerateCmp} is correct
\end{theorem}
\section{Problems}
\label{sec:org63fe12a}


\section{References}
\label{sec:org2f87942}
\label{bibliographystyle link}
\bibliographystyle{alpha}

\bibliography{/Users/wu/notes/notes/references.bib}
\end{document}
