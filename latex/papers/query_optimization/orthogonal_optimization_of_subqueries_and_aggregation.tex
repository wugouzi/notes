% Created 2025-02-14 Fri 13:55
% Intended LaTeX compiler: xelatex
\documentclass[11pt]{article}
\usepackage{capt-of}
\usepackage{hyperref}
% TIPS
% \substack{a\\b} for multiple lines text





% pdfplots will load xolor automatically without option
\usepackage[dvipsnames]{xcolor}

\usepackage{forest}
% two-line text in node by [two \\ lines]
% \begin{forest} qtree, [..] \end{forest}
\forestset{
  qtree/.style={
    baseline,
    for tree={
      parent anchor=south,
      child anchor=north,
      align=center,
      inner sep=1pt,
    }}}
%\usepackage{flexisym}
% load order of mathtools and mathabx, otherwise conflict overbrace

\usepackage{mathtools}
%\usepackage{fourier}
\usepackage{pgfplots}
\usepackage{amsthm, mathabx,  amsmath, commath}
\usepackage{amsfonts}

\usepackage{empheq}
\usepackage{tikz}
\usetikzlibrary{arrows.meta}
\usepackage[most]{tcolorbox}

\newtheorem{theorem}{Theorem}[section]
\newtheorem{definition}{Definition}[section]
\newtheorem{corollary}{Corollary}[section]
\newtheorem{example}{Example}[section]
\newtheorem{lemma}{Lemma}[section]
\newtheorem{proposition}{Proposition}[section]

\newcommand{\bl}[1] {\boldsymbol{#1}}
\newcommand{\Wt}[1] {\stackrel{\sim}{\smash{#1}\rule{0pt}{1.1ex}}}
\newcommand{\wt}[1] {\widetilde{#1}}


%For boxed texts in align, use Aboxed{}
%otherwise use boxed{}

\DeclareMathSymbol{\widehatsym}{\mathord}{largesymbols}{"62}
\newcommand\lowerwidehatsym{%
  \text{\smash{\raisebox{-1.3ex}{%
    $\widehatsym$}}}}
\newcommand\fixwidehat[1]{%
  \mathchoice
    {\accentset{\displaystyle\lowerwidehatsym}{#1}}
    {\accentset{\textstyle\lowerwidehatsym}{#1}}
    {\accentset{\scriptstyle\lowerwidehatsym}{#1}}
    {\accentset{\scriptscriptstyle\lowerwidehatsym}{#1}}
}

\usepackage{graphicx}
    
% text on arrow for xRightarrow
\makeatletter
%\newcommand{\xRightarrow}[2][]{\ext@arrow 0359\Rightarrowfill@{#1}{#2}}
\makeatother


\def \bx {\boldsymbol{x}}
\def \ba {\boldsymbol{a}}
\def \bI {\boldsymbol{I}}
\def \bt {\boldsymbol{t}}
\def \bb {\boldsymbol{b}}
\def \bA {\boldsymbol{A}}
\def \bX {\boldsymbol{X}}
\def \bu {\boldsymbol{u}}
\def \bS {\boldsymbol{S}}
\def \bZ {\boldsymbol{Z}}
\def \bz {\boldsymbol{z}}
\def \by {\boldsymbol{y}}
\def \bw {\boldsymbol{w}}
\def \bT {\boldsymbol{T}}
\def \bS {\boldsymbol{S}}
\def \bm {\boldsymbol{m}}
\def \bW {\boldsymbol{W}}
\def \bY {\boldsymbol{Y}}
\def \bH {\boldsymbol{H}}
\def \blambda {\boldsymbol{\lambda}}
\def \bPhi {\boldsymbol{\Phi}}
\def \btheta {\boldsymbol{\theta}}
\def \bmu {\boldsymbol{\mu}}
\def \bphi {\boldsymbol{\phi}}
\def \bSigma {\boldsymbol{\Sigma}}
\def \lb {\left\{}
\def \rb {\right\}}
\def \caln {\mathcal{N}}
\def \dissum {\displaystyle\Sigma}
\def \dispro {\displaystyle\prod}
\def \E {\mathbb{E}}
\def \Q {\mathbb{Q}}
\def \V {\mathbb{V}}
\def \R {\mathbb{R}}
\def \calq {\mathcal{Q}}
\def \calg {\mathcal{G}}
\def \caln {\mathcal{N}}
\def \calr {\mathcal{R}}
\def \calm {\mathcal{M}}
\def \calc {\mathcal{C}}
\def \bcup {\bigcup}

\graphicspath{{../../../paper/query_optimization/}}
\DeclareMathOperator{\SA}{\mathcal{SA}}
\definecolor{mintedbg}{rgb}{0.99,0.99,0.99}
\usepackage[cachedir=\detokenize{~/miscellaneous/trash}]{minted}
\setminted{breaklines,
mathescape,
bgcolor=mintedbg,
fontsize=\footnotesize,
frame=single,
linenos}

%% ox-latex features:
%   !announce-start, !guess-pollyglossia, !guess-babel, !guess-inputenc, caption,
%   image, !announce-end.

\usepackage{capt-of}

\usepackage{graphicx}

%% end ox-latex features


\date{\today}
\title{Orthogonal Optimization Of Subqueries And Aggregationn}
\hypersetup{
 pdfauthor={},
 pdftitle={Orthogonal Optimization Of Subqueries And Aggregationn},
 pdfkeywords={},
 pdfsubject={},
 pdfcreator={Emacs 31.0.50 (Org mode 9.8-pre)},
 pdflang={English}}
\begin{document}

\maketitle
\section{Introduction}
\label{sec:orgc25cbe9}
In this paper, we present subquery and aggregation techniques implemented in Microsoft SQL Server.
\subsection{Standard subquery execution strategies}
\label{sec:org2b0a137}
Before describing subquery strategies in detail, it is important to clarify the two forms of
aggregation in SQL, whose behavior diverges on an empty input.

“Vector” aggregation specifies grouping columns as well as aggregates to compute.
\begin{minted}[]{sql}
select o_orderdate, sum(o_totalprice)
from orders
group by o_orderdate
\end{minted}

And there are querys that \emph{always returns exactly one row}:
\begin{minted}[]{sql}
select sum(o_totalprice) from orders
\end{minted}

In algebraic expressions we denote vector aggregate as \(\calg_{A,F}\), where \(A\) are the grouping
columns and \(F\) are the aggregates to compute; and denote scalar aggregate as \(\calg^1_F\)

We review standard subquery execution strategies using the following SQL query, which finds customers
who have ordered more than \$1,000,000.
\begin{minted}[]{sql}
-- Q1
select c_custkey
from customer
where 100000 <
      (select sum(o_totalprice)
       from orders
       where o_custkey = c_custkey)
\end{minted}

\textbf{Outerjoin, then aggregate}:
\begin{minted}[]{sql}
select c_custkey
from customer left outer join
     orders on o_custkey = c_custkey
group by c_custkey
having 1000000 < sum(o_totalprice)
\end{minted}

\textbf{Aggregate, then join}:
\begin{minted}[]{sql}
select c_custkey
from customer,
     (select o_custkey from orders
      group by c_custkey
      having 1000000 < sum(o_totalprice))
     as aggresult
where o_custkey = c_custkey
\end{minted}
\subsection{Our technique: Use primitive, orthogonal pieces}
\label{sec:org246ed94}
\begin{center}
\includegraphics[width=.5\textwidth]{../../images/papers/95.png}
\captionof{figure}{\label{1}Primitives connecting different execution strategies}
\end{center}

By implementing all these orthogonal techniques, the query processor should then pro- duce the same
efficient execution plan for the various equiv- alent SQL formulations we have listed above, achieving
a degree of syntax-independence.

\begin{itemize}
\item \textbf{Algebrize into initial operator tree}
\item \textbf{Remove correlations}
\item \textbf{Simplify outerjoin}
\item \textbf{Reorder GroupBy}
\end{itemize}
\subsection{A useful tool: Represent parameterized execution algebraically}
\label{sec:org6fc5651}
\textbf{Apply} takes a relational input \(R\) and a parameterized expression \(E(r)\); it evaluates expression
\(E\) for each row \(r\in R\), and collects the results. Formally,
\begin{equation*}
R\cala^{\otimes}E=\bigcup_{r\in R}(\{r\}\otimes E(r))
\end{equation*}
where \(\otimes\) is either cross product, left outerjoin, left semijoin, or left antijoin.

The most primitive form is \(\cala^\times\), and cross product is assumed if no join variant is
specified.

All operators used in this paper are bag-oriented, and we assume no automatic removal of duplicates.
In particular, the union operator above is \texttt{UNION ALL}. Duplicates are removed explicitly using
\texttt{DISTINCT}.

Now Q1 is like
\begin{center}
\includegraphics[width=.5\textwidth]{../../images/papers/96.png}
\captionof{figure}{\label{2}Subquery execution using Apply}
\end{center}

Apply works on expressions that take scalar (or row-valued) parameters. A second useful construct is
\textbf{SegmentApply}, which deals with expressions using \emph{table}-valued parameters. It takes a relation input
\(R\), a parameterized expression \(E(S)\), and a set of segmenting columns \(A\) from \(R\). It
creates segments of \(R\) using columns \(A\), much like GroupBy, and for each such segment \(S\) it
executes \(E(S)\). Formally,
\begin{equation*}
R\SA_AE=\bigcup_{a}(\{a\}\times E(\sigma_{A=a}R))
\end{equation*}
where \(a\) takes all values in the domain of \(A\).
\section{Representing and normalizing subqueries}
\label{sec:orgc5e1c11}
\subsection{Direct algebraic representation with mutual recursion}
\label{sec:orgae05b9b}
\subsection{Algebraic representation with Apply}
\label{sec:orge4f5d82}
\subsection{Removal of Apply}
\label{sec:orgf052fcf}
\begin{align*}
R\cala^{\otimes}E&=R\otimes_{\text{true}}E\tag{1}\\
&\hspace{-0.7cm}\text{if no parameters in \(E\) resolved from \(R\)}\\
R\cala^{\otimes}(\sigma_pE)&=R\otimes_pE\tag{2}\\
&\hspace{-0.7cm}\text{if no parameters in \(E\) resolved from \(R\)}\\
R\cala^\times(\sigma_pE)&=\sigma_p(R\cala^\times E)\tag{3}\\
R\cala^\times(\pi_vE)&=\pi_{v\cup\text{columns}(R)}(R\cala^\times E)\tag{4}\\
R\cala^\times(E_1\cup E_2)&=(R\cala^\times E_1)\cup(R\cala^\times E_2)\tag{5}\\
R\cala^\times(E_1- E_2)&=(R\cala^\times E_1)-(R\cala^\times E_2)\tag{6}\\
R\cala^\times(E_1\times E_2)&=(R\cala^\times E_1)\bowtie_{R.key}(R\cala^\times E_2)\tag{7}\\
R\cala^\times(\calg_{A,F}E)&=\calg_{A\cup\text{columns}(R),F}(R\cala^\times E)\tag{8}\\
R\cala^\times(\calg^1_FE)&=\calg_{\text{columns}(R),F'}(R\cala^{LOJ}E)\tag{9}
\end{align*}


In (9), \(F'\) contains aggregates in \(F\) expressed over a single-column - for example, if \(F\) is
\texttt{COUNT(*)}, then \(F'\) is \texttt{COUNT(C)} for some not-nullable column \(C\) from \(E\). It is valid for all
aggregates s.t. \(agg(\emptyset)=agg(\{null\})\), which is true for SQL aggregates.

LOJ is left outerjoin

The proof is in \cite{galindo-legaria2000parameterized}
\subsection{All SQL subqueries}
\label{sec:orgb7e102b}
For boolean-valued subqueries, i.e., \texttt{EXISTS}, \texttt{NOT EXISTS}, \texttt{IN} subquery, and quantified comparisons, the
subquery can be rewritten as a scalar \texttt{COUNT} aggregate. From the utilization context of the aggregate
result, either equal to zero or greater than zero, it is possible for the aggregate operator to stop
requesting rows as soon as one has bee found, since additional rows do not affect the result of the
comparison.

A common case that is further optimized is when a relational select has an existential subquery as its
only predicate. In this case, the complete select operator is turned into Apply-semijoin for \emph{exists},
or Apply-antisemijoin for \emph{not exists}. Such Apply is then converted into a non-correlated expression,
if possible, using Identify (2).

There are two scenarios where normalization into standard relational algebra operators is hindered in
a fundamental way. We call those \textbf{exception subqueries} and they require scalar-specific features.
Consider the following query.
\begin{minted}[]{sql}
-- Q2
select c_name,
       (select o_orderkey from orders
        where o_custkey = c_custkey)
from customer
\end{minted}

For every customer, output the customer name, and the result of a subquery that retrieves an oderkey.
There are three cases: If exactly one row is returned from the subquery, then such value is used in
the scalar expression; if no rows are returned, then null is used; finally, if more than one row is
returned, then a run-time error is generated. We call such operator \texttt{Max1row}.
\subsection{Subquery classes}
\label{sec:org2be17e6}
\subsubsection{Class 1. Subqueries that can be removed with no additional common subexpressions}
\label{sec:org2591474}
In general, removing Apply requires introduction of additional common subexpressions. E.g., Identity
(5) introduces two copies of \(R\).

The common case of subqueries that are formed by a simple select/project/join/aggregate block are easy
to handle.
\subsubsection{Class 2. Subqueries that are removed by introducing common subexpressions}
\label{sec:org2d4aee7}
It is hard to formulate a short, meaningful query that fits in this class, using the TPCH schema.
\subsubsection{Class 3. Exception subqueries}
\label{sec:org88fe12c}
\section{Comprehensive optimization of aggregation}
\label{sec:org15a1c0e}
\subsection{Reordering GroupBy}
\label{sec:orgbc173a7}
We will denote GroupBy as \(\calg_{A,F}\), where \(A\) is the set of grouping columns and \(F\) are
the aggregate functions.

We can move a filter around a GroupBy iff all the columns used in the filter are functionally
determined by the grouping columns in the input relation.

A GroupBy can be pushed below a join if the grouping columns, the aggregate calculations and the join
predicate each satisfy certain conditions. Suppose we have a GroupBy above a join of two relations,
i.e., \(\calg_{A,F}(S\bowtie_pR)\), and we want to push the GroupBy below the join so that the
relation \(R\) is aggregated before it is joined, i.e.,
\begin{equation*}
S\bowtie_p(\calg_{A\cup\text{columns}(p)-\text{columns}(S),F}R)
\end{equation*}
\wu{
Assuming \(S\) and \(R\) have no common columns, and \(p\) has columns from \(S\), so we need to
filter them out.
}
This is feasible iff the following conditions are met:
\begin{enumerate}
\item If a column used in the join predicate \(p\) is defined by the relation \(R\) then it is part of
the grouping columns
\item The key of the relation \(R\) \wu{typo in the paper} is part of the grouping columns
\item The aggregate expressions only use columns defined by the relation \(R\)
\end{enumerate}

Pulling a GroupBy above a join is a lot easier. All that is required is that the relation being joined
has a key and that the join predicate does not use the results of the aggregate functions.
\begin{equation*}
S\bowtie_p(\calg_{A,F}R)=\calg_{A\cup\text{columns}(S),F}(S\bowtie_pR)
\end{equation*}

One can think of semijoins and antisemijoins as filters since they include or exclude rows of a
relation based on the column values. The conditions necessary to reorder these operators around
\subsection{Moving GroupBy around an outerjoin}
\label{sec:org5c2ce2e}
Removing correlations for scalar valued subqueries results in an outerjoin followed by a GroupBy.
\section{Problems}
\label{sec:orge6cf44d}


\section{References}
\label{sec:org58f5545}
\label{bibliographystyle link}
\bibliographystyle{alpha}

\bibliography{/Users/wu/notes/notes/references.bib}
\end{document}
