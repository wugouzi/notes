% Created 2025-07-17 Thu 13:37
% Intended LaTeX compiler: xelatex
\documentclass[11pt]{article}
\usepackage{capt-of}
\usepackage{hyperref}
% TIPS
% \substack{a\\b} for multiple lines text





% pdfplots will load xolor automatically without option
\usepackage[dvipsnames]{xcolor}

\usepackage{forest}
% two-line text in node by [two \\ lines]
% \begin{forest} qtree, [..] \end{forest}
\forestset{
  qtree/.style={
    baseline,
    for tree={
      parent anchor=south,
      child anchor=north,
      align=center,
      inner sep=1pt,
    }}}
%\usepackage{flexisym}
% load order of mathtools and mathabx, otherwise conflict overbrace

\usepackage{mathtools}
%\usepackage{fourier}
\usepackage{pgfplots}
\usepackage{amsthm, mathabx,  amsmath, commath}
\usepackage{amsfonts}

\usepackage{empheq}
\usepackage{tikz}
\usetikzlibrary{arrows.meta}
\usepackage[most]{tcolorbox}

\newtheorem{theorem}{Theorem}[section]
\newtheorem{definition}{Definition}[section]
\newtheorem{corollary}{Corollary}[section]
\newtheorem{example}{Example}[section]
\newtheorem{lemma}{Lemma}[section]
\newtheorem{proposition}{Proposition}[section]

\newcommand{\bl}[1] {\boldsymbol{#1}}
\newcommand{\Wt}[1] {\stackrel{\sim}{\smash{#1}\rule{0pt}{1.1ex}}}
\newcommand{\wt}[1] {\widetilde{#1}}


%For boxed texts in align, use Aboxed{}
%otherwise use boxed{}

\DeclareMathSymbol{\widehatsym}{\mathord}{largesymbols}{"62}
\newcommand\lowerwidehatsym{%
  \text{\smash{\raisebox{-1.3ex}{%
    $\widehatsym$}}}}
\newcommand\fixwidehat[1]{%
  \mathchoice
    {\accentset{\displaystyle\lowerwidehatsym}{#1}}
    {\accentset{\textstyle\lowerwidehatsym}{#1}}
    {\accentset{\scriptstyle\lowerwidehatsym}{#1}}
    {\accentset{\scriptscriptstyle\lowerwidehatsym}{#1}}
}

\usepackage{graphicx}
    
% text on arrow for xRightarrow
\makeatletter
%\newcommand{\xRightarrow}[2][]{\ext@arrow 0359\Rightarrowfill@{#1}{#2}}
\makeatother


\def \bx {\boldsymbol{x}}
\def \ba {\boldsymbol{a}}
\def \bI {\boldsymbol{I}}
\def \bt {\boldsymbol{t}}
\def \bb {\boldsymbol{b}}
\def \bA {\boldsymbol{A}}
\def \bX {\boldsymbol{X}}
\def \bu {\boldsymbol{u}}
\def \bS {\boldsymbol{S}}
\def \bZ {\boldsymbol{Z}}
\def \bz {\boldsymbol{z}}
\def \by {\boldsymbol{y}}
\def \bw {\boldsymbol{w}}
\def \bT {\boldsymbol{T}}
\def \bS {\boldsymbol{S}}
\def \bm {\boldsymbol{m}}
\def \bW {\boldsymbol{W}}
\def \bY {\boldsymbol{Y}}
\def \bH {\boldsymbol{H}}
\def \blambda {\boldsymbol{\lambda}}
\def \bPhi {\boldsymbol{\Phi}}
\def \btheta {\boldsymbol{\theta}}
\def \bmu {\boldsymbol{\mu}}
\def \bphi {\boldsymbol{\phi}}
\def \bSigma {\boldsymbol{\Sigma}}
\def \lb {\left\{}
\def \rb {\right\}}
\def \caln {\mathcal{N}}
\def \dissum {\displaystyle\Sigma}
\def \dispro {\displaystyle\prod}
\def \E {\mathbb{E}}
\def \Q {\mathbb{Q}}
\def \V {\mathbb{V}}
\def \R {\mathbb{R}}
\def \calq {\mathcal{Q}}
\def \calg {\mathcal{G}}
\def \caln {\mathcal{N}}
\def \calr {\mathcal{R}}
\def \calm {\mathcal{M}}
\def \calc {\mathcal{C}}
\def \bcup {\bigcup}

\graphicspath{{../../../paper/query_optimization/}}
\definecolor{mintedbg}{rgb}{0.99,0.99,0.99}
\usepackage[cachedir=\detokenize{~/miscellaneous/trash}]{minted}
\setminted{breaklines,
mathescape,
bgcolor=mintedbg,
fontsize=\footnotesize,
frame=single,
linenos}

%% ox-latex features:
%   !announce-start, !guess-pollyglossia, !guess-babel, !guess-inputenc, caption,
%   underline, image, !announce-end.

\usepackage{capt-of}

\usepackage[normalem]{ulem}

\usepackage{graphicx}

%% end ox-latex features


\date{\today}
\title{Query Optimization in Microsoft SQL Server PDW}
\hypersetup{
 pdfauthor={},
 pdftitle={Query Optimization in Microsoft SQL Server PDW},
 pdfkeywords={},
 pdfsubject={},
 pdfcreator={},
 pdflang={English}}
\begin{document}

\maketitle
\section{Introduction}
\label{sec:orgb8317d6}
\subsection{Overview of SQL Server PDW}
\label{sec:org2e7758f}
Microsoft SQL Server Parallel Data Warehouse is a shared-nothing parallel database appliance and is
one example of an MPP system.

\begin{center}
\includegraphics[width=.8\textwidth]{../../images/papers/184.png}
\label{1}
\end{center}


It has a \textbf{control node} that manages a number of compute nodes (see Figure \ref{1}). The control node
provides the external interface to the appliance, and query requests flow through it. The control node
is responsible for query parsing, creating a distributed execution plan, issuing plan steps to the
compute nodes, tracking the execution steps of the plan, and assembling the individual pieces of the
final results into the single result set that is returned to the user.

\textbf{Compute nodes} provide the data storage and the query processing backbone of the appliance. The control
and compute nodes each have a single instance of SQL Server RDBMS running on them. User data is stored
in tables that are hash-partitioned or replicate tables across the SQL Server instances on the compute
nodes.

To execute a query, the control node transforms the user query into a distributed execution plan
(called \textbf{DSQL plan}) that consists of a sequence of operations (called \textbf{DSQL Operations}). At a
high-level, every DSQL plan is composed of two types of operations:
\begin{enumerate}
\item SQL operations, which are SQL statements to be executed against the underlying compute nodes’ DBMS
instances
\item DMS(Data Movement Service) operations which are operations to transfer data between DBMS instances
on different nodes.
\end{enumerate}
\subsection{Query Optimization in PDW}
\label{sec:orgd53579c}
Queries executed in MPP environments tend to be complex - involving many joins, nested sub-queries and
aggregations - and are usually long-running and resource-intensive. The goal of the query optimizer is
to find the best execution plan for a given query, which is usually accomplished by examining a large
space of possible execution plans and comparing these plans according to their estimated execution
costs.

PDW invokes the SQL Server QO against a “shell database” to obtain a compact representation of the
\textbf{optimization search space} called a \texttt{MEMO}. This search space is then augmented with statistical
information on the distribution of data in the appliance to find a parallel execution plan for the
query, taking into consideration the available statistics and the actual data distribution in the
appliance.

The main idea of PDW QO can be summarized as follows:
\begin{enumerate}
\item We store the metadata of the distributed tables in a “shell database” on a single SQL Server
instance. The “shell database” provides the “single system image” of the data in the appliance. Importantly, it also stores aggregated statistical information on the user data.
\item Using the shell database, we use the existing compilation stack of SQL Server to parse a given
query, and generate and export the space of execution alternatives (\texttt{MEMO}).
\item We traverse the space of execution alternatives to introduce data movement operations, and make a
cost-based decision on the best execution plan for the distributed environment.
\end{enumerate}
\section{Microsoft SQL Server PDW Architecture Overview}
\label{sec:org1bb8648}
\subsection{Appliance}
\label{sec:org3f73d37}
There are two distinct types of nodes that implement the query processing functionality
\begin{enumerate}
\item \textbf{Control Node}. The control node manages the distribution of query execution across the compute
nodes, accepts client connections to the PDW appliance and manages client authentication. In
addition to containing a SQL Server instance, the control node contains additional software to
support the distributed architecture of the PDW. This includes the engine that coordinates the data
warehousing functions that are specific to processing parallel queries, stores appliance-wide
metadata and configuration data, and manages appliance and database authentication and
authorization. The control node also manages the Data Movement Service (DMS) that runs on the
appliance nodes and is the communication layer for transfering data between the nodes in the
appliance.
\item \textbf{Compute Nodes}. Each compute node is the host for a single SQL Server instance. It also runs a DMS
process for communication and data transfer with the other nodes in the appliance. Each compute
node stores a portion of the user data.
\end{enumerate}

Tables in a PDW appliance can either be
\begin{enumerate}
\item replicated on each compute node in the appliance, or
\item hash-partitioned on a specified column(s) across the compute nodes.
\end{enumerate}
\subsection{Shell Database}
\label{sec:org4721b79}
A “shell database” is a SQL Server database that defines all metadata and statistics about tables, but
does not contain any user data.

The shell database also contains global statistics for all the tables in the appliance. To compute
global statistics, local statistics are first computed on each node via the standard SQL Server
mechanisms, and are then merged together to derive global statistics.
\subsection{Data Movement}
\label{sec:org5f5732d}
The Data Movement Service (DMS) is responsible for moving data between all the nodes on the appliance.
Once instance of DMS runs on each of the control and compute nodes, certain steps of a user query may
require intermediate result sets to be moved from one compute node to another. In addition, sometimes
intermediate result sets from one or more compute nodes must be moved to the control node for final
aggregations and sorting prior to returning the result set to the client. PDW utilizes temporary
(\emph{temp}) tables on the compute and control nodes as necessary to move data or store intermediate result
sets. In some cases, queries can be written that generate no temp tables and results can be streamed
from the compute nodes directly back to the client that issued the query - such queries will not involve DMS.
\subsection{The DSQL Plan and its Execution}
\label{sec:orgd3707c4}
Given a user-specified SQL query, the PDW engine is responsible for creating a parallel execution plan
(known as a DSQL plan). A DSQL plan may include the following types of operations:
\begin{itemize}
\item \textbf{SQL Operations} that are executed directly on the SQL Server DBMS instances on one or more compute nodes.
\item \textbf{DMS Operations} which move data among the nodes in PDW for further processing, e.g. moving
intermediate result sets from one compute node to another.
\item \textbf{Temp table operations} that set up staging tables for further processing.
\item \textbf{Return operations} which push data back to the client.
\end{itemize}

Query plans are executed serially, one step at a time. However, a single step typically involves
parallel operations across multiple compute nodes.

\textbf{DSQL Plan Example}: Using the TPC-H schema as an example, let’s assume that the \texttt{Customer} table is
hash-partitioned on \texttt{c\_custkey}, and the \texttt{Orders} table is hash-partitioned on \texttt{o\_orderkey} and we want to
perform the following join between these two tables.
\begin{minted}[]{sql}
SELECT c_custkey,
       o_orderdate
FROM Orders, Customer
WHERE o_custkey = c_custkey AND o_totalprice > 100
\end{minted}
The table partitioning is not compatible with the join since \texttt{Orders} is not partitioned on \texttt{o\_custkey}.
Thus, a data movement operation is required in order to evaluate the query. The optimizer on the
control node may produce a DSQL plan consisting of the following two steps:

\begin{enumerate}
\item \uline{DMS Operation} that repartitions data in the \texttt{Orders} table on \texttt{o\_custkey} in preparation for the join.
\item \texttt{Return SQL Operation} that selects tuples for the final result set from each compute node and
returns them back to the client.
\end{enumerate}

\textbf{Step 1: DMS Operation}: In the example above, the first step in the DSQL plan is a DMS operation that
repartitions \texttt{Orders} data on \texttt{o\_custkey}. The DMS operation specifies the
\wu{How is repartition done}
\begin{enumerate}
\item SQL statement required to extract the source data
\item the tuple routing policy (e.g., replicate or hash-partition on a particular column), and
\item the name of a (temporary) destination table.
\end{enumerate}

The Engine service then begins broadcasting the DMS operation from the control node to the DMS
instance on each node. Upon receiving the DMS message, the DMS instance on each compute node begins
execution of the data movement operation by issuing the SQL statement below:
\begin{minted}[]{sql}
SELECT o_custkey,
       o_orderdate
FROM Orders
WHERE o_totalprice > 100
\end{minted}
against the local SQL Server instance. Each DMS instance reads the result tuples out of the local SQL
Server instance, routes the tuples to the appropriate DMS process by hashing on \texttt{o\_custkey}, and also
inserts the tuples it receives from other DMS instances into the specified local destination table
(\texttt{Temp\_Table} in this example). Once all of the tuples from the source SQL statement have been inserted
into their respective destinations the DMS operation is complete.

\textbf{Step 2: SQL Operation}: After the DMS operation has completed, the Engine service moves on to the
second step in the plan, which is the SQL operation that is used to pull the result tuples from each
compute node. To perform this operation, the Engine service obtains a connection to the SQL Server
instance on each compute node and issues a specified SQL statement. In this case, the SQL statement
that will be executed is:
\begin{minted}[]{sql}
SELECT c.c_custkey,
       tmp.o_orderdate
FROM Customer c,
     Temp_Table tmp
WHERE c.c_custkey = tmp.o_custkey
\end{minted}
\subsection{Cost-Based Query Optimization PDW}
\label{sec:org0bf2408}
\begin{center}
\includegraphics[width=.\textwidth]{../../images/papers/185.png}
\label{2}
\end{center}

Figure \ref{2} provides the high-level data flow for PDW query optimization. The key observation that
forms the basis of PDW QO is that the problem of
\begin{enumerate}
\item algebrizing input queries into operator trees, and
\item the logical (as opposed to physical, or partition-dependent) exploration done on operator trees to
find plan alternatives is the same for PDW as it is for a single SQL server instance.
\end{enumerate}



\begin{enumerate}
\item \textbf{PDW Parser}: This component is responsible for parsing the input query string and creating an
abstract syntax tree (AST) structure that can be validated against PDW syntax rules. Some PDW
queries may also need a few basic transformations before they are ready to be sent to SQL Server
against the shell database.
\item \textbf{SQL Server Compilation}: After validation by the PDW parser, the query is passed to SQL Server for
compilation against the shell database. The SQL Server optimizer performs the following functions:
\begin{enumerate}
\item Simplification of the input operator tree into a normalized form. This is inserted as the
initial plan into the \texttt{MEMO} data structure, which will hold the space of alternative plans for the query.
\item Logical transformations on the plans in the \texttt{MEMO} data structure to augment the set of choices.
These are based on relational algebra rules. For instance, all equivalent join orders are
generated in this stage.
\item Estimation of the size of intermediate results for each of the execution alternatives. These
estimations are based on the size of base tables and statistics on the column values.
\item The implementation phase which adds physical operator (algorithms) choices into the search
space. The optimizer costs them and prunes the plans that do not meet established lower bounds.
\item Extraction of the optimal execution plan.
\end{enumerate}
\item \textbf{XML Generator}: This component takes the search space generated by SQL Server optimizer represented
in the MEMO data structure as its input and encodes the information as XML.
\item \textbf{PDW Query Optimizer}: The PDW query optimizer is the consumer of the search space output from the
XML generator. There is a memo parser on the PDW side which is re- sponsible for constructing the
memo data structure for the PDW query optimizer. Once the memo data structure for the PDW side is
constructed, the PDW optimizer performs bottom-up optimization with the help of the PDW cost model.
The responsibilities of the PDW query optimizer include:
\begin{enumerate}
\item Enumeration of distributed execution plans by systematically adding appropriate data movement
strategies into the search space.
\item Costing the alternative plans generated using the PDW cost model.
\item Choosing the optimal (minimal cost) distributed tion plan
\end{enumerate}
\end{enumerate}


\begin{examplle}[]
Consider
\begin{minted}[]{sql}
SELECT *
FROM CUSTOMER C, ORDERS O
WHERE C.C_CUSTKEY = O.O_CUSTKEY
AND O.O_TOTALPRICE > 1000
\end{minted}
\end{examplle}
\section{Problems}
\label{sec:orge86dbea}


\section{References}
\label{sec:org236b051}
\label{bibliographystyle link}
\bibliographystyle{alpha}

\bibliography{/Users/wu/notes/notes/references.bib}
\end{document}
