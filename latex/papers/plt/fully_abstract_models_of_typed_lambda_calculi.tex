% Created 2025-06-24 Tue 13:18
% Intended LaTeX compiler: xelatex
\documentclass[11pt]{article}
\usepackage{capt-of}
\usepackage{hyperref}
% TIPS
% \substack{a\\b} for multiple lines text





% pdfplots will load xolor automatically without option
\usepackage[dvipsnames]{xcolor}

\usepackage{forest}
% two-line text in node by [two \\ lines]
% \begin{forest} qtree, [..] \end{forest}
\forestset{
  qtree/.style={
    baseline,
    for tree={
      parent anchor=south,
      child anchor=north,
      align=center,
      inner sep=1pt,
    }}}
%\usepackage{flexisym}
% load order of mathtools and mathabx, otherwise conflict overbrace

\usepackage{mathtools}
%\usepackage{fourier}
\usepackage{pgfplots}
\usepackage{amsthm, mathabx,  amsmath, commath}
\usepackage{amsfonts}

\usepackage{empheq}
\usepackage{tikz}
\usetikzlibrary{arrows.meta}
\usepackage[most]{tcolorbox}

\newtheorem{theorem}{Theorem}[section]
\newtheorem{definition}{Definition}[section]
\newtheorem{corollary}{Corollary}[section]
\newtheorem{example}{Example}[section]
\newtheorem{lemma}{Lemma}[section]
\newtheorem{proposition}{Proposition}[section]

\newcommand{\bl}[1] {\boldsymbol{#1}}
\newcommand{\Wt}[1] {\stackrel{\sim}{\smash{#1}\rule{0pt}{1.1ex}}}
\newcommand{\wt}[1] {\widetilde{#1}}


%For boxed texts in align, use Aboxed{}
%otherwise use boxed{}

\DeclareMathSymbol{\widehatsym}{\mathord}{largesymbols}{"62}
\newcommand\lowerwidehatsym{%
  \text{\smash{\raisebox{-1.3ex}{%
    $\widehatsym$}}}}
\newcommand\fixwidehat[1]{%
  \mathchoice
    {\accentset{\displaystyle\lowerwidehatsym}{#1}}
    {\accentset{\textstyle\lowerwidehatsym}{#1}}
    {\accentset{\scriptstyle\lowerwidehatsym}{#1}}
    {\accentset{\scriptscriptstyle\lowerwidehatsym}{#1}}
}

\usepackage{graphicx}
    
% text on arrow for xRightarrow
\makeatletter
%\newcommand{\xRightarrow}[2][]{\ext@arrow 0359\Rightarrowfill@{#1}{#2}}
\makeatother


\def \bx {\boldsymbol{x}}
\def \ba {\boldsymbol{a}}
\def \bI {\boldsymbol{I}}
\def \bt {\boldsymbol{t}}
\def \bb {\boldsymbol{b}}
\def \bA {\boldsymbol{A}}
\def \bX {\boldsymbol{X}}
\def \bu {\boldsymbol{u}}
\def \bS {\boldsymbol{S}}
\def \bZ {\boldsymbol{Z}}
\def \bz {\boldsymbol{z}}
\def \by {\boldsymbol{y}}
\def \bw {\boldsymbol{w}}
\def \bT {\boldsymbol{T}}
\def \bS {\boldsymbol{S}}
\def \bm {\boldsymbol{m}}
\def \bW {\boldsymbol{W}}
\def \bY {\boldsymbol{Y}}
\def \bH {\boldsymbol{H}}
\def \blambda {\boldsymbol{\lambda}}
\def \bPhi {\boldsymbol{\Phi}}
\def \btheta {\boldsymbol{\theta}}
\def \bmu {\boldsymbol{\mu}}
\def \bphi {\boldsymbol{\phi}}
\def \bSigma {\boldsymbol{\Sigma}}
\def \lb {\left\{}
\def \rb {\right\}}
\def \caln {\mathcal{N}}
\def \dissum {\displaystyle\Sigma}
\def \dispro {\displaystyle\prod}
\def \E {\mathbb{E}}
\def \Q {\mathbb{Q}}
\def \V {\mathbb{V}}
\def \R {\mathbb{R}}
\def \calq {\mathcal{Q}}
\def \calg {\mathcal{G}}
\def \caln {\mathcal{N}}
\def \calr {\mathcal{R}}
\def \calm {\mathcal{M}}
\def \calc {\mathcal{C}}
\def \bcup {\bigcup}

\graphicspath{{../../../paper/plt/}}
\newcommand{\CA}[1]{\cala\llbracket #1 \rrbracket}

%% ox-latex features:
%   !announce-start, !guess-pollyglossia, !guess-babel, !guess-inputenc,
%   !announce-end.

%% end ox-latex features


\author{Robin Milner}
\date{\today}
\title{Fully Abstract Models of Typed Lambda-Calculi}
\hypersetup{
 pdfauthor={Robin Milner},
 pdftitle={Fully Abstract Models of Typed Lambda-Calculi},
 pdfkeywords={},
 pdfsubject={},
 pdfcreator={},
 pdflang={English}}
\begin{document}

\maketitle
\section{Introduction}
\label{sec:org895c4ba}
A denotational semantic definition of \(L\) consists of a semantic domain \(D\) of meanings, and a
semantic interpretation \(\cala:L\to D\). We assume that we are mainly interested in the semantics of
programs. Denote by \(\calc[\quad]\) a program context - that is, a program with a hole in it, to be
filled by a pharase of some kind.

One desirable property of \(\cala\) is that for all phrases \(M\) and \(N\) we have
\(\cala\llb{\calc[M]}=\cala\llb{\calc[N]}\) whenever \(\cala\llb{M}=\cala\llb{N}\).

This is not hard to achieve, particularly if \(\cala\) is given as a homomorphism. But it is
unfortunate if for some \(M\) and \(N\) s.t. \(\cala\llb{M}\neq\cala\llb{N}\) it nevertheless holds
for \emph{all} program contexts that \(\cala\llb{\calc[M]}=\cala\llb{\calc[N]}\); it means that \(\cala\)
distinguishes finely among nonprogram phrases.

The reason for describing this situation as 'over-generous' is that it typically arises when there are
many objects in \(D\) which cannot be realized (i.e., denoted by a phrase). For example,
\(\cala\llb{M}\) and \(\cala\llb{N}\) may be functions which only differ at an unrealizable argument,
which can never be supplied to the functions in a program context.
\label{1}

So we wish to find \(D\) and \(\cala\) s.t.
\begin{equation*}
\CA{M}\sqsubseteq\CA{N}\quad\text{ iff }\quad\forall\calc[\quad].\CA{M}\sqsubseteq\CA{N}
\end{equation*}
\section{Problems}
\label{sec:org6c5aff6}
\begin{enumerate}
\item \ref{1}: ?
\end{enumerate}
\section{References}
\label{sec:orgb02b019}
\label{bibliographystyle link}
\bibliographystyle{alpha}

\bibliography{/Users/wu/notes/notes/references.bib}
\end{document}
