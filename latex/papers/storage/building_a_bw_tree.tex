% Created 2024-05-10 Fri 13:43
% Intended LaTeX compiler: xelatex
\documentclass[11pt]{article}
\usepackage{hyperref}
% TIPS
% \substack{a\\b} for multiple lines text





% pdfplots will load xolor automatically without option
\usepackage[dvipsnames]{xcolor}

\usepackage{forest}
% two-line text in node by [two \\ lines]
% \begin{forest} qtree, [..] \end{forest}
\forestset{
  qtree/.style={
    baseline,
    for tree={
      parent anchor=south,
      child anchor=north,
      align=center,
      inner sep=1pt,
    }}}
%\usepackage{flexisym}
% load order of mathtools and mathabx, otherwise conflict overbrace

\usepackage{mathtools}
%\usepackage{fourier}
\usepackage{pgfplots}
\usepackage{amsthm, mathabx,  amsmath, commath}
\usepackage{amsfonts}

\usepackage{empheq}
\usepackage{tikz}
\usetikzlibrary{arrows.meta}
\usepackage[most]{tcolorbox}

\newtheorem{theorem}{Theorem}[section]
\newtheorem{definition}{Definition}[section]
\newtheorem{corollary}{Corollary}[section]
\newtheorem{example}{Example}[section]
\newtheorem{lemma}{Lemma}[section]
\newtheorem{proposition}{Proposition}[section]

\newcommand{\bl}[1] {\boldsymbol{#1}}
\newcommand{\Wt}[1] {\stackrel{\sim}{\smash{#1}\rule{0pt}{1.1ex}}}
\newcommand{\wt}[1] {\widetilde{#1}}


%For boxed texts in align, use Aboxed{}
%otherwise use boxed{}

\DeclareMathSymbol{\widehatsym}{\mathord}{largesymbols}{"62}
\newcommand\lowerwidehatsym{%
  \text{\smash{\raisebox{-1.3ex}{%
    $\widehatsym$}}}}
\newcommand\fixwidehat[1]{%
  \mathchoice
    {\accentset{\displaystyle\lowerwidehatsym}{#1}}
    {\accentset{\textstyle\lowerwidehatsym}{#1}}
    {\accentset{\scriptstyle\lowerwidehatsym}{#1}}
    {\accentset{\scriptscriptstyle\lowerwidehatsym}{#1}}
}

\usepackage{graphicx}
    
% text on arrow for xRightarrow
\makeatletter
%\newcommand{\xRightarrow}[2][]{\ext@arrow 0359\Rightarrowfill@{#1}{#2}}
\makeatother


\def \bx {\boldsymbol{x}}
\def \ba {\boldsymbol{a}}
\def \bI {\boldsymbol{I}}
\def \bt {\boldsymbol{t}}
\def \bb {\boldsymbol{b}}
\def \bA {\boldsymbol{A}}
\def \bX {\boldsymbol{X}}
\def \bu {\boldsymbol{u}}
\def \bS {\boldsymbol{S}}
\def \bZ {\boldsymbol{Z}}
\def \bz {\boldsymbol{z}}
\def \by {\boldsymbol{y}}
\def \bw {\boldsymbol{w}}
\def \bT {\boldsymbol{T}}
\def \bS {\boldsymbol{S}}
\def \bm {\boldsymbol{m}}
\def \bW {\boldsymbol{W}}
\def \bY {\boldsymbol{Y}}
\def \bH {\boldsymbol{H}}
\def \blambda {\boldsymbol{\lambda}}
\def \bPhi {\boldsymbol{\Phi}}
\def \btheta {\boldsymbol{\theta}}
\def \bmu {\boldsymbol{\mu}}
\def \bphi {\boldsymbol{\phi}}
\def \bSigma {\boldsymbol{\Sigma}}
\def \lb {\left\{}
\def \rb {\right\}}
\def \caln {\mathcal{N}}
\def \dissum {\displaystyle\Sigma}
\def \dispro {\displaystyle\prod}
\def \E {\mathbb{E}}
\def \Q {\mathbb{Q}}
\def \V {\mathbb{V}}
\def \R {\mathbb{R}}
\def \calq {\mathcal{Q}}
\def \calg {\mathcal{G}}
\def \caln {\mathcal{N}}
\def \calr {\mathcal{R}}
\def \calm {\mathcal{M}}
\def \calc {\mathcal{C}}
\def \bcup {\bigcup}

\graphicspath{{../../../paper/storage/}}
\def \dinsert{\delta\textbf{insert}}
\def \dmerge {\delta\textbf{merge}}
\def \dremove{\delta\textbf{remove}}

%% ox-latex features:
%   !announce-start, !guess-pollyglossia, !guess-babel, !guess-inputenc, caption,
%   image, !announce-end.

\usepackage{capt-of}

\usepackage{graphicx}

%% end ox-latex features


\author{wu}
\date{\today}
\title{Building a Bw Tree Takes More Than Just Buzz Words}
\hypersetup{
 pdfauthor={wu},
 pdftitle={Building a Bw Tree Takes More Than Just Buzz Words},
 pdfkeywords={},
 pdfsubject={},
 pdfcreator={Emacs 29.1 (Org mode 9.7-pre)}, 
 pdflang={English}}
\begin{document}

\maketitle
\section{Abstract}
\label{sec:orgb53b260}
Bw-tree is a lock-free index that provides high throughput for transactional database workloads in SQL
Server’s Hekaton engine. The Bw-Tree avoids locks by appending delta record to tree nodes and using an
indirection layer that allows it to atomically update physical pointers using compare-and-swap (CaS).

This paper has two contributions:
\begin{enumerate}
\item First, it is the missing guide for how to build an \textbf{in-memory} lock-free Bw-Tree.
\item our evaluation shows that despite our improvements, the Bw-Tree still does not perform as well as
other concurrent data structures that use locks.
\end{enumerate}


Background HackerNews's \href{https://news.ycombinator.com/item?id=5521029}{discussion}.
\section{Introduction}
\label{sec:org2619408}
The high-level idea of the Bw-Tree is that it avoids locks by using an indirection layer that maps
logical identifiers to physical pointers for the tree’s internal components. Threads then apply
concurrent updates to a tree node by appending delta records to that node’s modification log.
Subsequent operations on that node must replay these deltas to obtain its current state.


The indirection layer and delta records provide two benefits.
\begin{enumerate}
\item it avoids coherence traffic of locks by decomposing every global state change into atomic steps.
\item it incurs fewer cache invalidations on a multi-core CPU because threads append delta records to
make changes to the index instead of overwriting existing nodes.
\end{enumerate}
\section{Bw-tree Essentials}
\label{sec:org63bb185}
Assume that the Bw-Tree is deployed inside of a database management system with a thread pool, and has
worker threads accessing the index to process queries. If non-cooperative garbage collection is used,pn
the DBMS also launches one or more background threads periodically to perform garbage collection on
the index.

The most prominent difference between the Bw-Tree and other B+Tree-based indexes is that the Bw-Tree
avoids directly editing tree nodes because it causes cache line invalidation. Instead, it stores
modifications to a node in a delta record (e.g., insert, update, delete), and maintains a chain of
such records for every node in the tree. This per-node structure, called a \textbf{Delta Chain}, allows the
Bw-Tree to perform atomic updates via CaS. The \textbf{Mapping Table} serves as an indirection layer that maps
logical node IDs to phisical pointers, making atomic updates of several references to a tree node
possible.

\begin{center}
\includegraphics[width=.9\textwidth]{../../images/db/29.png}
\label{fig1}
\end{center}

As in Fig. \ref{fig1}, every node in the Bw-tree has a unique logical node ID (64-bit). Instead of
using pointers, nodes refer to other nodes using these IDs (\textbf{logical links}). When a thread needs the
physical location of a node, it consults the Mapping Table to translate a node ID to its memory
address.
\subsection{Base Nodes and Delta Chains}
\label{sec:org9f0a616}
A \textbf{logical node} in the Bw-Tree has two components:
\begin{enumerate}
\item a base node:
\begin{enumerate}
\item \textbf{inner base node} holds a sorted (key, node ID) array
\item \textbf{leaf base node} holds a sorted (key, value) array
\end{enumerate}
\item a Delta Chain.
\end{enumerate}
Initially, the Bw-Tree consists of two nodes, an empty leaf base node, and an inner base node that
contains one separator item referring to the empty leaf node. Base nodes are immutable.

As shown in Fig. \ref{fig2}, a Delta Chain is a singly linked list that contains a
chronologically-ordered history of the modifications made to the base node. The entries in the Delta
Chain are connected using physical pointers, with the tail pointing to the base node. The entries in
the Delta Chain are connected using physical pointers, with the tail pointing to the base node. Both
the base node and its delta records contain additional meta-data that represent the state of the
logical node at that point in time. That is, when a worker thread updates a logical node, they compute
the latest attributes of the logical node and store them in the delta record.

\begin{center}
\includegraphics[width=.9\textwidth]{../../images/db/30.png}
\label{fig2}
\end{center}

\begin{table}[htbp]
\caption{\label{tab1}Node Attributes - The list of the attributes that are stored in the logical node's elementes}
\centering
\begin{tabular}{|l|l|}
\hline
Attribute & Description \\
\hline
low-key & The smallest key stored at the logical node. In a node \\
 & split, the low-key of the right sibling is set to the \\
 & split key. Otherwise it is inherited from the element's \\
 & predecessor. \\
\hline
high-key & The smallest key of a logical node's right sibling. ∆split \\
 & records use the split key for this attribute. ∆merge \\
 & records use the high-key of the right branch. Otherwise, \\
 & it is inherited from the element's predecessor. \\
\hline
right-sibling & The ID of the logical node's right sibling \\
 & \\
\hline
size & The number of items in the logical node. It is incremented \\
 & for ∆insert records and decremented for ∆delete records. \\
\hline
depth & The number of records in the logical node's Delta Chain \\
\hline
offset & The location of the inserted or deleted item in the base \\
 & node if they were applied to the base node. Only valid for \\
 & ∆insert and ∆delete records \\
\hline
\end{tabular}
\end{table}
\subsection{Mapping Table}
\label{sec:org2f765db}
Bw-Tree allows a thread to update all references to a node in a single CaS instruction that is
available on all modern CPUs. If a thread's CaS fails then it aborts its operation and restarts. This
restart is transparent to the higher-level DBMS components. Threads always restart an operation by
traversing again from the tree's root. The nodes that a thread will revisit after a restart will
likely be in the CPU cache anyway.
\subsection{Consolidation and Garbage Collection}
\label{sec:org329a132}
Worker threads will periodically consolidate a logical node’s delta records into a new base node.
Consolidation is triggered when Delta Chain’s length exceeds some threshold. Microsoft reported that a
length of eight was a good setting.

At the beginning of consolidation, the thread copies the logical node’s base node contents to its
private memory and then applies the Delta Chain. It then updates the node’s logical link in the
Mapping Table with the new base node. After consolidation, the index reclaims the old base node and
Delta Chain memory after all other threads in the system are finished accessing them.

The original Bw-Tree uses a centralized epoch-based garbage collection scheme to determine when it is
safe to reclaim memory
\subsection{Structural Modification}
\label{sec:org1552117}
As with a B+Tree, a Bw-Tree’s logical node is subject to overflow or underflow. These cases require
\textbf{splitting} a logical node with too many items into two separate nodes or \textbf{merging} together underfull
nodes into a new node. Bw-Tree’s \textbf{structural modification} (SMO) protocols for handling node splits and
merges without using locks. The main idea is to use special delta records to represent internal
structural modifications.

The SMO operation is divided into two phases:
\begin{enumerate}
\item \textbf{logical phase}: appends special deltas to notify other threads of an ongoing SMO

some thread \(t\) appends a \(\dinsert\), \(\dmerge\) or \(\dremove\)

\item \textbf{physical phase}: performs the SMO
\end{enumerate}
\end{document}
