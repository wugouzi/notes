% Created 2025-04-09 Wed 13:21
% Intended LaTeX compiler: xelatex
\documentclass[11pt]{article}
\usepackage{capt-of}
\usepackage{hyperref}
% TIPS
% \substack{a\\b} for multiple lines text





% pdfplots will load xolor automatically without option
\usepackage[dvipsnames]{xcolor}

\usepackage{forest}
% two-line text in node by [two \\ lines]
% \begin{forest} qtree, [..] \end{forest}
\forestset{
  qtree/.style={
    baseline,
    for tree={
      parent anchor=south,
      child anchor=north,
      align=center,
      inner sep=1pt,
    }}}
%\usepackage{flexisym}
% load order of mathtools and mathabx, otherwise conflict overbrace

\usepackage{mathtools}
%\usepackage{fourier}
\usepackage{pgfplots}
\usepackage{amsthm, mathabx,  amsmath, commath}
\usepackage{amsfonts}

\usepackage{empheq}
\usepackage{tikz}
\usetikzlibrary{arrows.meta}
\usepackage[most]{tcolorbox}

\newtheorem{theorem}{Theorem}[section]
\newtheorem{definition}{Definition}[section]
\newtheorem{corollary}{Corollary}[section]
\newtheorem{example}{Example}[section]
\newtheorem{lemma}{Lemma}[section]
\newtheorem{proposition}{Proposition}[section]

\newcommand{\bl}[1] {\boldsymbol{#1}}
\newcommand{\Wt}[1] {\stackrel{\sim}{\smash{#1}\rule{0pt}{1.1ex}}}
\newcommand{\wt}[1] {\widetilde{#1}}


%For boxed texts in align, use Aboxed{}
%otherwise use boxed{}

\DeclareMathSymbol{\widehatsym}{\mathord}{largesymbols}{"62}
\newcommand\lowerwidehatsym{%
  \text{\smash{\raisebox{-1.3ex}{%
    $\widehatsym$}}}}
\newcommand\fixwidehat[1]{%
  \mathchoice
    {\accentset{\displaystyle\lowerwidehatsym}{#1}}
    {\accentset{\textstyle\lowerwidehatsym}{#1}}
    {\accentset{\scriptstyle\lowerwidehatsym}{#1}}
    {\accentset{\scriptscriptstyle\lowerwidehatsym}{#1}}
}

\usepackage{graphicx}
    
% text on arrow for xRightarrow
\makeatletter
%\newcommand{\xRightarrow}[2][]{\ext@arrow 0359\Rightarrowfill@{#1}{#2}}
\makeatother


\def \bx {\boldsymbol{x}}
\def \ba {\boldsymbol{a}}
\def \bI {\boldsymbol{I}}
\def \bt {\boldsymbol{t}}
\def \bb {\boldsymbol{b}}
\def \bA {\boldsymbol{A}}
\def \bX {\boldsymbol{X}}
\def \bu {\boldsymbol{u}}
\def \bS {\boldsymbol{S}}
\def \bZ {\boldsymbol{Z}}
\def \bz {\boldsymbol{z}}
\def \by {\boldsymbol{y}}
\def \bw {\boldsymbol{w}}
\def \bT {\boldsymbol{T}}
\def \bS {\boldsymbol{S}}
\def \bm {\boldsymbol{m}}
\def \bW {\boldsymbol{W}}
\def \bY {\boldsymbol{Y}}
\def \bH {\boldsymbol{H}}
\def \blambda {\boldsymbol{\lambda}}
\def \bPhi {\boldsymbol{\Phi}}
\def \btheta {\boldsymbol{\theta}}
\def \bmu {\boldsymbol{\mu}}
\def \bphi {\boldsymbol{\phi}}
\def \bSigma {\boldsymbol{\Sigma}}
\def \lb {\left\{}
\def \rb {\right\}}
\def \caln {\mathcal{N}}
\def \dissum {\displaystyle\Sigma}
\def \dispro {\displaystyle\prod}
\def \E {\mathbb{E}}
\def \Q {\mathbb{Q}}
\def \V {\mathbb{V}}
\def \R {\mathbb{R}}
\def \calq {\mathcal{Q}}
\def \calg {\mathcal{G}}
\def \caln {\mathcal{N}}
\def \calr {\mathcal{R}}
\def \calm {\mathcal{M}}
\def \calc {\mathcal{C}}
\def \bcup {\bigcup}

\graphicspath{{../../../paper/storage/}}

%% ox-latex features:
%   !announce-start, !guess-pollyglossia, !guess-babel, !guess-inputenc, caption,
%   image, !announce-end.

\usepackage{capt-of}

\usepackage{graphicx}

%% end ox-latex features


\date{\today}
\title{Finding a needle in Haystack: Facebook's photo storage}
\hypersetup{
 pdfauthor={},
 pdftitle={Finding a needle in Haystack: Facebook's photo storage},
 pdfkeywords={},
 pdfsubject={},
 pdfcreator={},
 pdflang={English}}
\begin{document}

\maketitle
\section{Introduction}
\label{sec:org8bc8db7}
Data is written once, read often, never modified, and rarely deleted.

In our experience, we find that the disadvantages of a traditional POSIX based filesystem are
directories and per file metadata.
\begin{itemize}
\item For the Photos application most of this metadata, such as permissions, is unused and thereby wastes
storage capacity.
\item Yet the more significant cost is that the file’s metadata must be read from disk into memory in
order to find the file itself.
\end{itemize}
\section{Background \& Previous Design}
\label{sec:orgbb2da0f}
\subsection{Background}
\label{sec:org644e6f1}
\begin{center}
\includegraphics[width=.8\textwidth]{../../images/papers/132.png}
\captionof{figure}{\label{f1}Typical Design}
\end{center}

Figure \ref{f1} depicts the steps from the moment when a user visits a page containing an image until
she downloads that image from its location on disk.
\begin{enumerate}
\item When visiting a page the user’s browser first sends an HTTP request to a web server which is
responsible for generating the markup for the browser to render.
\item For each image the web server constructs a URL directing the browser to a location from which to
download the data.
\item For popular sites this URL often points to a CDN. If the CDN has the image cached then the CDN
responds immediately with the data.
\item Otherwise, the CDN examines the URL, which has enough information embedded to retrieve the photo
from the site’s storage systems.
\item The CDN then updates its cached data and sends the image to the user’s browser.
\end{enumerate}
\subsection{NFS-based Design}
\label{sec:org8322bf4}
In our first design we implemented the photo storage system using an NFS-based approach.

The major lesson we learned is that CDNs by themselves do not offer a practical solution to serving
photos on a social networking site. CDNs do effectively serve the hottest photos, but a social
networking site like Facebook also generates a large number of requests for less popular (often older)
content, which we refer to as the long tail. Requests from the long tail account for a significant
amount of our traffic, almost all of which accesses the backing photo storage hosts as these requests
typically miss in the CDN.

Our NFS-based design stores each photo in its own file on a set of commercial NAS appliances. A set of
machines, Photo Store servers, then mount all the volumes exported by these NAS appliances over NFS.

\begin{center}
\includegraphics[width=.8\textwidth]{../../images/papers/133.png}
\captionof{figure}{\label{f2}NFS-based Design}
\end{center}

We initially stored thousands of files in each directory of an NFS volume which led to an excessive
number of disk operations to read even a single image. Because of how the NAS appliances manage
directory metadata, placing thousands of files in a directory was extremely inefficient as the
directory’s blockmap was too large to be cached effectively by the appliance. Consequently it was
common to incur more than 10 disk operations to retrieve a single image. After reducing directory
sizes to hundreds of images per directory, the resulting system would still generally incur 3 disk
operations to fetch an image: one to read the directory metadata into memory, a second to load the
inode into memory, and a third to read the file contents.

To further reduce disk operations we let the Photo Store servers explicitly cache file handles
returned by the NAS appliances. When reading a file for the first time a Photo Store server opens a
file normally but also caches the filename to file handle mapping in memcache. When requesting a file
whose file handle is cached, a Photo Store server opens the file directly using a custom system call,
\texttt{open\_by\_filehandle}, that we added to the kernel. Regrettably, this file handle cache provides only a
minor improvement as less popular photos are less likely to be cached to begin with.

The major lesson we learned from the NAS approach is that focusing only on caching— whether the NAS
appliance’s cache or an external cache like memcache—has limited impact for reducing disk operations.
\subsection{Discussion}
\label{sec:org44b857f}
One could phrase the dilemma we faced as existing storage systems lacked the right RAM-to-disk ratio.
However, there is no right ratio. The system just needs enough main memory so that all of the
filesystem metadata can be cached at once. In our NAS-based approach, one photo corresponds to one
file and each file requires at least one inode, which is hundreds of bytes large. Having enough main
memory in this approach is not cost-effective. To achieve a better price/performance point, we decided
to build a custom storage system that reduces the amount of filesystem metadata per photo so that
having enough main memory is dramatically more cost-effective than buying more NAS appliances.
\section{Design \& Implementation}
\label{sec:org60fd19d}
In the following description of Haystack, we distinguish between two kinds of metadata. \textbf{Application
metadata} describes the information needed to construct a URL that a browser can use to retrieve a
photo. \textbf{Filesystem metadata} identifies the data necessary for a host to retrieve the photos that reside
on that host’s disk.
\subsection{Overview}
\label{sec:orgfae61f3}
The Haystack architecture consists of 3 core components: the Haystack Store, Haystack Directory, and
Haystack Cache.

The Store encapsulates the persistent storage system for photos and is the only component that manages
the filesystem metadata for photos. We organize the Store’s capacity by physical volumes. For example,
we can organize a server’s 10 terabytes of capacity into 100 physical volumes each of which provides
100 gigabytes of storage. We further group physical volumes on different machines into logical
volumes. When Haystack stores a photo on a logical volume, the photo is written to all corresponding
physical volumes. This redundancy allows us to mitigate data loss due to hard drive failures, disk
controller bugs, etc.

The Directory maintains the logical to physical mapping along with other application metadata, such as
the logical volume where each photo resides and the logical volumes with free space.

The Cache functions as our internal CDN, which shelters the Store from requests for the most popular
photos and provides insulation if upstream CDN nodes fail and need to refetch content.

\begin{center}
\includegraphics[width=.8\textwidth]{../../images/papers/134.png}
\captionof{figure}{\label{f3}Serving a photo}
\end{center}

When a user visits a page the web server uses the Directory to construct a URL for each photo. The URL
contains several pieces of information, each piece corresponding to the sequence of steps from when a
user’s browser contacts the CDN (or Cache) to ultimately retrieving a photo from a machine in the
Store. A typical URL that directs the browser to the CDN looks like the following:
\begin{center}
\url{http://<CDN>/<Cache>/}<Machine id>/<Logical volume, Photo>
\end{center}

\begin{center}
\includegraphics[width=.8\textwidth]{../../images/papers/135.png}
\captionof{figure}{\label{f4}Uploading a photo}
\end{center}
\subsection{Haysatack Directory}
\label{sec:org7fabe14}
The Directory serves four main functions.
\begin{enumerate}
\item It provides a mapping from logical volumes to physical volumes. Web servers use this mapping when
uploading photos and also when constructing the image URLs for a page request.
\item The Directory load balances writes across logical volumes and reads across physical volumes.
\item the Directory determines whether a photo request should be handled by the CDN or by the Cache. This
functionality lets us adjust our dependence on CDNs.
\item the Directory identifies those logical volumes that are read-only either because of operational
reasons or because those volumes have reached their storage capacity. We mark volumes as read-only
at the granularity of machines for operational ease.
\end{enumerate}

When we increase the capacity of the Store by adding new machines, those machines are write-enabled;
only write-enabled machines receive uploads. Over time the available capacity on these machines
decreases. When a machine exhausts its capacity, we mark it as read-only.
\subsection{Haystack Cache}
\label{sec:orgcc7baa5}
The Cache receives HTTP requests for photos from CDNs and also directly from users’ browsers. We
organize the Cache as a distributed hash table and use a photo’s id as the key to locate cached data.
If the Cache cannot immediately respond to the request, then the Cache fetches the photo from the
Store machine identified in the URL and replies to either the CDN or the user’s browser as
appropriate.

We now highlight an important behavioral aspect of the Cache. It caches a photo only if two conditions
are met:
\begin{enumerate}
\item the request comes directly from a user and not the CDN
\item the photo is fetched from a write-enabled Store machine.
\end{enumerate}

The justification for the first condition is that our experience with the NFS-based design showed
post-CDN caching is ineffective as it is un- likely that a request that misses in the CDN would hit in
our internal cache.

The reasoning for the second is indirect. We use the Cache to shelter write-enabled Store machines
from reads because of two interesting properties:
\begin{enumerate}
\item photos are most heavily accessed soon after they are uploaded
\item filesystems for our workload generally perform better when doing either reads or writes but not
both.
\end{enumerate}

Thus the write-enabled Store machines would see the most reads if it were not for the Cache. Given
this characteristic, an optimization we plan to implement is to proactively push recently uploaded
photos into the Cache as we expect those photos to be read soon and often.
\subsection{Haystack Store}
\label{sec:org36a396b}
Reads make very specific and well-contained requests asking for a photo with a given id, for a certain
logical volume, and from a particular physical Store machine.

Each Store machine manages multiple physical volumes. Each volume holds millions of photos. For
concreteness, the reader can think of a physical volume as simply a very large file (100 GB) saved as
\texttt{/hay/haystack <logical volume id>}. A Store machine can access a photo quickly using only the id of the
corresponding logical volume and the file offset at which the photo resides. This knowledge is the
keystone of the Haystack design: retrieving the filename, offset, and size for a particular photo
without needing disk operations. A Store machine keeps open file descriptors for each physical volume
that it manages and also an in-memory mapping of photo ids to the filesystem metadata (i.e., file,
offset and size in bytes) critical for retrieving that photo.

A Store machine represents a physical volume as a large file consisting of a superblock followed by a
sequence of \textbf{needles}. Each needle represents a photo stored in Haystack.

\begin{center}
\includegraphics[width=.8\textwidth]{../../images/papers/136.png}
\captionof{figure}{\label{f5}Layout of Haystack Store file}
\end{center}

\begin{center}
\includegraphics[width=.8\textwidth]{../../images/papers/137.png}
\captionof{figure}{\label{t1}Explanation of fields in a needle}
\end{center}


To retrieve needles quickly, each Store machine maintains an in-memory data structure for each of its
volumes. That data structure maps pairs of (key, alternate key) to the corresponding needle’s flags,
size in bytes, and volume offset. After a crash, a Store machine can reconstruct this mapping directly
from the volume file before processing requests.
\subsubsection{Photo Read}
\label{sec:orgbe22af3}
\subsubsection{Photo Write}
\label{sec:org8f3069e}
When uploading a photo into Haystack web servers provide the logical volume id, key, alternate key,
cookie, and data to Store machines. Each machine synchronously appends needle images to its physical
volume files and updates in-memory mappings as needed.

While simple, this append-only restriction complicates some operations that modify photos, such as
rotations. As Haystack disallows overwriting needles, photos can only be modified by adding an updated
needle with the same key and alternate key.
\begin{itemize}
\item If the new needle is written to a different logical volume than the original, the Directory updates
its application metadata and future requests will never fetch the older version.
\item If the new needle is written to the same logical volume, then Store machines append the new needle
to the same corresponding physical volumes.
\end{itemize}

Haystack distinguishes such duplicate needles based on their offsets. That is, the latest version of a
needle within a physical volume is the one at the highest offset.
\subsubsection{Photo Delete}
\label{sec:org19735d8}
Deleting a photo is straight-forward. A Store machine sets the delete flag in both the in-memory
mapping and synchronously in the volume file.
\subsubsection{The Index File}
\label{sec:org3148868}
Store machines use an important optimization—the \textbf{index file} —when rebooting.
Index files allow a Store machine to build its in-memory mappings quickly, shortening restart time.

Store machines maintain an index file for each of their volumes. The index file is a checkpoint of the
in-memory data structures used to locate needles efficiently on disk. An index file’s layout is
similar to a volume file’s, containing a superblock followed by a sequence of index records
corresponding to each needle in the su- perblock. These records must appear in the same order as the
corresponding needles appear in the volume file.

\begin{center}
\includegraphics[width=.8\textwidth]{../../images/papers/138.png}
\captionof{figure}{\label{f6}Layout of Haystack Index file}
\end{center}

Restarting using the index is slightly more complicated than just reading the indices and initializing
the in-memory mappings. The complications arise because index files are updated asynchronously,
meaning that index files may represent stale checkpoints.
\begin{itemize}
\item When we write a new photo the Store machine synchronously appends a needle to the end of the volume
file and asynchronously appends a record to the index file.
\item When we delete a photo, the Store machine synchronously sets the flag in that photo’s needle without
updating the index file.
\end{itemize}
These design decisions allow write and delete operations to return faster because they avoid
additional synchronous disk writes. They also cause two side effects we must address:
\begin{itemize}
\item needles can exist without corresponding index records
\item index records do not reflect deleted photos.
\end{itemize}

We refer to needles without corresponding index records as \textbf{orphans}. During restarts, a Store machine
sequentially examines each orphan, creates a matching index record, and appends that record to the
index file. Note that we can quickly identify orphans because the last record in the index file
corresponds to the last non-orphan needle in the volume file.

Since index records do not reflect deleted photos, a Store machine may retrieve a photo that has in
fact been deleted. To address this issue, after a Store machine reads the entire needle for a photo,
that machine can then inspect the deleted flag. If a needle is marked as deleted the Store machine
updates its in-memory map- ping accordingly and notifies the Cache that the object was not found.
\subsubsection{Filesystem}
\label{sec:org96191cf}
In particular, the Store machines should use a filesystem that does not need much memory to be able to
perform random seeks within a large file quickly.

Currently, each Store machine uses XFS, an extent based file system. XFS has two main advantages for
Haystack.
\begin{enumerate}
\item the blockmaps for several contiguous large files can be small enough to be stored in main memory.
\item XFS provides efficient file preallocation, mitigating fragmentation and reining in how large block
maps can grow.
\end{enumerate}
\subsubsection{Recovery from failures}
\label{sec:org67d7710}
To proactively find Store machines that are having problems, we maintain a background task, dubbed
\textbf{pitchfork}, that periodically checks the health of each Store machine. Pitchfork remotely tests the
connection to each Store machine, checks the availability of each volume file, and attempts to read
data from the Store machine. If pitchfork determines that a Store machine consistently fails these
health checks then pitchfork automatically marks all logical volumes that reside on that Store machine
as read-only. We manually address the underlying cause for the failed checks offline.

Once diagnosed, we may be able to fix the problem quickly. Occasionally, the situation requires a more
heavy-handed bulk sync operation in which we reset the data of a Store machine using the volume files
supplied by a replica. Bulk syncs happen rarely (a few each month) and are simple albeit slow to carry
out. The main bottleneck is that the amount of data to be bulk synced is often orders of magnitude
greater than the speed of the NIC on each Store machine, resulting in hours for mean time to recovery.
We are actively exploring techniques to address this constraint.
\subsection{Optimizations}
\label{sec:org1317fef}
\subsubsection{Compaction}
\label{sec:org2b3859a}
A Store machine compacts a volume file by copying needles into a new file while skipping any duplicate
or deleted entries. During compaction, deletes go to both files. Once this procedure reaches the end
of the file, it blocks any further modifications to the volume and atomically swaps the files and
in-memory structures.
\subsubsection{Saving more memory}
\label{sec:orgdebbaa0}
\subsubsection{Batch upload}
\label{sec:org746bc21}
\section{Problems}
\label{sec:org73cb39a}


\section{References}
\label{sec:org747dfd5}
\label{bibliographystyle link}
\bibliographystyle{alpha}

\bibliography{/Users/wu/notes/notes/references.bib}
\end{document}
