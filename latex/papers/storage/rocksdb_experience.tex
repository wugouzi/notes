% Created 2024-04-25 Thu 20:02
% Intended LaTeX compiler: xelatex
\documentclass[11pt]{article}
\usepackage{hyperref}
% TIPS
% \substack{a\\b} for multiple lines text





% pdfplots will load xolor automatically without option
\usepackage[dvipsnames]{xcolor}

\usepackage{forest}
% two-line text in node by [two \\ lines]
% \begin{forest} qtree, [..] \end{forest}
\forestset{
  qtree/.style={
    baseline,
    for tree={
      parent anchor=south,
      child anchor=north,
      align=center,
      inner sep=1pt,
    }}}
%\usepackage{flexisym}
% load order of mathtools and mathabx, otherwise conflict overbrace

\usepackage{mathtools}
%\usepackage{fourier}
\usepackage{pgfplots}
\usepackage{amsthm, mathabx,  amsmath, commath}
\usepackage{amsfonts}

\usepackage{empheq}
\usepackage{tikz}
\usetikzlibrary{arrows.meta}
\usepackage[most]{tcolorbox}

\newtheorem{theorem}{Theorem}[section]
\newtheorem{definition}{Definition}[section]
\newtheorem{corollary}{Corollary}[section]
\newtheorem{example}{Example}[section]
\newtheorem{lemma}{Lemma}[section]
\newtheorem{proposition}{Proposition}[section]

\newcommand{\bl}[1] {\boldsymbol{#1}}
\newcommand{\Wt}[1] {\stackrel{\sim}{\smash{#1}\rule{0pt}{1.1ex}}}
\newcommand{\wt}[1] {\widetilde{#1}}


%For boxed texts in align, use Aboxed{}
%otherwise use boxed{}

\DeclareMathSymbol{\widehatsym}{\mathord}{largesymbols}{"62}
\newcommand\lowerwidehatsym{%
  \text{\smash{\raisebox{-1.3ex}{%
    $\widehatsym$}}}}
\newcommand\fixwidehat[1]{%
  \mathchoice
    {\accentset{\displaystyle\lowerwidehatsym}{#1}}
    {\accentset{\textstyle\lowerwidehatsym}{#1}}
    {\accentset{\scriptstyle\lowerwidehatsym}{#1}}
    {\accentset{\scriptscriptstyle\lowerwidehatsym}{#1}}
}

\usepackage{graphicx}
    
% text on arrow for xRightarrow
\makeatletter
%\newcommand{\xRightarrow}[2][]{\ext@arrow 0359\Rightarrowfill@{#1}{#2}}
\makeatother


\def \bx {\boldsymbol{x}}
\def \ba {\boldsymbol{a}}
\def \bI {\boldsymbol{I}}
\def \bt {\boldsymbol{t}}
\def \bb {\boldsymbol{b}}
\def \bA {\boldsymbol{A}}
\def \bX {\boldsymbol{X}}
\def \bu {\boldsymbol{u}}
\def \bS {\boldsymbol{S}}
\def \bZ {\boldsymbol{Z}}
\def \bz {\boldsymbol{z}}
\def \by {\boldsymbol{y}}
\def \bw {\boldsymbol{w}}
\def \bT {\boldsymbol{T}}
\def \bS {\boldsymbol{S}}
\def \bm {\boldsymbol{m}}
\def \bW {\boldsymbol{W}}
\def \bY {\boldsymbol{Y}}
\def \bH {\boldsymbol{H}}
\def \blambda {\boldsymbol{\lambda}}
\def \bPhi {\boldsymbol{\Phi}}
\def \btheta {\boldsymbol{\theta}}
\def \bmu {\boldsymbol{\mu}}
\def \bphi {\boldsymbol{\phi}}
\def \bSigma {\boldsymbol{\Sigma}}
\def \lb {\left\{}
\def \rb {\right\}}
\def \caln {\mathcal{N}}
\def \dissum {\displaystyle\Sigma}
\def \dispro {\displaystyle\prod}
\def \E {\mathbb{E}}
\def \Q {\mathbb{Q}}
\def \V {\mathbb{V}}
\def \R {\mathbb{R}}
\def \calq {\mathcal{Q}}
\def \calg {\mathcal{G}}
\def \caln {\mathcal{N}}
\def \calr {\mathcal{R}}
\def \calm {\mathcal{M}}
\def \calc {\mathcal{C}}
\def \bcup {\bigcup}

\graphicspath{{../../../paper/storage/}}

%% ox-latex features:
%   !announce-start, !guess-pollyglossia, !guess-babel, !guess-inputenc, caption,
%   underline, !announce-end.

\usepackage{capt-of}

\usepackage[normalem]{ulem}

%% end ox-latex features


\author{wu}
\date{\today}
\title{Evolution of Development Priorities in Key-value Stores Serving Large-scale Applications: The RocksDB Experience}
\hypersetup{
 pdfauthor={wu},
 pdftitle={Evolution of Development Priorities in Key-value Stores Serving Large-scale Applications: The RocksDB Experience},
 pdfkeywords={},
 pdfsubject={},
 pdfcreator={Emacs 29.1 (Org mode 9.7-pre)}, 
 pdflang={English}}
\begin{document}

\maketitle
RocksDB is a key-value store targeting large-scale distributed systems and optimized for \uline{Solid State Drives}
(SSDs). This paper describes how our \uline{priorities} in developing RocksDB have evolved. We describe how and why
RocksDB’s resource optimization target migrated from write amplification, to space amplification, to CPU
utilization.

Lessons from running large-scale applications taught us that:
\begin{enumerate}
\item resource allocation needs to be managed across different RocksDB instances,
\item data format needs to remain backward and forward compatible to allow incremental software rollout,
\item appropriate support for database replication and backups are needed.
\end{enumerate}

Lessons from failure handling taught us that:
\begin{enumerate}
\item data corruption errors needed to be detected earlier and at every layer of the system.
\end{enumerate}
\section{Introduction}
\label{sec:org644f62f}
Each RocksDB instance manages data on storage devices of just a single server node; it does not handle
any inter-host operations, such as replication and load balancing, and it does not perform
high-level operations, such as checkpoints

RocksDB and its various components are highly customizable, customizations can include the
\uline{write-ahead log (WAL) treatment}, the \uline{compression strategy}, and the \uline{compaction strategy}. RocksDB may be
tuned for high write throughput or high read throughput, for space efficiency, or something in
between.


Used in
\begin{itemize}
\item \textbf{Database}:
\item \textbf{Stream processing}:
\item \textbf{Logging/queuing services}:
\item \textbf{Index service}:
\item \textbf{Caching on SSD}:
\end{itemize}



\begin{table}[htbp]
\caption{RocksDB use cases and their workload characteristics}
\centering
\begin{tabular}{llll}
\hline
 & Read/Write & Read Types & Special Characteristics\\
\hline
Databases & Mixed & Get + Iterator & Transactions and backups\\
Stream Processing & Write-Heavy & Get or Iterator & Time window and checkpoints\\
Logging/Queues & Write-Heavy & Iterator & Support on HDD too\\
Index Services & Read-Heavy & Iterator & Bulk loading\\
Cache & Write-Heavy & Get & Can drop data\\
\hline
\end{tabular}
\end{table}


\begin{table}[htbp]
\caption{System metrics for a typical use case from each application category}
\centering
\begin{tabular}{lllll}
\hline
 & CPU & Space Util & Flash Endurance & Read Bandwidth\\
\hline
Stream Processing & 11\% & 48\% & 16\% & 1.6\%\\
Logging/Queues & 46\% & 45\% & 7\% & 1.0\%\\
Index Services & 47\% & 61\% & 5\% & 10.0\%\\
Cache & 3\% & 78\% & 74\% & 3.5\%\\
\hline
\end{tabular}
\end{table}
\section{Background}
\label{sec:org7dae739}
\subsection{Embedded storage on flash based SSDs}
\label{sec:org3f768f2}
The high performance of the SSD, in many cases, also shifted the performance bottleneck from device
I/O to the network for both of latency and throughput. It became more attractive for applications to
design their architecture to store data on local SSDs rather than use a remote data storage ser- vice.
This increased the demand for a key-value store engines that are embedded in applications.
\subsection{RocksDB architecture}
\label{sec:orgc2f5fad}
\textbf{Writes}. Whenever data is written to RocksDB, it is added to an in-memory write buffer called \textbf{MemTable}, as well as an
on-disk \textbf{Write Ahead Log (WAL)}. Memtable is implemented as a skiplist so keep the data ordered with \(O(\log n)\) insert and
search overhead. The WAL is used for recovery after a failure, but is not mandatory. Once the size of the MemTable reaches
a configured size, then
\begin{enumerate}
\item the MemTable and WAL become immutable,
\item a new MemTable and WAL are allocated for subsequent writes,
\item the contents of the MemTable are flushed to a “Sorted String Table” (SSTable) data file on disk,
\item the flushed MemTable and associated WAL are discarded.
\end{enumerate}
Each SSTable stores data in sorted order, divided into uniformly-sized blocks. Each SSTable also has an index block
with one index entry per SSTable block for binary search.

\textbf{Compaction}. Levels higher than Level-0 are created by a process called \textbf{compaction}. The size of
SSTables on a given level are limited by configuration parameters. When level-L’s size target is exceeded, some
SSTables in level-L are selected and merged with the overlapping SSTables in level-(L+1). This process
gradually migrates written data from Level-0 to the last level. Compaction I/O is efficient as it can
be parallelized and only involves bulk reads and writes of entire files.

\textbf{Reads}. In the read path, a key lookup occurs at each successive level until the key is found or it is
determined that the key is not present in the last level.

RocksDB supports multiple different types of compaction:
\begin{itemize}
\item \textbf{Leveled Compaction}: levels are assigned exponentially increasing size
\item \textbf{Tiered Compaction} (\textbf{Universal Compaction} in RocksDB): Similar to Cassandra or HBase. Multiple sorted runs are lazily compacted
together, either when there are too many sorted runs, or the ratio between total DB size over the
size of the largest sorted run exceeds a configurable threshold.
\item \textbf{FIFO Compaction}: discards old files once the DB hits a size limit and only performs lightweight
compaction. It targets in-memory caching applications.
\end{itemize}


\#+CAPTION Write amplification, overhead and read I/O for three compaction types
\begin{center}
\begin{tabular}{lrrr}
\hline
Compaction & Leveled & Tiered & FIFO\\
\hline
Write Amplification & 16.07 & 4.8 & 2.14\\
Max Space Overhead & 9.8\% & 94.4\% & N/A\\
Avg Space Overhead & 9.5\% & 45.5\% & N/A\\
\# I/O per Get() with bloom filter & 0.99 & 1.03 & 1.16\\
\# I/O per Get() without bloom filter & 1.7 & 3.39 & 528\\
\# I/O per iterator seek & 1.84 & 4.80 & 967\\
\hline
\end{tabular}
\end{center}
\section{Evolution of resource optimization targets}
\label{sec:org19a3a20}
\subsection{Write amplification}
\label{sec:org3d2bfda}
Write amplification emerges at two levels:
\begin{enumerate}
\item SSDs themselves introduct write amplification: by their observation between 1.1 and 3.
\item Storage and database software also generae write amplification; this can sometimes be as high as
100 (e.g., when an entire 4KB/8KB/16KB page is written out for changes of less than 100 bytes)

Level Compaction in RocksDB usually exhibits write amplification between 10 and 30, which is
several times better than when using B-trees in many cases.
\end{enumerate}
\subsection{Space amplification}
\label{sec:org400aeb8}
We observed that for most applications, space utilization was far more important than write
amplification, given that neither flash write cycles nor write overhead were constraining.

In fact the number of IOPS utilized in practice was low compared to what the SSD could provide. As a
result, we shifted our resource optimization target to disk space.

We developed \textbf{Dynamic Leveled Compaction}, where the size of each level in the tree is automatically
adjusted based on the actual size of the last level.

\begin{center}
\begin{tabular}{|l|l|l|l|l|l|l|}
\hline
 & \# keys & 200 & 400 & 600 & 800 & 1000 \\
 & (millions) & & & & & \\
 & & & & & & \\
\hline
 & Fully & 12.0 & 24.0 & 36.0 & 48.0 & 60.1 \\
Dynamgic & compacted size & & & & & \\
Leveled & (GB) & & & & & \\
\cline{2-7}
 & Steady DB size & 13.5 & 26.9 & 40.4 & 54.2 & 67.5 \\
 & (GB) & & & & & \\
\cline{2-7}
 & Space overhead & 12.4 & 11.8 & 12.2 & 12.7 & 12.4 \\
 & (\%) & & & & & \\
 & & & & & & \\
\hline
 & Fully & 12.0 & 24.0 & 36.4 & 48.3 & 60.3 \\
LevelDB-style & Compacted size & & & & & \\
Compaction & (GB) & & & & & \\
\cline{2-7}
 & Steady DB & 15.1 & 26.9 & 42.5 & 57.9 & 73.8 \\
 & size (GB) & & & & & \\
\cline{2-7}
 & Space & 25.6 & 12.2 & 16.9 & 19.7 & 22.4 \\
 & overhead (\%) & & & & & \\
 & & & & & & \\
\hline
\end{tabular}
\end{center}
\subsection{CPU utilization}
\label{sec:org68fb47b}
\begin{enumerate}
\item prefix bloom filter
\item applying the bloom filter before index lookups
\item bloom filter improvements
\end{enumerate}
\subsection{Adapting to newer technologies}
\label{sec:org00cd418}
Disaggregated (remote) storage appears to be a much more interesting optimization target and is a
current priority. Faster networks currently allow many more I/Os to be served remotely, so the
performance of running RocksDB with remote storage has become viable for an increasing number of applications.
\subsection{Main Data Structure Revisited}
\label{sec:org27ed733}
WiscKey/ForrestDB
\section{Lessons on serving large-scale systems}
\label{sec:org95862dc}
\subsection{Resource management}
\label{sec:orgdf3af3a}
The fact that a host may run many RocksDB instances has implications on resource management. Given
that the instances share the host’s resources, the resources need to be managed both globally (per
host) and locally (per instance) to ensure they are used fairly and efficiently. When running in
single process mode, having global resource limits is im- portant, including for
\begin{enumerate}
\item memory for write buffer and block cache
\item compaction I/O bandwidth
\item compaction threads
\item total disk usage
\item file deletion rate
\end{enumerate}
\subsection{WAL treatment}
\label{sec:org0793894}
For example, if copies of the same data exist in multiple replicas, and one replica becomes corrupted
or inaccessible, then the storage system uses valid replica(s) from other unaffected hosts to rebuild
the replica of the failed host. For such systems, RocksDB WAL writes are less critical. Further,
distributed systems often have their own replication logs (e.g., Paxos logs), in which case RocksDB
WAL are not needed at all.
\subsection{Rate-limited file deletions}
\label{sec:org15b1c86}
Rate-limited file deletions RocksDB typically interacts with the underlying storage device via a file
system. These file systems are flash-SSD-aware; e.g., XFS, with realtime discard, may issue a \textbf{TRIM}
command to the SSD whenever a file is deleted. TRIM commands are commonly believed to improve
performance and flash endurance. However, it may also cause \uline{performance issue}. In addition to updating
the address mapping, the SSD firmware also needs to write these changes to FTL(Flash Translation
Layer)'s journal in flash, which in turn may trigger SSD's internal garbage collection. To avoid TRIM
activity spikes and associated increases in I/O latency, we introduced rate limiting for file deletion
to prevent multiple files from being deleted simultaneously.
\subsection{Data format compatibility}
\label{sec:orge4fc426}
It is important that the data on disk remain both backward and forward compatible across the different
software versions.
\subsection{Managing configurations}
\label{sec:org77edd22}
\subsection{Replication and backup support}
\label{sec:orge2a99b1}
Bootstraping a new replica by copying all the data from an existing one can be done in two ways:
\begin{enumerate}
\item read all keys from a source replica and then written to the destination replica (\textbf{logical copying}).
On the source side, RocksDB supports data scanning operations by offering the ability to minimize
the impact on concurrent online queries; e.g., by providing the option to not cache the result of
these operations
\item Copying SSTables and other files directly (\textbf{physical copying}). RocksDB assist physical copying by
identifying existing database files at a current point in time, and preventing them from being
deleted or mutated.
\end{enumerate}
\section{Lessons on failure handling}
\label{sec:orga9052b4}
\subsection{Frequency of silent corruptions}
\label{sec:org9390862}
CPU/memory corruption does happen rarely and it is difficult to accurately quantify.
\end{document}
