% Created 2025-07-10 Thu 19:41
% Intended LaTeX compiler: xelatex
\documentclass[11pt]{article}
\usepackage{capt-of}
\usepackage{hyperref}
% TIPS
% \substack{a\\b} for multiple lines text





% pdfplots will load xolor automatically without option
\usepackage[dvipsnames]{xcolor}

\usepackage{forest}
% two-line text in node by [two \\ lines]
% \begin{forest} qtree, [..] \end{forest}
\forestset{
  qtree/.style={
    baseline,
    for tree={
      parent anchor=south,
      child anchor=north,
      align=center,
      inner sep=1pt,
    }}}
%\usepackage{flexisym}
% load order of mathtools and mathabx, otherwise conflict overbrace

\usepackage{mathtools}
%\usepackage{fourier}
\usepackage{pgfplots}
\usepackage{amsthm, mathabx,  amsmath, commath}
\usepackage{amsfonts}

\usepackage{empheq}
\usepackage{tikz}
\usetikzlibrary{arrows.meta}
\usepackage[most]{tcolorbox}

\newtheorem{theorem}{Theorem}[section]
\newtheorem{definition}{Definition}[section]
\newtheorem{corollary}{Corollary}[section]
\newtheorem{example}{Example}[section]
\newtheorem{lemma}{Lemma}[section]
\newtheorem{proposition}{Proposition}[section]

\newcommand{\bl}[1] {\boldsymbol{#1}}
\newcommand{\Wt}[1] {\stackrel{\sim}{\smash{#1}\rule{0pt}{1.1ex}}}
\newcommand{\wt}[1] {\widetilde{#1}}


%For boxed texts in align, use Aboxed{}
%otherwise use boxed{}

\DeclareMathSymbol{\widehatsym}{\mathord}{largesymbols}{"62}
\newcommand\lowerwidehatsym{%
  \text{\smash{\raisebox{-1.3ex}{%
    $\widehatsym$}}}}
\newcommand\fixwidehat[1]{%
  \mathchoice
    {\accentset{\displaystyle\lowerwidehatsym}{#1}}
    {\accentset{\textstyle\lowerwidehatsym}{#1}}
    {\accentset{\scriptstyle\lowerwidehatsym}{#1}}
    {\accentset{\scriptscriptstyle\lowerwidehatsym}{#1}}
}

\usepackage{graphicx}
    
% text on arrow for xRightarrow
\makeatletter
%\newcommand{\xRightarrow}[2][]{\ext@arrow 0359\Rightarrowfill@{#1}{#2}}
\makeatother


\def \bx {\boldsymbol{x}}
\def \ba {\boldsymbol{a}}
\def \bI {\boldsymbol{I}}
\def \bt {\boldsymbol{t}}
\def \bb {\boldsymbol{b}}
\def \bA {\boldsymbol{A}}
\def \bX {\boldsymbol{X}}
\def \bu {\boldsymbol{u}}
\def \bS {\boldsymbol{S}}
\def \bZ {\boldsymbol{Z}}
\def \bz {\boldsymbol{z}}
\def \by {\boldsymbol{y}}
\def \bw {\boldsymbol{w}}
\def \bT {\boldsymbol{T}}
\def \bS {\boldsymbol{S}}
\def \bm {\boldsymbol{m}}
\def \bW {\boldsymbol{W}}
\def \bY {\boldsymbol{Y}}
\def \bH {\boldsymbol{H}}
\def \blambda {\boldsymbol{\lambda}}
\def \bPhi {\boldsymbol{\Phi}}
\def \btheta {\boldsymbol{\theta}}
\def \bmu {\boldsymbol{\mu}}
\def \bphi {\boldsymbol{\phi}}
\def \bSigma {\boldsymbol{\Sigma}}
\def \lb {\left\{}
\def \rb {\right\}}
\def \caln {\mathcal{N}}
\def \dissum {\displaystyle\Sigma}
\def \dispro {\displaystyle\prod}
\def \E {\mathbb{E}}
\def \Q {\mathbb{Q}}
\def \V {\mathbb{V}}
\def \R {\mathbb{R}}
\def \calq {\mathcal{Q}}
\def \calg {\mathcal{G}}
\def \caln {\mathcal{N}}
\def \calr {\mathcal{R}}
\def \calm {\mathcal{M}}
\def \calc {\mathcal{C}}
\def \bcup {\bigcup}

\graphicspath{{../../../paper/data_structure/}}

%% ox-latex features:
%   !announce-start, !guess-pollyglossia, !guess-babel, !guess-inputenc, caption,
%   image, !announce-end.

\usepackage{capt-of}

\usepackage{graphicx}

%% end ox-latex features


\date{\today}
\title{Memc3: Compact and Concurrent MemCache with Dumber Caching and Smarter Hashing}
\hypersetup{
 pdfauthor={},
 pdftitle={Memc3: Compact and Concurrent MemCache with Dumber Caching and Smarter Hashing},
 pdfkeywords={},
 pdfsubject={},
 pdfcreator={},
 pdflang={English}}
\begin{document}

\maketitle
\section{Background}
\label{sec:org52dca7c}
\subsection{Memcached Overview}
\label{sec:org52990d1}
\textbf{Interface}:
\begin{itemize}
\item \texttt{SET/ADD/REPLACE(key,value)}
\item \texttt{GET(key)}
\item \texttt{DELETE(key)}
\end{itemize}

\textbf{Hash Table}

\textbf{Memory Allocation}: Memcached uses \textbf{slab-based memory allocation}. Memory is divided into 1MB pages, and
each page is further sub-divided into fixed-length \textbf{chunks}. Key-value objects are stored in an
appropriately-size chunk. The size of a chunk, and thus the number of chunks per page, depends on the
particular slab class. For example, by default the chunk size of slab class 1 is 72 bytes and each
page of this class has 14563 chunks; while the chunk size of slab class 43 is 1 MB and thus there is
only 1 chunk spanning the whole page.

To insert a new key, Memcached looks up the slab class whose chunk size best fits this key-value
object. If a vacant chunk is available, it is assigned to this item; if the search fails, Memcached
will execute cache eviction

\textbf{Cache policy}: Each slab class maintains its own objects in an LRU queue.
\begin{center}
\includegraphics[width=.7\textwidth]{../../images/papers/174.png}
\captionof{figure}{\label{1}Memcached data structures}
\end{center}

\textbf{Threading}
\subsection{Real-world Workloads: Small and Read-only Requests Dominate}
\label{sec:org0bded42}
\section{Optimistic Concurrent Cuckoo Hashing}
\label{sec:orga683a29}
\textbf{Basic Cuckoo Hashing}: The basic idea of cuckoo hashing is to use two hash functions instead of one,
thus providing each key two possible locations where it can reside. Cuckoo hashing can dynamically
relocate existing keys and refine the table to make room for new keys during insertion.

Our hash table, as shown in Figure \ref{2}, consists of an array of \textbf{buckets}, each having 4 \textbf{slots}. Each
slot contains a \textbf{pointer} to the key-value object and a short summary of the key called a \textbf{tag}. To
support keys of variable length, the full keys and values are not stored in the hash table, but stored
with the associated metadata outside the table and referenced by the pointer. A null pointer indicates
this slot is not used.

\begin{center}
\includegraphics[width=.7\textwidth]{../../images/papers/175.png}
\label{2}
\end{center}

Each key is mapped to two random buckets, so \texttt{Lookup} checks all 8 candidate keys from every slot. To
insert a new key \(x\) into the table, if either of the two buckets has an empty slot, it is then
inserted in that bucket; if neither bucket has space, \texttt{Insert} selects a random key \(y\) from one
candidate bucket and relocates \(y\) to its own alternate location. Displacing \(y\) may also require
kicking out another existing key \(z\), so this procedure may repeat until a vacant slot is found, or
until a maximum number of displacements is reached (e.g., 500 times in our implementation). If no
vacant slot found, the hash table is considered too full to insert and an expansion process is
scheduled. Though it may execute a sequence of displacements, the amortized insertion time of cuckoo
hashing is O(1)
\subsection{Tag-based Lookup/Insert}
\label{sec:org9be56c5}
We propose a cache-aware technique to perform cuckoo hashing with minimum memory references by using
\textbf{tags} —a short hash of the keys (one-byte in our implementation).

\textbf{Cache-friendly Lookup}: Checking two buckets on each \texttt{Lookup} makes up to 8 (parallel) pointer
dereferences. In addition, displacing each key on \texttt{Insert} also requires a pointer dereference to
calculate the alternate location to swap, and each \texttt{Insert} may perform several displacement operations

Our hash table eliminates the need for pointer dereferences in the common case. We compute a 1-Byte
tag as the summary of each inserted key, and store the tag in the same bucket as its pointer. \texttt{Lookup}
first compares the tag, then retrieves the full key only if the tag matches.

Because each bucket fits in a CPU cacheline, on average each \texttt{Lookup} makes only 2 parallel
cacheline-sized reads for checking the two buckets.

\textbf{Cache-friendly Insert}: We also use the tags to avoid retrieving full keys on \texttt{Insert}. To this end, our
hashing scheme computes the two candidate buckets \(b_1\) and \(b_2\) for key \(x\) by
\begin{align*}
b_1&=HASH(x)\\
b_2&=b_1\oplus HASH(tag)
\end{align*}
Now \(b_1\) can be computed by the same formula from \(b_2\) and tag. This property ensures that to
displace a key originally in bucket \(b\) - no matter if \(b\) is \(b_1\) or \(b_2\) - it is possible
to calculate its alternative bucket \(b'\) from bucket index \(b\) and the tag stored in bucket \(b\)
by
\begin{equation*}
b'=b\oplus HASH(tag)
\end{equation*}
\subsection{Concurrent Cuckoo Hashing}
\label{sec:org7ed4236}
\begin{center}
\includegraphics[width=.7\textwidth]{../../images/papers/176.png}
\captionof{figure}{\label{3}Cuckoo path. \(\emptyset\) represents an empty slot}
\end{center}

Define a \textbf{cuckoo path} as the sequence of displaced keys in an \texttt{Insert} operation. In Figure \ref{3},
\(a\Rightarrow b\Rightarrow c\) is one cuckoo path to make one bucket available to insert key \(x\).

Two obstacles:
\begin{enumerate}
\item \textbf{Deadlock risk (writer/writer)}:
\item \textbf{False misses (reader/writer)}
\end{enumerate}

To avoid writer/writer deadlocks, it allows only one writer at a time - a tradeoff we accept as our
target workloads are read-heavy. To eliminate false misses, our design changes the order of the basic
cuckoo hashing insertion by:
\begin{enumerate}
\item \emph{separating discovering a valid cuckoo path from the execution of this path}. We first search for a
cuckoo path, but do not move keys during this search phase
\item \emph{moving keys backwards along the cuckoo path}. After a valid cuckoo path is known, we first move the
last key on the cuckoo path to the free slot, and then move the second to last key to the empty
slot left by the previous one, and so on. As a result, each swap affects only one key at a time,
which can always be successfully moved to its new location without any kickout. \wu\{So insert does
not affect get\}
\end{enumerate}
\subsubsection{Optimization: Optimistic Locks for Lookup}
\label{sec:org5e932a9}
Instead of locking on buckets, it assigns a version counter for each key, updates its version when
displacing this key on \texttt{Insert}, and looks for a version change during \texttt{Lookup} to detect any concurrent
displacement.

\textbf{Lock Striping}: The simplest way to maintain each key’s version is to store it inside each key-value
object. This approach, however, adds one counter for each key and there could be hundred of millions
of keys. More importantly, this approach leads to a race condition: to check or update the version of
a given key, we must first lookup in the hash table to find the key-value object (stored external to
the hash table), and this initial lookup is not protected by any lock and thus not thread-safe.

Instead, we create an array of counters (Figure \ref{2}). To keep this array small, each counter is
shared among multiple keys by hashing (e.g., the i-th counter is shared by all keys whose hash value
is \(i\)). Our implementation keeps 8192 counters in total (or 32 KB). This permits the counters to
fit in cache, but allows substantial concurrent access. It also keeps the chance of a “false retry”
(re-reading a key due to modification of an unrelated key) to roughly 0.01\%. All counters are
initialized to 0 and only read/updated by atomic memory operations to ensure the consistency among all
threads.

\textbf{Optimistic Locking}: Before displacing a key, an \texttt{Insert} process first increases the relevant counter by
one, indicating to the other Lookups an on-going update for this key; after the key is moved to its
new location, the counter is again increased by one to indicate the completion. As a result, the key
version is increased by 2 after each displacement.

Before a Lookup process reads the two buckets for a given key, it first snapshots the version stored
in its counter: If this version is odd, there must be a concurrent \texttt{Insert} working on the same key (or
another key sharing the same counter), and it should wait and retry; otherwise it proceeds to the two
buckets. After it finishes reading both buckets, it snapshots the counter again and compares its new
version with the old version. If two versions differ, the writer must have modified this key, and the
Lookup should retry.
\subsubsection{Optimization: Multiple Cuckoo Paths}
\label{sec:orgbe0927d}
Our revised \texttt{Insert} process first looks for a valid cuckoo path before swapping the key along the path.
Due to the separation of search and execution phases, we apply the following optimization to speed
path discovery and increase the chance of finding an empty slot.

Instead of searching for an empty slot along one cuckoo path, our \texttt{Insert} process keeps track of
multiple paths in parallel. At each step, multiple victim keys are “kicked out,” each key extending
its own cuckoo path. Whenever one path reaches an available bucket, this search phase completes.
\section{Concurrent Cache Management}
\label{sec:org2102f5e}
When serving small key-value objects, this too becomes a major source of space overhead in Memcached,
which requires 18 Bytes for each key (i.e., two pointers and a 2-Byte reference counter) to ensure
that keys can be evicted safely in a strict LRU order. String LRU cache management is also a
synchronization bottleneck, as all updates to the cache must be serialized in Memcached.

This section presents our efforts to make the cache management \emph{space efficient} (1 bit per key) and
\emph{concurrent} (no synchronization to update LRU) by implementing an approximate LRU cache based on the
CLOCK replacement algorithm. CLOCK is a well-known algorithm; our contribution lies in integrating
CLOCK replacement with the optimistic, striped locking in our cuckoo algorithm to reduce both locking
and space overhead.

\textbf{CLOCK Replacement}: A cache must implement two functions related to its replacement policy:
\begin{itemize}
\item \texttt{Update} to keep track of the recency after querying a key in the cache
\item \texttt{Evict} to select keys to purge when inserting keys into a full cache
\end{itemize}

Memcached keeps each key-value entry in a doubly-linked-list based LRU queue within its own slab
class. After each cache query, \texttt{Update} moves the accessed entry to the head of its own queue; to free
space when the cache is full, \texttt{Evict} replaces the entry on the tail of the queue by the new key-value
pair. This ensures strict LRU eviction in each queue, but unfortunately it also requires two pointers
per key for the doubly-linked list and, more importantly, all Updates to one linked list are
serialized. Every read access requires an update, and thus the queue permits no concurrency even for
read-only workloads.

CLOCK approximates LRU with improved concurrency and space efficiency. For each slab class, we
maintain a \textbf{circular buffer} and a \textbf{virtual hand}; each bit in the buffer represents the recency of a
differen key-value object: 1 for ``recently used'' and 0 otherwise. Each \texttt{Update} simply sets the recency
bit to 1 on each key access; each \texttt{Evict} checks the bit currently pointed by the hand. If the current
bit is 0, \texttt{Evict} selects the corresponding key-value object; otherwise we reset this bit to 0 and
advance the hand in the circular buffer until we see a bit of 0.

\textbf{Integration with Optimisitc Cuckoo Hashing}: The \texttt{Evict} process must coordinate with reader threads to
ensure the eviction is safe. Otherwise, a key-value entry may be overwritten by a new key-value pair
after eviction, but threads still accessing the entry for the evicted key may read dirty data. To
this end, the original Memcached adds to each entry a 2-Byte reference counter to avoid this rare
case. Reading this per-entry counter, the \texttt{Evict} process knows how many other threads are accessing
this entry concurrently and avoid evicting those busy entries.

Our cache integrates cache eviction with our optimistic locking scheme for cuckoo hashing. When \texttt{Evict}
selects a victim key \(x\) by CLOCK, it first increases key \(x\)’s version counter to inform other
threads currently reading \(x\) to retry; it then deletes \(x\) from the hash table to make \(x\)
unreachable for later readers, including those retries; and finally it increases key \(x\)’s version
counter again to complete the change for \(x\). Note that \texttt{Evict} and the hash table \texttt{Insert} are both
serialized (using locks) so when updating the counters they can not affect each other.

With Evict as above, our cache ensures consistent \texttt{GET}s by version checking. Each \texttt{GET} first
snapshots the version of the key before accessing the hash table; if the hash table returns a valid
pointer, it follows the pointer and reads the value assoicated. Afterwards, \texttt{GET} compares the latest
key version with the snapshot. If the verions differ, then \texttt{GET} may have observed an inconsistent
intermediate state and must retry.

\begin{center}
\includegraphics[width=.7\textwidth]{../../images/papers/177.png}
\label{a1}
\end{center}
\section{Problemsp}
\label{sec:org9a3e213}


\section{References}
\label{sec:org4453499}
\label{bibliographystyle link}
\bibliographystyle{alpha}

\bibliography{/Users/wu/notes/notes/references.bib}
\end{document}
