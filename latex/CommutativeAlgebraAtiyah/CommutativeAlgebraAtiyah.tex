% Created 2021-11-10 Wed 13:37
% Intended LaTeX compiler: pdflatex
\documentclass[11pt]{article}
\usepackage[utf8]{inputenc}
\usepackage[T1]{fontenc}
\usepackage{graphicx}
\usepackage{longtable}
\usepackage{wrapfig}
\usepackage{rotating}
\usepackage[normalem]{ulem}
\usepackage{amsmath}
\usepackage{amssymb}
\usepackage{capt-of}
\usepackage{hyperref}
\graphicspath{{../../books/}}
% TIPS
% \substack{a\\b} for multiple lines text





% pdfplots will load xolor automatically without option
\usepackage[dvipsnames]{xcolor}

\usepackage{forest}
% two-line text in node by [two \\ lines]
% \begin{forest} qtree, [..] \end{forest}
\forestset{
  qtree/.style={
    baseline,
    for tree={
      parent anchor=south,
      child anchor=north,
      align=center,
      inner sep=1pt,
    }}}
%\usepackage{flexisym}
% load order of mathtools and mathabx, otherwise conflict overbrace

\usepackage{mathtools}
%\usepackage{fourier}
\usepackage{pgfplots}
\usepackage{amsthm, mathabx,  amsmath, commath}
\usepackage{amsfonts}

\usepackage{empheq}
\usepackage{tikz}
\usetikzlibrary{arrows.meta}
\usepackage[most]{tcolorbox}

\newtheorem{theorem}{Theorem}[section]
\newtheorem{definition}{Definition}[section]
\newtheorem{corollary}{Corollary}[section]
\newtheorem{example}{Example}[section]
\newtheorem{lemma}{Lemma}[section]
\newtheorem{proposition}{Proposition}[section]

\newcommand{\bl}[1] {\boldsymbol{#1}}
\newcommand{\Wt}[1] {\stackrel{\sim}{\smash{#1}\rule{0pt}{1.1ex}}}
\newcommand{\wt}[1] {\widetilde{#1}}


%For boxed texts in align, use Aboxed{}
%otherwise use boxed{}

\DeclareMathSymbol{\widehatsym}{\mathord}{largesymbols}{"62}
\newcommand\lowerwidehatsym{%
  \text{\smash{\raisebox{-1.3ex}{%
    $\widehatsym$}}}}
\newcommand\fixwidehat[1]{%
  \mathchoice
    {\accentset{\displaystyle\lowerwidehatsym}{#1}}
    {\accentset{\textstyle\lowerwidehatsym}{#1}}
    {\accentset{\scriptstyle\lowerwidehatsym}{#1}}
    {\accentset{\scriptscriptstyle\lowerwidehatsym}{#1}}
}

\usepackage{graphicx}
    
% text on arrow for xRightarrow
\makeatletter
%\newcommand{\xRightarrow}[2][]{\ext@arrow 0359\Rightarrowfill@{#1}{#2}}
\makeatother


\def \bx {\boldsymbol{x}}
\def \ba {\boldsymbol{a}}
\def \bI {\boldsymbol{I}}
\def \bt {\boldsymbol{t}}
\def \bb {\boldsymbol{b}}
\def \bA {\boldsymbol{A}}
\def \bX {\boldsymbol{X}}
\def \bu {\boldsymbol{u}}
\def \bS {\boldsymbol{S}}
\def \bZ {\boldsymbol{Z}}
\def \bz {\boldsymbol{z}}
\def \by {\boldsymbol{y}}
\def \bw {\boldsymbol{w}}
\def \bT {\boldsymbol{T}}
\def \bS {\boldsymbol{S}}
\def \bm {\boldsymbol{m}}
\def \bW {\boldsymbol{W}}
\def \bY {\boldsymbol{Y}}
\def \bH {\boldsymbol{H}}
\def \blambda {\boldsymbol{\lambda}}
\def \bPhi {\boldsymbol{\Phi}}
\def \btheta {\boldsymbol{\theta}}
\def \bmu {\boldsymbol{\mu}}
\def \bphi {\boldsymbol{\phi}}
\def \bSigma {\boldsymbol{\Sigma}}
\def \lb {\left\{}
\def \rb {\right\}}
\def \caln {\mathcal{N}}
\def \dissum {\displaystyle\Sigma}
\def \dispro {\displaystyle\prod}
\def \E {\mathbb{E}}
\def \Q {\mathbb{Q}}
\def \V {\mathbb{V}}
\def \R {\mathbb{R}}
\def \calq {\mathcal{Q}}
\def \calg {\mathcal{G}}
\def \caln {\mathcal{N}}
\def \calr {\mathcal{R}}
\def \calm {\mathcal{M}}
\def \calc {\mathcal{C}}
\def \bcup {\bigcup}

\makeindex
\author{M. F. Atiyah \& I. G. MacDonald}
\date{\today}
\title{Introduction to Commutative Algebra}
\hypersetup{
 pdfauthor={M. F. Atiyah \& I. G. MacDonald},
 pdftitle={Introduction to Commutative Algebra},
 pdfkeywords={},
 pdfsubject={},
 pdfcreator={Emacs 27.2 (Org mode 9.5)}, 
 pdflang={English}}
\begin{document}

\maketitle
\tableofcontents


\section{Rings and Ideals}
\label{sec:org2f2a0f9}
A \textbf{ring homomorphism} is a mapping \(f\) of a ring \(A\) into a ring \(B\) s.t.
\begin{enumerate}
\item \(f(x+y)=f(x)+f(y)\)
\item \(f(xy)=f(x)f(y)\)
\item \(f(1)=1\)
\end{enumerate}


An \textbf{ideal} \(\fa\) of a ring \(A\) is a subset of \(A\) which is an additive subgroup and is
s.t. \(A\fa\subseteq\fa\). The quotient group \(A/\fa\) inherits a uniquely defined multiplication from \(A\)
which makes it into a ring, called the \textbf{quotient ring} \(A/\fa\). The elements of \(A/\fa\) are the
cosets of \(\fa\) in \(A\), and the mapping \(\phi:A\to A/\fa\) which maps each \(x\in A\) to its
coset \(x+\fa\) is a surjective ring homomorphism

\begin{proposition}[]
\label{1.1}
There is a one-to-one order-preserving correspondence between the ideals \(\fb\) of \(A\) which
contain \(\fa\), and the ideals \(\bar{\fb}\) of \(A/\fa\), given by \(\fb=\phi^{-1}(\bar{\fb})\).
\end{proposition}

\begin{proof}
Let \(S_1=\{\fb:\fb\text{ an ideal of $A$ and }\fa\subseteq\fb\}\)
and \(S_2=\{\bar{\fb}:\bar{\fb}\text{ an ideal of }A/\fa\}\), \(\pi\) is the natural map \(\pi(S)=S/\fa\), we prove that
\begin{equation*}
\varphi:S_1\to S_2\hspace{1cm}\fb\mapsto\pi(\fb)
\end{equation*}
is an bijection.

First assume that \(\fa\subseteq\fb\), we prove that \(\pi^{-1}\pi(\fb)=\fb\). Apparently \(\fb\subseteq\pi^{-1}\pi(\fb)\). For
any \(b\in\pi^{-1}\pi(\fb)\), there is a \(s\in\fb\) s.t. \(\pi(b)=\pi(s)\). Thus \(b-s\in\ker\pi=\fa\). As \(\fa\subseteq\fb\),
we have \(b\in\fb\). Hence \(\pi^{-1}\pi(\fb)=\fb\).

Thus for any \(\fb_1,\fb_2\in S_1\) and \(\varphi(\fb_1)=\pi(\fb_1)=\pi(\fb_2)=\varphi(\fb_2)\), we have \(\pi^{-1}\pi(\fb_1)=\pi^{-1}\pi(\fb_2)\).
Thus \(\varphi\) is injective.

For any \(\bar{\fb}\in S_2\), \(\pi^{-1}(\bar{\fb})\) contains \(\fa=\pi^{-1}(\{0\})\). Hence \(\varphi\) is surjective

Order-preserving means \(\fa\subseteq\fb\subseteq\fc\) iff \(\bar{\fb}\subseteq\bar{\fc}\)
\end{proof}

If \(f:A\to B\) is any ring homomorphism, the \textbf{kernel} of \(f\) is an ideal \(\fa\) of \(A\), and the
image of \(f\) is a subring \(C\) of \(B\); and \(f\) induces a ring isomorphism \(A/\fa\cong C\)

We shall sometimes use the notation \(x\equiv y\mod\fa\); this means that \(x-y\in\fa\)

A \textbf{zero-divisor} in a ring \(A\) is an element \(x\) which divides 0, i.e., for which there
exists \(y\neq 0\) in \(A\) s.t. \(xy=0\). A ring with no zero-divisor \(\neq 0\) (and in
which \(1\neq 0\)) is called an \textbf{integral domain}.

An element \(x\in A\) is \textbf{nilpotent} if \(x^n=0\) for some \(n>0\). A nilpotent element is a
zero-divisor (unless \(A=0\))

A \textbf{unit} in \(A\) is an element \(x\) which ``divides 1'', i.e., an element \(x\) s.t. \(xy=1\) for
some \(y\in A\). The element \(y\) is then uniquely determined by \(x\), and is
written \(x^{-1}\). The units in \(A\) form a (multiplicative) abelian group

The multiples \(ax\) of an element \(x\in A\) from a \textbf{principal} ideal, denoted by \((x)\)
or \(Ax\). \(x\) is a unit iff \((x)=A=(1)\). The \textbf{zero} ideal \((0)\) is denoted by 0

A \textbf{field} is a ring \(A\) in which \(1\neq 0\) and every non-zero element is a unit. Every field is
an integral domain

\begin{proposition}[]
Let \(A\) be a ring \(\neq 0\). Then the following are equivalent:
\begin{enumerate}
\item \(A\) is a field
\item the only ideals in \(A\) are 0 and (1)
\item every homomorphism of \(A\) into a non-zero ring \(B\) is injective
\end{enumerate}
\end{proposition}

\begin{proof}
\(2\to 3\). Let \(\phi:A\to B\) be a ring homomorphism. Then \(\ker\phi\) is an ideal \(\neq(1)\) in \(A\),
hence \(\ker\phi=0\), hence \(\phi\) is injective

\(3\to 1\). Let \(x\) be an element of \(A\) which is not a unit. Then \((x)\neq(1)\),
hence \(B=A/(x)\) is not the zero ring. Let \(\phi:A\to B\) be the natural homomorphism of \(A\)
onto \(B\) with kernel \((x)\). By hypothesis, \(\phi\) is injective, hence \((x)=0\), hence \(x=0\)
\end{proof}

\index{prime}
An ideal \(\fp\) in \(A\) is \textbf{prime} if \(\fp\neq(1)\) and if \(xy\in\fp\Rightarrow x\in\fp\) or \(y\in\fp\)

An ideal \(\fm\) in \(A\) is \textbf{maximal} if \(\fm\) in \(A\) is \textbf{maximal} if \(\fm\neq(1)\) and if no
ideal \(\fa\) s.t. \(\fm\subset\fa\subset(1)\) (\textbf{strict} inclusions). Equivalently
\begin{gather*}
\fp\text{ is prime } \Leftrightarrow A/\fp\text{ is an integral domain}\\
\fm\text{ is maximal } \Leftrightarrow A/\fm\text{ is a field}
\end{gather*}
\begin{proof}
If \(\fm\) is maximal and suppose \(a\notin\fm\). Then \(J=\{ra+i:i\in \fm\text{ and }r\in A\}\) is an ideal.
Hence \(J=A\). So there is \(r\in A, \fm\in I\) s.t. \(1=ra+i\). So we have \(1\equiv ra\mod \fm\). Hence we
find the inverse of \(a+\fm\)

If \(A/\fm\) is a field and suppose \(\fm\subset\fn\subset A\). Let \(a\in\fm\setminus\fn\), then there exists a \(b\in A\)
s.t. \(ab-1\in\fm\). So \(ab+m=1\) for some \(m\in\fm\). But \(ab\in\fn\) and \(m\in\fm\subset\fn\), then we
have \(1\in\fn\) and \(\fn=A\).
\end{proof}

Hence a maximal ideal is prime. The zero ideal is prime iff \(A\) is an integral domain

If \(f:A\to B\) is a ring homomorphism and \(\fq\) is a prime ideal in \(B\), then \(f^{-1}(\fq)\) is
a prime ideal in \(A\), for \(A/f^{-1}(\fq)\) is isomorphic to a subring of \(B/\fq\) and hence has
no zero-divisor \(\neq 0\). (\href{https://asgarli.wordpress.com/2013/04/21/inverse-image-of-a-prime-ideal-is-prime/}{Explanation}. Since \(\fq\) is prime, \(B/\fq\) is an integral domain and a
subring of an integral domain is still an integral domain. Define the map
\(\varphi(a+f^{-1}(\fq))=f(a)+\fq\) and we need to show its a homomorphism. Then we show its injective.)

But if \(\fn\) is a maximal ideal of \(B\) it is not necessarily true that \(f^{-1}(\fn)\) is
maximal in \(A\); all we can say for sure is that it is prime. (Example: \(A=\Z\), \(B=\Q\), \(\fn=0\)).

\begin{theorem}[]
\label{1.3}
Every ring \(A\neq 0\) has at least one maximal ideal
\end{theorem}

\begin{proof}
This is the standard application of Zorn's lemma. Let \(\Sigma\) be the set of all ideals \(\neq(1)\)
in \(A\). Order \(\Sigma\) by inclusion. \(\Sigma\) is not empty, since \(0\in\Sigma\). To apply Zorn's lemma we must
show that every chain in \(\Sigma\) has an upper bound in \(\Sigma\); let then \((\fa_\alpha)\) be a chain of ideals in
\(\Sigma\), so that for each pair of indices \(\alpha\), \(\beta\) we have either \(\fa_\alpha\subseteq \fa_\beta\) or \(\fa_\beta\subseteq\fa_\alpha\).
Let \(\fa=\bigcup_\alpha\fa_\alpha\). Then \(\fa\) is an ideal and \(1\notin\fa\). Hence \(\fa\in\Sigma\) and is an upper bound of the
chain. Hence \(\Sigma\) has a maximal element
\end{proof}

\begin{corollary}[]
If \(\fa\neq(1)\) is an ideal of \(A\), there exists a maximal ideal of \(A\) containing \(\fa\)
\end{corollary}

\begin{proof}
Apply \ref{1.3} to \(A/\fa\) and \ref{1.3}
\end{proof}

\begin{corollary}[]
\label{1.5}
Every non-unit of \(A\) is contained in a maximal ideal.
\end{corollary}

\index{local ring}
A ring \(A\) with exactly one maximal ideal \(\fm\) is called a \textbf{local ring}. The field \(k=A/\fm\) is
called the \textbf{residue field} of \(A\)

\begin{proposition}[]
\begin{enumerate}
\item Let \(A\) be a ring and \(\fm\neq(1)\) an ideal of \(A\) s.t. every \(x\in A-\fm\) is a unit in \(A\).
Then \(A\) is a local ring and \(\fm\) its maximal ideal.
\item Let \(A\) be a ring and \(\fm\) a maximal ideal of \(A\) s.t. every element of \(1+\fm\) is a
unit in \(A\). Then \(A\) is a local ring
\end{enumerate}
\end{proposition}

\begin{proof}
\begin{enumerate}
\setcounter{enumi}{1}
\item Let \(x\in A-\fm\). Since \(\fm\) is maximal, the ideal generated by \(x\) and \(\fm\)
is \((1)\), hence there exist \(y\in A\) and \(t\in\fm\) s.t. \(xy+t=1\); hence \(xy=1-t\) belongs
to \(1+\fm\) and therefore is a unit. Now use 1
\end{enumerate}
\end{proof}

A ring with only a finite number of maximal ideals is called \textbf{semi-local}

\begin{examplle}[]n
\begin{enumerate}
\item \(A=k[x_1,\dots,x_n]\), \(k\) a field. Let \(f\in A\) be an irreducible polynomial. By unique
factorization, the ideal \((f)\) is prime
\item \(A=\Z\). Every ideal in \(\Z\) is of the form \((m)\) for some \(m\ge 0\). The ideal \((m)\) is
prime iff \(m=0\) or a prime number. All the ideals \((p)\), where \(p\) is a prime number,
are maximal: \(\Z/(p)\) is the field of \(p\) elements
\item A \textbf{principal ideal domain} is an integral domain in which every ideal is principal. In such a
ring every non-zero prime ideal is maximal. For if \((x)\neq 0\) is a prime ideal
and \((y)\supset(x)\), we have \(x\in(y)\), say \(x=yz\), so that \(yz\in(x)\) and \(y\notin(x)\),
hence \(z\in(x)\); say \(z=tx\). Then \(x=yz=ytx\), so that \(yt=1\) and therefore \((y)=(1)\).
\end{enumerate}
\end{examplle}

\begin{proposition}[]
\label{1.7}
The set \(\fN\) of all nilpotent elements in a ring \(A\) is an ideal, and \(A/\fN\) has no
nilpotent \(\neq 0\)
\end{proposition}

\begin{proof}
If \(x\in\fN\), clearly \(ax\in\fN\) for all \(a\in A\). Let \(x,y\in\fN\): say \(x^m=0\), \(y^n=0\). By the
binomial theorem, \((x+y)^{n+m-1`}\) is a sum of integer multiples of products \(x^ry^s\),
where \(r+s=m+n-1\);

Let \(\barx\in A/\fN\) be represented by \(x\in A\). Then \(\barx^n\) is represented by \(x^n\), so
that \(\barx^n=0\Rightarrow x^n\in\fN\Rightarrow(x^n)^k=0\) for some \(k>0\Rightarrow x\in\fN\Rightarrow\barx=0\)
\end{proof}

The ideal \(\fN\) is called the \textbf{nilradical} of \(A\)

Check \href{https://math.stackexchange.com/questions/786393/when-is-nilradical-not-a-prime-ideal}{When is nilradical not a prime ideal}, which is related to Exercise \ref{ex1.19}.

\begin{proposition}[]
\label{1.8}
The nilradical of \(A\) is the intersection of all the prime ideals of \(A\)
\end{proposition}

\begin{proof}
Let \(\fN'\) denote the intersection of all the prime ideals of \(A\). If \(f\in A\) is nilpotent
and if \(\fp\) is a prime ideal, then \(f^n=0\in\fp\) for some \(n>0\), hence \(f\in\fp\). Hence \(f\in\fN'\)

Conversely, suppose that \(f\) is not nilpotent. Let \(\Sigma\) be the set of ideals \(\fa\) with the
property
\begin{equation*}
n>0\Rightarrow f^n\notin\fa
\end{equation*}
Then \(\Sigma\) is not empty because \(0\in\Sigma\). Zorn's lemma can be applied to the set \(\Sigma\), ordered by
inclusion, and therefore \(\Sigma\) has a maximal element. We shall show that \(\fp\) is a prime ideal.
Let \(x,y\notin\fp\). Then the ideals \(\fp+(x)\), \(\fp+(y)\) strictly contain \(\fp\) and therefore do not
belong to \(\Sigma\); hence
\begin{equation*}
f^m\in\fp+(x),\quad f^n\in\fp+(y)
\end{equation*}
for some \(m,n\). It follows that \(f^{m+n}\in\fp+(xy)\), hence the ideal \(\fp+(xy)\) is not in
\(\Sigma\) and therefore \(xy\notin\fp\). Hence we have a prime ideal \(\fp\) s.t. \(f\notin\fp\), so that \(f\notin\fN'\)
\end{proof}

The \textbf{Jacobson radical} \(\fR\) of \(A\) is defined to be the intersection of all the maximal ideals
of \(A\). It can be characterized as follows:

\begin{proposition}[]
\label{1.9}
\(x\in\fR\) iff \(1-xy\) is a unit in \(A\) for all \(y\in A\)
\end{proposition}

\begin{proof}
\(\Rightarrow\): Suppose \(1-xy\) is not a unit. By \ref{1.5} it belongs to some maximal ideal \(\fm\);
but \(x\in\fR\subseteq\fm\), hence \(xy\in\fm\) and therefore \(1\in\fm\), which is absurd

\(\Leftarrow\): Suppose \(x\notin\fm\) for some maximal ideal \(\fm\). Then \(\fm\) and \(x\) generate the unit
ideal \((1)\), so that we have \(u+xy=1\) for some \(u\in\fm\) and some \(y\in A\). Hence \(1-xy\in\fm\)
and is therefore not a unit.
\end{proof}

If \(\fa,\fb\) are ideals in a ring \(A\), their \textbf{sum} \(\fa+\fb\) is the set of all \(x+y\) where \(x\in\fa\)
and \(y\in\fb\). It is the smallest ideal containing \(\fa\) and \(\fb\). More generally, we may define
the sum \(\sum_{i\in I}a_i\) of any family (possibly infinite) of ideals \(\fa_i\) of \(A\); is elements
are all sums \(\sum x_i\), where \(x_i\in\fa_i\) for all \(i\in I\) and almost all of the \(x_i\) (i.e., all
but a finite set) are zero. It is the smallest ideal of \(A\) which contains all the ideals \(\fa_i\)

The \textbf{product} of two ideals \(\fa,\fb\) in \(A\) is the ideal \(\fa\fb\) \textbf{generated} by all products \(xy\),
where \(x\in\fa\) and \(y\in\fb\). It is the set of all finite sums \(\sum x_iy_i\) where each \(x_i\in\fa\) and
each \(y_i\in\fb\)

We have the \textbf{distributive law}
\begin{equation*}
\fa(\fb+\fc)=\fa\fb+\fa\fc
\end{equation*}
In the ring \(\Z\), \(\cap\) and + are distributive over each other. This is not the case in
general. \textbf{modular law}
\begin{equation*}
\fa\cap(\fb+\fc)=\fa\cap\fb+\fa\cap\fb\text{ if }\fa\supseteq\fb\text{ or }\fa\supseteq\fc
\end{equation*}
\begin{equation*}
\fa\cap\fb=\fa\fb\text{ provided }\fa+\fb=(1)
\end{equation*}
If \(x\in\fa\cap\fb\), there is \(a+b=1\). Hence \(xa+xb=x\in\fa\fb\)

Two ideals \(\fa,\fb\) are said to be \textbf{coprime} if \(\fa+\fb=(1)\). Thus for coprime ideals we
have \(\fa\cap\fb=\fa\fb\).

Let \(A\) be a ring and \(\fa_1,\dots,\fa_n\) ideals of \(A\). Define a homomorphism
\begin{equation*}
\phi:A\to\prod_{i=1}^n(A/\fa_i)
\end{equation*}
by the rule \(\phi(x)=(x+\fa_1,\dots,x+\fa_n)\)

\begin{proposition}[]
\begin{enumerate}
\item If \(\fa_i,\fa_j\) are coprime whenever \(i\neq j\), then \(\prod\fa_i=\bigcap\fa_i\)
\item \(\phi\) is surjective iff \(\fa_i\), \(\fa_j\) are coprime whenever \(i\neq j\)
\item \(\phi\) is injective iff \(\bigcap\fa_i=(0)\)
\end{enumerate}
\end{proposition}

\begin{proof}
\begin{enumerate}
\item Induction on \(n\). The case \(n=2\) is dealt with above. Suppose \(n>2\) and the result true
for \(\fa_1,\dots,\fa_{n-1}\), and let \(\fb=\prod_{i=1}^{n-1}\fa_i=\bigcap_{i=1}^{n-1}\fa_i\). As we have \(x_i+y_i=1\)
(\(x_i\in\fa_i,y_i\in\fa_n\)) and therefore
\begin{equation*}
\prod_{i=1}^{n-1}x_i=\prod_{i=1}^{n-1}(1-y_i)\equiv 1\mod \fa_n
\end{equation*}
Hence \(\fa_n+\fb=(1)\) and so
\begin{equation*}
\prod_{i=1}^n\fa_i=\fb\fa_n=\fb\cap\fa_n=\bigcap_{i=1}^n\fa_i
\end{equation*}
\item \(\Rightarrow\): Let's show for example that \(\fa_1,\fa_2\) are coprime. There exists \(x\in A\)
s.t. \(\phi(x)=(1,0,\dots,0)\); hence \(x\equiv 1\mod\fa_1\) and \(x\equiv 0\mod\fa_2\), so that
\begin{equation*}
1=(1-x)+x\in\fa_1+\fa_2
\end{equation*}
\(\Leftarrow\): It is enough to show, for example, that there is an element \(x\in A\)
s.t. \(\phi(x)=(1,0,\dots,0)\). Since \(\fa_1+\fa_i=(1)\) (\(i>1\)) we have \(u_i+v_i=1\) (\(u_i\in\fa_1,v_i\in\fa_i\)).
Take \(x=\prod_{i=2}^nv_i\), then \(x=\prod(1-u_i)\equiv 1\mod\fa_1\). Hence \(\phi(x)=(1,0,\dots,0)\)
\item \(\bigcap\fa_i\) is the kernel of \(\phi\)
\end{enumerate}
\end{proof}

\begin{proposition}[]
\label{1.11}
\begin{enumerate}
\item Let \(\fp_1,\dots,\fp_n\) be prime ideals and let \(\fa\) be an ideal contained in \(\bigcup_{i=1}^n\fp_i\).
Then \(\fa\subseteq\fp_i\) for some \(i\).
\item Let \(\fa_1,\dots,\fa_n\) be ideals and let \(\fp\) be a prime ideal containing \(\bigcap_{i=1}^n\fa_i\).
Then \(\fp\supseteq\fa_i\) for some \(i\). If \(\fp=\bigcap\fa_i\), then \(\fp=\fa_i\) for some \(i\)
\end{enumerate}
\end{proposition}

\begin{proof}
\begin{enumerate}
\item induction on \(n\) in the form
\begin{equation*}
\fa\not\subseteq\fp_i(1\le i\le n)\Rightarrow\fa\not\subseteq\bigcup_{i=1}^n\fp_i
\end{equation*}
It is true for \(n=1\). If \(n>1\) and the result is true for \(n-1\), then for each \(i\)
there exists \(x_i\in\fa\) s.t. \(x_i\notin\fp_j\) whenever \(j\neq i\). If for some \(i\) we have \(x_i\notin\fp_i\),
we are through. If not, then \(x_i\in\fp_i\) for all \(i\). Consider the element
\begin{equation*}
y=\sum_{i=1}^nx_1x_2\cdots x_{i-1}x_{i+1}\cdots x_n
\end{equation*}
we have \(y\in\fa\) and \(y\notin\fp_i\) (\(1\le i\le n\)). Hence \(\fa\not\subseteq\bigcup_{i=1}^n\fp_i\)
\item Suppose \(\fp\not\supseteq\fa_i\) for all \(i\). Then there exist \(x_i\in\fa_i\), \(x_i\notin\fp\) (\(1\le i\le n\)) and
therefore \(\prod x_i\in\prod\fa_i\subseteq\bigcap\fa_i\); but \(\prod x_i\notin\fp\) since \(\fp\) is prime. Hence \(\fp\not\supseteq\bigcap\fa_i\)

If \(\fp=\bigcap\fa_i\), then \(\fp\subseteq\fa_i\) and hence \(\fp=\fa_i\) for some \(i\).
\end{enumerate}
\end{proof}

\wu{
For prime ideals \(\fp_1,\dots,\fp_n\), if \(\bigcap_{i=1}^n\fp_i=\fp\) is a prime ideal, then \(\fp=\fp_i\) for
some \(i\). If there are more than one minimal ideal, this could never happen
}

If \(\fa,\fb\) are ideals in a ring \(A\), their \textbf{ideal quotient} is
\begin{equation*}
(\fa:\fb)=\{x\in A:x\fb\subseteq\fa\}
\end{equation*}
which is an ideal. In particular, \((0:\fb)\) is called the \textbf{annihilator} of \(\fb\) and is also
denoted by \(\Ann(\fb)\): it is the set of all \(x\in A\) s.t. \(x\fb=0\). In this notation the set of
all zero-divisors in \(A\) is
\begin{equation*}
D=\bigcup_{x\neq 0}\Ann(x)
\end{equation*}

If \(\fb\) is a principal ideal \((x)\), we shall write \((\fa:x)\) in place of \((\fa:(x))\)

\begin{examplle}[]
If \(A=\Z\), \(\fa=(m)\), \(\fb=(n)\), where say \(m=\prod_pp^{\mu_p}\), \(n=\prod_pp^{\nu_p}\),
then \((\fa:\fb)=(q)\) where \(q=\prod_pp^{\gamma_p}\) and
\begin{equation*}
\gamma_p=\max(\mu_p-\nu_p,0)=\mu_p-\min(\mu_p,\nu_p)
\end{equation*}
Hence \(q=m/(m,n)\), where \((m,n)\) is the h.c.f. of \(m\) and \(n\)
\end{examplle}

\begin{exercise}
\begin{enumerate}
\item \(\fa\subseteq(\fa:\fb)\)
\item \((\fa:\fb)\fb\subseteq\fa\)
\item \((\fa:\fb):\fc=(\fa:\fb\fc)=((\fa:\fc):\fb)\)
\item \((\bigcap_i\fa_i:\fb)=\bigcap_i(\fa_i:\fb)\)
\item \((\fa:\sum_i\fb_i)=\bigcap(\fa:\fb_i)\)
\end{enumerate}
\end{exercise}

\begin{proof}
\begin{enumerate}
\setcounter{enumi}{2}
\item \((\fa:\fb):\fc=\{x\in A:x\fc\subseteq\fa:\fb\}\). for any \(c\in\fc\), \(xc\fb\subseteq\fa\). Hence \(x\fc\fb\subseteq\fa\).
\setcounter{enumi}{4}
\item \((\fa:\sum_i\fb_i)=\{x\in A:x\sum_i\fb_i\subseteq\fa\}\)
\end{enumerate}
\end{proof}

If \(\fa\) is any ideal of \(A\), the \textbf{radical} of \(\fa\) is
\begin{equation*}
r(\fa)=\{x\in A:x^n\in\fa\text{ for some }n>0\}
\end{equation*}
if \(\phi:A\to A/\fa\) is the standard homomorphism, then \(r(\fa)=\phi^{-1}(\fN_{A/\fa})\) and hence \(r(\fa)\)
is an ideal by \ref{1.7}

\begin{exercise}
\begin{enumerate}
\item \(r(\fa)\supseteq\fa\)
\item \(r(r(\fa))=r(\fa)\)
\item \(r(\fa\fb)=r(\fa\cap\fb)=r(\fa)\cap r(\fb)\)
\item \(r(\fa)=(1)\) iff \(\fa=(1)\).
\item \(r(\fa+\fb)=r(r(\fa)+r(\fb))\)
\item if \(\fp\) is prime, \(r(\fp^n)=\fp\) for all \(n>0\)
\end{enumerate}
\end{exercise}

\begin{proof}
\begin{enumerate}
\setcounter{enumi}{4}
\item \(x\in r(\fa+\fb)\) iff \(x^n\in\fa+\fb\). \(y\in r(r(\fa)+r(\fb))\) iff \(y^m=a+b\), where \(a^{n_a}\in\fa\)
and \(b^{n_b}\in\fb\).
Then \((y^m)^{n_a+n_b}=(a+b)^{n_a+n_b}\in\fa+\fb\)
\item \(x\in r(\fp^n)\) iff \(x^m\in\fp^n\), then \(x^m=p_1\cdots p_n\in\fp\)
\end{enumerate}
\end{proof}

\begin{proposition}[]
\label{1.14}
The radical of an ideal \(\fa\) is the intersection of the prime ideals which contain \(\fa\)
\end{proposition}

\begin{proof}
Apply \ref{1.8} to \(A/\fa\).

Nilradical of \(A/\fa\) is the radical of \(\fa\).
\end{proof}

More generally, we may define the radical \(r(E)\) of any \textbf{subset} \(E\) of \(A\) in the same way.
It is \textbf{not} an ideal in general. We have \(r(\bigcup_\alpha E_\alpha)=\bigcup r(E_\alpha)\) for any family of subsets \(E_\alpha\)
of \(A\)

\begin{proposition}[]
\(D=\) set of zero-divisors of \(A=\bigcup_{x\neq 0}r(\Ann(x))\)
\end{proposition}

\begin{proof}
\(D=r(D)=r(\bigcup_{x\neq 0}\Ann(x))=\bigcup_{x\neq 0}r(\Ann(x))\)
\end{proof}

\begin{examplle}[]
If \(A=\Z\), \(\fa=(m)\), let \(p_i\) (\(1\le i\le r\)) be the distinct prime divisors of \(m\).
Then \(r(\fa)=(p_1\cdots p_r)=\bigcap_{i=1}^n(p_i)\)
\end{examplle}

\begin{proposition}[]
Let \(\fa\), \(\fb\) be ideals in a ring \(A\) s.t. \(r(\fa)\), \(r(\fb)\) are coprime. Then \(\fa\)
and \(\fb\) are coprime.
\end{proposition}

\begin{proof}
\(r(\fa+\fb)=r(r(\fa)+r(\fb))=r(1)=(1)\), hence \(\fa+\fb=(1)\)
\end{proof}

Let \(f:A\to B\) be a ring homomorphism. If \(\fa\) is an ideal in \(A\), the set \(f(\fa)\) is not
necessarily an ideal in \(B\) (e.g. \(\Z\to\Q\)). We define the \textbf{extension} \(\fa^e\) of \(\fa\) to be the
ideal \(Bf(\fa)\) generated by \(f(\fa)\) in \(B\): explicitly, \(\fa^e\) is the set of all
sums \(\sum y_if(x_i)\) where \(x_i\in\fa\), \(y_i\in B\)

If \(\fb\) is an ideal of \(B\), then \(f^{-1}(\fb)\) is always an ideal of \(A\), called the
\textbf{contraction} \(\fb^c\) of \(\fb\). If \(\fb\) is prime, then \(\fb^c\) is prime. If \(\fa\) is prime, \(\fa^e\)
need not be prime (\(f:\Z\to\Q\),\(\fa\neq 0\), then \(\fa^e=\Q\), which is not a prime ideal)

We can factorize \(f\) as follows:
\begin{equation*}
f\xrightarrow{p}f(A)\xrightarrow{j}B
\end{equation*}
where \(p\) is surjective and \(j\) is injective

\begin{examplle}[]
Consider \(\Z\to\Z[i]\), where \(i=\sqrt{-1}\). A prime ideal \((p)\) of \(\Z\) may or may not stay
prime when extended to \(\Z[i]\). In fact \(\Z[i]\) is a principal ideal domain (because it has a
Euclidean algorithm, i.e., a Euclidean ring) and the situation is as follows:
\begin{enumerate}
\item \((2^e)=((1+i)^2)\), the \textbf{square} of a prime ideal in \(\Z[i]\)
\item if \(p\equiv 1\mod 4\) then \((p)^e\) is the product of two distinct prime ideals
(for example, \((5)^e=(2+i)(2-i)\))
\item if \(p\equiv 3\mod 4\) then \((p)^e\) is prime in \(\Z[i]\)
\end{enumerate}
\end{examplle}

Let \(f:A\to B\), \(\fa\) and \(\fb\) be as before. Then
\begin{proposition}[]
\begin{enumerate}
\item \(\fa\subseteq\fa^{ec}\), \(\fb\supseteq\fb^{ce}\)
\item \(\fb^c=\fb^{cec}\), \(\fa^e=\fa^{ece}\)
\item If \(C\) is the set of contracted ideals in \(A\) and if \(E\) is the set of extended ideals
in \(B\), then \(C=\{\fa\mid\fa^{ec}=\fa\}\), \(E=\{\fb\mid\fb^{ce}=\fb\}\), and \(\fa\mapsto\fa^e\) is a bijective map
of \(C\) onto \(E\), whose inverse is \(\fb\mapsto\fb^c\).
\end{enumerate}
\end{proposition}

\begin{proof}
\begin{enumerate}
\setcounter{enumi}{2}
\item If \(\fa\in C\), then \(\fa=\fb^c=\fb^{cec}=\fa^{ec}\); conversely if \(\fa=\fa^{ec}\) then \(\fa\) is the
contraction of \(\fa^e\).
\end{enumerate}
\end{proof}

\begin{proof}
\begin{enumerate}
\item 
\end{enumerate}
\end{proof}

\begin{exercise}
If \(\fa_1,\fa_2\) are ideals of \(A\) and if \(\fb_1,\fb_2\) are ideals of \(B\), then
\begin{alignat*}{2}
&(\fa_1+\fa_2)^e=\fa_1^e+\fa_2^e\quad&&(\fb_1+\fb_2)^c\supseteq\fb_1^c+\fb_2^c\\
\end{alignat*}
\end{exercise}
\subsection{Exercise}
\label{sec:orgf2f7d8d}
\begin{exercise}
\label{ex1.1}
Let \(x\) be a nilpotent element of a ring \(A\). Show that \(1+x\) is a unit of \(A\). Deduce
that the sum of a nilpotent element and a unit is a unit
\end{exercise}

\begin{proof}
\(x\) is a element of a nilradical, which is the intersection all prime ideals. Since every
maximal ideal is a prime ideal, then nilradical is a subset of Jacobson radical.
Then \(1-(-u^{-1})x\) is a unit for some unit \(u\), hence \(u+x\) is a unit
\end{proof}

\begin{exercise}
\label{ex1.2}
Let \(A\) be a ring and let \(A[x]\) be the ring of polynomials in an indeterminate \(x\), with
coefficients in \(A\). Let \(f=a_0+a_1x+\cdots+a_nx^n\in A[x]\). Prove that
\begin{enumerate}
\item \(f\) is a unit in \(A[x]\) iff \(a_0\) is a unit in \(A\) and \(a_1,\dots,a_n\) are nilpotent [if
\(b_0+b_1x+\dots+b_mx^m\) is the inverse of \(f\), prove by induction on \(r\)
that \(a_n^{r+1}b_{m-r}=0\). Hence show that \(a_n\) is nilpotent and then use Exercise \ref{ex1.1}]
\item \(f\) is nilpotent iff \(a_0,\dots,a_n\) is nilpotent
\item \(f\) is a zero-divisor iff there exists \(a\neq 0\) in \(A\) s.t. \(af=0\)
\item \(f\) is said to be \textbf{primitive} if \((a_0,\dots,a_n)=(1)\). Prove that if \(f,g\in A[x]\),
then \(fg\) is primitive iff \(f\) and \(g\) are primitive
\end{enumerate}
\end{exercise}

\begin{proof}
\begin{enumerate}
\item Suppose \(g=\sum_{i=0}^mb_ix^i\) s.t. \(fg=1\). For \(r=0\), \(a_nb_m=0\) obviously.

Now suppose this is true for all \(p<r\). Now we prove \(a_n^{r+1}b_{m-r}=0\). The \(m+n-r\)th term's
coefficient is \(\sum_{i=0}^ra_{n-i}b_{m-r+i}=0\). Then
\begin{equation*}
a_n^{r+1}\sum_{i=0}^ra_{n-i}b_{m-r+i}=a_n^{r+1}b_{m-r}=0
\end{equation*}
Thus \(a_n^{m+1}b_0=0\) and hence \(a_n^{m+1}=0\) as \(b_0\) is a unit. So \(f-a_nx^n\) is a
unit and we can continue.

\item \(\Rightarrow\). Goal: for any prime ideal \(\fp\) in \(A\), \(f\) is 0 in \((A/\fp)[x]\). This is
because \(f^n\) is 0 in \((A/\fp)[x]\) and \(A/\fp\) is an integral domain. Then for \(a_0,\dots,a_n\)
is contained in every prime ideal and hence are nilpotent

If \(f\) is nilpotent and \(a_k\) is nilpotent, then \(f-a_kx^k\) is still nilpotent since
nilradical is an ideal

\(\Leftrightarrow\). Nilradical \(\fR\) is an ideal. As \(a_0,\dots,a_n\) is nilpotent in \(A[x]\),
their \(A[x]\)-combination is still nilpotent

\item Choose a polynomial \(g=b_0+b_1x+\dots+b_mx^m\) of least degree \(m\) s.t. \(fg=0\).
Then \(a_nb_m=0\) and \(a_ngf=0\). As \(g\) is of least degree, we have \(a_ng=0\). Then
\(fg=a_0g+\dots+a_{n-1}x^{n-1}g+a_ng=a_0g+\dots+a_{n-1}x^{n-1}g=0\). Hence for
all \(0\le i\le n\), \(a_ig=0\). Arbitrary coefficient of \(g\) is what we want

\item If \(fg\) is primitive, then \((\sum^{\min\{n,k\}}_{\max\{0,k-m\}}a_ib_{k-i})_{k\in[0,n+m]}=(1)\).
Change the coefficient one by one

By extract, we can get \((a_0^kb_k)_{k\in[0,n+m]}=(1)\). Then \((b_k)=(1)\).
\end{enumerate}
\end{proof}

\begin{exercise}
\label{ex1.4}
In the ring \(A[x]\), the Jacobson radical is equal to the nilradical
\end{exercise}

\begin{proof}
Suppose \(\fR\) is the Jacobson radical and \(f\in\fR\), then \(1-fx\)is a unit by Proposition
\ref{1.9}. By Exercise \ref{ex1.2} (1) all coefficients of \(f\) are nilpotent, then \(f\) is
nilpotent by Exercise \ref{ex1.2} (2)
\end{proof}

\begin{exercise}
\label{1.5}
Let \(A\) be the ring and let \(A[[x]]\) be the ring of formal power series \(f=\sum_{n=0}^\infty a_nx^n\)
with coefficients in \(A\). Show that
\begin{enumerate}
\item \(f\) is a unit in \(A[[x]]\) iff \(a_0\) is a unit in \(A\)
\item If \(f\) is nilpotent, then \(a_n\) is nilpotent for all \(n\ge 0\).
\item \(f\) belongs to the Jacobson radical of \(A[[x]]\) iff \(a_0\) belongs to the Jacobson radical
of \(A\)
\item The contraction of a maximal ideal \(\fm\) of \(A[[x]]\) is a maximal ideal of \(A\), and \(\fm\) is
generated by \(\fm^c\) and \(x\).
\item Every prime ideal of \(A\) is the contraction of a prime ideal of \(A[[x]]\).
\end{enumerate}
\end{exercise}

\begin{proof}
\begin{enumerate}
\item \(\Leftarrow\). We compute \(b_n\) from \(a_0,\dots,a_n\), \(b_0,\dots,b_{n-1}\)  and \(\sum_{i=0}^na_ib_{n-i}=0\).
Multiply it with \(a_0\), we get \(b_n+a_0\sum_{i=1}^na_ib_{n-i}=0\)
\item Note that nilradical is an ideal. If \(a_k\) is nilpotent in \(A\), then \(a_kx\) is
nilpotent in \(A[[x]]\), and \(f-a_kx^k\) is nilpotent. And we continue
\item For any \(b\in A\), \(1-bf\) is a unit, and by (1), \(1-ba_0\) is a unit.
\item From (3), a maximal ideal \(\fm\) at least contains \(xA[[x]]\). Let \(\fm=\fm^c+xA[[x]]\).
Now
\begin{equation*}
A[[x]]/\fm\cong(A[[x]]/xA[[x]])/(\fm/xA[[x]])\cong A/\fm^c
\end{equation*}
Thus \(\fm\) is maximal
\item Given a prime ideal \(\fp\) of \(A\), consider
\begin{equation*}
\phi:A[[x]]\to A\to A/\fp
\end{equation*}
Then \(\ker\phi=\fp+xA[[x]]\) and \(A[[x]]/\ker\phi\cong A/\fp\) and hence \(\ker\phi\) is a prime ideal.
\end{enumerate}
\end{proof}

\begin{exercise}
\label{ex1.6}
A ring \(A\) is s.t. every ideal not contained in the nilradical contains a nonzero idempotent
(that is, an element \(e\) s.t. \(e^2=e\neq 0\)). Prove that the nilradical and Jacobson radical
of \(A\) are equal
\end{exercise}

\begin{proof}
If there is a \(x\in A\) s.t. \(x\in\fR\) and \(x\notin\fN\). Then \((x)\not\subseteq\fN\) and there is \(y\in A\)
s.t. \(y^2x^2=x^2\) and hence \((y^2-1)x^2=0\). As \(x^2\neq 0\), \(y^2=1\). Hence \(\fR=(1)\), which is not possible
\end{proof}

\begin{exercise}
\label{ex1.7}
Let \(A\) be a ring where every element \(x\) satisfies \(x^n=x\) for some \(n>1\) (depending
on \(x\)). Show that every prime ideal in \(A\) is maximal
\end{exercise}

\begin{proof}
\(\fp\) the prime ideal and \(x\notin\fp\), as \(x(x^{n-1}-1)=0\in\fp\), \(x^{n-1}-1\in\fp\).
Then \(x^{n-1}\equiv 1\mod\fp\) and \((x+\fp)(x^{n-2}+\fp)=1+\fp\).
\end{proof}

\begin{exercise}
\label{ex1.8}
Let \(A\) be a ring \(\neq 0\). Show that the set of prime ideals of \(A\) has minimal elements
w.r.t. inclusion
\end{exercise}

\begin{proof}
Equivalently to say that nilradical is prime.
\end{proof}

\begin{exercise}
\label{ex1.9}
Let \(\fa\) be an ideal \(\neq(1)\) in a ring \(A\). Show that \(\fa=r(\fa)\) iff \(\fa\) is an
intersection of prime ideals
\end{exercise}

\begin{proof}
\(\Rightarrow\). From Proposition \ref{1.14}

\(\Leftarrow\). If \(x^n\in\fa\), then \(x\in\fa\).
\end{proof}

\begin{exercise}
\label{ex1.10}
Let \(A\) be a ring, \(\fN\) its nilradical. Show that the following are equivalent
\begin{enumerate}
\item \(A\) has exactly one prime ideal
\item every element of \(A\) is either a unit or nilpotent
\item \(A/\fN\) is a field
\end{enumerate}
\end{exercise}

\begin{proof}
\(2\to 3\). \(\fN\) is maximal

\(1\to 2\). Obvious:D

\(3\to 1\). Then \(\fN\) is maximal
\end{proof}

\begin{exercise}
\label{ex1.11}
A ring is \textbf{Boolean} if \(x^2=x\) for all \(x\in A\). In a Boolean ring \(A\), show that
\begin{enumerate}
\item \(2x=0\) for all \(x\in A\)
\item every prime ideal \(\fp\) is maximal, and \(A/\fp\) is a field with two elements
\item every finitely generated ideal in \(A\) is principal
\end{enumerate}
\end{exercise}

\begin{proof}
\begin{enumerate}
\item \(2x=x+x^2=0\)
\item Maximality by Exercise \ref{ex1.7}. For any \(x\notin\fp\), \((x+\fp)(1+\fp)=1+\fp\) and so \(x\equiv 1\mod\fp\).
For any \(x\in\fp\), \(x\equiv 0\mod\fp\).
\item Let \(x,y\) be elements of an ideal \(\fa\). Define \(z:=x+y+xy\), note that \(xz=x+y+y=x\).
Hence \((x,y)=(z)\)
\end{enumerate}
\end{proof}

\begin{exercise}
A local ring contains no idempotent \(\neq 0,1\)
\end{exercise}

\begin{proof}
If \(\fm\) is the unique maximal ring. Then \(x\in\fm\) iff for all \(y\in A\), \(1-xy\) is a unit.

If \(x^2=x\), then \(x(1-x)=0\). As \(1-x\) is not a unit, \(x\notin\fm\).
\end{proof}

\emph{Construction of an algebraic closure of a field}
\begin{exercise}
\label{ex1.13}
Let \(K\) be a field and let \(\Sigma\) be the set of all irreducible monic polynomials \(f\) in one
indeterminate with coefficients in \(K\). Let \(A\) be the polynomial ring over \(K\) generated
by indeterminate \(x_f\), one for each \(f\in\Sigma\). Let \(\fa\) be the ideal of \(A\) generated by the
polynomials \(f(x_f)\) for all \(f\in\Sigma\). Show that \(\fa\neq(1)\)

Let \(\fm\) be a maximal ideal of \(A\) containing \(\fa\), and let \(K_1=A/\fm\). Then \(K_1\) is an
extension field of \(K\) in which each \(f\in\Sigma\) has a root. Repeat the construction with \(K_1\)
in place of \(K\), obtaining a field \(K_2\), and so on. Let \(L=\bigcup_{n=1}^\infty K_n\). Then \(L\) is a
field in which each \(f\in\Sigma\) splits completely into linear factors. Let \(\barK\) be the set of
all elements of \(L\) which are algebraic over \(K\). Then \(\barK\) is an algebraic closure of \(K\).
\end{exercise}

\begin{proof}
Irreducible polynomials have degree greater than 1. There is no linear combination that the
degree of the sum is 0

Let \(K_0=K\) be a field. Given a non-negative integer \(n\) for which the field, \(K_n\), is
defined, let \(\Sigma_n\) be the set of monic irreducible elements of \(K_n[x]\) and let \(A_n\) be
the polynomial ring over \(K_n\) generated by the set of indeterminates \(\{x_f\mid f\in\Sigma\}\).
Define \(\fa_n\) be the ideal of \(A_n\) generated by the set \(\{f(x_f)\in A\mid f(\Sigma_n)\}\).
Since \(K_n\) is a field, \(A_n\) is a domain. Thus every element of \(\fa_n\) has positive degree
and \(\fa_n\) doesn't contain 1. Let \(\fm_n\) be a maximal ideal of \(A_n\) containing \(\fa_n\) and
define \(K_{n+1}=A_n/\fm_n\). The map
\begin{equation*}
K_n\to A_n\to A_n/\fm_n=K_{n+1}
\end{equation*}
given by the inclusion and quotient maps, is a field homomorphism. Thus it is injective and we
may identify \(K_n\) with a subfield of \(K_{n+1}\). Note that for any \(0\neq k\in K_n\), \(k\notin\fm\).
Thus the kernel of the map is only \(\{0\}\).

Let \(\barK=\bigcup_{n\ge 0}K_n\). If \(x,y\in\barK\) , then they are contained in some
subfields \(K_n,K_m\). Letting \(k=\max\{m,n\}\), \(x,y\in K_k\). Therefore the sum, difference, and
product of \(x,y\) are in \(K_k\).  Any field arithmetic of \(\barK\) can be performed in a
subfield, \(\barK\) is a field.

Let \(f\) be an irreducible monic polynomial in \(\barK[x]\). Since \(f\) has only finitely many
coefficients, there is some \(n\) s.t. \(f\) is an irreducible monic polynomial in \(K_n[x]\).
By construction, \(f\) has a root in \(K_{n+1}\), hence in \(\barK\). By the Euclidean
division, \(f\) must have degree 1. Therefore, \(\barK\) is algebraic closed.

By construction, the field extension \(K_{n+1}/K_n\) is algebraic for every \(n\).
\end{proof}

\begin{exercise}
\label{ex1.14}
In a ring \(A\), let \(\Sigma\) be the set of all ideals in which every element is a zero-divisor. Show
that the set \(\Sigma\) has minimal elements and that every maximal element of \(\Sigma\) is a prime ideal. Hence
the set of zero-divisors in \(A\) is a union of prime ideals
\end{exercise}

\begin{proof}
If \(x\) is a zero-divisor, then \(Ax\) is a set of zero-divisors. Thus \(\Sigma\) is not empty and has a
minimal element w.r.t. inclusion.

For a maximal ideal \(\fp\) in \(\Sigma\), suppose \(x,y\notin\fp\), then \(\fp+(x)+(y)\notin\Sigma\). Then there is an
element \(p+x'x+y'y\) that is not a zero-divisor. If \(xy\) is zero-divisor,
then \((p'xy)(p+x'x+y'y)=0\), a contradiction
\end{proof}

\emph{The prime spectrum of a ring}

\begin{exercise}
\label{ex1.15}
Let \(A\) be a ring and let \(X\) be the set of all prime ideals of \(A\). For each subset \(E\)
of \(A\), let \(V(E)\) denote the set of all prime ideals of \(A\) which contain \(E\). Prove
that
\begin{enumerate}
\item if \(\fa\) is the ideal generated by \(E\), then \(V(E)=V(\fa)=V(r(\fa))\)
\item \(V(0)=X\), \(V(1)=\emptyset\)
\item if \((E_i)_{i\in I}\) is any family of subsets of \(A\), then
\begin{equation*}
V\left( \bigcup_{i\in I}E_i \right)=\bigcap_{i\in I}V(E_i)
\end{equation*}
\item \(V(\fa\cap\fb)=V(\fa\fb)=V(\fa)\cup V(\fb)\) for any ideals \(\fa,\fb\) of \(A\)
\end{enumerate}


These results show that the sets \(V(E)\) satisfy the axioms for closed sets in a topological
space. The resulting topology is called the \textbf{Zariski topology}. The topological space \(X\) is
called the \textbf{prime spectrum} of \(A\), and is written as \(\Spec(A)\)
\end{exercise}

\begin{proof}
\begin{enumerate}
\item If \(\fa=(E)\), then \(\fa\) is the minimal ideal containing \(E\). Hence \(V(E)=V(\fa)\). For
any prime ideal \(\fp\) containing \(\fa\) and any \(a\in r(\fa)\). Then \(a^n\in\fa\) for some \(n\).
Then \(a^n\in\fp\), implying \(a\in\fp\). Hence \(V(\fa)\subseteq V(r(\fa))\).
\item Obvious
\item trivial
\item As \(\fa\fb\subseteq\fa\cap\fb\), if \(\fa\cap\fb\subseteq\fp\) then \(\fa\fb\subseteq\fp\). On the other hand, if \(\fa\fb\subseteq\fp\), then we have shown
either \(\fa\subseteq\fp\) or \(\fb\subseteq\fp\) (Proposition \ref{1.11}). Thus \(\fa\cap\fb\subseteq\fp\)
\end{enumerate}
\end{proof}

\begin{exercise}
\label{ex1.16}
Draw pictures of \(\Spec(\Z)\), \(\Spec(\R)\), \(\Spec(\C[x])\), \(\Spec(\R[x])\), \(\Spec(\Z[x])\)
\end{exercise}

\begin{proof}
\(\Z\) is PID, for any \(E\subseteq\Z\), let \(n=\min\{m\in E\mid m> 1\}\). Let \(\fa=(n)\). Then \((E)=\fa\).
Suppose \(n=p_1^{n_1}\dots p_r^{n_r}\), then \(V(E)=\{p_1\Z,\dots,p_r\Z\}\).

\(\R\) is a field and so there is only trivial ideals.

\(\C[x]\) is a PID. Prime ideals are of the form \((f)\), where \(f\) is a monic irreducible
or \(f=0\). As irreducible elements of \(\C[x]\) is of the form \(x-a\). Thus \(\Spec\C[x]\) is
actually the complex plane.

For any ideal \(\fa\) of \(\C[x]\), \(\fa=(f)\). By the Fundamental Theorem of
Algebra, \(f=\prod_{i=1}^k(x-a_i)^{\alpha_i}\) for some complex numbers \(a_1,\dots,a_k\) and positive
integers \(\alpha_1,\dots,\alpha_k\). Define \(\sqrt{f}\) as \(\prod_{i=1}^k(x-a_i)\). Since non-zero prime ideals
of \(\C[x]\) are maximal, we have
\begin{equation*}
V(\fa)=V(f)=V(\sqrt{f})=\bigcup_{i=1}^kV(x-a_i)=\{(x-a_1),\dots,(x-a_k)\}
\end{equation*}
Therefore non-empty open subsets of \(\Spec\C[x]\) are cofinite sets containing \(\{0\}\)

\label{Problem1}
\end{proof}

\begin{exercise}
\label{ex1.17}
For each \(f\in A\), let \(X_f\) denote the complement of \(V(f)\) in \(X=\Spec(A)\). The
sets \(X_f\) are open. Show that they form a basis of open sets for the Zariski topology, and
that
\begin{enumerate}
\item \(X_f\cap X_g=X_{fg}\)
\item \(X_f=\emptyset\) iff \(f\) is nilpotent
\item \(X_f=X\) iff \(f\) is a unit
\item \(X_f=X_g\) iff \(r((f))=r((g))\)
\item \(X\) is quasi-compact (that is, every open covering of \(X\) has a finite sub-covering)
\item More generally, each \(X_f\) is quasi-compact
\item An open subset of \(X\) is quasi-compact iff it is a finite union of sets \(X_f\)

The sets \(X_f\) are called \textbf{basic open sets} of \(X=\Spec(A)\)
\end{enumerate}
\end{exercise}

\begin{proof}
For any \(\fp\in X\), let \(x\in A\setminus\fp\). Then \(\fp\notin V(x)\). Hence \(\fp\in X_x\)

If \(\fp\in X_f\cap X_g\), then as \(V(f)\cup V(g)=V(fg)\), then \(\fp\in X_{fg}\). Hence this form a basis of
open sets for the Zariski topology

\begin{enumerate}
\item \(X_f\cap X_g=V(f)^c\cap V(g)^c=(V(f)\cup V(g))^c=(V(fg))^c=X_{fg}\)
\item \(X_f=\emptyset\) iff \(V(f)=X\) iff \(f\in\fN\)
\item \(X_f=X\) iff \(V(f)=\emptyset\). Note that any ideal can be extended to a maximal ideal which is
prime, thus \(f\) is not contained in any ideal, which means \(f\) is a unit
\item \(r((f))\subseteq r((g))\) iff every ideal containing \((g)\) contains \((f)\) iff \(V(f)\subseteq V(g)\).
\item A collection \(\calc\) of closed sets has finite intersection property iff for any
finite \(V(E_1),\dots,V(E_n)\in\calc\), \(\bigcap V(E_i)=V(\bigcup E_i)\neq\emptyset\) iff for any
finite \(V(E_1),\dots,V(E_n)\in\calc\), \(\bigcup E_i\) doesn't contain a unit. Thus \(\bigcup_{\calc}V(E_i)\) doesn't
contain a unit and hence \(\bigcap_{\calc} V(E_i)\neq\emptyset\)

Let \(\{X_f\}_{f\in E}\) be an open cover of \(X\). Taking complements shows that \(V(E)\) is
empty. Therefore \((E)=(1)\). This in turn implies that there are \(f_1,\dots,f_n\in E\)
and \(a_1,\dots,a_n\in A\) s.t. \(1=\sum_{i=1}^na_if_i\). Thus \(V(f_1,\dots,f_n)\) is empty
\item Suppose an open covering \(\{X_g\}_{g\in E}\) of \(X_f\),
then \(\bigcap_{g\in E}V(g)=V(\bigcup_{g\in E}g)=V(E)\subseteq V(f)\), which means that every prime containing \(E\)
contains \(f\), then \(f\in r((E))\) (Proposition \ref{1.14}). So there
are \(g_1,\dots,g_n\in E\), \(a_1,\dots,a_n\in A\) and a positive integer \(m\) s.t. \(f^m=\sum_{i=1}^na_ig_i\).
Thus \(V(f)\supseteq V(g_1,\dots,g_n)\). Hence \(X_f\subseteq\bigcup_{i=1}^n X_{g_i}\)
\item For any quasi-compact open sets \(U\) of \(X\), \(U=\bigcup_{f\in E}X_f\). And as it's quasi-compact,
there is \(E_0\subseteq_fE\) s.t. \(U=\bigcup_{f\in E_0}X_f\)
\end{enumerate}
\end{proof}

\begin{exercise}
\label{ex1.18}
It is sometimes convenient to denote a prime ideal of \(A\) by a letter such as \(x\) or \(y\)
when thinking of it as a point of \(X=\Spec(A)\). When thinking of \(x\) as a prime ideal
of \(A\), we denote it by \(\fp_x\). Show that
\begin{enumerate}
\item the set \(\{x\}\) is closed (we say that \(x\) is a ``closed point'') in \(\Spec(A)\) iff \(\fp_x\)
is maximal
\item \(\bbar{\{x\}}=V(\fp_x)\)
\item \(y\in\bbar{\{x\}}\) iff \(\fp_x\subseteq\fp_y\)
\item \(X\) is a \(T_0\)-space (this means that if \(x,y\) are disjoint points of \(X\), then
either there is a neighborhood of \(x\) which does not contain \(y\), or else there is a
neighborhood of \(y\) which does not contain \(x\))
\end{enumerate}
\end{exercise}

\begin{proof}
\begin{enumerate}
\item \(\{x\}\) is closed iff there is \(E\subseteq A\) s.t. \(\{x\}=V(E)\) which means \(\fp_x\) cannot be
expanded anymore
\item \(y\in\bbar{\{x\}}\) iff \(\forall\) open \(U\ni y\), \(x\in U\) iff \(\forall E\; y\notin V(E)\), \(x\notin V(E)\) iff
\(\forall E\; x\in V(E)\Rightarrow y\in V(E)\). As \(x\in V(x)\), \(y\in V(x)\). If \(y\in V(x)\), for any \(x\in V(E)\),
we have \(y\in V(x)\subseteq V(E)\)
\item \(y\in\bbar{\{x\}}\) iff \(y\in V(x)\) iff \(x\subseteq y\)
\item If \(x\subseteq y\), then \(x\notin V(y)\) and \(y\in V(y)\). If \(x\not\subseteq y\), then \((x)\not\subseteq y\) and
so \(y\notin V(x)\).

If every neighborhood of \(x\) contains \(y\) and vice versa. Then \(y\in\bbar{\{x\}}\)
and \(x\in\bbar{\{y\}}\). So \(x=y\)
\end{enumerate}
\end{proof}

\begin{exercise}
\label{ex1.19}
A topological space \(X\) is said to be \textbf{irreducible} if \(X\neq\emptyset\) and if every pair of non-empty
open sets in \(X\) intersect, or equivalently if every non-empty open set is dense in \(X\).
Show that \(\Spec(A)\) is irreducible iff the nilradical of \(A\) is a prime ideal
\end{exercise}

\section{{\bfseries\sffamily TODO} Problems}
\label{sec:orgb0c7cf7}
\ref{Problem1}: need more field knowledge to deal with \(\R[x]\) and \(\Z[x]\)
\end{document}
