% Created 2021-10-09 Sat 17:40
% Intended LaTeX compiler: pdflatex
\documentclass[11pt]{article}
\usepackage[utf8]{inputenc}
\usepackage[T1]{fontenc}
\usepackage{graphicx}
\usepackage{grffile}
\usepackage{longtable}
\usepackage{wrapfig}
\usepackage{rotating}
\usepackage[normalem]{ulem}
\usepackage{amsmath}
\usepackage{textcomp}
\usepackage{amssymb}
\usepackage{capt-of}
\usepackage{hyperref}
\graphicspath{{../../books/}}
% TIPS
% \substack{a\\b} for multiple lines text





% pdfplots will load xolor automatically without option
\usepackage[dvipsnames]{xcolor}

\usepackage{forest}
% two-line text in node by [two \\ lines]
% \begin{forest} qtree, [..] \end{forest}
\forestset{
  qtree/.style={
    baseline,
    for tree={
      parent anchor=south,
      child anchor=north,
      align=center,
      inner sep=1pt,
    }}}
%\usepackage{flexisym}
% load order of mathtools and mathabx, otherwise conflict overbrace

\usepackage{mathtools}
%\usepackage{fourier}
\usepackage{pgfplots}
\usepackage{amsthm, mathabx,  amsmath, commath}
\usepackage{amsfonts}

\usepackage{empheq}
\usepackage{tikz}
\usetikzlibrary{arrows.meta}
\usepackage[most]{tcolorbox}

\newtheorem{theorem}{Theorem}[section]
\newtheorem{definition}{Definition}[section]
\newtheorem{corollary}{Corollary}[section]
\newtheorem{example}{Example}[section]
\newtheorem{lemma}{Lemma}[section]
\newtheorem{proposition}{Proposition}[section]

\newcommand{\bl}[1] {\boldsymbol{#1}}
\newcommand{\Wt}[1] {\stackrel{\sim}{\smash{#1}\rule{0pt}{1.1ex}}}
\newcommand{\wt}[1] {\widetilde{#1}}


%For boxed texts in align, use Aboxed{}
%otherwise use boxed{}

\DeclareMathSymbol{\widehatsym}{\mathord}{largesymbols}{"62}
\newcommand\lowerwidehatsym{%
  \text{\smash{\raisebox{-1.3ex}{%
    $\widehatsym$}}}}
\newcommand\fixwidehat[1]{%
  \mathchoice
    {\accentset{\displaystyle\lowerwidehatsym}{#1}}
    {\accentset{\textstyle\lowerwidehatsym}{#1}}
    {\accentset{\scriptstyle\lowerwidehatsym}{#1}}
    {\accentset{\scriptscriptstyle\lowerwidehatsym}{#1}}
}

\usepackage{graphicx}
    
% text on arrow for xRightarrow
\makeatletter
%\newcommand{\xRightarrow}[2][]{\ext@arrow 0359\Rightarrowfill@{#1}{#2}}
\makeatother


\def \bx {\boldsymbol{x}}
\def \ba {\boldsymbol{a}}
\def \bI {\boldsymbol{I}}
\def \bt {\boldsymbol{t}}
\def \bb {\boldsymbol{b}}
\def \bA {\boldsymbol{A}}
\def \bX {\boldsymbol{X}}
\def \bu {\boldsymbol{u}}
\def \bS {\boldsymbol{S}}
\def \bZ {\boldsymbol{Z}}
\def \bz {\boldsymbol{z}}
\def \by {\boldsymbol{y}}
\def \bw {\boldsymbol{w}}
\def \bT {\boldsymbol{T}}
\def \bS {\boldsymbol{S}}
\def \bm {\boldsymbol{m}}
\def \bW {\boldsymbol{W}}
\def \bY {\boldsymbol{Y}}
\def \bH {\boldsymbol{H}}
\def \blambda {\boldsymbol{\lambda}}
\def \bPhi {\boldsymbol{\Phi}}
\def \btheta {\boldsymbol{\theta}}
\def \bmu {\boldsymbol{\mu}}
\def \bphi {\boldsymbol{\phi}}
\def \bSigma {\boldsymbol{\Sigma}}
\def \lb {\left\{}
\def \rb {\right\}}
\def \caln {\mathcal{N}}
\def \dissum {\displaystyle\Sigma}
\def \dispro {\displaystyle\prod}
\def \E {\mathbb{E}}
\def \Q {\mathbb{Q}}
\def \V {\mathbb{V}}
\def \R {\mathbb{R}}
\def \calq {\mathcal{Q}}
\def \calg {\mathcal{G}}
\def \caln {\mathcal{N}}
\def \calr {\mathcal{R}}
\def \calm {\mathcal{M}}
\def \calc {\mathcal{C}}
\def \bcup {\bigcup}

\makeindex
\author{M. F. Atiyah \& I. G. MacDonald}
\date{\today}
\title{Introduction to Commutative Algebra}
\hypersetup{
 pdfauthor={M. F. Atiyah \& I. G. MacDonald},
 pdftitle={Introduction to Commutative Algebra},
 pdfkeywords={},
 pdfsubject={},
 pdfcreator={Emacs 27.2 (Org mode 9.5)}, 
 pdflang={English}}
\begin{document}

\maketitle
\tableofcontents


\section{Rings and Ideals}
\label{sec:orgd80047a}
A \textbf{ring homomorphism} is a mapping \(f\) of a ring \(A\) into a ring \(B\) s.t.
\begin{enumerate}
\item \(f(x+y)=f(x)+f(y)\)
\item \(f(xy)=f(x)f(y)\)
\item \(f(1)=1\)
\end{enumerate}


An \textbf{ideal} \(\fa\) of a ring \(A\) is a subset of \(A\) which is an additive subgroup and is
s.t. \(A\fa\subseteq\fa\). The quotient group \(A/\fa\) inherits a uniquely defined multiplication from \(A\)
which makes it into a ring, called the \textbf{quotient ring} \(A/\fa\). The elements of \(A/\fa\) are the
cosets of \(\fa\) in \(A\), and the mapping \(\phi:A\to A/\fa\) which maps each \(x\in A\) to its
coset \(x+\fa\) is a surjective ring homomorphism

\begin{proposition}[]
\label{1.1}
There is a one-to-one order-preserving correspondence between the ideals \(\fb\) of \(A\) which
contain \(\fa\), and the ideals \(\bar{\fb}\) of \(A/\fa\), given by \(\fb=\phi^{-1}(\bar{\fb})\).
\end{proposition}

\begin{proof}
Let \(S_1=\{\fb:\fb\text{ an ideal of $A$ and }\fa\subseteq\fb\}\)
and \(S_2=\{\bar{\fb}:\bar{\fb}\text{ an ideal of }A/\fa\}\), \(\pi\) is the natural map \(\pi(S)=S/\fa\), we prove that
\begin{equation*}
\varphi:S_1\to S_2\hspace{1cm}\fb\mapsto\pi(\fb)
\end{equation*}
is an bijection.

First assume that \(\fa\subseteq\fb\), we prove that \(\pi^{-1}\pi(\fb)=\fb\). Apparently \(\fb\subseteq\pi^{-1}\pi(\fb)\). For
any \(b\in\pi^{-1}\pi(\fb)\), there is a \(s\in\fb\) s.t. \(\pi(b)=\pi(s)\). Thus \(b-s\in\ker\pi=\fa\). As \(\fa\subseteq\fb\),
we have \(b\in\fb\). Hence \(\pi^{-1}\pi(\fb)=\fb\).

Thus for any \(\fb_1,\fb_2\in S_1\) and \(\varphi(\fb_1)=\pi(\fb_1)=\pi(\fb_2)=\varphi(\fb_2)\), we have \(\pi^{-1}\pi(\fb_1)=\pi^{-1}\pi(\fb_2)\).
Thus \(\varphi\) is injective.

For any \(\bar{\fb}\in S_2\), \(\pi^{-1}(\bar{\fb})\) contains \(\fa=\pi^{-1}(\{0\})\). Hence \(\varphi\) is surjective

Order-preserving means \(\fa\subseteq\fb\subseteq\fc\) iff \(\bar{\fb}\subseteq\bar{\fc}\)
\end{proof}

If \(f:A\to B\) is any ring homomorphism, the \textbf{kernel} of \(f\) is an ideal \(\fa\) of \(A\), and the
image of \(f\) is a subring \(C\) of \(B\); and \(f\) induces a ring isomorphism \(A/\fa\cong C\)

We shall sometimes use the notation \(x\equiv y\mod\fa\); this means that \(x-y\in\fa\)

A \textbf{zero-divisor} in a ring \(A\) is an element \(x\) which divides 0, i.e., for which there
exists \(y\neq 0\) in \(A\) s.t. \(xy=0\). A ring with no zero-divisor \(\neq 0\) (and in
which \(1\neq 0\)) is called an \textbf{integral domain}.

An element \(x\in A\) is \textbf{nilpotent} if \(x^n=0\) for some \(n>0\). A nilpotent element is a
zero-divisor (unless \(A=0\))

A \textbf{unit} in \(A\) is an element \(x\) which ``divides 1'', i.e., an element \(x\) s.t. \(xy=1\) for
some \(y\in A\). The element \(y\) is then uniquely determined by \(x\), and is
written \(x^{-1}\). The units in \(A\) form a (multiplicative) abelian group

The multiples \(ax\) of an element \(x\in A\) from a \textbf{principal} ideal, denoted by \((x)\)
or \(Ax\). \(x\) is a unit iff \((x)=A=(1)\). The \textbf{zero} ideal \((0)\) is denoted by 0

A \textbf{field} is a ring \(A\) in which \(1\neq 0\) and every non-zero element is a unit. Every field is
an integral domain

\begin{proposition}[]
Let \(A\) be a ring \(\neq 0\). Then the following are equivalent:
\begin{enumerate}
\item \(A\) is a field
\item the only ideals in \(A\) are 0 and (1)
\item every homomorphism of \(A\) into a non-zero ring \(B\) is injective
\end{enumerate}
\end{proposition}

\begin{proof}
\(2\to 3\). Let \(\phi:A\to B\) be a ring homomorphism. Then \(\ker\phi\) is an ideal \(\neq(1)\) in \(A\),
hence \(\ker\phi=0\), hence \(\phi\) is injective

\(3\to 1\). Let \(x\) be an element of \(A\) which is not a unit. Then \((x)\neq(1)\),
hence \(B=A/(x)\) is not the zero ring. Let \(\phi:A\to B\) be the natural homomorphism of \(A\)
onto \(B\) with kernel \((x)\). By hypothesis, \(\phi\) is injective, hence \((x)=0\), hence \(x=0\)
\end{proof}

An ideal \(\fp\) in \(A\) is \textbf{prime} if \(\fp\neq(1)\) and if \(xy\in\fp\Rightarrow x\in\fp\) or \(y\in\fp\)

An ideal \(\fm\) in \(A\) is \textbf{maximal} if \(\fm\) in \(A\) is \textbf{maximal} if \(\fm\neq(1)\) and if no
ideal \(\fa\) s.t. \(\fm\subset\fa\subset(1)\) (\textbf{strict} inclusions). Equivalently
\begin{gather*}
\fp\text{ is prime } \Leftrightarrow A/\fp\text{ is an integral domain}\\
\fm\text{ is maximal } \Leftrightarrow A/\fm\text{ is a field}
\end{gather*}
\begin{proof}
If \(\fm\) is maximal and suppose \(a\notin A\). Then \(J=\{ra+i:i\in \fm\text{ and }r\in A\}\) is an ideal.
Hence \(J=A\). So there is \(r\in A, \fm\in I\) s.t. \(1=ra+i\). So we have \(1\equiv ra\mod \fm\). Hence we
find the inverse of \(a+\fm\)

If \(A/\fm\) is a field and suppose \(\fm\subset\fn\subset A\). Let \(a\in\fm\setminus\fn\), then there exists a \(b\in A\)
s.t. \(ab-1\in\fm\). So \(ab+m=1\) for some \(m\in\fm\). But \(ab\in\fn\) and \(m\in\fm\subset\fn\), then we
have \(1\in\fn\) and \(\fn=A\).
\end{proof}

Hence a maximal ideal is prime. The zero ideal is prime iff \(A\) is an integral domain

If \(f:A\to B\) is a ring homomorphism and \(\fq\) is a prime ideal in \(B\), then \(f^{-1}(\fq)\) is
a prime ideal in \(A\), for \(A/f^{-1}(\fq)\) is isomorphic to a subring of \(B/\fq\) and hence has
no zero-divisor \(\neq 0\). (\href{https://asgarli.wordpress.com/2013/04/21/inverse-image-of-a-prime-ideal-is-prime/}{Explanation}. Since \(\fq\) is prime, \(B/\fa\) is an integral domain and a
subring of an integral domain is still an integral domain. Define the map
\(\varphi(a+f^{-1}(\fq))=f(a)+\fq\) and we need to show its a homomorphism. Then we show its injective.)

But if \(\fn\) is a maximal ideal of \(B\) it is not necessarily true that \(f^{-1}(\fn)\) is
maximal in \(A\); all we can say for sure is that it is prime. (Example: \(A=\Z\), \(B=\Q\), \(\fn=0\)).

\begin{theorem}[]
\label{1.3}
Every ring \(A\neq 0\) has at least one maximal ideal
\end{theorem}

\begin{proof}
This is the standard application of Zorn's lemma. Let \(\Sigma\) be the set of all ideals \(\neq(1)\)
in \(A\). Order \(\Sigma\) by inclusion. \(\Sigma\) is not empty, since \(0\in\Sigma\). To apply Zorn's lemma we must
show that every chain in \(\Sigma\) has an upper bound in \(\Sigma\); let then \((\fa_\alpha)\) be a chain of ideals in
\(\Sigma\), so that for each pair of indices \(\alpha\), \(\beta\) we have either \(\fa_\alpha\subseteq \fa_\beta\) or \(\fa_\beta\subseteq\fa_\alpha\).
Let \(\fa=\bigcup_\alpha\fa_\alpha\). Then \(\fa\) is an ideal and \(1\notin\fa\). Hence \(\fa\in\Sigma\) and is an upper bound of the
chain. Hence \(\Sigma\) has a maximal element
\end{proof}

\begin{corollary}[]
If \(\fa\neq(1)\) is an ideal of \(A\), there exists a maximal ideal of \(A\) containing \(\fa\)
\end{corollary}

\begin{proof}
Apply \ref{1.3} to \(A/\fa\) and \ref{1.3}
\end{proof}

\begin{corollary}[]
\label{1.5}
Every non-unit of \(A\) is contained in a maximal ideal.
\end{corollary}

A ring \(A\) with exactly one maximal ideal \(\fm\) is called a \textbf{local ring}. The field \(k=A/\fm\) is
called the \textbf{residue field} of \(A\)

\begin{proposition}[]
\begin{enumerate}
\item Let \(A\) be a ring and \(\fm\neq(1)\) an ideal of \(A\) s.t. every \(x\in A-\fm\) is a unit in \(A\).
Then \(A\) is a local ring and \(\fm\) its maximal ideal.
\item Let \(A\) be a ring and \(\fm\) a maximal ideal of \(A\) s.t. every element of \(1+\fm\) is a
unit in \(A\). Then \(A\) is a local ring
\end{enumerate}
\end{proposition}

\begin{proof}
\begin{enumerate}
\setcounter{enumi}{1}
\item Let \(x\in A-\fm\). Since \(\fm\) is maximal, the ideal generated by \(x\) and \(\fm\)
is \((1)\), hence there exist \(y\in A\) and \(t\in\fm\) s.t. \(xy+t=1\); hence \(xy=1-t\) belongs
to \(1+\fm\) and therefore is a unit. Now use 1
\end{enumerate}
\end{proof}

A ring with only a finite number of maximal ideals is called \textbf{semi-local}

\begin{examplle}[]n
\begin{enumerate}
\item \(A=k[x_1,\dots,x_n]\), \(k\) a field. Let \(f\in A\) be an irreducible polynomial. By unique
factorization, the ideal \((f)\) is prime
\item \(A=\Z\). Every ideal in \(\Z\) is of the form \((m)\) for some \(m\ge 0\). The ideal \((m)\) is
prime iff \(m=0\) or a prime number. All the ideals \((p)\), where \(p\) is a prime number,
are maximal: \(\Z/(p)\) is the field of \(p\) elements
\item A \textbf{principal ideal domain} is an integral domain in which every ideal is principal. In such a
ring every non-zero prime ideal is maximal. For if \((x)\neq 0\) is a prime ideal
and \((y)\supset(x)\), we have \(x\in(y)\), say \(x=yz\), so that \(yz\in(x)\) and \(y\notin(x)\),
hence \(z\in(x)\); say \(z=tx\). Then \(x=yz=ytx\), so that \(yt=1\) and therefore \((y)=(1)\).
\end{enumerate}
\end{examplle}

\begin{proposition}[]
\label{1.7}
The set \(\fN\) of all nilpotent elements in a ring \(A\) is an ideal, and \(A/\fN\) has no
nilpotent \(\neq 0\)
\end{proposition}

\begin{proof}
If \(x\in\fN\), clearly \(ax\in\fN\) for all \(a\in A\). Let \(x,y\in\fN\): say \(x^m=0\), \(y^n=0\). By the
binomial theorem, \((x+y)^{n+m-1`}\) is a sum of integer multiples of products \(x^ry^s\),
where \(r+s=m+n-1\);

Let \(\barx\in A/\fN\) be represented by \(x\in A\). Then \(\barx^n\) is represented by \(x^n\), so
that \(\barx^n=0\Rightarrow x^n\in\fN\Rightarrow(x^n)^k=0\) for some \(k>0\Rightarrow x\in\fN\Rightarrow\barx=0\)
\end{proof}

The ideal \(\fN\) is called the \textbf{nilradical} of \(A\)

\begin{proposition}[]
\label{1.8}
The nilradical of \(A\) is the intersection of all the prime ideals of \(A\)
\end{proposition}

\begin{proof}
Let \(\fN'\) denote the intersection of all the prime ideals of \(A\). If \(f\in A\) is nilpotent
and if \(\fp\) is a prime ideal, then \(f^n=0\in\fp\) for some \(n>0\), hence \(f\in\fp\). Hence \(f\in\fN'\)

Conversely, suppose that \(f\) is not nilpotent. Let \(\Sigma\) be the set of ideals \(\fa\) with the
property
\begin{equation*}
n>0\Rightarrow f^n\notin\fa
\end{equation*}
Then \(\Sigma\) is not empty because \(0\in\Sigma\). Zorn's lemma can be applied to the set \(\Sigma\), ordered by
inclusion, and therefore \(\Sigma\) has a maximal element. We shall show that \(\fp\) is a prime ideal.
Let \(x,y\notin\fp\). Then the ideals \(\fp+(x)\), \(\fp+(y)\) strictly contain \(\fp\) and therefore do not
belong to \(\Sigma\); hence
\begin{equation*}
f^m\in\fp+(x),\quad f^n\in\fp+(y)
\end{equation*}
for some \(m,n\). It follows that \(f^{m+n}\in\fp+(xy)\), hence the ideal \(\fp+(xy)\) is not in
\(\Sigma\) and therefore \(xy\notin\fp\). Hence we have a prime ideal \(\fp\) s.t. \(f\notin\fp\), so that \(f\notin\fN'\)
\end{proof}

The \textbf{Jacobson radical} \(\fR\) of \(A\) is defined to be the intersection of all the maximal ideals
of \(A\). It can be characterized as follows:

\begin{proposition}[]
\(x\in\fR\) iff \(1-xy\) is a unit in \(A\) for all \(y\in A\)
\end{proposition}

\begin{proof}
\(\Rightarrow\): Suppose \(1-xy\) is not a unit. By \ref{1.5} it belongs to some maximal ideal \(\fm\);
but \(x\in\fR\subseteq\fm\), hence \(xy\in\fm\) and therefore \(1\in\fm\), which is absurd

\(\Leftarrow\): Suppose \(x\notin\fm\) for some maximal ideal \(\fm\). Then \(\fm\) and \(x\) generate the unit
ideal \((1)\), so that we have \(u+xy=1\) for some \(u\in\fm\) and some \(y\in A\). Hence \(1-xy\in\fm\)
and is therefore not a unit.
\end{proof}

If \(\fa,\fb\) are ideals in a ring \(A\), their \textbf{sum} \(\fa+\fb\) is the set of all \(x+y\) where \(x\in\fa\)
and \(y\in\fb\). It is the smallest ideal containing \(\fa\) and \(\fb\). More generally, we may define
the sum \(\sum_{i\in I}a_i\) of any family (possibly infinite) of ideals \(\fa_i\) of \(A\); is elements
are all sums \(\sum x_i\), where \(x_i\in\fa_i\) for all \(i\in I\) and almost all of the \(x_i\) (i.e., all
but a finite set) are zero. It is the smallest ideal of \(A\) which contains all the ideals \(\fa_i\)

The \textbf{product} of two ideals \(\fa,\fb\) in \(A\) is the ideal \(\fa\fb\) \textbf{generated} by all products \(xy\),
where \(x\in\fa\) and \(y\in\fb\). It is the set of all finite sums \(\sum x_iy_i\) where each \(x_i\in\fa\) and
each \(y_i\in\fb\)

We have the \textbf{distributive law}
\begin{equation*}
\fa(\fb+\fc)=\fa\fb+\fa\fc
\end{equation*}
In the ring \(\Z\), \(\cap\) and + are distributive over each other. This is not the case in
general. \textbf{modular law}
\begin{equation*}
\fa\cap(\fb+\fc)=\fa\cap\fb+\fa\cap\fb\text{ if }\fa\supseteq\fb\text{ or }\fa\supseteq\fc
\end{equation*}
\begin{equation*}
\fa\cap\fb=\fa\fb\text{ provided }\fa+\fb=(1)
\end{equation*}
If \(x\in\fa\cap\fb\), there is \(a+b=1\). Hence \(xa+xb=x\in\fa\fb\)

Two ideals \(\fa,\fb\) are said to be \textbf{coprime} if \(\fa+\fb=(1)\). Thus for coprime ideals we
have \(\fa\cap\fb=\fa\fb\).

Let \(A\) be a ring and \(\fa_1,\dots,\fa_n\) ideals of \(A\). Define a homomorphism
\begin{equation*}
\phi:A\to\prod_{i=1}^n(A/\fa_i)
\end{equation*}
by the rule \(\phi(x)=(x+\fa_1,\dots,x+\fa_n)\)

\begin{proposition}[]
\begin{enumerate}
\item If \(\fa_i,\fa_j\) are coprime whenever \(i\neq j\), then \(\prod\fa_i=\bigcap\fa_i\)
\item \(\phi\) is surjective iff \(\fa_i\), \(\fa_j\) are coprime whenever \(i\neq j\)
\item \(\phi\) is injective iff \(\bigcap\fa_i=(0)\)
\end{enumerate}
\end{proposition}

\begin{proof}
\begin{enumerate}
\item Induction on \(n\). The case \(n=2\) is dealt with above. Suppose \(n>2\) and the result true
for \(\fa_1,\dots,\fa_{n-1}\), and let \(\fb=\prod_{i=1}^{n-1}\fa_i=\bigcap_{i=1}^{n-1}\fa_i\). As we have \(x_i+y_i=1\)
(\(x_i\in\fa_i,y_i\in\fa_n\)) and therefore
\begin{equation*}
\prod_{i=1}^{n-1}x_i=\prod_{i=1}^{n-1}(1-y_i)\equiv 1\mod \fa_n
\end{equation*}
Hence \(\fa_n+\fb=(1)\) and so
\begin{equation*}
\prod_{i=1}^n\fa_i=\fb\fa_n=\fb\cap\fa_n=\bigcap_{i=1}^n\fa_i
\end{equation*}
\item \(\Rightarrow\): Let's show for example that \(\fa_1,\fa_2\) are coprime. There exists \(x\in A\)
s.t. \(\phi(x)=(1,0,\dots,0)\); hence \(x\equiv 1\mod\fa_1\) and \(x\equiv 0\mod\fa_2\), so that
\begin{equation*}
1=(1-x)+x\in\fa_1+\fa_2
\end{equation*}
\(\Leftarrow\): It is enough to show, for example, that there is an element \(x\in A\)
s.t. \(\phi(x)=(1,0,\dots,0)\). Since \(\fa_1+\fa_i=(1)\) (\(i>1\)) we have \(u_i+v_i=1\) (\(u_i\in\fa_1,v_i\in\fa_i\)).
Take \(x=\prod_{i=2}^nv_i\), then \(x=\prod(1-u_i)\equiv 1\mod\fa_1\). Hence \(\phi(x)=(1,0,\dots,0)\)
\item \(\bigcap\fa_i\) is the kernel of \(\phi\)
\end{enumerate}
\end{proof}

\begin{proposition}[]
\begin{enumerate}
\item Let \(\fp_1,\dots,\fp_n\) be prime ideals and let \(\fa\) be an ideal contained in \(\bigcup_{i=1}^n\fp_i\).
Then \(\fa\subseteq\fp_i\) for some \(i\).
\item Let \(\fa_1,\dots,\fa_n\) be ideals and let \(\fp\) be a prime ideal containing \(\bigcap_{i=1}^n\fa_i\).
Then \(\fp\supseteq\fa_i\) for some \(i\). If \(\fp=\bigcap\fa_i\), then \(\fp=\fa_i\) for some \(i\)
\end{enumerate}
\end{proposition}

\begin{proof}
\begin{enumerate}
\item induction on \(n\) in the form
\begin{equation*}
\fa\not\subseteq\fp_i(1\le i\le n)\Rightarrow\fa\not\subseteq\bigcup_{i=1}^n\fp_i
\end{equation*}
It is true for \(n=1\). If \(n>1\) and the result is true for \(n-1\), then for each \(i\)
there exists \(x_i\in\fa\) s.t. \(x_i\notin\fp_j\) whenever \(j\neq i\). If for some \(i\) we have \(x_i\notin\fp_i\),
we are through. If not, then \(x_i\in\fp_i\) for all \(i\). Consider the element
\begin{equation*}
y=\sum_{i=1}^nx_1x_2\cdots x_{i-1}x_{i+1}\cdots x_n
\end{equation*}
we have \(y\in\fa\) and \(y\notin\fp_i\) (\(1\le i\le n\)). Hence \(\fa\not\subseteq\bigcup_{i=1}^n\fp_i\)
\item Suppose \(\fp\not\supseteq\fa_i\) for all \(i\). Then there exist \(x_i\in\fa_i\), \(x_i\notin\fp\) (\(1\le i\le n\)) and
therefore \(\prod x_i\in\prod\fa_i\subseteq\bigcap\fa_i\); but \(\prod x_i\notin\fp\) since \(\fp\) is prime. Hence \(\fp\not\supseteq\bigcap\fa_i\)

If \(\fp=\bigcap\fa_i\), then \(\fp\subseteq\fa_i\) and hence \(\fp=\fa_i\) for some \(i\).
\end{enumerate}
\end{proof}

If \(\fa,\fb\) are ideals in a ring \(A\), their \textbf{ideal quotient} is
\begin{equation*}
(\fa:\fb)=\{x\in A:x\fb\subseteq\fa\}
\end{equation*}
which is an ideal. In particular, \((0:\fb)\) is called the \textbf{annihilator} of \(\fb\) and is also
denoted by \(\Ann(\fb)\): it is the set of all \(x\in A\) s.t. \(x\fb=0\). In this notation the set of
all zero-divisors in \(A\) is
\begin{equation*}
D=\bigcup_{x\neq 0}\Ann(x)
\end{equation*}

If \(\fb\) is a principal ideal \((x)\), we shall write \((\fa:x)\) in place of \((\fa:(x))\)

\begin{examplle}[]
If \(A=\Z\), \(\fa=(m)\), \(\fb=(n)\), where say \(m=\prod_pp^{\mu_p}\), \(n=\prod_pp^{\nu_p}\),
then \((\fa:\fb)=(q)\) where \(q=\prod_pp^{\gamma_p}\) and
\begin{equation*}
\gamma_p=\max(\mu_p-\nu_p,0)=\mu_p-\min(\mu_p,\nu_p)
\end{equation*}
Hence \(q=m/(m,n)\), where \((m,n)\) is the h.c.f. of \(m\) and \(n\)
\end{examplle}

\begin{exercise}
\begin{enumerate}
\item \(\fa\subseteq(\fa:\fb)\)
\item \((\fa:\fb)\fb\subseteq\fa\)
\item \((\fa:\fb):\fc=(\fa:\fb\fc)=((\fa:\fc):\fb)\)
\item \((\bigcap_i\fa_i:\fb)=\bigcap_i(\fa_i:\fb)\)
\item \((\fa:\sum_i\fb_i)=\bigcap(\fa:\fb_i)\)
\end{enumerate}
\end{exercise}

\begin{proof}
\begin{enumerate}
\setcounter{enumi}{2}
\item \((\fa:\fb):\fc=\{x\in A:x\fc\subseteq\fa:\fb\}\). for any \(c\in\fc\), \(xc\fb\subseteq\fa\). Hence \(x\fc\fb\subseteq\fa\).
\setcounter{enumi}{4}
\item \((\fa:\sum_i\fb_i)=\{x\in A:x\sum_i\fb_i\subseteq\fa\}\)
\end{enumerate}
\end{proof}

If \(\fa\) is any ideal of \(A\), the \textbf{radical} of \(\fa\) is
\begin{equation*}
r(\fa)=\{x\in A:x^n\in\fa\text{ for some }n>0\}
\end{equation*}
if \(\phi:A\to A/\fa\) is the standard homomorphism, then \(r(\fa)=\phi^{-1}(\fN_{A/\fa})\) and hence \(r(\fa)\)
is an ideal by \ref{1.7}

\begin{exercise}
\begin{enumerate}
\item \(r(\fa)\supseteq\fa\)
\item \(r(r(\fa))=r(\fa)\)
\item \(r(\fa\fb)=r(\fa\cap\fb)=r(\fa)\cap r(\fb)\)
\item \(r(\fa)=(1)\) iff \(\fa=(1)\).
\item \(r(\fa+\fb)=r(r(\fa)+r(\fb))\)
\item if \(\fp\) is prime, \(r(\fp^n)=\fp\) for all \(n>0\)
\end{enumerate}
\end{exercise}

\begin{proof}
\begin{enumerate}
\setcounter{enumi}{4}
\item \(x\in r(\fa+\fb)\) iff \(x^n\in\fa+\fb\). \(y\in r(r(\fa)+r(\fb))\) iff \(y^m=a+b\), where \(a^{n_a}\in\fa\)
and \(b^{n_b}\in\fb\).
Then \((y^m)^{n_a+n_b}=(a+b)^{n_a+n_b}\in\fa+\fb\)
\item \(x\in r(\fp^n)\) iff \(x^m\in\fp^n\), then \(x^m=p_1\cdots p_n\in\fp\)
\end{enumerate}
\end{proof}

\begin{proposition}[]
The radical of an ideal \(\fa\) is the intersection of the prime ideals which contain \(\fa\)
\end{proposition}

\begin{proof}
Apply \ref{1.8} to \(A/\fa\).

Nilradical of \(A/\fa\) is the radical of \(\fa\).
\end{proof}

More generally, we may define the radical \(r(E)\) of any \textbf{subset} \(E\) of \(A\) in the same way.
It is \textbf{not} an ideal in general. We have \(r(\bigcup_\alpha E_\alpha)=\bigcup r(E_\alpha)\) for any family of subsets \(E_\alpha\)
of \(A\)

\begin{proposition}[]
\(D=\) set of zero-divisors of \(A=\bigcup_{x\neq 0}r(\Ann(x))\)
\end{proposition}

\begin{proof}
\(D=r(D)=r(\bigcup_{x\neq 0}\Ann(x))=\bigcup_{x\neq 0}r(\Ann(x))\)
\end{proof}

\begin{examplle}[]
If \(A=\Z\), \(\fa=(m)\), let \(p_i\) (\(1\le i\le r\)) be the distinct prime divisors of \(m\).
Then \(r(\fa)=(p_1\cdots p_r)=\bigcap_{i=1}^n(p_i)\)
\end{examplle}

\begin{proposition}[]
Let \(\fa\), \(\fb\) be ideals in a ring \(A\) s.t. \(r(\fa)\), \(r(\fb)\) are coprime. Then \(\fa\)
and \(\fb\) are coprime.
\end{proposition}

\begin{proof}
\(r(\fa+\fb)=r(r(\fa)+r(\fb))=r(1)=(1)\), hence \(\fa+\fb=(1)\)
\end{proof}

Let \(f:A\to B\) be a ring homomorphism. If \(\fa\) is an ideal in \(A\), the set \(f(\fa)\) is not
necessarily an ideal in \(B\) (e.g. \(\Z\to\Q\)). We define the \textbf{extension} \(\fa^e\) of \(\fa\) to be the
ideal \(Bf(\fa)\) generated by \(f(\fa)\) in \(B\): explicitly, \(\fa^e\) is the set of all
sums \(\sum y_if(x_i)\) where \(x_i\in\fa\), \(y_i\in B\)

If \(\fb\) is an ideal of \(B\), then \(f^{-1}(\fb)\) is always an ideal of \(A\), called the
\textbf{contraction} \(\fb^c\) of \(\fb\). If \(\fb\) is prime, then \(\fb^c\) is prime. If \(\fa\) is prime, \(\fa^e\)
need not be prime (\(f:\Z\to\Q\),\(\fa\neq 0\), then \(\fa^e=\Q\), which is not a prime ideal)

We can factorize \(f\) as follows:
\begin{equation*}
f\xrightarrow{p}f(A)\xrightarrow{j}B
\end{equation*}
where \(p\) is surjective and \(j\) is injective

\begin{examplle}[]
Consider \(\Z\to\Z[i]\), where \(i=\sqrt{-1}\). A prime ideal \((p)\) of \(\Z\) may or may not stay
prime when extended to \(\Z[i]\). In fact \(\Z[i]\) is a principal ideal domain (because it has a
Euclidean algorithm, i.e., a Euclidean ring) and the situation is as follows:
\begin{enumerate}
\item \((2^e)=((1+i)^2)\), the \textbf{square} of a prime ideal in \(\Z[i]\)
\item if \(p\equiv 1\mod 4\) then \((p)^e\) is the product of two distinct prime ideals
(for example, \((5)^e=(2+i)(2-i)\))
\item if \(p\equiv 3\mod 4\) then \((p)^e\) is prime in \(\Z[i]\)
\end{enumerate}
\end{examplle}

Let \(f:A\to B\), \(\fa\) and \(\fb\) be as before. Then
\begin{proposition}[]
\begin{enumerate}
\item \(\fa\subseteq\fa^{ec}\), \(\fb\supseteq\fb^{ce}\)
\item \(\fb^c=\fb^{cec}\), \(\fa^e=\fa^{ece}\)
\item If \(C\) is the set of contracted ideals in \(A\) and if \(E\) is the set of extended ideals
in \(B\), then \(C=\{\fa\mid\fa^{ec}=\fa\}\), \(E=\{\fb\mid\fb^{ce}=\fb\}\), and \(\fa\mapsto\fa^e\) is a bijective map
of \(C\) onto \(E\), whose inverse is \(\fb\mapsto\fb^c\).
\end{enumerate}
\end{proposition}

\begin{proof}
\begin{enumerate}
\setcounter{enumi}{2}
\item If \(\fa\in C\), then \(\fa=\fb^c=\fb^{cec}=\fa^{ec}\); conversely if \(\fa=\fa^{ec}\) then \(\fa\) is the
contraction of \(\fa^e\).
\end{enumerate}
\end{proof}

\begin{proof}
\begin{enumerate}
\item 
\end{enumerate}
\end{proof}

\begin{exercise}
If \(\fa_1,\fa_2\) are ideals of \(A\) and if \(\fb_1,\fb_2\) are ideals of \(B\), then
\begin{alignat*}{2}
&(\fa_1+\fa_2)^e=\fa_1^e+\fa_2^e\quad&&(\fb_1+\fb_2)^c\supseteq\fb_1^c+\fb_2^c\\
\end{alignat*}
\end{exercise}
\subsection{Exercise}
\label{sec:org92bcb9f}
\begin{exercise}
\label{ex1.1}
Let \(x\) be a nilpotent element of a ring \(A\). Show that \(1+x\) is a unit of \(A\). Deduce
that the sum of a nilpotent element and a unit is a unit
\end{exercise}

\begin{proof}
\(x\) is a element of a nilradical, which is the intersection all prime ideals. Since every
maximal ideal is a prime ideal, then nilradical is a subset of Jacobson radical.
Then \(1-(-u^{-1})x\) is a unit for some unit \(u\), hence \(u+x\) is a unit
\end{proof}
\end{document}
