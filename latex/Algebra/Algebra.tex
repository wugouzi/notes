% Created 2020-06-16 二 14:45
% Intended LaTeX compiler: pdflatex
\documentclass[11pt]{article}
\usepackage[utf8]{inputenc}
\usepackage[T1]{fontenc}
\usepackage{graphicx}
\usepackage{grffile}
\usepackage{longtable}
\usepackage{wrapfig}
\usepackage{rotating}
\usepackage[normalem]{ulem}
\usepackage{amsmath}
\usepackage{textcomp}
\usepackage{amssymb}
\usepackage{capt-of}
\usepackage{hyperref}
\usepackage{minted}
% TIPS
% \substack{a\\b} for multiple lines text





% pdfplots will load xolor automatically without option
\usepackage[dvipsnames]{xcolor}

\usepackage{forest}
% two-line text in node by [two \\ lines]
% \begin{forest} qtree, [..] \end{forest}
\forestset{
  qtree/.style={
    baseline,
    for tree={
      parent anchor=south,
      child anchor=north,
      align=center,
      inner sep=1pt,
    }}}
%\usepackage{flexisym}
% load order of mathtools and mathabx, otherwise conflict overbrace

\usepackage{mathtools}
%\usepackage{fourier}
\usepackage{pgfplots}
\usepackage{amsthm, mathabx,  amsmath, commath}
\usepackage{amsfonts}

\usepackage{empheq}
\usepackage{tikz}
\usetikzlibrary{arrows.meta}
\usepackage[most]{tcolorbox}

\newtheorem{theorem}{Theorem}[section]
\newtheorem{definition}{Definition}[section]
\newtheorem{corollary}{Corollary}[section]
\newtheorem{example}{Example}[section]
\newtheorem{lemma}{Lemma}[section]
\newtheorem{proposition}{Proposition}[section]

\newcommand{\bl}[1] {\boldsymbol{#1}}
\newcommand{\Wt}[1] {\stackrel{\sim}{\smash{#1}\rule{0pt}{1.1ex}}}
\newcommand{\wt}[1] {\widetilde{#1}}


%For boxed texts in align, use Aboxed{}
%otherwise use boxed{}

\DeclareMathSymbol{\widehatsym}{\mathord}{largesymbols}{"62}
\newcommand\lowerwidehatsym{%
  \text{\smash{\raisebox{-1.3ex}{%
    $\widehatsym$}}}}
\newcommand\fixwidehat[1]{%
  \mathchoice
    {\accentset{\displaystyle\lowerwidehatsym}{#1}}
    {\accentset{\textstyle\lowerwidehatsym}{#1}}
    {\accentset{\scriptstyle\lowerwidehatsym}{#1}}
    {\accentset{\scriptscriptstyle\lowerwidehatsym}{#1}}
}

\usepackage{graphicx}
    
% text on arrow for xRightarrow
\makeatletter
%\newcommand{\xRightarrow}[2][]{\ext@arrow 0359\Rightarrowfill@{#1}{#2}}
\makeatother


\def \bx {\boldsymbol{x}}
\def \ba {\boldsymbol{a}}
\def \bI {\boldsymbol{I}}
\def \bt {\boldsymbol{t}}
\def \bb {\boldsymbol{b}}
\def \bA {\boldsymbol{A}}
\def \bX {\boldsymbol{X}}
\def \bu {\boldsymbol{u}}
\def \bS {\boldsymbol{S}}
\def \bZ {\boldsymbol{Z}}
\def \bz {\boldsymbol{z}}
\def \by {\boldsymbol{y}}
\def \bw {\boldsymbol{w}}
\def \bT {\boldsymbol{T}}
\def \bS {\boldsymbol{S}}
\def \bm {\boldsymbol{m}}
\def \bW {\boldsymbol{W}}
\def \bY {\boldsymbol{Y}}
\def \bH {\boldsymbol{H}}
\def \blambda {\boldsymbol{\lambda}}
\def \bPhi {\boldsymbol{\Phi}}
\def \btheta {\boldsymbol{\theta}}
\def \bmu {\boldsymbol{\mu}}
\def \bphi {\boldsymbol{\phi}}
\def \bSigma {\boldsymbol{\Sigma}}
\def \lb {\left\{}
\def \rb {\right\}}
\def \caln {\mathcal{N}}
\def \dissum {\displaystyle\Sigma}
\def \dispro {\displaystyle\prod}
\def \E {\mathbb{E}}
\def \Q {\mathbb{Q}}
\def \V {\mathbb{V}}
\def \R {\mathbb{R}}
\def \calq {\mathcal{Q}}
\def \calg {\mathcal{G}}
\def \caln {\mathcal{N}}
\def \calr {\mathcal{R}}
\def \calm {\mathcal{M}}
\def \calc {\mathcal{C}}
\def \bcup {\bigcup}

\author{Serge Lang}
\date{\today}
\title{Algebra}
\hypersetup{
 pdfauthor={Serge Lang},
 pdftitle={Algebra},
 pdfkeywords={},
 pdfsubject={},
 pdfcreator={Emacs 26.3 (Org mode 9.4)}, 
 pdflang={English}}
\begin{document}

\maketitle
\tableofcontents \clearpage\section{Real Fields}
\label{sec:org5990011}
\subsection{Ordered Fields}
\label{sec:org6f67a81}
Let \(K\) be a field. An \textbf{ordering} of \(K\) is a subset \(P\) of \(K\)
having the following properties
\bigskip
\begin{itemize}[itemindent=3em]
\item[\textbf{ORD 1.}] Given \(x\in K\), we have either \(x\in P\) ,or \(x=0\) or
\(-x\in P\), and these three possibilities are mutually exclusive
\item[\textbf{ORD 2.}] If \(x,y\in P\), then \(x+y,xy\in P\)
\end{itemize}

\(K\) is \textbf{ordered by} \(P\), and we call \(P\) the set of \textbf{positive
elements}

Suppose \(K\) is ordered by \(P\). Since \(1\neq0\) and \(1=1^2=(-1)^2\), we
see that \(1\in P\). By \textbf{ORD 2}, it follows that \(1+\dots+1\in P\), whence \(K\)
has characteristic 0. If \(x\in P\) and \(x\neq0\), then \(xx^{-1}=1\in P\) implies
that \(x^{-1}\in P\)

\begin{center}
\emph{Let \(E\) be a field. Then a product of sums of squares in \(E\) is a sum
of squares.}

\emph{If \(a,b\in E\) are sum of squares and \(b\neq0\), then \(a/b\) is a sum of
squares}
\end{center}

Consider complex number:)

Let \(x,y\in K\). We define \(x<y\) to mean that \(y-x\in P\). If \(x<0\) we say
that \(x\) is \textbf{negative}.

If \(K\) is ordered and \(x\in K\), \(x\neq0\), then \(x^2\) is positive

If \(E\) has characteristic \(\neq2\), and \(-1\) is a sum of squares in \(E\),
then every element \(a\in E\) is a sum of squares, because
\(4a=(1+a)^2-(1-a)^2\)

If \(K\) is a field with an ordering \(P\), and \(F\) is a subfield, then
obviously, \(P\cap F\) defines an ordering of \(F\), which is called the
\textbf{induced} ordering

Let \(K\) be an ordered field and let \(F\) be a subfield with the induced
ordering. We put \(\abs{x}=x\) if \(x>0\) and \(\abs{x}=-x\) if \(x<0\). An
element \(\alpha\in K\) is \textbf{infinitely large} over \(F\) if \(\abs{\alpha}\ge x\) for all
\(x\in F\). It is \textbf{infinitely small} over \(F\) if \(0\le\abs{\alpha}\le\abs{x}\) for
all \(x\in F\), \(x\neq0\). \(\alpha\) is infinitely large if and only if \(\alpha^{-1}\) is
infinitely small. \(K\) is \textbf{archimedean} over \(F\) if \(K\) has no elements
which are infinitely large over \(F\). An intermediate field \(F_1\),
\(K\supset F_1\supset F\) is \textbf{maximal archimedean over} \(F\) in \(K\) if it is
archimedean over \(F\) and no other intermediate field containing \(F_1\) is
archimedean over \(F\). We say that \(F\) is \textbf{maximal archimedean in} \(K\)
if it is maximal archimedean over itself in \(K\)

Let \(K\) be an ordered field and \(F\) a subfield. Let \(K\) be an ordered
field and \(F\) a subfield. Let \(\fo\) be the set of elements of \(K\)
which are not infinitely large over \(F\). Then \(\fo\) is a ring and that
for any \(\alpha\in K\), we have \(\alpha\) or \(\alpha^{-1}\in\fo\). Hence \(\fo\) is what is
called a valuation ring, containing \(F\). Let \(\fm\) be the ideal of all
\(\alpha\in K\) which are infinitely small over \(F\). Then \(\fm\) is the unique
maximal ideal of \(\fo\), because any element in \(\fo\) which is not in
\(\fm\) has an inverse in \(\fo\). We call \(\fo\) the
\textbf{valuation ring determined by the ordering of} \(K/F\)

\begin{proposition}[]
Let \(K\) be an ordered field and \(F\) a subfield. Let \(\fo\) be the
valuation ring determined by the ordering of \(K/F\), and let \(\fm\) be its
maximal ideal. Then \(\fo/\fm\) is a real field.
\end{proposition}

\begin{proof}
Otherwise, we could write
\begin{equation*}
-1=\displaystyle\sum\alpha_i^2+a
\end{equation*}
with \(\alpha_i\in\fo\) and \(a\in\fm\). Since \(\sum\alpha_i^2\) is positive and \(a\) is
infinitely small, such a relation is clearly impossible
\end{proof}
\end{document}
