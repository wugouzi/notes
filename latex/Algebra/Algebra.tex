% Created 2021-08-06 Fri 00:02
% Intended LaTeX compiler: pdflatex
\documentclass[11pt]{article}
\usepackage[utf8]{inputenc}
\usepackage[T1]{fontenc}
\usepackage{graphicx}
\usepackage{grffile}
\usepackage{longtable}
\usepackage{wrapfig}
\usepackage{rotating}
\usepackage[normalem]{ulem}
\usepackage{amsmath}
\usepackage{textcomp}
\usepackage{amssymb}
\usepackage{capt-of}
\usepackage{hyperref}
% wrong resolution of image
% https://tex.stackexchange.com/questions/21627/image-from-includegraphics-showing-in-wrong-image-size?rq=1

%%%%%%%%%%%%%%%%%%%%%%%%%%%%%%%%%%%%%%
%% TIPS                                 %%
%%%%%%%%%%%%%%%%%%%%%%%%%%%%%%%%%%%%%%
% \substack{a\\b} for multiple lines text
% \usepackage{expl3}
% \expandafter\def\csname ver@l3regex.sty\endcsname{}
% \usepackage{pkgloader}
\usepackage[utf8]{inputenc}

% nfss error
% \usepackage[B1,T1]{fontenc}
\usepackage{fontspec}

% \usepackage[Emoticons]{ucharclasses}
\newfontfamily\DejaSans{DejaVu Sans}
% \setDefaultTransitions{\DejaSans}{}

% pdfplots will load xolor automatically without option
\usepackage[dvipsnames]{xcolor}

%                                                             ┳┳┓   ┓
%                                                             ┃┃┃┏┓╋┣┓
%                                                             ┛ ┗┗┻┗┛┗
% \usepackage{amsmath} mathtools loads the amsmath
\usepackage{amsmath}
\usepackage{mathtools}

\usepackage{amsthm}
\usepackage{amsbsy}

%\usepackage{commath}

\usepackage{amssymb}

\usepackage{mathrsfs}
%\usepackage{mathabx}
\usepackage{stmaryrd}
\usepackage{empheq}

\usepackage{scalerel}
\usepackage{stackengine}
\usepackage{stackrel}



\usepackage{nicematrix}
\usepackage{tensor}
\usepackage{blkarray}
\usepackage{siunitx}
\usepackage[f]{esvect}

% centering \not on a letter
\usepackage{slashed}
\usepackage[makeroom]{cancel}

%\usepackage{merriweather}
\usepackage{unicode-math}
\setmainfont{TeX Gyre Pagella}
% \setmathfont{STIX}
%\setmathfont{texgyrepagella-math.otf}
%\setmathfont{Libertinus Math}
\setmathfont{Latin Modern Math}

 % \setmathfont[range={\smwhtdiamond,\enclosediamond,\varlrtriangle}]{Latin Modern Math}
\setmathfont[range={\rightrightarrows,\twoheadrightarrow,\leftrightsquigarrow,\triangledown,\vartriangle,\precneq,\succneq,\prec,\succ,\preceq,\succeq,\tieconcat}]{XITS Math}
 \setmathfont[range={\int,\setminus}]{Libertinus Math}
 % \setmathfont[range={\mathalpha}]{TeX Gyre Pagella Math}
%\setmathfont[range={\mitA,\mitB,\mitC,\mitD,\mitE,\mitF,\mitG,\mitH,\mitI,\mitJ,\mitK,\mitL,\mitM,\mitN,\mitO,\mitP,\mitQ,\mitR,\mitS,\mitT,\mitU,\mitV,\mitW,\mitX,\mitY,\mitZ,\mita,\mitb,\mitc,\mitd,\mite,\mitf,\mitg,\miti,\mitj,\mitk,\mitl,\mitm,\mitn,\mito,\mitp,\mitq,\mitr,\mits,\mitt,\mitu,\mitv,\mitw,\mitx,\mity,\mitz}]{TeX Gyre Pagella Math}
% unicode is not good at this!
%\let\nmodels\nvDash

 \usepackage{wasysym}

 % for wide hat
 \DeclareSymbolFont{yhlargesymbols}{OMX}{yhex}{m}{n} \DeclareMathAccent{\what}{\mathord}{yhlargesymbols}{"62}

%                                                               ┏┳┓•┓
%                                                                ┃ ┓┃┏┓
%                                                                ┻ ┗┛┗┗

\usepackage{pgfplots}
\pgfplotsset{compat=1.18}
\usepackage{tikz}
\usepackage{tikz-cd}
\tikzcdset{scale cd/.style={every label/.append style={scale=#1},
    cells={nodes={scale=#1}}}}
% TODO: discard qtree and use forest
% \usepackage{tikz-qtree}
\usepackage{forest}

\usetikzlibrary{arrows,positioning,calc,fadings,decorations,matrix,decorations,shapes.misc}
%setting from geogebra
\definecolor{ccqqqq}{rgb}{0.8,0,0}

%                                                          ┳┳┓•    ┓┓
%                                                          ┃┃┃┓┏┏┏┓┃┃┏┓┏┓┏┓┏┓┓┏┏
%                                                          ┛ ┗┗┛┗┗ ┗┗┗┻┛┗┗ ┗┛┗┻┛
%\usepackage{twemojis}
\usepackage[most]{tcolorbox}
\usepackage{threeparttable}
\usepackage{tabularx}

\usepackage{enumitem}
\usepackage[indLines=false]{algpseudocodex}
\usepackage[]{algorithm2e}
% \SetKwComment{Comment}{/* }{ */}
% \algrenewcommand\algorithmicrequire{\textbf{Input:}}
% \algrenewcommand\algorithmicensure{\textbf{Output:}}
% wrong with preview
\usepackage{subcaption}
\usepackage{caption}
% {\aunclfamily\Huge}
\usepackage{auncial}

\usepackage{float}

\usepackage{fancyhdr}

\usepackage{ifthen}
\usepackage{xargs}

\definecolor{mintedbg}{rgb}{0.99,0.99,0.99}
\usepackage[cachedir=\detokenize{~/miscellaneous/trash}]{minted}
\setminted{breaklines,
  mathescape,
  bgcolor=mintedbg,
  fontsize=\footnotesize,
  frame=single,
  linenos}
\usemintedstyle{xcode}
\usepackage{tcolorbox}
\usepackage{etoolbox}



\usepackage{imakeidx}
\usepackage{hyperref}
\usepackage{soul}
\usepackage{framed}

% don't use this for preview
%\usepackage[margin=1.5in]{geometry}
% \usepackage{geometry}
% \geometry{legalpaper, landscape, margin=1in}
\usepackage[font=itshape]{quoting}

%\LoadPackagesNow
%\usepackage[xetex]{preview}
%%%%%%%%%%%%%%%%%%%%%%%%%%%%%%%%%%%%%%%
%% USEPACKAGES end                       %%
%%%%%%%%%%%%%%%%%%%%%%%%%%%%%%%%%%%%%%%

%%%%%%%%%%%%%%%%%%%%%%%%%%%%%%%%%%%%%%%
%% Algorithm environment
%%%%%%%%%%%%%%%%%%%%%%%%%%%%%%%%%%%%%%%
\SetKwIF{Recv}{}{}{upon receiving}{do}{}{}{}
\SetKwBlock{Init}{initially do}{}
\SetKwProg{Function}{Function}{:}{}

% https://github.com/chrmatt/algpseudocodex/issues/3
\algnewcommand\algorithmicswitch{\textbf{switch}}%
\algnewcommand\algorithmiccase{\textbf{case}}
\algnewcommand\algorithmicof{\textbf{of}}
\algnewcommand\algorithmicotherwise{\texttt{otherwise} $\Rightarrow$}

\makeatletter
\algdef{SE}[SWITCH]{Switch}{EndSwitch}[1]{\algpx@startIndent\algpx@startCodeCommand\algorithmicswitch\ #1\ \algorithmicdo}{\algpx@endIndent\algpx@startCodeCommand\algorithmicend\ \algorithmicswitch}%
\algdef{SE}[CASE]{Case}{EndCase}[1]{\algpx@startIndent\algpx@startCodeCommand\algorithmiccase\ #1}{\algpx@endIndent\algpx@startCodeCommand\algorithmicend\ \algorithmiccase}%
\algdef{SE}[CASEOF]{CaseOf}{EndCaseOf}[1]{\algpx@startIndent\algpx@startCodeCommand\algorithmiccase\ #1 \algorithmicof}{\algpx@endIndent\algpx@startCodeCommand\algorithmicend\ \algorithmiccase}
\algdef{SE}[OTHERWISE]{Otherwise}{EndOtherwise}[0]{\algpx@startIndent\algpx@startCodeCommand\algorithmicotherwise}{\algpx@endIndent\algpx@startCodeCommand\algorithmicend\ \algorithmicotherwise}
\ifbool{algpx@noEnd}{%
  \algtext*{EndSwitch}%
  \algtext*{EndCase}%
  \algtext*{EndCaseOf}
  \algtext*{EndOtherwise}
  %
  % end indent line after (not before), to get correct y position for multiline text in last command
  \apptocmd{\EndSwitch}{\algpx@endIndent}{}{}%
  \apptocmd{\EndCase}{\algpx@endIndent}{}{}%
  \apptocmd{\EndCaseOf}{\algpx@endIndent}{}{}
  \apptocmd{\EndOtherwise}{\algpx@endIndent}{}{}
}{}%

\pretocmd{\Switch}{\algpx@endCodeCommand}{}{}
\pretocmd{\Case}{\algpx@endCodeCommand}{}{}
\pretocmd{\CaseOf}{\algpx@endCodeCommand}{}{}
\pretocmd{\Otherwise}{\algpx@endCodeCommand}{}{}

% for end commands that may not be printed, tell endCodeCommand whether we are using noEnd
\ifbool{algpx@noEnd}{%
  \pretocmd{\EndSwitch}{\algpx@endCodeCommand[1]}{}{}%
  \pretocmd{\EndCase}{\algpx@endCodeCommand[1]}{}{}
  \pretocmd{\EndCaseOf}{\algpx@endCodeCommand[1]}{}{}%
  \pretocmd{\EndOtherwise}{\algpx@endCodeCommand[1]}{}{}
}{%
  \pretocmd{\EndSwitch}{\algpx@endCodeCommand[0]}{}{}%
  \pretocmd{\EndCase}{\algpx@endCodeCommand[0]}{}{}%
  \pretocmd{\EndCaseOf}{\algpx@endCodeCommand[0]}{}{}
  \pretocmd{\EndOtherwise}{\algpx@endCodeCommand[0]}{}{}
}%
\makeatother
% % For algpseudocode
% \algnewcommand\algorithmicswitch{\textbf{switch}}
% \algnewcommand\algorithmiccase{\textbf{case}}
% \algnewcommand\algorithmiccaseof{\textbf{case}}
% \algnewcommand\algorithmicof{\textbf{of}}
% % New "environments"
% \algdef{SE}[SWITCH]{Switch}{EndSwitch}[1]{\algorithmicswitch\ #1\ \algorithmicdo}{\algorithmicend\ \algorithmicswitch}%
% \algdef{SE}[CASE]{Case}{EndCase}[1]{\algorithmiccase\ #1}{\algorithmicend\ \algorithmiccase}%
% \algtext*{EndSwitch}%
% \algtext*{EndCase}
% \algdef{SE}[CASEOF]{CaseOf}{EndCaseOf}[1]{\algorithmiccaseof\ #1 \algorithmicof}{\algorithmicend\ \algorithmiccaseof}
% \algtext*{EndCaseOf}



%\pdfcompresslevel0

% quoting from
% https://tex.stackexchange.com/questions/391726/the-quotation-environment
\NewDocumentCommand{\bywhom}{m}{% the Bourbaki trick
  {\nobreak\hfill\penalty50\hskip1em\null\nobreak
   \hfill\mbox{\normalfont(#1)}%
   \parfillskip=0pt \finalhyphendemerits=0 \par}%
}

\NewDocumentEnvironment{pquotation}{m}
  {\begin{quoting}[
     indentfirst=true,
     leftmargin=\parindent,
     rightmargin=\parindent]\itshape}
  {\bywhom{#1}\end{quoting}}

\indexsetup{othercode=\small}
\makeindex[columns=2,options={-s /media/wu/file/stuuudy/notes/index_style.ist},intoc]
\makeatletter
\def\@idxitem{\par\hangindent 0pt}
\makeatother


% \newcounter{dummy} \numberwithin{dummy}{section}
\newtheorem{dummy}{dummy}[section]
\theoremstyle{definition}
\newtheorem{definition}[dummy]{Definition}
\theoremstyle{plain}
\newtheorem{corollary}[dummy]{Corollary}
\newtheorem{lemma}[dummy]{Lemma}
\newtheorem{proposition}[dummy]{Proposition}
\newtheorem{theorem}[dummy]{Theorem}
\newtheorem{notation}[dummy]{Notation}
\newtheorem{conjecture}[dummy]{Conjecture}
\newtheorem{fact}[dummy]{Fact}
\newtheorem{warning}[dummy]{Warning}
\theoremstyle{definition}
\newtheorem{examplle}{Example}[section]
\theoremstyle{remark}
\newtheorem*{remark}{Remark}
\newtheorem{exercise}{Exercise}[subsection]
\newtheorem{problem}{Problem}[subsection]
\newtheorem{observation}{Observation}[section]
\newenvironment{claim}[1]{\par\noindent\textbf{Claim:}\space#1}{}

\makeatletter
\DeclareFontFamily{U}{tipa}{}
\DeclareFontShape{U}{tipa}{m}{n}{<->tipa10}{}
\newcommand{\arc@char}{{\usefont{U}{tipa}{m}{n}\symbol{62}}}%

\newcommand{\arc}[1]{\mathpalette\arc@arc{#1}}

\newcommand{\arc@arc}[2]{%
  \sbox0{$\m@th#1#2$}%
  \vbox{
    \hbox{\resizebox{\wd0}{\height}{\arc@char}}
    \nointerlineskip
    \box0
  }%
}
\makeatother

\setcounter{MaxMatrixCols}{20}
%%%%%%% ABS
\DeclarePairedDelimiter\abss{\lvert}{\rvert}%
\DeclarePairedDelimiter\normm{\lVert}{\rVert}%

% Swap the definition of \abs* and \norm*, so that \abs
% and \norm resizes the size of the brackets, and the
% starred version does not.
\makeatletter
\let\oldabs\abss
%\def\abs{\@ifstar{\oldabs}{\oldabs*}}
\newcommand{\abs}{\@ifstar{\oldabs}{\oldabs*}}
\newcommand{\norm}[1]{\left\lVert#1\right\rVert}
%\let\oldnorm\normm
%\def\norm{\@ifstar{\oldnorm}{\oldnorm*}}
%\renewcommand{norm}{\@ifstar{\oldnorm}{\oldnorm*}}
\makeatother

% \stackMath
% \newcommand\what[1]{%
% \savestack{\tmpbox}{\stretchto{%
%   \scaleto{%
%     \scalerel*[\widthof{\ensuremath{#1}}]{\kern-.6pt\bigwedge\kern-.6pt}%
%     {\rule[-\textheight/2]{1ex}{\textheight}}%WIDTH-LIMITED BIG WEDGE
%   }{\textheight}%
% }{0.5ex}}%
% \stackon[1pt]{#1}{\tmpbox}%
% }

% \newcommand\what[1]{\ThisStyle{%
%     \setbox0=\hbox{$\SavedStyle#1$}%
%     \stackengine{-1.0\ht0+.5pt}{$\SavedStyle#1$}{%
%       \stretchto{\scaleto{\SavedStyle\mkern.15mu\char'136}{2.6\wd0}}{1.4\ht0}%
%     }{O}{c}{F}{T}{S}%
%   }
% }

% \newcommand\wtilde[1]{\ThisStyle{%
%     \setbox0=\hbox{$\SavedStyle#1$}%
%     \stackengine{-.1\LMpt}{$\SavedStyle#1$}{%
%       \stretchto{\scaleto{\SavedStyle\mkern.2mu\AC}{.5150\wd0}}{.6\ht0}%
%     }{O}{c}{F}{T}{S}%
%   }
% }

% \newcommand\wbar[1]{\ThisStyle{%
%     \setbox0=\hbox{$\SavedStyle#1$}%
%     \stackengine{.5pt+\LMpt}{$\SavedStyle#1$}{%
%       \rule{\wd0}{\dimexpr.3\LMpt+.3pt}%
%     }{O}{c}{F}{T}{S}%
%   }
% }

\newcommand{\bl}[1] {\boldsymbol{#1}}
\newcommand{\Wt}[1] {\stackrel{\sim}{\smash{#1}\rule{0pt}{1.1ex}}}
\newcommand{\wt}[1] {\widetilde{#1}}
\newcommand{\tf}[1] {\textbf{#1}}

\newcommand{\wu}[1]{{\color{red} #1}}

%For boxed texts in align, use Aboxed{}
%otherwise use boxed{}

\DeclareMathSymbol{\widehatsym}{\mathord}{largesymbols}{"62}
\newcommand\lowerwidehatsym{%
  \text{\smash{\raisebox{-1.3ex}{%
    $\widehatsym$}}}}
\newcommand\fixwidehat[1]{%
  \mathchoice
    {\accentset{\displaystyle\lowerwidehatsym}{#1}}
    {\accentset{\textstyle\lowerwidehatsym}{#1}}
    {\accentset{\scriptstyle\lowerwidehatsym}{#1}}
    {\accentset{\scriptscriptstyle\lowerwidehatsym}{#1}}
  }


\newcommand{\cupdot}{\mathbin{\dot{\cup}}}
\newcommand{\bigcupdot}{\mathop{\dot{\bigcup}}}

\usepackage{graphicx}

\usepackage[toc,page]{appendix}

% text on arrow for xRightarrow
\makeatletter
%\newcommand{\xRightarrow}[2][]{\ext@arrow 0359\Rightarrowfill@{#1}{#2}}
\makeatother

% Arbitrary long arrow
\newcommand{\Rarrow}[1]{%
\parbox{#1}{\tikz{\draw[->](0,0)--(#1,0);}}
}

\newcommand{\LRarrow}[1]{%
\parbox{#1}{\tikz{\draw[<->](0,0)--(#1,0);}}
}


\makeatletter
\providecommand*{\rmodels}{%
  \mathrel{%
    \mathpalette\@rmodels\models
  }%
}
\newcommand*{\@rmodels}[2]{%
  \reflectbox{$\m@th#1#2$}%
}
\makeatother

% Roman numerals
\makeatletter
\newcommand*{\rom}[1]{\expandafter\@slowromancap\romannumeral #1@}
\makeatother
% \\def \\b\([a-zA-Z]\) {\\boldsymbol{[a-zA-z]}}
% \\DeclareMathOperator{\\b\1}{\\textbf{\1}}

\DeclareMathOperator*{\argmin}{arg\,min}
\DeclareMathOperator*{\argmax}{arg\,max}

\DeclareMathOperator{\bone}{\textbf{1}}
\DeclareMathOperator{\bx}{\textbf{x}}
\DeclareMathOperator{\bz}{\textbf{z}}
\DeclareMathOperator{\bff}{\textbf{f}}
\DeclareMathOperator{\ba}{\textbf{a}}
\DeclareMathOperator{\bk}{\textbf{k}}
\DeclareMathOperator{\bs}{\textbf{s}}
\DeclareMathOperator{\bh}{\textbf{h}}
\DeclareMathOperator{\bc}{\textbf{c}}
\DeclareMathOperator{\br}{\textbf{r}}
\DeclareMathOperator{\bi}{\textbf{i}}
\DeclareMathOperator{\bj}{\textbf{j}}
\DeclareMathOperator{\bn}{\textbf{n}}
\DeclareMathOperator{\be}{\textbf{e}}
\DeclareMathOperator{\bo}{\textbf{o}}
\DeclareMathOperator{\bU}{\textbf{U}}
\DeclareMathOperator{\bL}{\textbf{L}}
\DeclareMathOperator{\bV}{\textbf{V}}
\def \bzero {\mathbf{0}}
\def \bbone {\mathbb{1}}
\def \btwo {\mathbf{2}}
\DeclareMathOperator{\bv}{\textbf{v}}
\DeclareMathOperator{\bp}{\textbf{p}}
\DeclareMathOperator{\bI}{\textbf{I}}
\def \dbI {\dot{\bI}}
\DeclareMathOperator{\bM}{\textbf{M}}
\DeclareMathOperator{\bN}{\textbf{N}}
\DeclareMathOperator{\bK}{\textbf{K}}
\DeclareMathOperator{\bt}{\textbf{t}}
\DeclareMathOperator{\bb}{\textbf{b}}
\DeclareMathOperator{\bA}{\textbf{A}}
\DeclareMathOperator{\bX}{\textbf{X}}
\DeclareMathOperator{\bu}{\textbf{u}}
\DeclareMathOperator{\bS}{\textbf{S}}
\DeclareMathOperator{\bZ}{\textbf{Z}}
\DeclareMathOperator{\bJ}{\textbf{J}}
\DeclareMathOperator{\by}{\textbf{y}}
\DeclareMathOperator{\bw}{\textbf{w}}
\DeclareMathOperator{\bT}{\textbf{T}}
\DeclareMathOperator{\bF}{\textbf{F}}
\DeclareMathOperator{\bmm}{\textbf{m}}
\DeclareMathOperator{\bW}{\textbf{W}}
\DeclareMathOperator{\bR}{\textbf{R}}
\DeclareMathOperator{\bC}{\textbf{C}}
\DeclareMathOperator{\bD}{\textbf{D}}
\DeclareMathOperator{\bE}{\textbf{E}}
\DeclareMathOperator{\bQ}{\textbf{Q}}
\DeclareMathOperator{\bP}{\textbf{P}}
\DeclareMathOperator{\bY}{\textbf{Y}}
\DeclareMathOperator{\bH}{\textbf{H}}
\DeclareMathOperator{\bB}{\textbf{B}}
\DeclareMathOperator{\bG}{\textbf{G}}
\def \blambda {\symbf{\lambda}}
\def \boldeta {\symbf{\eta}}
\def \balpha {\symbf{\alpha}}
\def \btau {\symbf{\tau}}
\def \bbeta {\symbf{\beta}}
\def \bgamma {\symbf{\gamma}}
\def \bxi {\symbf{\xi}}
\def \bLambda {\symbf{\Lambda}}
\def \bGamma {\symbf{\Gamma}}

\newcommand{\bto}{{\boldsymbol{\to}}}
\newcommand{\Ra}{\Rightarrow}
\newcommand{\xrsa}[1]{\overset{#1}{\rightsquigarrow}}
\newcommand{\xlsa}[1]{\overset{#1}{\leftsquigarrow}}
\newcommand\und[1]{\underline{#1}}
\newcommand\ove[1]{\overline{#1}}
%\def \concat {\verb|^|}
\def \bPhi {\mbfPhi}
\def \btheta {\mbftheta}
\def \bTheta {\mbfTheta}
\def \bmu {\mbfmu}
\def \bphi {\mbfphi}
\def \bSigma {\mbfSigma}
\def \la {\langle}
\def \ra {\rangle}

\def \caln {\mathcal{N}}
\def \dissum {\displaystyle\Sigma}
\def \dispro {\displaystyle\prod}

\def \caret {\verb!^!}

\def \A {\mathbb{A}}
\def \B {\mathbb{B}}
\def \C {\mathbb{C}}
\def \D {\mathbb{D}}
\def \E {\mathbb{E}}
\def \F {\mathbb{F}}
\def \G {\mathbb{G}}
\def \H {\mathbb{H}}
\def \I {\mathbb{I}}
\def \J {\mathbb{J}}
\def \K {\mathbb{K}}
\def \L {\mathbb{L}}
\def \M {\mathbb{M}}
\def \N {\mathbb{N}}
\def \O {\mathbb{O}}
\def \P {\mathbb{P}}
\def \Q {\mathbb{Q}}
\def \R {\mathbb{R}}
\def \S {\mathbb{S}}
\def \T {\mathbb{T}}
\def \U {\mathbb{U}}
\def \V {\mathbb{V}}
\def \W {\mathbb{W}}
\def \X {\mathbb{X}}
\def \Y {\mathbb{Y}}
\def \Z {\mathbb{Z}}

\def \cala {\mathcal{A}}
\def \cale {\mathcal{E}}
\def \calb {\mathcal{B}}
\def \calq {\mathcal{Q}}
\def \calp {\mathcal{P}}
\def \cals {\mathcal{S}}
\def \calx {\mathcal{X}}
\def \caly {\mathcal{Y}}
\def \calg {\mathcal{G}}
\def \cald {\mathcal{D}}
\def \caln {\mathcal{N}}
\def \calr {\mathcal{R}}
\def \calt {\mathcal{T}}
\def \calm {\mathcal{M}}
\def \calw {\mathcal{W}}
\def \calc {\mathcal{C}}
\def \calv {\mathcal{V}}
\def \calf {\mathcal{F}}
\def \calk {\mathcal{K}}
\def \call {\mathcal{L}}
\def \calu {\mathcal{U}}
\def \calo {\mathcal{O}}
\def \calh {\mathcal{H}}
\def \cali {\mathcal{I}}
\def \calj {\mathcal{J}}

\def \bcup {\bigcup}

% set theory

\def \zfcc {\textbf{ZFC}^-}
\def \BGC {\textbf{BGC}}
\def \BG {\textbf{BG}}
\def \ac  {\textbf{AC}}
\def \gl  {\textbf{L }}
\def \gll {\textbf{L}}
\newcommand{\zfm}{$\textbf{ZF}^-$}

\def \ZFm {\text{ZF}^-}
\def \ZFCm {\text{ZFC}^-}
\DeclareMathOperator{\WF}{WF}
\DeclareMathOperator{\On}{On}
\def \on {\textbf{On }}
\def \cm {\textbf{M }}
\def \cn {\textbf{N }}
\def \cv {\textbf{V }}
\def \zc {\textbf{ZC }}
\def \zcm {\textbf{ZC}}
\def \zff {\textbf{ZF}}
\def \wfm {\textbf{WF}}
\def \onm {\textbf{On}}
\def \cmm {\textbf{M}}
\def \cnm {\textbf{N}}
\def \cvm {\textbf{V}}

\renewcommand{\restriction}{\mathord{\upharpoonright}}
%% another restriction
\newcommand\restr[2]{{% we make the whole thing an ordinary symbol
  \left.\kern-\nulldelimiterspace % automatically resize the bar with \right
  #1 % the function
  \vphantom{\big|} % pretend it's a little taller at normal size
  \right|_{#2} % this is the delimiter
  }}

\def \pred {\text{pred}}

\def \rank {\text{rank}}
\def \Con {\text{Con}}
\def \deff {\text{Def}}


\def \uin {\underline{\in}}
\def \oin {\overline{\in}}
\def \uR {\underline{R}}
\def \oR {\overline{R}}
\def \uP {\underline{P}}
\def \oP {\overline{P}}

\def \dsum {\displaystyle\sum}

\def \Ra {\Rightarrow}

\def \e {\enspace}

\def \sgn {\operatorname{sgn}}
\def \gen {\operatorname{gen}}
\def \Hom {\operatorname{Hom}}
\def \hom {\operatorname{hom}}
\def \Sub {\operatorname{Sub}}

\def \supp {\operatorname{supp}}

\def \epiarrow {\twoheadarrow}
\def \monoarrow {\rightarrowtail}
\def \rrarrow {\rightrightarrows}

% \def \minus {\text{-}}
% \newcommand{\minus}{\scalebox{0.75}[1.0]{$-$}}
% \DeclareUnicodeCharacter{002D}{\minus}


\def \tril {\triangleleft}

\def \ISigma {\text{I}\Sigma}
\def \IDelta {\text{I}\Delta}
\def \IPi {\text{I}\Pi}
\def \ACF {\textsf{ACF}}
\def \pCF {\textit{p}\text{CF}}
\def \ACVF {\textsf{ACVF}}
\def \HLR {\textsf{HLR}}
\def \OAG {\textsf{OAG}}
\def \RCF {\textsf{RCF}}
\DeclareMathOperator{\GL}{GL}
\DeclareMathOperator{\PGL}{PGL}
\DeclareMathOperator{\SL}{SL}
\DeclareMathOperator{\Inv}{Inv}
\DeclareMathOperator{\res}{res}
\DeclareMathOperator{\Sym}{Sym}
%\DeclareMathOperator{\char}{char}
\def \equal {=}

\def \degree {\text{degree}}
\def \app {\text{App}}
\def \FV {\text{FV}}
\def \conv {\text{conv}}
\def \cont {\text{cont}}
\DeclareMathOperator{\cl}{\text{cl}}
\DeclareMathOperator{\trcl}{\text{trcl}}
\DeclareMathOperator{\sg}{sg}
\DeclareMathOperator{\trdeg}{trdeg}
\def \Ord {\text{Ord}}

\DeclareMathOperator{\cf}{cf}
\DeclareMathOperator{\zfc}{ZFC}

%\DeclareMathOperator{\Th}{Th}
%\def \th {\text{Th}}
% \newcommand{\th}{\text{Th}}
\DeclareMathOperator{\type}{type}
\DeclareMathOperator{\zf}{\textbf{ZF}}
\def \fa {\mathfrak{a}}
\def \fb {\mathfrak{b}}
\def \fc {\mathfrak{c}}
\def \fd {\mathfrak{d}}
\def \fe {\mathfrak{e}}
\def \ff {\mathfrak{f}}
\def \fg {\mathfrak{g}}
\def \fh {\mathfrak{h}}
%\def \fi {\mathfrak{i}}
\def \fj {\mathfrak{j}}
\def \fk {\mathfrak{k}}
\def \fl {\mathfrak{l}}
\def \fm {\mathfrak{m}}
\def \fn {\mathfrak{n}}
\def \fo {\mathfrak{o}}
\def \fp {\mathfrak{p}}
\def \fq {\mathfrak{q}}
\def \fr {\mathfrak{r}}
\def \fs {\mathfrak{s}}
\def \ft {\mathfrak{t}}
\def \fu {\mathfrak{u}}
\def \fv {\mathfrak{v}}
\def \fw {\mathfrak{w}}
\def \fx {\mathfrak{x}}
\def \fy {\mathfrak{y}}
\def \fz {\mathfrak{z}}
\def \fA {\mathfrak{A}}
\def \fB {\mathfrak{B}}
\def \fC {\mathfrak{C}}
\def \fD {\mathfrak{D}}
\def \fE {\mathfrak{E}}
\def \fF {\mathfrak{F}}
\def \fG {\mathfrak{G}}
\def \fH {\mathfrak{H}}
\def \fI {\mathfrak{I}}
\def \fJ {\mathfrak{J}}
\def \fK {\mathfrak{K}}
\def \fL {\mathfrak{L}}
\def \fM {\mathfrak{M}}
\def \fN {\mathfrak{N}}
\def \fO {\mathfrak{O}}
\def \fP {\mathfrak{P}}
\def \fQ {\mathfrak{Q}}
\def \fR {\mathfrak{R}}
\def \fS {\mathfrak{S}}
\def \fT {\mathfrak{T}}
\def \fU {\mathfrak{U}}
\def \fV {\mathfrak{V}}
\def \fW {\mathfrak{W}}
\def \fX {\mathfrak{X}}
\def \fY {\mathfrak{Y}}
\def \fZ {\mathfrak{Z}}

\def \sfA {\textsf{A}}
\def \sfB {\textsf{B}}
\def \sfC {\textsf{C}}
\def \sfD {\textsf{D}}
\def \sfE {\textsf{E}}
\def \sfF {\textsf{F}}
\def \sfG {\textsf{G}}
\def \sfH {\textsf{H}}
\def \sfI {\textsf{I}}
\def \sfJ {\textsf{J}}
\def \sfK {\textsf{K}}
\def \sfL {\textsf{L}}
\def \sfM {\textsf{M}}
\def \sfN {\textsf{N}}
\def \sfO {\textsf{O}}
\def \sfP {\textsf{P}}
\def \sfQ {\textsf{Q}}
\def \sfR {\textsf{R}}
\def \sfS {\textsf{S}}
\def \sfT {\textsf{T}}
\def \sfU {\textsf{U}}
\def \sfV {\textsf{V}}
\def \sfW {\textsf{W}}
\def \sfX {\textsf{X}}
\def \sfY {\textsf{Y}}
\def \sfZ {\textsf{Z}}
\def \sfa {\textsf{a}}
\def \sfb {\textsf{b}}
\def \sfc {\textsf{c}}
\def \sfd {\textsf{d}}
\def \sfe {\textsf{e}}
\def \sff {\textsf{f}}
\def \sfg {\textsf{g}}
\def \sfh {\textsf{h}}
\def \sfi {\textsf{i}}
\def \sfj {\textsf{j}}
\def \sfk {\textsf{k}}
\def \sfl {\textsf{l}}
\def \sfm {\textsf{m}}
\def \sfn {\textsf{n}}
\def \sfo {\textsf{o}}
\def \sfp {\textsf{p}}
\def \sfq {\textsf{q}}
\def \sfr {\textsf{r}}
\def \sfs {\textsf{s}}
\def \sft {\textsf{t}}
\def \sfu {\textsf{u}}
\def \sfv {\textsf{v}}
\def \sfw {\textsf{w}}
\def \sfx {\textsf{x}}
\def \sfy {\textsf{y}}
\def \sfz {\textsf{z}}

\def \ttA {\texttt{A}}
\def \ttB {\texttt{B}}
\def \ttC {\texttt{C}}
\def \ttD {\texttt{D}}
\def \ttE {\texttt{E}}
\def \ttF {\texttt{F}}
\def \ttG {\texttt{G}}
\def \ttH {\texttt{H}}
\def \ttI {\texttt{I}}
\def \ttJ {\texttt{J}}
\def \ttK {\texttt{K}}
\def \ttL {\texttt{L}}
\def \ttM {\texttt{M}}
\def \ttN {\texttt{N}}
\def \ttO {\texttt{O}}
\def \ttP {\texttt{P}}
\def \ttQ {\texttt{Q}}
\def \ttR {\texttt{R}}
\def \ttS {\texttt{S}}
\def \ttT {\texttt{T}}
\def \ttU {\texttt{U}}
\def \ttV {\texttt{V}}
\def \ttW {\texttt{W}}
\def \ttX {\texttt{X}}
\def \ttY {\texttt{Y}}
\def \ttZ {\texttt{Z}}
\def \tta {\texttt{a}}
\def \ttb {\texttt{b}}
\def \ttc {\texttt{c}}
\def \ttd {\texttt{d}}
\def \tte {\texttt{e}}
\def \ttf {\texttt{f}}
\def \ttg {\texttt{g}}
\def \tth {\texttt{h}}
\def \tti {\texttt{i}}
\def \ttj {\texttt{j}}
\def \ttk {\texttt{k}}
\def \ttl {\texttt{l}}
\def \ttm {\texttt{m}}
\def \ttn {\texttt{n}}
\def \tto {\texttt{o}}
\def \ttp {\texttt{p}}
\def \ttq {\texttt{q}}
\def \ttr {\texttt{r}}
\def \tts {\texttt{s}}
\def \ttt {\texttt{t}}
\def \ttu {\texttt{u}}
\def \ttv {\texttt{v}}
\def \ttw {\texttt{w}}
\def \ttx {\texttt{x}}
\def \tty {\texttt{y}}
\def \ttz {\texttt{z}}

\def \bara {\bbar{a}}
\def \barb {\bbar{b}}
\def \barc {\bbar{c}}
\def \bard {\bbar{d}}
\def \bare {\bbar{e}}
\def \barf {\bbar{f}}
\def \barg {\bbar{g}}
\def \barh {\bbar{h}}
\def \bari {\bbar{i}}
\def \barj {\bbar{j}}
\def \bark {\bbar{k}}
\def \barl {\bbar{l}}
\def \barm {\bbar{m}}
\def \barn {\bbar{n}}
\def \baro {\bbar{o}}
\def \barp {\bbar{p}}
\def \barq {\bbar{q}}
\def \barr {\bbar{r}}
\def \bars {\bbar{s}}
\def \bart {\bbar{t}}
\def \baru {\bbar{u}}
\def \barv {\bbar{v}}
\def \barw {\bbar{w}}
\def \barx {\bbar{x}}
\def \bary {\bbar{y}}
\def \barz {\bbar{z}}
\def \barA {\bbar{A}}
\def \barB {\bbar{B}}
\def \barC {\bbar{C}}
\def \barD {\bbar{D}}
\def \barE {\bbar{E}}
\def \barF {\bbar{F}}
\def \barG {\bbar{G}}
\def \barH {\bbar{H}}
\def \barI {\bbar{I}}
\def \barJ {\bbar{J}}
\def \barK {\bbar{K}}
\def \barL {\bbar{L}}
\def \barM {\bbar{M}}
\def \barN {\bbar{N}}
\def \barO {\bbar{O}}
\def \barP {\bbar{P}}
\def \barQ {\bbar{Q}}
\def \barR {\bbar{R}}
\def \barS {\bbar{S}}
\def \barT {\bbar{T}}
\def \barU {\bbar{U}}
\def \barVV {\bbar{V}}
\def \barW {\bbar{W}}
\def \barX {\bbar{X}}
\def \barY {\bbar{Y}}
\def \barZ {\bbar{Z}}

\def \baralpha {\bbar{\alpha}}
\def \bartau {\bbar{\tau}}
\def \barsigma {\bbar{\sigma}}
\def \barzeta {\bbar{\zeta}}

\def \hata {\hat{a}}
\def \hatb {\hat{b}}
\def \hatc {\hat{c}}
\def \hatd {\hat{d}}
\def \hate {\hat{e}}
\def \hatf {\hat{f}}
\def \hatg {\hat{g}}
\def \hath {\hat{h}}
\def \hati {\hat{i}}
\def \hatj {\hat{j}}
\def \hatk {\hat{k}}
\def \hatl {\hat{l}}
\def \hatm {\hat{m}}
\def \hatn {\hat{n}}
\def \hato {\hat{o}}
\def \hatp {\hat{p}}
\def \hatq {\hat{q}}
\def \hatr {\hat{r}}
\def \hats {\hat{s}}
\def \hatt {\hat{t}}
\def \hatu {\hat{u}}
\def \hatv {\hat{v}}
\def \hatw {\hat{w}}
\def \hatx {\hat{x}}
\def \haty {\hat{y}}
\def \hatz {\hat{z}}
\def \hatA {\hat{A}}
\def \hatB {\hat{B}}
\def \hatC {\hat{C}}
\def \hatD {\hat{D}}
\def \hatE {\hat{E}}
\def \hatF {\hat{F}}
\def \hatG {\hat{G}}
\def \hatH {\hat{H}}
\def \hatI {\hat{I}}
\def \hatJ {\hat{J}}
\def \hatK {\hat{K}}
\def \hatL {\hat{L}}
\def \hatM {\hat{M}}
\def \hatN {\hat{N}}
\def \hatO {\hat{O}}
\def \hatP {\hat{P}}
\def \hatQ {\hat{Q}}
\def \hatR {\hat{R}}
\def \hatS {\hat{S}}
\def \hatT {\hat{T}}
\def \hatU {\hat{U}}
\def \hatVV {\hat{V}}
\def \hatW {\hat{W}}
\def \hatX {\hat{X}}
\def \hatY {\hat{Y}}
\def \hatZ {\hat{Z}}

\def \hatphi {\hat{\phi}}

\def \barfM {\bbar{\fM}}
\def \barfN {\bbar{\fN}}

\def \tila {\tilde{a}}
\def \tilb {\tilde{b}}
\def \tilc {\tilde{c}}
\def \tild {\tilde{d}}
\def \tile {\tilde{e}}
\def \tilf {\tilde{f}}
\def \tilg {\tilde{g}}
\def \tilh {\tilde{h}}
\def \tili {\tilde{i}}
\def \tilj {\tilde{j}}
\def \tilk {\tilde{k}}
\def \till {\tilde{l}}
\def \tilm {\tilde{m}}
\def \tiln {\tilde{n}}
\def \tilo {\tilde{o}}
\def \tilp {\tilde{p}}
\def \tilq {\tilde{q}}
\def \tilr {\tilde{r}}
\def \tils {\tilde{s}}
\def \tilt {\tilde{t}}
\def \tilu {\tilde{u}}
\def \tilv {\tilde{v}}
\def \tilw {\tilde{w}}
\def \tilx {\tilde{x}}
\def \tily {\tilde{y}}
\def \tilz {\tilde{z}}
\def \tilA {\tilde{A}}
\def \tilB {\tilde{B}}
\def \tilC {\tilde{C}}
\def \tilD {\tilde{D}}
\def \tilE {\tilde{E}}
\def \tilF {\tilde{F}}
\def \tilG {\tilde{G}}
\def \tilH {\tilde{H}}
\def \tilI {\tilde{I}}
\def \tilJ {\tilde{J}}
\def \tilK {\tilde{K}}
\def \tilL {\tilde{L}}
\def \tilM {\tilde{M}}
\def \tilN {\tilde{N}}
\def \tilO {\tilde{O}}
\def \tilP {\tilde{P}}
\def \tilQ {\tilde{Q}}
\def \tilR {\tilde{R}}
\def \tilS {\tilde{S}}
\def \tilT {\tilde{T}}
\def \tilU {\tilde{U}}
\def \tilVV {\tilde{V}}
\def \tilW {\tilde{W}}
\def \tilX {\tilde{X}}
\def \tilY {\tilde{Y}}
\def \tilZ {\tilde{Z}}

\def \tilalpha {\tilde{\alpha}}
\def \tilPhi {\tilde{\Phi}}

\def \barnu {\bar{\nu}}
\def \barrho {\bar{\rho}}
%\DeclareMathOperator{\ker}{ker}
\DeclareMathOperator{\im}{im}

\DeclareMathOperator{\Inn}{Inn}
\DeclareMathOperator{\rel}{rel}
\def \dote {\stackrel{\cdot}=}
%\DeclareMathOperator{\AC}{\textbf{AC}}
\DeclareMathOperator{\cod}{cod}
\DeclareMathOperator{\dom}{dom}
\DeclareMathOperator{\card}{card}
\DeclareMathOperator{\ran}{ran}
\DeclareMathOperator{\textd}{d}
\DeclareMathOperator{\td}{d}
\DeclareMathOperator{\id}{id}
\DeclareMathOperator{\LT}{LT}
\DeclareMathOperator{\Mat}{Mat}
\DeclareMathOperator{\Eq}{Eq}
\DeclareMathOperator{\irr}{irr}
\DeclareMathOperator{\Fr}{Fr}
\DeclareMathOperator{\Gal}{Gal}
\DeclareMathOperator{\lcm}{lcm}
\DeclareMathOperator{\alg}{\text{alg}}
\DeclareMathOperator{\Th}{Th}
%\DeclareMathOperator{\deg}{deg}


% \varprod
\DeclareSymbolFont{largesymbolsA}{U}{txexa}{m}{n}
\DeclareMathSymbol{\varprod}{\mathop}{largesymbolsA}{16}
% \DeclareMathSymbol{\tonm}{\boldsymbol{\to}\textbf{Nm}}
\def \tonm {\bto\textbf{Nm}}
\def \tohm {\bto\textbf{Hm}}

% Category theory
\DeclareMathOperator{\ob}{ob}
\DeclareMathOperator{\Ab}{\textbf{Ab}}
\DeclareMathOperator{\Alg}{\textbf{Alg}}
\DeclareMathOperator{\Rng}{\textbf{Rng}}
\DeclareMathOperator{\Sets}{\textbf{Sets}}
\DeclareMathOperator{\Set}{\textbf{Set}}
\DeclareMathOperator{\Grp}{\textbf{Grp}}
\DeclareMathOperator{\Met}{\textbf{Met}}
\DeclareMathOperator{\BA}{\textbf{BA}}
\DeclareMathOperator{\Mon}{\textbf{Mon}}
\DeclareMathOperator{\Top}{\textbf{Top}}
\DeclareMathOperator{\hTop}{\textbf{hTop}}
\DeclareMathOperator{\HTop}{\textbf{HTop}}
\DeclareMathOperator{\Aut}{\text{Aut}}
\DeclareMathOperator{\RMod}{R-\textbf{Mod}}
\DeclareMathOperator{\RAlg}{R-\textbf{Alg}}
\DeclareMathOperator{\LF}{LF}
\DeclareMathOperator{\op}{op}
\DeclareMathOperator{\Rings}{\textbf{Rings}}
\DeclareMathOperator{\Ring}{\textbf{Ring}}
\DeclareMathOperator{\Groups}{\textbf{Groups}}
\DeclareMathOperator{\Group}{\textbf{Group}}
\DeclareMathOperator{\ev}{ev}
% Algebraic Topology
\DeclareMathOperator{\obj}{obj}
\DeclareMathOperator{\Spec}{Spec}
\DeclareMathOperator{\spec}{spec}
% Model theory
\DeclareMathOperator*{\ind}{\raise0.2ex\hbox{\ooalign{\hidewidth$\vert$\hidewidth\cr\raise-0.9ex\hbox{$\smile$}}}}
\def\nind{\cancel{\ind}}
\DeclareMathOperator{\acl}{acl}
\DeclareMathOperator{\tspan}{span}
\DeclareMathOperator{\acleq}{acl^{\eq}}
\DeclareMathOperator{\Av}{Av}
\DeclareMathOperator{\ded}{ded}
\DeclareMathOperator{\EM}{EM}
\DeclareMathOperator{\dcl}{dcl}
\DeclareMathOperator{\Ext}{Ext}
\DeclareMathOperator{\eq}{eq}
\DeclareMathOperator{\ER}{ER}
\DeclareMathOperator{\tp}{tp}
\DeclareMathOperator{\stp}{stp}
\DeclareMathOperator{\qftp}{qftp}
\DeclareMathOperator{\Diag}{Diag}
\DeclareMathOperator{\MD}{MD}
\DeclareMathOperator{\MR}{MR}
\DeclareMathOperator{\RM}{RM}
\DeclareMathOperator{\el}{el}
\DeclareMathOperator{\depth}{depth}
\DeclareMathOperator{\ZFC}{ZFC}
\DeclareMathOperator{\GCH}{GCH}
\DeclareMathOperator{\Inf}{Inf}
\DeclareMathOperator{\Pow}{Pow}
\DeclareMathOperator{\ZF}{ZF}
\DeclareMathOperator{\CH}{CH}
\def \FO {\text{FO}}
\DeclareMathOperator{\fin}{fin}
\DeclareMathOperator{\qr}{qr}
\DeclareMathOperator{\Mod}{Mod}
\DeclareMathOperator{\Def}{Def}
\DeclareMathOperator{\TC}{TC}
\DeclareMathOperator{\KH}{KH}
\DeclareMathOperator{\Part}{Part}
\DeclareMathOperator{\Infset}{\textsf{Infset}}
\DeclareMathOperator{\DLO}{\textsf{DLO}}
\DeclareMathOperator{\PA}{\textsf{PA}}
\DeclareMathOperator{\DAG}{\textsf{DAG}}
\DeclareMathOperator{\ODAG}{\textsf{ODAG}}
\DeclareMathOperator{\sfMod}{\textsf{Mod}}
\DeclareMathOperator{\AbG}{\textsf{AbG}}
\DeclareMathOperator{\sfACF}{\textsf{ACF}}
\DeclareMathOperator{\DCF}{\textsf{DCF}}
% Computability Theorem
\DeclareMathOperator{\Tot}{Tot}
\DeclareMathOperator{\graph}{graph}
\DeclareMathOperator{\Fin}{Fin}
\DeclareMathOperator{\Cof}{Cof}
\DeclareMathOperator{\lh}{lh}
% Commutative Algebra
\DeclareMathOperator{\ord}{ord}
\DeclareMathOperator{\Idem}{Idem}
\DeclareMathOperator{\zdiv}{z.div}
\DeclareMathOperator{\Frac}{Frac}
\DeclareMathOperator{\rad}{rad}
\DeclareMathOperator{\nil}{nil}
\DeclareMathOperator{\Ann}{Ann}
\DeclareMathOperator{\End}{End}
\DeclareMathOperator{\coim}{coim}
\DeclareMathOperator{\coker}{coker}
\DeclareMathOperator{\Bil}{Bil}
\DeclareMathOperator{\Tril}{Tril}
\DeclareMathOperator{\tchar}{char}
\DeclareMathOperator{\tbd}{bd}

% Topology
\DeclareMathOperator{\diam}{diam}
\newcommand{\interior}[1]{%
  {\kern0pt#1}^{\mathrm{o}}%
}

\DeclareMathOperator*{\bigdoublewedge}{\bigwedge\mkern-15mu\bigwedge}
\DeclareMathOperator*{\bigdoublevee}{\bigvee\mkern-15mu\bigvee}

% \makeatletter
% \newcommand{\vect}[1]{%
%   \vbox{\m@th \ialign {##\crcr
%   \vectfill\crcr\noalign{\kern-\p@ \nointerlineskip}
%   $\hfil\displaystyle{#1}\hfil$\crcr}}}
% \def\vectfill{%
%   $\m@th\smash-\mkern-7mu%
%   \cleaders\hbox{$\mkern-2mu\smash-\mkern-2mu$}\hfill
%   \mkern-7mu\raisebox{-3.81pt}[\p@][\p@]{$\mathord\mathchar"017E$}$}

% \newcommand{\amsvect}{%
%   \mathpalette {\overarrow@\vectfill@}}
% \def\vectfill@{\arrowfill@\relbar\relbar{\raisebox{-3.81pt}[\p@][\p@]{$\mathord\mathchar"017E$}}}

% \newcommand{\amsvectb}{%
% \newcommand{\vect}{%
%   \mathpalette {\overarrow@\vectfillb@}}
% \newcommand{\vecbar}{%
%   \scalebox{0.8}{$\relbar$}}
% \def\vectfillb@{\arrowfill@\vecbar\vecbar{\raisebox{-4.35pt}[\p@][\p@]{$\mathord\mathchar"017E$}}}
% \makeatother
% \bigtimes

\DeclareFontFamily{U}{mathx}{\hyphenchar\font45}
\DeclareFontShape{U}{mathx}{m}{n}{
      <5> <6> <7> <8> <9> <10>
      <10.95> <12> <14.4> <17.28> <20.74> <24.88>
      mathx10
      }{}
\DeclareSymbolFont{mathx}{U}{mathx}{m}{n}
\DeclareMathSymbol{\bigtimes}{1}{mathx}{"91}
% \odiv
\DeclareFontFamily{U}{matha}{\hyphenchar\font45}
\DeclareFontShape{U}{matha}{m}{n}{
      <5> <6> <7> <8> <9> <10> gen * matha
      <10.95> matha10 <12> <14.4> <17.28> <20.74> <24.88> matha12
      }{}
\DeclareSymbolFont{matha}{U}{matha}{m}{n}
\DeclareMathSymbol{\odiv}         {2}{matha}{"63}


\newcommand\subsetsim{\mathrel{%
  \ooalign{\raise0.2ex\hbox{\scalebox{0.9}{$\subset$}}\cr\hidewidth\raise-0.85ex\hbox{\scalebox{0.9}{$\sim$}}\hidewidth\cr}}}
\newcommand\simsubset{\mathrel{%
  \ooalign{\raise-0.2ex\hbox{\scalebox{0.9}{$\subset$}}\cr\hidewidth\raise0.75ex\hbox{\scalebox{0.9}{$\sim$}}\hidewidth\cr}}}

\newcommand\simsubsetsim{\mathrel{%
  \ooalign{\raise0ex\hbox{\scalebox{0.8}{$\subset$}}\cr\hidewidth\raise1ex\hbox{\scalebox{0.75}{$\sim$}}\hidewidth\cr\raise-0.95ex\hbox{\scalebox{0.8}{$\sim$}}\cr\hidewidth}}}
\newcommand{\stcomp}[1]{{#1}^{\mathsf{c}}}

\setlength{\baselineskip}{0.5in}

\stackMath
\newcommand\yrightarrow[2][]{\mathrel{%
  \setbox2=\hbox{\stackon{\scriptstyle#1}{\scriptstyle#2}}%
  \stackunder[0pt]{%
    \xrightarrow{\makebox[\dimexpr\wd2\relax]{$\scriptstyle#2$}}%
  }{%
   \scriptstyle#1\,%
  }%
}}
\newcommand\yleftarrow[2][]{\mathrel{%
  \setbox2=\hbox{\stackon{\scriptstyle#1}{\scriptstyle#2}}%
  \stackunder[0pt]{%
    \xleftarrow{\makebox[\dimexpr\wd2\relax]{$\scriptstyle#2$}}%
  }{%
   \scriptstyle#1\,%
  }%
}}
\newcommand\yRightarrow[2][]{\mathrel{%
  \setbox2=\hbox{\stackon{\scriptstyle#1}{\scriptstyle#2}}%
  \stackunder[0pt]{%
    \xRightarrow{\makebox[\dimexpr\wd2\relax]{$\scriptstyle#2$}}%
  }{%
   \scriptstyle#1\,%
  }%
}}
\newcommand\yLeftarrow[2][]{\mathrel{%
  \setbox2=\hbox{\stackon{\scriptstyle#1}{\scriptstyle#2}}%
  \stackunder[0pt]{%
    \xLeftarrow{\makebox[\dimexpr\wd2\relax]{$\scriptstyle#2$}}%
  }{%
   \scriptstyle#1\,%
  }%
}}

\newcommand\altxrightarrow[2][0pt]{\mathrel{\ensurestackMath{\stackengine%
  {\dimexpr#1-7.5pt}{\xrightarrow{\phantom{#2}}}{\scriptstyle\!#2\,}%
  {O}{c}{F}{F}{S}}}}
\newcommand\altxleftarrow[2][0pt]{\mathrel{\ensurestackMath{\stackengine%
  {\dimexpr#1-7.5pt}{\xleftarrow{\phantom{#2}}}{\scriptstyle\!#2\,}%
  {O}{c}{F}{F}{S}}}}

\newenvironment{bsm}{% % short for 'bracketed small matrix'
  \left[ \begin{smallmatrix} }{%
  \end{smallmatrix} \right]}

\newenvironment{psm}{% % short for ' small matrix'
  \left( \begin{smallmatrix} }{%
  \end{smallmatrix} \right)}

\newcommand{\bbar}[1]{\mkern 1.5mu\overline{\mkern-1.5mu#1\mkern-1.5mu}\mkern 1.5mu}

\newcommand{\bigzero}{\mbox{\normalfont\Large\bfseries 0}}
\newcommand{\rvline}{\hspace*{-\arraycolsep}\vline\hspace*{-\arraycolsep}}

\font\zallman=Zallman at 40pt
\font\elzevier=Elzevier at 40pt

\newcommand\isoto{\stackrel{\textstyle\sim}{\smash{\longrightarrow}\rule{0pt}{0.4ex}}}
\newcommand\embto{\stackrel{\textstyle\prec}{\smash{\longrightarrow}\rule{0pt}{0.4ex}}}

% from http://www.actual.world/resources/tex/doc/TikZ.pdf

\tikzset{
modal/.style={>=stealth’,shorten >=1pt,shorten <=1pt,auto,node distance=1.5cm,
semithick},
world/.style={circle,draw,minimum size=0.5cm,fill=gray!15},
point/.style={circle,draw,inner sep=0.5mm,fill=black},
reflexive above/.style={->,loop,looseness=7,in=120,out=60},
reflexive below/.style={->,loop,looseness=7,in=240,out=300},
reflexive left/.style={->,loop,looseness=7,in=150,out=210},
reflexive right/.style={->,loop,looseness=7,in=30,out=330}
}


\makeatletter
\newcommand*{\doublerightarrow}[2]{\mathrel{
  \settowidth{\@tempdima}{$\scriptstyle#1$}
  \settowidth{\@tempdimb}{$\scriptstyle#2$}
  \ifdim\@tempdimb>\@tempdima \@tempdima=\@tempdimb\fi
  \mathop{\vcenter{
    \offinterlineskip\ialign{\hbox to\dimexpr\@tempdima+1em{##}\cr
    \rightarrowfill\cr\noalign{\kern.5ex}
    \rightarrowfill\cr}}}\limits^{\!#1}_{\!#2}}}
\newcommand*{\triplerightarrow}[1]{\mathrel{
  \settowidth{\@tempdima}{$\scriptstyle#1$}
  \mathop{\vcenter{
    \offinterlineskip\ialign{\hbox to\dimexpr\@tempdima+1em{##}\cr
    \rightarrowfill\cr\noalign{\kern.5ex}
    \rightarrowfill\cr\noalign{\kern.5ex}
    \rightarrowfill\cr}}}\limits^{\!#1}}}
\makeatother

% $A\doublerightarrow{a}{bcdefgh}B$

% $A\triplerightarrow{d_0,d_1,d_2}B$

\def \uhr {\upharpoonright}
\def \rhu {\rightharpoonup}
\def \uhl {\upharpoonleft}


\newcommand{\floor}[1]{\lfloor #1 \rfloor}
\newcommand{\ceil}[1]{\lceil #1 \rceil}
\newcommand{\lcorner}[1]{\llcorner #1 \lrcorner}
\newcommand{\llb}[1]{\llbracket #1 \rrbracket}
\newcommand{\ucorner}[1]{\ulcorner #1 \urcorner}
\newcommand{\emoji}[1]{{\DejaSans #1}}
\newcommand{\vprec}{\rotatebox[origin=c]{-90}{$\prec$}}

\newcommand{\nat}[6][large]{%
  \begin{tikzcd}[ampersand replacement = \&, column sep=#1]
    #2\ar[bend left=40,""{name=U}]{r}{#4}\ar[bend right=40,',""{name=D}]{r}{#5}\& #3
          \ar[shorten <=10pt,shorten >=10pt,Rightarrow,from=U,to=D]{d}{~#6}
    \end{tikzcd}
}


\providecommand\rightarrowRHD{\relbar\joinrel\mathrel\RHD}
\providecommand\rightarrowrhd{\relbar\joinrel\mathrel\rhd}
\providecommand\longrightarrowRHD{\relbar\joinrel\relbar\joinrel\mathrel\RHD}
\providecommand\longrightarrowrhd{\relbar\joinrel\relbar\joinrel\mathrel\rhd}
\def \lrarhd {\longrightarrowrhd}


\makeatletter
\providecommand*\xrightarrowRHD[2][]{\ext@arrow 0055{\arrowfill@\relbar\relbar\longrightarrowRHD}{#1}{#2}}
\providecommand*\xrightarrowrhd[2][]{\ext@arrow 0055{\arrowfill@\relbar\relbar\longrightarrowrhd}{#1}{#2}}
\makeatother

\newcommand{\metalambda}{%
  \mathop{%
    \rlap{$\lambda$}%
    \mkern3mu
    \raisebox{0ex}{$\lambda$}%
  }%
}

%% https://tex.stackexchange.com/questions/15119/draw-horizontal-line-left-and-right-of-some-text-a-single-line
\newcommand*\ruleline[1]{\par\noindent\raisebox{.8ex}{\makebox[\linewidth]{\hrulefill\hspace{1ex}\raisebox{-.8ex}{#1}\hspace{1ex}\hrulefill}}}

% https://www.dickimaw-books.com/latex/novices/html/newenv.html
\newenvironment{Block}[1]% environment name
{% begin code
  % https://tex.stackexchange.com/questions/19579/horizontal-line-spanning-the-entire-document-in-latex
  \noindent\textcolor[RGB]{128,128,128}{\rule{\linewidth}{1pt}}
  \par\noindent
  {\Large\textbf{#1}}%
  \bigskip\par\noindent\ignorespaces
}%
{% end code
  \par\noindent
  \textcolor[RGB]{128,128,128}{\rule{\linewidth}{1pt}}
  \ignorespacesafterend
}

\mathchardef\mhyphen="2D % Define a "math hyphen"

\def \QQ {\quad}
\def \QW {​\quad}

\def \Map {\operatorname{Map}}
\def \ev {\text{ev}}
\def \Mor {\text{Mor}}
\def \ord {\operatorname{ord}}
\def \irr {\operatorname{irr}}
\author{Serge Lang}
\date{\today}
\title{Algebra}
\hypersetup{
 pdfauthor={Serge Lang},
 pdftitle={Algebra},
 pdfkeywords={},
 pdfsubject={},
 pdfcreator={Emacs 27.2 (Org mode 9.5)}, 
 pdflang={English}}
\begin{document}

\maketitle
\tableofcontents

\section{Groups}
\label{sec:org189c23d}
\subsection{Monoids}
\label{sec:org5ff5aea}
\subsection{Groups}
\label{sec:orgd8d508f}
\subsection{Normal Subgroups}
\label{sec:orgd47e9fd}
Let \(f:G\to G'\) be a group homomorphism, and let \(H\) be its kernel. If \(x\) is an element
of \(G\), then \(xH=Hx\), because both are equal to \(f^{-1}(f(x))\). We can also rewrite this
relation as \(xHx^{-1}=H\)

Conversely, let \(G\) be a group and let \(H\) be a subgroup. Assume that for all
elements \(x\in G\), we have \(xH\subset Hx\) (or equivalently, \(xHx^{-1}\subset H\)), which
implies \(H\subset xHx^{-1}\). Thus our condition is equivalent to the condition \(xHx^{-1}=H\) for
all \(x\in G\). A subgroup \(H\) satisfying this condition will be called \textbf{normal}

Let \(G'\) be the set of cosets of \(H\). (A left coset is equal to a right coset). If \(xH\)
and \(yH\) are cosets, then their product
\begin{equation*}
xHyH=xyHH=xyH
\end{equation*}
is also a coset. Hence \(G'\) is a group.

Let \(f:G\to G'\) be the mapping s.t. \(f(x)\) is the coset \(xH\). Then \(f\) is clearly a
homomorphism and \(H\) is equal to the kernel.

The group of cosets of a normal subgroup \(H\) is denoted by \(G/H\) (which we read \(G\)
modulo \(H\), or \(G\) mod \(H\)). The map \(f\) of \(G\) onto \(G/H\) constructed above is
called the \textbf{canonical map}, and \(G/H\) is called the \textbf{factor group} of \(G\) by \(H\)
\subsection{Direct Sums and Free Abelian Groups}
\label{sec:orgdd4d840}
Let \(\{A_i\}_{i\in I}\) be a family of abelian groups. We define their \textbf{direct sum}
\begin{equation*}
A=\bigoplus_{i\in I}A_i
\end{equation*}
to be the subset of the direct product \(\prod A_i\) consisting of all families \((x_i)_{i\in I}\)
with \(x_i\in A_i\) s.t. \(x_i=0\) for all but a finite number of indices \(i\). For each
index \(j\in I\), we map
\begin{equation*}
\lambda_j:A_j\to A
\end{equation*}
by letting \(\lambda_j(x)\) be the element whose \(j\)-th component is \(x\), and having all other
components equal to 0. Then \(\lambda_j\) is an injective homomorphism

\begin{proposition}[]
\label{prop7.1}
Let \(\{f_i:A_i\to B\}\) be a family of homomorphisms into an abelian group \(B\). Let \(A=\bigoplus A_i\).
There exists a unique homomorphism
\begin{equation*}
f:A\to B
\end{equation*}
s.t. \(f\circ\lambda_j=f_j\) for all \(j\)
\end{proposition}

\begin{proof}
Define
\begin{equation*}
f((x_i)_{i\in I})=\sum_{i\in I}f_i(x_i)
\end{equation*}
\end{proof}

The property in Proposition \ref{prop7.1} is called the \textbf{universal property} of the direct sum.

Let \(A\) be an abelian group and \(B,C\) subgroups. If \(B+C=A\) and \(B\cap C=\{0\}\) then the map
\begin{equation*}
B\times C\to A
\end{equation*}
given by \((x,y)\mapsto x+y\) is an isomorphism. Instead of writing \(A=B\times C\) we shall
write \(A=B\oplus C\) and say that \(A\) is the \textbf{direct sum} of \(B\) and \(C\). We sue a similar
notation for the direct sum of a finite number of subgroups \(B_1,\dots,B_n\) s.t.
\begin{equation*}
B_1+\dots+B_n=A
\end{equation*}
and
\begin{equation*}
B_{i+1}\cap(B_1+\dots+B_i)=0
\end{equation*}
In that case, we write
\begin{equation*}
A=B_1\oplus B_2\oplus\dots\oplus B_n
\end{equation*}
Let \(A\) be an abelian group. Let \(\{e_i\}_{i\in I}\) be a family of elements of \(A\). We say that
this family is a \textbf{basis} of \(A\) if the family is not empty, and if every element of \(A\) has a
unique expression as a linear combination
\begin{equation*}
x=\sum x_ie_i
\end{equation*}
with \(x_i\in\Z\) and almost all \(x_i=0\). Thus the sum is actually a finite sum. An abelian group is
\textbf{free} if it has a basis. If that is the case, then if we let \(Z_i=\Z\) for all \(i\), then \(A\) is
isomorphic to the direct sum
\begin{equation*}
A\cong\bigoplus_{i\in I}Z_i
\end{equation*}
Now let \(S\) be a set. Let \(\Z\la S\ra\) be the set of all maps \(\varphi:S\to\Z\) s.t. \(\varphi(x)=0\) for
almost all \(x\in S\). Then \(\Z\la S\ra\) is an abelian group. if \(k\) is an integer and \(x\in S\),
we denote by \(k\cdot x\) the map \(\varphi\) s.t. \(\varphi(x)=k\) and \(\varphi(y)=0\) if \(y\neq x\). Then every element
\(\varphi\) of \(\Z\la S\ra\) can be written in the form
\begin{equation*}
\varphi=k_1\cdot x_1+\dots+k_n\cdot x_n
\end{equation*}
for \(k_i\in\Z\) and \(x_i\in S\), all the \(x_i\) being distinct. Furthermore, \(\varphi\)
\textbf{admits a unique such expression}, because if we have
\begin{equation*}
\varphi=\sum_{x\in S}k_x\cdot x=\sum_{x\in S}k_x'\cdot x
\end{equation*}
then
\begin{equation*}
0=\sum_{x\in S}(k_x-k'_x)\cdot x
\end{equation*}
whence \(k_x'=k_x\) for all \(x\in S\)

We map \(S\) into \(\Z\la S\ra\) by the map \(f_S=f\) s.t. \(f(x)=1\cdot x\). \(f(S)\)
generates \(\Z\la S\ra\). If \(g:S\to B\) is a mapping of \(S\) into some abelian group \(B\), then
we define a map
\begin{equation*}
g_*:\Z\la S\ra\to B
\end{equation*}
s.t.
\begin{equation*}
g_*\left( \sum_{x\in S}k_x\cdot x \right)=\sum_{x\in S}k_xg(x)
\end{equation*}
It's unique for any such homomorphism \(g_*\) must be s.t. \(g_*(1\cdot x)=g(x)\)
\begin{proposition}[]
if \(\lambda:S\to S'\) is a mapping of sets, there is a unique homomorphism \(\bar{\lambda}\) making the
following diagram commutative
\begin{center}\begin{tikzcd}
S\ar[r,"f_S"]\ar[d,"\lambda"]&\Z\la S\ra\ar[d,"\bar{\lambda}"]\\
S'\ar[r,"f_{S'}"']&\Z\la S'\ra
\end{tikzcd}\end{center}
In fact, \(\bar{\lambda}\) is none other than \((f_S\circ\lambda)_*\)
\end{proposition}

We shall denote \(\Z\la S\ra\) also \(F_{ab}(S)\) and call \(F_{ab}(S)\) the \textbf{free abelian group}
\textbf{generated by} \(S\). We call elements of \(S\) its \textbf{free generators}


\section{Rings}
\label{sec:org1770ed0}
\subsection{Rings and Homomorphisms}
\label{sec:org3af7b52}
A \textbf{ring} \(A\) is a set
\begin{enumerate}
\item w.r.t. addition, \(A\) is a commutative group
\item the multiplication is associative, and has a unit element
\item for all \(x,y,z\in A\) we have
\begin{equation*}
(x+y)z=xz+yz \quad\text{ and }\quad z(x+y)=zx+zy
\end{equation*}
\end{enumerate}
(called \textbf{distributivity})

We denote the unit element for addition by 0, and the unit element for multiplication by 1.
Observe that \(0x=0\) for all \(x\in A\). \emph{Proof:} \(0x+x=(0+1)x=x\)

For any \(x,y\in A\) we have \((-x)y=-(xy)\)

Let \(A\) be a ring, and let \(U\) be the set of elements of \(A\) which have both a right and
left inverse. Then \(U\) is a multiplicative group. Indeed, if \(a\) has a right inverse \(b\),
so that \(ab=1\), and a left inverse \(c\), so that \(ca=1\), then \(cab=b\), whence \(c=b\), and
we see that \(c\) is a two-sided inverse, and that \(c\) itself has a two-sided inverse,
namely \(a\). Therefore \(U\) satisfies all the axioms of a multiplicative group, and is called
the group of \textbf{units} of \(A\). It is sometimes denoted by \(A^*\), and is also called the group of
\textbf{invertible} elements of \(A\). A ring \(A\) s.t. \(1\neq 0\) and s.t. every non-zero element is
invertible is called a \textbf{division ring}.

\begin{examplle}[The Shift Operator]
Let \(E\) be the set of all sequences
\begin{equation*}
a=(a_1,a_2,a_3,\dots`)
\end{equation*}
of integers. One can define addition componentwise. Let \(R\) be the set of all
mappings \(f:E\to E\) of \(E\) into itself s.t. \(f(a+b)=f(a)+f(b)\). Then \(R\) is a ring. Let
\begin{equation*}
T(a_1,a_2,a_3,\dots)=(0,a_1,a_2,a_3,\dots)
\end{equation*}
Verify that \(T\) is left invertible but not right invertible
\end{examplle}

A ring \(A\) is said to be \textbf{commutative} if \(xy=yx\) for all \(x,y\in A\). A commutative division
ring is called a \textbf{field}. By definition, a field contains at least two elements, namely 0 and 1.

   A subset \(B\) of ring \(A\) is called a \textbf{subring} if it is an additive subgroup, if it contains
   the multiplicative unit, and if \(x,y\in B\) implies \(xy\in B\). If that is the case, then \(B\) is
n   itself a ring, the laws of operation in \(B\) being the same as the laws of operation in \(A\)

For example, the \textbf{center} of a ring \(A\) is the subset of \(A\) consisting of all
elements \(a\in A\) s.t. \(ax=xa\) for all \(x\in A\). The center of \(A\) is a subring.

If \(x,y_1,\dots,y_n\) are elements of a ring, then by induction one sees that
\begin{equation*}
x(y_1+\dots+y_n)=xy_1+\dots+xy_n
\end{equation*}
If \(x_i(i=1,\dots,n)\) and \(y_j(j=1,\dots,m)\) are elements of \(A\), then it is also easily proved that
\begin{equation*}
\left( \sum_{i=1}^nx_i \right)\left( \sum_{j=1}^my_j \right)=
\sum_{i=1}^n\sum_{j=1}^mx_iy_j
\end{equation*}
Furthermore, distributivity holds for subtraction, e.g.
\begin{equation*}
x(y_1-y_2)=xy_1-xy_2
\end{equation*}

\begin{examplle}[]
Let \(S\) be a set and \(A\) a ring. Let \(\Map(S,A)\) be the set of mappings of \(S\)
into \(A\). Then \(\Map(S,A)\) is a ring if for \(f,g\in\Map(S,A)\) we define
\begin{equation*}
(fg)(x)=f(x)g(x)\quad\text{ and }\quad (f+g)(x)=f(x)+g(x)
\end{equation*}
for all \(x\in S\).

Let \(M\) be an additive abelian group, and let \(A\) be the set \(\End(M)\) of
group-homomorphisms of \(M\) into itself. We define addition in \(A\) to be the addition of
mappings, and we define multiplication to be \textbf{composition} of mappings
\end{examplle}

\begin{examplle}[The convolution product]
Let \(G\) be a group and let \(K\) be a field. Denote by \(K[G]\) the set of all formal linear
combinations \(\alpha=\sum a_xx\) with \(x\in G\) and \(a_x\in K\), s.t. all but finite number of \(a_x\) are
equal to 0. If \(\beta=\sum b_xx\in K[G]\), then one can define the product
\begin{equation*}
\alpha\beta=\sum_{x\in G}\sum_{y\in G}a_xb_yxy=\sum_{z\in G}\left( \sum_{xy=z}a_xb_y \right)z
\end{equation*}
With this product, the \textbf{group ring} \(K[G]\) is a ring. \(K[G]\) is commutative iff \(G\) is
commutative. The second sum on the right  defines what is called a \textbf{convolution product}.
If \(f,g\) are functions on a group \(G\), we define their \textbf{convolution} \(f*g\) by
\begin{equation*}
(f*g)(z)=\sum_{xy=z}f(x)g(y)
\end{equation*}
\end{examplle}

A \textbf{left ideal} \(\fa\) in a ring \(A\) is a subset of \(A\) which is a subgroup of the additive group
of \(A\), s.t. \(A\fa\subset\fa\) (and hence \(A\fa=\fa\) since \(A\) contains 1). To define a right ideal, we
quire \(\fa A=\fa\), and a \textbf{two-sided ideal} is a subset which is both a left and right ideal. A
two-sided ideal is called an \textbf{ideal} in this section.

If \(A\) is a ring and \(a\in A\), then \(Aa\) is a left ideal, called \textbf{principal}. We say that \(a\)
is a generator of \(\fa\) (over \(A\)). \(AaA\) is a principal two-sided ideal
if \(AaA=\{\sum x_iay_i\mid x_i,y_i\in A\}\). More generally, let \(a_1,\dots,a_n\in A\). We denote by \((a_1,\dots,a_n)\)
the set of elements of \(A\) which can be written in the form
\begin{equation*}
x_1a_1+\dots+x_na_n\quad\text{with}\quad x_i\in A
\end{equation*}
Then this set of elements is immediately verified to be a left ideal, and \(a_1,\dots,a_n\) are called
\textbf{generators} of the left ideal.

If \(\{\fa_i\}_{i\in I}\) is a family of ideals, then their intersection
\begin{equation*}
\bigcap_{i\in I}\fa_i
\end{equation*}
is also an ideal

A \textbf{commutative} ring s.t. every ideal is principal and s.t. \(1\neq 0\) is called a \textbf{principal} ring

\begin{examplle}[]
The integers \(\Z\) form a ring, which is commutative. Let \(\fa\) be an ideal \(\neq\Z\) and \(\neq 0\).
If \(n\in\fa\) then \(-n\in\fa\). Let \(d\) be the smallest integer \(>0\) lying in \(\fa\). If \(n\in\fa\)
then there exists integers \(q,r\) with \(0\le r<d\) s.t.
\begin{equation*}
n=dq+r
\end{equation*}
Since \(\fa\) is an ideal, it follows that \(r\) lies in \(\fa\), hence \(r=0\). Hence \(\fa\) consists
of all multiples \(qd\) of \(d\), which \(q\in\Z\), and \(\Z\) is a principal ring.
\end{examplle}

Let \(\fa,\fb\) be ideals of \(A\). We define \(\fa\fb\) to be the set of all sums
\begin{equation*}
x_1y_1+\dots+x_ny_n
\end{equation*}
with \(x_i\in\fa\) and \(y_i\in\fb\). \(\fa\fb\) is an ideal, and that the set of ideals forms a multiplicative
monoid, the unit element being the ring itself. This unit element is called the \textbf{unit ideal} and is
often written (1).

If \(\fa,\fb\) are left ideals of \(A\), then \(\fa+\fb\) (the sum being taken as additive subgroup
of \(A\)) is obviously a left ideal. Thus ideals also form a monoid under addition. We also have
distributivity: if \(\fa_1,\dots,\fa_n,\fb\) are ideals of \(A\), then
\begin{equation*}
\fb(\fa_1+\dots+\fa_n)=\fb\fa_1+\dots+\fb\fa_n
\end{equation*}

Let \(\fa\) be a left ideal. Define \(\fa A\) to be the set of all sums \(a_1x_1+\dots+a_nx_n\)
with \(a_i\in\fa\) and \(x_i\in A\). Then \(\fa A\) is an ideal.

Suppose that \(A\) is commutative. Let \(\fa,\fb\) be ideals. Then trivially
\begin{equation*}
\fa\fb\subset\fa\cap\fb
\end{equation*}
If \(\fa+\fb=A\) then \(\fa\fb=\fa\cap \fb\).  Suppose \(x\in\fa\cap\fb\) and \(x=a_x+b_x\), where \(a_x\in\fa\) and \(b_x\in\fb\).
Then \(a_x\in\fb\) and \(b_x\in\fa\). If \(1=a_1+b_1\) then \(x\cdot 1=(a_x+b_x)(a_1+b_1)\in\fa\fb\)

\index{ring homomorphism}
By a \textbf{ring homomorphism} one means a mapping \(f:A\to B\) where \(A,B\) are rings, and s.t. \(f\) is
a monoid-homomorphism for the multiplicative structures on \(A\) and \(B\), and also a monoid
homomorphism for the additive structure. In other words
\begin{alignat*}{2}
&f(a+a')=f(a)+f(a')\quad&&f(aa')=f(a)f(a')\\
&f(1)=1&&f(0)=0
\end{alignat*}
for all \(a,a'\in A\).

The kernel of a ring homomorphism \(f:A\to B\) is an ideal of \(A\).

Conversely, let \(\fa\) be an ideal of the ring \(A\). We can construct the \textbf{factor ring} \(A/\fa\) as
follows. Viewing \(A\) and \(\fa\) as additive groups, let \(A/\fa\) be the factor group. If \(x+\fa\)
and \(y+\fa\) are two cosets of \(\fa\), we define \((x+\fa)(y+\fa)\) to be the coset \(xy+\fa\). This
coset is well-defined, for if \(x_1,y_1\) are in the same coset as \(x,y\) respectively, then one
verifies that \(x_1y_1\) is in the same coset as \(xy\). Unit element is \(1+\fa\).

We therefore defined a ring structure on \(A/\fa\) and the caonical map
\begin{equation*}
f:A\to A/\fa
\end{equation*}
is then clearly a ring homomorphism

\begin{proposition}[]
If \(g:A\to A'\) is a ring homomorphism whose kernel contains \(\fa\), then there exists a unique
ring homomorphism \(g_*:A/\fa\to A'\) making the following diagram commutative
\begin{center}\begin{tikzcd}[column sep=small]
A\ar[rr,"g"]\ar[rd,"f"']&&A'\\
&A/\fa\ar[ur,"g_*"']
\end{tikzcd}\end{center}
\end{proposition}

Indeed, viewing \(f,g\) as group homomorphisms, there is a unique group homomorphism \(g_*\)
making our diagram commutative

\begin{proof}
If \(x\in A\) then \(g(x)=g_*f(x)\). Hence for \(x,y\in A\)
\begin{align*}
g_*(f(x)f(y))&=g_*(f(xy))=g(xy)=g(x)g(y)\\
&=g_*f(x)g_*f(y)
\end{align*}
Given \(\xi,\eta\in A/\fa\), there exists \(x,y\in A\) s.t. \(f(x)=\xi\) and \(f(y)=\eta\). Since \(f(1)=1\), we
get \(g_*f(1)=g(1)=1\) and hence the two conditions that \(g_*\) be a multiplicative
monoid-homomorphism are satisfied
\end{proof}

Let \(A\) be a ring, and denote its unit element by \(e\) for the moment. The map
\begin{equation*}
\lambda:\Z\to A
\end{equation*}
s.t. \(\lambda(n)=ne\) is a ring homomorphism, and its kernel is an ideal \((n)\), generated by an
integer \(n\ge 0\). We have a canonical injective homomorphism \(\Z/n\Z\to A\) which is a (ring)
isomorphism between \(\Z/n\Z\) and a subring of \(A\). If \(n\Z\) is a prime ideal, then \(n=0\)
or \(n=p\) for some prime number \(p\). In the first place, \(A\) contains as a subring a ring
which is isomorphic to \(\Z\), and which is often identified with \(\Z\). In that case, we say
that \(A\) has \textbf{characteristic} 0. if on the other hand \(n=p\) then we say that \(A\) has
\textbf{characteristic} \(p\), and \(A\) contains (an isomorphic image of) \(\Z/p\Z\) as a subring. We
abbreviate \(\Z/p\Z\) by \(\F_p\).

If \(K\) is a field, then \(K\) has characteristic 0 or \(p>0\). (if its characteristic
is \(a\cdot b\), then \(a\cdot b\cdot 1=0\) but field is an integral domain). In the first case, \(K\)
contains as a subfield an isomorphic image of the rational numbers, and in the second case, it
contains an isomorphic image of \(\F_p\). In either case, this subfield will be called the \textbf{prime
field} (contained in \(K\)). Since this prime field is the smallest subfield of \(K\) containing 1
and has no automorphism except the identity, it is customary to identiy it with \(\Q\) or \(\F_p\)
as the case may be. By the \textbf{prime ring} (in \(K\)) we shall mean either the integers \(\Z\) if \(K\)
has characteristic 0 or \(\F_p\) if \(K\) has characteristic \(p\). \label{Problem1}

Let \(A\) be a subring of a ring \(B\). Let \(S\) be a subset of \(B\) commuting with \(A\). We
denote by \(A[S]\) the set of all elements
\begin{equation*}
\sum a_{i_1\dots i_n}s_1^{i_1}\dots s_n^{i_n}
\end{equation*}
the sum ranging over a finite number of \(n\)-tuples \((i_1,\dots,i_n)\) of integers \(\ge 0\),
and \(a_{i_1,\dots,i_n}\in A\), \(s_1,\dots,s_n\in S\). If \(B=A[S]\) , we say that \(S\) is a set of
\textbf{generators} (or \textbf{ring generators}) for \(B\) over \(A\), or that \(B\) is \textbf{generated} by \(S\)
over \(A\). If \(S\) is finite, \(B\) is \textbf{finitely generated as a ring over} \(A\). Note that \(S\)
is not commutative.

Let \(A\) be a ring, \(\fa\) an ideal, and \(S\) a subset of \(A\). We write
\begin{equation*}
S\equiv 0\mod \fa
\end{equation*}
if \(S\subset\fa\). If \(x,y\in A\) we write
\begin{equation*}
x\equiv y\mod\fa
\end{equation*}
if \(x-a\in\fa\).  If \(\fa\) is principal, equal to \((a)\), then we also write
\begin{equation*}
x\equiv y\mod a
\end{equation*}
If \(f:A\to A/\fa\) is the canonical homomorphism, then \(x\equiv y\mod\fa\) means that \(f(x)=f(y)\)

The factor ring \(A/\fa\) is also called a \textbf{residue class ring}. Cosets of \(\fa\) in \(A\) are called
\textbf{residue classes} modulo \(\fa\), and if \(x\in A\), then the coset \(x+\fa\) is called the \textbf{residue class}
\textbf{of \(x\) modulo \(\fa\)}

An injective ring homomorphism \(f:A\to B\) establishes a ring isomorphism between \(A\) and its
image. Such a homomorphism will be called an \textbf{embedding}

Let \(f:A\to A'\) be a ring homomorphism, and let \(\fa'\) be an ideal of \(A'\). Then \(f^{-1}(a')\)
is an ideal \(\fa\) in \(A\), and we have an induced injective homomorphism
\begin{equation*}
A/\fa\to A'/\fa'
\end{equation*}

\begin{proposition}[]
Products exist in the category of rings
\end{proposition}

Let \(A\) be a ring. Elements \(x,y\in A\) are said to be \textbf{zero divisors} if \(x\neq 0\), \(y\neq 0\)
and \(xy=0\). A ring \(A\) is \textbf{entire} if \(1\neq 0\), if \(A\) is commutative and if there are no
zero divisors in the ring. (Entire rings are also called \textbf{integral domains})

Let \(m\) be a positive integer \(\neq 1\). The ring \(\Z/m\Z\) has zero divisors iff \(m\) is not
prime.

\begin{proposition}[]
Let \(A\) be an entire ring, and let \(a,b\) be non-zero elements of \(A\). Then \(a,b\) generate
the same ideal iff there exists a unit \(u\) of \(A\) s.t. \(b=au\).
\end{proposition}

\begin{proof}
Assume \(Aa=Ab\). Then \(a=bc\) and \(b=ad\) for some \(c,d\in A\). Hence \(a=adc\)
whence \(a(1-dc)=0\) and therefore \(dc=1\). Hence \(c\) is a unit
\end{proof}

\subsection{Commutative Rings}
\label{sec:org41c3a4e}
Assume \(A\) is commutative

A \textbf{prime} ideal in \(A\) is an ideal \(\fp\neq A\) s.t. \(A/\fp\) is entire. Equivalently, we could say
that it is an ideal \(\fp\neq A\) s.t. whenever \(x,y\in A\) and \(xy\in\fp\) then \(x\in\fp\) or \(y\in\fp\). A
prime ideal is often called simply a \textbf{prime}

\begin{proposition}[]
Every maximal ideal is prime
\end{proposition}

\begin{proof}
Let \(\fm\) be maximal and let \(x,y\in A\) s.t. \(xy\in\fm\). Suppose \(x\not\in\fm\), then \(\fm+Ax\) is an
ideal properly containing \(\fm\), hence equal to \(A\). Hence we can write
\begin{equation*}
1=u+ax
\end{equation*}
with \(u\in\fm\) and \(a\in A\). Multiplying by \(y\) we find
\begin{equation*}
y=yu+axy
\end{equation*}
whence \(y\in\fm\).
\end{proof}

\begin{proposition}[]
Let \(\fa\) be an ideal \(\neq A\). Then \(\fa\) is contained in some maximal ideal \(\fm\)
\end{proposition}

\begin{proposition}[]
The ideal \(\{0\}\) is a prime ideal of \(A\) iff \(A\) is entire
\end{proposition}

The only ideals of a field are itself and the zero ideal

\begin{proposition}[]
If \(\fm\) is a maximal ideal of \(A\), then \(A/\fm\) is a field
\end{proposition}

\begin{proof}
If \(x\in A\), we denote by \(\barx\) its residue class mod \(\fm\). Since \(\fm\neq A\) we note
that \(A/\fm\)  has a unit element \(\neq 0\). Any non-zero element of \(A/\fm\) can be written
as \(\barx\) for some \(x\in A\), \(x\not\in\fm\). To find its inverse, note that \(\fm+Ax\) is an ideal
of \(A\neq\fm\) and hence equal to \(A\). Hence we can write
\begin{equation*}
1=u+yx
\end{equation*}
with \(u\in\fm\) and \(y\in A\). This means that \(\bary\barx=1=\bar{1}\) and hence that \(\barx\) has
an inverse.
\end{proof}

\begin{proposition}[]
Let \(f:A\to A'\) be a homomorphism of commutative rings. Let \(\fp'\) be a prime ideal of \(A'\) and
let \(\fp=f^{-1}\fp'\). Then \(\fp\) is prime
\end{proposition}

\begin{examplle}[]
Let \(\Z\) be the ring of integers. Since an ideal is also an additive subgroup of \(\Z\), every
ideal \(\neq\{0\}\) is principal, of the form \(n\Z\) for some integer \(n>0\). (\href{https://math.stackexchange.com/questions/101348/show-that-every-ideal-of-the-ring-mathbb-z-is-principal}{proof})

Let \(\fp\) be a prime ideal \(\neq\{0\}\), \(\fp=n\Z\). Then \(n\) must be a prime number. Conversely,
if \(p\) is a prime number, then \(p\Z\) is a prime ideal. Furthermore, \(p\Z\) is a maximal ideal.
Suppose \(p\Z\) is contained in some ideal \(n\Z\), then \(p=nm\) for some integer \(m\),
whence \(n=p\) or \(n=1\), thereby proving \(p\Z\) maximal
\end{examplle}

if \(n\) is an integer, the factor ring \(\Z/n\Z\) is called the ring of \textbf{integers modulo} \(n\). We
also denote
\begin{equation*}
\Z/n\Z=\Z(n)
\end{equation*}
If \(n\) is a prime number \(p\), then the ring of integers modulo \(p\) is in fact a field,
denoted by \(\F_p\). In particular, the multiplicative group of \(\F_p\) is called the group of
non-zero integers modulo \(p\). From the elementary properties of groups, we get a standard fact
of elementary number theory: if \(x\) is an integer \(\neq 0\mod p\), then \(x^{p-1}\equiv 1\mod p\)
(Fermat's Theorem). Similarly given an integer \(n>1\), the units in the ring \(\Z/n\Z\) consist
of those residue class mod \(n\Z\) which are represented by integers \(m\neq 0\) and prime to \(n\).
The order of the group of units in \(\Z/n\Z\) is called by definition \(\varphi(n)\) (where \(\varphi\) is known as
the \textbf{Euler phi-function}). Consequently, if \(x\) is an integer prime to \(n\),
then \(x^{\varphi(n)}\equiv 1\mod n\)


\begin{theorem}[Chinese Remainder Theorem]
Let \(\fa_1,\dots,\fa_n\) be ideals of \(A\) s.t. \(\fa_i+\fa_j=A\) for all \(i\neq j\). Given
elements \(x_1,\dots,x_n\in A\) ,there exists \(x\in A\) s.t. \(x\equiv x_i\mod\fa_i\) for all \(i\)
\end{theorem}

\begin{proof}
For \(n=2\) we have an expression
\begin{equation*}
1=a_1+a_2
\end{equation*}
for some \(a_i\in\fa_i\), and we let \(x=x_2a_1+x_1a_2\)

For each \(i\ge 2\) we can find elements \(a_i\in\fa_1\) and \(b_i\in\fa_i\) s.t.
\begin{equation*}
a_i+b_i=1,\quad i\ge 2
\end{equation*}
The products \(\prod_{i=2}^n(a_i+b_i)\) is equal to 1, and lies in
\begin{equation*}
\fa_1+\prod_{i=2}^n\fa_i
\end{equation*}
Hence
\begin{equation*}
\fa_1+\prod_{i=2}^n\fa_i=A
\end{equation*}
By theorem for \(n=2\), we can find an element \(y_1\in A\) s.t.
\begin{align*}
&y_1\equiv 1\mod\fa_1\\
&y_1\equiv 0\mod\prod_{i=2}^n\fa_i
\end{align*}
We find similarly elements \(y_2,\dots,y_n\) s.t.
\begin{equation*}
y_j\equiv 1\mod\fa_j \quad\text{ and }\quad y_j\equiv 0\mod\fa_i\text{ for }i\neq j
\end{equation*}
Then \(x=x_1y_1+\dots+x_ny_n\) satisfies our requirements
\end{proof}

In the same vein as above, we observe that if \(\fa_1,\dots,\fa_n\) are ideals of a ring \(A\) s.t.
\begin{equation*}
\fa_1+\dots+\fa_n=A
\end{equation*}
and if \(v_1,\dots,v_n\) are positive integers, then
\begin{equation*}
\fa_1^{v_1}+\dots+\fa_n^{v_n}=A
\end{equation*}
\label{Problem2}

\begin{corollary}[]
Let \(\fa_1,\dots,\fa_n\) be ideals of \(A\). Assume that \(\fa_i+\fa_j=A\) for \(i\neq j\). Let
\begin{equation*}
f:A\to\prod_{i=1}^nA/\fa_i=(A/\fa_1)\times\dots\times(A/\fa_n)
\end{equation*}
be the map of \(A\) into the product induced by the canonical map of \(A\) onto \(A/\fa_i\) for each
factor. Then the kernel of \(f\) is \(\bigcap_{i=1}^n\fa_i\) and \(f\) is surjective, thus giving an
isomorphism
\begin{equation*}
A/\bigcap\fa_i\cong\prod A/\fa_i
\end{equation*}
\end{corollary}

\begin{proof}
Surjectivity follows from the theorem
\end{proof}

Let \(m\) be an integer \(>1\), and let
\begin{equation*}
m=\prod_ip_i^{r_i}
\end{equation*}
be a factorization of \(m\) into primes, with exponents \(r_i\ge 1\). Then we have a ring
isomorphism
\begin{equation*}
\Z/m\Z\cong\prod_i\Z/p_i^{r_i}\Z
\end{equation*}
If \(A\) is a ring, we denote as usual by \(A^*\) the multiplicative group of invertible elements
of \(A\)

\begin{proposition}[]
The preceding ring isomorphism of \(\Z/m\Z\) onto the product induces a group isomorphism
\begin{equation*}
(\Z/m\Z)^*\cong\prod_i(\Z/p_i^{r_i}\Z)^*
\end{equation*}
\end{proposition}

In view of our isomorphism, we have
\begin{equation*}
\varphi(m)=\prod_i\varphi(p_i^{r_i})
\end{equation*}
If \(p\) is a prime number and \(r\) an integer \(\ge 1\), then
\begin{equation*}
\varphi(p^r)=(p-1)p^{r-1}
\end{equation*}
If \(r=1\), then \(\Z/p\Z\) is a field, and the multiplicative group of that field has
order \(p-1\). Let \(r\) be \(\ge 1\), and consider the canonical ring homomorphism
\begin{equation*}
\Z/p^{r+1}\Z\to\Z/p^r\Z
\end{equation*}
arising from the inclusion of ideals \((p^{r+1})\subset(p^r)\). We get an induced group homomorphism
\begin{equation*}
\lambda:(Z/p^{r+1}\Z)^*\to(\Z/p^r\Z)^*
\end{equation*}
which is surjective because any integer \(a\) which represents an element of \(\Z/p^r\Z\) and is
prime to \(p\) will represent an element of \((\Z/p^{r+1}\Z)^*\). Let \(a\) be an integer
representing an element of \((\Z/p^{r+1}\Z)^*\) s.t. \(\lambda(a)=1\). Then
\begin{equation*}
a\equiv 1\mod p^{r}\Z
\end{equation*}
P96

\textbf{Application: The ring of endomorphisms of a cyclic group}.
\begin{theorem}[]
Let \(A\) be a cyclic group of order \(n\). For each \(k\in\Z\) let \(f_k:A\to A\) be the
endomorphism \(x\mapsto kx\) (writing \(A\) additively). Then \(k\mapsto f_k\) induces a ring
homomorphism \(\Z/n\Z\cong\End(A)\), and a group isomorphism \((\Z/n\Z)^*\cong\Aut(A)\)
\end{theorem}

\begin{proof}
The fact that \(k\mapsto f_k\) is ring homomorphism is a restatement of the formulas
\begin{equation*}
1a=a,\quad (k+k')a=ka+k'a,\quad (kk')a=k(k'a)
\end{equation*}
\end{proof}


\subsection{Polynomials and Group Rings}
\label{sec:orga2614c2}
Consider an infinite cyclic group generated by an element \(X\). We let \(S\) be the subset
consisting of powers \(X^r\) with \(r\ge 0\). Then \(S\) is a monoid. We define the set of
\textbf{polynomials} \(A[X]\) to be the set of functions \(S\to A\) which are equal to 0 except for a finite
number  of elements of \(S\). For each element \(a\in A\) we denote by \(aX^n\) the function which
has the value \(a\) on \(X^n\) and the value 0 for all other elements of \(S\). Then it is
immediate that a polynomial can be written uniquely as a finite sum
\begin{equation*}
a_0X^0+\dots+a_nX^n
\end{equation*}
for some integer \(n\in\N\) and  \(a_i\in A\). Such a polynomial is denoted by \(f(X)\). The
elements \(a_i\in A\) are called the \textbf{coefficients} of \(f\). We define the product according to the
convolution rule. Thus, given polynomials
\begin{equation*}
f(X)=\sum_{i=0}^na_iX^i \quad\text{ and }\quad g(X)=\sum_{j=0}^mb_jX^j
\end{equation*}
we define the product to be
\begin{equation*}
f(X)g(X)=\sum_{k=0}^{m+n}\left( \sum_{i+j=k}a_ib_j \right)X^k
\end{equation*}
This product is associative and distributive. \(1X^0\) is the unit element.  There is also an
embedding
\begin{gather*}
A\to A[X]\\
a\mapsto aX^0
\end{gather*}
Let \(A\) be a subring of a commutative ring \(B\). Let \(x\in B\). If \(f\in A[X]\) is a polynomial,
we define the associated \textbf{polynomial function}
\begin{equation*}
f_B:B\to B
\end{equation*}
by letting
\begin{equation*}
f_B(x)=f(x)=a_0+a_1x+\dots+a_nx^n
\end{equation*}
Given an element \(b\in B\), directly from the definition of multiplication of polynomials, we find
\begin{proposition}[]
The association
\begin{equation*}
\ev_b:f\mapsto f(b)
\end{equation*}
is a ring homomorphism of \(A[X]\) into \(B\)
\end{proposition}

This homomorphism is called the \textbf{evaluation homomorphism}, and is also said to be obtained by
\textbf{substituting} \(b\) for \(X\) in the polynomial

Let \(x\in B\). We see that the subring \(A[x]\) of \(B\) generated by \(x\) over \(A\) is a ring
of all polynomial values \(f(x)\) for \(f\in A[X]\). If the evaluation map \(f\mapsto f(x)\) gives an
isomorphism of \(A[X]\) with \(A[x]\), then we say that \(x\) is \textbf{transcendental} over \(A\), or
that \(x\) is a \textbf{variable} over \(A\). In particular, \(X\) is a variable over \(A\)

\begin{examplle}[]
Let \(\alpha=\sqrt{2}\). Then the set of all real numbers of the form \(a+b\alpha\), with \(a,b\in\Z\) is a
subring of the real numbers, generated by \(\sqrt{2}\). \(\alpha\) is not transcendental over \(\Z\),
because the polynomial \(X^2-2\) lies in the kernel of the evaluation map \(f\mapsto f(\sqrt{2})\). On
the other hand, it can be shown that \(e\) and \(\pi\) are transcendental over \(\Q\)
\end{examplle}

\begin{examplle}[]
Let \(p\) be a prime number and let \(K=\Z/p\Z\). Then \(K\) is a field. Let \(f(X)=X^p-X\in K[X]\).
Then \(f\) is not the zero polynomials. But \(f_K\) is the zero function. Indeed, \(f_K(0)=0\).
If \(x\in K\), \(x\neq 0\), then since the multiplicative group of \(K\) has order \(p-1\). it follows
that \(x^{p-1}=1\), whence \(x^p=x\), so \(f(x)\). Thus a non-zero polynomial gives rise to the
zero function on \(K\)
\end{examplle}

Let
\begin{equation*}
\varphi:A\to B
\end{equation*}
be a homomorphism of commutative rings. Then there is an associated homomorphism of the
polynomial rings \(A[X]\to B[X]\) s.t.
\begin{equation*}
f(X)=\sum a_iX^i\mapsto\sum\varphi(a_i)X^i=(\varphi f)(X)
\end{equation*}
We call \(f\mapsto\varphi f\) the \textbf{reduction map}

Let \(\fp\) be a prime ideal of \(A\). Let \(\varphi:A\to A'\) be the canonical homomorphism of \(A\)
onto \(A/\fp\). If \(f(X)\) is a polynomial in \(A[X]\), then \(\varphi f\) will sometimes be called the
\textbf{reduction of \(f\) modulo \(\fp\)}.

For example, taking \(A=\Z\) and \(\fp=(p)\) for some prime number \(p\), we can speak of the
polynomial \(3X^4-X+2\) as a polynomial mod 5, viewing the coefficients as elements of \(\Z/5\Z\)

\begin{proposition}[]
Let \(\varphi:A\to B\) be a homomorphism of commutative rings. Let \(x\in B\). There is a unique
homomorphism extending \(\varphi\)
\begin{equation*}
A[X]\to B\quad\text{ s.t. }\quad X\mapsto x
\end{equation*}
and for this homomorphism \(\sum a_iX^i\mapsto\sum\varphi(a_i)x^i\)
\end{proposition}

The homomorphism of the above statement may be views as the composite
\begin{center}\begin{tikzcd}
A[X\ar[r]]&B[X]\ar[r,"\ev_x"]&B
\end{tikzcd}\end{center}

When writing a polynomial \(f(X)=\displaystyle\sum_{i=1}^na_iX^i\), if \(a_n\neq 0\) then we define \(n\) to be the
\textbf{degree} of \(f\). Thus the degree of \(f\) is the smallest integer \(n\) s.t. \(a_r=0\)
for \(r>n\). If \(f=0\) (i.e. \(f\) is the zero polynomial), then by convention, we define the
degree of \(f\) to be \(-\infty\). We agree to the convention that
\begin{equation*}
-\infty+-\infty=-\infty,\quad-\infty+n=-\infty,\quad-\infty<n
\end{equation*}
for all \(n\in\Z\), and no other operation with \(-\infty\) is defined. A polynomial of degree 1 is also
called a \textbf{linear} polynomial. If \(f\neq 0\) and \(\deg f=n\) then we call \(a_n\) the \textbf{leading
coefficient} of \(f\). We call \(a_0\) its \textbf{constant term}

Let
\begin{equation*}
g(X)=b_0+\dots+b_mX^m
\end{equation*}
be a polynomial in \(A[X]\), of degree \(m\), and assume \(g\neq 0\). Then
\begin{equation*}
f(X)g(X)=a_0b_0+\dots+a_nb_mX^{m+n}
\end{equation*}
Therefore
\begin{proposition}[]
If we assume that at least one of the leading coefficients \(a_n\) or \(b_m\) is not a divisor of
0 in \(A\), then
\begin{equation*}
\deg(fg)=\deg f+\deg g
\end{equation*}
and the leading coefficient of \(fg\) is \(a_nb_m\). This holds in particular when \(a_n\)
or \(b_m\) is a unit in \(A\), or when \(A\) is entire. Consequently, when \(A\) is
entire, \(A[X]\) is also entire
\end{proposition}

If \(f=0\) or \(g=0\) we still have
\begin{equation*}
\deg(fg)=\deg f+\deg g
\end{equation*}
if we agree that \(-\infty+m=-\infty\) for any integer \(m\)

Let \(A\) be a subring of a commutative ring \(B\). Let \(x_1,\dots,x_n\in B\). For each \(n\)-tuple of
integers \((v_1,\dots,v_n)=\bv\in\N^n\), let \(\bx=(x_1,\dots,x_n)\), and
\begin{equation*}
M_{\bv}(\bx)=x_1^{v_1}\dots x_n^{v_n}
\end{equation*}
The set of such elements forms a monoid under multiplication. Let \(A[x]=A[x_1,\dots,x_n]\) be the
subring of \(B\) generated by \(x_1,\dots,x_n\) over \(A\). Then every element of \(A[x]\) can be
written as a finite sum
\begin{equation*}
\sum a_{\bv}M_{\bv}(\bx) \quad\text{ and }\quad a_{\bv}\in A
\end{equation*}

Using the construction of polynomials in one variable repeatedly, we may form the ring
\begin{equation*}
A[X_1,\dots,X_n]=A[X_1][X_2]\dots[X_n]
\end{equation*}
selecting \(X_n\) to be variable over \(A[X_1,\dots,X_{n-1}]\). Then every element \(f\)
of \(A[X_1,\dots,X_n]=A[X]\) has a \emph{unique} expression as a finite sum
\begin{equation*}
f=\sum_{j=0}^{d_n}f_j(X_1,\dots,X_{n-1})X_n^j \quad\text{with}\quad f_j\in A[X_1,\dots,X_{n-1}]
\end{equation*}
Therefore by induction we can write \(f\) uniquely as a sum
\begin{align*}
f&=\sum_{v_n=0}^{d_n}\left(
\sum_{v_1,\dots,v_{n-1}}a_{v_1\dots v_n}X_1^{v_1}\dots X_{n-1}^{v_{n-1}} \right)X^{v_n}_n\\
&=\sum a_{\bv}M_{\bv}(X)=\sum a_{\bv}X_1^{v_1}\dots X_n^{v_n}
\end{align*}
with elements \(a_{\bv\in A}\), which are called the \textbf{coefficients} of \(f\). The products
\begin{equation*}
M_{\bv}(X)=X_1^{v_1}\dots X_n^{v_n}
\end{equation*}
will be called \textbf{primitive monomials}. Elements of \(A[X]\) are called \textbf{polynomials} (in \(n\)
variables). We call \(a_{\bv}\) its \textbf{coefficients}

GIven \(\bx=(x_1,\dots,x_n)\) and \(f\), we define
\begin{equation*}
f(x)=\sum a_{\bv}M_{\bv}(\bx)=\sum a_{\bv}x_1^{v_1}\dots x_n^{v_n}
\end{equation*}
Then the \textbf{evaluation map}
\begin{equation*}
\ev_{\bx}:A[X]\to B \quad\text{with}\quad f\mapsto f(x)
\end{equation*}
is a ring homomorphism

Elements \(x_1,\dots,x_n\in B\) are called \textbf{algebraically independent} over \(A\) if the evaluation map
\begin{equation*}
f\mapsto f(x)
\end{equation*}
is injective. Equivalently, we could say that if \(f\in A[X]\) is a polynomial and \(f(x)=0\)
then \(f=0\).; in other words, there are no non-trivial polynomial relations among \(x_1,\dots,x_n\)
over \(A\).

By the \textbf{degree} of a primitive monomial
\begin{equation*}
M_{\bv}(X)=X_1^{v_1}\dots X_n^{v_n}
\end{equation*}
we shall mean the integer \(\abs{v}=v_1+\dots+v_n\)

A polynomial
\begin{equation*}
aX_1^{v_1}\dots X_n^{v_n}\quad(a\in A)
\end{equation*}
will be called a \textbf{monomial}

If \(f(X)\) is a polynomial in \(A[X]\) written as
\begin{equation*}
f(X)=\sum a_{\bv}X_1^{v_1}\dots X_n^{v_n}
\end{equation*}
we define the \textbf{degree} of \(f\) to be the maximum of the degrees of the monomials \(M_{\bv}(X)\)
s.t. \(a_{\bv}\neq 0\). (Such monomials are said to \textbf{occur} in the polynomial)

For each integer \(d\ge 0\), given a polynomial \(f\), let \(f^{(d)}\) be the sum of all monomials
occuring in \(f\) and having degree \(d\). Then
\begin{equation*}
f=\sum_df^{(d)}
\end{equation*}
Suppose \(f\neq 0\), we say that \(f\) is \textbf{homogeneous} of degree \(d\) if \(f=f^{(d)}\)

Algebraically independent elements will also be called \textbf{variables}

\subsection{Localization}
\label{sec:orgba17010}
\(A\) a commutative ring

By a \textbf{multiplicative subset} of \(A\) we shall mean a submonoid of \(A\)

We shall now construct the \textbf{quotient ring of \(A\) by \(S\)}, also known as the \textbf{ring of fractions}
\textbf{of \(A\) by \(S\)}

We consider pairs \((a,s)\) with \(a\in A\) and \(s\in S\). We define a relation
\begin{equation*}
(a,s)\sim (a',s')
\end{equation*}
if there exists \(s_1\in S\) s.t.
\begin{equation*}
s_1(s'a-sa')=0
\end{equation*}
The equivalence class containing a pair \((a,s)\) is denoted by \(a/s\). The set of equivalence
classes is denoted by \(S^{-1}A\)

if \(0\in  S\), then \(S^{-1}A\) has precisely one element \(0/1\)
\begin{gather*}
(a/s)(a'/s')=aa'/ss'\\
\frac{a}{s}+\frac{a'}{s'}=\frac{s'a+sa'}{ss'}
\end{gather*}

Let \(\varphi_S:A\to S^{-1}A\) be the s.t. \(\varphi_S(a)=a/1\). Every element of \(\varphi_S(S)\) is invertible
in \(S^{-1}(A)\) (the inverse of \(s/1\) is \(1/s\))

Let \(\calc\) be the category whose objects are ring homomorphism
\begin{equation*}
f:A\to B
\end{equation*}
s.t. for every \(s\in S\) the elements \(f(s)\) is invertible in \(B\). If \(f:A\to B\) and

\begin{proposition}[]
Let \(A\) be an entire ring, and let \(S\) be a multiplicative subset which does not contain 0.
Then
\begin{equation*}
\varphi_S:A\to S^{-1}A
\end{equation*}
is injective
\end{proposition}

Let \(A\) be an entire ring, and let \(S\) be the set of non-zero elements of \(A\). Then \(S\)
is a multiplicative set, and \(S^{-1}A\) is then a field, called the \textbf{quotient field} or the *field
of fractions of \(A\).

\subsection{Principal and Factorial Rings}
\label{sec:orgd4fa35d}
Let \(A\) be an entire ring. An element \(a\neq 0\) is called \textbf{irreducible} if it is not a unit, and
if whenever one can write \(a=bc\) with \(b\in A\) and \(c\in A\), then \(b\) or \(c\) is a unit

\emph{Let \(a\neq 0\) be an element of \(A\) and assume that the principal ideal \((a)\) is prime. Then}
\((a)\) \emph{is irreducible}. If we write \(a=bc\)., then \(b\) or \(c\) lies in \((a)\), say \(b\).
Then we can write \(b=ad\) with some \(d\in A\) and hence \(a=acd\). Since \(A\) is entire, it
follows that \(cd=1\), in other words, \(c\) is a unit.

The converse of the preceding assertion is not always true. We shall discuss under which
conditions it is true. An element \(a\in A\), \(a\neq 0\) is said to have a
\textbf{unique factorization into irreducible elements} if there exists a unit \(u\) and there exist
irreducible elements \(p_i\) in \(A\) s.t.
\begin{equation*}
a=u\prod_{i=1}^rp_i
\end{equation*}
and if given two factorization into irreducible elements
\begin{equation*}
a=u\prod_{i=1}^rp_i=u'\prod_{j=1}^sq_j
\end{equation*}
we have \(r=s\) and after a permutation of the indices \(i\), we have \(p_i=u_iq_i\) for some
unit \(u_i\in A\)

A ring is called \textbf{factorial} (or \textbf{unique factorization ring}) if it is entire and if every
element \(\neq 0\) has a unique factorization into irreducible elements.

Let \(A\) be an entire ring and \(a,b\in A\), \(ab\neq 0\). We say that \(a\) \textbf{divides} \(b\) and
write \(a\mid b\) if there exists \(c\in A\) s.t. \(ac=b\). We say that \(d\in A\), \(d\neq 0\) is a
\textbf{greatest common divisor} (\textbf{g.c.d.}) of \(a\) and \(b\) if \(d\mid a\) and \(d\mid b\) and if any
element \(e\) of \(A\) \(e\neq 0\) which divides both \(a\) and \(b\) also divides \(d\)

\begin{proposition}[]
Let \(A\) be a principal entire ring and \(a,b\in A\), \(a,b\neq 0\). Let \((a,b)=(c)\). Then \(c\) is
a greatest common divisor of \(a\) and \(b\)
\end{proposition}

\begin{theorem}[]
Let \(A\) be a principal entire ring. Then \(A\) is factorial
\end{theorem}

\begin{proof}
We first prove that every non-zero element of \(A\) has a factorization into irreducible
elements. Let \(S\) be the set of principal ideals \(\neq 0\) whose generators do not have a
factorization into irreducible elements, and suppose \(S\) is not empty. Let \((a_1)\in S\)  be
in \(S\). Consider an ascending chain
\begin{equation*}
(a_1)\subsetneq(a_2)\subsetneq\dots\subsetneq(a_n)\subsetneq\dots
\end{equation*}
of ideals in \(S\). We contend that such a chain cannot be infinite. Indeed, the union of such a
chain is an ideal of \(A\), which is principal, say equal to \((a)\). The generator \(a\) must
already lie in some element of the chain, say \((a_n)\), and then we see
that \((a_n)\subset(a)\subset(a_n)\), whence the chain stops at \((a_n)\). Hence \(S\)  is inductively
ordered, and has a maximal element \((a)\). Therefore any ideal of \(A\) containing \((a)\)
and \(\neq(a)\) has a generator admitting a factorization.

We note that \(a_n\) cannot be irreducible and hence we can write \(a=bc\) with neither \(b\)
nor \(c\) equal to a unit. But then \((b)\neq(a)\) and \((c)\neq(a)\) and hence both \(b\) and \(c\)
admit factorizations into irreducible elements. The product of these factorizations is a
factorization for \(a\), contradicting the assumption that \(S\) is not empty

To prove uniqueness, we first remark that if \(p\) is an irreducible element of \(A\)
and \(a,b\in A\), \(p\mid ab\), then \(p\mid a\) or \(p\mid b\). \emph{Proof}: if \(p\not\mid a\), then the g.c.d.
of \(p,a\) is 1 and hence we can write
\begin{equation*}
1=xp+ya
\end{equation*}
for some \(x,y\in A\). Then \(b=bxp+yab\) and since \(p\mid ab\) we conclude that \(p\mid b\)

Suppose that \(a\) has two factorizations
\begin{equation*}
a=p_1\dots p_r=q_1\dots q_s
\end{equation*}
into irreducible elements. Since \(p_1\) divides \(q_1\dots q_s\), \(p_1\) divides one of the factors,
which we may assume to be \(q_1\) after renumbering these factors. Then there exists a unit \(u_1\)
s.t. \(q_1=u_1p_1\). We can now cancel \(p_1\) from both factorizations and get
\begin{equation*}
p_2\dots p_r=u_1q_2\dots q_s
\end{equation*}
\end{proof}

We could call two elements \(a,b\in A\) equivalent if there exists a unit \(u\) s.t. \(a=bu\). let
us select irreducible element \(p\) out of each equivalence class belonging to such an
irreducible element, and let us denote by \(P\) the set of such representatives.
Let \(a\in A,a\neq 0\). Then there exists a unit \(u\) and integers \(v(p)\ge 0\), equal to 0 for almost
all \(p\in P\) s.t.
\begin{equation*}
a=u\prod_{p\in P}p^{v(p)}
\end{equation*}
Furthermore, the unit \(u\) and the integers \(v(p)\) are uniquely determined by \(a\). We
call \(v(p)\) the \textbf{order} of \(a\) at \(p\), also written as \(\ord_pa\)

If \(A\) is a factorial ring, then an irreducible element \(p\) generates a prime ideal \((p)\).
Thus in a factorial ring, an irreducible element will also be called a \textbf{prime element}, or simply \textbf{prime}


\section{Modules}
\label{sec:orgdcee263}

\subsection{Basic Definitions}
\label{sec:org659b735}
Let \(A\) be a ring. A \textbf{left module} over \(A\), or a left \(A\)-module \(M\) is an abelian group,
together with an operation of \(A\) on \(M\), s.t. for all \(a,b\in A\) and \(x,y\in M\)
\begin{equation*}
(a+b)x=ax+bx \quad\text{ and }\quad a(x+y)=ax+ay
\end{equation*}

Let \(A\) be an entire ring and let \(M\) be an \(A\)-module. We define the \textbf{torsion
submodule} \(M_{tor}\) to be the subset of elements \(x\in M\) s.t. there exist\(a\in A\)s , \(a\neq 0\)
s.t. \(ax=0\).

By a \textbf{module homomorphism} we means a map
\begin{equation*}
f:M\to M'
\end{equation*}
which is an additive group homomorphism and s.t.
\begin{equation*}
f(ax)=af(x)
\end{equation*}
for all \(a\in A\) and \(x\in M\). If we wish to refer to the ring \(A\), we also say that \(f\) is
an \textbf{\(A\)-homomorphism}, or also that it is an \textbf{\(A\)-linear map}

For any module \(M\) and \(M'\), the map \(\zeta:M\to M'\) s.t.
\(\zeta(x)=0\) for all \(x\in M\) is a homomorphism, called \textbf{zero}

Let \(f:M\to M'\) be a homomorphism. By the \textbf{cokernel} of \(f\) we mean the factor module
\(M'/\im f=M'/f(M)\).

Like groups
\begin{proposition}[]
Let \(N,N'\) be two submodules of a module of \(M\). Then \(N+N'\) is also a submodule, and we
have an isomorphism
\begin{equation*}
N/(N\cap N')\cong(N+N')/N'
\end{equation*}
If \(M\supset M'\supset M''\) are modules, then
\begin{equation*}
(M/M'')/(M'/M'')\cong M/M'
\end{equation*}
If \(f:M\to M'\) is a module homomorphism, and \(N'\) is a submodule of \(M'\), then \(f^{-1}(N')\)
is a submodule of \(M\) and we have a canonical injective homomorphism
\begin{equation*}
\barf:M/f^{-1}(N')\to M'/N'
\end{equation*}
If \(f\) is surjective, then \(\barf\) is a module isomorphism
\end{proposition}

A sequence of module homomorphisms
\begin{center}\begin{tikzcd}
M'\ar[r,"f"]&M\ar[r,"g"]&M''
\end{tikzcd}\end{center}
is \textbf{exact} if \(\im f=\ker g\). If \(N\) is a submodule of \(M\), then
\begin{center}\begin{tikzcd}
0\ar[r]&N\ar[r]&M\ar[r]&M/N\ar[r]&0
\end{tikzcd}\end{center}

If a homomorphism \(u:N\to M\) is s.t.
\begin{center}\begin{tikzcd}
0\ar[r]&N\ar[r,"u"]&M
\end{tikzcd}\end{center}

is exact, then we also say that \(u\) is a \textbf{monomorphism} or an \textbf{embedding}. Dually
if
\begin{center}\begin{tikzcd}
N\ar[r,"u"]&M\ar[r]&0
\end{tikzcd}\end{center}
is exact, we say that \(u\) is an \textbf{epimorphism}

Let \(A\) be a commutative ring. Let \(E,F\) be modules. By a \textbf{bilinear map}
\begin{equation*}
g:E\times E\to F
\end{equation*}
we mean a map s.t. given \(x\in E\) the map \(y\mapsto g(x,y)\) is \(A\)-linear and given \(y\in E\), the
map \(x\mapsto g(x,y)\) is \(A\)-linear. By an \textbf{\(A\)-algebra} we mean a module together with a bilinear
map \(g:E\times E\to E\) . We view such a map as a law of composition on \(E\).

\subsection{The Group of Homomorphisms}
\label{sec:org80ee62c}
Let \(A\) be a ring, and let \(X,X'\) be \(A\)-modules. We denote by \(\Hom_A(X',X)\) the set
of \(A\)-homomorphisms of \(X'\) into \(X\). Then \(\Hom_A(X',X)\) is an abelian group, the law
of addition being that of addition for mappings into an abelian group.

If \(A\) is \emph{commutative} then we can make \(\Hom_A(X',X)\) into an \(A\)-module by defining \(af\)
for \(a\in A\) and \(f\in\Hom_A(X',X)\) to be the map s.t.
\begin{equation*}
(af)(x)=af(x)
\end{equation*}

Let \(Y\) be an \(A\)-module, and let
\begin{center}\begin{tikzcd}
X'\ar[r,"f"]&X
\end{tikzcd}\end{center}
be an \(A\)-homomorphism. Then we get an induced homomorphism
\begin{equation*}
\Hom_A(f,Y):\Hom_{A}(X,Y)\to\Hom_A(X',Y)
\end{equation*}
given by \(g\mapsto g\circ f\). The fact that \(\Hom_A(f,Y)\) is a homomorphism is a rephrasing of the
\((g_1+g_2)\circ f=g_1\circ f+g_2\circ f\)

If we have a sequence of \(A\)-homomorphisms
\begin{center}\begin{tikzcd}
X'\ar[r]&X\ar[r]&X''
\end{tikzcd}\end{center}
then we get an induced sequence
\begin{center}\begin{tikzcd}
\Hom_A(X',Y)&\Hom_A(X,Y)\ar[l]&\Hom_A(X'',Y)\ar[l]
\end{tikzcd}\end{center}

\begin{proposition}[]
A sequence
\begin{center}\begin{tikzcd}
X'\ar[r,"\lambda"]&X\ar[r]&X''\ar[r]&0
\end{tikzcd}\end{center}
is exact iff the sequence
\begin{center}\begin{tikzcd}
\Hom_A(X',Y)&\Hom_A(X,Y)\ar[l]&\Hom_A(X'',Y)\ar[l]&0\ar[l]
\end{tikzcd}\end{center}
is exact for all \(Y\)
\end{proposition}

\begin{proof}
Suppose the first sequence is exact. If \(g:X''\to Y\) is an \(A\)-homomorphism, its image
in \(\Hom_A(X,Y)\) is obtained by composing \(g\) with the surjective map of \(X\) on \(X''\). If
this composition is 0, it follows that \(g=0\).  Consider a homomorphism \(g:X\to Y\) s.t. the
composition
\begin{center}\begin{tikzcd}
X'\ar[r,"\lambda"]&X\ar[r,"g"]&Y
\end{tikzcd}\end{center}
is 0. Then \(g\) vanishes on the image of \(\lambda\). Hence we can factor \(g\) through the factor module
\begin{center}\begin{tikzcd}[column sep=small]
&X/\im\lambda\ar[rd]\\
X\ar[ur]\ar[rr,"g"']&&Y
\end{tikzcd}\end{center}
Since \(X\to X''\) is surjective, we have an isomorphism
\begin{equation*}
X/\im\lambda\cong X''
\end{equation*}
Hence we can factor \(g\) through \(X''\), thereby showing that the kernel of
\begin{center}\begin{tikzcd}
\Hom_A(X',Y)&\Hom_A(X,Y)\ar[l]
\end{tikzcd}\end{center}
is contained in the image of
\begin{center}\begin{tikzcd}
\Hom_A(X,Y)&\Hom_A(X'',Y)\ar[l]
\end{tikzcd}\end{center}
\end{proof}

similarly, we have
\begin{proposition}[]
A sequence
\begin{center}\begin{tikzcd}
0\ar[r]&Y'\ar[r]&Y\ar[r]&Y''
\end{tikzcd}\end{center}
is exact iff
\begin{center}\begin{tikzcd}
0\ar[r]&\Hom_A(X,Y')\ar[r]&\Hom_A(X,Y)\ar[r]&\Hom_A(X,Y'')
\end{tikzcd}\end{center}
is exact for all \(X\)
\end{proposition}

Let \(\Mod(A)\) and \(\Mod(B)\) be the categories of modules over rings \(A\) and \(B\), and
let \(F:\Mod(A)\to\Mod(B)\) be a functor. One says that \(F\) is \textbf{exact} if \(F\) transforms exact
sequences into exact sequences.

let \(M\) be an \(A\)-module. From the relations
\begin{align*}
&(g_1+g_2)\circ f=g_1\circ f+g_2\circ f\\
&g\circ(f_1+f_2)=g\circ f_1+g\circ f_2
\end{align*}
and the fact that there is an identity for composition, namely \(id_M\), we conclude
that \(\Hom_A(M,M)\) is a ring. We call \(\End_A(M)=\Hom_A(M,M)\) the ring of \textbf{endomorphisms}

\subsection{Direct Products and Sums of Modules}
\label{sec:org70042c0}
\begin{proposition}[]
Let \(M\) be an \(A\)-module and \(n\) an integer \(\ge 1\). For each \(i=1,\dots,n\) let \(\varphi_i:M\to M\)
be an \(A\)-homomorphism s.t.
\begin{equation*}
\sum_{i=1}^n\varphi_i=\id \quad\text{ and }\quad\varphi_i\circ\varphi_j=0\quad \text{ if }i\neq j
\end{equation*}
Then \(\varphi_i^2=\varphi_i\) for all \(i\). Let \(M_i=\varphi_i(M)\) , and let \(\varphi:M\to\prod M_i\) be s.t.
\begin{equation*}
\varphi(x)=(\varphi_1(x),\dots,\varphi_n(x))
\end{equation*}
Then \(\varphi\) is an \(A\)-isomorphism of \(M\) onto the direct product \(\prod M_i\)
\end{proposition}

\begin{proof}
for each \(j\), we have
\begin{equation*}
\varphi_j=\varphi_j\circ\id=\varphi_j\circ\sum_{i=1}^n\varphi_i=\varphi_j\circ\varphi_j=\varphi_j^2
\end{equation*}
thereby proving the first assertion. It is clear that \(\varphi\) is an \(A\)-homomorphism.
Let \(x\in\ker\varphi\). Since
\begin{equation*}
x=\id(x)=\sum_{i=1}^n\varphi_i(x)
\end{equation*}
we conclude that \(x=0\), so \(\varphi\) is injective.
\end{proof}

Let \(M\) be a module over a ring \(A\) and let \(S\) be a subset of \(M\). By a \textbf{linear
combination} of elements of \(S\) (with coefficients in \(A\)) one means a sum
\begin{equation*}
\sum_{x\in S}a_xx
\end{equation*}
where \(\{a_x\}\) is a set of elements of \(A\), almost all of which are equal to 0. Let \(N\) be the
set of all linear combinations of elements of \(S\). Then \(N\) is a submodule of \(M\), for if
\begin{equation*}
\sum_{x\in S}a_xx \quad\text{ and }\quad\sum_{x\in S}b_xx
\end{equation*}
are two linear combinations, then their sum is equal to
\begin{equation*}
\sum_{x\in S}(a_x+b_x)x
\end{equation*}
and if \(c\in A\), then
\begin{equation*}
c\left( \sum_{x\in S}a_xx \right)=\sum_{x\in S}ca_xx
\end{equation*}
We shall call \(N\) the submodule \textbf{generated} by \(S\), and we call \(S\) a set of \textbf{generators}
for \(N\). We sometimes write \(N=A\la S\ra\). If \(S\) consists of one element \(x\), the module
generated by \(x\) is also written \(Ax\), or simply \((x)\), and sometimes we say that \((x)\)
is a \textbf{principal module}

A module \(M\) is said to be \textbf{finitely generated}, or of \textbf{finite type} or \textbf{finite} over \(A\), if it
has a finite number of generators

A subset \(S\) of a module \(M\) is said to be \textbf{linearly independent} (over \(A\)) if whenever we
have a linear combination
\begin{equation*}
\sum_{x\in S}a_xx
\end{equation*}
which is equal to 0, then \(a_x=0\) for all \(x\in S\). If \(S\) is linearly independent and if two
linear combinations
\begin{equation*}
\sum a_xx \quad\text{ and }\quad\sum b_xx
\end{equation*}
are equal, then \(a_x=b_x\) for all \(x\in S\).

Let \(M\) be an \(A\)-module, and let \(\{M_i\}_{i\in I}\) be a family of submodules. Since we have
inclusion-homomorphism
\begin{equation*}
\lambda_i:M_i\to M
\end{equation*}
we have an induced homomorphism
\begin{equation*}
\lambda_*:\bigoplus M_i\to M
\end{equation*}
which is s.t. for any family of elements \((x_i)_{i\in I}\) all but a finite number of which are 0,
we have
\begin{equation*}
\lambda_*((x_i))=\sum_{i\in I}x_i
\end{equation*}
if \(\lambda_*\) is an isomorphism, then we say that \(\{M_i\}_{i\in I}\) is a \textbf{direct sum decomposition}
of \(M\). This is equivalent to saying that every element of \(M\) has a unique expression as a
sum
\begin{equation*}
\sum x_i
\end{equation*}
with \(x_i\in M\) and almost all \(x_i=0\). By abuse of notation, we also write
\begin{equation*}
M=\bigoplus M_i
\end{equation*}
in this case

If \(M\) is a module and \(N,N'\) are two submodules s.t. \(N+N'=M\) and \(N\cap N'=0\), then we
have a module isomorphism
\begin{equation*}
M\cong N\oplus N'
\end{equation*}
\begin{proposition}[]
Let \(M,M',N\) be modules. Then we have an isomorphism of abelian groups
\begin{equation*}
\Hom_A(M\oplus M',N)\cong\Hom_A(M,N)\times\Hom_A(M',N)
\end{equation*}
and
\begin{equation*}
\Hom_A(N,M\times M')\cong\Hom_A(N,M)\times\Hom_A(N,M')
\end{equation*}
\end{proposition}

\begin{proof}
if \(f:M\oplus M'\to N\) is a homomorphism, then \(f\) induces a homomorphism \(f_1:M\to N\) and a
homomorphism \(f_2:M'\to N\) by composing injections
\begin{align*}
&M\to M\oplus\{0\}\subset M\oplus M'\xrightarrow{f}N\\
&M'\to\{0\}\oplus M'\subset M\oplus M'\xrightarrow{f}N
\end{align*}
Then
\begin{equation*}
f\mapsto(f_1,f_2)
\end{equation*}
is an isomorphism
\end{proof}

\begin{proposition}[]
Let \(0\to M'\xrightarrow{f}M\xrightarrow{g}M''\to 0\) be an exact sequence of modules. The following are equivalent
\begin{enumerate}
\item there exists a homomorphism \(\varphi:M''\to M\) s.t. \(g\circ\varphi=\id\)
\item there exists a homomorphism \(\psi:M\to M'\) s.t. \(\psi\circ f=\id\)
\end{enumerate}


if these conditions are satisfied, then we have isomorphisms
\begin{gather*}
M=\im f\oplus\ker\psi,\hspace{1cm}M=\ker g\oplus\im\varphi\\
M\cong M'\oplus M''
\end{gather*}
\end{proposition}

\begin{proof}
Let \(x\in M\), then \(x-\varphi(g(x))\in\ker g\), and hence \(M=\ker g+\im\varphi\). If \(x\in\ker g\cap\im\varphi\),
then \(x=\varphi(w)\) and \(g(x)=g(\varphi(w))=w=0\), thus \(\ker g\cap\im\varphi=\{0\}\)

\label{Problem3}
\end{proof}

when these conditions are satisfied, the exact sequence is said to \textbf{split}. \(\psi\) \textbf{splits} \(f\) and
\(\varphi\) \textbf{splits} \(g\)

Consider first a category \(\fC\) s.t. \(\Mor(E,F)\) is an abelian group for each pair of
objects \(E,F\) of \(\fC\), satisfying the following two conditions
\begin{center}
\begin{tabular}{ll}
AB 1 & The law of composition of morphisms is bilinear, and there exists\\
 & a zero object 0, i.e., s.t. \(\Mor(0,E)\) and \(\Mor(E,0)\) have precisely\\
 & one element for each object \(E\)\\
AB 2 & Finite products and finite coproducts exists in the category\\
\end{tabular}
\end{center}

Then we say that \(\fC\) is an \textbf{additive category}

Given a morphism \(E\xrightarrow{f}F\) in \(\fC\), we define a \textbf{kernel} of \(f\) to be a morphism \(E'\to E\) s.t.
for all objects \(X\) in the category, the following sequence is exact
\begin{center}\begin{tikzcd}
0\ar[r]&\Mor(X,E')\ar[r]&\Mor(X,E)\ar[r]&\Mor(X,F)
\end{tikzcd}\end{center}
we define a \textbf{cokernel} for \(f\) to be a morphism \(F\to F''\) s.t. for all objects \(X\) in the
category, the following sequence is exact
\begin{center}\begin{tikzcd}
0\ar[r]&\Mor(F'',X)\ar[r]&\Mor(F,X)\ar[r]&\Mor(E,X)
\end{tikzcd}\end{center}
\begin{center}
\begin{tabular}{ll}
AB 3 & Kernels and cokernels exist\\
AB 4 & If \(f:E\to F\) is a morphism whose kernel is 0, then \(f\) is the kernel\\
 & of its cokernel. If \(f:E\to F\) is a morphism whose cokernel is 0,\\
 & then \(f\) is the cokernel of its kernel. A morphism whose kernel\\
 & and cokernel are 0 is an isomorphism\\
\end{tabular}
\end{center}


A category \(\fC\) satisfying the above four axioms is called an \textbf{abelian category}

In an abelian category, the group of morphisms is usually denote by Hom, so
\begin{equation*}
\Mor(E,F)=\Hom(E,F)
\end{equation*}
The morphisms are usually called \textbf{homomorphisms}. Given an exact sequence
\begin{center}\begin{tikzcd}
0\ar[r]&M'\ar[r]&M
\end{tikzcd}\end{center}
we say that \(M'\) is a \textbf{subobject} of \(M\), or that the homomorphism of \(M'\) into \(M\) is a
\textbf{monomorphism}. Dually, in an exact sequence
\begin{center}\begin{tikzcd}
M\ar[r]&M''\ar[r]&0
\end{tikzcd}\end{center}
we say that \(M''\) is a \textbf{quotient object} of \(M\), or that the homomorphism of \(M\) to \(M''\)
is an \textbf{epimorphism}

\subsection{Free Modules}
\label{sec:org9936416}
Let \(M\) be a module over a ring \(A\) and let \(S\) be a subset of \(M\). \(S\) is a \textbf{basis}
of \(M\) if \(S\) is not empty, if \(S\) generates \(M\), and if \(S\) is linearly independent.
If \(S\) is a basis of \(M\), then in particular \(M\neq\{0\}\) if \(A\neq\{0\}\) and every element
of \(M\) has a unique expression as a linear combination of elements of \(S\)

If \(A\) is a ring, then as a module over itself, \(A\) admits a basis, consisting of the unit
element 1.

Let \(I\) be a non-empty set, and for each \(i\in I\), let \(A_i=A\), viewed as an \(A\)-module. Let
\begin{equation*}
F=\bigoplus_{i\in I}A_i
\end{equation*}
then \(F\) admits a basis, which consists of the elements \(e_i\) of \(F\) whose \(i\)-th
component is the unit element of \(A_i\), and having all other components equal to 0

By a \textbf{free} module we mean a module which admits a basis, or the zero module

\begin{theorem}[]
Let \(A\) be a ring and \(M\) a module over \(A\). Let \(I\) be a non-empty set, and
let \(\{x_i\}_{i\in I}\) be a basis of \(M\). Let \(N\) be an \(A\)-module, and let \(\{y_i\}_{i\in I}\) be
a family of elements of \(N\). Then there exists a unique homomorphism \(f:M\to N\)
s.t. \(f(x_i)=y_i\) for all \(i\).
\end{theorem}

\begin{corollary}[]
Let the notation be as in the theorem, and assume that \(\{y_i\}_{i\in I}\) is a basis of \(N\). Then
the homomorphism \(f\) is an isomorphism
\end{corollary}

\begin{corollary}[]
Two modules having bases whose cadinalities are equal are isomorphic
\end{corollary}

Let \(M\) be a free module over \(A\), with basis \(\{x_i\}_{i\in I}\), so that
\begin{equation*}
M=\bigoplus_{i\in I}Ax_i
\end{equation*}
Let \(\fa\) be a two sided ideal of \(A\). Then \(\fa M\) is a submodule of \(M\). Each \(\fa x_i\) is a
submodule of \(Ax_i\). We \emph{have an isomorphism}
   \begin{equation*}
M/\fa M\cong\bigoplus_{i\in I}Ax_i/\fa x_i
   \end{equation*}

A module \(M\) is called \textbf{principal} if there exists an element \(x\in M\) s.t. \(M=Ax\). The map
 \begin{equation*}
a\mapsto ax
 \end{equation*}
is an \(A\)-module homomorphism of \(A\) onto \(M\), whose kernel is a left ideal \(\fa\).

\subsection{Vector Spaces}
\label{sec:org00afee4}
A module over a field is called a \textbf{vector space}

\begin{theorem}[]
Let \(V\) be a vector space over a field \(K\), and assume that \(V\neq\{0\}\). Let \(\Gamma\) be a set of
generators of \(V\) over \(K\) and let \(S\) be a subset of \(\Gamma\) which is linearly independent. Then
there exists a basis \(\fB\) of \(V\) s.t. \(S\subset\fB\subset\Gamma\).
\end{theorem}

\begin{proof}
Zorn's lemma
\end{proof}

\begin{theorem}[]
Let \(V\) be a vector space over a field \(K\). Then two bases of \(V\) over \(K\) have the same cardinality
\end{theorem}

\begin{proof}
First assume that there exists a basis of \(V\) with a finite number of elements,
say \(\{v_1,\dots,v_m\}\), \(m\ge 1\). It is suffice to prove: if \(w_1,\dots,w_n\) are elements of \(V\) which
are linearly independent over \(K\), then \(n\le m\) (for then we can use symmetry). We proceed by
induction. There exist elements \(c_1,\dots,c_m\) of \(K\) s.t.
\begin{equation*}
w_1=c_1v_1+\dots+c_mv_m
\end{equation*}
and some \(c_i\), say \(c_1\) is not equal to 0. Then \(v_1\) lies in the space generated
by \(w_1,v_2,\dots,v_m\) over \(K\), and this space must therefore be equal to \(V\) itself.
Furthermore, \(w_1,v_2,\dots,v_m\) are linearly independent, for suppose \(b_1,\dots,b_m\) are elements
of \(K\) s.t.
\begin{equation*}
b_1w_1+\dots+b_mv_m=0
\end{equation*}
if \(b_1\neq 0\), divide by \(b_1\) and express \(w_1\) as a linear combination of \(v_2,\dots,v_m\), would
yield a relation of linear dependence among the \(v_i\). Hence \(b_1=0\), and again we must have
all \(b_i=0\)

Suppose inductively that after a suitable renumbering of the \(v_i\), we have found \(w_1,\dots,w_r\)
(\(r<n\)) s.t.
\begin{equation*}
\{w_1,\dots,w_r,v_{r+1},\dots,v_m\}
\end{equation*}
is a basis of \(V\).
\begin{equation*}
w_{r+1}=c_1w_1+\dots+c_rw_r+c_{r+1}w_{r+1}+\dots+c_mv_m
\end{equation*}
with \(c_i\in K\). Similarly we still can replace \(v_{r+1}\) by \(w_{r+1}\).
\end{proof}

\begin{theorem}[]
Let \(V\) be a vector space over a field \(K\), and let \(W\) be a subspace. Then
\begin{equation*}
\dim_KV=\dim_KW+\dim_KV/W
\end{equation*}
If \(f:V\to U\) is a homomorphism of vector spaces over \(K\), then
\begin{equation*}
\dim V=\dim\ker f+\dim\im f
\end{equation*}
\end{theorem}

\begin{proof}
The first statement is a special case of the second, taking for \(f\) the canonical map.
Let \(\{u_i\}_{i\in I}\) be a basis of \(\im f\) and \(\{w_i\}_{i\in J}\) a basis of \(\ker f\).
Let \(\{v_i\}_{i\in I}\) be a family of \(V\) s.t. \(f(v_i)=u_i\) for each \(i\in I\). We contend that
\begin{equation*}
\{v_i,w_j\}_{i\in I,j\in J}
\end{equation*}
is a basis for \(V\)

Let \(x\in V\). Then there exist elements \(\{a_i\}_{i\in I}\) of \(K\) almost all of which are 0 s.t.
\begin{equation*}
f(x)=\sum_{i\in I}a_iu_i
\end{equation*}
Hence \(f(x-\sum a_iv_i)=0\). Thus
\begin{equation*}
x-\sum a_iv_i\in\ker f
\end{equation*}
thus there exists elements \(\{b_j\}_{j\in J}\) of \(K\) almost all of which are 0 s.t.
\begin{equation*}
x-\sum a_iv_i=\sum b_jw_j
\end{equation*}
From this we see that \(x=\sum a_iv_i+\sum b_jw_j\), and that \(\{v_i,w_j\}\) generated \(V\). It remains to
show that the family is linearly independent. Suppose that there exists elements \(c_i,d_j\) s.t.
\begin{equation*}
0=\sum c_iv_i+\sum d_jw_j
\end{equation*}
applying \(f\) yields
\begin{equation*}
0=\sum c_if(v_i)=\sum c_iu_i
\end{equation*}
whence all \(c_i=0\).From this we conclude that all \(d_j=0\)
\end{proof}

\begin{corollary}[]
Let \(V\) be a vector space and \(W\) a subspace. Then
\begin{equation*}
\dim W\le\dim V
\end{equation*}
If \(V\) is finite dimensional and \(\dim W=\dim V\) then \(W=V\)
\end{corollary}


\section{Polynomials}
\label{sec:org7b9ac8a}
\subsection{Basic Properties for Polynomials in One Variable}
\label{sec:orgd0c232e}
\begin{theorem}[]
Let \(A\) be a commutative ring, let \(f,g\in A[X]\)  be polynomials in one variable, of
degree \(\ge 0\), and assume that the leading coefficient of \(g\) is a unit in \(A\). Then there
exist unique polynomials \(q,r\in A[X]\) s.t.
\begin{equation*}
f=gq+r
\end{equation*}
and \(\deg r<\deg g\)
\end{theorem}

\begin{proof}
Write
\begin{align*}
&f(X)=a_nX^n+\dots+a_0\\
&g(X)=b_dX^d+\dots+b_0
\end{align*}
where \(n=\deg f\), \(d=\deg g\) so that \(a_n,b_d\neq 0\) and \(b_d\) is a unit in \(A\). We use
induction on \(n\)

if \(n=0\) and \(\deg g>\deg f\), we let \(q=0\), \(r=f\). If \(\deg g=\deg f=0\), then
let \(r=0\) and \(q=a_nb_d^{-1}\)

Assume the theorem proved for polynomials of degree \(<n\). We may assume \(\deg g\le\deg f\)
(otherwise take \(q=0\) and \(r=f\)). Then
\begin{equation*}
f(X)=a_nb^{-1}_dX^{n-d}g(X)+f_1(X)
\end{equation*}
where \(f_1(X)\) has degree \(<n\). By induction, we can find \(q_1,r\) s.t.
\begin{equation*}
f(X)=a_nb_d^{-1}X^{n-d}g(X)+q_1(X)g(X)+r(X)
\end{equation*}
and \(\deg r<\deg g\). Then we let
\begin{equation*}
q(X)=a_nb_d^{-1}X^{n-d}+q_1(X)
\end{equation*}

For uniqueness, suppose that
\begin{equation*}
f=q_1g+r_1=q_2g+r_2
\end{equation*}
with \(\deg r_1<\deg g\) and \(\deg r_2<\deg g\). Subtracting yields
\begin{equation*}
(q_1-q_2)g=r_2-r_1
\end{equation*}
Since the leading coefficient of \(g\) is assumed to be a unit, we have
\begin{equation*}
\deg(q_1-q_2)g=\deg(q_1-q_2)+\deg g
\end{equation*}
Since \(\deg(r_2-r_1)<\deg g\), this relation can hold only if \(q_1-q_2=0\). Hence \(r_1=r_2\)
\end{proof}

\begin{theorem}[]
Let \(k\) be a field. Then the polynomial ring in one variable \(k[X]\) is principal
\end{theorem}

\begin{proof}
Let \(\fa\) be an ideal of \(k[X]\) and assume \(\fa\neq 0\). Let \(g\) be an element of \(\fa\) of
smallest degree \(\ge 0\). Let \(f\) be an element of \(\fa\) s.t. \(f\neq 0\). By the Euclidean
algorithm we can find \(q,r\in k[X]\) s.t.
\begin{equation*}
f=qg+r
\end{equation*}
and \(\deg r<\deg g\). But \(r=f-qg\) whence \(r\in\fa\). It follows that \(r=0\), hence that \(\fa\)
consists of all polynomials \(qg\).
\end{proof}

A polynomial \(f(X)\in k[X]\) is called \textbf{irreducible} if it has degree \(\ge 1\), and if one cannot
write \(f(X)\) as a product
\begin{equation*}
f(X)=g(X)h(X)
\end{equation*}
with \(g,h\in k[X]\) and both \(g,h\not\in k\). Elements of \(k\) are usually called \textbf{constant
polynomials}. A polynomial is called \textbf{monic} if it has leading coefficient 1

Let \(A\) be a commutative ring and \(f(X)\) a polynomial in \(A[X]\). Let \(A\) be
a subring of \(B\). An element \(b\in B\) is called a \textbf{root} or a \textbf{zero} of \(f\) in \(B\)
if \(f(b)=0\).

\begin{theorem}[]
Let \(k\) be a field and \(f\) a polynomial in one variable \(X\) in \(k[X]\) of degree \(n\ge 0\).
Then \(f\) has at most \(n\) roots in \(k\) and if \(a\) is a root of \(f\) in \(k\),
then \(X-a\) divides \(f(X)\)
\end{theorem}

\begin{proof}
Suppose \(f(a)=0\). Find \(q,r\) s.t.
\begin{equation*}
f(X)=q(X)(X-a)+r(X)
\end{equation*}
and \(\deg r<1\). Then
\begin{equation*}
0=f(a)=r(a)
\end{equation*}
Since \(r=0\) or \(r\) is a non-zero constant, we must have \(r=0\), whence \(X-a\)
divides \(f(X)\).
\end{proof}

\begin{corollary}[]
Let \(k\) be a field and \(T\) an infinite subset of \(k\). Let \(f(X)\in k[X]\) be a polynomial in
one variable. If \(f(a)=0\) for all \(a\in T\), then \(f=0\)
\end{corollary}

\begin{corollary}[]
Let \(k\) be a field, and let \(S_1,\dots,S_n\) be infinite subsets of \(k\). Let \(f(X_1,\dots,X_n)\) be a
polynomial in \(n\) variables over \(k\). If \(f(a_1,\dots,a_n)=0\) for all \(a_i\in S_i\) (\(i=1,\dots,n\)),
then \(f=0\)
\end{corollary}

\begin{proof}
By induction. Let \(n\ge 2\) and write
\begin{equation*}
f(X_1,\dots,X_n)=\sum_jf_i(X_1,\dots,X_{n-1})X^j_n
\end{equation*}
\end{proof}

\begin{corollary}[]
Let \(k\) be an infinite field and \(f\) a polynomial in \(n\) variables over \(k\) . If \(f\)
induces the zero function on \(k^{(n)}\), then \(f=0\)
\end{corollary}

Let \(k\) be a finite field with \(q\) elements. Let \(f(X_1,\dots,X_n)\) be a polynomial in \(n\)
variables over \(k\). Write
\begin{equation*}
f(X_1,\dots,X_n)=\sum a_{\barv}X_1^{v_1}\dots X_n^{v_n}
\end{equation*}
If \(a_{\barv}\neq 0\)  we recall that the monomial \(M_{\barv}(X)\) \textbf{occurs} in \(f\). Suppose this
is the case, and that in this monomial \(M_{\barv}(X)\) some variable \(X_i\) occurs with an
exponent \(v_i\ge q\). We can write
\begin{equation*}
X_i^{v_i}=X_i^{q+\mu}
\end{equation*}
If we replace \(X_i^{v_i}\) by \(X_i^{\mu+1}\) in this monomial, then we obtain a new polynomial which
gives rise to the same function as \(f\). The degree of this new polynomial is at most equal to
the degree of \(f\)

Performing the above operation a finite number of times, for all the monomials occuring in \(f\)
and all the variables \(X_1,\dots,X_n\) we obtain some polynomial \(f^*\) giving rise to the same
function as \(f\), but whose degree in each variable is \(<q\)

\begin{corollary}[]
Let \(k\) be a finite field with \(q\) elements. Let \(f\) be a polynomial in \(n\) variables
over \(k\) s.t. the degree of \(f\) in each variable is \(<q\). If \(f\) induces the zero
function on \(k^n\), then \(f=0\)
\end{corollary}

Let \(f\) be a polynomial in \(n\) variables over the finite field \(k\). A polynomial \(g\)
whose degree in each variable is \(<q\) will be said to be \textbf{reduced}. There exists a unique reduced
polynomial \(f^*\) which gives the same function as \(f\) on \(k^n\)

Let \(k\) be a field. By a \textbf{multiplicative subgroup} of \(k\) we shall mean a subgroup of the
group \(k^*\) (non-zero elements of \(k\))

\begin{theorem}[]
\label{thm1.4.1.9}
Let \(k\) be a field and let \(U\) be a finite multiplicative subgroup of \(k\). Then \(U\) is cyclic
\end{theorem}

\begin{proof}
\ref{prop1.4.3}
Write \(U\) as a product of subgroups \(U(p)\) for each prime \(p\), where \(U(p)\) is a \(p\)-group.
\end{proof}

\begin{corollary}[]
If \(k\) is a finite field, then \(k^*\) is cyclic
\end{corollary}

An element \(\zeta\) in a field \(k\) s.t. there exists an integer \(n\ge 1\) s.t. \(\zeta^n=1\) is called a
\textbf{root of unity}, or \(n\)-th root of unity. Thus the set of \(n\)-th roots of unity is the set of
roots of the polynomial \(X^n-1\). There are at most \(n\) such roots, and they form a group,
which is cyclic by Theorem \ref{thm1.4.1.9}

The group of roots of unity is denoted by \(\mbfmu\). The group of roots of unity in a field \(K\)
is denoted by \(\bmu(K)\)


A field \(k\) is said to be \textbf{algebraically closed} if every polynomial in \(k[X]\) of
degree \(\ge 1\) has a root in \(k\). If \(k\) is algebraically closed then the irreducible
polynomials in \(k[X]\) are the polynomials of degree 1. In such a case, the unique factorization
of a polynomial \(f\) of degree \(\ge 0\)  can be written in the form
\begin{equation*}
f(X)=c\prod_{i=1}^r(X-\alpha_i)^{m_i}
\end{equation*}

Let \(A\) be a commutative ring. We define a map
\begin{equation*}
D:A[X]\to A[X]
\end{equation*}
if \(f(X)=a_nX^n+\dots+a_0\) with \(a_i\in A\), we define the \textbf{derivative}
\begin{equation*}
Df(X)=f'(X)=\sum_{v=1}^nva_vX^{v-1}
\end{equation*}
Let \(K\) be a field and \(f\) a non-zero polynomial in \(K[X]\). Let \(a\) be a root of \(f\)
in \(K\). We can write
\begin{equation*}
f(X)=(X-a)^mg(X)
\end{equation*}
with some polynomial \(g(X)\) relatively prime to \(X-a\). We call \(m\) the \textbf{multiplicity}
of \(a\) in \(f\), and say that \(a\) is a \textbf{multiple root} if \(m>1\)

\begin{proposition}[]
Let \(K,f\) be as above. The element \(a\) of \(K\) is a multiple root of \(f\) iff it is a root
and \(f'(a)=0\)
\end{proposition}

\begin{proposition}[]
Let \(f\in K[X]\). If \(K\) has characteristic 0, and \(f\) has degree \(\ge 1\), then \(f'\neq 0\).
Let \(K\) have characteristic \(p>0\) and \(f\) have degree \(\ge 1\). Then \(f'=0\) iff in the
expression for \(f(X)\) given by
\begin{equation*}
f(X)=\sum_{v=1}^na_vX^v
\end{equation*}
\(p\) divides each integer \(v\) s.t. \(a_v\neq 0\)
\end{proposition}

Since the binomial coefficients \(\binom{p}{v}\) are divisible by \(p\) for \(1\le v\le p-1\) we see
that if \(K\) has characteristic \(p\), then for \(a,b\in K\) we have
\begin{equation*}
(a+b)^p=a^p+b^p
\end{equation*}
Since obviously \((ab)^p=a^pb^p\) the map
\begin{equation*}
x\mapsto x^p
\end{equation*}
is a homomorphism of \(K\) into itself, which has trivial kernel, hence is injective. Iterating,
we conclude that for each integer \(r\ge 1\), the map \(x\mapsto x^{p^r}\) is an endomorphism of \(K\),
called the \textbf{Frobenius endomorphism}.
\subsection{Polynomials Over a Factorial Ring}
\label{sec:orgf6118ef}

\section{Algebraic Extensions}
\label{sec:orgd1fb4cc}
\subsection{Finite and Algebraic Extensions}
\label{sec:org04334b6}
Let \(F\) be a field. If \(F\) is a subfield of a field \(E\), then we also say that \(E\) is an
\textbf{extension field} of \(F\). We may view \(E\) as a vector space over \(F\), and we say \(E\) is
\textbf{finite} or \textbf{infinite} extension of \(F\) according as the dimension of this vector space is finite
or infinite.

Let \(F\) be a subfield of a field \(E\). An element \(\alpha\) of \(E\) is said to be \textbf{algebraic}
over \(F\) if there exists elements \(a_0,\dots,a_n\in F\), not all equal to 0, s.t.
\begin{equation*}
a_0+a_1\alpha+\dots+a_n\alpha^n=0
\end{equation*}
If \(\alpha\neq 0\), and \(\alpha\) is algebraic, then we can always find elements \(a_i\) as above s.t. \(a_0\neq 0\)

Let \(X\) be a variable over \(F\). We can also say that \(\alpha\) is algebraic over \(F\) if the
homomorphism
\begin{equation*}
F[X]\to E
\end{equation*}
which is the identity on \(F\) and maps \(X\) on \(\alpha\) has a non-zero kernel. In that case the kernel
is an ideal which is principal, generated by a single polynomial \(p(X)\), which we may assume
has leading coefficient 1. We then have an isomorphism
\begin{equation*}
F[X]/(p(X))\cong F[\alpha]
\end{equation*}
and since \(F[\alpha]\) is entire, it follows that \(p(X)\) is irreducible. \label{Problem4} Having
normalized \(p(X)\) so that its leading coefficient is 1, we see that \(p(X)\) is uniquely
determined by \(\alpha\) and will be called the \textbf{irreducible polynomial of \(\alpha\) over \(F\)}, denoted by \(\irr(\alpha,F,X)\)

An extension \(E\) of \(F\) is said to be \textbf{algebraic} if every element of \(E\) is algebraic
over \(F\)

\begin{proposition}[]
\label{prop5.1.1}
Let \(E\) be a finite extension of \(F\). Then \(E\) is algebraic over \(F\)
\end{proposition}

\begin{proof}
Let \(\alpha\in E,\alpha\neq 0\). The powers of \(\alpha\)
\begin{equation*}
1,\alpha,\alpha^2,\dots,\alpha^n
\end{equation*}
cannot be linearly independent over \(F\) for all positive integers \(n\), otherwise the
dimension of \(E\) over \(F\) would be infinite. A linear relation between these powers shows
that \(\alpha\) is algebraic over \(F\).
\end{proof}

If \(E\) is an extension of \(F\), we denote by
\begin{equation*}
[E:F]
\end{equation*}
the dimension of \(E\) as a vector space over \(F\).

\begin{proposition}[]
Let \(k\) be a field and \(F\subset E\) extension fields of \(k\). Then
\begin{equation*}
[E:k]=[E:F][F:k]
\end{equation*}
if \(\{x_i\}_{i\in I}\) is a basis for \(F\) over \(k\) and \(\{y_j\}_{j\in J}\) is a basis for \(E\)
over \(F\), then \(\{x_iy_j\}_{(i,j)\in I\times J}\) is a basis for \(E\) over \(k\)
\end{proposition}

\begin{proof}
Let \(z\in E\). By hypothesis there exist elements \(\alpha_j\in F\), almost all \(\alpha_j=0\), s.t.
\begin{equation*}
z=\sum_{j\in J}\alpha_jy_j
\end{equation*}
For each \(j\in J\) there exists elements \(b_{ji}\in k\), almost all of which are equal to 0, s.t.
\begin{equation*}
\alpha_j=\sum_{i\in I}b_{ji}x_i
\end{equation*}
and hence
\begin{equation*}
z=\sum_j\sum_ib_{ji}x_iy_j
\end{equation*}
This shows that \(\{x_iy_j\}\) is a family of generators for \(E\) over \(k\). We must show that it
is linearly independent. Let \(\{c_{ij}\}\) be a family of elements of \(k\), almost all of which
are 0, s.t.
\begin{equation*}
\sum_j\sum_ic_{ij}x_iy_j=0
\end{equation*}
Then for each \(j\)
\begin{equation*}
\sum_ic_{ij}x_i=0
\end{equation*}
since the elements \(y_j\) are linearly independent over \(F\). Hence \(c_{ij}=0\)
\end{proof}

\begin{corollary}[]
\label{cor5.1.3}
The extension \(E\) of \(k\) is finite iff \(E\) is finite over \(F\) and \(F\) is finite over \(k\)
\end{corollary}

A \textbf{tower} of fields is a sequence
\begin{equation*}
F_1\subset F_2\subset\dots\subset F_n
\end{equation*}
of extension fields. The tower is called \textbf{finite} iff each step is finite

Let \(k\) be a field, \(E\) an extension field, and \(\alpha\in E\). We denote by \(k(\alpha)\) the smallest
subfield of \(E\) containing both \(k\) and \(\alpha\). It consists of all quotients \(f(\alpha)/g(\alpha)\)
where \(f,g\) are polynomials with coefficients in \(k\) and \(g(\alpha)\neq 0\).

\begin{proposition}[]
\label{prop5.1.4}
Let \(\alpha\) be algebraic over \(k\). Then \(k(\alpha)=k[\alpha]\), and \(k(\alpha)\) is finite over \(k\). The
degree \([k(\alpha):k]\) is equal to the degree of \(\irr(\alpha,k,X)\)
\end{proposition}

Let \(E,F\) be extensions of a field \(k\). If \(E\) and \(F\) are contained in some field \(L\)
then we denote by \(EF\) the smallest subfield of \(L\) containing both \(E\) and \(F\), and call
it the \textbf{compositum} of \(E\) and \(F\), in \(L\).

Let \(k\) be a subfield of \(E\) and let \(\alpha_1,\dots,\alpha_n\in E\). We denote by
\begin{equation*}
k(\alpha_1,\dots,\alpha_n)
\end{equation*}
the smallest subfield of \(E\) containing \(k\) and \(\alpha_1,\dots,\alpha_n\). Its elements consist of all
quotients
\begin{equation*}
\frac{f(\alpha_1,\dots,f_n)}{g(\alpha_1,\dots,\alpha_n)}
\end{equation*}
where \(f,g\) are polynomials in \(n\) variables with coefficients in \(k\), and
\begin{equation*}
g(\alpha_1,\dots,\alpha_n)\neq 0
\end{equation*}

We observe that \(E\) is the union of all its subfields \(k(\alpha_1,\dots,\alpha_n)\) as \((\alpha_1,\dots,\alpha_n)\) ranges
over finite subfamilies of elements of \(E\). We could define the \textbf{compositum of an arbitrary}
\textbf{subfamily of subfields of a field \(L\)} as the smallest subfield containing all fields in the
family. We say that \(E\) is \textbf{finitely generated} over \(k\) if there is a finite family of
elements \(\alpha_1,\dots,\alpha_n\) of \(E\) s.t.
\begin{equation*}
E=k(\alpha_1,\dots,\alpha_n)
\end{equation*}

\begin{proposition}[]
Let \(E\) be a finite extension of \(k\). Then \(E\) is finitely generated
\end{proposition}

\begin{proof}
Let \(\{\alpha_1,\dots,\alpha_n\}\) be a basis of \(E\) as vector space over \(k\). Then certainly
\begin{equation*}
E=k(\alpha_1,\dots,\alpha_n)
\end{equation*}
\end{proof}

If \(E=k(\alpha_1,\dots,\alpha_n)\) is finitely generated, and \(F\) is an extension of \(k\), both \(F,E\)
contained in \(L\), then
\begin{equation*}
EF=F(\alpha_1,\dots,\alpha_n)
\end{equation*}
and \(EF\) is finitely generated over \(F\)
\begin{center}\begin{tikzcd}
&EF&\\
&&F\ar[ul,dash]\ar[ddl,dash]\\
E\ar[uur,dash]\ar[dr,dash]&&\\
&k
\end{tikzcd}\end{center}

Lines slanting up indicate an inclusion relation between fields. We also call the
extension \(EF\) of \(F\) the \textbf{translation} of \(E\) to \(F\), or also the \textbf{lifting} of \(E\)
to \(F\)

Let \(\alpha\) be algebraic over the field \(k\). Let \(F\) be an extension of \(k\), and
assume \(k(\alpha)\), \(F\) both contained in some field \(L\). Then \(\alpha\) is algebraic over \(F\).
Consider the irreducible polynomial for \(\alpha\).

Suppose that we have a tower of fields
\begin{equation*}
k\subset k(\alpha_1)\subset k(\alpha_1,\alpha_2)\subset\dots\subset k(\alpha_1,\dots,\alpha_n)
\end{equation*}
each one generated from the preceding field by a single element. Assume that each \(\alpha_i\) is
algebraic over \(k\), \(i=1,\dots,n\). As a special case of our preceding remark, we note
that \(\alpha_{i+1}\) is algebraic over \(k(\alpha_1,\dots,\alpha_i)\). Hence each step of the tower is algebraic

\begin{proposition}[]
Let \(E=k(\alpha_1,\dots,\alpha_n)\) be a finitely genrated extension of a field \(k\), and assume \(\alpha_i\)
algebraic over \(k\) for each \(i=1,\dots,n\). Then \(E\) is finite algebraic over \(k\)
\end{proposition}

\begin{proof}
\(E\) is finite by Proposition \ref{prop5.1.4} and Corollary \ref{cor5.1.3}. Algebraic by Proposition \ref{prop5.1.1}
\end{proof}

Let \(\calc\) be a certain class of extension fields \(F\subset E\). \(\calc\) is \textbf{distinguished} if it satisfies
the following conditions
\begin{enumerate}
\item Let \(k\subset F\subset E\) be a tower of fields. The extension \(k\subset E\) is in \(\calc\) iff \(k\subset F\) is
in \(\calc\) and \(F\subset E\) is in \(\calc\)
\item if \(k\subset E\) is in \(\calc\), if \(F\) is any extension of \(k\), and \(E,F\) are both contained in
some field, then \(F\subset EF\) is in \(\calc\)
\item if \(k\subset F\) and \(k\subset E\) are in \(\calc\) and \(F,E\) are subfields of a common field,
then \(k\subset FE\) is in \(\calc\)
\end{enumerate}



\begin{center}
   \begin{tikzcd}
E\ar[d,dash]\\
F\ar[d,dash]\\
k
\end{tikzcd}\quad
\begin{tikzcd}
&EF\ar[ddl,dash]\ar[dr,dash]\\
&&F\\
E\\
&k\ar[ul,dash]\ar[uur,dash]
\end{tikzcd}\hspace{1cm}
\begin{tikzcd}
&EF\ar[dl,dash]\ar[rd,dash]\\
E&&F\\
&k\ar[ul,dash]\ar[ur,dash]
\end{tikzcd}
\end{center}
\end{equation*}

It is convenient to write \(E/F\) instead of \(F\subset E\) to denote an extension

\begin{proposition}[]
The class of algebraic extensions is distinguished, and so is the class of finite extensions
\end{proposition}
\subsection{Algebraic Closure}
\label{sec:org15df0d5}
Let \(E\) be an extension of a field \(F\) and let
\begin{equation*}
\sigma:F\to L
\end{equation*}
be an embedding (i.e. an injective homomorphism) of \(F\) into \(L\). Then \(\sigma\) induces an
isomorphism of \(F\) with its image \(\sigma F\), which is sometimes written \(F^\sigma\). An embedding
\(\tau\) of \(E\) in \(L\) will be said to be \textbf{over} \(\sigma\) if the restriction of \(\tau\) to \(F\) is equal
to \(\sigma\). We also say that \(\tau\) \textbf{extends} \(\sigma\). If \(\sigma\) is the identity then we say that \(\tau\) is an
embedding of \(E\) \textbf{over} \(F\)
\begin{center}
\begin{tikzcd}
E\ar[r,"\tau"]&L\\
F\ar[u,"\text{inc}"]\ar[r,"\sigma"']&L\ar[u,"\id"']
\end{tikzcd}\hspace{2cm}
\begin{tikzcd}[column sep=small]
E\ar[rr,"\tau"]&&L\\
&F\ar[ul,"\text{inc}"]\ar[ur,"\text{inc}"']
\end{tikzcd}
\end{center}
\section{Real Fields}
\label{sec:org048cf5f}
\subsection{Ordered Fields}
\label{sec:orgaf9d4e6}
Let \(K\) be a field. An \textbf{ordering} of \(K\) is a subset \(P\) of \(K\)
having the following properties
\bigskip
\begin{itemize}[itemindent=3em]
\item[\textbf{ORD 1.}] Given \(x\in K\), we have either \(x\in P\) ,or \(x=0\) or
\(-x\in P\), and these three possibilities are mutually exclusive
\item[\textbf{ORD 2.}] If \(x,y\in P\), then \(x+y,xy\in P\)
\end{itemize}

\(K\) is \textbf{ordered by} \(P\), and we call \(P\) the set of \textbf{positive
elements}

Suppose \(K\) is ordered by \(P\). Since \(1\neq0\) and \(1=1^2=(-1)^2\), we
see that \(1\in P\). By \textbf{ORD 2}, it follows that \(1+\dots+1\in P\), whence \(K\)
has characteristic 0. If \(x\in P\) and \(x\neq0\), then \(xx^{-1}=1\in P\) implies
that \(x^{-1}\in P\)

\begin{center}
\emph{Let \(E\) be a field. Then a product of sums of squares in \(E\) is a sum
of squares.}

\emph{If \(a,b\in E\) are sum of squares and \(b\neq0\), then \(a/b\) is a sum of
squares}
\end{center}

Consider complex number:)

Let \(x,y\in K\). We define \(x<y\) to mean that \(y-x\in P\). If \(x<0\) we say
that \(x\) is \textbf{negative}.

If \(K\) is ordered and \(x\in K\), \(x\neq0\), then \(x^2\) is positive

If \(E\) has characteristic \(\neq2\), and \(-1\) is a sum of squares in \(E\),
then every element \(a\in E\) is a sum of squares, because
\(4a=(1+a)^2-(1-a)^2\)

If \(K\) is a field with an ordering \(P\), and \(F\) is a subfield, then
obviously, \(P\cap F\) defines an ordering of \(F\), which is called the
\textbf{induced} ordering

Let \(K\) be an ordered field and let \(F\) be a subfield with the induced
ordering. We put \(\abs{x}=x\) if \(x>0\) and \(\abs{x}=-x\) if \(x<0\). An
element \(\alpha\in K\) is \textbf{infinitely large} over \(F\) if \(\abs{\alpha}\ge x\) for all
\(x\in F\). It is \textbf{infinitely small} over \(F\) if \(0\le\abs{\alpha}\le\abs{x}\) for
all \(x\in F\), \(x\neq0\). \(\alpha\) is infinitely large if and only if \(\alpha^{-1}\) is
infinitely small. \(K\) is \textbf{archimedean} over \(F\) if \(K\) has no elements
which are infinitely large over \(F\). An intermediate field \(F_1\),
\(K\supset F_1\supset F\) is \textbf{maximal archimedean over} \(F\) in \(K\) if it is
archimedean over \(F\) and no other intermediate field containing \(F_1\) is
archimedean over \(F\). We say that \(F\) is \textbf{maximal archimedean in} \(K\)
if it is maximal archimedean over itself in \(K\)

Let \(K\) be an ordered field and \(F\) a subfield. Let \(K\) be an ordered
field and \(F\) a subfield. Let \(\fo\) be the set of elements of \(K\)
which are not infinitely large over \(F\). Then \(\fo\) is a ring and that
for any \(\alpha\in K\), we have \(\alpha\) or \(\alpha^{-1}\in\fo\). Hence \(\fo\) is what is
called a valuation ring, containing \(F\). Let \(\fm\) be the ideal of all
\(\alpha\in K\) which are infinitely small over \(F\). Then \(\fm\) is the unique
maximal ideal of \(\fo\), because any element in \(\fo\) which is not in
\(\fm\) has an inverse in \(\fo\). We call \(\fo\) the
\textbf{valuation ring determined by the ordering of} \(K/F\)

\begin{proposition}[]
Let \(K\) be an ordered field and \(F\) a subfield. Let \(\fo\) be the
valuation ring determined by the ordering of \(K/F\), and let \(\fm\) be its
maximal ideal. Then \(\fo/\fm\) is a real field.
\end{proposition}

\begin{proof}
Otherwise, we could write
\begin{equation*}
-1=\displaystyle\sum\alpha_i^2+a
\end{equation*}
with \(\alpha_i\in\fo\) and \(a\in\fm\). Since \(\sum\alpha_i^2\) is positive and \(a\) is
infinitely small, such a relation is clearly impossible
\end{proof}
\end{document}
