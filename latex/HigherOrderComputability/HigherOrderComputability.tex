% Created 2022-03-28 Mon 16:37
% Intended LaTeX compiler: pdflatex
\documentclass[11pt]{article}
\usepackage[utf8]{inputenc}
\usepackage[T1]{fontenc}
\usepackage{graphicx}
\usepackage{longtable}
\usepackage{wrapfig}
\usepackage{rotating}
\usepackage[normalem]{ulem}
\usepackage{amsmath}
\usepackage{amssymb}
\usepackage{capt-of}
\usepackage{hyperref}
\graphicspath{{../../books/}}
% wrong resolution of image
% https://tex.stackexchange.com/questions/21627/image-from-includegraphics-showing-in-wrong-image-size?rq=1

%%%%%%%%%%%%%%%%%%%%%%%%%%%%%%%%%%%%%%
%% TIPS                                 %%
%%%%%%%%%%%%%%%%%%%%%%%%%%%%%%%%%%%%%%
% \substack{a\\b} for multiple lines text
% \usepackage{expl3}
% \expandafter\def\csname ver@l3regex.sty\endcsname{}
% \usepackage{pkgloader}
\usepackage[utf8]{inputenc}

% nfss error
% \usepackage[B1,T1]{fontenc}
\usepackage{fontspec}

% \usepackage[Emoticons]{ucharclasses}
\newfontfamily\DejaSans{DejaVu Sans}
% \setDefaultTransitions{\DejaSans}{}

% pdfplots will load xolor automatically without option
\usepackage[dvipsnames]{xcolor}

%                                                             ┳┳┓   ┓
%                                                             ┃┃┃┏┓╋┣┓
%                                                             ┛ ┗┗┻┗┛┗
% \usepackage{amsmath} mathtools loads the amsmath
\usepackage{amsmath}
\usepackage{mathtools}

\usepackage{amsthm}
\usepackage{amsbsy}

%\usepackage{commath}

\usepackage{amssymb}

\usepackage{mathrsfs}
%\usepackage{mathabx}
\usepackage{stmaryrd}
\usepackage{empheq}

\usepackage{scalerel}
\usepackage{stackengine}
\usepackage{stackrel}



\usepackage{nicematrix}
\usepackage{tensor}
\usepackage{blkarray}
\usepackage{siunitx}
\usepackage[f]{esvect}

% centering \not on a letter
\usepackage{slashed}
\usepackage[makeroom]{cancel}

%\usepackage{merriweather}
\usepackage{unicode-math}
\setmainfont{TeX Gyre Pagella}
% \setmathfont{STIX}
%\setmathfont{texgyrepagella-math.otf}
%\setmathfont{Libertinus Math}
\setmathfont{Latin Modern Math}

 % \setmathfont[range={\smwhtdiamond,\enclosediamond,\varlrtriangle}]{Latin Modern Math}
\setmathfont[range={\rightrightarrows,\twoheadrightarrow,\leftrightsquigarrow,\triangledown,\vartriangle,\precneq,\succneq,\prec,\succ,\preceq,\succeq,\tieconcat}]{XITS Math}
 \setmathfont[range={\int,\setminus}]{Libertinus Math}
 % \setmathfont[range={\mathalpha}]{TeX Gyre Pagella Math}
%\setmathfont[range={\mitA,\mitB,\mitC,\mitD,\mitE,\mitF,\mitG,\mitH,\mitI,\mitJ,\mitK,\mitL,\mitM,\mitN,\mitO,\mitP,\mitQ,\mitR,\mitS,\mitT,\mitU,\mitV,\mitW,\mitX,\mitY,\mitZ,\mita,\mitb,\mitc,\mitd,\mite,\mitf,\mitg,\miti,\mitj,\mitk,\mitl,\mitm,\mitn,\mito,\mitp,\mitq,\mitr,\mits,\mitt,\mitu,\mitv,\mitw,\mitx,\mity,\mitz}]{TeX Gyre Pagella Math}
% unicode is not good at this!
%\let\nmodels\nvDash

 \usepackage{wasysym}

 % for wide hat
 \DeclareSymbolFont{yhlargesymbols}{OMX}{yhex}{m}{n} \DeclareMathAccent{\what}{\mathord}{yhlargesymbols}{"62}

%                                                               ┏┳┓•┓
%                                                                ┃ ┓┃┏┓
%                                                                ┻ ┗┛┗┗

\usepackage{pgfplots}
\pgfplotsset{compat=1.18}
\usepackage{tikz}
\usepackage{tikz-cd}
\tikzcdset{scale cd/.style={every label/.append style={scale=#1},
    cells={nodes={scale=#1}}}}
% TODO: discard qtree and use forest
% \usepackage{tikz-qtree}
\usepackage{forest}

\usetikzlibrary{arrows,positioning,calc,fadings,decorations,matrix,decorations,shapes.misc}
%setting from geogebra
\definecolor{ccqqqq}{rgb}{0.8,0,0}

%                                                          ┳┳┓•    ┓┓
%                                                          ┃┃┃┓┏┏┏┓┃┃┏┓┏┓┏┓┏┓┓┏┏
%                                                          ┛ ┗┗┛┗┗ ┗┗┗┻┛┗┗ ┗┛┗┻┛
%\usepackage{twemojis}
\usepackage[most]{tcolorbox}
\usepackage{threeparttable}
\usepackage{tabularx}

\usepackage{enumitem}
\usepackage[indLines=false]{algpseudocodex}
\usepackage[]{algorithm2e}
% \SetKwComment{Comment}{/* }{ */}
% \algrenewcommand\algorithmicrequire{\textbf{Input:}}
% \algrenewcommand\algorithmicensure{\textbf{Output:}}
% wrong with preview
\usepackage{subcaption}
\usepackage{caption}
% {\aunclfamily\Huge}
\usepackage{auncial}

\usepackage{float}

\usepackage{fancyhdr}

\usepackage{ifthen}
\usepackage{xargs}

\definecolor{mintedbg}{rgb}{0.99,0.99,0.99}
\usepackage[cachedir=\detokenize{~/miscellaneous/trash}]{minted}
\setminted{breaklines,
  mathescape,
  bgcolor=mintedbg,
  fontsize=\footnotesize,
  frame=single,
  linenos}
\usemintedstyle{xcode}
\usepackage{tcolorbox}
\usepackage{etoolbox}



\usepackage{imakeidx}
\usepackage{hyperref}
\usepackage{soul}
\usepackage{framed}

% don't use this for preview
%\usepackage[margin=1.5in]{geometry}
% \usepackage{geometry}
% \geometry{legalpaper, landscape, margin=1in}
\usepackage[font=itshape]{quoting}

%\LoadPackagesNow
%\usepackage[xetex]{preview}
%%%%%%%%%%%%%%%%%%%%%%%%%%%%%%%%%%%%%%%
%% USEPACKAGES end                       %%
%%%%%%%%%%%%%%%%%%%%%%%%%%%%%%%%%%%%%%%

%%%%%%%%%%%%%%%%%%%%%%%%%%%%%%%%%%%%%%%
%% Algorithm environment
%%%%%%%%%%%%%%%%%%%%%%%%%%%%%%%%%%%%%%%
\SetKwIF{Recv}{}{}{upon receiving}{do}{}{}{}
\SetKwBlock{Init}{initially do}{}
\SetKwProg{Function}{Function}{:}{}

% https://github.com/chrmatt/algpseudocodex/issues/3
\algnewcommand\algorithmicswitch{\textbf{switch}}%
\algnewcommand\algorithmiccase{\textbf{case}}
\algnewcommand\algorithmicof{\textbf{of}}
\algnewcommand\algorithmicotherwise{\texttt{otherwise} $\Rightarrow$}

\makeatletter
\algdef{SE}[SWITCH]{Switch}{EndSwitch}[1]{\algpx@startIndent\algpx@startCodeCommand\algorithmicswitch\ #1\ \algorithmicdo}{\algpx@endIndent\algpx@startCodeCommand\algorithmicend\ \algorithmicswitch}%
\algdef{SE}[CASE]{Case}{EndCase}[1]{\algpx@startIndent\algpx@startCodeCommand\algorithmiccase\ #1}{\algpx@endIndent\algpx@startCodeCommand\algorithmicend\ \algorithmiccase}%
\algdef{SE}[CASEOF]{CaseOf}{EndCaseOf}[1]{\algpx@startIndent\algpx@startCodeCommand\algorithmiccase\ #1 \algorithmicof}{\algpx@endIndent\algpx@startCodeCommand\algorithmicend\ \algorithmiccase}
\algdef{SE}[OTHERWISE]{Otherwise}{EndOtherwise}[0]{\algpx@startIndent\algpx@startCodeCommand\algorithmicotherwise}{\algpx@endIndent\algpx@startCodeCommand\algorithmicend\ \algorithmicotherwise}
\ifbool{algpx@noEnd}{%
  \algtext*{EndSwitch}%
  \algtext*{EndCase}%
  \algtext*{EndCaseOf}
  \algtext*{EndOtherwise}
  %
  % end indent line after (not before), to get correct y position for multiline text in last command
  \apptocmd{\EndSwitch}{\algpx@endIndent}{}{}%
  \apptocmd{\EndCase}{\algpx@endIndent}{}{}%
  \apptocmd{\EndCaseOf}{\algpx@endIndent}{}{}
  \apptocmd{\EndOtherwise}{\algpx@endIndent}{}{}
}{}%

\pretocmd{\Switch}{\algpx@endCodeCommand}{}{}
\pretocmd{\Case}{\algpx@endCodeCommand}{}{}
\pretocmd{\CaseOf}{\algpx@endCodeCommand}{}{}
\pretocmd{\Otherwise}{\algpx@endCodeCommand}{}{}

% for end commands that may not be printed, tell endCodeCommand whether we are using noEnd
\ifbool{algpx@noEnd}{%
  \pretocmd{\EndSwitch}{\algpx@endCodeCommand[1]}{}{}%
  \pretocmd{\EndCase}{\algpx@endCodeCommand[1]}{}{}
  \pretocmd{\EndCaseOf}{\algpx@endCodeCommand[1]}{}{}%
  \pretocmd{\EndOtherwise}{\algpx@endCodeCommand[1]}{}{}
}{%
  \pretocmd{\EndSwitch}{\algpx@endCodeCommand[0]}{}{}%
  \pretocmd{\EndCase}{\algpx@endCodeCommand[0]}{}{}%
  \pretocmd{\EndCaseOf}{\algpx@endCodeCommand[0]}{}{}
  \pretocmd{\EndOtherwise}{\algpx@endCodeCommand[0]}{}{}
}%
\makeatother
% % For algpseudocode
% \algnewcommand\algorithmicswitch{\textbf{switch}}
% \algnewcommand\algorithmiccase{\textbf{case}}
% \algnewcommand\algorithmiccaseof{\textbf{case}}
% \algnewcommand\algorithmicof{\textbf{of}}
% % New "environments"
% \algdef{SE}[SWITCH]{Switch}{EndSwitch}[1]{\algorithmicswitch\ #1\ \algorithmicdo}{\algorithmicend\ \algorithmicswitch}%
% \algdef{SE}[CASE]{Case}{EndCase}[1]{\algorithmiccase\ #1}{\algorithmicend\ \algorithmiccase}%
% \algtext*{EndSwitch}%
% \algtext*{EndCase}
% \algdef{SE}[CASEOF]{CaseOf}{EndCaseOf}[1]{\algorithmiccaseof\ #1 \algorithmicof}{\algorithmicend\ \algorithmiccaseof}
% \algtext*{EndCaseOf}



%\pdfcompresslevel0

% quoting from
% https://tex.stackexchange.com/questions/391726/the-quotation-environment
\NewDocumentCommand{\bywhom}{m}{% the Bourbaki trick
  {\nobreak\hfill\penalty50\hskip1em\null\nobreak
   \hfill\mbox{\normalfont(#1)}%
   \parfillskip=0pt \finalhyphendemerits=0 \par}%
}

\NewDocumentEnvironment{pquotation}{m}
  {\begin{quoting}[
     indentfirst=true,
     leftmargin=\parindent,
     rightmargin=\parindent]\itshape}
  {\bywhom{#1}\end{quoting}}

\indexsetup{othercode=\small}
\makeindex[columns=2,options={-s /media/wu/file/stuuudy/notes/index_style.ist},intoc]
\makeatletter
\def\@idxitem{\par\hangindent 0pt}
\makeatother


% \newcounter{dummy} \numberwithin{dummy}{section}
\newtheorem{dummy}{dummy}[section]
\theoremstyle{definition}
\newtheorem{definition}[dummy]{Definition}
\theoremstyle{plain}
\newtheorem{corollary}[dummy]{Corollary}
\newtheorem{lemma}[dummy]{Lemma}
\newtheorem{proposition}[dummy]{Proposition}
\newtheorem{theorem}[dummy]{Theorem}
\newtheorem{notation}[dummy]{Notation}
\newtheorem{conjecture}[dummy]{Conjecture}
\newtheorem{fact}[dummy]{Fact}
\newtheorem{warning}[dummy]{Warning}
\theoremstyle{definition}
\newtheorem{examplle}{Example}[section]
\theoremstyle{remark}
\newtheorem*{remark}{Remark}
\newtheorem{exercise}{Exercise}[subsection]
\newtheorem{problem}{Problem}[subsection]
\newtheorem{observation}{Observation}[section]
\newenvironment{claim}[1]{\par\noindent\textbf{Claim:}\space#1}{}

\makeatletter
\DeclareFontFamily{U}{tipa}{}
\DeclareFontShape{U}{tipa}{m}{n}{<->tipa10}{}
\newcommand{\arc@char}{{\usefont{U}{tipa}{m}{n}\symbol{62}}}%

\newcommand{\arc}[1]{\mathpalette\arc@arc{#1}}

\newcommand{\arc@arc}[2]{%
  \sbox0{$\m@th#1#2$}%
  \vbox{
    \hbox{\resizebox{\wd0}{\height}{\arc@char}}
    \nointerlineskip
    \box0
  }%
}
\makeatother

\setcounter{MaxMatrixCols}{20}
%%%%%%% ABS
\DeclarePairedDelimiter\abss{\lvert}{\rvert}%
\DeclarePairedDelimiter\normm{\lVert}{\rVert}%

% Swap the definition of \abs* and \norm*, so that \abs
% and \norm resizes the size of the brackets, and the
% starred version does not.
\makeatletter
\let\oldabs\abss
%\def\abs{\@ifstar{\oldabs}{\oldabs*}}
\newcommand{\abs}{\@ifstar{\oldabs}{\oldabs*}}
\newcommand{\norm}[1]{\left\lVert#1\right\rVert}
%\let\oldnorm\normm
%\def\norm{\@ifstar{\oldnorm}{\oldnorm*}}
%\renewcommand{norm}{\@ifstar{\oldnorm}{\oldnorm*}}
\makeatother

% \stackMath
% \newcommand\what[1]{%
% \savestack{\tmpbox}{\stretchto{%
%   \scaleto{%
%     \scalerel*[\widthof{\ensuremath{#1}}]{\kern-.6pt\bigwedge\kern-.6pt}%
%     {\rule[-\textheight/2]{1ex}{\textheight}}%WIDTH-LIMITED BIG WEDGE
%   }{\textheight}%
% }{0.5ex}}%
% \stackon[1pt]{#1}{\tmpbox}%
% }

% \newcommand\what[1]{\ThisStyle{%
%     \setbox0=\hbox{$\SavedStyle#1$}%
%     \stackengine{-1.0\ht0+.5pt}{$\SavedStyle#1$}{%
%       \stretchto{\scaleto{\SavedStyle\mkern.15mu\char'136}{2.6\wd0}}{1.4\ht0}%
%     }{O}{c}{F}{T}{S}%
%   }
% }

% \newcommand\wtilde[1]{\ThisStyle{%
%     \setbox0=\hbox{$\SavedStyle#1$}%
%     \stackengine{-.1\LMpt}{$\SavedStyle#1$}{%
%       \stretchto{\scaleto{\SavedStyle\mkern.2mu\AC}{.5150\wd0}}{.6\ht0}%
%     }{O}{c}{F}{T}{S}%
%   }
% }

% \newcommand\wbar[1]{\ThisStyle{%
%     \setbox0=\hbox{$\SavedStyle#1$}%
%     \stackengine{.5pt+\LMpt}{$\SavedStyle#1$}{%
%       \rule{\wd0}{\dimexpr.3\LMpt+.3pt}%
%     }{O}{c}{F}{T}{S}%
%   }
% }

\newcommand{\bl}[1] {\boldsymbol{#1}}
\newcommand{\Wt}[1] {\stackrel{\sim}{\smash{#1}\rule{0pt}{1.1ex}}}
\newcommand{\wt}[1] {\widetilde{#1}}
\newcommand{\tf}[1] {\textbf{#1}}

\newcommand{\wu}[1]{{\color{red} #1}}

%For boxed texts in align, use Aboxed{}
%otherwise use boxed{}

\DeclareMathSymbol{\widehatsym}{\mathord}{largesymbols}{"62}
\newcommand\lowerwidehatsym{%
  \text{\smash{\raisebox{-1.3ex}{%
    $\widehatsym$}}}}
\newcommand\fixwidehat[1]{%
  \mathchoice
    {\accentset{\displaystyle\lowerwidehatsym}{#1}}
    {\accentset{\textstyle\lowerwidehatsym}{#1}}
    {\accentset{\scriptstyle\lowerwidehatsym}{#1}}
    {\accentset{\scriptscriptstyle\lowerwidehatsym}{#1}}
  }


\newcommand{\cupdot}{\mathbin{\dot{\cup}}}
\newcommand{\bigcupdot}{\mathop{\dot{\bigcup}}}

\usepackage{graphicx}

\usepackage[toc,page]{appendix}

% text on arrow for xRightarrow
\makeatletter
%\newcommand{\xRightarrow}[2][]{\ext@arrow 0359\Rightarrowfill@{#1}{#2}}
\makeatother

% Arbitrary long arrow
\newcommand{\Rarrow}[1]{%
\parbox{#1}{\tikz{\draw[->](0,0)--(#1,0);}}
}

\newcommand{\LRarrow}[1]{%
\parbox{#1}{\tikz{\draw[<->](0,0)--(#1,0);}}
}


\makeatletter
\providecommand*{\rmodels}{%
  \mathrel{%
    \mathpalette\@rmodels\models
  }%
}
\newcommand*{\@rmodels}[2]{%
  \reflectbox{$\m@th#1#2$}%
}
\makeatother

% Roman numerals
\makeatletter
\newcommand*{\rom}[1]{\expandafter\@slowromancap\romannumeral #1@}
\makeatother
% \\def \\b\([a-zA-Z]\) {\\boldsymbol{[a-zA-z]}}
% \\DeclareMathOperator{\\b\1}{\\textbf{\1}}

\DeclareMathOperator*{\argmin}{arg\,min}
\DeclareMathOperator*{\argmax}{arg\,max}

\DeclareMathOperator{\bone}{\textbf{1}}
\DeclareMathOperator{\bx}{\textbf{x}}
\DeclareMathOperator{\bz}{\textbf{z}}
\DeclareMathOperator{\bff}{\textbf{f}}
\DeclareMathOperator{\ba}{\textbf{a}}
\DeclareMathOperator{\bk}{\textbf{k}}
\DeclareMathOperator{\bs}{\textbf{s}}
\DeclareMathOperator{\bh}{\textbf{h}}
\DeclareMathOperator{\bc}{\textbf{c}}
\DeclareMathOperator{\br}{\textbf{r}}
\DeclareMathOperator{\bi}{\textbf{i}}
\DeclareMathOperator{\bj}{\textbf{j}}
\DeclareMathOperator{\bn}{\textbf{n}}
\DeclareMathOperator{\be}{\textbf{e}}
\DeclareMathOperator{\bo}{\textbf{o}}
\DeclareMathOperator{\bU}{\textbf{U}}
\DeclareMathOperator{\bL}{\textbf{L}}
\DeclareMathOperator{\bV}{\textbf{V}}
\def \bzero {\mathbf{0}}
\def \bbone {\mathbb{1}}
\def \btwo {\mathbf{2}}
\DeclareMathOperator{\bv}{\textbf{v}}
\DeclareMathOperator{\bp}{\textbf{p}}
\DeclareMathOperator{\bI}{\textbf{I}}
\def \dbI {\dot{\bI}}
\DeclareMathOperator{\bM}{\textbf{M}}
\DeclareMathOperator{\bN}{\textbf{N}}
\DeclareMathOperator{\bK}{\textbf{K}}
\DeclareMathOperator{\bt}{\textbf{t}}
\DeclareMathOperator{\bb}{\textbf{b}}
\DeclareMathOperator{\bA}{\textbf{A}}
\DeclareMathOperator{\bX}{\textbf{X}}
\DeclareMathOperator{\bu}{\textbf{u}}
\DeclareMathOperator{\bS}{\textbf{S}}
\DeclareMathOperator{\bZ}{\textbf{Z}}
\DeclareMathOperator{\bJ}{\textbf{J}}
\DeclareMathOperator{\by}{\textbf{y}}
\DeclareMathOperator{\bw}{\textbf{w}}
\DeclareMathOperator{\bT}{\textbf{T}}
\DeclareMathOperator{\bF}{\textbf{F}}
\DeclareMathOperator{\bmm}{\textbf{m}}
\DeclareMathOperator{\bW}{\textbf{W}}
\DeclareMathOperator{\bR}{\textbf{R}}
\DeclareMathOperator{\bC}{\textbf{C}}
\DeclareMathOperator{\bD}{\textbf{D}}
\DeclareMathOperator{\bE}{\textbf{E}}
\DeclareMathOperator{\bQ}{\textbf{Q}}
\DeclareMathOperator{\bP}{\textbf{P}}
\DeclareMathOperator{\bY}{\textbf{Y}}
\DeclareMathOperator{\bH}{\textbf{H}}
\DeclareMathOperator{\bB}{\textbf{B}}
\DeclareMathOperator{\bG}{\textbf{G}}
\def \blambda {\symbf{\lambda}}
\def \boldeta {\symbf{\eta}}
\def \balpha {\symbf{\alpha}}
\def \btau {\symbf{\tau}}
\def \bbeta {\symbf{\beta}}
\def \bgamma {\symbf{\gamma}}
\def \bxi {\symbf{\xi}}
\def \bLambda {\symbf{\Lambda}}
\def \bGamma {\symbf{\Gamma}}

\newcommand{\bto}{{\boldsymbol{\to}}}
\newcommand{\Ra}{\Rightarrow}
\newcommand{\xrsa}[1]{\overset{#1}{\rightsquigarrow}}
\newcommand{\xlsa}[1]{\overset{#1}{\leftsquigarrow}}
\newcommand\und[1]{\underline{#1}}
\newcommand\ove[1]{\overline{#1}}
%\def \concat {\verb|^|}
\def \bPhi {\mbfPhi}
\def \btheta {\mbftheta}
\def \bTheta {\mbfTheta}
\def \bmu {\mbfmu}
\def \bphi {\mbfphi}
\def \bSigma {\mbfSigma}
\def \la {\langle}
\def \ra {\rangle}

\def \caln {\mathcal{N}}
\def \dissum {\displaystyle\Sigma}
\def \dispro {\displaystyle\prod}

\def \caret {\verb!^!}

\def \A {\mathbb{A}}
\def \B {\mathbb{B}}
\def \C {\mathbb{C}}
\def \D {\mathbb{D}}
\def \E {\mathbb{E}}
\def \F {\mathbb{F}}
\def \G {\mathbb{G}}
\def \H {\mathbb{H}}
\def \I {\mathbb{I}}
\def \J {\mathbb{J}}
\def \K {\mathbb{K}}
\def \L {\mathbb{L}}
\def \M {\mathbb{M}}
\def \N {\mathbb{N}}
\def \O {\mathbb{O}}
\def \P {\mathbb{P}}
\def \Q {\mathbb{Q}}
\def \R {\mathbb{R}}
\def \S {\mathbb{S}}
\def \T {\mathbb{T}}
\def \U {\mathbb{U}}
\def \V {\mathbb{V}}
\def \W {\mathbb{W}}
\def \X {\mathbb{X}}
\def \Y {\mathbb{Y}}
\def \Z {\mathbb{Z}}

\def \cala {\mathcal{A}}
\def \cale {\mathcal{E}}
\def \calb {\mathcal{B}}
\def \calq {\mathcal{Q}}
\def \calp {\mathcal{P}}
\def \cals {\mathcal{S}}
\def \calx {\mathcal{X}}
\def \caly {\mathcal{Y}}
\def \calg {\mathcal{G}}
\def \cald {\mathcal{D}}
\def \caln {\mathcal{N}}
\def \calr {\mathcal{R}}
\def \calt {\mathcal{T}}
\def \calm {\mathcal{M}}
\def \calw {\mathcal{W}}
\def \calc {\mathcal{C}}
\def \calv {\mathcal{V}}
\def \calf {\mathcal{F}}
\def \calk {\mathcal{K}}
\def \call {\mathcal{L}}
\def \calu {\mathcal{U}}
\def \calo {\mathcal{O}}
\def \calh {\mathcal{H}}
\def \cali {\mathcal{I}}
\def \calj {\mathcal{J}}

\def \bcup {\bigcup}

% set theory

\def \zfcc {\textbf{ZFC}^-}
\def \BGC {\textbf{BGC}}
\def \BG {\textbf{BG}}
\def \ac  {\textbf{AC}}
\def \gl  {\textbf{L }}
\def \gll {\textbf{L}}
\newcommand{\zfm}{$\textbf{ZF}^-$}

\def \ZFm {\text{ZF}^-}
\def \ZFCm {\text{ZFC}^-}
\DeclareMathOperator{\WF}{WF}
\DeclareMathOperator{\On}{On}
\def \on {\textbf{On }}
\def \cm {\textbf{M }}
\def \cn {\textbf{N }}
\def \cv {\textbf{V }}
\def \zc {\textbf{ZC }}
\def \zcm {\textbf{ZC}}
\def \zff {\textbf{ZF}}
\def \wfm {\textbf{WF}}
\def \onm {\textbf{On}}
\def \cmm {\textbf{M}}
\def \cnm {\textbf{N}}
\def \cvm {\textbf{V}}

\renewcommand{\restriction}{\mathord{\upharpoonright}}
%% another restriction
\newcommand\restr[2]{{% we make the whole thing an ordinary symbol
  \left.\kern-\nulldelimiterspace % automatically resize the bar with \right
  #1 % the function
  \vphantom{\big|} % pretend it's a little taller at normal size
  \right|_{#2} % this is the delimiter
  }}

\def \pred {\text{pred}}

\def \rank {\text{rank}}
\def \Con {\text{Con}}
\def \deff {\text{Def}}


\def \uin {\underline{\in}}
\def \oin {\overline{\in}}
\def \uR {\underline{R}}
\def \oR {\overline{R}}
\def \uP {\underline{P}}
\def \oP {\overline{P}}

\def \dsum {\displaystyle\sum}

\def \Ra {\Rightarrow}

\def \e {\enspace}

\def \sgn {\operatorname{sgn}}
\def \gen {\operatorname{gen}}
\def \Hom {\operatorname{Hom}}
\def \hom {\operatorname{hom}}
\def \Sub {\operatorname{Sub}}

\def \supp {\operatorname{supp}}

\def \epiarrow {\twoheadarrow}
\def \monoarrow {\rightarrowtail}
\def \rrarrow {\rightrightarrows}

% \def \minus {\text{-}}
% \newcommand{\minus}{\scalebox{0.75}[1.0]{$-$}}
% \DeclareUnicodeCharacter{002D}{\minus}


\def \tril {\triangleleft}

\def \ISigma {\text{I}\Sigma}
\def \IDelta {\text{I}\Delta}
\def \IPi {\text{I}\Pi}
\def \ACF {\textsf{ACF}}
\def \pCF {\textit{p}\text{CF}}
\def \ACVF {\textsf{ACVF}}
\def \HLR {\textsf{HLR}}
\def \OAG {\textsf{OAG}}
\def \RCF {\textsf{RCF}}
\DeclareMathOperator{\GL}{GL}
\DeclareMathOperator{\PGL}{PGL}
\DeclareMathOperator{\SL}{SL}
\DeclareMathOperator{\Inv}{Inv}
\DeclareMathOperator{\res}{res}
\DeclareMathOperator{\Sym}{Sym}
%\DeclareMathOperator{\char}{char}
\def \equal {=}

\def \degree {\text{degree}}
\def \app {\text{App}}
\def \FV {\text{FV}}
\def \conv {\text{conv}}
\def \cont {\text{cont}}
\DeclareMathOperator{\cl}{\text{cl}}
\DeclareMathOperator{\trcl}{\text{trcl}}
\DeclareMathOperator{\sg}{sg}
\DeclareMathOperator{\trdeg}{trdeg}
\def \Ord {\text{Ord}}

\DeclareMathOperator{\cf}{cf}
\DeclareMathOperator{\zfc}{ZFC}

%\DeclareMathOperator{\Th}{Th}
%\def \th {\text{Th}}
% \newcommand{\th}{\text{Th}}
\DeclareMathOperator{\type}{type}
\DeclareMathOperator{\zf}{\textbf{ZF}}
\def \fa {\mathfrak{a}}
\def \fb {\mathfrak{b}}
\def \fc {\mathfrak{c}}
\def \fd {\mathfrak{d}}
\def \fe {\mathfrak{e}}
\def \ff {\mathfrak{f}}
\def \fg {\mathfrak{g}}
\def \fh {\mathfrak{h}}
%\def \fi {\mathfrak{i}}
\def \fj {\mathfrak{j}}
\def \fk {\mathfrak{k}}
\def \fl {\mathfrak{l}}
\def \fm {\mathfrak{m}}
\def \fn {\mathfrak{n}}
\def \fo {\mathfrak{o}}
\def \fp {\mathfrak{p}}
\def \fq {\mathfrak{q}}
\def \fr {\mathfrak{r}}
\def \fs {\mathfrak{s}}
\def \ft {\mathfrak{t}}
\def \fu {\mathfrak{u}}
\def \fv {\mathfrak{v}}
\def \fw {\mathfrak{w}}
\def \fx {\mathfrak{x}}
\def \fy {\mathfrak{y}}
\def \fz {\mathfrak{z}}
\def \fA {\mathfrak{A}}
\def \fB {\mathfrak{B}}
\def \fC {\mathfrak{C}}
\def \fD {\mathfrak{D}}
\def \fE {\mathfrak{E}}
\def \fF {\mathfrak{F}}
\def \fG {\mathfrak{G}}
\def \fH {\mathfrak{H}}
\def \fI {\mathfrak{I}}
\def \fJ {\mathfrak{J}}
\def \fK {\mathfrak{K}}
\def \fL {\mathfrak{L}}
\def \fM {\mathfrak{M}}
\def \fN {\mathfrak{N}}
\def \fO {\mathfrak{O}}
\def \fP {\mathfrak{P}}
\def \fQ {\mathfrak{Q}}
\def \fR {\mathfrak{R}}
\def \fS {\mathfrak{S}}
\def \fT {\mathfrak{T}}
\def \fU {\mathfrak{U}}
\def \fV {\mathfrak{V}}
\def \fW {\mathfrak{W}}
\def \fX {\mathfrak{X}}
\def \fY {\mathfrak{Y}}
\def \fZ {\mathfrak{Z}}

\def \sfA {\textsf{A}}
\def \sfB {\textsf{B}}
\def \sfC {\textsf{C}}
\def \sfD {\textsf{D}}
\def \sfE {\textsf{E}}
\def \sfF {\textsf{F}}
\def \sfG {\textsf{G}}
\def \sfH {\textsf{H}}
\def \sfI {\textsf{I}}
\def \sfJ {\textsf{J}}
\def \sfK {\textsf{K}}
\def \sfL {\textsf{L}}
\def \sfM {\textsf{M}}
\def \sfN {\textsf{N}}
\def \sfO {\textsf{O}}
\def \sfP {\textsf{P}}
\def \sfQ {\textsf{Q}}
\def \sfR {\textsf{R}}
\def \sfS {\textsf{S}}
\def \sfT {\textsf{T}}
\def \sfU {\textsf{U}}
\def \sfV {\textsf{V}}
\def \sfW {\textsf{W}}
\def \sfX {\textsf{X}}
\def \sfY {\textsf{Y}}
\def \sfZ {\textsf{Z}}
\def \sfa {\textsf{a}}
\def \sfb {\textsf{b}}
\def \sfc {\textsf{c}}
\def \sfd {\textsf{d}}
\def \sfe {\textsf{e}}
\def \sff {\textsf{f}}
\def \sfg {\textsf{g}}
\def \sfh {\textsf{h}}
\def \sfi {\textsf{i}}
\def \sfj {\textsf{j}}
\def \sfk {\textsf{k}}
\def \sfl {\textsf{l}}
\def \sfm {\textsf{m}}
\def \sfn {\textsf{n}}
\def \sfo {\textsf{o}}
\def \sfp {\textsf{p}}
\def \sfq {\textsf{q}}
\def \sfr {\textsf{r}}
\def \sfs {\textsf{s}}
\def \sft {\textsf{t}}
\def \sfu {\textsf{u}}
\def \sfv {\textsf{v}}
\def \sfw {\textsf{w}}
\def \sfx {\textsf{x}}
\def \sfy {\textsf{y}}
\def \sfz {\textsf{z}}

\def \ttA {\texttt{A}}
\def \ttB {\texttt{B}}
\def \ttC {\texttt{C}}
\def \ttD {\texttt{D}}
\def \ttE {\texttt{E}}
\def \ttF {\texttt{F}}
\def \ttG {\texttt{G}}
\def \ttH {\texttt{H}}
\def \ttI {\texttt{I}}
\def \ttJ {\texttt{J}}
\def \ttK {\texttt{K}}
\def \ttL {\texttt{L}}
\def \ttM {\texttt{M}}
\def \ttN {\texttt{N}}
\def \ttO {\texttt{O}}
\def \ttP {\texttt{P}}
\def \ttQ {\texttt{Q}}
\def \ttR {\texttt{R}}
\def \ttS {\texttt{S}}
\def \ttT {\texttt{T}}
\def \ttU {\texttt{U}}
\def \ttV {\texttt{V}}
\def \ttW {\texttt{W}}
\def \ttX {\texttt{X}}
\def \ttY {\texttt{Y}}
\def \ttZ {\texttt{Z}}
\def \tta {\texttt{a}}
\def \ttb {\texttt{b}}
\def \ttc {\texttt{c}}
\def \ttd {\texttt{d}}
\def \tte {\texttt{e}}
\def \ttf {\texttt{f}}
\def \ttg {\texttt{g}}
\def \tth {\texttt{h}}
\def \tti {\texttt{i}}
\def \ttj {\texttt{j}}
\def \ttk {\texttt{k}}
\def \ttl {\texttt{l}}
\def \ttm {\texttt{m}}
\def \ttn {\texttt{n}}
\def \tto {\texttt{o}}
\def \ttp {\texttt{p}}
\def \ttq {\texttt{q}}
\def \ttr {\texttt{r}}
\def \tts {\texttt{s}}
\def \ttt {\texttt{t}}
\def \ttu {\texttt{u}}
\def \ttv {\texttt{v}}
\def \ttw {\texttt{w}}
\def \ttx {\texttt{x}}
\def \tty {\texttt{y}}
\def \ttz {\texttt{z}}

\def \bara {\bbar{a}}
\def \barb {\bbar{b}}
\def \barc {\bbar{c}}
\def \bard {\bbar{d}}
\def \bare {\bbar{e}}
\def \barf {\bbar{f}}
\def \barg {\bbar{g}}
\def \barh {\bbar{h}}
\def \bari {\bbar{i}}
\def \barj {\bbar{j}}
\def \bark {\bbar{k}}
\def \barl {\bbar{l}}
\def \barm {\bbar{m}}
\def \barn {\bbar{n}}
\def \baro {\bbar{o}}
\def \barp {\bbar{p}}
\def \barq {\bbar{q}}
\def \barr {\bbar{r}}
\def \bars {\bbar{s}}
\def \bart {\bbar{t}}
\def \baru {\bbar{u}}
\def \barv {\bbar{v}}
\def \barw {\bbar{w}}
\def \barx {\bbar{x}}
\def \bary {\bbar{y}}
\def \barz {\bbar{z}}
\def \barA {\bbar{A}}
\def \barB {\bbar{B}}
\def \barC {\bbar{C}}
\def \barD {\bbar{D}}
\def \barE {\bbar{E}}
\def \barF {\bbar{F}}
\def \barG {\bbar{G}}
\def \barH {\bbar{H}}
\def \barI {\bbar{I}}
\def \barJ {\bbar{J}}
\def \barK {\bbar{K}}
\def \barL {\bbar{L}}
\def \barM {\bbar{M}}
\def \barN {\bbar{N}}
\def \barO {\bbar{O}}
\def \barP {\bbar{P}}
\def \barQ {\bbar{Q}}
\def \barR {\bbar{R}}
\def \barS {\bbar{S}}
\def \barT {\bbar{T}}
\def \barU {\bbar{U}}
\def \barVV {\bbar{V}}
\def \barW {\bbar{W}}
\def \barX {\bbar{X}}
\def \barY {\bbar{Y}}
\def \barZ {\bbar{Z}}

\def \baralpha {\bbar{\alpha}}
\def \bartau {\bbar{\tau}}
\def \barsigma {\bbar{\sigma}}
\def \barzeta {\bbar{\zeta}}

\def \hata {\hat{a}}
\def \hatb {\hat{b}}
\def \hatc {\hat{c}}
\def \hatd {\hat{d}}
\def \hate {\hat{e}}
\def \hatf {\hat{f}}
\def \hatg {\hat{g}}
\def \hath {\hat{h}}
\def \hati {\hat{i}}
\def \hatj {\hat{j}}
\def \hatk {\hat{k}}
\def \hatl {\hat{l}}
\def \hatm {\hat{m}}
\def \hatn {\hat{n}}
\def \hato {\hat{o}}
\def \hatp {\hat{p}}
\def \hatq {\hat{q}}
\def \hatr {\hat{r}}
\def \hats {\hat{s}}
\def \hatt {\hat{t}}
\def \hatu {\hat{u}}
\def \hatv {\hat{v}}
\def \hatw {\hat{w}}
\def \hatx {\hat{x}}
\def \haty {\hat{y}}
\def \hatz {\hat{z}}
\def \hatA {\hat{A}}
\def \hatB {\hat{B}}
\def \hatC {\hat{C}}
\def \hatD {\hat{D}}
\def \hatE {\hat{E}}
\def \hatF {\hat{F}}
\def \hatG {\hat{G}}
\def \hatH {\hat{H}}
\def \hatI {\hat{I}}
\def \hatJ {\hat{J}}
\def \hatK {\hat{K}}
\def \hatL {\hat{L}}
\def \hatM {\hat{M}}
\def \hatN {\hat{N}}
\def \hatO {\hat{O}}
\def \hatP {\hat{P}}
\def \hatQ {\hat{Q}}
\def \hatR {\hat{R}}
\def \hatS {\hat{S}}
\def \hatT {\hat{T}}
\def \hatU {\hat{U}}
\def \hatVV {\hat{V}}
\def \hatW {\hat{W}}
\def \hatX {\hat{X}}
\def \hatY {\hat{Y}}
\def \hatZ {\hat{Z}}

\def \hatphi {\hat{\phi}}

\def \barfM {\bbar{\fM}}
\def \barfN {\bbar{\fN}}

\def \tila {\tilde{a}}
\def \tilb {\tilde{b}}
\def \tilc {\tilde{c}}
\def \tild {\tilde{d}}
\def \tile {\tilde{e}}
\def \tilf {\tilde{f}}
\def \tilg {\tilde{g}}
\def \tilh {\tilde{h}}
\def \tili {\tilde{i}}
\def \tilj {\tilde{j}}
\def \tilk {\tilde{k}}
\def \till {\tilde{l}}
\def \tilm {\tilde{m}}
\def \tiln {\tilde{n}}
\def \tilo {\tilde{o}}
\def \tilp {\tilde{p}}
\def \tilq {\tilde{q}}
\def \tilr {\tilde{r}}
\def \tils {\tilde{s}}
\def \tilt {\tilde{t}}
\def \tilu {\tilde{u}}
\def \tilv {\tilde{v}}
\def \tilw {\tilde{w}}
\def \tilx {\tilde{x}}
\def \tily {\tilde{y}}
\def \tilz {\tilde{z}}
\def \tilA {\tilde{A}}
\def \tilB {\tilde{B}}
\def \tilC {\tilde{C}}
\def \tilD {\tilde{D}}
\def \tilE {\tilde{E}}
\def \tilF {\tilde{F}}
\def \tilG {\tilde{G}}
\def \tilH {\tilde{H}}
\def \tilI {\tilde{I}}
\def \tilJ {\tilde{J}}
\def \tilK {\tilde{K}}
\def \tilL {\tilde{L}}
\def \tilM {\tilde{M}}
\def \tilN {\tilde{N}}
\def \tilO {\tilde{O}}
\def \tilP {\tilde{P}}
\def \tilQ {\tilde{Q}}
\def \tilR {\tilde{R}}
\def \tilS {\tilde{S}}
\def \tilT {\tilde{T}}
\def \tilU {\tilde{U}}
\def \tilVV {\tilde{V}}
\def \tilW {\tilde{W}}
\def \tilX {\tilde{X}}
\def \tilY {\tilde{Y}}
\def \tilZ {\tilde{Z}}

\def \tilalpha {\tilde{\alpha}}
\def \tilPhi {\tilde{\Phi}}

\def \barnu {\bar{\nu}}
\def \barrho {\bar{\rho}}
%\DeclareMathOperator{\ker}{ker}
\DeclareMathOperator{\im}{im}

\DeclareMathOperator{\Inn}{Inn}
\DeclareMathOperator{\rel}{rel}
\def \dote {\stackrel{\cdot}=}
%\DeclareMathOperator{\AC}{\textbf{AC}}
\DeclareMathOperator{\cod}{cod}
\DeclareMathOperator{\dom}{dom}
\DeclareMathOperator{\card}{card}
\DeclareMathOperator{\ran}{ran}
\DeclareMathOperator{\textd}{d}
\DeclareMathOperator{\td}{d}
\DeclareMathOperator{\id}{id}
\DeclareMathOperator{\LT}{LT}
\DeclareMathOperator{\Mat}{Mat}
\DeclareMathOperator{\Eq}{Eq}
\DeclareMathOperator{\irr}{irr}
\DeclareMathOperator{\Fr}{Fr}
\DeclareMathOperator{\Gal}{Gal}
\DeclareMathOperator{\lcm}{lcm}
\DeclareMathOperator{\alg}{\text{alg}}
\DeclareMathOperator{\Th}{Th}
%\DeclareMathOperator{\deg}{deg}


% \varprod
\DeclareSymbolFont{largesymbolsA}{U}{txexa}{m}{n}
\DeclareMathSymbol{\varprod}{\mathop}{largesymbolsA}{16}
% \DeclareMathSymbol{\tonm}{\boldsymbol{\to}\textbf{Nm}}
\def \tonm {\bto\textbf{Nm}}
\def \tohm {\bto\textbf{Hm}}

% Category theory
\DeclareMathOperator{\ob}{ob}
\DeclareMathOperator{\Ab}{\textbf{Ab}}
\DeclareMathOperator{\Alg}{\textbf{Alg}}
\DeclareMathOperator{\Rng}{\textbf{Rng}}
\DeclareMathOperator{\Sets}{\textbf{Sets}}
\DeclareMathOperator{\Set}{\textbf{Set}}
\DeclareMathOperator{\Grp}{\textbf{Grp}}
\DeclareMathOperator{\Met}{\textbf{Met}}
\DeclareMathOperator{\BA}{\textbf{BA}}
\DeclareMathOperator{\Mon}{\textbf{Mon}}
\DeclareMathOperator{\Top}{\textbf{Top}}
\DeclareMathOperator{\hTop}{\textbf{hTop}}
\DeclareMathOperator{\HTop}{\textbf{HTop}}
\DeclareMathOperator{\Aut}{\text{Aut}}
\DeclareMathOperator{\RMod}{R-\textbf{Mod}}
\DeclareMathOperator{\RAlg}{R-\textbf{Alg}}
\DeclareMathOperator{\LF}{LF}
\DeclareMathOperator{\op}{op}
\DeclareMathOperator{\Rings}{\textbf{Rings}}
\DeclareMathOperator{\Ring}{\textbf{Ring}}
\DeclareMathOperator{\Groups}{\textbf{Groups}}
\DeclareMathOperator{\Group}{\textbf{Group}}
\DeclareMathOperator{\ev}{ev}
% Algebraic Topology
\DeclareMathOperator{\obj}{obj}
\DeclareMathOperator{\Spec}{Spec}
\DeclareMathOperator{\spec}{spec}
% Model theory
\DeclareMathOperator*{\ind}{\raise0.2ex\hbox{\ooalign{\hidewidth$\vert$\hidewidth\cr\raise-0.9ex\hbox{$\smile$}}}}
\def\nind{\cancel{\ind}}
\DeclareMathOperator{\acl}{acl}
\DeclareMathOperator{\tspan}{span}
\DeclareMathOperator{\acleq}{acl^{\eq}}
\DeclareMathOperator{\Av}{Av}
\DeclareMathOperator{\ded}{ded}
\DeclareMathOperator{\EM}{EM}
\DeclareMathOperator{\dcl}{dcl}
\DeclareMathOperator{\Ext}{Ext}
\DeclareMathOperator{\eq}{eq}
\DeclareMathOperator{\ER}{ER}
\DeclareMathOperator{\tp}{tp}
\DeclareMathOperator{\stp}{stp}
\DeclareMathOperator{\qftp}{qftp}
\DeclareMathOperator{\Diag}{Diag}
\DeclareMathOperator{\MD}{MD}
\DeclareMathOperator{\MR}{MR}
\DeclareMathOperator{\RM}{RM}
\DeclareMathOperator{\el}{el}
\DeclareMathOperator{\depth}{depth}
\DeclareMathOperator{\ZFC}{ZFC}
\DeclareMathOperator{\GCH}{GCH}
\DeclareMathOperator{\Inf}{Inf}
\DeclareMathOperator{\Pow}{Pow}
\DeclareMathOperator{\ZF}{ZF}
\DeclareMathOperator{\CH}{CH}
\def \FO {\text{FO}}
\DeclareMathOperator{\fin}{fin}
\DeclareMathOperator{\qr}{qr}
\DeclareMathOperator{\Mod}{Mod}
\DeclareMathOperator{\Def}{Def}
\DeclareMathOperator{\TC}{TC}
\DeclareMathOperator{\KH}{KH}
\DeclareMathOperator{\Part}{Part}
\DeclareMathOperator{\Infset}{\textsf{Infset}}
\DeclareMathOperator{\DLO}{\textsf{DLO}}
\DeclareMathOperator{\PA}{\textsf{PA}}
\DeclareMathOperator{\DAG}{\textsf{DAG}}
\DeclareMathOperator{\ODAG}{\textsf{ODAG}}
\DeclareMathOperator{\sfMod}{\textsf{Mod}}
\DeclareMathOperator{\AbG}{\textsf{AbG}}
\DeclareMathOperator{\sfACF}{\textsf{ACF}}
\DeclareMathOperator{\DCF}{\textsf{DCF}}
% Computability Theorem
\DeclareMathOperator{\Tot}{Tot}
\DeclareMathOperator{\graph}{graph}
\DeclareMathOperator{\Fin}{Fin}
\DeclareMathOperator{\Cof}{Cof}
\DeclareMathOperator{\lh}{lh}
% Commutative Algebra
\DeclareMathOperator{\ord}{ord}
\DeclareMathOperator{\Idem}{Idem}
\DeclareMathOperator{\zdiv}{z.div}
\DeclareMathOperator{\Frac}{Frac}
\DeclareMathOperator{\rad}{rad}
\DeclareMathOperator{\nil}{nil}
\DeclareMathOperator{\Ann}{Ann}
\DeclareMathOperator{\End}{End}
\DeclareMathOperator{\coim}{coim}
\DeclareMathOperator{\coker}{coker}
\DeclareMathOperator{\Bil}{Bil}
\DeclareMathOperator{\Tril}{Tril}
\DeclareMathOperator{\tchar}{char}
\DeclareMathOperator{\tbd}{bd}

% Topology
\DeclareMathOperator{\diam}{diam}
\newcommand{\interior}[1]{%
  {\kern0pt#1}^{\mathrm{o}}%
}

\DeclareMathOperator*{\bigdoublewedge}{\bigwedge\mkern-15mu\bigwedge}
\DeclareMathOperator*{\bigdoublevee}{\bigvee\mkern-15mu\bigvee}

% \makeatletter
% \newcommand{\vect}[1]{%
%   \vbox{\m@th \ialign {##\crcr
%   \vectfill\crcr\noalign{\kern-\p@ \nointerlineskip}
%   $\hfil\displaystyle{#1}\hfil$\crcr}}}
% \def\vectfill{%
%   $\m@th\smash-\mkern-7mu%
%   \cleaders\hbox{$\mkern-2mu\smash-\mkern-2mu$}\hfill
%   \mkern-7mu\raisebox{-3.81pt}[\p@][\p@]{$\mathord\mathchar"017E$}$}

% \newcommand{\amsvect}{%
%   \mathpalette {\overarrow@\vectfill@}}
% \def\vectfill@{\arrowfill@\relbar\relbar{\raisebox{-3.81pt}[\p@][\p@]{$\mathord\mathchar"017E$}}}

% \newcommand{\amsvectb}{%
% \newcommand{\vect}{%
%   \mathpalette {\overarrow@\vectfillb@}}
% \newcommand{\vecbar}{%
%   \scalebox{0.8}{$\relbar$}}
% \def\vectfillb@{\arrowfill@\vecbar\vecbar{\raisebox{-4.35pt}[\p@][\p@]{$\mathord\mathchar"017E$}}}
% \makeatother
% \bigtimes

\DeclareFontFamily{U}{mathx}{\hyphenchar\font45}
\DeclareFontShape{U}{mathx}{m}{n}{
      <5> <6> <7> <8> <9> <10>
      <10.95> <12> <14.4> <17.28> <20.74> <24.88>
      mathx10
      }{}
\DeclareSymbolFont{mathx}{U}{mathx}{m}{n}
\DeclareMathSymbol{\bigtimes}{1}{mathx}{"91}
% \odiv
\DeclareFontFamily{U}{matha}{\hyphenchar\font45}
\DeclareFontShape{U}{matha}{m}{n}{
      <5> <6> <7> <8> <9> <10> gen * matha
      <10.95> matha10 <12> <14.4> <17.28> <20.74> <24.88> matha12
      }{}
\DeclareSymbolFont{matha}{U}{matha}{m}{n}
\DeclareMathSymbol{\odiv}         {2}{matha}{"63}


\newcommand\subsetsim{\mathrel{%
  \ooalign{\raise0.2ex\hbox{\scalebox{0.9}{$\subset$}}\cr\hidewidth\raise-0.85ex\hbox{\scalebox{0.9}{$\sim$}}\hidewidth\cr}}}
\newcommand\simsubset{\mathrel{%
  \ooalign{\raise-0.2ex\hbox{\scalebox{0.9}{$\subset$}}\cr\hidewidth\raise0.75ex\hbox{\scalebox{0.9}{$\sim$}}\hidewidth\cr}}}

\newcommand\simsubsetsim{\mathrel{%
  \ooalign{\raise0ex\hbox{\scalebox{0.8}{$\subset$}}\cr\hidewidth\raise1ex\hbox{\scalebox{0.75}{$\sim$}}\hidewidth\cr\raise-0.95ex\hbox{\scalebox{0.8}{$\sim$}}\cr\hidewidth}}}
\newcommand{\stcomp}[1]{{#1}^{\mathsf{c}}}

\setlength{\baselineskip}{0.5in}

\stackMath
\newcommand\yrightarrow[2][]{\mathrel{%
  \setbox2=\hbox{\stackon{\scriptstyle#1}{\scriptstyle#2}}%
  \stackunder[0pt]{%
    \xrightarrow{\makebox[\dimexpr\wd2\relax]{$\scriptstyle#2$}}%
  }{%
   \scriptstyle#1\,%
  }%
}}
\newcommand\yleftarrow[2][]{\mathrel{%
  \setbox2=\hbox{\stackon{\scriptstyle#1}{\scriptstyle#2}}%
  \stackunder[0pt]{%
    \xleftarrow{\makebox[\dimexpr\wd2\relax]{$\scriptstyle#2$}}%
  }{%
   \scriptstyle#1\,%
  }%
}}
\newcommand\yRightarrow[2][]{\mathrel{%
  \setbox2=\hbox{\stackon{\scriptstyle#1}{\scriptstyle#2}}%
  \stackunder[0pt]{%
    \xRightarrow{\makebox[\dimexpr\wd2\relax]{$\scriptstyle#2$}}%
  }{%
   \scriptstyle#1\,%
  }%
}}
\newcommand\yLeftarrow[2][]{\mathrel{%
  \setbox2=\hbox{\stackon{\scriptstyle#1}{\scriptstyle#2}}%
  \stackunder[0pt]{%
    \xLeftarrow{\makebox[\dimexpr\wd2\relax]{$\scriptstyle#2$}}%
  }{%
   \scriptstyle#1\,%
  }%
}}

\newcommand\altxrightarrow[2][0pt]{\mathrel{\ensurestackMath{\stackengine%
  {\dimexpr#1-7.5pt}{\xrightarrow{\phantom{#2}}}{\scriptstyle\!#2\,}%
  {O}{c}{F}{F}{S}}}}
\newcommand\altxleftarrow[2][0pt]{\mathrel{\ensurestackMath{\stackengine%
  {\dimexpr#1-7.5pt}{\xleftarrow{\phantom{#2}}}{\scriptstyle\!#2\,}%
  {O}{c}{F}{F}{S}}}}

\newenvironment{bsm}{% % short for 'bracketed small matrix'
  \left[ \begin{smallmatrix} }{%
  \end{smallmatrix} \right]}

\newenvironment{psm}{% % short for ' small matrix'
  \left( \begin{smallmatrix} }{%
  \end{smallmatrix} \right)}

\newcommand{\bbar}[1]{\mkern 1.5mu\overline{\mkern-1.5mu#1\mkern-1.5mu}\mkern 1.5mu}

\newcommand{\bigzero}{\mbox{\normalfont\Large\bfseries 0}}
\newcommand{\rvline}{\hspace*{-\arraycolsep}\vline\hspace*{-\arraycolsep}}

\font\zallman=Zallman at 40pt
\font\elzevier=Elzevier at 40pt

\newcommand\isoto{\stackrel{\textstyle\sim}{\smash{\longrightarrow}\rule{0pt}{0.4ex}}}
\newcommand\embto{\stackrel{\textstyle\prec}{\smash{\longrightarrow}\rule{0pt}{0.4ex}}}

% from http://www.actual.world/resources/tex/doc/TikZ.pdf

\tikzset{
modal/.style={>=stealth’,shorten >=1pt,shorten <=1pt,auto,node distance=1.5cm,
semithick},
world/.style={circle,draw,minimum size=0.5cm,fill=gray!15},
point/.style={circle,draw,inner sep=0.5mm,fill=black},
reflexive above/.style={->,loop,looseness=7,in=120,out=60},
reflexive below/.style={->,loop,looseness=7,in=240,out=300},
reflexive left/.style={->,loop,looseness=7,in=150,out=210},
reflexive right/.style={->,loop,looseness=7,in=30,out=330}
}


\makeatletter
\newcommand*{\doublerightarrow}[2]{\mathrel{
  \settowidth{\@tempdima}{$\scriptstyle#1$}
  \settowidth{\@tempdimb}{$\scriptstyle#2$}
  \ifdim\@tempdimb>\@tempdima \@tempdima=\@tempdimb\fi
  \mathop{\vcenter{
    \offinterlineskip\ialign{\hbox to\dimexpr\@tempdima+1em{##}\cr
    \rightarrowfill\cr\noalign{\kern.5ex}
    \rightarrowfill\cr}}}\limits^{\!#1}_{\!#2}}}
\newcommand*{\triplerightarrow}[1]{\mathrel{
  \settowidth{\@tempdima}{$\scriptstyle#1$}
  \mathop{\vcenter{
    \offinterlineskip\ialign{\hbox to\dimexpr\@tempdima+1em{##}\cr
    \rightarrowfill\cr\noalign{\kern.5ex}
    \rightarrowfill\cr\noalign{\kern.5ex}
    \rightarrowfill\cr}}}\limits^{\!#1}}}
\makeatother

% $A\doublerightarrow{a}{bcdefgh}B$

% $A\triplerightarrow{d_0,d_1,d_2}B$

\def \uhr {\upharpoonright}
\def \rhu {\rightharpoonup}
\def \uhl {\upharpoonleft}


\newcommand{\floor}[1]{\lfloor #1 \rfloor}
\newcommand{\ceil}[1]{\lceil #1 \rceil}
\newcommand{\lcorner}[1]{\llcorner #1 \lrcorner}
\newcommand{\llb}[1]{\llbracket #1 \rrbracket}
\newcommand{\ucorner}[1]{\ulcorner #1 \urcorner}
\newcommand{\emoji}[1]{{\DejaSans #1}}
\newcommand{\vprec}{\rotatebox[origin=c]{-90}{$\prec$}}

\newcommand{\nat}[6][large]{%
  \begin{tikzcd}[ampersand replacement = \&, column sep=#1]
    #2\ar[bend left=40,""{name=U}]{r}{#4}\ar[bend right=40,',""{name=D}]{r}{#5}\& #3
          \ar[shorten <=10pt,shorten >=10pt,Rightarrow,from=U,to=D]{d}{~#6}
    \end{tikzcd}
}


\providecommand\rightarrowRHD{\relbar\joinrel\mathrel\RHD}
\providecommand\rightarrowrhd{\relbar\joinrel\mathrel\rhd}
\providecommand\longrightarrowRHD{\relbar\joinrel\relbar\joinrel\mathrel\RHD}
\providecommand\longrightarrowrhd{\relbar\joinrel\relbar\joinrel\mathrel\rhd}
\def \lrarhd {\longrightarrowrhd}


\makeatletter
\providecommand*\xrightarrowRHD[2][]{\ext@arrow 0055{\arrowfill@\relbar\relbar\longrightarrowRHD}{#1}{#2}}
\providecommand*\xrightarrowrhd[2][]{\ext@arrow 0055{\arrowfill@\relbar\relbar\longrightarrowrhd}{#1}{#2}}
\makeatother

\newcommand{\metalambda}{%
  \mathop{%
    \rlap{$\lambda$}%
    \mkern3mu
    \raisebox{0ex}{$\lambda$}%
  }%
}

%% https://tex.stackexchange.com/questions/15119/draw-horizontal-line-left-and-right-of-some-text-a-single-line
\newcommand*\ruleline[1]{\par\noindent\raisebox{.8ex}{\makebox[\linewidth]{\hrulefill\hspace{1ex}\raisebox{-.8ex}{#1}\hspace{1ex}\hrulefill}}}

% https://www.dickimaw-books.com/latex/novices/html/newenv.html
\newenvironment{Block}[1]% environment name
{% begin code
  % https://tex.stackexchange.com/questions/19579/horizontal-line-spanning-the-entire-document-in-latex
  \noindent\textcolor[RGB]{128,128,128}{\rule{\linewidth}{1pt}}
  \par\noindent
  {\Large\textbf{#1}}%
  \bigskip\par\noindent\ignorespaces
}%
{% end code
  \par\noindent
  \textcolor[RGB]{128,128,128}{\rule{\linewidth}{1pt}}
  \ignorespacesafterend
}

\mathchardef\mhyphen="2D % Define a "math hyphen"

\def \QQ {\quad}
\def \QW {​\quad}

\newcommand{\ssmile}[1]{\mathord{\stackrel{\smallsmile}{#1}}}
\DeclareMathOperator{\lv}{lv}
\newcommand{\FF}{f\mspace{-7mu}f}
\newcommand{\TT}{t\mspace{-3mu}t}
\newcommand{\IF}{i\mspace{-4mu}f}
\DeclareMathOperator{\Asm}{\mathcal{A}sm}
\DeclareMathOperator{\nMod}{\mathcal{M}od}
\makeindex
\author{John Longley \& Dag Normann}
\date{\today}
\title{Higher Order Computability}
\hypersetup{
 pdfauthor={John Longley \& Dag Normann},
 pdftitle={Higher Order Computability},
 pdfkeywords={},
 pdfsubject={},
 pdfcreator={Emacs 28.0.92 (Org mode 9.6)}, 
 pdflang={English}}
\begin{document}

\maketitle
\tableofcontents


\section{Theory of Computability Models}
\label{sec:orga9b4bc4}
\subsection{Notations}
\label{sec:orgc508af1}
\begin{itemize}
\item \(e\downarrow\) 'the value of \(e\) is defined'
\item \(e\uparrow\) 'the value of \(e\) is undefined'
\item \(e=e'\) 'the values of both \(e\) and \(e'\) are defined and they are equal'
\item \(e\simeq e'\) 'if either \(e\) or \(e'\) is defined then so is the other and they are equal'
\item \(e\succeq e'\) 'if \(e'\) is defined then so is \(e\) and they are equal'
\end{itemize}


if \(e\) is a mathematical expression possibly involving the variable \(x\), we write \(\Lambda x.e\)
to mean the ordinary (possibly partial) function \(f\) defined by \(f(x)\simeq e\)

Finite sequences of length \(n\) starts from index 0.
\subsection{Higher-Order Computability Models}
\label{sec:orgfacab31}
\subsubsection{Computability Models}
\label{sec:org39ef3d5}
\begin{definition}[]
\label{3.1.1}
A \textbf{computability model} \(\bC\)  over a set \(\sfT\) of \textbf{type names} consists of
\begin{itemize}
\item an indexed family \(\abs{\bC}=\{\bC(\tau)\mid\tau\in\sfT\}\) of sets, called the \textbf{datatypes} of \(\bC\)
\item for each \(\sigma,\tau\in\sfT\), a set \(\bC[\sigma,\tau]\) of partial functions \(f:\bC(\sigma)\to\bC(\tau)\), called the
\textbf{operations} of \(\bC\)
\end{itemize}


s.t.
\begin{enumerate}
\item for each \(\tau\in\sfT\), the identity function \(\id:\bC(\tau)\to\bC(\tau)\) is in \(\bC(\tau,\tau)\)
\item for any \(f\in\bC[\rho,\sigma]\) and \(g\in\bC[\sigma,\tau]\) we have \(g\circ f\in\bC[\rho,\tau]\) where \(\circ\) denotes ordinary
composition of partial functions
\end{enumerate}
\end{definition}

We shall use uppercase letters \(A,B,C,\dots\) to denote \textbf{occurrences} of sets within \(\abs{\bC}\):
that is, sets \(\bC(\tau)\) implicitly tagged with a type name \(\tau\). We shall write \(\bC[A,B]\)
for \(\bC[\sigma,\tau]\) if \(A=\bC(\sigma)\) and \(B=\bC(\tau)\)

In typical cases of interest, the operations of \(\bC\) will be 'computable' maps of some kind between datatypes

\begin{definition}[]
A computability model \(\bC\) is \textbf{total} if every operation \(f\in\bC[A,B]\) is a total
function \(f:A\to B\)
\end{definition}

\begin{definition}[]
A computability model \(\bC\) has \textbf{weak (binary cartesian) products} if there is an operation
assigning to each \(A,B\in\abs{\bC}\) a datatype \(A\bowtie B\in\abs{\bC}\) along with
operations \(\pi_A\in\bC[A\bowtie B,A]\) and \(\pi_B\in\bC[A\bowtie B,B]\) (known as \textbf{projections}) s.t. for
any \(f\in\bC[C,A]\) and \(g\in\bC[C,B]\) there exists \(\la f,g\ra\in\bC[C,A\bowtie B]\) satisfying the following for
all \(c\in C\)
\begin{enumerate}
\item \(\la f,g\ra(c)\downarrow\) iff \(f(c)\downarrow\) and \(g(c)\downarrow\)
\item \(\pi_A(\la f,g\ra(c))=f(c)\) and \(\pi_B(\la f,g\ra(c))=g(c)\)
\end{enumerate}


We say that \(d\in A\bowtie B\) \textbf{represents} the pair \((a,b)\) if \(\pi_A(d)=a\) and \(\pi_B(d)=b\)
\end{definition}

In contrast to the usual definition of categorical products, the operation \(\la f,g\ra\) need not be
unique, since many elements of \(A\bowtie B\) may represent the same pair \((a,b)\). We do not formally
require that every \((a,b)\) is represented in \(A\bowtie B\), though in all cases of interest this will be
so. The reader is also warned that \(\pi_A\circ\la f,g\ra\) will not in general coincide with \(f\) .

\begin{definition}[]
A \textbf{weak terminal} in a computability model \(\bC\) consists of a datatype \(I\in\abs{\bC}\) and an
element \(i\in I\) s.t. for any \(A\in\abs{\bC}\) the constant function \(\Lambda a.i\) is in \(\bC[A,I]\)
\end{definition}

If \(\bC\) has weak products and a weak terminal \((I,i)\), then for any \(A\in\abs{\bC}\) there is an
operation \(t_A\in\bC[A,I\bowtie A]\) s.t. \(\pi_A\circ t_A=\id_A\)
\subsubsection{Examples of Computability Models}
\label{sec:org5ea0d36}
\begin{examplle}[]
\label{3.1.5}
Model with single datatype \(\N\) and whose operations \(\N\rightharpoonup\N\) are precisely the
Turing-computable partial functions. The model has standard products, since the well-known
computable pairing operation
\begin{equation*}
\la m,n\ra=(m+n)(m+n+1)/2+m
\end{equation*}
defines a bijection \(\N\times\N\to\N\). Any element \(i\in\N\) may serve as a weak terminal,
since \(\Lambda n.i\) is computable
\end{examplle}

\begin{examplle}[]
\label{3.1.6}
untyped \(\lambda\)-calculus

Terms \(M\) of the \(\lambda\)-calculus are generated from a set of variable symbols \(x\) by means of the following
grammar:
\begin{equation*}
M::=x\mid MM'\mid\lambda x.M
\end{equation*}
Writing \(\sfL\) for the quotient set \(\Lambda/=_\beta\)

We write \(M[x\mapsto N]\) for the result of substituting \(N\) for all free occurrences of \(x\)
within \(M\)

We define \(\Lambda\)  to be the set of untyped \(\lambda\)-terms modulo \(\alpha\)-equivalence.

Let \(\sim\) be any equivalence relation on \(\Lambda\) with the following properties:
\begin{equation*}
(\lambda x.M)N\sim M[x\mapsto N],\quad M\sim N\Rightarrow PM\sim PN
\end{equation*}
\begin{enumerate}
\item \((\lambda x.x)M\sim M\)
\item If \(M\sim N\), then \((\lambda x.N)M\sim(\lambda x.M)N\) and hence \(N\sim M\).
\item If \(M\sim N\) and \(N\sim O\), then
\end{enumerate}

Then we have \(M\sim N\Rightarrow MP\sim NP\) since \((\lambda y.yP)M\sim(\lambda y.yP)N\Rightarrow MP\sim NP\).

As a example, we may define \(=_\beta\) to be the smallest equivalence relation \(\sim\) satisfying the
above properties and also
\begin{equation*}
M\sim N\Rightarrow \lambda x.M\sim\lambda x.N
\end{equation*}

Writing \([M]\) for the \(\sim\)-equivalence class of \(M\), any term \(P\in A\) induces a
well-defined mapping \([M]\mapsto[PM]\) on \(\Lambda/\sim\). The mappings induced by some \(P\) in this way are
called \textbf{\(\lambda\)-definable}

We may regard \(\Lambda/\sim\) as a total computability model: the sole datatype is \(\Lambda/\sim\) itself, and
the operations on it are exactly the \(\lambda\)-definable mappings. It also has weak products: a
pair \((M,N)\) may be represented by the term \(pair\;M\;N\) where \(pair=\lambda xyz.zxy\)
the terms \(fst=\lambda p.p(\lambda xy.x)\) and \(snd=\lambda p.p(\lambda xy.y)\). We can check that
\(fst(pair\; M\;N)\sim M\) and \(snd(pair\;M\;N)\sim N\)

We can also obtain a submodel \(\Lambda^0/\sim\) consisting of the equivalence classes of closed terms \(M\)
\end{examplle}

\begin{examplle}[]
\label{3.1.7}
Let \(B\) be any family of \textbf{base sets}, and let \(\la B\ra\) denote the family of sets generated
from \(B\) by adding the singleton set \(1=\{()\}\) and closing under binary products \(X\times Y\) and
set-theoretic function spaces \(Y^X\). We shall consider some computability models whose family
of datatypes is \(\la B\ra\)

First we may define a computability model \(\sfS(B)\) with \(\abs{\sfS(B)}=\la B\ra\) (often called
the \textbf{full set-theoretic model over} \(B\)) by letting \(\sfS(B)[X,Y]\) consist of all
set-theoretic functions \(X\to Y\) for \(X,Y\in\la B\ra\); that is, we consider all functions to be
computable. However this model is of limited interest since it does not represent an interesting
concept of computability

To do better we may start by noting that whatever the 'computable' functions between these sets
are supposed to be, it is reasonable to expect that they will enjoy the following closure
properties
\begin{enumerate}
\item For any \(X\in\la B\ra\), the unique function \(X\to 1\) is computable
\item For any \(X,Y\in\la B\ra\), the projections \(X\times Y\to X\), \(X\times Y\to Y\) is computable
\item For any \(X,Y\in\la B\ra\), the application function \(Y^X\times X\to Y\) is computable
\item If \(f:Z\to X\) and \(g:Z\to Y\) is computable, so is their pairing \((f,g):Z\to X\times Y\)
\item If \(f:X\to Y\) and \(g:Y\to Z\) are computable, so is their composition \(g\circ f:X\to Z\)
\item If \(f:Z\times X\to Y\) is computable, so is its transpose \(\hatf:Z\to Y^X\)
\end{enumerate}


One possible approach is therefore to start by specifying some set \(C\) of functions between
out datatypes that we wish to regard as ``basic computable operations'', and define a
computability model \(\sfK(B;C)\) over \(\la B\ra\) whose operations are exactly the functions
generated from \(C\) under the above closure conditions

Take \(B=\{\N\}\); we shall often denote \(\sfS(\{\N\})\) by \(\sfS\). Let \(C\) consist of the
following basic operations: the zero function \(\Lambda x.0:\N\to 1\), the successor function \(suc:\N\to\N\);
and for each \(X\in\la B\ra\), the primitive recursion operator \(rec_X:(X\times X^{X\times\N}\times\N)\to X\) defined by
\begin{align*}
&rec_X(x,f,0)=0\\
&rec_X(x,f,n+1)=f(rec_X(x,f,n),n)
\end{align*}
the resulting model \(\sfK(B;C)\) consists of exactly those operations of \(\sfS\) definable in
Gödel's \textbf{System T}
\end{examplle}
\subsubsection{Weakly Cartesian Closed Models}
\label{sec:orge1fb0fc}
\begin{definition}[]
\label{3.1.8}
Suppose \(\bC\) has weak products and a weak terminal. We say \(\bC\) is \textbf{weakly cartesian closed} if
it is endowed with the following for each \(A,B\in\abs{\bC}\):
\begin{itemize}
\item a choice of datatype \(A\Rightarrow B\in\abs{\bC}\)
\item a partial function \(\cdot_{AB}:(A\Rightarrow B)\times A\rhu B\), external to the structure of \(\bC\)
\end{itemize}


s.t. for any partial function \(f:C\times A\rhu B\) the following are equivalent
\begin{enumerate}
\item \(f\) is represented by some \(\barf:\C[C\bowtie A,B]\), in the sense that if \(d\)
represents \((c,a)\) then \(\barf(d)\simeq f(c,a)\)
\item \(f\) is represented by some total operation \(\hatf:\bC[C,A\Rightarrow B]\), in the sense that
\begin{equation*}
\forall c\in C,a\in A\quad\hatf(c)\cdot_{AB}a\simeq f(c,a)
\end{equation*}
\end{enumerate}
\end{definition}

\(\cdot_{AB}\) is represented by an operation \(app_{AB}\in\C[(A\Rightarrow B)\bowtie A,B]\)

Crucially, and in contrast to the definition of cartesian closed category, there is no
requirement that \(f\) is unique. This highlights an important feature of our framework: in many
models of interest, elements of \(A\Rightarrow B\) will be \textbf{intensional} objects (programs or algorithms),
and there may be many intensional objects giving rise to the same partial function \(A\to B\)

\begin{examplle}[]
Consider again the model of Example \ref{3.1.5}, comprising the partial Turing-computable
functions \(\N\rhu\N\). Here \(\N\Rightarrow\N\) can only be \(\N\), so we must provide a suitable
operation \(\cdot:\N\times\N\to\N\). This is done using the concept of a \textbf{universal Turing machine}.
Let \(T_0, T_1,\dots\)  be some sensibly chosen enumeration of all Turing machines for computing
partial functions \(\N\rhu\N\). Then there is a Turing machine that accepts two inputs \(e,a\) and
returns the result of applying the machine \(T_e\) to the single input \(a\). We may therefore
take \(\cdot\) to be the partial function computed by \(U\)

Clearly the partial functions \(f:\N\times\N\rhu\N\)  representable within the model via the pairing
operation from Example \ref{3.1.5} are just the partial computable ones. We may also see that
these coincide exactly with those represented by some total computable \(\barf:\N\to\N\), in the
sense that \(f(c,a)\simeq\tilf(c)\cdot a\).

\(\Leftarrow\): Given a computable \(\tilf\) the operation \(\Lambda(c,a).\tilf(c)\cdot a\) is clearly computable

\(\Rightarrow\): \(s\)-\(m\)-\(n\) theorem

When endowed with this weakly cartisian closed structure, this computability model is known as
\textbf{Kleene's first model} of \(K_1\)
\end{examplle}

\begin{examplle}[]
\label{3.1.10}
Now consider the model \(\Lambda/\sim\) ; we shall write \(\sfL\) for the set \(\Lambda/\sim\) considered as the
sole datatype in this model. Set \(\sfL\Rightarrow\sfL=\sfL\bowtie\sfL=\sfL\). We may obtain a weakly cartesian
closed structure by letting \(\cdot\) be given by application. If \(M\in\Lambda\) induces an operation
in \([\sfL\bowtie \sfL,\sfL]\) representing some \(f:\sfL\times\sfL\to\sfL\), then \(\lambda x.\lambda y.M(pair\;x\;y)\)
induces the corresponding operation in \([\sfL,\sfL\Rightarrow\sfL]\); conversely if \(N\) induces an
operation in \([\sfL,\sfL\Rightarrow\sfL]\) then \(\lambda z.N(fst\;z)(snd\;z)\) induces the corresponding one
in \([\sfL\bowtie\sfL,\sfL]\)
\end{examplle}

\begin{examplle}[]
For models of the form \(\sfK(B;C)\), we naturally define \(X\Rightarrow Y=Y^X\) and take \(\cdot_{XY}\) to be
ordinary function application. These models are endowed with binary products, and it is
immediate from closure condition 6 in Example \ref{3.1.7} that they are weakly cartesian closed

Such models show that not every element of \(X\Rightarrow Y\) need represent an operation in \(\bC[X,Y]\),
or equivalently one in \(\bC[1,X\Rightarrow Y]\). This accords with the idea that our models consist of
'computable' operations acting on potentially 'non-computable' data: operations in \(\bC[X,Y]\)
are computable, whereas elements of \(X\) need not be
\end{examplle}
\subsubsection{Higher-Order Models}
\label{sec:orgfb16c45}
\begin{definition}[]
A \textbf{higher-order structure} is a computability model \(\bC\) possessing a weak terminal \((I,i)\) and
endowed with the following for each \(A,B\in\abs{\bC}\)
\begin{itemize}
\item a choice of datatype \(A\Rightarrow B\in\abs{\bC}\)
\item a partial function \(\cdot_{AB}:(A\Rightarrow B)\times A\rhu B\)
\end{itemize}
\end{definition}

We treat \(\Rightarrow\) as right-associative and \(\cdot\) as left-associative

The significance of the weak terminal \((I,i)\) here is that it allows us to pick out a
subset \(A^\sharp\) of each \(A\in\abs{\bC}\), namely the set of elements of the form \(f(i)\)
where \(f\in\bC[I,A]\) and \(f(i)\downarrow\).

This is independent of the choice of \((I,i)\): if \(a=f(i)\) and \((J,j)\) is another weak
terminal, then composing \(f\) with \(\Lambda x.i\in\bC[J,I]\) gives \(f'\in\bC[J,A]\) with \(f'(j)=a\).

Intuitively, we think of \(A^\sharp\) as playing the role of the 'computable' elements of \(A\), and \(i\) as
some generic computable element.
On the one hand, if \(a\in A\) were computable, we would expect each
\(\Lambda x.a\) to be computable so that \(a\in A^\sharp\); on the other hand, the image of a computable element
under a computable operation should be computable, so that every element of \(A^\sharp\) is
computable.

Any weakly cartesian closed model \(\bC\) is a higher-structure.

\begin{definition}[]
\label{3.1.13}
A \textbf{higher-order (computability) model} is a higher-order structure \(\bC\) satisfying the following
conditions for some (or equivalently any) weak terminal \((I,i)\)
\begin{enumerate}
\item A partial function \(f:A\rhu B\) is present in \(\bC[A,B]\) iff there
exists \(\hatf\in\bC[I,A\Rightarrow B]\) s.t.
\begin{equation*}
\hatf(i)\downarrow,\quad\forall a\in A.\hatf(i)\cdot a\simeq f(a)
\end{equation*}
\item For any \(A,B\in\abs{\bC}\), there exists \(k_{AB}\in(A\Rightarrow B\Rightarrow A)^\sharp\) s.t.
\begin{equation*}
\forall a.k_{AB}\cdot a\downarrow,\quad \forall a,b.k_{AB}\cdot a\cdot b=a
\end{equation*}
\item For any \(A,B,C\in\abs{\bC}\) there exits
\begin{equation*}
s_{ABC}\in((A\Rightarrow B\Rightarrow C)\Rightarrow(A\Rightarrow B)\Rightarrow(A\Rightarrow C))^\sharp
\end{equation*}
s.t.
\begin{equation*}
\forall f,g.s_{ABC}\cdot f\cdot g\downarrow,\quad\forall f,g,a.s_{ABC}\cdot f\cdot g\cdot a\simeq(f\cdot a)\cdot(g\cdot a)
\end{equation*}
\end{enumerate}
\end{definition}

The elements \(k\) and \(s\) correspond to combinators from combinatory logic.

\(k\) allows us to construct \textbf{constant} maps in a computable way

A possible intuition for \(s\) is that it somehow does duty for an application
operation \((B\Rightarrow C)\times B\rhu C\)
within \(\bC\) itself, where the application may be performed uniformly in a parameter of type A.p

\begin{proposition}[]
\label{3.1.14}
Suppose \(\bC\) is a higher-order model
\begin{enumerate}
\item for any \(j<m\), there exists \(\pi_j^m\in(A_0\Rightarrow\cdots\Rightarrow A_{m-1}\Rightarrow A_j)^\sharp\) s.t.
\begin{equation*}
\forall a_0,\dots,a_{m-1}.\pi_j^m\cdot a_0\cdot\dots\cdot a_{m-1}=a_j
\end{equation*}
\item Suppose \(m,n>0\). Given
\begin{gather*}
f_j\in(A_0\Rightarrow\dots\Rightarrow A_{m-1}\Rightarrow B_j)^\sharp,\quad(j=0,\dots,n-1),\\
g\in(B_0\Rightarrow\dots\Rightarrow B_{n-1}\Rightarrow C)^\sharp
\end{gather*}
there exists \(h\in (A_0\Rightarrow\dots\Rightarrow A_{m-1}\Rightarrow C)^\sharp\) s.t.
\begin{equation*}
\forall a_0,\dots,a_{m-1}.h\cdot a_0\cdot\dots\cdot a_{m-1}\simeq g\cdot(f_0\cdot a_0\cdot\dots\cdot a_{m-1})\cdot\dots\cdot(f_{n-1}\cdot a_0\cdot\dots\cdot a_{m-1})
\end{equation*}
\item Suppose \(m>0\). For any element \(f\in (A_0\Rightarrow\cdots\Rightarrow A_{m-1}\Rightarrow B)^\sharp\), there
exists \(f^\dagger\in(A_0\Rightarrow\dots\Rightarrow A_{m-1}\Rightarrow B)^\sharp\) s.t.
\begin{gather*}
\forall a_0,\dots,a_{m-1}.f^\dagger\cdot a_0\cdot\dots\cdot a_{m-1}\simeq f\cdot a_0\cdot\dots\cdot a_{m-1}\\
\forall k<m.\forall a_0,\dots,a_{k-1}.f^\dagger\cdot a_0\cdot\dots\cdot a_{k-1}\downarrow
\end{gather*}
\end{enumerate}
\end{proposition}

\(i_A=s_{A(A\Rightarrow A)A}\cdot k_{A\Rightarrow A}\cdot k_{AA}\in(A\Rightarrow A)^\sharp\)

\begin{proof}
\begin{enumerate}
\item consider
\begin{align*}
&T[x]\Rightarrow x\\
&T[(E_1\;E_2)]\Rightarrow(T[E_1]\;T[E_2])\text{if $x$ does not occur free in $E$}\\
&T[\lambda x.E]\Rightarrow(\bK\;T[E])\\
&T[\lambda x.x]\Rightarrow\bI\\
&T[\lambda x.\lambda y.E]\Rightarrow T[\lambda x.T[\lambda y.E]]\text{if $x$ occurs free in $E$}\\
&T[\lambda x.(E_1\;E_2)]\Rightarrow(\bS\;T[\lambda x.E_1]\;T[\lambda x.E_2])\text{if $x$ occurs free in $E_1$ or $E_2$}
\end{align*}
so \(A\Rightarrow B\Rightarrow B\to\lambda x^Ay^B.y^B\to \bK_{B\Rightarrow B,A}\cdot I_B\)
\end{enumerate}
\end{proof}

If \(\bC,\bD\) are higher-order structures, we say \(\bC\) is a \textbf{full substructure} of \(\bD\) if
\begin{itemize}
\item \(\abs{\bC}\subseteq\abs{\bD}\)
\item \(\bC[A,B]=\bD[A,B]\) for all \(A,B\in\abs{\bC}\)
\item some (or equivalently any) weak terminal in \(\bC\) is also a weak terminal in \(\bD\)
\item the meaning of \(A\Rightarrow B\) and \(\cdot_{AB}\) in \(\bC\) and \(\bD\) coincide
\end{itemize}


Note that if \((I,i)\) and \((J,j)\) are weak terminals in \(\bC\) then \(\Lambda x.j\in\bC[I,J]\), so
if \((I,i)\) is a weak terminal in \(\bD\) then so is \((J,j)\)

\begin{theorem}[]
A higher-order structure is a higher-order model iff it is a full substructure of a weakly
cartesian closed model
\end{theorem}

\begin{proof}
Let \(\bC\) be a higher-order structure.

\(\Leftarrow\): suppose \(\bD\) is weakly cartesian closed and \(\bC\) is a
full substructure of \(\bD\) with a weak terminal \((I,i)\)
\begin{enumerate}
\item For any \(f\in\bC[A,B]=\bD[A,B]\) we have that \(f\circ\pi_A\in\bD[I\bowtie A,B]\) represents \(\Lambda(x,a).f(a)\),
which by definition \ref{3.1.8} is in turn represented by some total \(\hatf\in\bD[I,A\Rightarrow B]\).

Conversely, given \(f:A\rhu B\) and \(\hatf\in\bC[I,A\Rightarrow B]\) with \(\hatf(i)\downarrow\)
and \(\hatf(i)\cdot a\simeq f(a)\) for all \(a\), take \(\hatg=\hatf\circ(\Lambda x.i)\in\bC[I,A\Rightarrow B]=\bD[I,A\Rightarrow B]\) so
that \(\hatg\) is total and represents \(g=\Lambda(x,a).f(a):I\times A\rhu B\). Now
let \(\barg\in\bD[I\bowtie A,B]\) also represents \(g\). Then \(\barg\circ\la\Lambda a.i,\id_A\ra\in\bD[A,B]=\bC[A,B]\) and
it is routine to check that \(\barg\circ\la\Lambda a.i,\id_A\ra=f\)

\item Suppose \(A,B\in\abs{\bC}\). Let \(k'\in\bD[A,B\Rightarrow A]\) correspond to \(\pi_A\in\bD[A\bowtie B,A]\) as in
definition \ref{3.1.8}, then \(k'(a)\cdot b\simeq\pi_A(d)\). Let \(\hatk'\in\bD[I,A\Rightarrow(B\Rightarrow A)]\) correspond
to \(k'\circ\pi_A'\in\bD[I\bowtie A,B\Rightarrow A]\) where \(\pi_A'\in\bD[I\bowtie A,A]\) and take \(k=\hatk'(i)\)
\(k\cdot a\cdot b=\hatk'(i)\cdot a\cdot b=(k'\circ\pi_A'(i,a))\cdot b=k'(a)\cdot b=a\)

\item 
\end{enumerate}

\(\Rightarrow\): Suppose \(\bC\) is a higher-order model, with \((I,i)\) a weak terminal. We build a weakly
cartesian closed model \(\bC^\times\) into which \(\bC\) embeds fully as follows:
\begin{itemize}
\item Datatypes of \(\bC^\times\) are sets \(A_0\times\dots\times A_{m-1}\), where \(m>0\) and \(A_0,\dots,A_{m-1}\in\abs{\bC}\)
\item If \(D=A_0\times\dots\times A_{m-1}\) and \(E=B_0\times\dots\times B_{n-1}\) where \(m,n>0\) the operations
in \(\bC^\times[D,E]\) are those partial functions \(f:D\rhu E\) of the form
\begin{equation*}
f=\Lambda(a_0,\dots,a_{m-1}).(f_0\cdot a_0\cdot\dots\cdot a_{m-1},\dots,f_{n-1}\cdot a_0\cdot\dots\cdot a_{m-1})
\end{equation*}
where \(f_j\in(A_0\Rightarrow\dots\Rightarrow A_{m-1}\Rightarrow B_j)^\sharp\) for each \(j\); we say that \(f_0,\dots,f_{n-1}\) \textbf{witness}
the operation \(f\). Note that for \((f_0\cdot a_0\cdot\dots\cdot a_{m-1},\dots,f_{n-1}\cdot a_0\cdot\dots\cdot a_{m-1})\) to be
defined, it is necessary that all its components be defined
\end{itemize}


It remains to check the relevant properties of \(\bC^\times\). That \(\bC^\times\) is a computability model is
straightforward: the existence of identities follows from part 1 of Proposition \ref{3.1.14}
and composition from part 2. \(\bC^\times\) has standard products and that \((I,i)\) is a weak terminal
in \(\bC^\times\).

Now let's show that \(\bC^\times\) is weakly cartesian closed. Given \(D=A_0\times\dots\times A_{m-1}\)
and \(E=B_0\times\dots\times B_{n-1}\) with \(m,n>0\), take \(C_j=A_0\Rightarrow\dots\Rightarrow A_{m-1}\Rightarrow B_j\) for each \(j\), and
let \(D\Rightarrow E\) be the set of tuples \((f_0,\dots,f_{n-1})\in C_0\times\dots\times C_{n-1}\) witnessing operations
in \(\bC^\times[D,E]\). The application \(\cdot_{DE}\) is then given by
\begin{equation*}
(f_0,\dots,f_{n-1})\cdot_{DE}(a_0,\dots,a_{m-1})\simeq(f_0\cdot a_0\cdot\dots\cdot a_{m-1},\dots,f_{n-1}\cdot a_0\cdot\dots\cdot a_{m-1})
\end{equation*}

Next, given an operation \(g\in\bC^\times[G\times D,E]\) witnessed by operations \(g_0,\dots,g_{n-1}\) in \(\bC\),
take \(g_0^\dagger,\dots,g_{n-1}^\dagger\) as in Proposition \ref{3.1.14} (3); then \(g_0^\dagger,\dots,g_{n-1}^\dagger\) witness
the corresponding total operation \(\hatg\in\bC^\times[G,D\Rightarrow E]\). Conversely, the witnesses for any such
total \(\hatg\) also witness the corresponding \(g\)
\end{proof}
\subsubsection{Typed Partial Combinatory Algebras}
\label{sec:orgfcc1831}
The following definition captures roughly what is left of a higher-order model once the
operations are discarded

\begin{definition}[]
\label{3.1.16}
\begin{enumerate}
\item A \textbf{partial applicative structure} \(\bA\) consists of
\begin{itemize}
\item an inhabited family \(\abs{\bA}\) of datatypes \(A,B,\dots\) (indexed by some set \(T\))
\item a (right-associative) binary operation \(\Rightarrow\) on \(\abs{\bA}\)
\item for each \(A,B\in\abs{\bA}\), a partial function \(\cdot_{AB}:(A\Rightarrow B)\times A\rhu B\)
\end{itemize}
\item A \textbf{typed partial combinatory algebra} (TPCA) is a partial applicative structure \(\bA\)
satisfying the following conditions
\begin{enumerate}
\item For any \(A,B\in\abs{\bA}\), there exists \(k_{AB}\in A\Rightarrow B\Rightarrow A\) s.t.
\begin{equation*}
\forall a.k\cdot a\downarrow,\quad\forall a,b.k\cdot a\cdot b=a
\end{equation*}
\item For any \(A,B,C\in\abs{\bA}\), there exists \(s_{ABC}\in(A\Rightarrow B\Rightarrow C)\Rightarrow(A\Rightarrow B)\Rightarrow(A\Rightarrow C)\) s.t.
\begin{equation*}
\forall f,g. s\cdot f\cdot g\downarrow,\quad\forall f,g,a.s\cdot f\cdot g\cdot a\simeq(f\cdot a)\cdot(g\cdot a)
\end{equation*}
\end{enumerate}
\end{enumerate}


A TPCA is \textbf{total} if all the application operations \(\cdot_{AB}\) are total
\end{definition}

Any higher-order model yields an underlying TPCA. However, in passing to this TPCA we lose the
information that says which element of \(A\Rightarrow B\) are supposed to represent operations.

\begin{definition}[]


\begin{enumerate}
\item If \(\bA^\circ\) denotes a partial applicative structure, a \textbf{partial applicative
substructure} \(\bA^\sharp\) of \(\bA^\circ\) consists of a subset \(A^\sharp\subseteq A\) for each \(A\in\abs{\bA^\circ}\) s.t.
\begin{itemize}
\item if \(f\in(A\Rightarrow B)^\sharp\), \(a\in A^\sharp\) and \(f\cdot a\downarrow\) in \(\bA^\circ\), then \(f\cdot a\in B^\sharp\)
\end{itemize}

such a pair \((\bA^\circ;\bA^\sharp)\) is called a \textbf{relative partial applicative structure}

\item A \textbf{relative TPCA} is a relative partial applicative structure \((\bA^\circ,\bA^\sharp)\) s.t. there exist
elements \(k_{AB}, s_{ABC}\) in \(\bA^\sharp\) witnessing that \(\bA^\circ\) is a TPCA
\end{enumerate}
\end{definition}

A relative TPCA \((\bA^\circ,\bA^\sharp)\) is \textbf{full} if \(\bA^\sharp=\bA^\circ\). We will use \(\bA\) to range over both
ordinary TPCAs and relative ones (writing \(\bA^\circ\), \(\bA^\sharp\) for the two components of \(\bA\) in
the latter case), so that in effect we identify an ordinary TPCA \(\bA\) with the relative
TPCA \((\bA;\bA)\). Indeed, we may sometimes refer to ordinary TPCAs as 'full TPCAs' . Clearly the
models \(K_1\) and \(\Lambda/\sim\) are full, while in general \(\sfK(B;C)\) is not: rather, it is a
relative TPCA \(\bA\) in which \(\bA^\circ\) is a full set-theoretic type structure whilst \(\bA^\sharp\)
consists of only the \(C\)-computable elements

\begin{theorem}[]
\label{3.1.18}
There is a canonical bijection between higher-order models and relative TPCAs
\end{theorem}

\begin{proof}
First suppose \(\bC\) is a higher-order model, and let \(\bA^\circ\) be its underlying partial
applicative structure. Take \((I,i)\) a weak terminal in \(\bC\), and for any \(A\in\abs{\bC}\),
define \(A^\sharp=\{g(i)\mid g\in\bC[I,A],g(i)\downarrow\}\). As noted there, this is independent of the choice
of \((I,i)\); in fact, it is easy to see that \(a\in A^\sharp\) iff \((A,a)\) is a weak terminal. To
see that the \(A^\sharp\) form an applicative substructure, suppose \(f\in(A\Rightarrow B)^\sharp\) is witnessed
by \(f'\in\bC[I,A\Rightarrow B]\) and \(a\in A^\sharp\) is witnessed by \(a'\in\bC[I,A]\), and suppose further
that \(f\cdot a=b\). Take \(\check{f'}\in\bC[A\Rightarrow B]\) corresponding to \(f'\); then \(\check{f'}(a)=b\)
and so \(\check{f'}\circ a'\) witnesses \(b\in B^\sharp\)

Let \(\bA^\sharp\) denote the substructure formed by the sets \(A^\sharp\). It is directly build into
Definition \ref{def3.1.13} that there are elements \(k_{AB}, s_{ABC}\) in \(\bA^\sharp\) with the
properties required by Definition \ref{3.1.16}; thus \((\bA^\circ;\bA^\sharp)\) is a relative TPCA

For the converse, suppose \(\bA\) is a relative TPCA. Take \(\abs{\bC}=\abs{\bA^\circ}\) and
for \(A,B\in\abs{\bC}\), let \(\bC[A,B]\) consist of all partial functions \(\Lambda a.f\cdot a\)
for \(f\in(A\Rightarrow B)^\sharp\). To see that \(\bC\) has identities, for any \(A\in\abs{\bC}\), we have
\begin{equation*}
i_A=s_{A(A\Rightarrow A)A}\cdot k_{A(A\Rightarrow A)}\cdot k_{AA}\in(A\Rightarrow A)^\sharp
\end{equation*}
and clearly \(i_A\) induces \(\id_A\in\bC[A,A]\).  For composition, given
operations \(f\in\bC[A,B]\), \(g\in\bC[B,C]\) induced by \(f'\in(A\Rightarrow B)^\sharp\), \(g'\in(B\Rightarrow C)^\sharp\), we have
that \(g\circ f\in\bC[A,C]\) is induced by \(s_{ABC}\cdot(k_{(B\Rightarrow C)}\cdot g)\cdot f\). Thus \(\bC\) is a computability
mode

For a weak terminal, take any \(U\in\abs{\bC}\) and let \(I=U\Rightarrow U\) and \(i=i_U\) as defined above.
Then for any \(A\) we have that \(k_{IA}\cdot i\in(A\Rightarrow U\Rightarrow U)^\sharp\) induces \(\Lambda a.i\in\bC[A,I]\)

To turn \(\bC\) into a higher-order structure, we take \(\Rightarrow\) and \(\cdot\) as in \(\bA^\circ\). We may now
verify that for any \(A\) we have
\begin{equation*}
A^\sharp=\{g(i)\mid g\in\bC[I,A],g(i)\downarrow\}
\end{equation*}
so that the present meaning of \(A^\sharp\) coincides with its meaning in Section \ref{3.1.4}. For
given \(a\in A^\sharp\) we have \(k_{AI}\cdot a\in(I\Rightarrow A)^\sharp\) inducing an operation \(g\) with \(g(i)=a\).
Conversely, given \(g\in\bC[I,A]\) with \(g(i)\downarrow\) we have that \(g(i)=g'\cdot i\) for
some \(g'\in(I\Rightarrow A)^\sharp\) (by definition, \(g\) is of the form \(\Lambda a.f\cdot a\)); but \(i\in I^\sharp\) so \(g(i)\in A^\sharp\)

By applying the above equation to the type \(A\Rightarrow B\), we see that conditions 1 and 2 of
Definition \ref{3.1.13} are satisfied, and conditions 3 and 4 are immediate from the \(k,s\)
conditions in Definition \ref{3.1.16}. Thus \(\bC\) is a higher-order model
\end{proof}

In the setting of a relative TPCA \(\bA\), we have a natural \textbf{degree structure} on the elements
of \(\bA^\circ\). Specifically, if \(a\in A\) and \(b\in B\) where \(A,B\in\abs{\bA^\circ}\), let us
write \(a\gg b\) if there exists \(f\in\bA^\sharp(A\Rightarrow B)\) with \(f\cdot a=b\)

If \(\abs{\bA}\) consists of just a single datatype, then TPCA is just a single set \(A\) equipped
with a partial 'application' operation \(\cdot:A\times A\rhu A\) s.t. for some \(k,s\in A\) we have
\begin{equation*}
\forall x,y.k\cdot x\cdot y=x,\quad\forall x,y.s\cdot x\cdot y\downarrow,\quad\forall x,y,z.s\cdot x\cdot y\cdot z\simeq(x\cdot z)\cdot(y\cdot z)
\end{equation*}
We call such a structure an \textbf{partial combinatory algebra} (or PCA)
\subsubsection{Lax Models}
\label{sec:orgf2e2871}
For simplicity, we have worked so far with a simple definition of computability model in which
operations are required to be closed under ordinary composition of partial functions. It turns
out, however, that with a few refinements, practically all the general theory presented in this
chapter goes through under a somewhat milder assumption.

\begin{definition}[]
A \textbf{lax computability model} \(\bC\)  over a set \(\sfT\) of \textbf{type names} consists of
\begin{itemize}
\item an indexed family \(\abs{\bC}=\{\bC(\tau)\mid\tau\in\sfT\}\) of sets, called the \textbf{datatypes} of \(\bC\)
\item for each \(\sigma,\tau\in\sfT\), a set \(\bC[\sigma,\tau]\) of partial functions \(f:\bC(\sigma)\to\bC(\tau)\), called the
\textbf{operations} of \(\bC\)
\end{itemize}


s.t.
\begin{enumerate}
\item for each \(\tau\in\sfT\), the identity function \(\id:\bC(\tau)\to\bC(\tau)\) is in \(\bC(\tau,\tau)\)
\item for any \(f\in\bC[\rho,\sigma]\) and \(g\in\bC[\sigma,\tau]\), there exists \(h\in\bC[\rho,\tau]\) with \(h(a)\succeq g(f(a))\) for
all \(a\in\bC(\rho)\)
\end{enumerate}


We may refer to \(h\) here as a \textbf{supercomposition} of \(f\) and \(g\).
\end{definition}

We sometimes refer to our standard computability models as \textbf{strict} when we wish to emphasize the
contrast with lax models. Of course, for total computability models, the distinction
evaporates completely.

One possible motivation for the concept of lax model is that it is often natural to think of an
application \(f(a)\) in terms of some computational agent \(F\) representing \(f\) being placed ‘alongside’
a representation \(A\) of a to yield a composite system \(F\mid A\), which may then evolve in certain ways
via interactions between \(F\) and \(A\). If an agent \(G\) representing \(g\) is then placed alongside this to
yield a system \(G\mid F\mid A\), there is the possibility that \(G\) may interact ‘directly’ with \(F\) rather
than just with the result obtained from \(F\mid A\); thus, \(G\mid F\) might admit other behaviours not
accounted for by \(g\circ f\) . (For a precise example of this in process algebra, see Longley
[183].)

The notion of a \textbf{(relative) lax TPCA} is given by replacing the axioms for \(s_{ABC}\) in
Definition \ref{3.1.16} with
\begin{equation*}
\forall f,g.s\cdot f\cdot g\downarrow,\quad\forall f,g,a.s\cdot f\cdot g\cdot a\succeq (f\cdot a)\cdot(g\cdot a)
\end{equation*}

The definitions of weak products and weak terminal may be carried over unchanged to the
setting of lax computability models; note that \(\la f,g\ra\) is still required to be a pairing in the
‘strict’ sense that its domain coincides precisely with \(\dom f\cap\dom g\). The definition of weakly
cartesian closed model is likewise unchanged, although one should note that in the lax
setting, whether a given model is weakly cartesian closed may be sensitive to the choice of the
type operator \(\bowtie\).

For the definition of a lax higher-order model, we simply replace '\(\simeq\)' by '\(\succeq\)' in
condition 4(?) of Definition \ref{3.1.13}

\begin{theorem}[]
\begin{enumerate}
\item Any lax higher-order model is a full substructure of a lax weakly cartesian closed model
\item If \(\bD\) is a lax weakly cartesian closed model in which some weak terminal \((I,i)\) is a
weak unit, any full substructure of \(\bD\) containing \(I\) is a lax higher-order model
\end{enumerate}
\end{theorem}


\subsubsection{Type worlds}
\label{sec:orgbae8f17}
\begin{definition}[]
\begin{enumerate}
\item A \textbf{type world} is simply a set \(\sfT\) of \textbf{type names} \(\sigma\), optionally endowed with any or
all of the following:
\begin{enumerate}
\item a \textbf{fixing map}, assigning a set \(\sfT[\sigma]\) to certain type names \(\sigma\in\sfT\)
\item a \textbf{product structure}, consisting of a total binary operation \((\sigma,\tau)\mapsto\sigma\times\tau\)
\item an \textbf{arrow structure}, consisting of a total binary operation \((\sigma,\tau)\mapsto\sigma\to\tau\)
\end{enumerate}
\item A \textbf{computability model over} a type world \(\sfT\) is a computability model \(\bC\) with index
set \(\sfT\) (so that \(\abs{\bC}=\{\bC(\sigma)\mid\sigma\in\sfT\}\)) subject to the following conventions
\begin{enumerate}
\item If \(\sfT\) has a fixing map, then \(\bC(\sigma)=T[\sigma]\) whenever \(\sfT(\sigma)\) is defined
\item If \(\sfT\) has a product structure, then \(\bC\) has weak products and for any \(\sigma,\tau\in\sfT\)
we have \(\bC(\sigma\times\tau)=\bC(\sigma)\bowtie\bC(\tau)\)
\item If \(\sfT\) has an arrow structure, then \(\bC\) is a higher-order model and for
any \(\sigma,\tau\in\sfT\) we have \(\bC(\sigma\to\tau)=\bC(\sigma)\Rightarrow\bC(\tau)\)
\item If \(\sfT\) has both a product and an arrow structure, then \(\bC\) is weakly cartesian closed
\end{enumerate}
\end{enumerate}
\end{definition}

\begin{examplle}[]
The one-element type world \(\sfO=\{*\}\) with just the arrow structure \(*\to*=*\). TPCAs over this
type world are precisely (untyped) PCAs; both \(K_1\) and \(\Lambda/\sim\) are examples
\end{examplle}

\begin{examplle}[]
If \(\beta_0,\dots,\beta_{n-1}\) are distinct \textbf{basic type names} and \(B_0,\dots,B_{n-1}\) are sets, we may define
the type word \(\sfT^{\to}(\beta_0=B_0,\dots,\beta_{n-1}=B_{n-1})\) to consist of formal type expressions
freely constructed from \(\beta_0,\dots,\beta_{n-1}\) via \(\to\), fixing the interpretation of each \(\beta_i\)
at \(B_i\). This type world has a fixing map and an arrow structure, but no product. We may write
just \(\sfT^{\to}(\beta_0,\dots,\beta_{n-1})\) if we do not wish to constrain the interpretation of the \(\beta_i\)

A typical example is the type world \(\sfT^\to(\ttN=\N)\). Models over this type would correspond
to \textbf{finite type structures} over \(\N\); the models \(\sfK(B;C)\) are examples

Type world \(\sfT^{\to}(\ttN=\N_\bot)\) where \(\N_\bot\) is the set of natural numbers together with an
additional element \(\bot\) representing 'non-termination'. Whereas \(\N\) may be used to model
actual \textbf{results} of computation, we may think of \(\N_\bot\) as representing some computational
\textbf{process} which may or may not return a natural number.
\end{examplle}

\begin{examplle}[]
Similarly, we define \(\sfT^{\to\times}=(\beta_0=B_0,\dots,\beta_{n-1}=B_{n-1})\) to be the type world consisting
of type expressions freely constructed form \(\beta_0,\dots,\beta_{n-1}\) via \(\to\) and \(\times\), fixing the
interpretation of each \(\beta_i\) at \(B_i\). If no fixing map is required, we write
just \(\sfT^{\to\times}(\beta_0,\dots,\beta_{n-1})\)

Type worlds featuring a \textbf{unit type} (denoted by \(\texttt{1}\)) are also useful. We shall
write \(\sfT^{\to\times\texttt{1}}(\beta_0=B_0,\dots,\beta_{n-1}=B_{n-1})\) for the type world
\begin{equation*}
\sfT^{\to\times}(\texttt{1}=\{()\},\beta_0=B_0,\dots,\beta_{n-1}=B_{n-1})
\end{equation*}
We will often refer to the type names in a type world simply as \textbf{types}, and use \(\rho\), \(\sigma\), \(\tau\) to range
over them. When dealing with formal type expressions, we adopt the usual convention that \(\to\)
is right-associative, so that \(\rho\to\sigma\to\tau\) means \(\rho\to(\sigma\to\tau)\). For definiteness, we may also declare
that \(\times\) is right-associative, although in practice we shall not always bother to distinguish
between \((\rho\times\sigma)\times\tau\) and \(\rho\times(\sigma\times\tau)\). We consider \(\times\) as binding more tightly than \(\to\)

We shall use the notation \(\sigma_0,\dots,\sigma_{r-1}\to\tau\) as an abbreviation for \(\sigma_0\to\sigma_1\to\dots\to\sigma_{r-1}\to\tau\)
(allowing this to mean \(\tau\) in the sense \(r=0\)). This allows us to express our intention
regarding which objects are to be thought of as 'arguments' to a given operation: for instance,
the types \(\ttN,\ttN,\ttN\to\ttN\) and \(\ttN,\ttN\to(\ttN\to\ttN)\) are formally the same, but in
the first case we are thinking of a three-argument operation returning a natural number, while
in the second we are thinking of a two-argument operation returning a function \(\N\to\N\). We also
write \(\sigma^{(r)}\to\tau\) for the type \(\sigma,\dots,\sigma\to\tau\) with \(r\) arguments. The notation \(\sigma^r\) is
reserved for the \(r\)-fold product type \(\sigma\times\dots\times\sigma\)
\end{examplle}

\begin{proposition}[]
Any type \(\sigma\in\sfT^{\to}(\beta_0,\dots,\beta_{n-1})\) may be uniquely written in the form \(\sigma_0,\dots,\sigma_{r-1}\to\beta_i\)
\end{proposition}

We shall call this the \textbf{argument form} of \(\sigma\). The importance of this is that it provides a useful
induction principle for types: if a property holds for \(\sigma_0,\dots,\sigma_{r-1}\to\beta_i\) whenever it holds
for each of \(\sigma_0,\dots,\sigma_{r-1}\), then it holds for all \(\sigma\in\sfT^{\to}(\beta_0,\dots,\beta_{n-1})\). We shall
refer to this as \textbf{argument induction}; it is often preferable as an alternative to the usual
\textbf{structural induction} on types

Closely associated with argument form is the notion of the \textbf{level} of a type \(\sigma\): informally, the
stage at which \(\sigma\) appears in the generation of \(\sfT^\to(\beta_0,\dots,\beta_{n-1})\) via argument induction:
\begin{align*}
\lv(\beta_i)&=0\\
\lv(\sigma_0,\dots,\sigma_{r-1}\to\beta_i)&=1+\max_{i<r}\lv(\sigma_i)\quad(r\ge 1)
\end{align*}
When working with \(\sfT^{\to\times}(\beta_0,\dots,\beta_{n-1})\), it is natural to augment this definition with
\begin{equation*}
\lv(\sigma\times\tau)=\max(\lv(\sigma),\lv(\tau))
\end{equation*}
We may define the \textbf{pure type of level \(k\) over \(\sigma\)}, written \(\bark[\sigma]\):
\begin{equation*}
\bbar{0}[\sigma]=\sigma,\quad\ove{k+1}[\sigma]=\bark[\sigma]\to\sigma
\end{equation*}
For type worlds generated by a single base type \(\beta\), we may write simply \(\bark\)
for \(\bark[\beta]\). For instance, in the type word \(\sfT^{\to}(\N)\) we write \(\bbar{2}\) for the
type \((\ttN\to\ttN)\to\ttN\).
\subsection{Computational Structure in Higher-Order Models}
\label{sec:orgad3ec4f}
\subsubsection{Combinatory Completeness}
\label{sec:org945a430}
Combinatory completeness can be seen as a syntactic counterpart to the notion of weakly
cartesian closed model. In essence, combinatory completeness asserts that any operation
definable by means of a formal expression over A (constructed using application) is
representable by an element of A itself.

\begin{definition}[]
\label{3.1.16}
\begin{enumerate}
\item A \textbf{partial applicative structure} \(\bA\) consists of
\begin{itemize}
\item an inhabited family \(\abs{\bA}\) of datatypes \(A,B,\dots\) (indexed by some set \(T\))
\item a (right-associative) binary operation \(\Rightarrow\) on \(\abs{\bA}\)
\item for each \(A,B\in\abs{\bA}\), a partial function \(\cdot_{AB}:(A\Rightarrow B)\times A\rhu B\)
\end{itemize}
\item A \textbf{typed partial combinatory algebra} (TPCA) is a partial applicative structure \(\bA\)
satisfying the following conditions
\begin{enumerate}
\item For any \(A,B\in\abs{\bA}\), there exists \(k_{AB}\in A\Rightarrow B\Rightarrow A\) s.t.
\begin{equation*}
\forall a.k\cdot a\downarrow,\quad\forall a,b.k\cdot a\cdot b=a
\end{equation*}
\item For any \(A,B,C\in\abs{\bA}\), there exists \(s_{ABC}\in(A\Rightarrow B\Rightarrow C)\Rightarrow(A\Rightarrow B)\Rightarrow(A\Rightarrow C)\) s.t.
\begin{equation*}
\forall f,g. s\cdot f\cdot g\downarrow,\quad\forall f,g,a.s\cdot f\cdot g\cdot a\simeq(f\cdot a)\cdot(g\cdot a)
\end{equation*}
A \textbf{lax TPCA} is obtained from a TPCA change '\(\simeq\)' to '\(\succeq\)' in the axiom \(s\)
\end{enumerate}
\item If \(\bA^\circ\) denotes a partial applicative structure, a \textbf{partial applicative
substructure} \(\bA^\sharp\) of \(\bA^\circ\) consists of a subset \(A^\sharp\subseteq A\) for each \(A\in\abs{\bA^\circ}\) s.t.
\begin{itemize}
\item if \(f\in(A\Rightarrow B)^\sharp\), \(a\in A^\sharp\) and \(f\cdot a\downarrow\) in \(\bA^\circ\), then \(f\cdot a\in B^\sharp\)
\end{itemize}

such a pair \((\bA^\circ;\bA^\sharp)\) is called a \textbf{relative partial applicative structure}
\item A \textbf{relative TPCA} is a relative partial applicative structure \((\bA^\circ,\bA^\sharp)\) s.t. there exist
elements \(k_{AB}, s_{ABC}\) in \(\bA^\sharp\) witnessing that \(\bA^\circ\) is a TPCA
\end{enumerate}
\end{definition}

\begin{definition}[]
Suppose \(\bA\) is a relative partial applicative structure over \(\sfT\)
\begin{enumerate}
\item The set of well-typed \textbf{applicative expressions} \(e:\sigma\) over \(\bA\) is defined inductively as
follows
\begin{itemize}
\item for each \(\sigma\in\sfT\), we have an unlimited supply of variables \(x^\sigma:\sigma\)
\item for each \(\sigma\in\sfT\) and \(a\in\bA^\sharp(\sigma)\), we have a \textbf{constant} symbol \(c_a:\sigma\) (we shall often
write \(c_a\) simply as \(a\))
\item If \(e:\sigma\to\tau\) and \(e':\sigma\) are applicative expressions, then \(ee'\) is an applicative
expression of type \(\tau\).
\end{itemize}

We write \(V(e)\) for the set of variables appearing in \(e\)

\item A \textbf{valuation} in \(\bA\) is a function \(v\) assigning to certain variables \(x^\sigma\) an
element \(v(x^\sigma)\in\bA^\circ(\sigma)\). Given an applicative expression \(e\) and a valuation \(v\)
covering \(V(e)\), the value \(\llb{e}_v\), when defined, is given inductively by
\begin{equation*}
\llb{x^\sigma}_v=v(x),\quad\llb{c_a}_v=a,\quad\llb{ee'}_\nu\simeq\llb{e}_v\cdot\llb{e'}_v
\end{equation*}
Note that if \(e:\tau\) and \(\llb{e}_v\) is defined then \(\llb{e}_v\in\bA^\circ(\tau)\).
\end{enumerate}
\end{definition}

Note that for any \(v\) with \(\ran(v)\in\bA^\sharp\), we can prove \(\llb{e:\tau}_v\in\bA^\sharp(\tau)\) by induction:
\begin{enumerate}
\item If \(e\) is of the form \(x^\tau\)
\item If \(e\) is of the form \(c_a\) where \(a\in\bA^\sharp(\tau)\)
\item If \(e\) is of the form \(e'e''\) where \(e':\sigma\to\tau\) and \(e'':\sigma\).

\(\llb{e}_v=\llb{e'}_v\cdot\llb{e''}_v\) where \(\llb{e'}_v\in\bA^\sharp(\sigma\to\tau)\) and \(\llb{e''}_v\in\bA^\sharp(\sigma)\).
Since \(\bA^\sharp\) is a substructure of \(\bA^\circ\), if \(\llb{e'}_v\cdot\llb{e''}_v\downarrow\), then \(\llb{e}\in\bA^\sharp(\tau)\)
\end{enumerate}

\begin{definition}[]
Let \(\bA\) be a relative partial applicative structure. We say \(\bA\) is \textbf{lax combinatory complete}
if for every applicative expression \(e:\tau\) over \(\bA\) and every variable \(x^\sigma\), there is an
applicative expression \(\lambda^*x^\sigma.e\) with \(V(\lambda^*x^\sigma.e)=V(e)-\{x^\sigma\}\) s.t. for any valuation \(v\)
covering \(V(\lambda^*x^\sigma.e)\) and any \(a\in\bA^\circ(\sigma)\) we have
\begin{equation*}
\llb{\lambda^*x^\sigma.e}_v\downarrow,\quad\llb{\lambda^*x^\sigma.e}_v\cdot a\succeq\llb{e}_{v,x\mapsto a}
\end{equation*}
We say \(\bA\) is \textbf{strictly combinatory complete} if this holds with '\(\simeq\)' in place of '\(\succeq\)'
\end{definition}

\begin{theorem}[]
\label{3.3.3}
A (relative) partial applicative structure \(\bA\) is a lax (relative) TPCA iff it is lax
combinatory complete
\end{theorem}

\begin{proof}
If \(\bA\) is lax combinatory complete, then for any \(\rho\), \(\sigma\), \(\tau\) we may define
\begin{align*}
k_{\sigma\tau}&=\llb{\lambda^*x^\sigma.(\lambda^*y^\tau.x)}_\emptyset\\
s_{\rho\sigma\tau}&=\llb{\lambda^*x^{\rho\to\sigma\to\tau}.(\lambda^*y^{\rho\to\sigma}.(\lambda^*z^\rho.xz(yz)))}_\emptyset
\end{align*}

Conversely, if \(\bA\) is a lax TPCA, then given any suitable choice of elements \(k\) and \(s\)
for \(\bA\), we may define \(\lambda^*x^\sigma.e\) by induction on the structure of \(e\):
\begin{align*}
\lambda^*x^\sigma.x&=s_{\sigma(\sigma\to\sigma)}k_{\sigma(\sigma\to\sigma)}k_{\sigma\sigma}&&\\
\lambda^*x^\sigma.a&=k_{\tau\sigma}a&&\text{for each }a\in\bA^\sharp(\tau)\\
\lambda^*x^\sigma.ee'&=s_{\sigma\tau\tau'}(\lambda^*x^\sigma.e)(\lambda^*x^\sigma.e')&&\text{if }e:\tau\to\tau',e':\tau\text{ and }ee'\text{ contains }x
\end{align*}
\end{proof}

The same argument shows that \(\bA\) is a strict TPCA iff it is strictly combinatory complete

we often tacitly suppose that a TPCA \(\bA\) comes equipped with some choice of k and s drawn from A♯,
and in this case we shall use the notation \(\lambda^*x.e\) for the applicative expression given by the
above proof. Since all the constants appearing in e are drawn from \(A^\sharp\), the same will be true for
\(\lambda^*x.e\).


In TPCAs constructed as syntactic models for untyped or typed \(\lambda\)-calculi (as in Example 3.1.6 or
Section 3.2.3), the value of \(\lambda^*x.e\) coincides with \(\lambda x.e\). However, the notational distinction is worth
retaining, since the term \(\lambda^*x.e\) as defined above is not syntactically identical
to \(\lambda x.e\).

More generally, we may consider terms of the \(\lambda\)-calculus as \textbf{meta-expressions} for applicative
expressions. Specifically any such \(\lambda\)-term \(M\) can be regarded as denoting an applicative
expression \(M^\dagger\) as follows:
\begin{equation*}
x^\dag=x,\quad c_a^\dag=c_a,\quad (MN)^\dag=M^\dag N^\dag,\quad(\lambda x.M)^\dag=\lambda^*x.(M^\dag)
\end{equation*}

Some caution is needed here, however, because \(\beta\)-equivalent meta-expressions
do not always have the same meaning
\begin{examplle}[]
Consider the two meta-expressions \((\lambda x.(\lambda y.y)x)\) and \(\lambda x.x\). Although these are
\(\beta\)-equivalent, the first expands to \(s(ki)i\) and the second to \(i\), where \(i\equiv skk\).
\end{examplle}

The moral here is that \(\beta\)-reductions are not valid underneath \(\lambda^*\)-abstractions: in this case,
the reduction \((\lambda^*y.y)x\rightsquigarrow x\) is not valid underneath \(\lambda^*\). However at
least for the definition of \(\lambda^*\) given above, \(\beta\)-reductions at top level are valid.

\begin{proposition}[]
\label{3.5.5}
\begin{enumerate}
\item If \(M\) is a meta-expression, \(x\) is a variable and \(a\) is a constant or variable,
then \(\llb{((\lambda x.M)a)^\dag}_v\succeq\llb{M[x\mapsto a]^\dag}\)
\item If \(M,N\) are meta-expressions, \(x\notin FV(N)\), no free occurrence of \(x\) in \(M\) occurs
under a \(\lambda\), and \(\llb{N^\dag}_v\downarrow\), then \(\llb{((\lambda x.M)N)^\dag}_v\succeq\llb{M[x\mapsto N]^\dag}_v\)
\end{enumerate}
\end{proposition}

\begin{proof}
Longley's PhD thesis
\end{proof}

From now on, we will not need to distinguish formally between meta-expressions and the
applicative expressions they denote. For the remainder of this chapter we shall use the \(\lambda^*\)
notation for such (meta-)expressions, retaining the asterisk as a reminder that the usual rules
of \(\lambda\)-calculus are not always valid.
\subsubsection{Pairing}
\label{sec:org387e8c7}
\begin{definition}[]
\begin{enumerate}
\item A \textbf{type world} is simply a set \(\sfT\) of \textbf{type names} \(\sigma\), optionally endowed with any or
all of the following:
\begin{enumerate}
\item a \textbf{fixing map}, assigning a set \(\sfT[\sigma]\) to certain type names \(\sigma\in\sfT\)
\item a \textbf{product structure}, consisting of a total binary operation \((\sigma,\tau)\mapsto\sigma\times\tau\)
\item an \textbf{arrow structure}, consisting of a total binary operation \((\sigma,\tau)\mapsto\sigma\to\tau\)
\end{enumerate}
\item A \textbf{computability model over} a type world \(\sfT\) is a computability model \(\bC\) with index
set \(\sfT\) (so that \(\abs{\bC}=\{\bC(\sigma)\mid\sigma\in\sfT\}\)) subject to the following conventions
\begin{enumerate}
\item If \(\sfT\) has a fixing map, then \(\bC(\sigma)=T[\sigma]\) whenever \(\sfT(\sigma)\) is defined
\item If \(\sfT\) has a product structure, then \(\bC\) has weak products and for any \(\sigma,\tau\in\sfT\)
we have \(\bC(\sigma\times\tau)=\bC(\sigma)\bowtie\bC(\tau)\)
\item If \(\sfT\) has an arrow structure, then \(\bC\) is a higher-order model and for
any \(\sigma,\tau\in\sfT\) we have \(\bC(\sigma\to\tau)=\bC(\sigma)\Rightarrow\bC(\tau)\)
\item If \(\sfT\) has both a product and an arrow structure, then \(\bC\) is weakly cartesian closed
\end{enumerate}
\end{enumerate}
\end{definition}

\begin{theorem}[]
There is a canonical bijection between higher-order models and relative TPCAs
\end{theorem}

Let \(\bA\) be a relative TPCA (which is combinatory complete) over a type world \(\sfT\) with arrow structure, and suppose
that \(\bA\) (considered as a higher-order model) has weak products, inducing a product
structure \(\times\) on \(\sfT\). This means that for any \(\sigma,\tau\in\sfT\) there are elements
\begin{equation*}
fst\in\bA^\sharp((\sigma\times\tau)\to\sigma),\quad snd\in\bA^\sharp((\sigma\times\tau)\to\tau)
\end{equation*}
And for each \(\sigma,\tau\in\sfT\) a \textbf{paring} operation
\begin{equation*}
pair\in\bA^\sharp(\sigma\to\tau\to(\sigma\times\tau))
\end{equation*}
s.t.
\begin{equation*}
\forall a\in\bA^\circ(\sigma),b\in\bA^\circ(\tau).\;fst\cdot(pair\cdot a\cdot b)=a\wedge snd\cdot(pair\cdot a\cdot b)=b
\end{equation*}

\begin{proposition}[]
A higher-order model with weak products has pairing iff it is weakly cartesian closed
\end{proposition}

\begin{lemma}[\ref{3.1.14}]
Suppose \(m,n>0\). Given
\begin{gather*}
f_j\in(A_0\Rightarrow\dots\Rightarrow A_{m-1}\Rightarrow B_j)^\sharp,\quad(j=0,\dots,n-1),\\
g\in(B_0\Rightarrow\dots\Rightarrow B_{n-1}\Rightarrow C)^\sharp
\end{gather*}
there exists \(h\in (A_0\Rightarrow\dots\Rightarrow A_{m-1}\Rightarrow C)^\sharp\) s.t.
\begin{equation*}
\forall a_0,\dots,a_{m-1}.h\cdot a_0\cdot\dots\cdot a_{m-1}\simeq g\cdot(f_0\cdot a_0\cdot\dots\cdot a_{m-1})\cdot\dots\cdot(f_{n-1}\cdot a_0\cdot\dots\cdot a_{m-1})
\end{equation*}
\end{lemma}

\begin{proof}
The binary partial functions representable in \(\bA^\sharp((\rho\times\sigma)\to\tau)\) are exactly those representable
in \(\bA^\sharp(\rho\to\sigma\to\tau)\)

Given \(f\in\bA^\sharp((\rho\times\sigma)\to\tau)\), by Proposition \ref{3.1.14}, we have \(h\in\bA^\sharp(\rho\to\sigma\to\tau)\) where
\begin{equation*}
\forall a,b.\; h\cdot a\cdot b\simeq f\cdot(pair\;\cdot a\cdot b)
\end{equation*}

Given \(f\in\bA^\sharp(\rho\to\sigma\to\tau)\), by the same Proposition, we have \(h\in\bA^\sharp((\rho\times\sigma)\to\tau)\) where
\begin{equation*}
\forall a,b.\; h\cdot c\simeq f\cdot(fst\cdot c)\cdot(snd\cdot c)
\end{equation*}
\end{proof}

Henceforth we shall generally work with pair in preference to the ‘external’ pairing of
operations, and will write \(pair\cdot a\cdot b\) when there is no
danger of confusion.

In untyped models, pairing is automatic
 \begin{equation*}
pair=\lambda^*xyz.zxy,\quad fst=\lambda^*p.p(\lambda^*xy.x),\quad snd=\lambda^*p.p(\lambda^*xy.y)
 \end{equation*}
\subsubsection{Booleans}
\label{sec:org95c9e64}
\begin{definition}[]
A model \(\bA\) has \textbf{booleans} if for some type \(\ttB\) there exist elements
\begin{align*}
\TT,\FF&\in\bA^\sharp(\ttB)\\
\IF_\sigma&\in\bA(\ttB,\sigma,\sigma\to\sigma)\text{ for each }\sigma
\end{align*}
s.t. for all \(x,y\in\bA^\circ(\sigma)\) we have
\begin{equation*}
\IF_\sigma\cdot\TT\cdot x\cdot y=x,\quad\IF_\sigma\cdot\FF\cdot x\cdot y=y
\end{equation*}
Note that \(\TT,\FF\) need not be the sole element of \(\bA^\sharp(\bB)\)
\end{definition}

Alternatively, we may define a notion of having booleans in the setting of computability
model \(\bC\) with weak products: replace \(\IF_\sigma\) with \(\IF'_\sigma\in\bC[\B\times\sigma\times\sigma,\sigma]\). In a TPCA with
products and pairing the two definitions coincide

In untyped models, the existence of booleans is automatic: \(\TT=\lambda^*xy.x\), \(\FF=\lambda^*xy.y\)
and \(\IF=\lambda^*zxy.zxy\)

Obviously, the value of an expression \(\IF_\sigma\cdot b\cdot e\cdot e'\) cannot be defined unless the values of
both \(e\) and \(e'\) are defined. However, there is a useful trick that allows us to build conditional
expressions whose definedness requires only that the chosen branch of the conditional is
defined. This trick is specific to the higher-order setting, and is known as \textbf{strong definition
by cases}:

\begin{proposition}[]
Suppose \(\bA\) has booleans as above. Given applicative expressions \(e,e':\sigma\) there is an
applicative expression \((e\mid e'):\ttB\to\sigma\) s.t. for any valuation \(v\) covering \(V(e)\)
and \(V(e')\) we have
\begin{equation*}
\llb{(e\mid e')}_v\downarrow,\quad\llb{(e\mid e')\cdot\TT}_v\succeq\llb{e}_v,\quad
\llb{(e\mid e')\cdot\FF}_v\succeq\llb{e'}_v
\end{equation*}
\end{proposition}

\begin{proof}
Let \(\rho\) be any type s.t. \(\bA^\circ(\rho)\) is inhabited by some element \(a\), and define
\begin{equation*}
(e\mid e')=\lambda^*z^{\ttB}\cdot(\IF_\sigma z(\lambda^*r^\rho.e)(\lambda^*r^\rho.e')c_a)
\end{equation*}
where \(z,r\) are fresh variables

\(\llb{(e\mid e')}_v\downarrow\)  since by lax combinatory completeness

\(\llb{(e\mid e')\cdot\TT}_v\succeq\llb{e}_v\) by \ref{3.5.5}
\end{proof}

The expressions \(\lambda^*r.e\), \(\lambda^*r.e'\) in the above proof are known as \textbf{suspensions} or \textbf{thunks}: the idea is
that \(\llb{\lambda^*r.e}_v\) is guaranteed to be defined, but the actual evaluation of \(e_v\) (which may be
undefined) is ‘suspended’ until the argument \(c_a\) is supplied.
\subsubsection{Numerals}
\label{sec:org53ea72a}
\begin{definition}[]
A model \(\bA\) has \textbf{numerals} if for some type \(\ttN\) there exist
\begin{align*}
\hat{0},\hat{1},\hat{2},\dots&\in\bA^\sharp(\ttN)\\
suc&\in\bA^\sharp(\ttN\to\ttN)
\end{align*}
and for any \(x\in\bA^\sharp(\sigma)\) and \(f\in\bA^\sharp(\ttN\to\sigma\to\sigma)\) an element
\begin{equation*}
Rec_\sigma(x,f)\in\bA^\sharp(\ttN\to\sigma)
\end{equation*}
s.t. for all \(x\in\bA^\sharp(\sigma)\), \(f\in\bA^\sharp(\ttN\to\sigma\to\sigma)\) and \(n\in\N\) we have
\begin{align*}
suc\cdot\hatn&=\what{n+1}\\
Rec_\sigma(x,f)\cdot\hat{0}&=x\\
Rec_\sigma(x,f)\cdot\what{n+1}&\succeq f\cdot\hatn\cdot(Rec_\sigma(x,f)\cdot\hatn)
\end{align*}
\end{definition}

The above definition has the advantage that it naturally adapts to the setting of a
computability model C with products: just replace the types of \(f\) and \(Rec_\sigma(x,f)\) above
with \(\bC[\ttN\times\sigma,\sigma]\) and \(\bC[\ttN,\sigma]\) respectively.

\begin{proposition}[]
A model \(\bA\) has numerals iff it has elements \(\hatn\), \(suc\) as above and
\begin{equation*}
rec_\sigma\in\bA^\sharp(\sigma\to(\ttN\to\sigma\to\sigma)\to\ttN\to\sigma)\quad\text{for each }\sigma
\end{equation*}
s.t. for all \(x\in\bA^\circ(\sigma)\), \(f\in\bA^\circ(\ttN\to\sigma\to\sigma)\) and \(n\in\N\) we have
\begin{align*}
suc\cdot\hatn&=\what{n+1}\\
rec_\sigma\cdot x\cdot f\cdot\hat{0}&=x\\
rec_\sigma\cdot x\cdot f\cdot\what{n+1}&\succeq f\cdot\hatn\cdot(rec_\sigma\cdot x\cdot f\cdot\hatn)
\end{align*}
\end{proposition}

\begin{proof}
\(\Leftarrow\): Let \(Rec_\sigma(x,f)=rec_\sigma\cdot x\cdot f\)

\(\Rightarrow\): define
\begin{equation*}
rec_\sigma=Rec_{\sigma\to(\ttN\to\sigma\to\sigma)\to\sigma}(\lambda^*xf.x,\lambda^*nr.\lambda^*xf.fn(rxf))
\end{equation*}
?
\end{proof}

\begin{exercise}
Show that \(\bA\) has numerals, then \(\bA\) has booleans
\end{exercise}

\begin{proposition}[]
\label{3.3.13}
Every untyped model has numerals
\end{proposition}

\begin{proof}
Using the encodings for pairings and booleans given above, we may define the \textbf{Curry
numerals} \(\hatn\) in any untyped models as follows:
\begin{equation*}
\hat{0}=\la\TT,\TT\ra,\quad\what{n+1}=\la\FF,\hatn\ra
\end{equation*}
and \(suc=\lambda^*x.\la\FF,x\ra\). We also have elements for the zero testing and predecessor operations:
take \(iszero=fst\) and \(pre=\lambda^*x.\IF(iszero\;x)\hat{0}(snd\;x)\)
\end{proof}

In any model with numerals, a rich class of functions \(\N^r\to\N\) is representable. For example, the
(first-order) primitive recursive functions on \(\N\)

\begin{proposition}[]
\label{3.3.14}
For any primitive recursive \(f:\N^r\to\N\) there is an applicative expression \(e_f:\N^{(r)}\to\N\)
(involving constants 0, \(suc\), \(rec_{\ttN}\)) s.t. in any model \((\bA^\circ;\bA^\sharp)\) with numerals
we have \(\llb{e_f}_v\in\bA^\sharp\) (where \(v\) is the obvious valuation of the constants) and
\begin{equation*}
\forall n_0,\dots,n_{r-1},m. f(n_0,\dots,n_{r-1})=m\Rightarrow\llb{e_f}_v\cdot\hatn_0\cdot\hatn_{r-1}=\hatm
\end{equation*}
\end{proposition}

\begin{proof}

\end{proof}
\subsubsection{Recursion and Minimization}
\label{sec:orga7f2bcf}
\begin{definition}[]
\begin{enumerate}
\item A total model \(\bA\) \textbf{has general recursion}, or \textbf{has fixed points}, if for every
element \(f\in\bA^\sharp(\rho\to\rho)\) there is an element \(Fix_\rho(f)\in\bA^\sharp(\rho)\) s.t. \(Fix_\rho(f)=f\cdot Fix_\rho(f)\)
\item An arbitrary model \(\bA\) \textbf{has guarded recursion}, or \textbf{guarded fixed points}, if for every
element \(f\in\bA^\sharp(\rho\to\rho)\) where \(\rho=\sigma\to\tau\) there is an element \(GFix_\rho(f)\in\bA^\sharp(\rho)\)
s.t. \(GFix_\rho(f)\cdot x\succeq f\cdot GFix_\rho(f)\cdot x\) for all \(x\in\bA^\circ(\sigma)\)
\end{enumerate}
\end{definition}

\begin{proposition}[]
\begin{enumerate}
\item A total model \(\bA\) has general recursion iff for every type \(\rho\) there is an
element \(Y_\rho\in\bA^\sharp((\rho\to\rho)\to\rho)\) s.t. for all \(f\in\bA^\circ(\rho\to\rho)\) we have
\begin{equation*}
Y_\rho\cdot f=f\cdot(Y_\rho\cdot f)
\end{equation*}
\item \(\bA\) has guarded recursion iff for every type \(\rho=\sigma\to\tau\) there is an
element \(Z_\rho\in\bA^\sharp((\rho\to\rho)\to\rho)\) s.t. for all \(f\in\bA^\circ(\rho\to\rho)\) and \(x\in\bA^\circ(\sigma)\) we have
\begin{equation*}
Z_\rho\cdot f\downarrow,\quad Z_\rho\cdot f\cdot x\succeq f\cdot(Z_\rho\cdot f)\cdot x
\end{equation*}
\end{enumerate}
\end{proposition}

\begin{proof}
Define
\begin{equation*}
Y_\rho=Fix_{(\rho\to\rho)\to\rho}(\lambda^*y.\lambda^*f.f(yf)),\quad Z_\rho=GFix_{(\rho\to\rho)\to\rho}(\lambda^*z.\lambda^*fx.f(zf)x)
\end{equation*}
\end{proof}

Not all models of interest possess such recursion operators. Clearly, if \(\bA\) is a \textbf{total} model
with \(\bA(\ttN)=\N\) a type of numerals as above, then \(\bA\) cannot have general or even guarded
recursion: if \(\rho=\ttN\to\ttN\)  and \(f=\lambda^*gx.suc(gx)\) then we would
have \(Z\cdot f\cdot\hatn=suc\cdot Z\cdot f\cdot\hatn\), which is impossible. However, many models
with \(\bA(\ttN)=\N_\bot\) will have general recursion

Any \textbf{untyped} total model has general recursion, since we may take
\begin{equation*}
W=\lambda^*wf.f(wwf),\quad Y=WW
\end{equation*}
(This element \(Y\) is known as the \textbf{Turing fixed point combinator}). Likewise, every untyped
model, total or not, has guarded recursion, since we may take
\begin{equation*}
V=\lambda^*vfx.f(vvf)x,\quad Z=VV
\end{equation*}

Note in passing that Kleene’s \textbf{second recursion theorem} from classical computability theory is
tantamount to the existence of a guarded recursion operator in \(K_1\)

We can now prove \ref{3.3.13}. In any untyped model, let \(Z\) be a guarded recursion operator,
define
\begin{equation*}
R=\lambda^*rxfm.\IF(iszero\;m)(kx)(\lambda^*y.f(pre\;m))(rxf(pre\; m)\hat{0})
\end{equation*}
and take \(rec=\lambda^*xfm.(ZR)xfmi\).

\begin{definition}[]
A model \(\bA\) with numerals \textbf{has minimization} if it contains an
element \(min\in\bA^\sharp((\ttN\to\ttN)\to\ttN)\) s.t. whenever \(\hatg\in\bA^\circ(\ttN\to\ttN)\) represents some
total \(g:\N\to\N\) and \(m\) is the least number s.t. \(g(m)=0\), we have \(min\cdot\hatg=\hatm\)
\end{definition}

\begin{proposition}[]
\label{3.3.19}
There is an applicative expression Min involving constants \(\hat{0}\), suc, iszero, if
and \(Z\) s.t. in any model with numerals and guarded recursion, \(\llb{Min}_v\) is a
minimization operator
\end{proposition}

\begin{proof}
Take \(Min=Z(\lambda^*M.\lambda^*g.\IF(iszero(g\;\hat{0}))\hat{0}(M(\lambda^*n.g(suc\;n))))\)
\end{proof}

\begin{proposition}[]
For any partial computable \(f:\N^r\rhu\N\) there is an applicative
expression \(e_f:\ttN^{(r)}\to\ttN\) (involving constants 0, suc, \(rec_{\ttN}\), min) s.t. in any
model \(\bA\) with numerals and minimization we have \(\llb{e_f}_v\in\bA^\sharp\) (with the obvious
valuation \(v\)) and
\begin{equation*}
\forall n_0,\dots,n_{r-1},m. f(n_0,\dots,n_{r-1})=m\Rightarrow\llb{e_f}_v\cdot\hatn_0\cdot\dots\cdot\hatn_{r-1}=\hatm
\end{equation*}
\end{proposition}

\begin{proof}
Since our definition of minimization refers only to total functions \(g:\N\to\N\), we appeal to the
\emph{Kleene normal form} theorem: there are primitive recursive functions \(T:\N^{r+2}\to\N\) and \(U:\N\to\N\)
such that any partial computable \(f\) has an ‘index’ \(e\in\N\) such
that \(f(\barn)\simeq U(\mu y.T(e,\barn,y)=0)\)
for all \(\barn\). Using this, the result follows easily from Propositions \ref{3.3.14} and \ref{3.3.19}.
\end{proof}
\subsubsection{The Category of Assemblies}
\label{sec:org5b9d219}
\begin{definition}[]
Let \(\bC\) be a lax computability model over \(T\). The \textbf{category of assemblies over} \(\bC\),
written \(\Asm(\bC)\) is defined as follows:
\begin{itemize}
\item Objects \(X\) are triples \((\abs{X},\rho_X,\Vdash_X)\) where \(\abs{X}\) is a set, \(\rho_X\in T\)
names some type, and \(\Vdash_X\subseteq\bC(\rho_X)\times\abs{X}\) is a relation
s.t. \(\forall x\in\abs{X}.\exists a\in\bC(\rho_X).a\Vdash_Xx\) (The formula \(a\Vdash_Xx\) may be read as '\(a\)
\textbf{realizes} \(x\)')
\item A morphism \(f:X\to Y\) is a function \(f:\abs{X}\to\abs{Y}\) that is \textbf{tracked} by
some \(\barf\in\bC[\rho_X,\rho_Y]\), in the sense that for any \(x\in\abs{X}\) and \(a\in\bC(\rho_X)\) we have
\begin{equation*}
a\Vdash_Xx\Rightarrow\barf(a)\Vdash_Yf(x)
\end{equation*}
\end{itemize}


An assembly \(X\) is called \textbf{modest} if \(a\Vdash_Xx\wedge a\Vdash_Xx'\) implies \(x=x'\). We
write \(\nMod(\bC)\) for the full subcategory of \(\Asm(\bC)\) consisting of modest assemblies
\end{definition}

Intuitively, we regard an assembly \(X\) as an ``abstract datatype'' for which we have a concrete
implementation on the ``machine'' \(\bC\). The underlying set \(\abs{X}\) is the set of values of
the abstract type, and for each \(x\in\abs{X}\), the elements \(a\Vdash_Xx\) are the possible
machine representations of this abstract value. (Note that an abstract value \(x\) may have many
possible machine representations \(a\).) The morphisms \(f:X\to Y\) may then be regarded as the
``computable mappings'' between such datatypes

In the case that \(\bC\) is a lax TPCA \(\bA\), we may also denote the above
categories \(\Asm(\bA)\), \(\nMod(\bA)\), or by \(\Asm(\bA^\circ;\bA^\sharp)\), \(\nMod(\bA^\circ;\bA^\sharp)\). Note that
realizers for elements \(x\in\abs{X}\) may be arbitrary elements of \(\bA^\circ(\rho_X)\), whereas a
morphism \(f:X\to Y\) must be tracked by an element of \(\bA^\sharp(\rho_X\to\rho_Y)\)

Viewed in this way, all the datatypes we shall typically wish to consider in fact live in the
subcategory \(\nMod(\bC)\): an abstract data value is uniquely determined by any of its machine
representations. Note also that if \(Y\) is modest, a morphism \(f:X\to Y\) is completely determined by
any \(\barf\) that tracks it.

\begin{definition}[]
Let the category \(\bC\) have binary products. An \textbf{exponential} of objects \(B\) and \(C\) consists
of an object \(C^B\) and an arrow \(\epsilon:C^B\times B\to C\) s.t. for any object \(A\) and
arrow \(f:A\times B\to C\) there is a unique arrow \(\tilf:A\to C^B\) s.t. \(\epsilon\circ(\tilf\times 1_B)=f\)
\begin{center}\begin{tikzcd}
C^B\\
A\ar[u,"\tilf"']
\end{tikzcd}\hspace{1cm}\begin{tikzcd}
C^B\times B\ar[r,"\epsilon"]&C\\
A\times B\ar[u,"\tilf\times 1_B"]\ar[ur,"f"']
\end{tikzcd}\end{center}
\end{definition}


\begin{theorem}[]
\label{3.3.22}
Let \(\bC\) be a lax computability model
\begin{enumerate}
\item If \(\bC\) has a weak terminal, then \(\Asm(\bC)\) has a terminal object 1
\item If \(\bC\) has weak products, then \(\Asm(\bC)\) has binary cartesian products
\item If \(\bC\) weakly cartesian closed, then \(\Asm(\bC)\) is cartesian closed
\item If \(\bC\) has a weak terminal and booleans, \(\Asm(\bC)\) has the coproduct \(1+1\)
\item If \(\bC\) has a weak terminal and numerals, \(\Asm(\bC)\) has a natural number object
\end{enumerate}
\end{theorem}

\begin{proof}
\begin{enumerate}
\item If \((I,i)\) is a weak terminal, define \(1=(\{i\},I,\Vdash_1=\{(i,i)\})\). Then for any \(X\in\Asm(\bC)\),
\(f=\Lambda x.i\) is the unique morphism where \(\barf=\Lambda x.i\).
\setcounter{enumi}{1}
\item If \(X\) and \(Y\) are assemblies and \(\rho\) is a weak product of \(\rho_X\) and \(\rho_Y\), define
the assembly \(X\times Y\) by
\begin{equation*}
\abs{X\times Y}=\abs{X}\times\abs{Y},\quad\rho_{X\times Y}=\rho,\quad a\Vdash_{X\times Y}(x,y)\text{ iff }
\pi_X(a)\Vdash_Xx\wedge\pi_Y(a)\Vdash_Yy
\end{equation*}
\item If \(X\) and \(Y\) are assemblies, let us say an element \(t\in\bC(\rho_X\to\rho_Y)\) \textbf{tracks} a
function \(f:\abs{X}\to\abs{Y}\) if
\begin{equation*}
\forall x\in\abs{X},a\in\bC(\rho_X).\;a\Vdash_Xx\Rightarrow t\cdot_{XY}a\Vdash_Yf(x)
\end{equation*}
Now define the assembly \(Y^X\) as follows:
\begin{align*}
\abs{Y^X}&=\{f:\abs{X}\to\abs{Y}\mid f\text{ is tracked by some }t\in\bC(\rho_X\to\rho_Y)\}\\
\rho_{Y^X}&=\rho_X\to\rho_Y\\
t\Vdash_{Y^X}f&\Leftrightarrow t\text{ tracks }f
\end{align*}
\end{enumerate}
\end{proof}

Theorem \ref{3.2.22} also holds with \(\nMod(\bC)\), and the
inclusion \(\nMod(\bC)\hookrightarrow\Asm(\bC)\) preserves all the relevant structure
\end{document}
