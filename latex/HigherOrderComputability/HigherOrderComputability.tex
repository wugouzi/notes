% Created 2022-04-04 Mon 12:56
% Intended LaTeX compiler: pdflatex
\documentclass[11pt]{article}
\usepackage[utf8]{inputenc}
\usepackage[T1]{fontenc}
\usepackage{graphicx}
\usepackage{longtable}
\usepackage{wrapfig}
\usepackage{rotating}
\usepackage[normalem]{ulem}
\usepackage{amsmath}
\usepackage{amssymb}
\usepackage{capt-of}
\usepackage{hyperref}
\graphicspath{{../../books/}}
% TIPS
% \substack{a\\b} for multiple lines text





% pdfplots will load xolor automatically without option
\usepackage[dvipsnames]{xcolor}

\usepackage{forest}
% two-line text in node by [two \\ lines]
% \begin{forest} qtree, [..] \end{forest}
\forestset{
  qtree/.style={
    baseline,
    for tree={
      parent anchor=south,
      child anchor=north,
      align=center,
      inner sep=1pt,
    }}}
%\usepackage{flexisym}
% load order of mathtools and mathabx, otherwise conflict overbrace

\usepackage{mathtools}
%\usepackage{fourier}
\usepackage{pgfplots}
\usepackage{amsthm, mathabx,  amsmath, commath}
\usepackage{amsfonts}

\usepackage{empheq}
\usepackage{tikz}
\usetikzlibrary{arrows.meta}
\usepackage[most]{tcolorbox}

\newtheorem{theorem}{Theorem}[section]
\newtheorem{definition}{Definition}[section]
\newtheorem{corollary}{Corollary}[section]
\newtheorem{example}{Example}[section]
\newtheorem{lemma}{Lemma}[section]
\newtheorem{proposition}{Proposition}[section]

\newcommand{\bl}[1] {\boldsymbol{#1}}
\newcommand{\Wt}[1] {\stackrel{\sim}{\smash{#1}\rule{0pt}{1.1ex}}}
\newcommand{\wt}[1] {\widetilde{#1}}


%For boxed texts in align, use Aboxed{}
%otherwise use boxed{}

\DeclareMathSymbol{\widehatsym}{\mathord}{largesymbols}{"62}
\newcommand\lowerwidehatsym{%
  \text{\smash{\raisebox{-1.3ex}{%
    $\widehatsym$}}}}
\newcommand\fixwidehat[1]{%
  \mathchoice
    {\accentset{\displaystyle\lowerwidehatsym}{#1}}
    {\accentset{\textstyle\lowerwidehatsym}{#1}}
    {\accentset{\scriptstyle\lowerwidehatsym}{#1}}
    {\accentset{\scriptscriptstyle\lowerwidehatsym}{#1}}
}

\usepackage{graphicx}
    
% text on arrow for xRightarrow
\makeatletter
%\newcommand{\xRightarrow}[2][]{\ext@arrow 0359\Rightarrowfill@{#1}{#2}}
\makeatother


\def \bx {\boldsymbol{x}}
\def \ba {\boldsymbol{a}}
\def \bI {\boldsymbol{I}}
\def \bt {\boldsymbol{t}}
\def \bb {\boldsymbol{b}}
\def \bA {\boldsymbol{A}}
\def \bX {\boldsymbol{X}}
\def \bu {\boldsymbol{u}}
\def \bS {\boldsymbol{S}}
\def \bZ {\boldsymbol{Z}}
\def \bz {\boldsymbol{z}}
\def \by {\boldsymbol{y}}
\def \bw {\boldsymbol{w}}
\def \bT {\boldsymbol{T}}
\def \bS {\boldsymbol{S}}
\def \bm {\boldsymbol{m}}
\def \bW {\boldsymbol{W}}
\def \bY {\boldsymbol{Y}}
\def \bH {\boldsymbol{H}}
\def \blambda {\boldsymbol{\lambda}}
\def \bPhi {\boldsymbol{\Phi}}
\def \btheta {\boldsymbol{\theta}}
\def \bmu {\boldsymbol{\mu}}
\def \bphi {\boldsymbol{\phi}}
\def \bSigma {\boldsymbol{\Sigma}}
\def \lb {\left\{}
\def \rb {\right\}}
\def \caln {\mathcal{N}}
\def \dissum {\displaystyle\Sigma}
\def \dispro {\displaystyle\prod}
\def \E {\mathbb{E}}
\def \Q {\mathbb{Q}}
\def \V {\mathbb{V}}
\def \R {\mathbb{R}}
\def \calq {\mathcal{Q}}
\def \calg {\mathcal{G}}
\def \caln {\mathcal{N}}
\def \calr {\mathcal{R}}
\def \calm {\mathcal{M}}
\def \calc {\mathcal{C}}
\def \bcup {\bigcup}

\newcommand{\ssmile}[1]{\mathord{\stackrel{\smallsmile}{#1}}}
\DeclareMathOperator{\lv}{lv}
\newcommand{\FF}{f\mspace{-7mu}f}
\newcommand{\TT}{t\mspace{-3mu}t}
\newcommand{\IF}{i\mspace{-4mu}f}
\DeclareMathOperator{\Asm}{\mathcal{A}sm}
\DeclareMathOperator{\nMod}{\mathcal{M}od}
\DeclareMathOperator{\PC}{\textsf{PC}}
\DeclareMathOperator{\PCF}{PCF}
\DeclareMathOperator{\eff}{eff}
\DeclareMathOperator{\tT}{T}
\makeindex
\author{John Longley \& Dag Normann}
\date{\today}
\title{Higher Order Computability}
\hypersetup{
 pdfauthor={John Longley \& Dag Normann},
 pdftitle={Higher Order Computability},
 pdfkeywords={},
 pdfsubject={},
 pdfcreator={Emacs 28.0.92 (Org mode 9.6)}, 
 pdflang={English}}
\begin{document}

\maketitle
\tableofcontents


\section{Theory of Computability Models}
\label{sec:org772aa94}
\subsection{Further Examples of Higher-Order Models}
\label{sec:org1838659}
\subsubsection{Kleene's Second Model}
\label{sec:orgef1c5ca}
Whereas \(K_1\) captures a notion of computability for finitary data such as numbers, our next
example embodies a notion of computability in an \emph{infinitary} setting.

When our computation returns a result \(F(g)\in\N\), it must do so after only a finite number of
'computation steps'

If \(F:\N^{\N}\to\N\) is computable, the function \(F\) can be completely described by recording, for
each finite sequence \(m_0,\dots,m_{r-1}\), whether the information '\(g(0)=m_0,\dots,g(r-1)=m-1\)'
suffices to determine the value of \(F(g)\), and if so, what the value is. We write \(\la\cdots\ra\)for
standard computable operation \(\N^*\to\N\)
\begin{align*}
&f(\la m_0,\dots,m_{r-1}\ra)=m-1&&\text{if }F(g)=m\text{ whenever }g(0)=m_0,\dots,g(r-1)=m_{r-1}\\
&f(\la m_0,\dots,m_{r-1}\ra)=0&&\text{'}g(0)=m_0,\dots,g(r-1)=m_{r-1}\text{' does not}\\
&&&\text{suffice to determine }F(g)
\end{align*}
We can then compute \(F(g)\) from just \(f\) and \(g\)
\begin{align*}
F(g)=(f\mid g):=&f(\la g(0),\dots,g(r-1)\ra)-1\\
&\text{where }r=\mu r.f(\la g(0),\dots,g(r-1)\ra)>0
\end{align*}
Even if \(f\in\N^{\N}\) does not represent some \(F\) in this way, we can regard the above formula
as defining a \emph{partial} computable function \((f\mid-):\N^{\N}\rhu\N\)

A small tweak is to obtain an application operation with codomain \(\N^{\N}\) rather than \(\N\).
In effect, the computaion now accepts an additional argument \(n\in N\), which can be assumed to
be known before any queries to \(g\) are made. We may therefore define
\begin{align*}
(f\odot g)(n):=&f(\la n,g(0),\dots,g(r-1)\ra)-1\\
&\text{where }r=\mu r.f(\la n,g(0),\dots,g(r-1)\ra)>0
\end{align*}
In general, this will define a \textbf{partial} function \(\N\rhu\N\). We shall henceforth
consider \(f\odot g\) to be 'defined' only if the above formula yields a \emph{total} function \(\N\to\N\). In
this way, we obtain a partial application operation
\begin{equation*}
\odot:\N^{\N}\times\N^{\N}\rhu\N^{\N}
\end{equation*}
The structure \((\N^{\N},\odot)\) is known to be \textbf{Kleene's second model}

We can also restrict \(\odot\) to the set of total \emph{computable}' functions \(\N\to\N\). Since the natural
candidates for \(k,s\) in \(K_2\) are computable, this gives us both a PCA \(K_2^{\eff}\) and a
relative PCA \((K_2;K_2^{\eff})\)
\subsubsection{Typed \texorpdfstring{\(\lambda\)}{λ}-Calculi}
\label{sec:org167de23}

We work in the type world \(\sfT=\sfT^{\to}(\beta_0,\dots,\beta_{r-1})\) where \(\beta_i\) are base types. We
assume given some set \(C\) of \textbf{constant symbols} \(c\), each with an assigned type \(\tau_c\in\sfT\);
we also suppose we have an unlimited supply of \textbf{variable symbols} \(x^\sigma\) for each \(\sigma\).

\(\Xi\) specifies \textbf{basic evaluation contexts} \(K[-]\)

\begin{examplle}[]
Let \(C\) consist of the following constants
\begin{align*}
\hat{0}&:\ttN\\
suc&:\ttN\to\ttN\\
rec_\sigma&:\sigma\to(\sigma\to\ttN\to\sigma)\to\ttN\to\sigma
\end{align*}
Let \(\Delta\) consist of the \(\delta\)-rules
\begin{gather*}
rec_\sigma xf\hat{0}\rightsquigarrow x\\
rec_\sigma xf(suc\; n)\rightsquigarrow f(rec_\sigma xfn)n
\end{gather*}
and let \(\Xi\) consist of the contexts of the forms
\begin{equation*}
[-]N,\quad suc[-],\quad rec_\sigma PQ[-]
\end{equation*}
The first means that if \(M\rightsquigarrow M'\) then \(MN\rightsquigarrow MN'\)

This defines Gödel's \textbf{System \(T\)}
\end{examplle}

\begin{examplle}[Plotkin's PCF]
Let \(C\) consists of the following constants
\begin{align*}
\hat{0}&:\ttN\quad\forall n\in\N\\
suc,pre&:\ttN\to\ttN\\
ifzero&:\ttN,\ttN,\ttN\to\ttN\\
Y_\sigma&:(\sigma\to\sigma)\to\sigma\quad\forall\sigma\in\sfT
\end{align*}
Let \(\Delta\) consist of rules of the forms
\begin{align*}
suc\;\hatn&\rightsquigarrow\what{n+1}\\
pre\;\what{n+1}&\rightsquigarrow\hatn\\
pre\;\hat{0}&\rightsquigarrow\hat{0}\\
ifzero\;\hat{0}&\rightsquigarrow\lambda xy.x\\
ifzero\;\what{n+1}&\rightsquigarrow\lambda xy.y\\
Y_\sigma f&\rightsquigarrow f(Y_\sigma f)
\end{align*}
and let \(\Xi\) consist of the contexts of the forms
\begin{equation*}
[-]N,\quad suc[-],\quad pre[-],\quad ifzero[-]
\end{equation*}
The resulting language is known as Plotkin's PCF (\emph{Programming language for Computable Functions})
\end{examplle}

Given \(\Delta\) and \(\Xi\) as above, let us define a \textbf{one-step reduction} relation \(\rightsquigarrow\) between closed terms
of the same type by means of the following inductive definition
\begin{enumerate}
\item \((\lambda x^\sigma.M)N\rightsquigarrow M[x^\sigma\mapsto N]\) (\textbf{\(\beta\)-reduction})
\item If \((cM_0\dots M_{r-1}\rightsquigarrow N)\in\Delta\) and \(-^*\) denotes some type-respecting substitution of closed
terms for the free variables of \(cM_0\dots M_{r-1}\),
then \((cM_0^*\dots M_{r-1}^*)\rightsquigarrow N^*\)
\item If \(M\rightsquigarrow M'\) and \(K[-]\in\Xi\), then \(K[M]\rightsquigarrow K[M']\)
\end{enumerate}


We call the entire system the \textbf{typed \(\lambda\)-calculus} specified by \(\vec{\beta},C,\Delta,\Xi\)

Depending on the choice of \(\Delta\) and \(\Xi\), the relation \(\rightsquigarrow\) may or may not be \textbf{deterministic} in the
sense that \(M\rightsquigarrow M'\wedge M\rightsquigarrow M''\Rightarrow M'=M''\). Note that both System T and PCF have deterministic
reduction relations.

Typically we are interested in the possible values of a closed term \(M\), that is, those
terms \(M'\) s.t. \(M\rightsquigarrow^*M'\) and \(M'\) can be reduced no further. In System T, every
closed \(M:\ttN\) reduces to a unique numeral \(suc^k\hat{0}\). In PCF, there are terms such
as \(Y_{\ttN}(\lambda x^{\ttN}.x)\)  that can never be reduced to a value. Thus System T is in essence
a language for \textbf{total} functions, whilst PCF is a language for \textbf{partial} ones


Let \(=_{op}\) be the \textbf{operational equivalence} relation on closed terms generated by
\begin{equation*}
M\rightsquigarrow M'\Rightarrow M=_{op}M',\quad M=_{op}M'\Rightarrow MN=_{op}M'N\wedge PM=_{op}PM'
\end{equation*}

\subsection{Simulations Between Computability Models}
\label{sec:org2edba36}
\subsubsection{Simulations and Transformations}
\label{sec:org1530e10}
\begin{definition}[]
Let \(\bC\) and \(\bD\) be lax computability models over type worlds \(T,U\) respectively. A
\textbf{simulation} \(\gamma\) of \(\bC\) in \(\bD\) (written in \(\gamma:\bC\lrarhd\bD\))
consist of
\begin{itemize}
\item a mapping \(\sigma\mapsto\gamma\sigma:T\to U\)
\item for each \(\sigma\in T\), a relation \(\Vdash_\sigma^\gamma\subseteq\bD(\gamma\sigma)\times\bC(\sigma)\)
\end{itemize}
satisfying the following
\begin{enumerate}
\item For all \(a\in\bC(\sigma)\) there exists \(a'\in\bD(\gamma\sigma)\) s.t. \(a'\Vdash_\sigma^\gamma a\)
\item Every operation \(f\in\bC[\sigma,\tau]\) is \textbf{tracked} by some \(f'\in\bD[\gamma\sigma,\gamma\tau]\), in the sense that
whenever \(f(a)\downarrow\) and \(a'\Vdash_\sigma^\gamma a\), we have \(f'(a)\Vdash_\tau^\gamma f(a)\)
\end{enumerate}
\end{definition}

For any \(\bC\) we have the \textbf{identity} simulation \(\id_{\bC}:\bC\lrarhd\bC\) given by \(\id_{\bC}\sigma=\sigma\)
and \(a'\Vdash_{\sigma}^{\id_{\bC}}a\) iff \(a'=a\)

Given simulations \(\gamma:\bC\lrarhd\bD\) and \(\delta:\bD\lrarhd\bE\) we have the composite
simulation \(\delta\circ\gamma:\bC\lrarhd\gamma\bE\) defined by \((\delta\circ\gamma)\sigma=\delta(\gamma\sigma)\) and \(a'\Vdash_\sigma^{\delta\circ\gamma}a\)  iff there
exists \(a''\in\bD(\gamma\sigma)\) with \(a''\Vdash_\sigma^\gamma a\) and \(a'\Vdash_{\gamma\sigma}^\delta a''\).

\begin{definition}[]
Let \(\bC\), \(\bD\) be lax computability models and suppose \(\gamma,\delta:\bC\lrarhd\bD\) are simulations. We
say \(\gamma\) is \textbf{transformable} to \(\delta\), and write \(\gamma\preceq\delta\), if for each \(\sigma\in\abs{\bC}\) there is an
operation \(t\in\bD[\gamma\sigma,\delta\sigma]\) s.t.
\begin{equation*}
\forall a\in\bC(\sigma),a'\in\bD(\gamma\sigma).a'\Vdash_\sigma^\gamma a\Rightarrow t(a')\Vdash_\sigma^\delta a
\end{equation*}
We write \(\gamma\sim\delta\) if both \(\gamma\preceq\delta\) and \(\delta\preceq\gamma\)
\end{definition}

\begin{definition}[]
Models \(\bC,\bD\) are \textbf{equivalent} (\(\bC\simeq\bD\)) if there exist simulations \(\gamma:\bC\lrarhd\bD\)
and \(\delta:\bD\lrarhd\bC\) s.t. \(\delta\circ\gamma\sim\id_{\bC}\) and \(\gamma\circ\delta\sim\id_{\bD}\)
\end{definition}

A model is \textbf{essentially untyped} if it is equivalent to a model over the singleton type world \(\sfO\)

\begin{exercise}
Show that a model \(\bC\) is essentially untyped iff it contains a \textbf{universal type}: that is, a
datatype \(U\) s.t. for each \(A\in\abs{\bC}\) there exists operations \(e\in\bC[A,U]\), \(r\in\bC[U,A]\)
with \(r(e(a))=a\) for all \(a\in A\)
\end{exercise}

\begin{proof}
\(\Leftarrow\): Let \(\sfO=\{U\}\). For each \(f\in\bC[A,B]\), \(\bD\) contains \(\barf\in\bD[U,U]\) s.t. \(\barf()\)
Let
\begin{equation*}
\bD[U,U]=\{\barf:e[A]\to e[B]:f\in\bC[A,B]\}
\end{equation*}
where \(\barf(e(a))=e(f(a))\)

each \(A\in\abs{\bC}\), let \(\gamma(A)=U\) and define \(a'\Vdash_A^\gamma a\) iff \(a'=e(a)\)
\end{proof}

\begin{definition}[]
Suppose \(\bC,\bD\) are lax models with weak products and weak terminals \((I,i)\), \((J,j)\)
respectively. A simulation \(\gamma:\bC\lrarhd\bD\) is \textbf{cartesian} if
\begin{enumerate}
\item for each \(\sigma,\tau\in\abs{\bC}\) there exists \(t\in\bD[\gamma\sigma\bowtie\gamma\tau,\gamma(\sigma\bowtie\tau)]\) s.t.
\begin{gather*}
\pi_{\gamma\sigma}(d)\Vdash_\sigma^\gamma a\wedge\pi_{\gamma\tau}(d)\Vdash_\tau^\gamma b\Rightarrow\\
\exists c\in\bC(\sigma\bowtie\tau).\pi_\sigma(c)=a\wedge\pi_\tau(c)=b\wedge t(d)\Vdash_{\sigma\bowtie\tau}^\gamma c
\end{gather*}
\item there exists \(u\in\bD[J,\gamma I]\) s.t. \(u(j)\Vdash_I^\gamma i\)
\end{enumerate}
\end{definition}


\begin{definition}[]
Let \(\bA\) and \(\bB\) be lax relative TPCAs over the type worlds \(\sfT,\sfU\) respectively. An
\textbf{applicative simulation} \(\gamma:\bA\lrarhd\bB\) consists of
\begin{itemize}
\item a mapping \(\sigma\mapsto\gamma\sigma:\sfT\to\sfU\)
\item for each \(\sigma\in\sfT\), a relation \(\Vdash_\sigma^\gamma\subseteq\bB^\circ(\gamma\sigma)\times\bA^\circ(\sigma)\)
\end{itemize}
satisfying the following
\begin{enumerate}
\item For all \(a\in\bA^\circ(\sigma)\) there exists \(b\in\bB^\circ(\gamma\sigma)\) with \(b\Vdash_\sigma^\gamma a\)
\item For all \(a\in\bA^\sharp(\sigma)\) there exists \(b\in\bB^\sharp(\gamma\sigma)\) with \(b\Vdash_\sigma^\gamma a\)
\item 'Application in \(\bA\) is effective in \(\bB\)': that is, for each \(\sigma,\tau\in\sfT\), there exists
some \(r\in\bB^\sharp(\gamma(\sigma\to\tau)\to\gamma\sigma\to\gamma\tau)\), called a \textbf{realizer for \(\gamma\) at \(\sigma\),\(\tau\)}, s.t. for
all \(f\in\bA^\circ(\sigma\to\tau)\), \(f'\in\bB^\circ(\gamma(\sigma\to\tau))\), \(a\in\bA^\circ(\sigma)\) and \(a'\in\bB^\circ(\gamma\sigma)\) we have
\begin{equation*}
f'\Vdash_{\sigma\to\tau}f\wedge a'\Vdash_\sigma a\wedge f\cdot a\downarrow\Rightarrow r\cdot f'\cdot a'\Vdash_\tau f\cdot a
\end{equation*}
\end{enumerate}
\end{definition}

\begin{theorem}[]
Suppose \(\bC\) and \(\bD\) are (lax) weakly cartesian closed models, and suppose \(\bA\) and \(\bB\)
are the corresponding (lax) relative TPCAs with pairing via the correspondence of Theorem
\ref{3.1.18}. Then cartesian simulations \(\bC\lrarhd\bD\) correspond precisely to applicative
simulations \(\bA\lrarhd\bB\)
\end{theorem}

\begin{proof}
Suppose first that \(\gamma:\bC\lrarhd\bD\) is a cartesian simulation
\begin{enumerate}
\item Definition
\item Suppose \(a\in\bA^\sharp(\sigma)\) where \(\bA^\circ(\sigma)=A\). Then we may
find \(g\in\bC[I,A]\) with \(g(i)=a\), where \((I,i)\) is a weak terminal in \(\bC\).
Take \(g'\in\bD[\gamma I,\gamma A]\) tracking \(g\), and compose it with \(u\in\bD[J,\gamma I]\), we
obtain \(g''\in\bD[J,\gamma A]\). Then \(g''(j)\in\bB^\sharp(\gamma\sigma)\), and it is easy to see
that \(g''(j)\Vdash_\sigma^\gamma a\)
\item Let \(\sigma,\tau\) be any types; then by the definition of weakly cartesian closedness, we
have \(app_{\sigma\tau}\in\bC[(\sigma\to\tau)\times\sigma,\tau]\) tracked by some \(app_{\sigma\tau}'\in\bD[\gamma((\sigma\to\tau)\times\sigma),\gamma\tau]\). By definition
of cartesian simulation, we have \(t\in\bD[\gamma(\sigma\to\tau)\times\gamma\sigma,\gamma((\sigma\to\tau)\times\sigma)]\), we have an operation
\(\bD[\gamma(\sigma\to\tau)\times\gamma\sigma,\gamma\tau]\), and hence an operation \(\bD[\gamma(\sigma\to\tau),\gamma\sigma\to\gamma\tau]\), and then
an operation \(\bD[J, \gamma(\sigma\to\tau)\to\gamma\sigma\to\gamma\tau]\), and hence
realizer \(r\in\bB^\sharp(\gamma(\sigma\to\tau)\to\gamma\sigma\to\gamma\tau)\) with the required properties:
for all \(f\in\bA^\circ(\sigma\to\tau)\), \(f'\in\bB^\circ(\gamma(\sigma\to\tau))\), \(a\in\bA^\circ(\sigma)\), \(a'\in\bB^\circ(\gamma\sigma)\),
and \(f'\Vdash_{\sigma\to\tau}f\), \(a'\Vdash_\sigma a\), \(f\cdot a\downarrow\), we
have \(t(f',a')\Vdash_{(\sigma\to\tau)\times\sigma}(f,a)\).
Then \(app'_{\sigma\tau}(t(f',a'))\Vdash_{\tau}^\gamma app_{\sigma\tau}(f,a)\)
\end{enumerate}


Conversely, suppose \(\gamma:\bA\lrarhd\bB\) is an applicative simulation. To see that \(\gamma\) is a
simulation \(\bC\lrarhd\bD\), it suffices to show that every operation in \(\bC\) is tracked by one
in \(\bD\). But given \(f\in\bC[\sigma,\tau]\), we may find a corresponding element \(a\in\bA^\sharp(\sigma\to\tau)\), whence
some \(a'\in\bB^\sharp(\gamma(\sigma\to\tau))\) with \(a'\Vdash^\gamma_{\sigma\to\tau}a\); by using a realizer \(r\in\bB^\sharp\) for \(\gamma\) at \(\sigma\),\(\tau\),
we have an element \(a''\in\bB^\sharp(\gamma\sigma\to\gamma\tau)\)  and so a corresponding operation \(f'\in\bD[\gamma\sigma,\gamma\tau]\).

It remains to show that \(\gamma\) is cartesian. For any types \(\sigma\),\(\tau\), we have by assumption an element
\(pair_{\sigma\tau}\in\bA^\sharp(\sigma\to\tau\to\sigma\times\tau)\), yielding some \(p\in\bC[\sigma,\tau\to\sigma\times\tau]\). Since \(\gamma\) is a simulation, this is
tracked by some \(p'\in\bD[\gamma\sigma,\gamma(\tau\to\sigma\times\tau)]\). From the weak product structure in \(\bD\) we may thence
obtain an operation
\begin{equation*}
p''\in\bD[\gamma\sigma\times\gamma\tau,\gamma(\tau\to\sigma\times\tau)\times\gamma\tau]
\end{equation*}
and together with a realizer for \(\gamma\) at \(\tau\) and \(\sigma\times\tau\), this yields an
operation \(t\in\bD[\gamma\sigma\times\gamma\tau,\gamma(\sigma\times\tau)]\) with the required properties.

\(i\in\bA^\sharp(I)\), hence there is \(b\in\bB^\sharp(\gamma I)\) with \(b\Vdash_I^\gamma i\). But \(b=u(j)\) for
some \(u\in\bD[J,\gamma I]\)
\end{proof}

The notion of a transformation between simulations carries across immediately to the relative
TPCA setting: an applicative simulation \(\gamma:\bA\lrarhd\bB\) is transformable to \(\delta\) if for each type
\(\sigma\) there exists \(t\in\bB^\sharp(\gamma\sigma\to\delta\sigma)\) s.t. \(a'\Vdash_\sigma^\gamma a\) implies \(t\cdot a'\Vdash_\sigma^\delta a\)

\begin{definition}[]
Suppose \(\bA\) and \(\bB\) are (lax) relative TPCAs over \(\sfT\) and \(\sfU\) respectively, and
suppose \(\gamma:\bA\lrarhd\bB\) is an applicative simulation
\begin{enumerate}
\item \(\gamma\) is \textbf{discrete} if \(b\Vdash^\gamma a\) and \(b\Vdash^\gamma a'\) imply \(a=a'\)
\item \(\gamma\) is \textbf{single-valued} if for all \(a\in\bA\) there is exactly one \(b\in\bB\) with \(b\Vdash^\gamma a\). \(\gamma\) is
\textbf{projective} if \(\gamma\sim\gamma'\) for some single-valued \(\gamma'\)
\item If \(\bA\) and \(\bB\) have booleans \(\TT_{\bA}\), \(\FF_{\bA}\) and \(\TT_{\bB}\), \(\FF_{\bB}\)
respectively, \(\gamma\) \textbf{respects booleans} if there exists \(q\in\bB^\sharp\) s.t.
\begin{equation*}
b\Vdash^\gamma\TT_{\bA}\Rightarrow q\cdot b=\TT_{\bB},\quad b\Vdash^\gamma\FF_{\bA}\Rightarrow q\cdot b=\FF_{\bB}
\end{equation*}
\item If \(\bA\) and \(\bB\) have numerals \(\hatn\) and \(\tiln\) respectively, \(\gamma\) \textbf{respects numerals} if
there exists \(q\in\bB^\sharp\) s.t. for all \(n\in\N\)
\begin{equation*}
b\Vdash^\gamma\hatn\Rightarrow q\cdot b=\tiln
\end{equation*}
\item If \(\sfT=\sfU\), we say \(\gamma\) is \textbf{type-respecting} if \(\gamma\) is the identity on types, and moreover:
\begin{itemize}
\item \(\Vdash_\sigma^\gamma=\id_{\sfT[\sigma]}\) whenever \(\sfT\) fixes the interpretation of \(\sigma\)
\item Application is itself a realizer for \(\gamma\) at each \(\sigma,\tau\): that is,
\begin{equation*}
f'\Vdash_{\sigma\to\tau}^\gamma f\wedge a'\Vdash_\sigma^\gamma a\wedge f\cdot a\downarrow\Rightarrow f'\cdot a'\Vdash_\tau^\gamma f\cdot a
\end{equation*}
\item If \(\sfT\) has product structure, then \(\bA,\bB\) have pairing and
\begin{align*}
a'\Vdash_\sigma^\gamma a\wedge b'\Vdash_\tau^\gamma b&\Rightarrow pair\cdot a'\cdot b'\Vdash_{\sigma\times\tau}^\gamma pair\cdot a\cdot b\\
d'\Vdash_{\sigma\times\tau}^\gamma d&\Rightarrow fst\cdot d'\Vdash_\sigma^\gamma fst\cdot d\wedge snd\cdot d'\Vdash_\sigma^\gamma snd\cdot d
\end{align*}
\end{itemize}
\end{enumerate}
\end{definition}

\begin{remark}
Show that if \(\gamma\) respects numerals then \(\gamma\) respects booleans
\end{remark}

\subsubsection{Simulations and Assemblies}
\label{sec:orgebc39b6}
A simulation \(\gamma:\bC\lrarhd\bD\) naturally induces a functor \(\gamma_*:\Asm(\bC)\to\Asm(\bD)\) , capturing the
evident idea that any datatypes implementable within \(\bC\) must also be implementable in \(\bD\)
\begin{itemize}
\item On objects \(X\), define \(\gamma_*(X)\) by \(\abs{\gamma_*(X)}=\abs{X}\), \(\rho_{\gamma_*(X)}=\gamma\rho_X\)
and \(b\Vdash_{\gamma_*(X)}x\) iff \(\exists a\in\bC(\rho_X)\), \(a\Vdash_Xx\wedge b\Vdash_{\rho_X}^\gamma a\)
\item On morphisms \(f:X\to Y\), define \(\gamma_*(f)=f\). Note that if \(r\in\bC[\rho_X,\rho_Y]\) tracks \(f\) as a
morphism in \(\Asm(\bC)\), and \(r'\in\bD[\gamma\rho_X,\gamma\rho_Y]\) tracks \(r\) w.r.t. \(\gamma\), then \(r'\)
tracks \(f\) as a morphism in \(\Asm(\bC)\)
\end{itemize}


Moreover, a transformation \(\xi:\gamma\to\delta\) yields a natural transformation \(\xi_*:\gamma_*\to\delta_*\): just
take \(\xi_{*X}=\id_{\abs{X}}:\gamma_*(X)\to\delta_*​ (X)\), and note that if \(t\) witnesses \(\gamma\preceq\delta\) at \(X\),
then \(t\) tracks \(\xi_{*​X}\).

\begin{definition}[]
\begin{enumerate}
\item \(\gamma\) respects numerals iff \(\gamma_*\) preserves the natural number object
\end{enumerate}
\end{definition}

\subsection{Examples of Simulations and Transformations}
\label{sec:org55dd87e}
\begin{examplle}[]
Suppose \(\bC\) is any (lax) computability model with weak products, and consider the following
variation on the 'product completion' construction described in the proof of Theorem \ref{3.1.15}.
Let \(\bC^\times\) be the computability model whose datatypes are sets \(A_0\times\dots\times A_{m-1}\)
where \(A_i\in\abs{\bC}\), and whose operations \(f\in\bC^\times[A_0\times\dots\times A_{m-1},B_0\times\dots\times B_{n-1}]\) are those
partial functions represented by some operation in \(\bC[A_0\bowtie\dots\bowtie A_{m-1},B_0\bowtie\dots\bowtie B_{n-1}]\). Clearly
the inclusion \(\bC\hookrightarrow\bC^\times\) and \(\bC^\times\to\bC\) sending \(A_0\times\dots A_{m-1}\) to \(A_0\bowtie\dots\bowtie A_{m-1}\) are
simulations. Moreover, they constitute an equivalence \(\bC\simeq\bC^\times\). This shows that every strict
(lax) computability model with weak products is equivalent to one with standard products
\end{examplle}

\begin{proposition}[]
For any partial computable \(f:\N^r\rhu\N\) there is an applicative
expression \(e_f:\ttN^{(r)}\to\ttN\) (involving constants 0, suc, \(rec_{\ttN}\), min) s.t. in any
model \(\bA\) with numerals and minimization we have \(\llb{e_f}_v\in\bA^\sharp\) (with the obvious
valuation \(v\)) and
\begin{equation*}
\forall n_0,\dots,n_{r-1},m. f(n_0,\dots,n_{r-1})=m\Rightarrow\llb{e_f}_v\cdot\hatn_0\cdot\dots\cdot\hatn_{r-1}=\hatm
\end{equation*}
\end{proposition}

\begin{examplle}[]
Let \(\bA\) be any untyped (lax relative) PCA, or more generally any model with
numerals \(\bar{0},\bar{1},\dots\) and minimization. We may then define a single-valued applicative
simulation \(\kappa:K_1\lrarhd\bA\) by taking \(a\Vdash^\kappa n\) iff \(a=\hat{n}\). Condition 3 is
satisfied because the application operation \(\cdot:\N\times\N\rhu\N\) of \(K_1\) is representable by an
element of \(\bA^\sharp\)

Model with single datatype \(\N\) and whose operations \(\N\rightharpoonup\N\) are precisely the
Turing-computable partial functions. \(K_1\)is weakly cartesian closed.

For any partial computable \(f:\N^r\rhu\N\) there is an applicative
expression \(e_f:\ttN^{(r)}\to\ttN\) (involving constants 0, suc, \(rec_{\ttN}\), min) s.t. in any
model \(\bA\) with numerals and minimization we have \(\llb{e_f}_v\in\bA^\sharp\) (with the obvious
valuation \(v\)) and
\begin{equation*}
\forall n_0,\dots,n_{r-1},m. f(n_0,\dots,n_{r-1})=m\Rightarrow\llb{e_f}_v\cdot\hatn_0\cdot\dots\cdot\hatn_{r-1}=\hatm
\end{equation*}

In particular models, many choices of numerals may be available. For instance, if \(\bA=\bA^\sharp=\Lambda/\sim\),
then besides the \emph{Curry numerals}, we also have the \textbf{Church numerals} \(\tiln=\lambda f.\lambda x.f^nx\)
\end{examplle}

\begin{examplle}[]
The model \(\Lambda^0/=\beta\) consisting of closed \(\lambda\)-terms modulo \(\beta\)-equivalence. Let \(\ceil{-}\) be
any effective \textbf{Gödel numbering} of \(\lambda\)-terms as natural numbers, and define \(\gamma:\Lambda^0/=_\beta\lrarhd K_1\) by
\begin{equation*}
m\Vdash^\gamma[M]\quad\text{ iff }\quad m=\ceil{M'}\text{ for some }M'=_\beta M
\end{equation*}
Justaposition of \(\lambda\)-terms is tracked by a computable function \((\ceil{M},\ceil{N})\mapsto\ceil{MN}\)
at the level of Gödel numbers. Thus \(\gamma\) is applicative

We have transformations \(\id_{K_1}\preceq\gamma\circ\kappa\) and \(\gamma\circ\kappa\preceq\id_{K_1}\); these correspond to the
observation that the 'encoding' and 'decoding' mappings \(n\mapsto\ceil{\hatn}\)
and \(\ceil{\hatn}\mapsto n\)
are computable. Also there is a term \(P\in\Lambda^0\) s.t. \(P(\what{\ceil{M}})=_\beta M\) for any \(M\in\Lambda^0\) by
Kleene's enumeration theorem, therefore \(\kappa\circ\gamma\preceq\id_{\Lambda^0/=_\beta}\)

However we don't have \(\id_{\Lambda^0/=_\beta}\preceq\kappa\circ\gamma\)

It can be easily shown that \(K_1\) is not equivalent to \(\Lambda^0/=_\beta\). Let us say a relative
PCA \(\bA\) has \textbf{decidable equality} if there is an element \(q\in\bA^\sharp\) s.t.
\begin{equation*}
q\cdot x\cdot y=
\begin{cases}
\TT&x=y\\
\FF&\text{otherwise}
\end{cases}
\end{equation*}
Clearly \(K_1\) has decidable equality, and it is easy t see that if \(\bA\simeq\bB\) then \(\bA\) has
decidable equality iff \(\bB\) does. However, if a total relative PCA \(\bA\) were to contain such
an element, we could define \(v=Y(q\;\FF)\) so that \(v=q\;\FF\;v\) which would yield a
contradiction

Here \(Y\;f=f(Y\; f)\)
\end{examplle}



\begin{examplle}[]
We may translate System T terms \(M\) to PCF terms \(M^\theta\) simply by replacing all
constants \(rec_\sigma\) by suitable implementations of these recursors in PCF

In any untyped model, let \(Z\) be a guarded recursion operator,
define
\begin{equation*}
R=\lambda rxfm.\IF(iszero\;m)(kx)(\lambda y.f(pre\;m))(rxf(pre\; m)\hat{0})
\end{equation*}
and take \(rec=\lambda xfm.(ZR)xfmi\).

Such a translation induces a type-respecting applicative
simulation \(\theta:\tT^0/=_{op}\lrarhd \PCF^0/=_{op}\). This simulation is single-valued
as \(M=_{op}M'\Rightarrow M^\theta=_{op}M'^\theta\). It is easy to see that \(\theta\) respects numerals
\end{examplle}

\begin{examplle}[]
We may also translate PCF into the untyped \(\lambda\)-calculus. One such translation \(\phi\) may be defined on
constants as follows. We write \(\la M,N\ra\) for \(pair\;M\;N\) and
define \(\hat{0}=\la\lambda xy.x,\lambda xy.x\) and \(\what{n+1}=\la\lambda xy.y,\hatn\ra\)
\begin{align*}
\hatn^\phi&=\hatn\\
suc^\phi&=\lambda z.\la\lambda xy.y,z\ra\\
pre^\phi&=\lambda z.(fst\;z)\hat{0}(snd\;z)\\
ifzero^\phi&=fst\\
Y_\sigma^\phi&=(\lambda xy.y(xxy))(\lambda xy.y(xxy))
\end{align*}
This induces an applicative simulation \(\phi:\PCF^0/=_{op}\lrarhd\Lambda^0/=_\beta\) respecting numerals.
Moreover if \(M=_{op}M'\) then \(M^\phi=_\beta M'^\phi\)
\end{examplle}

\begin{examplle}[Interpretations of System \(T\)]
Let \(\bA\) be any total relative TPCA with numerals \(\hat{0},\hat{1},\dots\) of type \(\ttN\), and
associated operations \(suc\), \(rec_\sigma\). To any closed term \(M:\sigma\) in Gödel's System \(T\), we
may associate an element \(\llb{M}\in\bA^\sharp(\sigma)\) as follows: replace each occurrence of \(\lambda\) by \(\lambda^*\)
to obtain a meta-expression \(M^*\), then expand \(M^*\) to an applicative expression \(M^\dag\)
and evaluate it in \(\bA\), interpreting the constants \(\hat{0}\), \(suc\), \(rec_\sigma\) in the
obvious way

Clearly \(\llb{MN}=\llb{M}\cdot\llb{N}\) and if \(M\rightsquigarrow M'\) then \(\llb{M}=\llb{M'}\). We therefore
obtain a type-respecting simulation \(\llb{-}:T^0/=_{op}\lrarhd\bA\)

Now suppose \(\gamma:\bA\lrarhd\bB\) is a type- and numeral-respecting applicative simulation between
total models with numerals. By the above, we have applicative simulations
\(\llb{-}_{\bA}:\tT^0/=_{op}\lrarhd\bA\) and \(\llb{-}_{\bB}:\tT^0/=_{op}\lrarhd\bB\), but it nee not in
general be the case that \(\gamma\circ\llb{-}_{\bA}\sim\llb{-}_{\bB}\). This shows that a model \(\bB\) may in
general admit several inequivalent applicative simulations of \(\tT^0/=_{op}\)
\end{examplle}

\begin{examplle}[Interpretations of PCF]
Let \(\bA\) be any strict relative TPCA with numerals and general recursion. Since \(\bA\) contains
elements playing playing the role of the PCF constants \(\hatn\), \(suc\), \(pre\), \(ifzero\)
and \(Y_\sigma\), we may translate closed PCF terms \(M:\sigma\) into expressions \(\tilM:\sigma\) over \(A\)
by simply replacing \(\lambda\) with \(\lambda^*\) and expanding. Note that \(M=_{op}M'\)
implies \(\llb{\tilM}\simeq\llb{\tilM'}\) in \(\bA\). This does not itself give us an applicative
simulation \(\PCF^0/=_{op}\lrarhd\bA\) since \(\llb{\tilM}\) may sometimes be undefined. However,
we may obtain such a simulation \(\theta_{\bA}\) via a suspension trick: let \(\theta_{\bA}\sigma=\N\to\sigma\) for each \(\sigma\),
and take \(a\Vdash_\sigma^{\theta_{\bA}}[M]\) iff \(a\cdot\hat{0}\simeq\llb{\tilM}\). (For example, we
have \(\llb{\lambda^*u.\tilM}\Vdash[M]\) for any \(M\)) \(\theta_{\bA}\) is an applicative morphism, being
realized at each \(\sigma\), \(\tau\) by \(\lambda^*fxu.(fu)(xu)\)
\end{examplle}
\subsubsection{Effective and Continuous Models}
\label{sec:org609a5ad}
It is natural to want to identify a class of computability models that are genuinely
‘effective’, in the sense that their data can be represented in some reasonable finitary way
and their operations are ultimately implementable on a Turing machine.

\begin{definition}[]
An \textbf{effective (relative) TPCA} is a (relative) TPCA \(\bA\) with numerals equipped with a
numeral-respecting applicative simulation \(\gamma:\bA\lrarhd K_1\)
\end{definition}

Syntactic models in general, such as \(\Lambda^0/\sim\) is effective TPCAs: The simulation \(\gamma\) is given by a
Gödel numbering of syntactic terms.

We may define \(\gamma:K_2^{\eff}\lrarhd K_1\) by setting \(m\Vdash^\gamma f\) iff \(\forall n.m\cdot n=f(n)\)

\begin{examplle}[]
The following shows what can happen if the numeral-respecting condition in definition is
omitted. Let \(f:\N\to\N\) be some fixed non-computable function. The definition of \(K_1\) may be
relativized to \(f\) as follows: let \(T_0',T_1',\dots\) be a sensibly chosen enumeration of all
Turing machines equipped with an \emph{oracle} for \(f\), so that a computation may ask for the value
of some \(f(m)\) as a single step. Now take \(e\cdot^fn\) to be the result of applying \(T_e'\) to the
input \(n\), and let \(K_1^f=(\N,\cdot^f)\). The proof that \(K_1\) is a PCA readily carries over
to \(K_1^f\)

There is an applicative simulation \(\gamma:K_1^f\lrarhd K_1\): we may take \(m\Vdash^\gamma n\) iff \(m\cdot^f0=n\)
\end{examplle}

\begin{definition}[]
A \textbf{continuous (relative) TPCA} is a (relative) TPCA \(\bA\) with numerals equipped with a
numeral-respecting applicative simulation \(\gamma:\bA\lrarhd K_2\)
\end{definition}

\begin{definition}[]
A continuous TPCA \((\bA,\gamma)\) is \textbf{full continuous} if the following hold:
\begin{enumerate}
\item For every \(f:\N\to\N\) there is some \(a\in\bA(\ttN\to\ttN)\) that represents \(f\)
\item Moreover, a realizer for some such \(a\) may be computed from \(f\) within \(K_2\) - that is,
there exists \(h\in K_2\) s.t. for all \(f\in\N^{\N}\) we have
\end{enumerate}
\end{definition}
\end{document}
