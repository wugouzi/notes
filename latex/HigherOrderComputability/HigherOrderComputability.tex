% Created 2022-03-01 Tue 11:04
% Intended LaTeX compiler: pdflatex
\documentclass[11pt]{article}
\usepackage[utf8]{inputenc}
\usepackage[T1]{fontenc}
\usepackage{graphicx}
\usepackage{longtable}
\usepackage{wrapfig}
\usepackage{rotating}
\usepackage[normalem]{ulem}
\usepackage{amsmath}
\usepackage{amssymb}
\usepackage{capt-of}
\usepackage{hyperref}
\graphicspath{{../../books/}}
% TIPS
% \substack{a\\b} for multiple lines text





% pdfplots will load xolor automatically without option
\usepackage[dvipsnames]{xcolor}

\usepackage{forest}
% two-line text in node by [two \\ lines]
% \begin{forest} qtree, [..] \end{forest}
\forestset{
  qtree/.style={
    baseline,
    for tree={
      parent anchor=south,
      child anchor=north,
      align=center,
      inner sep=1pt,
    }}}
%\usepackage{flexisym}
% load order of mathtools and mathabx, otherwise conflict overbrace

\usepackage{mathtools}
%\usepackage{fourier}
\usepackage{pgfplots}
\usepackage{amsthm, mathabx,  amsmath, commath}
\usepackage{amsfonts}

\usepackage{empheq}
\usepackage{tikz}
\usetikzlibrary{arrows.meta}
\usepackage[most]{tcolorbox}

\newtheorem{theorem}{Theorem}[section]
\newtheorem{definition}{Definition}[section]
\newtheorem{corollary}{Corollary}[section]
\newtheorem{example}{Example}[section]
\newtheorem{lemma}{Lemma}[section]
\newtheorem{proposition}{Proposition}[section]

\newcommand{\bl}[1] {\boldsymbol{#1}}
\newcommand{\Wt}[1] {\stackrel{\sim}{\smash{#1}\rule{0pt}{1.1ex}}}
\newcommand{\wt}[1] {\widetilde{#1}}


%For boxed texts in align, use Aboxed{}
%otherwise use boxed{}

\DeclareMathSymbol{\widehatsym}{\mathord}{largesymbols}{"62}
\newcommand\lowerwidehatsym{%
  \text{\smash{\raisebox{-1.3ex}{%
    $\widehatsym$}}}}
\newcommand\fixwidehat[1]{%
  \mathchoice
    {\accentset{\displaystyle\lowerwidehatsym}{#1}}
    {\accentset{\textstyle\lowerwidehatsym}{#1}}
    {\accentset{\scriptstyle\lowerwidehatsym}{#1}}
    {\accentset{\scriptscriptstyle\lowerwidehatsym}{#1}}
}

\usepackage{graphicx}
    
% text on arrow for xRightarrow
\makeatletter
%\newcommand{\xRightarrow}[2][]{\ext@arrow 0359\Rightarrowfill@{#1}{#2}}
\makeatother


\def \bx {\boldsymbol{x}}
\def \ba {\boldsymbol{a}}
\def \bI {\boldsymbol{I}}
\def \bt {\boldsymbol{t}}
\def \bb {\boldsymbol{b}}
\def \bA {\boldsymbol{A}}
\def \bX {\boldsymbol{X}}
\def \bu {\boldsymbol{u}}
\def \bS {\boldsymbol{S}}
\def \bZ {\boldsymbol{Z}}
\def \bz {\boldsymbol{z}}
\def \by {\boldsymbol{y}}
\def \bw {\boldsymbol{w}}
\def \bT {\boldsymbol{T}}
\def \bS {\boldsymbol{S}}
\def \bm {\boldsymbol{m}}
\def \bW {\boldsymbol{W}}
\def \bY {\boldsymbol{Y}}
\def \bH {\boldsymbol{H}}
\def \blambda {\boldsymbol{\lambda}}
\def \bPhi {\boldsymbol{\Phi}}
\def \btheta {\boldsymbol{\theta}}
\def \bmu {\boldsymbol{\mu}}
\def \bphi {\boldsymbol{\phi}}
\def \bSigma {\boldsymbol{\Sigma}}
\def \lb {\left\{}
\def \rb {\right\}}
\def \caln {\mathcal{N}}
\def \dissum {\displaystyle\Sigma}
\def \dispro {\displaystyle\prod}
\def \E {\mathbb{E}}
\def \Q {\mathbb{Q}}
\def \V {\mathbb{V}}
\def \R {\mathbb{R}}
\def \calq {\mathcal{Q}}
\def \calg {\mathcal{G}}
\def \caln {\mathcal{N}}
\def \calr {\mathcal{R}}
\def \calm {\mathcal{M}}
\def \calc {\mathcal{C}}
\def \bcup {\bigcup}

\makeindex
\author{John Longley \& Dag Normann}
\date{\today}
\title{Higher Order Computability}
\hypersetup{
 pdfauthor={John Longley \& Dag Normann},
 pdftitle={Higher Order Computability},
 pdfkeywords={},
 pdfsubject={},
 pdfcreator={Emacs 28.0.90 (Org mode 9.6)}, 
 pdflang={English}}
\begin{document}

\maketitle
\tableofcontents


\section{Theory of Computability Models}
\label{sec:org8b4882b}
\begin{itemize}
\item \(e\downarrow\) 'the value of \(e\) is defined'
\item \(e\uparrow\) 'the value of \(e\) is undefined'
\item \(e=e'\) 'the values of both \(e\) and \(e'\) are defined and they are equal'
\item \(e\simeq e'\) 'if either \(e\) or \(e'\) is defined then so is the other and they are equal'
\item \(e\succeq e'\) 'if \(e'\) is defined then so it \(e\) and they are equal'
\end{itemize}


if \(e\) is a mathematical expression possibly involving the variable \(x\), we write \(\Lambda x.e\)
to mean the ordinary (possibly partial) function \(f\) defined by \(f(x)\simeq e\)

Finite sequences of length \(n\) starts from index 0.
\subsection{Higher-Order Computability Models}
\label{sec:org1f2d573}
\subsubsection{Computability Models}
\label{sec:org5c780b8}
\begin{definition}[]
A \textbf{computability model} \(\bC\)  over a set \(\sfT\) of \textbf{type names} consists of
\begin{itemize}
\item an indexed family \(\abs{\bC}=\{\bC(\tau)\mid\tau\in\sfT\}\) of sets, called the \textbf{datatypes} of \(\bC\)
\item for each \(\sigma,\tau\in\sfT\), a set \(\bC[\sigma,\tau]\) of partial functions \(f:\bC(\sigma)\to\bC(\tau)\), called the
\textbf{operations} of \(\bC\)
\end{itemize}
\end{definition}

We shall use uppercase letters \(A,B,C,\dots\) to denote \textbf{occurrences} of sets within \(\abs{\bC}\):
that is, sets \(\bC(\tau)\) implicitly tagged with a type name \(\tau\). We shall write \(\bC[A,B]\)
for \(\bC[\sigma,\tau]\) if \(A=\bC(\sigma)\) and \(B=\bC(\tau)\)

In typical cases of interest, the operations of \(\bC\) will be ‘computable’ maps of some kind between datatypes
\end{document}
