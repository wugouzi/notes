% Created 2021-04-09 五 11:01
% Intended LaTeX compiler: pdflatex
\documentclass[11pt]{article}
\usepackage[utf8]{inputenc}
\usepackage[T1]{fontenc}
\usepackage{graphicx}
\usepackage{grffile}
\usepackage{longtable}
\usepackage{wrapfig}
\usepackage{rotating}
\usepackage[normalem]{ulem}
\usepackage{amsmath}
\usepackage{textcomp}
\usepackage{amssymb}
\usepackage{capt-of}
\usepackage{hyperref}
% TIPS
% \substack{a\\b} for multiple lines text





% pdfplots will load xolor automatically without option
\usepackage[dvipsnames]{xcolor}

\usepackage{forest}
% two-line text in node by [two \\ lines]
% \begin{forest} qtree, [..] \end{forest}
\forestset{
  qtree/.style={
    baseline,
    for tree={
      parent anchor=south,
      child anchor=north,
      align=center,
      inner sep=1pt,
    }}}
%\usepackage{flexisym}
% load order of mathtools and mathabx, otherwise conflict overbrace

\usepackage{mathtools}
%\usepackage{fourier}
\usepackage{pgfplots}
\usepackage{amsthm, mathabx,  amsmath, commath}
\usepackage{amsfonts}

\usepackage{empheq}
\usepackage{tikz}
\usetikzlibrary{arrows.meta}
\usepackage[most]{tcolorbox}

\newtheorem{theorem}{Theorem}[section]
\newtheorem{definition}{Definition}[section]
\newtheorem{corollary}{Corollary}[section]
\newtheorem{example}{Example}[section]
\newtheorem{lemma}{Lemma}[section]
\newtheorem{proposition}{Proposition}[section]

\newcommand{\bl}[1] {\boldsymbol{#1}}
\newcommand{\Wt}[1] {\stackrel{\sim}{\smash{#1}\rule{0pt}{1.1ex}}}
\newcommand{\wt}[1] {\widetilde{#1}}


%For boxed texts in align, use Aboxed{}
%otherwise use boxed{}

\DeclareMathSymbol{\widehatsym}{\mathord}{largesymbols}{"62}
\newcommand\lowerwidehatsym{%
  \text{\smash{\raisebox{-1.3ex}{%
    $\widehatsym$}}}}
\newcommand\fixwidehat[1]{%
  \mathchoice
    {\accentset{\displaystyle\lowerwidehatsym}{#1}}
    {\accentset{\textstyle\lowerwidehatsym}{#1}}
    {\accentset{\scriptstyle\lowerwidehatsym}{#1}}
    {\accentset{\scriptscriptstyle\lowerwidehatsym}{#1}}
}

\usepackage{graphicx}
    
% text on arrow for xRightarrow
\makeatletter
%\newcommand{\xRightarrow}[2][]{\ext@arrow 0359\Rightarrowfill@{#1}{#2}}
\makeatother


\def \bx {\boldsymbol{x}}
\def \ba {\boldsymbol{a}}
\def \bI {\boldsymbol{I}}
\def \bt {\boldsymbol{t}}
\def \bb {\boldsymbol{b}}
\def \bA {\boldsymbol{A}}
\def \bX {\boldsymbol{X}}
\def \bu {\boldsymbol{u}}
\def \bS {\boldsymbol{S}}
\def \bZ {\boldsymbol{Z}}
\def \bz {\boldsymbol{z}}
\def \by {\boldsymbol{y}}
\def \bw {\boldsymbol{w}}
\def \bT {\boldsymbol{T}}
\def \bS {\boldsymbol{S}}
\def \bm {\boldsymbol{m}}
\def \bW {\boldsymbol{W}}
\def \bY {\boldsymbol{Y}}
\def \bH {\boldsymbol{H}}
\def \blambda {\boldsymbol{\lambda}}
\def \bPhi {\boldsymbol{\Phi}}
\def \btheta {\boldsymbol{\theta}}
\def \bmu {\boldsymbol{\mu}}
\def \bphi {\boldsymbol{\phi}}
\def \bSigma {\boldsymbol{\Sigma}}
\def \lb {\left\{}
\def \rb {\right\}}
\def \caln {\mathcal{N}}
\def \dissum {\displaystyle\Sigma}
\def \dispro {\displaystyle\prod}
\def \E {\mathbb{E}}
\def \Q {\mathbb{Q}}
\def \V {\mathbb{V}}
\def \R {\mathbb{R}}
\def \calq {\mathcal{Q}}
\def \calg {\mathcal{G}}
\def \caln {\mathcal{N}}
\def \calr {\mathcal{R}}
\def \calm {\mathcal{M}}
\def \calc {\mathcal{C}}
\def \bcup {\bigcup}

\author{Steve Awodey}
\date{\today}
\title{Category Theory}
\hypersetup{
 pdfauthor={Steve Awodey},
 pdftitle={Category Theory},
 pdfkeywords={},
 pdfsubject={},
 pdfcreator={Emacs 27.1 (Org mode 9.3)}, 
 pdflang={English}}
\begin{document}

\maketitle
\tableofcontents

\section{Categories}
\label{sec:orgbd94cd0}
\subsection{Examples of categories}
\label{sec:orgefc2503}
\begin{definition}[]
A \textbf{functor}
\begin{equation*}
F:\bC\to\bC
\end{equation*}
between categories \(\bC\) and \(\bD\) is a mapping of objects to objects and
arrows to arrows, in such a way that
\begin{enumerate}
\item \(F(f:A\to B)=F(f):F(A)\to F(B)\)
\item \(F(1_A)=1_{F(A)}\)
\item \(F(g\circ f)=F(g)\circ F(f)\)
\end{enumerate}
\end{definition}
\subsection{Free categories}
\label{sec:org1dd03ec}
The "Kleene closure" of \(A\) is defined to be the set
\begin{equation*}
A^*=\{\text{words over $A$}\}
\end{equation*}
Also
\begin{equation*}
i:A\to A^*
\end{equation*}
defined by \(i(a)=a\) and called the "intersection of generators"

A monoid \(M\) is \textbf{freely generated} by a subset \(A\) of \(M\) if the
following conditions hold:
\begin{enumerate}
\item Every element \(m\in M\) can be written as a product of elements of \(A\)
\item No "nontrivial" relations hold in \(M\), that is, if \(a_1\dots
      a_j=a_1'\dots a_k'\) ,then this is required by the axioms for monoids
\end{enumerate}


Every monoid \(N\) has an underlying set \(\abs{N}\), and every monoid
homomorphism \(f:N\to M\) has an underlying function
\(\abs{f}:\abs{N}\to\abs{M}\). The free monoid \(M(A)\) on a set \(A\) is by
definition "the" monoid with the following UMP

\emph{Universal Mapping Property of \(M(A)\)}

There is a function \(i:A\to\abs{M(A)}\), and given any monoid \(N\) and any
function \(f:A\to\abs{N}\), there is a \textbf{unique} monoid homomorphism
\(\ove{f}:M(A)\to N\) s.t. \(\abs{\over{f}}\circ i=f\)

\emph{in} \(\Mon\)
\begin{center}\begin{tikzcd}
M(A)\arrow[r,"\ove{f}",dotted]&N
\end{tikzcd}\end{center}

\emph{in} \(\Sets\)
\begin{center}\begin{tikzcd}
\abs{M(A)}\arrow[r,"\abs{\ove{f}}"]&\abs{N}\\
A\arrow[u,"i"]\arrow[ur,"f"']&
\end{tikzcd}\end{center}
\begin{proposition}[]
\(A^*\) has the UMP of the free monoid on \(A\)
\end{proposition}

\begin{proof}
Given \(f:A\to\abs{N}\), define \(\ove{f}:A^*\to N\) by
\begin{gather*}
\ove{f}(-)=u_N,\quad\text{ the unit of } N\\
\ove{f}(a_1\dots a_i)=f(a_1)\cdot_N\dots\cdot_N f(a_i)
\end{gather*}
\end{proof}


\section{Abstract structures}
\label{sec:orgbdcaf14}

\subsection{Initial and terminal objects}
\label{sec:orga9a9617}
\begin{examplle}[]
A \textbf{Boolean algebra} is a poset \(B\) equipped with distinguished elements 0,1,
binary operations \(a\vee b\) of join and \(a\wedge b\) of meet, and a unary
operation \(\neg b\) of complementation. These are required to satisfy the
conditions
\begin{align*}
0&\le a\\
a&\le 1\\
a\le c \quad\text{ and }\quad b\le c \quad&\text{ iff }\quad a\vee b\le c\\
c\le a \quad\text{ and }\quad c\le b \quad&\text{ iff }\quad c\le a\wedge b\\
a\le\neg b \quad&\text{ iff }\quad a\wedge b=0\\
\neg\neg a&=a
\end{align*}

\(\textbf{2}=\{0,1\}\) is an initial elements of \(\BA\). \(\BA\) has as
arrows the Boolean homomorphisms that \(h(0)=0,h(a\vee b)=h(a)\vee h(b)\), etc.
\end{examplle}
\subsection{Products}
\label{sec:org3628568}
\begin{definition}[]
In any category \(\bC\), a \textbf{product diagram} for the objects \(A\) and \(B\)
consists of an object \(P\) and arrows
\begin{center}\begin{tikzcd}
A&P\arrow[l,"p_1"']\arrow[r,"p_2"]&B
\end{tikzcd}\end{center}

satisfying the following UMP:

Given any diagram of the form
\begin{center}\begin{tikzcd}
A&X\arrow[l,"x_1"']\arrow[r,"x_2"]&B
\end{tikzcd}\end{center}

there exists a unique \(u:X\to P\) making the diagram
\begin{center}\begin{tikzcd}
&X\arrow[dl,"x_1"']\arrow[dr,"x_2"]\arrow[d,dashed,"u"]\\
A&P\arrow[l,"p_1"]\arrow[r,"p_2"']&B
\end{tikzcd}\end{center}
\end{definition}

\subsection{Categories with products}
\label{sec:org92bc992}
Let \(\bC\) be a category that has a product diagram for every pair of
objects. Suppose we have objects and arrows
\begin{center}\begin{tikzcd}
A\arrow[d,"f"']&A\times A'\arrow[l,"p_1"']\arrow[r,"p_2"]&A'\arrow[d,"f'"]\\
B&B\times B'\arrow[l,"q_1"]\arrow[r,"q_2"']&B'
\end{tikzcd}\end{center}
with indicated products. Then we write
\begin{equation*}
f\times f':A\times A'\to B\times B
\end{equation*}
for \(f\times f'=\la f\circ p_1,f'\circ p_2\ra\)
\begin{center}\begin{tikzcd}
A\arrow[d,"f"']&A\times A'\arrow[l,"p_1"']\arrow[r,"p_2"]
\arrow[d,dashed,"f\times f'"]
&A'\arrow[d,"f'"]\\
B&B\times B'\arrow[l,"q_1"]\arrow[r,"q_2"']&B'
\end{tikzcd}\end{center}     In this way, if we choose a product for each
pair of objects, we get a functor
\begin{equation*}
\times:\bC\times\bC\to\bC
\end{equation*}

\begin{center}\begin{tikzcd}
A\arrow[d,"f"']&A\times A'\arrow[l,"p_1"']\arrow[r,"p_2"]\arrow[d,dashed,"f\times f'"]
\arrow[dd,dashed,bend right=60]
&A'\arrow[d,"f'"]\\
B\arrow[d,"g"']&B\times B'\arrow[l,"q_1"']\arrow[r,"q_2"]\arrow[d,dashed,"g\times g'"]&B'\arrow[d,"g'"]\\
C&C\times C'\arrow[l,"o_1"']\arrow[r,"o_2"]&C'\\
\end{tikzcd}\end{center}
\((g\circ f)\times(g'\circ f')=(f\times f')\circ(g\times g')\)



To prove
\begin{equation*}
(A\times B)\times C\cong A\times (B\times C)
\end{equation*}
Consider
\begin{center}\begin{tikzcd}
&A\times (B\times C)\arrow[r]\arrow[ldd]&B\times C\arrow[ld]\arrow[rdd]\\
&B&&\\
A&A\times B\arrow[u]\arrow[l]\arrow[rd]
&(A\times B)\times C\arrow[l]\arrow[r]\arrow[uu,dashed]\arrow[luu,dashed,"g"']
&C\\
&&B\\
&A\times(B\times C)\arrow[r]\arrow[luu]\arrow[uu,dashed]\arrow[ruu,dashed,"f"]&
B\times C\arrow[u]\arrow[ruu]
\end{tikzcd}\end{center}


Given no objects, there is an object 1 with no maps, and give nany other
object \(X\) and no maps, there is a unique arrow
\begin{equation*}
!:X\to 1
\end{equation*}

\begin{definition}[]
A category \(\bC\) is said to \textbf{have all finite products}, if it has a terminal
object and all binary products (and therewith products of any finite
cardinality). The category \(\bC\) \textbf{has all (small) products} if every set of
objects in \(\bC\) has a product
\end{definition}


\subsection{Hom-sets}
\label{sec:org1ada9a0}
In this section, we assume that all categories are locally small

Given any objects \(A\) and \(B\)  in category \(\bC\),we write
\begin{equation*}
\Hom(A,B)=\{f\in\bC\mid f:A\to B\}
\end{equation*}
and call such a set of arrows a \textbf{Hom-set}

Note that any arrow \(g:B\to B'\) in \(\bC\) induces a function
\begin{gather*}
\Hom(A,g):\Hom(A,B)\to\Hom(A,B')\\
(f:A\to B)\mapsto(g\circ f:A\to B\to B')
\end{gather*}

Let's show that this determines a functor
\begin{equation*}
\Hom(A,-):\bC\to\Sets
\end{equation*}
called the (covariant) \textbf{representable functor} of \(A\). We need to show that
\begin{equation*}
\Hom(A,1_X)=1_{\Hom(A,X)}
\end{equation*}
and that
\begin{equation*}
\Hom(A,g\circ f)=\Hom(A,g)\circ\Hom(A,f)
\end{equation*}

For any object \(P\), a pair of arrows \(p_1:P\to A\) and \(p_2:P\to B\)
determine an element \((p_1,p_2)\) of the set
\begin{equation*}
\Hom(P,A)\times\Hom(P,B)
\end{equation*}
Now given any arrow
\begin{equation*}
x:X\to P
\end{equation*}
composing with \(p_1\) and \(p_2\) gives a pair of arrows
\(x_1=p_1\circ x:X\to A\) and \(x_2=p_2\circ x:X\to B\)

In this way, we have a function
\begin{equation*}
\theta_X=(\Hom(X,p_1),\Hom(X,p_2)):\Hom(X,P)\to\Hom(X,A)\times\Hom(X,B)
\end{equation*}
defined by
\begin{equation*}
\theta_X(x)=(x_1,x_2)
\end{equation*}
\begin{proposition}[]
A diagram of the form
\begin{center}\begin{tikzcd}
A&P\arrow[l,"p_1"]\arrow[r,"p_2"']&B
\end{tikzcd}\end{center}
is a product for \(A\) and \(B\) iff for every object \(X\), the canonical
function \(\theta_X\) is an isomorphism
\begin{equation*}
\theta_X:\Hom(X,P)\cong\Hom(X,A)\times\Hom(X,B)
\end{equation*}
\end{proposition}

\begin{proof}
Note that we are talking about isomorphism on the set
\end{proof}

\begin{definition}[]
Let \(\bC,\bD\) be categories with binary products. A functor
\(F:\bC\to\bD\) is said to \textbf{preserve binary products} if it takes every
product diagram
\begin{center}\begin{tikzcd}
A&A\times B\arrow[l,"p_1"]\arrow[r,"p_2"']&B
\end{tikzcd}\end{center}
to a product diagram
\begin{center}\begin{tikzcd}
FA&F(A\times B)\arrow[l,"Fp_1"]\arrow[r,"Fp_2"']&FB
\end{tikzcd}\end{center}
\end{definition}
\(F\) preserves products just if
\begin{equation*}
F(A\times B)\cong FA\times FB
\end{equation*}
\begin{corollary}[]
For any object \(X\) in a category \(\bC\) with products, the (covariant)
representable functor
\begin{equation*}
\Hom_{\bC}(X,-):\bC\to\Sets
\end{equation*}
preserves products
\end{corollary}

\section{Duality}
\label{sec:org2143621}

\subsection{Coproducts}
\label{sec:orgc35d225}
\begin{definition}[]
A diagram
\begin{tikzcd}
A\arrow[r,"q_1"]&Q&B\arrow[l,"q_2"']
\end{tikzcd}
is a coproduct of \(A,B\) if for any \(Z\) and
\begin{tikzcd}
A\arrow[r,"z_1"]&Z&B\arrow[l,"z_2"']
\end{tikzcd}
there is a unique \(u:Q\to Z\) with \(u\circ q_i=z_i\)
\begin{center}\begin{tikzcd}
&Z\\
A\arrow[ur,"z_1"]\arrow[r,"q_1"]&Q\arrow[u,dotted,"u"]&B\arrow[l,"q_2"]\arrow[ul,"z_2"']
\end{tikzcd}\end{center}
written as \(A+B\)
\end{definition}

In \(\Sets\), every finite set \(A\) is a coproduct
\begin{equation*}
A\cong1+1+\dots+1\quad(n\text{-times})
\end{equation*}
\begin{examplle}[]
If \(M(A)\) and \(M(B)\) are free monoids on sets \(A\) and \(B\), then in
\(\Mon\) we can construct their coproduct as
\begin{equation*}
M(A)+M(B)\cong M(A+B)
\end{equation*}

\begin{center}\begin{tikzcd}
&N\\
M(A)\arrow[r]\arrow[ur]&M(A+B)\arrow[u,dotted]&M(B)\arrow[l]\arrow[ul]\\
A\arrow[u,"\eta_A"]\arrow[r]&A+B\arrow[u,"\eta_{A+B}"]&B\arrow[l]\arrow[u,"\eta_B"']
\end{tikzcd}\end{center}
Here we are working in two different categories. Half below is in \(\Sets\),
the other is \(\Mon\)
\end{examplle}

Product of two powerset Boolean algebras \(\calp(A)\) and \(\calp(B)\) is
also a powerset
\begin{equation*}
\calp(A)\times\calp(B)\cong\calp(A+B)
\end{equation*}  
\begin{examplle}[]
Two monoids \(M(\abs{A}+\abs{B})\) is strings over the disjoint union
\(\abs{A}+\abs{B}\) of the underlying sets. That is, the elements of \(A\)
and \(B\) and the equivalence relation \(v\sim w\) is the least one
containing all instances of the following equations
\begin{align*}
(\dots xu_Ay\dots)&=(\dots xy\dots)\\
(\dots xu_By\dots)&=(\dots xy\dots)\\
(\dots aa'\dots)&=(\dots a\cdot_A a'\dots)\\
(\dots bb'\dots)&=(\dots b\cdot_B b'\dots)
\end{align*}
The unit is the equivalence class \([-]\) of the empty word. Multiplication
is
\begin{equation*}
[x\dots y]\cdot[x'\dots y']=[x\dots yx'\dots y']
\end{equation*}
The coproduct injections \(i_A:A\to A+B\) and \(i_B:B\to A+B\) are
\begin{equation*}
i_A(a)=[a],\quad i_B(b)=[b]
\end{equation*}
Given any homomorphisms \(f:A\to M\) and \(g:B\to M\) into a monoid, the
unique homomorphism
\begin{equation*}
[f,g]:A+B\to M
\end{equation*}
is defined by
\begin{center}\begin{tikzcd}
\abs{A}+\abs{B}\arrow[r,"{[\abs{f}+\abs{g}]}"]&\abs{M}\\
M(\abs{A}+\abs{B})\arrow[r,"{[f,g]}'"]\arrow[d,twoheadrightarrow]&M\\
M(\abs{A}+\abs{B})/\sim\arrow[ur,dotted,"{[f,g]}"']
\end{tikzcd}\end{center}
\end{examplle}
\end{document}