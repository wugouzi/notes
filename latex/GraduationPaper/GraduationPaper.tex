% Created 2021-03-06 六 14:54
% Intended LaTeX compiler: pdflatex
\documentclass[11pt]{article}
\usepackage[utf8]{inputenc}
\usepackage[T1]{fontenc}
\usepackage{graphicx}
\usepackage{grffile}
\usepackage{longtable}
\usepackage{wrapfig}
\usepackage{rotating}
\usepackage[normalem]{ulem}
\usepackage{amsmath}
\usepackage{textcomp}
\usepackage{amssymb}
\usepackage{capt-of}
\usepackage{hyperref}
% TIPS
% \substack{a\\b} for multiple lines text





% pdfplots will load xolor automatically without option
\usepackage[dvipsnames]{xcolor}

\usepackage{forest}
% two-line text in node by [two \\ lines]
% \begin{forest} qtree, [..] \end{forest}
\forestset{
  qtree/.style={
    baseline,
    for tree={
      parent anchor=south,
      child anchor=north,
      align=center,
      inner sep=1pt,
    }}}
%\usepackage{flexisym}
% load order of mathtools and mathabx, otherwise conflict overbrace

\usepackage{mathtools}
%\usepackage{fourier}
\usepackage{pgfplots}
\usepackage{amsthm, mathabx,  amsmath, commath}
\usepackage{amsfonts}

\usepackage{empheq}
\usepackage{tikz}
\usetikzlibrary{arrows.meta}
\usepackage[most]{tcolorbox}

\newtheorem{theorem}{Theorem}[section]
\newtheorem{definition}{Definition}[section]
\newtheorem{corollary}{Corollary}[section]
\newtheorem{example}{Example}[section]
\newtheorem{lemma}{Lemma}[section]
\newtheorem{proposition}{Proposition}[section]

\newcommand{\bl}[1] {\boldsymbol{#1}}
\newcommand{\Wt}[1] {\stackrel{\sim}{\smash{#1}\rule{0pt}{1.1ex}}}
\newcommand{\wt}[1] {\widetilde{#1}}


%For boxed texts in align, use Aboxed{}
%otherwise use boxed{}

\DeclareMathSymbol{\widehatsym}{\mathord}{largesymbols}{"62}
\newcommand\lowerwidehatsym{%
  \text{\smash{\raisebox{-1.3ex}{%
    $\widehatsym$}}}}
\newcommand\fixwidehat[1]{%
  \mathchoice
    {\accentset{\displaystyle\lowerwidehatsym}{#1}}
    {\accentset{\textstyle\lowerwidehatsym}{#1}}
    {\accentset{\scriptstyle\lowerwidehatsym}{#1}}
    {\accentset{\scriptscriptstyle\lowerwidehatsym}{#1}}
}

\usepackage{graphicx}
    
% text on arrow for xRightarrow
\makeatletter
%\newcommand{\xRightarrow}[2][]{\ext@arrow 0359\Rightarrowfill@{#1}{#2}}
\makeatother


\def \bx {\boldsymbol{x}}
\def \ba {\boldsymbol{a}}
\def \bI {\boldsymbol{I}}
\def \bt {\boldsymbol{t}}
\def \bb {\boldsymbol{b}}
\def \bA {\boldsymbol{A}}
\def \bX {\boldsymbol{X}}
\def \bu {\boldsymbol{u}}
\def \bS {\boldsymbol{S}}
\def \bZ {\boldsymbol{Z}}
\def \bz {\boldsymbol{z}}
\def \by {\boldsymbol{y}}
\def \bw {\boldsymbol{w}}
\def \bT {\boldsymbol{T}}
\def \bS {\boldsymbol{S}}
\def \bm {\boldsymbol{m}}
\def \bW {\boldsymbol{W}}
\def \bY {\boldsymbol{Y}}
\def \bH {\boldsymbol{H}}
\def \blambda {\boldsymbol{\lambda}}
\def \bPhi {\boldsymbol{\Phi}}
\def \btheta {\boldsymbol{\theta}}
\def \bmu {\boldsymbol{\mu}}
\def \bphi {\boldsymbol{\phi}}
\def \bSigma {\boldsymbol{\Sigma}}
\def \lb {\left\{}
\def \rb {\right\}}
\def \caln {\mathcal{N}}
\def \dissum {\displaystyle\Sigma}
\def \dispro {\displaystyle\prod}
\def \E {\mathbb{E}}
\def \Q {\mathbb{Q}}
\def \V {\mathbb{V}}
\def \R {\mathbb{R}}
\def \calq {\mathcal{Q}}
\def \calg {\mathcal{G}}
\def \caln {\mathcal{N}}
\def \calr {\mathcal{R}}
\def \calm {\mathcal{M}}
\def \calc {\mathcal{C}}
\def \bcup {\bigcup}

\def \EQ {\text{EQ}}
\def \Truth {\text{Truth}}
\def \EQUAL {\text{EQUAL}}
\def \Table {\text{Table}}
\author{wugouzi}
\date{\today}
\title{Graduation Paper Collection}
\hypersetup{
 pdfauthor={wugouzi},
 pdftitle={Graduation Paper Collection},
 pdfkeywords={},
 pdfsubject={},
 pdfcreator={Emacs 27.1 (Org mode 9.3)}, 
 pdflang={English}}
\begin{document}

\maketitle
\tableofcontents


\section{Mining stock price using fuzzy rough set system}
\label{sec:org522293b}
\subsection{Preliminaries}
\label{sec:orgca4a0f3}
To store data, we have to define our fuzzy relational model. A fuzzy
relational schema \(R(A_1,\dots,A_n,u_r)\) is made up by a relation name
\(R\) and a list of attributes \(A_1,\dots,A_n,u_r\). Each attribute \(A_i\)
is the name of a role played by some domain, \(\dom(A_i)\), and \(u_r\) is
characterized by the following membership function:
\begin{equation*}
u_r:\dom(A_1)\times\dots\times\dom(A_n)\to[0,1]
\end{equation*}

\begin{definition}[]
Let \(R(A_1,\dots,A_n,u_r(t))\) be a fuzzy relational schema. An \(n\)-ary
fuzzy relation \(r\) over \(R\) is a set of
\(\dom(A_1)\times\dots\times\dom(A_n)\), \(t[A_i]\) refers to the value in
\(t\) for attribute \(A_i\), and each tuple
\(t\in\dom(A_1)\times\dots\times\dom(A_n)\). That is
\begin{align*}
r=&\{(t,u_r(t))\mid
t=((t[A_1],u_1(t[A_1])),\dots,(t[A_n],u_n(t[A_n]))),\\
&\text{and for }i=1,\dots,n\\
&t[A_i]\in\dom(A_i),u_i(t[A_i])\in[0,1],\\
&u_r(t)=\min(u_1(t[A_1]),\dots,u_n(t[A_n]))
\}
\end{align*}
\end{definition}

In this paper, we use a fuzzy relation \(\EQUAL(\EQ)\) to compute the
membership degree of an object w.r.t. a given criteria, \(\varphi(A)\)

\begin{definition}[]
A fuzzy relation \(\EQUAL(\EQ)\) over \(\dom(A_j)\) is characterized by the
membership function \(u_{\EQ}\), where \(u_{\EQ}\) satisfies the following
conditions
\begin{equation*}
\text{For all }x,y\in\dom(A_j),u_{\EQ}(x,x)=1\text{ and }u_{\EQ}(x,y)=u_{\EQ}(y,x)
\end{equation*}

According to Zadeh's possibility theory, \(u_{\EQ}(x,y)\) can be interpreted
as the possibility of treating the two values \((x,u(x))\) and \((u,u(y))\)
equally under the same fuzzy term.

Since an object may partially match a certain property in many situations, we
define a prespecified threshold value, \(\alpha_j\), and let
\(u_r(t_i)\ge\alpha_j\). When \(u_r(t_i)<\alpha_j\), we assume that the tuple
\(t_i\) doesn't satisfy to the given property

For example, let \(\varphi(A)=y\). Then a fuzzy relation \(\EQUAL(\EQ)\) over an
attribute domain, \(\dom(A_j)\), w.r.t. \(y\) can be defined as
\begin{align*}
u_{\EQ}(x,y)&=\\
\bigtimes
&\begin{cases}
0&\quad\text{for }1-\abs{u_j(x)-u_j(y)}<\alpha_j\\
1-\abs{u_j(x)-u_j(y)}<\alpha_j&\quad\text{for }1-\abs{u_j(x)-u_j(y)}\ge\alpha_j
\end{cases}
\end{align*}
\end{definition}

\begin{definition}[]
Let the criteria \(\varphi(A)=\varphi(A_i)\) and \(\varphi(A_i)=y_i\). Then, a fuzzy set
\(X\) w.r.t. the criteria \(\varphi(A)\) in a fuzzy relation \(r\) can be defined
as
\begin{equation*}
X=\{(t,u_r(t))\mid t\in r\text{ and }u_r(t)=u_{\EQ}(t[A_i],y_i)\}
\end{equation*}
\end{definition}

Let \(X=\{x_1,\dots,x_n\}\) be a set of objects. Denote \(P(X)\) the set of
all crisp subsets from \(X\) and by \(F(X)\) the set of all fuzzy subsets
from \(X\). Any subset \(A\subset X\) will be called a \textbf{concept} in \(X\).
\(u_A\) is the membership function of \(A\). Let \(C=\{C_1,\dots,C_n\}\),
\(C_i\in P(X)\) be a family of concepts in \(X\) that forms a partition of
\(X\) and let \(Y\) be a crisp subset of \(X\).Lower and upper approximation
of \(Y\)
\begin{equation*}
\und{C}(Y)=\bigcup_{C_i\subset Y}C_i,\quad\ove{C}(Y)=\bigcup_{C_i\cap Y\neq\emptyset}C_i
\end{equation*}

The pair \((\und{C}(Y),\ove{C}(Y))\) is called a \textbf{rough set}.

Consider a situation where concepts \(C_1,\dots,C_n\in F(X)\)  from a weak
fuzzy partition and concept \(Y\) is a fuzzy set on \(X\).

Then the lower and the upper approximation of \(Y\) by means of \(C\) are
defined as fuzzy sets of \(X/C\) with membership functions
\begin{align*}
&u_{\und{C}(Y)}(C_i)=\inf_x\max\{1-u_{ci}(x),u_Y(X)\}\\
&u_{\ove{C}(Y)}(C_i)=\sup_x\min\{u_{ci}(x),u_Y(x)\}
\end{align*}
where \(u_{\und{C}(Y)}(C_i)\) is the degree of certain membership of \(C_i\)
in \(Y\) and \(u_{\ove{C}(Y)}(C_i)\) is the corresponding degree of possible
membership. The pair \((u_{\und{C}(Y)}(C_i),u_{\ove{C}(Y)}(C_i))\) is called
a \textbf{fuzzy rough set}

The stock price at a specific time can be influenced by numerous factors. For
example

\begin{center}
\begin{tabular}{rrrr}
\hline
Date & Time & Stock price & \(u_{price}(t[price])\)\\
\hline
20010205 & 09:00 & 35.81 & 1\\
20010205 & 10:00 & 35.20 & 0.9662\\
20010205 & 11:00 & 34.29 & 0.9169\\
20010205 & 12:00 & 33.48 & 0.8741\\
20010205 & 13:00 & 35.20 & 0.9662\\
20010205 & 13:30 & 35.01 & 0.9777\\
\hline
\end{tabular}
\end{center}

\begin{center}
\begin{tabular}{rrrr}
\hline
Date & Time & Stock price & \(u_{price}(t[price])\)\\
\hline
20000815 & 09:00 & 86.91 & 1\\
20000815 & 10:00 & 85.23 & 0.9617\\
20000815 & 11:00 & 83.02 & 0.9125\\
20000815 & 12:00 & 80.92 & 0.8669\\
20000815 & 13:00 & 82.47 & 0.9004\\
20000815 & 13:30 & 82.15 & 0.9452\\
\hline
\end{tabular}
\end{center}

The value \(u_{price}(t[price])\) is the degree of the stock price at a
specfic trading hour on that day, and the membership function of
\(u_{price}\) is given as
\begin{equation*}
u_{price}(x)=(x/y)^2
\end{equation*}
where \(x\) is the stock price at a specific trading hour on one day, and
\(y\) is the highest stock price during trading hours on the same day

Particularly, the change rate at an hour interval of the stock price may vary
little each day. Since different times present different specific behavior,
we employ the roughness technique to pre-classify stock price data into
different groups by time and consider only the behavior of stock price of
trading hours on weekdays. Each group has the same values for the attribute
time.

\begin{center}
\begin{tabular}{lrr}
\hline
 & Time & \(u_{price}(t[price])\)\\
\hline
Group 1 & 09:00 & 0.9864\\
 & \(\vdots\) & \(\vdots\)\\
 & 09:00 & 1\\
Group 2 & 10:00 & 0.9327\\
 & \(\vdots\) & \(\vdots\)\\
 & 10:00 & 0.9665\\
\(\vdots\) & \(\vdots\) & \(\vdots\)\\
Group 3 & 14:00 & 0.9887\\
 & 14:00 & 0.9678\\
\hline
\end{tabular}
\end{center}

For concentrating the change rate at hour intervals of the stock price, we
also used roughness technique and assume the change rate of stock price is in
following ranks

\begin{center}
\begin{tabular}{l}
\hline
Rank 0: 6\textasciitilde{}7\%\\
Rank 1: 5\textasciitilde{}6\%\\
Rank 2: 4\textasciitilde{}5\%\\
Rank 3: 3\textasciitilde{}4\%\\
Rank 4: 2\textasciitilde{}3\%\\
Rank 5: 1\textasciitilde{}2\%\\
Rank 6: 0\textasciitilde{}1\%\\
Rank 7: -1\textasciitilde{}0\%\\
Rank 8: -2\textasciitilde{}-1\%\\
Rank 9: -3\textasciitilde{}-2\%\\
Rank 10: -4\textasciitilde{}-3\%\\
Rank 11: -5\textasciitilde{}-4\%\\
Rank 12: -6\textasciitilde{}-5\%\\
Rank 13: -7\textasciitilde{}-6\%\\
\hline
\end{tabular}
\end{center}

A \textbf{linguistic summary} is a linguistically quantified proposition containing
metaknowledge about a set of particular objects.

The general form of a linguistically quantified proposition is usually
written as:
\begin{equation*}
\bQ\bX\text{ are }\bF
\end{equation*}
where \(\bQ\) is a (fuzzy) linguistic quantifier, \(\bX\) is a set of
objects, and \(\bF\) is a property

To a relational database, let '\(\bQ\{t_1,\dots,t_n\}\) are \(\bF\)' be a
linguistically quantified proposition, \(t_1,\dots,t_n\) be a set of tuples
in  a fuzzy relation \(r\). Then according to the semantics of the property,
the truth value of '\(\bQ\{t_1,\dots,t_n\}\) are \(\bF\)' over the fuzzy
relation \(r\) can be computed by the following definition

\begin{definition}[]
Let \(\{t_1,\dots,t_n\}\) be a set of tuples in a group over a relation
\(r\). Then
\begin{equation*}
\Truth(\bQ\{t_1,\dots,t_n\}\text{ are }\bF)=
u_Q\left(\frac{1}{n}\left(
\sum_{i=1}^n u_r(t_i)
\right)
\right)
\end{equation*}
where \(u_t(t_i)\) is a degree membership value of \(t_i\) w.r.t. \(\bF\).
When \(\bF=F_1\vee\dots\vee F_m\), \(u_r(t_i)=\max_{j=1}^m(u_{F_j}(t_i))\)
and when \(\bF=F_1\wedge\dots\wedge F_m\),
\(u_r(t_i)=\min_{j=1}^m(u_{F_j}(t_i))\). 
\end{definition}

\subsection{A fuzzy rough set system}
\label{sec:org19b0288}

\subsubsection{Mining agent using fuzzy rough set method}
\label{sec:org6494ac5}
In the system, the corresponding domain knowledge is represented as
\begin{equation*}
\{\Table.A_1,\dots,\Table.A_k\}\approx\bB
\end{equation*}
where \(\approx\) stands for 'relate to', and for \(i=1,\dots,k\) attribute
\(\Table.A_i\) relates to knowledge \(\bB\). For example, when we want to
predict the ranks of stock price at a specifc time, the corresponding domain
knowledge w.r.t. the WSP relation in table can be represented as
\begin{align*}
&\{\text{WSP.Date,WSP.Time,WSP.Stock\_Price,}\\
&\text{WSP.}(t[\text{price}])\}\approx\bB
\end{align*}
\end{document}