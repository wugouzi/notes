% Created 2020-09-17 四 13:21
% Intended LaTeX compiler: pdflatex
\documentclass[11pt]{article}
\usepackage[utf8]{inputenc}
\usepackage[T1]{fontenc}
\usepackage{graphicx}
\usepackage{grffile}
\usepackage{longtable}
\usepackage{wrapfig}
\usepackage{rotating}
\usepackage[normalem]{ulem}
\usepackage{amsmath}
\usepackage{textcomp}
\usepackage{amssymb}
\usepackage{capt-of}
\usepackage{hyperref}
\usepackage{minted}
% TIPS
% \substack{a\\b} for multiple lines text





% pdfplots will load xolor automatically without option
\usepackage[dvipsnames]{xcolor}

\usepackage{forest}
% two-line text in node by [two \\ lines]
% \begin{forest} qtree, [..] \end{forest}
\forestset{
  qtree/.style={
    baseline,
    for tree={
      parent anchor=south,
      child anchor=north,
      align=center,
      inner sep=1pt,
    }}}
%\usepackage{flexisym}
% load order of mathtools and mathabx, otherwise conflict overbrace

\usepackage{mathtools}
%\usepackage{fourier}
\usepackage{pgfplots}
\usepackage{amsthm, mathabx,  amsmath, commath}
\usepackage{amsfonts}

\usepackage{empheq}
\usepackage{tikz}
\usetikzlibrary{arrows.meta}
\usepackage[most]{tcolorbox}

\newtheorem{theorem}{Theorem}[section]
\newtheorem{definition}{Definition}[section]
\newtheorem{corollary}{Corollary}[section]
\newtheorem{example}{Example}[section]
\newtheorem{lemma}{Lemma}[section]
\newtheorem{proposition}{Proposition}[section]

\newcommand{\bl}[1] {\boldsymbol{#1}}
\newcommand{\Wt}[1] {\stackrel{\sim}{\smash{#1}\rule{0pt}{1.1ex}}}
\newcommand{\wt}[1] {\widetilde{#1}}


%For boxed texts in align, use Aboxed{}
%otherwise use boxed{}

\DeclareMathSymbol{\widehatsym}{\mathord}{largesymbols}{"62}
\newcommand\lowerwidehatsym{%
  \text{\smash{\raisebox{-1.3ex}{%
    $\widehatsym$}}}}
\newcommand\fixwidehat[1]{%
  \mathchoice
    {\accentset{\displaystyle\lowerwidehatsym}{#1}}
    {\accentset{\textstyle\lowerwidehatsym}{#1}}
    {\accentset{\scriptstyle\lowerwidehatsym}{#1}}
    {\accentset{\scriptscriptstyle\lowerwidehatsym}{#1}}
}

\usepackage{graphicx}
    
% text on arrow for xRightarrow
\makeatletter
%\newcommand{\xRightarrow}[2][]{\ext@arrow 0359\Rightarrowfill@{#1}{#2}}
\makeatother


\def \bx {\boldsymbol{x}}
\def \ba {\boldsymbol{a}}
\def \bI {\boldsymbol{I}}
\def \bt {\boldsymbol{t}}
\def \bb {\boldsymbol{b}}
\def \bA {\boldsymbol{A}}
\def \bX {\boldsymbol{X}}
\def \bu {\boldsymbol{u}}
\def \bS {\boldsymbol{S}}
\def \bZ {\boldsymbol{Z}}
\def \bz {\boldsymbol{z}}
\def \by {\boldsymbol{y}}
\def \bw {\boldsymbol{w}}
\def \bT {\boldsymbol{T}}
\def \bS {\boldsymbol{S}}
\def \bm {\boldsymbol{m}}
\def \bW {\boldsymbol{W}}
\def \bY {\boldsymbol{Y}}
\def \bH {\boldsymbol{H}}
\def \blambda {\boldsymbol{\lambda}}
\def \bPhi {\boldsymbol{\Phi}}
\def \btheta {\boldsymbol{\theta}}
\def \bmu {\boldsymbol{\mu}}
\def \bphi {\boldsymbol{\phi}}
\def \bSigma {\boldsymbol{\Sigma}}
\def \lb {\left\{}
\def \rb {\right\}}
\def \caln {\mathcal{N}}
\def \dissum {\displaystyle\Sigma}
\def \dispro {\displaystyle\prod}
\def \E {\mathbb{E}}
\def \Q {\mathbb{Q}}
\def \V {\mathbb{V}}
\def \R {\mathbb{R}}
\def \calq {\mathcal{Q}}
\def \calg {\mathcal{G}}
\def \caln {\mathcal{N}}
\def \calr {\mathcal{R}}
\def \calm {\mathcal{M}}
\def \calc {\mathcal{C}}
\def \bcup {\bigcup}

\usepackage[UTF8]{ctex}
\author{喵喵喵}
\date{\today}
\title{概率论与数理统计}
\hypersetup{
 pdfauthor={喵喵喵},
 pdftitle={概率论与数理统计},
 pdfkeywords={},
 pdfsubject={},
 pdfcreator={Emacs 26.3 (Org mode 9.4)}, 
 pdflang={English}}
\begin{document}

\maketitle
\tableofcontents \clearpage\section{概率论的基本概念}
\label{sec:org88b7bd7}
\subsection{随机实验}
\label{sec:org62e8fd6}
\subsection{样本空间、随机事件}
\label{sec:org8e337dc}
将随机试验\(E\)的所有可能结果组成的集合称为\(E\)的 \textbf{样本空间} ,记为\(S\),样本
空间的元素,即\(E\)的每个结果,称为 \textbf{样本点}

称试验\(E\)的样本空间\(S\)的子集为\(E\) \textbf{随机事件} ,简称 \textbf{事件} 。在每次试验中,
当且仅当这一子集中的一个样本点出现时,称这一 \textbf{事件发生}

由一个样本点组成的单点集称为 \textbf{基本事件}

样本空间\(S\)包含所有的样本点,他是\(S\)自身的子集,在每次试验中它总是出现,
\(S\)称为 \textbf{必然事件} ,空集\(\emptyset\)称为 \textbf{不可能事件}

设试验\(E\)的样本空间为\(S\),而\(A,B,A_k\subseteq S\)
\begin{enumerate}
\item 若\(A\subset B\),则称事件\(B\)包含事件\(A\),事件\(A\)的发生必导致事件
\(B\)发生

若\(A\subset B,B\subset A\)即\(A=B\),则称事件\(A\)与事件\(B\) \textbf{相等}

\item 事件\(A\cup B=\{x\mid x\in A\text{ or }x\in B\}\)称为事件\(A\)与事件\(B\)
的 \textbf{和事件} ,当且仅当\(A,B\)中至少由一个发生时,事件\(A\cup B\)发生

\item 事件\(A\cap B=\{x\mid x\in A\text{ and }x\in B\}\)称为事件\(A\)与事件\(B\)
的 \textbf{积事件} , 也记作\(AB\)

称\(\bigcap_{k=1}^\infty A_k\)为可列个事件\(A_1,A_2,\dots\)的积事件

\item 事件\(A-B=\{x\mid x\in A\text{ and }x\not\in B\}\)称为事件\(A\)与事件\(B\)
的 \textbf{差事件}

\item 若\(A\cap B=\emptyset\),则称事件\(A\)与事件\(B\) \textbf{互不相容} 或 \textbf{互斥} 的

\item 若\(A\cup B=S,A\cap B=\emptyset\),则称事件\(A\)与事件\(B\)互为 \textbf{逆事件} ,
又称事件\(A\)与事件\(B\)互为 \textbf{对立事件}
\end{enumerate}
\subsection{频率与概率}
\label{sec:org36f631c}
\begin{definition}[]
在相同的条件下,进行了\(n\)次试验,在这\(n\)次试验中,事件\(A\)发生次数
\(n_A\)称为事件\(A\)发生的 \textbf{频数} ,比值\(n_A/n\)称为事件\(A\)发生的 \textbf{频率} ,并
记成 \(f_n(A)\)
\end{definition}

基本性质
\begin{enumerate}
\item \(0\le f_n(A)\le1\)
\item \(f_n(S)=1\)
\item 若\(A_1,\dots,A_k\)是两两互不相容的事件,则
\begin{equation*}
f_n(A_1\cup\dots\cup A_k)=f_n(A_1)+\dots+f_n(A_k)
\end{equation*}
\end{enumerate}


\begin{definition}[]
设\(E\)是随机试验,\(S\)是它的样本空间,对于\(E\)的每一事件\(A\)赋予一个实数,
记为\(P(A)\),称为事件\(A\)的 \textbf{概率} ,如果集合函数\(P(\cdot)\)满足下列条件
\begin{enumerate}
\item \textbf{非负性} :对于每一个事件\(A\),有\(P(A)\ge0\)
\item \textbf{规范性} :对于必然事件,有\(P(A)=1\)
\item \textbf{可列可加性} :设\(A_1,A_2,\dots\)是两两互不相容事件,有
\begin{equation*}
P(A_1\cup A_2\cup\dots)=P(A_1)+P(A_2)+\dots
\end{equation*}
\end{enumerate}
\end{definition}

\begin{proposition}[]
\(P(\emptyset)=0\)
\end{proposition}

\begin{proof}
令\(A_n=\emptyset\),则\(\bigcup_{n=1}^\infty A_n=\emptyset\),且
\(A_iA_j=\emptyset\),由可列可加性
\begin{equation*}
P(\emptyset)=P(\bigcup_{n=1}^\infty A_n)=\sum_{n=1}^\infty P(A_n)=
\sum_{n=1}^\infty P(\emptyset)
\end{equation*}
由概率的非负性,\(P(\emptyset)=0\)
\end{proof}

\begin{proposition}[有限可加性]
若\(A_1,\dots,A_n\)是两两互不相容的事件,则有
\begin{equation*}
P(A_1\cup A_2\cup\dots\cup A_n)=P(A_1)+\dots+P(A_n)
\end{equation*}
\end{proposition}

\begin{proof}
令\(A_{n+1}=A_{n+2}=\dots=\emptyset\),即有\(A_iA_j=\emptyset\)

\begin{align*}
P(A_1&\cup A_2\cup\dots\cup A_n)\\
&=P(\bigcup_{k=1}^\infty A_k)=\sum_{k=1}^\infty P(A_k)\\
&=\sum_{k=1}^\infty P(A_k)+0=P(A_1)+\dots+P(A_n)
\end{align*}
\end{proof}

\begin{proposition}[]
设\(A,B\)是两个事件,若\(A\subset B\),则有
\begin{gather*}
P(B-A)=P(B)-P(A)\\
P(B)\ge P(A)
\end{gather*}
\end{proposition}

\begin{proof}
\(B=A\cup (B-A)\)
\end{proof}

\begin{proposition}[]
对于任一事件\(A\)
\begin{equation*}
P(A)\le 1
\end{equation*}
\end{proposition}

\begin{proof}
\(P(A)\le P(S)=1\)
\end{proof}

\begin{proposition}[逆事件的概率]
对于任一事件\(A\),有
\begin{equation*}
P(\bbar{A})=1-P(A)
\end{equation*}
\end{proposition}

\begin{proof}
\(P(S)=P(A\cup\bbar{A})\)
\end{proof}

\begin{proposition}[加法公式]
对于任一两事件\(A,B\)有
\begin{equation*}
P(A\cup B)=P(A)+P(B)-P(AB)
\end{equation*}
\end{proposition}

\begin{proof}
\(A\cup B=A\cup (B-AB)\)
\begin{equation*}
P(A\cup B)=P(A)+P(B-AB)=P(A)+P(B)-P(AB)
\end{equation*}
\end{proof}

可推广到
\begin{align*}
P(A_1\cup A_2\cup\dots\cup A_n)&=\sum_{i=1}^nP(A_i)
-\sum_{1\le i<j\le n}P(A_iA_j)\\
+\sum_{1\le i<j<k\le n}P(A_iA_jA_k)+\dots+(-1)^{n-1}P(A_1\dots A_n)
\end{align*}
\subsection{等可能概型(古典概型)}
\label{sec:org177bc39}
\textbf{等可能概型} (古典概型)
\begin{enumerate}
\item 试验的样本空间只包含有限个元素
\item 试验中每个基本事件发生的可能性相同
\end{enumerate}


设试验的样本空间为\(S=\{e_1,\dots,e_n\}\),由于在试验中每个基本事件发生的可能
性相同,即
\begin{equation*}
P(\{e_1\})=\dots=P(\{e_n\})
\end{equation*}
又由于基本事件两两互不相容,于是
\begin{align*}
1&=P(S)=P(\{e_1\}\cup\dots\cup\{e_n\})\\
&=P(\{e_1\})+\dots+P(\{e_n\})\\
&=nP(\{e_i\})
\end{align*}
于是
\begin{equation*}
P(\{e_i\})=\frac{1}{n}
\end{equation*}
若事件\(A\)包含\(k\)个基本 i 事件,即\(A=\{e_{i_1}\}\cup\dots\cup\{e_{i_k}\}\),
则有
\begin{align*}
 P(A)=\sum_{j=1}^nP(\{e_{i_j}\})=\frac{k}{n}
\end{align*}

\begin{examplle}[]
设有\(N\)件产品,其中有\(D\)件次品,今从中任取\(n\)件,文其中恰有\(k(k\le
   D)\)件次品的概率

\begin{equation*}
p=\frac{\binom{D}{k}\binom{N-D}{n-k}}{\binom{N}{n}}
\end{equation*}
\end{examplle}

\begin{examplle}[]
袋中有\(a\)只白球,\(b\)只红球,\(k\)个人依次在袋中取一只球,求第\(i\)人取到
白球(记为事件\(B\))的概率(\(k\le a+b\))

共有\(A_{a+b}^k\)个基本事件,事件\(B\)发生时,第\(i\)人取的应是白球,有\(a\)
中取法,剩余\(k-1\)只球有\(A_{a+b-1}^{k-1}\)种取法,则
\begin{equation*}
P(B)=\frac{a\cdot A_{a+b-1}^{k-1}}{A_{a+b}^k}=\frac{a}{a+b}
\end{equation*}
\end{examplle}
\subsection{条件概率}
\label{sec:org39583f7}
\begin{definition}[]
设\(A,B\)是两个事件,且\(P(A)>0\),称
\begin{equation*}
P(B|A)=\frac{P(AB)}{P(A)}
\end{equation*}
为在事件\(A\)发生的条件下事件\(B\)发生的 \textbf{条件概率}
\end{definition}

条件概率\(P(\cdot|A)\)符合
\begin{enumerate}
\item \textbf{非负性} :对于每一事件\(B\),有\(P(B|A)\ge0\)
\item \textbf{规范性} :对于必然事件\(S\),有\(P(S|A)=1\)
\item \textbf{可列可加性} :设\(B_1,B_2,\dots\)是两两互不相容的事件,则有
\begin{equation*}
P(\bigcup_{i=1}^\infty B_i|A)=\sum_{i=1}^\infty P(B_i|A)
\end{equation*}
\end{enumerate}


\begin{theorem}[乘法定理]
设\(P(A)>0\),则有
\begin{equation*}
P(AB)=P(B|A)P(A)
\end{equation*}
\end{theorem}
一般地,设\(A_1,\dots,A_n\)为\(n\)个事件,\(n\ge2\),且\(P(A_1\dots
   A_{n-1})>0\),则有
\begin{equation*}
P(A_1\dots A_n)=P(A_n|A_1\dots A_{n-1})P(A_{n-1}|A_1\dots A_{n-2})\dots P(A_2|A_1)P(A_1)
\end{equation*}

\begin{examplle}[]
设袋中装有\(r\)只红球,\(t\)只白球,每次自袋中任取一只球,观察其颜色再放回,
并再放入\(a\)只与所取出的那只球同色的球,若在袋中连续取球四次,试求第一、二次
取到红球且第三、四次取到白球的概率

以\(A_i\)表示事件“第\(i\)次取到红球”,则
\begin{align*}
 P(A_1A_2\bbar{A_3}\bbar{A_4})&=P(\bbar{A_4}|A_1A_2\bbar{A_3})
P(\bbar{A_3}|A_1A_2)P(A_2|A_1)P(A_1)\\
&=\frac{t+a}{r+t+3a}\cdot\frac{t}{r+t+2a}\cdot\frac{r+a}{r+t+a}\cdot\frac{r}{r+t}
\end{align*}
\end{examplle}

\begin{definition}[]
设\(S\)为试验\(E\)的样本空间,\(B_1,\dots,B_n\)为\(E\)的一组事件,若
\begin{enumerate}
\item \(B_iB_j=\emptyset,i\neq j,i,j=1,2,\dots,n\)
\item \(B_1\cup B_2\cup\dots\cup B_n=S\)
\end{enumerate}


则称\(B_1,\dots,B_n\)为样本空间\(S\)的一个 \textbf{划分}
\end{definition}

\begin{theorem}[]
设试验\(E\)的样本空间为\(S\),\(A\)为\(E\)的事件,\(B_1,\dots,B_n\)为\(S\)的
一个划分,且\(\iffalse<\fi P(B_i)>0\),则
\begin{align*}
P(A)&=P(A|B_1)P(B_1)+\dots+P(A|B_n)P(B_n)
\end{align*}
称为 \textbf{全概率公式} 
\end{theorem}

\begin{proof}
\begin{equation*}
A=AS=A(B_1\cup\dots\cup B_n)=AB_1\cup\dots\cup AB_n
\end{equation*}
\end{proof}

\begin{theorem}[]
设试验\(E\)的样本空间\(S\),\(A\)为\(E\)的事件,\(B_1,\dots,B_n\)为\(S\)的一
个划分,且\(P(A)>0,P(B_i)>0\),则
\begin{equation*}
P(B_i|A)=\frac{P(A|B_i)P(B_i)}{\sum_{j=1}^nP(A|B_j)P(B_j)}
\end{equation*}
称为 \textbf{贝叶斯公式}
\end{theorem}

特别的,取\(n=2\),则
\begin{gather*}
P(A)=P(A|B)P(B)+P(A|\bbar{B})P(\bbar{B})\\
P(B|A)=\frac{AB}{A}=\frac{P(A|B)P(B)}
{P(A|B)P(B)+P(A|\bbar{B})P(\bbar{B})}
\end{gather*}

\begin{examplle}[]
患肺癌的概率约为\(0.1\%\),在人群中有\(20\%\)是吸烟者,他们患肺癌的概率约为
\(0.4\%\),求不吸者患肺癌的概率

以\(C\)记事件“患肺癌”,以\(A\)记事件“吸烟”,则
\(P(C)=0.001,P(A)=0.2,P(C|A)=0.004\),由全概率公式
\begin{equation*}
P(C)=P(C|A)P(A)+P(C|\bbar{A})P(\bbar{A})
\end{equation*}
因此
\begin{equation*}
P(C|\bbar{A})=0.00025
\end{equation*}
\end{examplle}
\subsection{独立性}
\label{sec:org769250c}
\begin{definition}[]
设\(A,B\)是两事件,如果满足
\begin{equation*}
P(AB)=P(A)P(B)
\end{equation*}
则称事件\(A,B\) \textbf{相互独立} ,简称 \(A,B\) \textbf{独立}
\end{definition}

\begin{theorem}[]
设\(A,B\)是两事件,且\(P(A)>0\),若\(A,B\)相互独立,则\(P(B|A)=P(B)\)
\end{theorem}

\begin{theorem}[]
若事件\(A,B\)相互独立,则\(A\)与\(\bbar{B}\),\(\bbar{A}\)与\(B\),
\(\bbar{A}\)与\(B\)也相互独立
\end{theorem}

\begin{definition}[]
设\(A,B,C\)是三个事件,满足
\begin{equation*}
\begin{cases}
P(AB)=P(A)P(B)\\
P(BC)=P(B)P(C)\\
P(AC)=P(A)P(C)\\
P(ABC)=P(A)P(B)P(C)
\end{cases}
\end{equation*}
则称事件\(A,B,C\) \textbf{相互独立}
\end{definition}

一般地,设\(A_1,\dots,A_n\),如果对于 q 其中任意\(2,3,\dots,n\)个事件的积事件的
概率,都等于各事件概率之积,则称事件\(A_1,\dots,A_n\) \textbf{相互独立}

\begin{examplle}[]
要验收一批(100)件乐器,验收方案如下:自该批乐器中随机地取 3 件测试(相互独立),
如果 3 件中至少有一件在测试中被认为音色不纯,则这批乐器被拒绝接收。设一件音色不
纯的乐器经测试查出其音色不纯的概率为 0.95,而一件音色纯的乐器被误认为不纯的概
率为 0.01,已知 100 中有 4 件音色不纯,试问这批乐器被接收的概率是多少

设以\(H_i\)表示 3 件中恰有\(i\)件不纯,\(A\)表示这批批乐器被接收,则
\begin{align*}
&P(A|H_0)=0.99^3,P(A|H_1)=0.99^2\times 0.05\\
&P(A|H_2)=0.99\times 0.05^2,P(A|H_3)=0.05^3
\end{align*}
而
\begin{align*}
&P(H_0)=\frac{\binom{96}{3}}{\binom{100}{3}},P(H_1)=\frac{\binom{4}{1}\binom{96}{2}}{\binom{100}{3}}\\
&P(H_2)=\frac{\binom{4}{2}\binom{96}{1}}{\binom{100}{3}},
P(H_3)=\frac{\binom{4}{3}}{\binom{100}{3}}
\end{align*}
故
\begin{equation*}
P(A)=\sum P(A|H_i)P(H_i)
\end{equation*}
\end{examplle}
\section{随机变量及其分布}
\label{sec:org3d21e4b}
\subsection{随机变量}
\label{sec:orgb543e83}
\begin{definition}[]
设随机试验的样本空间为\(S=\{e\},X=X(e)\)是定义在样本空间\(S\)上的实值单值函数,
称\(X=X(e)\)为随机变量
\end{definition}
\subsection{离散型随机变量及其分布律}
\label{sec:org3f566f8}
设离散型随机变量\(X\)所有可能取的值为\(x_k(k=1,2,\dots)\),\(X\)取各个可能值
的概率,即事件\(\{X=x_k\}\)的概率,为
\begin{equation}
P\{X=x_k\}=p_k,k=1,2,\dots\label{eq2.1}
\end{equation}
由概率的定义,\(p_k\)满足如下两个条件
\begin{enumerate}
\item \(p_k\ge0,k=1,2,\dots\)
\item \(\sum_{k=1}^{\infty}p_k=1\)
\end{enumerate}


我们称 \eqref{eq2.1} 为离散型随机变量\(X\)的 \textbf{分布律} ,分布律也可以用表格表示

\begin{center}
\begin{tabular}{llllll}
\(X\) & \(x_1\) & \(x_2\) & \(\dots\) & \(x_n\) & \(\dots\)\\
\hline
\(p_k\) & \(p_1\) & \(p_2\) & \(\dots\) & \(p_n\) & \(\dots\)\\
\end{tabular}
\end{center}
\subsubsection{(0-1)分布}
\label{sec:org557b542}
设随机变量\(X\)只可能取 0 与 1 两个值,它的分布律是
\begin{equation*}
P\{X=k\}=p^k(1-p)^{1-k},k=0,1\quad(0<p<1)
\end{equation*}
则称\(X\)服从以\(p\)为参数的(0-1)分布或两点分布
\subsubsection{伯努利试验、二项分布}
\label{sec:orgb6b4577}
设试验\(E\)只有两个可能结果:\(A\)及\(\bbar{A}\),则称\(E\)为 \textbf{伯努利试验} 。
设\(P(A)=p(0<p<1)\),此时\(P(\bbar{A}=1-p)\)。将\(E\)独立重复地进行\(n\)次,
则称这一串重复的独立试验为 \textbf{\(n\)重伯努利试验}

以\(X\)表示\(n\)重伯努利事件\(A\)发生的次数,\(X\)是一个随机变量。记
\(q=1-p\),即有
\begin{equation*}
P\{X=k\}=\binom{n}{k}p^kq^{1-k}
\end{equation*}
注意到\(\binom{n}{k}p^kq^{1-k}\)刚好是\((p+q)^n\)的展开式中出现\(p^k\)的那一
项,我们称随机变量\(X\)服从参数\(n,p\)的 \textbf{二项分布} ,并记为\(X\sim b(n,p)\)
\subsubsection{泊松分布}
\label{sec:org7a950b5}
设随机变量\(X\) 所有可能的值为\(0,1,2,\dots\),而各个值的概率为
\begin{equation*}
P\{X=k\}=\frac{\lambda^ke^{-\lambda}}{k!}
\end{equation*}
其中\(\lambda>0\)是常数,则称\(X\)服从参数 \(\lambda\) 的 \textbf{泊松分布} ,记为 \(X\sim\pi(\lambda)\)

易知\(P\{X=k\}\ge0\),\(k=0,1,2,\dots\),且有
\begin{equation*}
\sum_{k=0}^\infty P\{X=k\}=\sum_{k=0}^\infty\frac{\lambda^ke^{-\lambda}}{k!}=
e^{-\lambda}\sum_{k=0}^\infty \frac{\lambda^k}{k!}=e^{-\lambda}e^\lambda=1
\end{equation*}

\begin{theorem}[泊松定理]
设\(\lambda>0\)是一个常数,\(n\)是任意正整数,设\(np_n=\lambda\),则对于任意固定的
非负整数\(k\),有
\begin{equation*}
\lim_{n\to\infty}\binom{n}{k}p_n^k(1-p_n)^{n-k}=\frac{\lambda^ke^{-\lambda}}{k!}
\end{equation*}
\end{theorem}

\begin{proof}
设\(p_n=\frac{\lambda}{n}\),有
\begin{align*}
\binom{n}{k}p_n^k(1-p_k)^{n-k}&=
\frac{n(n-1)\dots(n-k+1)}{k!}(\frac{\lambda}{n})^k(1-\frac{\lambda}{n})^{n-k}\\
&=\frac{\lambda^k}{k!}[(1-\frac{1}{n})\dots(1-\frac{k-1}{n})](1-\frac{\lambda}{n})^n(1-\frac{\lambda}{n})^{-k}\\
\end{align*}
当\(n\to\infty\)时\((1-\frac{\lambda}{n})^n\to e^{-\lambda}\),故有
\begin{equation*}
\lim_{n\to\infty}\binom{n}{k}p_n^k(1-p_n)^{n-k}=\frac{\lambda^ke^{-\lambda}}{k!}
\end{equation*}
\end{proof}
也就是说以\(n,p\)为参数的二项分布的概率值可以由参数为\(\lambda=np\)的泊松分
布的概率值近似
\subsection{随机变量的分布函数}
\label{sec:org0515ac6}
\begin{definition}[]
设\(X\)是一个随机变量,\(x\)是任意实数,函数
\begin{equation*}
F(x)=P\{X\le x\},-\infty<x<\infty
\end{equation*}
称为\(X\)的 \textbf{分布函数}
\end{definition}

对于任意实数\(x_1,x_2(x_1<x_2)\),有
\begin{align*}
P\{x_1<X\le x_2\}&=P\{X\le x_2\}-P\{X\le x_1\}\\
&=F(x_2)-F(x_1)
\end{align*}

分布函数\(F(x)\)具有以下的基本性质
\begin{enumerate}
\item \(F(x)\)是一个不减函数
\item \(0\le F(x)\le1\),且
\begin{align*}
&F(-\infty)=\lim_{x\to-\infty}F(x)=0\\
&F(\infty)=\lim_{x\to\infty}F(x)=1
\end{align*}
\item \(F(x+0)=F(x)\)
\end{enumerate}
\subsection{连续型随机变量及其概率密度}
\label{sec:org5b40594}
如果对于随机变量\(X\)的分布函数\(F(x)\),存在非负函数\(f(x)\)使对于任意实数
\(x\)有
\begin{equation*}
F(x)=\int_{-\infty}^xf(t)dt
\end{equation*}
则称\(X\)为 \textbf{连续型随机变量} ,其中函数\(f(x)\)称为\(X\)的 \textbf{概率密度函数} ,简称
\textbf{概率密度}

概率密度\(f(x)\)具有以下性质
\begin{enumerate}
\item \(f(x)\ge0\)
\item \(\int_{-\infty}^{\infty}f(x)dx=1\)
\item 对于任意实数\(x_1,x_2(x_1\le x_2)\),
\begin{equation*}
P\{x_1<X\le x_2\}=F(x_2)-F(x_1)=\int_{x_1}^{x_2}f(x)dx
\end{equation*}
\item 若\(f(x)\)在点\(x\)处连续,则有\(F'(x)=f(x)\)
\end{enumerate}
\subsubsection{均匀分布}
\label{sec:org31d4841}
若连续型随机变量\(X\)有概率密度
\begin{equation*}
f(x)=
\begin{cases}
\frac{1}{b-a}&a<x<b\\
0&
\end{cases}
\end{equation*}
则称\(X\)在区间\((a,b)\)上服从 \textbf{均匀分布} ,记为\(X\sim U(a,b)\)。分布函数
\begin{equation*}
F(x)=
\begin{cases}
0&x<a\\
\frac{x-a}{b-a}&a\le x<b\\
1&x\ge b
\end{cases}
\end{equation*}
\subsubsection{指数分布}
\label{sec:org26cba62}
若连续型随机变量\(X\)的概率密度为
\begin{equation*}
f(x)=
\begin{cases}
\frac{1}{\theta}e^{-x/\theta}&x>0\\
0&
\end{cases}
\end{equation*}
其中\(\theta>0\)为常数,则称\(X\)服从参数 \(\theta\) 的 \textbf{指数分布}
\begin{equation*}
F(x)=
\begin{cases}
1-e^{-x/\theta}&x>0\\
0&
\end{cases}
\end{equation*}
对于任意\(s,t>0\),有
\begin{equation*}
P\{X>s+t|X>s\}=P\{X>t\}
\end{equation*}
事实上
\begin{align*}
P\{X>s+t|X>s\}&=
\frac{P\{(X>s+t)\cap (X>s)\}}{P\{X>s\}}\\
&=\frac{P\{X>s+t\}}{P\{X>s\}}=\frac{1-F(s+t)}{1-F(s)}\\
&=e^{-t/\theta}=P\{X>t\}
\end{align*}
这个性质称为无记忆性
\subsubsection{正态分布}
\label{sec:org321e101}
若连续型随机变量\(X\)的概率密度为
\begin{equation*}
f(x)=\frac{1}{\sqrt{2\pi}\sigma}e^{\frac{(x-\mu)^2}{2\sigma^2}},-\infty<x<\infty
\end{equation*}
其中\(\mu,\sigma(\sigma>0)\)为常数,则称\(X\)服从参数为 $\backslash$\(\mu\),\(\sigma\) 的 \textbf{正态分布} 或 \textbf{高斯} 分布,
记为 \(X\sim N(\mu,\sigma^2)\)

令\((x-\mu)/\sigma=t\),记\(I=\int_{-\infty}^\infty e^{-t^2/2}dt\),则有
\(I^2=\int_{-\infty}^\infty\int_{-\infty}^\infty e^{-(t^2+u^2)/2}dtdu\),利
用极坐标得
\begin{equation*}
I^2=\int_0^{2\pi}\int_0^\infty re^{-r^2/2}drd\theta=2\pi
\end{equation*}

\(f(x)\)有以下性质
\begin{enumerate}
\item 曲线关于\(x=\mu\)对称
\item 当\(x=\mu\)时取得最大值
\begin{equation*}
f(\mu)=\frac{1}{\sqrt{2\pi}\sigma}
\end{equation*}
\end{enumerate}


特别的,当\(\mu=0,\sigma=1\)时称随机变量\(X\)服从 \textbf{标准正态分布} ,其概率密度
和分布函数分别用\(\varphi(x),\Phi(x)\)表示,即
\begin{gather*}
\varphi(x)=\frac{1}{\sqrt{2\pi}}e^{-t^2/2}\\
\Phi(x)=\frac{1}{\sqrt{2\pi}}\int_{-\infty}^xe^{-t^2/2}dt
\end{gather*}
易知
\begin{equation*}
\Phi(-x)=1-\Phi(x)
\end{equation*}

\begin{lemma}[]
若\(X\sim N(\mu,\sigma^2)\),则\(Z=\frac{X-\mu}{\sigma}\sim N(0,1)\)
\end{lemma}
于是,若\(X\sim N(\mu,\sigma^2)\),则
\begin{equation*}
F(x)=P\{X\le x\}=P\{\frac{X-\mu}{\sigma}\le\frac{x-\mu}{\sigma}\}=\Phi(\frac{x-\mu}{\sigma})
\end{equation*}
\subsection{随机变量的函数的分布}
\label{sec:org59ac4ca}
\begin{theorem}[]
设随机变量\(X\)具有概率密度\(f_X(x),-\infty<x<\infty\),又设函数\(g(x)\)处处
可导且恒有\(g'(x)>0\)(或恒有\(g'(x)<0\)),则\(Y=g(x)\)是连续型随机变量,其
概率密度为
\begin{equation*}
f_Y(y)=
\begin{cases}
f_X[h(y)]\abs{h'(y)}&\alpha<y<\beta\\
0&
\end{cases}
\end{equation*}
其中
\(\alpha=\min\{g(-\infty),g(\infty)\},\beta=\max\{g(-\infty),g(\infty)\}\),
\(h(y)\)是\(g(x)\)的反函数
\end{theorem}

\begin{proposition}[]
设随机变量\(X\sim N(\mu,\sigma^2)\),试证明\(X\)的线性函数\(Y=aX+b(a\neq0)\)也服从正
态分布
\end{proposition}

\begin{proof}
\(X\)的概率密度为
\begin{equation*}
f_X(x)=\frac{1}{\sqrt{2\pi}\sigma}e^{-\frac{(x-\mu)^2}{2\sigma^2}},-\infty<x<\infty
\end{equation*}
由\(Y=aX+b\)
\begin{equation*}
x=\frac{y-b}{a},h'(y)=\frac{1}{a}
\end{equation*}
因此
\begin{align*}
f_Y(y)&=\frac{1}{\abs{a}}f_X(\frac{y-b}{a}),-\infty<y<\infty\\
&=\frac{1}{\abs{a}\sigma\sqrt{2\pi}}e^{
-\frac{(\frac{y-b}{a}-\mu)^2}{2(a\sigma)^2}}
\end{align*}
即有 \(Y=aX+B\sim N(a\mu+b,(a\sigma)^2)\)
\end{proof}
\section{多维随机变量及其分布}
\label{sec:org6e6468f}
\subsection{二维随机变量}
\label{sec:org1a3f656}
设\(E\)是一个随机试验,它 的样本空间是\(S=\{e\}\),设\(X=X(e),Y=Y(e)\)是定义
在\(S\)上的随机变量,它们构成的一个向量\((X,Y)\)叫做 \textbf{二维随机向量} 或 \textbf{二维随机
变量}

\begin{definition}[]
设\((X,Y)\)是二维随机变量,对于任意实数\(x,y\),二元函数
\begin{equation*}
F(x,y)=P\{(X\le x)\cap(Y\le y)\}(\text{written as }P\{X\le x,Y\le y\})
\end{equation*}
称为二维随机变量的 \textbf{分布函数} ,或称为随机变量\(X,Y\)的 \textbf{联合分布函数}
\end{definition}

分布函数\(F(x,y)\)具有以下性质
\begin{enumerate}
\item \(F(x,y)\)是变量\(x,y\)的不减函数
\item \(0\le F(x,y)\le 1\),且
\begin{align*}
&\forall y,F(-\infty,y)=0\\
&\forall x,F(x,-\infty)=0\\
&F(-\infty,-\infty)=0,F(\infty,\infty)=1
\end{align*}
\item \(F(x+0,y)=F(x,y),F(x,y+0)=F(x,y)\),即\(F(x,y)\)关于\(x\)又连续,关于
\(y\)也右连续
\item 对于任意\((x_1,y_1),(x_2,y_2),x_1<x_2,y_1<y_2\),下列不等式成立
\begin{equation*}
F(x_2,y_2)-F(x_2,y_1)+F(x_1,y_1)-F(x_1,y_2)\ge0
\end{equation*}
\end{enumerate}


如果二维随机变量\((X,Y)\)全部可能取到的值是有限对或可列无限多对,则称
\((X,Y)\)是 \textbf{离散型的随机变量}

设二维离散型随机变量\((X,Y)\)所有可能取的值为\((x_i,y_j),i,j=1,2,\dots\),
记\(P\{X=x_i,Y=y_j\}=p_{ij}\),称为二维离散型随机变量\((X,Y)\)的 \textbf{分布律} ,或
随机变量\(X,Y\)的 \textbf{联合分布律}

\begin{center}
\begin{tabular}{c|cccc}
\(Y,X\) & \(x_1\) & \(\dots\) & \(x_i\) & \(\dots\)\\
\hline
\(y_1\) & \(p_{11}\) & \(\dots\) & \(p_{i1}\) & \(\dots\)\\
\(\vdots\) &  &  &  & \\
\(y_j\) & \(p_{1j}\) &  & \(p_{ij}\) & \\
\end{tabular}
\end{center}

对于二维随机变量\((X,Y)\)的分布函数\(F(x,y)\),如果存在非负的函数\(f(x,y)\)使
对于任意\(x,y\)有
\begin{equation*}
F(x,y)=\int_{-\infty}^y\int_{-\infty}^xf(u,v)dudv
\end{equation*}
则称\((X,Y)\)是 \textbf{连续型的二维随机变量} ,函数\(f(x,y)\)称为二维随机变量的 \textbf{概率
密度} ,或称为随机变量\(X,Y\)的 \textbf{联合概率密度}

概率密度\(f(x,y)\)具有以下性质
\begin{enumerate}
\item \(f(x,y)\ge0\)
\item \(\int_{-\infty}^\infty\int_{-\infty}^\infty f(x,y)dxdy=1\)
\item 设\(G\)是\(xOy\)平面的区域,点\((X,Y)\)落在\(G\)内的概率为
\begin{equation*}
P\{(X,Y)\in G\}=\iint_Gf(x,y)dxdy
\end{equation*}
\item 若\(f(x,y)\)在点\((x,y)\)连续,则有
\begin{equation*}
\frac{\partial ^2F(x,y)}{\partial x\partial y}=f(x,y)
\end{equation*}
\end{enumerate}
\subsection{边缘分布}
\label{sec:org2a7da19}
\(F_X(x),F_Y(y)\)称为二维随机变量\((X,Y)\)关于\(X,Y\)的 \textbf{边缘分布函数}
\begin{equation*}
F_x(x)=P\{X\le x\}=P\{X\le x,Y<\infty\}=F(x,\infty)
\end{equation*}

记
\begin{align*}
&p_{i\cdot}\sum_{j=1}^\infty p_{ij}=P\{X=x_i\}\\
&p_{\cdot j}\sum_{i=1}^\infty p_{ij}=P\{X=y_j\}\\
\end{align*}
分别称为\(p_{i\cdot}\)和\(p_{\cdot j}\)为\((X,Y)\)关于\(X,Y\)的 \textbf{边缘分布律}


\begin{align*}
&f_X(x)=\int_{-\infty}^\infty f(x,y)dy\\
&f_Y(y)=\int_{-\infty}^\infty f(x,y)dx\\
\end{align*}
分别为 \(X,Y\)的 \textbf{边缘概率密度}
\begin{examplle}[]
设二维随机变量\((X,Y)\)的概率密度
\begin{align*}
f(x,y)=&
\frac{1}{2\pi\sigma_1\sigma_2\sqrt{1-\rho^2}}\exp\left\{
\frac{-1}{2(1-\rho^2)}\left[
\frac{(x-\mu_1)^2}{\sigma_1^2}\right.\right.\\
&\left.\left.
-2\rho\frac{(x-\mu_1)(x-\mu_2)}{\sigma_1\sigma_2}+
\frac{(y-\mu_2)^2}{\sigma_2^2}
\right]
\right\}
\end{align*}
其中\(\mu_1,\mu_2,\sigma_1,\sigma_2,\rho\)都是常数,且
\(\sigma_1,\sigma_2>0,-1<\rho<1\),我们称\((X,Y)\)服从参数为
\(\mu_1,\mu_2,\sigma_1,\sigma_2,\rho\)的 \textbf{二维正态分布} ,记为
\((X,Y)\sim N(\mu_1,\mu_2,\sigma_1^2,\sigma_2^2,\rho)\),试求二维正态分布随机变量
的边缘概率密度

令
\(t=\frac{1}{\sqrt{1-\rho^2}}(\frac{y-\mu_2}{\sigma_2}-\rho\frac{x-\mu_1}{\sigma_1})\)
,则有
\begin{equation*}
f_X(x)=\frac{1}{2\pi\sigma^1}e^{-\frac{(x-\mu_1)^2}{2\sigma_1^2}}
\int_{-\infty}^\infty e^{-\frac{t^2}{2}}dt
\end{equation*}
即
\begin{equation*}
f_X(x)=\frac{1}{\sqrt{2\pi}\sigma_1}e^{-\frac{(x-\mu_1)^2}{2\sigma_1^2}}
\end{equation*}
\end{examplle}
\subsection{条件分布}
\label{sec:org749796e}
\begin{definition}[]
设\((X,Y)\)是二维离散型随机变量,对于固定的\(j\),若\(P\{Y=y_j\}>0\),则称
\begin{equation*}
P\{X=x_i|Y=y_j\}=\frac{P\{X=x_i,Y=y_j\}}{P\{Y=y_j\}}=\frac{p_{ij}}{p_{\cdot j}}
\end{equation*}
为在\(Y=y_j\)条件下随机变量 \(X\)的 \textbf{条件分布律}
\end{definition}

\begin{definition}[]
设二维随机变量\((X,Y)\)的概率密度为\(f(x,y)\),\((X,Y)\)关于\(Y\)的边缘概率密
度为\(f_Y(y)\),若对于固定的\(y\),\(f_Y(y)>0\),则称
\(\frac{f(x,y)}{f_Y(y)}\)为在\(Y=y\)的条件下\(X\)的 \textbf{条件概率密度} ,记为
\begin{equation*}
 f_{X|Y}(x|y)=\frac{f(x,y)}{f_Y(y)}
\end{equation*}
称\(F_{X|Y}(x|y)=\int_{-\infty}^xf_{X|Y}(x|y)dx=\int_{-\infty}^x\frac{f(x,y)}{f_Y(y)}dx\)
为在\(Y=y\)下\(X\)的条件分布函数
\end{definition}
\subsection{相互独立的随机变量}
\label{sec:org0f52590}
\begin{definition}[]
设\(F(x,y)\)及\(F_X(x),F_Y(y)\)分别是二维随机变量\((X,Y)\)的分布函数及边缘分
布函数,若对于所有\(x,y\)有
\begin{gather*}
P\{X\le x,Y\le y\}=P\{X\le x\}P\{Y\le y\}\\
F(x,y)=F_X(x)F_Y(y)\\
f(x,y)=f_X(x)f_Y(y)
\end{gather*}
则称\(X,Y\)是 \textbf{相互独立的}
\end{definition}

下面考查二维正态随机变量\((X,Y)\),它的概率密度为
   \begin{align*}
f(x,y)=&
\frac{1}{2\pi\sigma_1\sigma_2\sqrt{1-\rho^2}}\exp\left\{
\frac{-1}{2(1-\rho^2)}\left[
\frac{(x-\mu_1)^2}{\sigma_1^2}\right.\right.\\
&\left.\left.
-2\rho\frac{(x-\mu_1)(x-\mu_2)}{\sigma_1\sigma_2}+
\frac{(y-\mu_2)^2}{\sigma_2^2}
\right]
\right\}
\end{align*}
如果\(\rho=0\)则对于所有\(x,y\),\(f(x,y)=f_X(x)f_Y(y)\)。如果\(X,Y\)相互独立,
令\(x=\mu_1,y=\mu_2\),则\(\rho=0\)

对于二维正态随机变量\((X,Y)\), \(X,Y\)相互独立的充要条件是参数\(\rho=0\)

\begin{theorem}[]
设\((X_1,\dots,X_m)\)和\((Y_1,\dots,Y_n)\)相互独立,则\(X_i\)和\(Y_j\)相互独
立。又若\(h,g\)是连续函数,则\(h(X_1,\dots,X_m)\)和\(g(Y_1,\dots,Y_n)\)相互独
立
\end{theorem}
\subsection{两个随机变量的函数的分布}
\label{sec:orge3f7ea8}
\end{document}
