% Created 2021-09-08 Wed 23:31
% Intended LaTeX compiler: pdflatex
\documentclass[11pt]{article}
\usepackage[utf8]{inputenc}
\usepackage[T1]{fontenc}
\usepackage{graphicx}
\usepackage{grffile}
\usepackage{longtable}
\usepackage{wrapfig}
\usepackage{rotating}
\usepackage[normalem]{ulem}
\usepackage{amsmath}
\usepackage{textcomp}
\usepackage{amssymb}
\usepackage{capt-of}
\usepackage{hyperref}
\graphicspath{{../../books/}}
% wrong resolution of image
% https://tex.stackexchange.com/questions/21627/image-from-includegraphics-showing-in-wrong-image-size?rq=1

%%%%%%%%%%%%%%%%%%%%%%%%%%%%%%%%%%%%%%
%% TIPS                                 %%
%%%%%%%%%%%%%%%%%%%%%%%%%%%%%%%%%%%%%%
% \substack{a\\b} for multiple lines text
% \usepackage{expl3}
% \expandafter\def\csname ver@l3regex.sty\endcsname{}
% \usepackage{pkgloader}
\usepackage[utf8]{inputenc}

% nfss error
% \usepackage[B1,T1]{fontenc}
\usepackage{fontspec}

% \usepackage[Emoticons]{ucharclasses}
\newfontfamily\DejaSans{DejaVu Sans}
% \setDefaultTransitions{\DejaSans}{}

% pdfplots will load xolor automatically without option
\usepackage[dvipsnames]{xcolor}

%                                                             ┳┳┓   ┓
%                                                             ┃┃┃┏┓╋┣┓
%                                                             ┛ ┗┗┻┗┛┗
% \usepackage{amsmath} mathtools loads the amsmath
\usepackage{amsmath}
\usepackage{mathtools}

\usepackage{amsthm}
\usepackage{amsbsy}

%\usepackage{commath}

\usepackage{amssymb}

\usepackage{mathrsfs}
%\usepackage{mathabx}
\usepackage{stmaryrd}
\usepackage{empheq}

\usepackage{scalerel}
\usepackage{stackengine}
\usepackage{stackrel}



\usepackage{nicematrix}
\usepackage{tensor}
\usepackage{blkarray}
\usepackage{siunitx}
\usepackage[f]{esvect}

% centering \not on a letter
\usepackage{slashed}
\usepackage[makeroom]{cancel}

%\usepackage{merriweather}
\usepackage{unicode-math}
\setmainfont{TeX Gyre Pagella}
% \setmathfont{STIX}
%\setmathfont{texgyrepagella-math.otf}
%\setmathfont{Libertinus Math}
\setmathfont{Latin Modern Math}

 % \setmathfont[range={\smwhtdiamond,\enclosediamond,\varlrtriangle}]{Latin Modern Math}
\setmathfont[range={\rightrightarrows,\twoheadrightarrow,\leftrightsquigarrow,\triangledown,\vartriangle,\precneq,\succneq,\prec,\succ,\preceq,\succeq,\tieconcat}]{XITS Math}
 \setmathfont[range={\int,\setminus}]{Libertinus Math}
 % \setmathfont[range={\mathalpha}]{TeX Gyre Pagella Math}
%\setmathfont[range={\mitA,\mitB,\mitC,\mitD,\mitE,\mitF,\mitG,\mitH,\mitI,\mitJ,\mitK,\mitL,\mitM,\mitN,\mitO,\mitP,\mitQ,\mitR,\mitS,\mitT,\mitU,\mitV,\mitW,\mitX,\mitY,\mitZ,\mita,\mitb,\mitc,\mitd,\mite,\mitf,\mitg,\miti,\mitj,\mitk,\mitl,\mitm,\mitn,\mito,\mitp,\mitq,\mitr,\mits,\mitt,\mitu,\mitv,\mitw,\mitx,\mity,\mitz}]{TeX Gyre Pagella Math}
% unicode is not good at this!
%\let\nmodels\nvDash

 \usepackage{wasysym}

 % for wide hat
 \DeclareSymbolFont{yhlargesymbols}{OMX}{yhex}{m}{n} \DeclareMathAccent{\what}{\mathord}{yhlargesymbols}{"62}

%                                                               ┏┳┓•┓
%                                                                ┃ ┓┃┏┓
%                                                                ┻ ┗┛┗┗

\usepackage{pgfplots}
\pgfplotsset{compat=1.18}
\usepackage{tikz}
\usepackage{tikz-cd}
\tikzcdset{scale cd/.style={every label/.append style={scale=#1},
    cells={nodes={scale=#1}}}}
% TODO: discard qtree and use forest
% \usepackage{tikz-qtree}
\usepackage{forest}

\usetikzlibrary{arrows,positioning,calc,fadings,decorations,matrix,decorations,shapes.misc}
%setting from geogebra
\definecolor{ccqqqq}{rgb}{0.8,0,0}

%                                                          ┳┳┓•    ┓┓
%                                                          ┃┃┃┓┏┏┏┓┃┃┏┓┏┓┏┓┏┓┓┏┏
%                                                          ┛ ┗┗┛┗┗ ┗┗┗┻┛┗┗ ┗┛┗┻┛
%\usepackage{twemojis}
\usepackage[most]{tcolorbox}
\usepackage{threeparttable}
\usepackage{tabularx}

\usepackage{enumitem}
\usepackage[indLines=false]{algpseudocodex}
\usepackage[]{algorithm2e}
% \SetKwComment{Comment}{/* }{ */}
% \algrenewcommand\algorithmicrequire{\textbf{Input:}}
% \algrenewcommand\algorithmicensure{\textbf{Output:}}
% wrong with preview
\usepackage{subcaption}
\usepackage{caption}
% {\aunclfamily\Huge}
\usepackage{auncial}

\usepackage{float}

\usepackage{fancyhdr}

\usepackage{ifthen}
\usepackage{xargs}

\definecolor{mintedbg}{rgb}{0.99,0.99,0.99}
\usepackage[cachedir=\detokenize{~/miscellaneous/trash}]{minted}
\setminted{breaklines,
  mathescape,
  bgcolor=mintedbg,
  fontsize=\footnotesize,
  frame=single,
  linenos}
\usemintedstyle{xcode}
\usepackage{tcolorbox}
\usepackage{etoolbox}



\usepackage{imakeidx}
\usepackage{hyperref}
\usepackage{soul}
\usepackage{framed}

% don't use this for preview
%\usepackage[margin=1.5in]{geometry}
% \usepackage{geometry}
% \geometry{legalpaper, landscape, margin=1in}
\usepackage[font=itshape]{quoting}

%\LoadPackagesNow
%\usepackage[xetex]{preview}
%%%%%%%%%%%%%%%%%%%%%%%%%%%%%%%%%%%%%%%
%% USEPACKAGES end                       %%
%%%%%%%%%%%%%%%%%%%%%%%%%%%%%%%%%%%%%%%

%%%%%%%%%%%%%%%%%%%%%%%%%%%%%%%%%%%%%%%
%% Algorithm environment
%%%%%%%%%%%%%%%%%%%%%%%%%%%%%%%%%%%%%%%
\SetKwIF{Recv}{}{}{upon receiving}{do}{}{}{}
\SetKwBlock{Init}{initially do}{}
\SetKwProg{Function}{Function}{:}{}

% https://github.com/chrmatt/algpseudocodex/issues/3
\algnewcommand\algorithmicswitch{\textbf{switch}}%
\algnewcommand\algorithmiccase{\textbf{case}}
\algnewcommand\algorithmicof{\textbf{of}}
\algnewcommand\algorithmicotherwise{\texttt{otherwise} $\Rightarrow$}

\makeatletter
\algdef{SE}[SWITCH]{Switch}{EndSwitch}[1]{\algpx@startIndent\algpx@startCodeCommand\algorithmicswitch\ #1\ \algorithmicdo}{\algpx@endIndent\algpx@startCodeCommand\algorithmicend\ \algorithmicswitch}%
\algdef{SE}[CASE]{Case}{EndCase}[1]{\algpx@startIndent\algpx@startCodeCommand\algorithmiccase\ #1}{\algpx@endIndent\algpx@startCodeCommand\algorithmicend\ \algorithmiccase}%
\algdef{SE}[CASEOF]{CaseOf}{EndCaseOf}[1]{\algpx@startIndent\algpx@startCodeCommand\algorithmiccase\ #1 \algorithmicof}{\algpx@endIndent\algpx@startCodeCommand\algorithmicend\ \algorithmiccase}
\algdef{SE}[OTHERWISE]{Otherwise}{EndOtherwise}[0]{\algpx@startIndent\algpx@startCodeCommand\algorithmicotherwise}{\algpx@endIndent\algpx@startCodeCommand\algorithmicend\ \algorithmicotherwise}
\ifbool{algpx@noEnd}{%
  \algtext*{EndSwitch}%
  \algtext*{EndCase}%
  \algtext*{EndCaseOf}
  \algtext*{EndOtherwise}
  %
  % end indent line after (not before), to get correct y position for multiline text in last command
  \apptocmd{\EndSwitch}{\algpx@endIndent}{}{}%
  \apptocmd{\EndCase}{\algpx@endIndent}{}{}%
  \apptocmd{\EndCaseOf}{\algpx@endIndent}{}{}
  \apptocmd{\EndOtherwise}{\algpx@endIndent}{}{}
}{}%

\pretocmd{\Switch}{\algpx@endCodeCommand}{}{}
\pretocmd{\Case}{\algpx@endCodeCommand}{}{}
\pretocmd{\CaseOf}{\algpx@endCodeCommand}{}{}
\pretocmd{\Otherwise}{\algpx@endCodeCommand}{}{}

% for end commands that may not be printed, tell endCodeCommand whether we are using noEnd
\ifbool{algpx@noEnd}{%
  \pretocmd{\EndSwitch}{\algpx@endCodeCommand[1]}{}{}%
  \pretocmd{\EndCase}{\algpx@endCodeCommand[1]}{}{}
  \pretocmd{\EndCaseOf}{\algpx@endCodeCommand[1]}{}{}%
  \pretocmd{\EndOtherwise}{\algpx@endCodeCommand[1]}{}{}
}{%
  \pretocmd{\EndSwitch}{\algpx@endCodeCommand[0]}{}{}%
  \pretocmd{\EndCase}{\algpx@endCodeCommand[0]}{}{}%
  \pretocmd{\EndCaseOf}{\algpx@endCodeCommand[0]}{}{}
  \pretocmd{\EndOtherwise}{\algpx@endCodeCommand[0]}{}{}
}%
\makeatother
% % For algpseudocode
% \algnewcommand\algorithmicswitch{\textbf{switch}}
% \algnewcommand\algorithmiccase{\textbf{case}}
% \algnewcommand\algorithmiccaseof{\textbf{case}}
% \algnewcommand\algorithmicof{\textbf{of}}
% % New "environments"
% \algdef{SE}[SWITCH]{Switch}{EndSwitch}[1]{\algorithmicswitch\ #1\ \algorithmicdo}{\algorithmicend\ \algorithmicswitch}%
% \algdef{SE}[CASE]{Case}{EndCase}[1]{\algorithmiccase\ #1}{\algorithmicend\ \algorithmiccase}%
% \algtext*{EndSwitch}%
% \algtext*{EndCase}
% \algdef{SE}[CASEOF]{CaseOf}{EndCaseOf}[1]{\algorithmiccaseof\ #1 \algorithmicof}{\algorithmicend\ \algorithmiccaseof}
% \algtext*{EndCaseOf}



%\pdfcompresslevel0

% quoting from
% https://tex.stackexchange.com/questions/391726/the-quotation-environment
\NewDocumentCommand{\bywhom}{m}{% the Bourbaki trick
  {\nobreak\hfill\penalty50\hskip1em\null\nobreak
   \hfill\mbox{\normalfont(#1)}%
   \parfillskip=0pt \finalhyphendemerits=0 \par}%
}

\NewDocumentEnvironment{pquotation}{m}
  {\begin{quoting}[
     indentfirst=true,
     leftmargin=\parindent,
     rightmargin=\parindent]\itshape}
  {\bywhom{#1}\end{quoting}}

\indexsetup{othercode=\small}
\makeindex[columns=2,options={-s /media/wu/file/stuuudy/notes/index_style.ist},intoc]
\makeatletter
\def\@idxitem{\par\hangindent 0pt}
\makeatother


% \newcounter{dummy} \numberwithin{dummy}{section}
\newtheorem{dummy}{dummy}[section]
\theoremstyle{definition}
\newtheorem{definition}[dummy]{Definition}
\theoremstyle{plain}
\newtheorem{corollary}[dummy]{Corollary}
\newtheorem{lemma}[dummy]{Lemma}
\newtheorem{proposition}[dummy]{Proposition}
\newtheorem{theorem}[dummy]{Theorem}
\newtheorem{notation}[dummy]{Notation}
\newtheorem{conjecture}[dummy]{Conjecture}
\newtheorem{fact}[dummy]{Fact}
\newtheorem{warning}[dummy]{Warning}
\theoremstyle{definition}
\newtheorem{examplle}{Example}[section]
\theoremstyle{remark}
\newtheorem*{remark}{Remark}
\newtheorem{exercise}{Exercise}[subsection]
\newtheorem{problem}{Problem}[subsection]
\newtheorem{observation}{Observation}[section]
\newenvironment{claim}[1]{\par\noindent\textbf{Claim:}\space#1}{}

\makeatletter
\DeclareFontFamily{U}{tipa}{}
\DeclareFontShape{U}{tipa}{m}{n}{<->tipa10}{}
\newcommand{\arc@char}{{\usefont{U}{tipa}{m}{n}\symbol{62}}}%

\newcommand{\arc}[1]{\mathpalette\arc@arc{#1}}

\newcommand{\arc@arc}[2]{%
  \sbox0{$\m@th#1#2$}%
  \vbox{
    \hbox{\resizebox{\wd0}{\height}{\arc@char}}
    \nointerlineskip
    \box0
  }%
}
\makeatother

\setcounter{MaxMatrixCols}{20}
%%%%%%% ABS
\DeclarePairedDelimiter\abss{\lvert}{\rvert}%
\DeclarePairedDelimiter\normm{\lVert}{\rVert}%

% Swap the definition of \abs* and \norm*, so that \abs
% and \norm resizes the size of the brackets, and the
% starred version does not.
\makeatletter
\let\oldabs\abss
%\def\abs{\@ifstar{\oldabs}{\oldabs*}}
\newcommand{\abs}{\@ifstar{\oldabs}{\oldabs*}}
\newcommand{\norm}[1]{\left\lVert#1\right\rVert}
%\let\oldnorm\normm
%\def\norm{\@ifstar{\oldnorm}{\oldnorm*}}
%\renewcommand{norm}{\@ifstar{\oldnorm}{\oldnorm*}}
\makeatother

% \stackMath
% \newcommand\what[1]{%
% \savestack{\tmpbox}{\stretchto{%
%   \scaleto{%
%     \scalerel*[\widthof{\ensuremath{#1}}]{\kern-.6pt\bigwedge\kern-.6pt}%
%     {\rule[-\textheight/2]{1ex}{\textheight}}%WIDTH-LIMITED BIG WEDGE
%   }{\textheight}%
% }{0.5ex}}%
% \stackon[1pt]{#1}{\tmpbox}%
% }

% \newcommand\what[1]{\ThisStyle{%
%     \setbox0=\hbox{$\SavedStyle#1$}%
%     \stackengine{-1.0\ht0+.5pt}{$\SavedStyle#1$}{%
%       \stretchto{\scaleto{\SavedStyle\mkern.15mu\char'136}{2.6\wd0}}{1.4\ht0}%
%     }{O}{c}{F}{T}{S}%
%   }
% }

% \newcommand\wtilde[1]{\ThisStyle{%
%     \setbox0=\hbox{$\SavedStyle#1$}%
%     \stackengine{-.1\LMpt}{$\SavedStyle#1$}{%
%       \stretchto{\scaleto{\SavedStyle\mkern.2mu\AC}{.5150\wd0}}{.6\ht0}%
%     }{O}{c}{F}{T}{S}%
%   }
% }

% \newcommand\wbar[1]{\ThisStyle{%
%     \setbox0=\hbox{$\SavedStyle#1$}%
%     \stackengine{.5pt+\LMpt}{$\SavedStyle#1$}{%
%       \rule{\wd0}{\dimexpr.3\LMpt+.3pt}%
%     }{O}{c}{F}{T}{S}%
%   }
% }

\newcommand{\bl}[1] {\boldsymbol{#1}}
\newcommand{\Wt}[1] {\stackrel{\sim}{\smash{#1}\rule{0pt}{1.1ex}}}
\newcommand{\wt}[1] {\widetilde{#1}}
\newcommand{\tf}[1] {\textbf{#1}}

\newcommand{\wu}[1]{{\color{red} #1}}

%For boxed texts in align, use Aboxed{}
%otherwise use boxed{}

\DeclareMathSymbol{\widehatsym}{\mathord}{largesymbols}{"62}
\newcommand\lowerwidehatsym{%
  \text{\smash{\raisebox{-1.3ex}{%
    $\widehatsym$}}}}
\newcommand\fixwidehat[1]{%
  \mathchoice
    {\accentset{\displaystyle\lowerwidehatsym}{#1}}
    {\accentset{\textstyle\lowerwidehatsym}{#1}}
    {\accentset{\scriptstyle\lowerwidehatsym}{#1}}
    {\accentset{\scriptscriptstyle\lowerwidehatsym}{#1}}
  }


\newcommand{\cupdot}{\mathbin{\dot{\cup}}}
\newcommand{\bigcupdot}{\mathop{\dot{\bigcup}}}

\usepackage{graphicx}

\usepackage[toc,page]{appendix}

% text on arrow for xRightarrow
\makeatletter
%\newcommand{\xRightarrow}[2][]{\ext@arrow 0359\Rightarrowfill@{#1}{#2}}
\makeatother

% Arbitrary long arrow
\newcommand{\Rarrow}[1]{%
\parbox{#1}{\tikz{\draw[->](0,0)--(#1,0);}}
}

\newcommand{\LRarrow}[1]{%
\parbox{#1}{\tikz{\draw[<->](0,0)--(#1,0);}}
}


\makeatletter
\providecommand*{\rmodels}{%
  \mathrel{%
    \mathpalette\@rmodels\models
  }%
}
\newcommand*{\@rmodels}[2]{%
  \reflectbox{$\m@th#1#2$}%
}
\makeatother

% Roman numerals
\makeatletter
\newcommand*{\rom}[1]{\expandafter\@slowromancap\romannumeral #1@}
\makeatother
% \\def \\b\([a-zA-Z]\) {\\boldsymbol{[a-zA-z]}}
% \\DeclareMathOperator{\\b\1}{\\textbf{\1}}

\DeclareMathOperator*{\argmin}{arg\,min}
\DeclareMathOperator*{\argmax}{arg\,max}

\DeclareMathOperator{\bone}{\textbf{1}}
\DeclareMathOperator{\bx}{\textbf{x}}
\DeclareMathOperator{\bz}{\textbf{z}}
\DeclareMathOperator{\bff}{\textbf{f}}
\DeclareMathOperator{\ba}{\textbf{a}}
\DeclareMathOperator{\bk}{\textbf{k}}
\DeclareMathOperator{\bs}{\textbf{s}}
\DeclareMathOperator{\bh}{\textbf{h}}
\DeclareMathOperator{\bc}{\textbf{c}}
\DeclareMathOperator{\br}{\textbf{r}}
\DeclareMathOperator{\bi}{\textbf{i}}
\DeclareMathOperator{\bj}{\textbf{j}}
\DeclareMathOperator{\bn}{\textbf{n}}
\DeclareMathOperator{\be}{\textbf{e}}
\DeclareMathOperator{\bo}{\textbf{o}}
\DeclareMathOperator{\bU}{\textbf{U}}
\DeclareMathOperator{\bL}{\textbf{L}}
\DeclareMathOperator{\bV}{\textbf{V}}
\def \bzero {\mathbf{0}}
\def \bbone {\mathbb{1}}
\def \btwo {\mathbf{2}}
\DeclareMathOperator{\bv}{\textbf{v}}
\DeclareMathOperator{\bp}{\textbf{p}}
\DeclareMathOperator{\bI}{\textbf{I}}
\def \dbI {\dot{\bI}}
\DeclareMathOperator{\bM}{\textbf{M}}
\DeclareMathOperator{\bN}{\textbf{N}}
\DeclareMathOperator{\bK}{\textbf{K}}
\DeclareMathOperator{\bt}{\textbf{t}}
\DeclareMathOperator{\bb}{\textbf{b}}
\DeclareMathOperator{\bA}{\textbf{A}}
\DeclareMathOperator{\bX}{\textbf{X}}
\DeclareMathOperator{\bu}{\textbf{u}}
\DeclareMathOperator{\bS}{\textbf{S}}
\DeclareMathOperator{\bZ}{\textbf{Z}}
\DeclareMathOperator{\bJ}{\textbf{J}}
\DeclareMathOperator{\by}{\textbf{y}}
\DeclareMathOperator{\bw}{\textbf{w}}
\DeclareMathOperator{\bT}{\textbf{T}}
\DeclareMathOperator{\bF}{\textbf{F}}
\DeclareMathOperator{\bmm}{\textbf{m}}
\DeclareMathOperator{\bW}{\textbf{W}}
\DeclareMathOperator{\bR}{\textbf{R}}
\DeclareMathOperator{\bC}{\textbf{C}}
\DeclareMathOperator{\bD}{\textbf{D}}
\DeclareMathOperator{\bE}{\textbf{E}}
\DeclareMathOperator{\bQ}{\textbf{Q}}
\DeclareMathOperator{\bP}{\textbf{P}}
\DeclareMathOperator{\bY}{\textbf{Y}}
\DeclareMathOperator{\bH}{\textbf{H}}
\DeclareMathOperator{\bB}{\textbf{B}}
\DeclareMathOperator{\bG}{\textbf{G}}
\def \blambda {\symbf{\lambda}}
\def \boldeta {\symbf{\eta}}
\def \balpha {\symbf{\alpha}}
\def \btau {\symbf{\tau}}
\def \bbeta {\symbf{\beta}}
\def \bgamma {\symbf{\gamma}}
\def \bxi {\symbf{\xi}}
\def \bLambda {\symbf{\Lambda}}
\def \bGamma {\symbf{\Gamma}}

\newcommand{\bto}{{\boldsymbol{\to}}}
\newcommand{\Ra}{\Rightarrow}
\newcommand{\xrsa}[1]{\overset{#1}{\rightsquigarrow}}
\newcommand{\xlsa}[1]{\overset{#1}{\leftsquigarrow}}
\newcommand\und[1]{\underline{#1}}
\newcommand\ove[1]{\overline{#1}}
%\def \concat {\verb|^|}
\def \bPhi {\mbfPhi}
\def \btheta {\mbftheta}
\def \bTheta {\mbfTheta}
\def \bmu {\mbfmu}
\def \bphi {\mbfphi}
\def \bSigma {\mbfSigma}
\def \la {\langle}
\def \ra {\rangle}

\def \caln {\mathcal{N}}
\def \dissum {\displaystyle\Sigma}
\def \dispro {\displaystyle\prod}

\def \caret {\verb!^!}

\def \A {\mathbb{A}}
\def \B {\mathbb{B}}
\def \C {\mathbb{C}}
\def \D {\mathbb{D}}
\def \E {\mathbb{E}}
\def \F {\mathbb{F}}
\def \G {\mathbb{G}}
\def \H {\mathbb{H}}
\def \I {\mathbb{I}}
\def \J {\mathbb{J}}
\def \K {\mathbb{K}}
\def \L {\mathbb{L}}
\def \M {\mathbb{M}}
\def \N {\mathbb{N}}
\def \O {\mathbb{O}}
\def \P {\mathbb{P}}
\def \Q {\mathbb{Q}}
\def \R {\mathbb{R}}
\def \S {\mathbb{S}}
\def \T {\mathbb{T}}
\def \U {\mathbb{U}}
\def \V {\mathbb{V}}
\def \W {\mathbb{W}}
\def \X {\mathbb{X}}
\def \Y {\mathbb{Y}}
\def \Z {\mathbb{Z}}

\def \cala {\mathcal{A}}
\def \cale {\mathcal{E}}
\def \calb {\mathcal{B}}
\def \calq {\mathcal{Q}}
\def \calp {\mathcal{P}}
\def \cals {\mathcal{S}}
\def \calx {\mathcal{X}}
\def \caly {\mathcal{Y}}
\def \calg {\mathcal{G}}
\def \cald {\mathcal{D}}
\def \caln {\mathcal{N}}
\def \calr {\mathcal{R}}
\def \calt {\mathcal{T}}
\def \calm {\mathcal{M}}
\def \calw {\mathcal{W}}
\def \calc {\mathcal{C}}
\def \calv {\mathcal{V}}
\def \calf {\mathcal{F}}
\def \calk {\mathcal{K}}
\def \call {\mathcal{L}}
\def \calu {\mathcal{U}}
\def \calo {\mathcal{O}}
\def \calh {\mathcal{H}}
\def \cali {\mathcal{I}}
\def \calj {\mathcal{J}}

\def \bcup {\bigcup}

% set theory

\def \zfcc {\textbf{ZFC}^-}
\def \BGC {\textbf{BGC}}
\def \BG {\textbf{BG}}
\def \ac  {\textbf{AC}}
\def \gl  {\textbf{L }}
\def \gll {\textbf{L}}
\newcommand{\zfm}{$\textbf{ZF}^-$}

\def \ZFm {\text{ZF}^-}
\def \ZFCm {\text{ZFC}^-}
\DeclareMathOperator{\WF}{WF}
\DeclareMathOperator{\On}{On}
\def \on {\textbf{On }}
\def \cm {\textbf{M }}
\def \cn {\textbf{N }}
\def \cv {\textbf{V }}
\def \zc {\textbf{ZC }}
\def \zcm {\textbf{ZC}}
\def \zff {\textbf{ZF}}
\def \wfm {\textbf{WF}}
\def \onm {\textbf{On}}
\def \cmm {\textbf{M}}
\def \cnm {\textbf{N}}
\def \cvm {\textbf{V}}

\renewcommand{\restriction}{\mathord{\upharpoonright}}
%% another restriction
\newcommand\restr[2]{{% we make the whole thing an ordinary symbol
  \left.\kern-\nulldelimiterspace % automatically resize the bar with \right
  #1 % the function
  \vphantom{\big|} % pretend it's a little taller at normal size
  \right|_{#2} % this is the delimiter
  }}

\def \pred {\text{pred}}

\def \rank {\text{rank}}
\def \Con {\text{Con}}
\def \deff {\text{Def}}


\def \uin {\underline{\in}}
\def \oin {\overline{\in}}
\def \uR {\underline{R}}
\def \oR {\overline{R}}
\def \uP {\underline{P}}
\def \oP {\overline{P}}

\def \dsum {\displaystyle\sum}

\def \Ra {\Rightarrow}

\def \e {\enspace}

\def \sgn {\operatorname{sgn}}
\def \gen {\operatorname{gen}}
\def \Hom {\operatorname{Hom}}
\def \hom {\operatorname{hom}}
\def \Sub {\operatorname{Sub}}

\def \supp {\operatorname{supp}}

\def \epiarrow {\twoheadarrow}
\def \monoarrow {\rightarrowtail}
\def \rrarrow {\rightrightarrows}

% \def \minus {\text{-}}
% \newcommand{\minus}{\scalebox{0.75}[1.0]{$-$}}
% \DeclareUnicodeCharacter{002D}{\minus}


\def \tril {\triangleleft}

\def \ISigma {\text{I}\Sigma}
\def \IDelta {\text{I}\Delta}
\def \IPi {\text{I}\Pi}
\def \ACF {\textsf{ACF}}
\def \pCF {\textit{p}\text{CF}}
\def \ACVF {\textsf{ACVF}}
\def \HLR {\textsf{HLR}}
\def \OAG {\textsf{OAG}}
\def \RCF {\textsf{RCF}}
\DeclareMathOperator{\GL}{GL}
\DeclareMathOperator{\PGL}{PGL}
\DeclareMathOperator{\SL}{SL}
\DeclareMathOperator{\Inv}{Inv}
\DeclareMathOperator{\res}{res}
\DeclareMathOperator{\Sym}{Sym}
%\DeclareMathOperator{\char}{char}
\def \equal {=}

\def \degree {\text{degree}}
\def \app {\text{App}}
\def \FV {\text{FV}}
\def \conv {\text{conv}}
\def \cont {\text{cont}}
\DeclareMathOperator{\cl}{\text{cl}}
\DeclareMathOperator{\trcl}{\text{trcl}}
\DeclareMathOperator{\sg}{sg}
\DeclareMathOperator{\trdeg}{trdeg}
\def \Ord {\text{Ord}}

\DeclareMathOperator{\cf}{cf}
\DeclareMathOperator{\zfc}{ZFC}

%\DeclareMathOperator{\Th}{Th}
%\def \th {\text{Th}}
% \newcommand{\th}{\text{Th}}
\DeclareMathOperator{\type}{type}
\DeclareMathOperator{\zf}{\textbf{ZF}}
\def \fa {\mathfrak{a}}
\def \fb {\mathfrak{b}}
\def \fc {\mathfrak{c}}
\def \fd {\mathfrak{d}}
\def \fe {\mathfrak{e}}
\def \ff {\mathfrak{f}}
\def \fg {\mathfrak{g}}
\def \fh {\mathfrak{h}}
%\def \fi {\mathfrak{i}}
\def \fj {\mathfrak{j}}
\def \fk {\mathfrak{k}}
\def \fl {\mathfrak{l}}
\def \fm {\mathfrak{m}}
\def \fn {\mathfrak{n}}
\def \fo {\mathfrak{o}}
\def \fp {\mathfrak{p}}
\def \fq {\mathfrak{q}}
\def \fr {\mathfrak{r}}
\def \fs {\mathfrak{s}}
\def \ft {\mathfrak{t}}
\def \fu {\mathfrak{u}}
\def \fv {\mathfrak{v}}
\def \fw {\mathfrak{w}}
\def \fx {\mathfrak{x}}
\def \fy {\mathfrak{y}}
\def \fz {\mathfrak{z}}
\def \fA {\mathfrak{A}}
\def \fB {\mathfrak{B}}
\def \fC {\mathfrak{C}}
\def \fD {\mathfrak{D}}
\def \fE {\mathfrak{E}}
\def \fF {\mathfrak{F}}
\def \fG {\mathfrak{G}}
\def \fH {\mathfrak{H}}
\def \fI {\mathfrak{I}}
\def \fJ {\mathfrak{J}}
\def \fK {\mathfrak{K}}
\def \fL {\mathfrak{L}}
\def \fM {\mathfrak{M}}
\def \fN {\mathfrak{N}}
\def \fO {\mathfrak{O}}
\def \fP {\mathfrak{P}}
\def \fQ {\mathfrak{Q}}
\def \fR {\mathfrak{R}}
\def \fS {\mathfrak{S}}
\def \fT {\mathfrak{T}}
\def \fU {\mathfrak{U}}
\def \fV {\mathfrak{V}}
\def \fW {\mathfrak{W}}
\def \fX {\mathfrak{X}}
\def \fY {\mathfrak{Y}}
\def \fZ {\mathfrak{Z}}

\def \sfA {\textsf{A}}
\def \sfB {\textsf{B}}
\def \sfC {\textsf{C}}
\def \sfD {\textsf{D}}
\def \sfE {\textsf{E}}
\def \sfF {\textsf{F}}
\def \sfG {\textsf{G}}
\def \sfH {\textsf{H}}
\def \sfI {\textsf{I}}
\def \sfJ {\textsf{J}}
\def \sfK {\textsf{K}}
\def \sfL {\textsf{L}}
\def \sfM {\textsf{M}}
\def \sfN {\textsf{N}}
\def \sfO {\textsf{O}}
\def \sfP {\textsf{P}}
\def \sfQ {\textsf{Q}}
\def \sfR {\textsf{R}}
\def \sfS {\textsf{S}}
\def \sfT {\textsf{T}}
\def \sfU {\textsf{U}}
\def \sfV {\textsf{V}}
\def \sfW {\textsf{W}}
\def \sfX {\textsf{X}}
\def \sfY {\textsf{Y}}
\def \sfZ {\textsf{Z}}
\def \sfa {\textsf{a}}
\def \sfb {\textsf{b}}
\def \sfc {\textsf{c}}
\def \sfd {\textsf{d}}
\def \sfe {\textsf{e}}
\def \sff {\textsf{f}}
\def \sfg {\textsf{g}}
\def \sfh {\textsf{h}}
\def \sfi {\textsf{i}}
\def \sfj {\textsf{j}}
\def \sfk {\textsf{k}}
\def \sfl {\textsf{l}}
\def \sfm {\textsf{m}}
\def \sfn {\textsf{n}}
\def \sfo {\textsf{o}}
\def \sfp {\textsf{p}}
\def \sfq {\textsf{q}}
\def \sfr {\textsf{r}}
\def \sfs {\textsf{s}}
\def \sft {\textsf{t}}
\def \sfu {\textsf{u}}
\def \sfv {\textsf{v}}
\def \sfw {\textsf{w}}
\def \sfx {\textsf{x}}
\def \sfy {\textsf{y}}
\def \sfz {\textsf{z}}

\def \ttA {\texttt{A}}
\def \ttB {\texttt{B}}
\def \ttC {\texttt{C}}
\def \ttD {\texttt{D}}
\def \ttE {\texttt{E}}
\def \ttF {\texttt{F}}
\def \ttG {\texttt{G}}
\def \ttH {\texttt{H}}
\def \ttI {\texttt{I}}
\def \ttJ {\texttt{J}}
\def \ttK {\texttt{K}}
\def \ttL {\texttt{L}}
\def \ttM {\texttt{M}}
\def \ttN {\texttt{N}}
\def \ttO {\texttt{O}}
\def \ttP {\texttt{P}}
\def \ttQ {\texttt{Q}}
\def \ttR {\texttt{R}}
\def \ttS {\texttt{S}}
\def \ttT {\texttt{T}}
\def \ttU {\texttt{U}}
\def \ttV {\texttt{V}}
\def \ttW {\texttt{W}}
\def \ttX {\texttt{X}}
\def \ttY {\texttt{Y}}
\def \ttZ {\texttt{Z}}
\def \tta {\texttt{a}}
\def \ttb {\texttt{b}}
\def \ttc {\texttt{c}}
\def \ttd {\texttt{d}}
\def \tte {\texttt{e}}
\def \ttf {\texttt{f}}
\def \ttg {\texttt{g}}
\def \tth {\texttt{h}}
\def \tti {\texttt{i}}
\def \ttj {\texttt{j}}
\def \ttk {\texttt{k}}
\def \ttl {\texttt{l}}
\def \ttm {\texttt{m}}
\def \ttn {\texttt{n}}
\def \tto {\texttt{o}}
\def \ttp {\texttt{p}}
\def \ttq {\texttt{q}}
\def \ttr {\texttt{r}}
\def \tts {\texttt{s}}
\def \ttt {\texttt{t}}
\def \ttu {\texttt{u}}
\def \ttv {\texttt{v}}
\def \ttw {\texttt{w}}
\def \ttx {\texttt{x}}
\def \tty {\texttt{y}}
\def \ttz {\texttt{z}}

\def \bara {\bbar{a}}
\def \barb {\bbar{b}}
\def \barc {\bbar{c}}
\def \bard {\bbar{d}}
\def \bare {\bbar{e}}
\def \barf {\bbar{f}}
\def \barg {\bbar{g}}
\def \barh {\bbar{h}}
\def \bari {\bbar{i}}
\def \barj {\bbar{j}}
\def \bark {\bbar{k}}
\def \barl {\bbar{l}}
\def \barm {\bbar{m}}
\def \barn {\bbar{n}}
\def \baro {\bbar{o}}
\def \barp {\bbar{p}}
\def \barq {\bbar{q}}
\def \barr {\bbar{r}}
\def \bars {\bbar{s}}
\def \bart {\bbar{t}}
\def \baru {\bbar{u}}
\def \barv {\bbar{v}}
\def \barw {\bbar{w}}
\def \barx {\bbar{x}}
\def \bary {\bbar{y}}
\def \barz {\bbar{z}}
\def \barA {\bbar{A}}
\def \barB {\bbar{B}}
\def \barC {\bbar{C}}
\def \barD {\bbar{D}}
\def \barE {\bbar{E}}
\def \barF {\bbar{F}}
\def \barG {\bbar{G}}
\def \barH {\bbar{H}}
\def \barI {\bbar{I}}
\def \barJ {\bbar{J}}
\def \barK {\bbar{K}}
\def \barL {\bbar{L}}
\def \barM {\bbar{M}}
\def \barN {\bbar{N}}
\def \barO {\bbar{O}}
\def \barP {\bbar{P}}
\def \barQ {\bbar{Q}}
\def \barR {\bbar{R}}
\def \barS {\bbar{S}}
\def \barT {\bbar{T}}
\def \barU {\bbar{U}}
\def \barVV {\bbar{V}}
\def \barW {\bbar{W}}
\def \barX {\bbar{X}}
\def \barY {\bbar{Y}}
\def \barZ {\bbar{Z}}

\def \baralpha {\bbar{\alpha}}
\def \bartau {\bbar{\tau}}
\def \barsigma {\bbar{\sigma}}
\def \barzeta {\bbar{\zeta}}

\def \hata {\hat{a}}
\def \hatb {\hat{b}}
\def \hatc {\hat{c}}
\def \hatd {\hat{d}}
\def \hate {\hat{e}}
\def \hatf {\hat{f}}
\def \hatg {\hat{g}}
\def \hath {\hat{h}}
\def \hati {\hat{i}}
\def \hatj {\hat{j}}
\def \hatk {\hat{k}}
\def \hatl {\hat{l}}
\def \hatm {\hat{m}}
\def \hatn {\hat{n}}
\def \hato {\hat{o}}
\def \hatp {\hat{p}}
\def \hatq {\hat{q}}
\def \hatr {\hat{r}}
\def \hats {\hat{s}}
\def \hatt {\hat{t}}
\def \hatu {\hat{u}}
\def \hatv {\hat{v}}
\def \hatw {\hat{w}}
\def \hatx {\hat{x}}
\def \haty {\hat{y}}
\def \hatz {\hat{z}}
\def \hatA {\hat{A}}
\def \hatB {\hat{B}}
\def \hatC {\hat{C}}
\def \hatD {\hat{D}}
\def \hatE {\hat{E}}
\def \hatF {\hat{F}}
\def \hatG {\hat{G}}
\def \hatH {\hat{H}}
\def \hatI {\hat{I}}
\def \hatJ {\hat{J}}
\def \hatK {\hat{K}}
\def \hatL {\hat{L}}
\def \hatM {\hat{M}}
\def \hatN {\hat{N}}
\def \hatO {\hat{O}}
\def \hatP {\hat{P}}
\def \hatQ {\hat{Q}}
\def \hatR {\hat{R}}
\def \hatS {\hat{S}}
\def \hatT {\hat{T}}
\def \hatU {\hat{U}}
\def \hatVV {\hat{V}}
\def \hatW {\hat{W}}
\def \hatX {\hat{X}}
\def \hatY {\hat{Y}}
\def \hatZ {\hat{Z}}

\def \hatphi {\hat{\phi}}

\def \barfM {\bbar{\fM}}
\def \barfN {\bbar{\fN}}

\def \tila {\tilde{a}}
\def \tilb {\tilde{b}}
\def \tilc {\tilde{c}}
\def \tild {\tilde{d}}
\def \tile {\tilde{e}}
\def \tilf {\tilde{f}}
\def \tilg {\tilde{g}}
\def \tilh {\tilde{h}}
\def \tili {\tilde{i}}
\def \tilj {\tilde{j}}
\def \tilk {\tilde{k}}
\def \till {\tilde{l}}
\def \tilm {\tilde{m}}
\def \tiln {\tilde{n}}
\def \tilo {\tilde{o}}
\def \tilp {\tilde{p}}
\def \tilq {\tilde{q}}
\def \tilr {\tilde{r}}
\def \tils {\tilde{s}}
\def \tilt {\tilde{t}}
\def \tilu {\tilde{u}}
\def \tilv {\tilde{v}}
\def \tilw {\tilde{w}}
\def \tilx {\tilde{x}}
\def \tily {\tilde{y}}
\def \tilz {\tilde{z}}
\def \tilA {\tilde{A}}
\def \tilB {\tilde{B}}
\def \tilC {\tilde{C}}
\def \tilD {\tilde{D}}
\def \tilE {\tilde{E}}
\def \tilF {\tilde{F}}
\def \tilG {\tilde{G}}
\def \tilH {\tilde{H}}
\def \tilI {\tilde{I}}
\def \tilJ {\tilde{J}}
\def \tilK {\tilde{K}}
\def \tilL {\tilde{L}}
\def \tilM {\tilde{M}}
\def \tilN {\tilde{N}}
\def \tilO {\tilde{O}}
\def \tilP {\tilde{P}}
\def \tilQ {\tilde{Q}}
\def \tilR {\tilde{R}}
\def \tilS {\tilde{S}}
\def \tilT {\tilde{T}}
\def \tilU {\tilde{U}}
\def \tilVV {\tilde{V}}
\def \tilW {\tilde{W}}
\def \tilX {\tilde{X}}
\def \tilY {\tilde{Y}}
\def \tilZ {\tilde{Z}}

\def \tilalpha {\tilde{\alpha}}
\def \tilPhi {\tilde{\Phi}}

\def \barnu {\bar{\nu}}
\def \barrho {\bar{\rho}}
%\DeclareMathOperator{\ker}{ker}
\DeclareMathOperator{\im}{im}

\DeclareMathOperator{\Inn}{Inn}
\DeclareMathOperator{\rel}{rel}
\def \dote {\stackrel{\cdot}=}
%\DeclareMathOperator{\AC}{\textbf{AC}}
\DeclareMathOperator{\cod}{cod}
\DeclareMathOperator{\dom}{dom}
\DeclareMathOperator{\card}{card}
\DeclareMathOperator{\ran}{ran}
\DeclareMathOperator{\textd}{d}
\DeclareMathOperator{\td}{d}
\DeclareMathOperator{\id}{id}
\DeclareMathOperator{\LT}{LT}
\DeclareMathOperator{\Mat}{Mat}
\DeclareMathOperator{\Eq}{Eq}
\DeclareMathOperator{\irr}{irr}
\DeclareMathOperator{\Fr}{Fr}
\DeclareMathOperator{\Gal}{Gal}
\DeclareMathOperator{\lcm}{lcm}
\DeclareMathOperator{\alg}{\text{alg}}
\DeclareMathOperator{\Th}{Th}
%\DeclareMathOperator{\deg}{deg}


% \varprod
\DeclareSymbolFont{largesymbolsA}{U}{txexa}{m}{n}
\DeclareMathSymbol{\varprod}{\mathop}{largesymbolsA}{16}
% \DeclareMathSymbol{\tonm}{\boldsymbol{\to}\textbf{Nm}}
\def \tonm {\bto\textbf{Nm}}
\def \tohm {\bto\textbf{Hm}}

% Category theory
\DeclareMathOperator{\ob}{ob}
\DeclareMathOperator{\Ab}{\textbf{Ab}}
\DeclareMathOperator{\Alg}{\textbf{Alg}}
\DeclareMathOperator{\Rng}{\textbf{Rng}}
\DeclareMathOperator{\Sets}{\textbf{Sets}}
\DeclareMathOperator{\Set}{\textbf{Set}}
\DeclareMathOperator{\Grp}{\textbf{Grp}}
\DeclareMathOperator{\Met}{\textbf{Met}}
\DeclareMathOperator{\BA}{\textbf{BA}}
\DeclareMathOperator{\Mon}{\textbf{Mon}}
\DeclareMathOperator{\Top}{\textbf{Top}}
\DeclareMathOperator{\hTop}{\textbf{hTop}}
\DeclareMathOperator{\HTop}{\textbf{HTop}}
\DeclareMathOperator{\Aut}{\text{Aut}}
\DeclareMathOperator{\RMod}{R-\textbf{Mod}}
\DeclareMathOperator{\RAlg}{R-\textbf{Alg}}
\DeclareMathOperator{\LF}{LF}
\DeclareMathOperator{\op}{op}
\DeclareMathOperator{\Rings}{\textbf{Rings}}
\DeclareMathOperator{\Ring}{\textbf{Ring}}
\DeclareMathOperator{\Groups}{\textbf{Groups}}
\DeclareMathOperator{\Group}{\textbf{Group}}
\DeclareMathOperator{\ev}{ev}
% Algebraic Topology
\DeclareMathOperator{\obj}{obj}
\DeclareMathOperator{\Spec}{Spec}
\DeclareMathOperator{\spec}{spec}
% Model theory
\DeclareMathOperator*{\ind}{\raise0.2ex\hbox{\ooalign{\hidewidth$\vert$\hidewidth\cr\raise-0.9ex\hbox{$\smile$}}}}
\def\nind{\cancel{\ind}}
\DeclareMathOperator{\acl}{acl}
\DeclareMathOperator{\tspan}{span}
\DeclareMathOperator{\acleq}{acl^{\eq}}
\DeclareMathOperator{\Av}{Av}
\DeclareMathOperator{\ded}{ded}
\DeclareMathOperator{\EM}{EM}
\DeclareMathOperator{\dcl}{dcl}
\DeclareMathOperator{\Ext}{Ext}
\DeclareMathOperator{\eq}{eq}
\DeclareMathOperator{\ER}{ER}
\DeclareMathOperator{\tp}{tp}
\DeclareMathOperator{\stp}{stp}
\DeclareMathOperator{\qftp}{qftp}
\DeclareMathOperator{\Diag}{Diag}
\DeclareMathOperator{\MD}{MD}
\DeclareMathOperator{\MR}{MR}
\DeclareMathOperator{\RM}{RM}
\DeclareMathOperator{\el}{el}
\DeclareMathOperator{\depth}{depth}
\DeclareMathOperator{\ZFC}{ZFC}
\DeclareMathOperator{\GCH}{GCH}
\DeclareMathOperator{\Inf}{Inf}
\DeclareMathOperator{\Pow}{Pow}
\DeclareMathOperator{\ZF}{ZF}
\DeclareMathOperator{\CH}{CH}
\def \FO {\text{FO}}
\DeclareMathOperator{\fin}{fin}
\DeclareMathOperator{\qr}{qr}
\DeclareMathOperator{\Mod}{Mod}
\DeclareMathOperator{\Def}{Def}
\DeclareMathOperator{\TC}{TC}
\DeclareMathOperator{\KH}{KH}
\DeclareMathOperator{\Part}{Part}
\DeclareMathOperator{\Infset}{\textsf{Infset}}
\DeclareMathOperator{\DLO}{\textsf{DLO}}
\DeclareMathOperator{\PA}{\textsf{PA}}
\DeclareMathOperator{\DAG}{\textsf{DAG}}
\DeclareMathOperator{\ODAG}{\textsf{ODAG}}
\DeclareMathOperator{\sfMod}{\textsf{Mod}}
\DeclareMathOperator{\AbG}{\textsf{AbG}}
\DeclareMathOperator{\sfACF}{\textsf{ACF}}
\DeclareMathOperator{\DCF}{\textsf{DCF}}
% Computability Theorem
\DeclareMathOperator{\Tot}{Tot}
\DeclareMathOperator{\graph}{graph}
\DeclareMathOperator{\Fin}{Fin}
\DeclareMathOperator{\Cof}{Cof}
\DeclareMathOperator{\lh}{lh}
% Commutative Algebra
\DeclareMathOperator{\ord}{ord}
\DeclareMathOperator{\Idem}{Idem}
\DeclareMathOperator{\zdiv}{z.div}
\DeclareMathOperator{\Frac}{Frac}
\DeclareMathOperator{\rad}{rad}
\DeclareMathOperator{\nil}{nil}
\DeclareMathOperator{\Ann}{Ann}
\DeclareMathOperator{\End}{End}
\DeclareMathOperator{\coim}{coim}
\DeclareMathOperator{\coker}{coker}
\DeclareMathOperator{\Bil}{Bil}
\DeclareMathOperator{\Tril}{Tril}
\DeclareMathOperator{\tchar}{char}
\DeclareMathOperator{\tbd}{bd}

% Topology
\DeclareMathOperator{\diam}{diam}
\newcommand{\interior}[1]{%
  {\kern0pt#1}^{\mathrm{o}}%
}

\DeclareMathOperator*{\bigdoublewedge}{\bigwedge\mkern-15mu\bigwedge}
\DeclareMathOperator*{\bigdoublevee}{\bigvee\mkern-15mu\bigvee}

% \makeatletter
% \newcommand{\vect}[1]{%
%   \vbox{\m@th \ialign {##\crcr
%   \vectfill\crcr\noalign{\kern-\p@ \nointerlineskip}
%   $\hfil\displaystyle{#1}\hfil$\crcr}}}
% \def\vectfill{%
%   $\m@th\smash-\mkern-7mu%
%   \cleaders\hbox{$\mkern-2mu\smash-\mkern-2mu$}\hfill
%   \mkern-7mu\raisebox{-3.81pt}[\p@][\p@]{$\mathord\mathchar"017E$}$}

% \newcommand{\amsvect}{%
%   \mathpalette {\overarrow@\vectfill@}}
% \def\vectfill@{\arrowfill@\relbar\relbar{\raisebox{-3.81pt}[\p@][\p@]{$\mathord\mathchar"017E$}}}

% \newcommand{\amsvectb}{%
% \newcommand{\vect}{%
%   \mathpalette {\overarrow@\vectfillb@}}
% \newcommand{\vecbar}{%
%   \scalebox{0.8}{$\relbar$}}
% \def\vectfillb@{\arrowfill@\vecbar\vecbar{\raisebox{-4.35pt}[\p@][\p@]{$\mathord\mathchar"017E$}}}
% \makeatother
% \bigtimes

\DeclareFontFamily{U}{mathx}{\hyphenchar\font45}
\DeclareFontShape{U}{mathx}{m}{n}{
      <5> <6> <7> <8> <9> <10>
      <10.95> <12> <14.4> <17.28> <20.74> <24.88>
      mathx10
      }{}
\DeclareSymbolFont{mathx}{U}{mathx}{m}{n}
\DeclareMathSymbol{\bigtimes}{1}{mathx}{"91}
% \odiv
\DeclareFontFamily{U}{matha}{\hyphenchar\font45}
\DeclareFontShape{U}{matha}{m}{n}{
      <5> <6> <7> <8> <9> <10> gen * matha
      <10.95> matha10 <12> <14.4> <17.28> <20.74> <24.88> matha12
      }{}
\DeclareSymbolFont{matha}{U}{matha}{m}{n}
\DeclareMathSymbol{\odiv}         {2}{matha}{"63}


\newcommand\subsetsim{\mathrel{%
  \ooalign{\raise0.2ex\hbox{\scalebox{0.9}{$\subset$}}\cr\hidewidth\raise-0.85ex\hbox{\scalebox{0.9}{$\sim$}}\hidewidth\cr}}}
\newcommand\simsubset{\mathrel{%
  \ooalign{\raise-0.2ex\hbox{\scalebox{0.9}{$\subset$}}\cr\hidewidth\raise0.75ex\hbox{\scalebox{0.9}{$\sim$}}\hidewidth\cr}}}

\newcommand\simsubsetsim{\mathrel{%
  \ooalign{\raise0ex\hbox{\scalebox{0.8}{$\subset$}}\cr\hidewidth\raise1ex\hbox{\scalebox{0.75}{$\sim$}}\hidewidth\cr\raise-0.95ex\hbox{\scalebox{0.8}{$\sim$}}\cr\hidewidth}}}
\newcommand{\stcomp}[1]{{#1}^{\mathsf{c}}}

\setlength{\baselineskip}{0.5in}

\stackMath
\newcommand\yrightarrow[2][]{\mathrel{%
  \setbox2=\hbox{\stackon{\scriptstyle#1}{\scriptstyle#2}}%
  \stackunder[0pt]{%
    \xrightarrow{\makebox[\dimexpr\wd2\relax]{$\scriptstyle#2$}}%
  }{%
   \scriptstyle#1\,%
  }%
}}
\newcommand\yleftarrow[2][]{\mathrel{%
  \setbox2=\hbox{\stackon{\scriptstyle#1}{\scriptstyle#2}}%
  \stackunder[0pt]{%
    \xleftarrow{\makebox[\dimexpr\wd2\relax]{$\scriptstyle#2$}}%
  }{%
   \scriptstyle#1\,%
  }%
}}
\newcommand\yRightarrow[2][]{\mathrel{%
  \setbox2=\hbox{\stackon{\scriptstyle#1}{\scriptstyle#2}}%
  \stackunder[0pt]{%
    \xRightarrow{\makebox[\dimexpr\wd2\relax]{$\scriptstyle#2$}}%
  }{%
   \scriptstyle#1\,%
  }%
}}
\newcommand\yLeftarrow[2][]{\mathrel{%
  \setbox2=\hbox{\stackon{\scriptstyle#1}{\scriptstyle#2}}%
  \stackunder[0pt]{%
    \xLeftarrow{\makebox[\dimexpr\wd2\relax]{$\scriptstyle#2$}}%
  }{%
   \scriptstyle#1\,%
  }%
}}

\newcommand\altxrightarrow[2][0pt]{\mathrel{\ensurestackMath{\stackengine%
  {\dimexpr#1-7.5pt}{\xrightarrow{\phantom{#2}}}{\scriptstyle\!#2\,}%
  {O}{c}{F}{F}{S}}}}
\newcommand\altxleftarrow[2][0pt]{\mathrel{\ensurestackMath{\stackengine%
  {\dimexpr#1-7.5pt}{\xleftarrow{\phantom{#2}}}{\scriptstyle\!#2\,}%
  {O}{c}{F}{F}{S}}}}

\newenvironment{bsm}{% % short for 'bracketed small matrix'
  \left[ \begin{smallmatrix} }{%
  \end{smallmatrix} \right]}

\newenvironment{psm}{% % short for ' small matrix'
  \left( \begin{smallmatrix} }{%
  \end{smallmatrix} \right)}

\newcommand{\bbar}[1]{\mkern 1.5mu\overline{\mkern-1.5mu#1\mkern-1.5mu}\mkern 1.5mu}

\newcommand{\bigzero}{\mbox{\normalfont\Large\bfseries 0}}
\newcommand{\rvline}{\hspace*{-\arraycolsep}\vline\hspace*{-\arraycolsep}}

\font\zallman=Zallman at 40pt
\font\elzevier=Elzevier at 40pt

\newcommand\isoto{\stackrel{\textstyle\sim}{\smash{\longrightarrow}\rule{0pt}{0.4ex}}}
\newcommand\embto{\stackrel{\textstyle\prec}{\smash{\longrightarrow}\rule{0pt}{0.4ex}}}

% from http://www.actual.world/resources/tex/doc/TikZ.pdf

\tikzset{
modal/.style={>=stealth’,shorten >=1pt,shorten <=1pt,auto,node distance=1.5cm,
semithick},
world/.style={circle,draw,minimum size=0.5cm,fill=gray!15},
point/.style={circle,draw,inner sep=0.5mm,fill=black},
reflexive above/.style={->,loop,looseness=7,in=120,out=60},
reflexive below/.style={->,loop,looseness=7,in=240,out=300},
reflexive left/.style={->,loop,looseness=7,in=150,out=210},
reflexive right/.style={->,loop,looseness=7,in=30,out=330}
}


\makeatletter
\newcommand*{\doublerightarrow}[2]{\mathrel{
  \settowidth{\@tempdima}{$\scriptstyle#1$}
  \settowidth{\@tempdimb}{$\scriptstyle#2$}
  \ifdim\@tempdimb>\@tempdima \@tempdima=\@tempdimb\fi
  \mathop{\vcenter{
    \offinterlineskip\ialign{\hbox to\dimexpr\@tempdima+1em{##}\cr
    \rightarrowfill\cr\noalign{\kern.5ex}
    \rightarrowfill\cr}}}\limits^{\!#1}_{\!#2}}}
\newcommand*{\triplerightarrow}[1]{\mathrel{
  \settowidth{\@tempdima}{$\scriptstyle#1$}
  \mathop{\vcenter{
    \offinterlineskip\ialign{\hbox to\dimexpr\@tempdima+1em{##}\cr
    \rightarrowfill\cr\noalign{\kern.5ex}
    \rightarrowfill\cr\noalign{\kern.5ex}
    \rightarrowfill\cr}}}\limits^{\!#1}}}
\makeatother

% $A\doublerightarrow{a}{bcdefgh}B$

% $A\triplerightarrow{d_0,d_1,d_2}B$

\def \uhr {\upharpoonright}
\def \rhu {\rightharpoonup}
\def \uhl {\upharpoonleft}


\newcommand{\floor}[1]{\lfloor #1 \rfloor}
\newcommand{\ceil}[1]{\lceil #1 \rceil}
\newcommand{\lcorner}[1]{\llcorner #1 \lrcorner}
\newcommand{\llb}[1]{\llbracket #1 \rrbracket}
\newcommand{\ucorner}[1]{\ulcorner #1 \urcorner}
\newcommand{\emoji}[1]{{\DejaSans #1}}
\newcommand{\vprec}{\rotatebox[origin=c]{-90}{$\prec$}}

\newcommand{\nat}[6][large]{%
  \begin{tikzcd}[ampersand replacement = \&, column sep=#1]
    #2\ar[bend left=40,""{name=U}]{r}{#4}\ar[bend right=40,',""{name=D}]{r}{#5}\& #3
          \ar[shorten <=10pt,shorten >=10pt,Rightarrow,from=U,to=D]{d}{~#6}
    \end{tikzcd}
}


\providecommand\rightarrowRHD{\relbar\joinrel\mathrel\RHD}
\providecommand\rightarrowrhd{\relbar\joinrel\mathrel\rhd}
\providecommand\longrightarrowRHD{\relbar\joinrel\relbar\joinrel\mathrel\RHD}
\providecommand\longrightarrowrhd{\relbar\joinrel\relbar\joinrel\mathrel\rhd}
\def \lrarhd {\longrightarrowrhd}


\makeatletter
\providecommand*\xrightarrowRHD[2][]{\ext@arrow 0055{\arrowfill@\relbar\relbar\longrightarrowRHD}{#1}{#2}}
\providecommand*\xrightarrowrhd[2][]{\ext@arrow 0055{\arrowfill@\relbar\relbar\longrightarrowrhd}{#1}{#2}}
\makeatother

\newcommand{\metalambda}{%
  \mathop{%
    \rlap{$\lambda$}%
    \mkern3mu
    \raisebox{0ex}{$\lambda$}%
  }%
}

%% https://tex.stackexchange.com/questions/15119/draw-horizontal-line-left-and-right-of-some-text-a-single-line
\newcommand*\ruleline[1]{\par\noindent\raisebox{.8ex}{\makebox[\linewidth]{\hrulefill\hspace{1ex}\raisebox{-.8ex}{#1}\hspace{1ex}\hrulefill}}}

% https://www.dickimaw-books.com/latex/novices/html/newenv.html
\newenvironment{Block}[1]% environment name
{% begin code
  % https://tex.stackexchange.com/questions/19579/horizontal-line-spanning-the-entire-document-in-latex
  \noindent\textcolor[RGB]{128,128,128}{\rule{\linewidth}{1pt}}
  \par\noindent
  {\Large\textbf{#1}}%
  \bigskip\par\noindent\ignorespaces
}%
{% end code
  \par\noindent
  \textcolor[RGB]{128,128,128}{\rule{\linewidth}{1pt}}
  \ignorespacesafterend
}

\mathchardef\mhyphen="2D % Define a "math hyphen"

\def \QQ {\quad}
\def \QW {​\quad}

\makeindex
\author{Thomas Jech}
\date{\today}
\title{Set Theory}
\hypersetup{
 pdfauthor={Thomas Jech},
 pdftitle={Set Theory},
 pdfkeywords={},
 pdfsubject={},
 pdfcreator={Emacs 27.2 (Org mode 9.5)}, 
 pdflang={English}}
\begin{document}

\maketitle
\tableofcontents


\section{Ordinal Numbers}
\label{sec:org21364c4}

\subsection{Linear and Partial Ordering}
\label{sec:org393a1ea}
\begin{definition}[]
A binary relation < on a set \(P\) is a \textbf{partial ordering} of \(P\) if
\begin{enumerate}
\item \(p\not<p\) for any \(p\in P\)
\item if \(p<q\) and \(q<r\) then \(p<r\)
\end{enumerate}


\((P,<)\) is called a \textbf{partially ordered set}. A partial ordering < of \(P\) is a \textbf{linear ordering}
if moreover
\begin{enumerate}
\setcounter{enumi}{2}
\item \(p<q\) or \(p=q\) or \(q<p\) for all \(p,q\in P\)

If < is a partial ordering, then \(\le\) is also a partial ordering
\end{enumerate}
\end{definition}

if \((P,<)\) and \((Q,<)\) are partially ordered sets and \(f:P\to Q\), then \(f\) is
\textbf{order-preserving} if \(x<y\) implies \(f(x)<f(y)\). If \(P\) and \(Q\) are linearly ordered, then
an order-preserving function is also called \textbf{increasing}

\subsection{Well-Ordering}
\label{sec:org55660f4}
\begin{definition}[]
A linear ordering < of a set \(P\) is a \textbf{well-ordering} if every nonempty subset of \(P\) has a
least element
\end{definition}

\begin{lemma}[]
\label{lemma2.4}
If \((W,<)\) is a well-ordered set and \(f:W\to W\) is an increasing function, then \(f(x)\ge x\)
for each \(x\in W\)
\end{lemma}

\begin{proof}
Assume that the set \(X=\{x\in W:f(x)<x\}\) is nonempty and let \(z\) be the least element of \(X\).
If \(w=f(z)\), then \(f(w)<w\), a contradiction
\end{proof}

\begin{corollary}[]
The only automorphism of a well-ordered set is the identity
\end{corollary}

\begin{proof}
By Lemma \ref{lemma2.4}, \(f(x)\ge x\) for all \(x\), and \(f^{-1}(x)\ge x\) for all \(x\)
\end{proof}

\begin{corollary}[]
If two well-ordered sets \(W_1,W_2\) are isomorphic, then the isomorphism of \(W_1\) onto \(W_2\) is unique
\end{corollary}

if \(W\) is a well-ordered set and \(u\in W\), then \(\{x\in W:x<u\}\) is an \textbf{initial segment} of \(W\)

\begin{lemma}[]
\label{lemma2.7}
No well-ordered set is isomorphic to an initial segment of itself
\end{lemma}

\begin{proof}
If \(\ran(f)=\{x:x<u\}\), then \(f(u)<u\), contrary to Lemma \ref{lemma2.4}
\end{proof}

\begin{theorem}[]
\label{thm2.8}
If \(W_1\) and \(W_2\) are well-ordered sets, then exactly one of the following three cases holds
\begin{enumerate}
\item \(W_1\) is isomorphic to \(W_2\)
\item \(W_1\) is isomorphic to an initial segment of \(W_2\)
\item \(W_2\) is isomorphic to an initial segment of \(W_1\)
\end{enumerate}
\end{theorem}

\begin{proof}
For \(u\in W_i\), (\(i=1,2\)), let \(W_i(u)\) denote the initial segment of \(W_i\) given by \(u\).
Let
\begin{equation*}
f=\{(x,y)\in W_1\times W_2:W_1(x)\text{ is isomorphic to }W_2(y)\}
\end{equation*}
Using Lemma \ref{lemma2.7}, \(f\) is a injective: if \(f(x_1)=f(x_2)=y\),
then \(W_1(x_1)\cong W_2(y)\cong W_1(x_2)\), and \(x_1<x_2\) or \(x_2<x_1\) fail. If \(h\) is an isomorphism
between \(W_1(x)\) and \(W_2(y)\), and \(x'<x\), then \(W_1(x')\) and \(W_2(h(x'))\) are isomorphic.
It follows that \(f\) is order-preserving

If \(\dom(f)=W_1\) and \(\ran(f)=W_2\), then case 1 holds

if \(y_1<y_2\) and \(y_2\in\ran(f)\), then \(y_1\in\ran(f)\). Thus if \(\ran(f)\neq W_2\) and \(y_0\) is the
least element of \(W_2-\ran(f)\), we have \(\ran(f)=W_2(y_0)\). Necessarily, \(\dom(f)=W_1\), for
otherwise we would have \((x_0,y_0)\in f\), where \(x_0\)=the least element of \(W_1-\dom(f)\)
\end{proof}

if \(W_1\) and \(W_2\) are isomorphic, we say that they have the same \textbf{order-type}.

\subsection{Ordinal Numbers}
\label{sec:org15c5ab6}
\begin{definition}[]
A set \(T\) is \textbf{transitive} if every element of \(T\) is a subset of \(T\)
\end{definition}

\begin{definition}[]
A set is an \textbf{ordinal number} (an \textbf{ordinal}) if it is transitive and well-ordered by \(\in\)
\end{definition}

Define
\begin{equation*}
\alpha<\beta \quad\text{ iff }\quad \alpha\in\beta
\end{equation*}

\begin{lemma}[]
\label{lemma2.11}
\begin{enumerate}
\item \(0=\emptyset\) is an ordinal
\item if \(\alpha\) is an ordinal and \(\beta\in\alpha\), then \(\beta\) is an ordinal
\item if \(\alpha\neq\beta\) are ordinals and \(\alpha\subset\beta\), then \(\alpha\in\beta\)
\item if \(\alpha\), \(\beta\) are ordinals, then either \(\alpha\subset\beta\) or \(\beta\subset\alpha\)
\end{enumerate}
\end{lemma}

\begin{proof}
1,2 by definition

\begin{enumerate}
\setcounter{enumi}{2}
\item if \(\alpha\subset\beta\), let \(\gamma\) be the least element of the set \(\beta-\alpha\). Since \(\alpha\) is transitive, it
follows that \(\alpha\) is the initial segment of \(\beta\) given by \(\gamma\): for \(\eta\in\alpha\), \(\eta\neq\gamma\) and \(\gamma\not\in\eta\),
hence \(\eta\in\gamma\) since ordinals are well-ordered by \(\in\). Thus \(\alpha=\{\xi\in\beta:\xi<\gamma\}=\gamma\), and
so \(\alpha\in\beta\).
\item \(\alpha\cap\beta\) is an ordinal, \(\alpha\cap\beta=\gamma\). We have \(\gamma=\alpha\) or \(\gamma=\beta\), for otherwise \(\gamma\in\alpha\)
and \(\gamma\in\beta\), by 3. Then \(\gamma\in\gamma\), which contradicts the definition of an ordinal (namely
that \(\in\) is a \textbf{strict} ordering of \(\alpha\))
\end{enumerate}
\end{proof}

Using Lemma \ref{lemma2.11} one gets the followings
\begin{enumerate}
\item < is a linear ordering of the class \(\Ord\)
\item for each \(\alpha\), \(\alpha=\{\beta:\beta<\alpha\}\) \label{Question1}
\item if \(C\) is a nonempty class of ordinals, then \(\bigcap C\) is an ordinal, \(\bigcap C\in C\)
and \(\bigcap C=\inf C\)
\item if \(X\) is a nonempty set of ordinals, then \(\bigcup X\) is an ordinal, and \(\bigcup X=\sup X\)
\item for every \(\alpha\), \(\alpha\cup\{\alpha\}\) is an ordinal and \(\alpha\cup\{\alpha\}=\inf\{\beta:\beta>\alpha\}\)
\end{enumerate}


We thus define \(\alpha+1=\alpha\cup\{\alpha\}\). In view of 4, the class \(\Ord\) is a proper class; otherwise
consider \(\sup\Ord+1\)

\begin{theorem}[]
Every well-ordered set is isomorphic to a unique ordinal number
\end{theorem}

\begin{proof}
The uniqueness follows from Lemma \ref{lemma2.7}: suppose \(\alpha\cong\beta\) and \(\alpha\neq\beta\). As \(\alpha\neq\beta\),
either \(\alpha\in\beta\) or \(\beta\in\alpha\), thus \(\alpha\) is isomorphic to an initial segment of \(\beta\) or vice versa.
But by Lemma \ref{lemma2.7},  we get a contradiction.

Given a well-ordered set \(W\),
define \(F(x)=\alpha\) is \(\alpha\) is isomorphic to the initial segment of \(W\) given by \(x\). If such an
\(\alpha\) exists, then it is unique. By the Replacement Axioms, \(F(W)\) is a set. For each \(x\in W\),
such an \(\alpha\) exists (otherwise consider the least \(x\) for which such an \(\alpha\) does not exists). If
\(\gamma\) is the least \(\gamma\not\in F(W)\), then \(F(W)=\gamma\) and we have an isomorphism of \(W\) onto \(\gamma\)
\end{proof}

0 is a limit ordinal and define \(\sup\emptyset=0\)

\begin{definition}[Natural Numbers]
We denote the least nonzero limit ordinal \(\omega\) (or \(\N\)). The ordinals less than \(\omega\) are call
\textbf{finite ordinals}, or \textbf{natural numbers}
\end{definition}

\subsection{Induction and Recursion}
\label{sec:orgd8523a7}
\begin{theorem}[Transfinite Induction]
Let \(C\) be a class of ordinals and assume that
\begin{enumerate}
\item \(0\in C\)
\item if \(\alpha\in C\), then \(\alpha+1\in C\)
\item if \(\alpha\) is a nonzero limit ordinal and \(\beta\in C\) for all \(\beta<\alpha\), then \(\alpha\in C\)
\end{enumerate}


Then \(C\) is the class of all ordinals
\end{theorem}

\begin{proof}
Otherwise, let \(\alpha\) be the least \(\alpha\not\in C\) and apply 1,2 and 3.
\end{proof}

A function whose domain is the set \(\N\) is called an (\textbf{infinite}) \textbf{sequence} (A \textbf{sequence in} \(X\)
is a function \(f:\N\to X\)). The standard notation for a sequence is
\begin{equation*}
\la a_n:n<\omega\ra
\end{equation*}
A \textbf{finite sequence} is a function \(s\) s.t. \(\dom(s)=\{i:i<n\}\) for some \(n\in\N\); then \(s\) is a
\textbf{sequence  of length} \(n\)

A \textbf{transfinite sequence} is a function whose domain is an ordinal
\begin{equation*}
\la a_\xi:\xi<\alpha\ra
\end{equation*}
It is also called an \textbf{\(\alpha\)-sequence} or a \textbf{sequence of length} \(\alpha\). We also say that a
sequence \(\la a_\xi:\xi<\alpha\ra\) is an \textbf{enumeration} of its range \(\{a_\xi:\xi<\alpha\}\). If \(s\) is a sequence of
length \(\alpha\), then \(s^\smallfrown x\) or simply \(sx\) denotes the sequence of length \(\alpha+1\) that
extends \(s\) and whose \(\alpha\)th term is \(x\):
\begin{equation*}
s^\smallfrown x=sx=s\cup\{(\alpha,x)\}
\end{equation*}
Sometimes we call a ``sequence''
\begin{equation*}
\la a_\alpha:\alpha\in\Ord\ra
\end{equation*}
a function (a proper class) on \(\Ord\)


``Definition by transfinite recursion'' usually takes the following form: Given a function \(G\)
(on the class of transfinite sequence), then for every \(\theta\) there exists a unique \(\theta\)-sequence
\begin{equation*}
\la a_\alpha:\alpha<\theta\ra
\end{equation*}
s.t.
\begin{equation*}
a_\alpha=G(\la a_\xi:\xi<\alpha\ra)
\end{equation*}
for every \(\alpha<\theta\)

\begin{theorem}[Transfinite Recursion]
Let \(G\) be a function (on \(V\)), then \eqref{2.6} below defines a unique function \(F\)
on \(\Ord\) s.t.
\begin{equation*}
F(\alpha)=G(F\restriction\alpha)
\end{equation*}
for each \(\alpha\)
\end{theorem}

In other words, if we let \(a_\alpha=F(\alpha)\), then for each \(\alpha\)
\begin{equation*}
a_\alpha=G(\la a_\xi:\xi<\alpha\ra)
\end{equation*}
(Note that we tacitly use Replacement: \(F\restriction\alpha\) is a set for each \(\alpha\))

\begin{corollary}[]
Let \(X\) be a set and \(\theta\) an ordinal number. For every function \(G\) on the set of all
transfinite sequences in \(X\) of length \(<\theta\) s.t. \(\ran(G)\subset X\) there exists a unique
\(\theta\)-sequence \(\la a_\alpha:\alpha<\theta\ra\) in \(X\) s.t. \(a_\alpha=G(\la a_\xi:\xi<\alpha\ra)\) for every \(\alpha<\theta\)
\end{corollary}

\begin{proof}
Let
\begin{align}
\label{2.6}
F(\alpha)=x\leftrightarrow&\text{ there is a sequence }\la a_\xi:\xi<\alpha\ra\text{ s.t.:}\\
&1.\; (\forall\xi<\alpha)a_\xi=G(\la a_\eta:\eta<\xi\ra)\nonumber\\
&2.\; x=G(\la a_\xi:\xi<\alpha\ra)\nonumber
\end{align}
For every \(\alpha\), if there is an \(\alpha\)-sequence that satisfies 1, then such a sequence is unique:
if \(\la a_\xi:\xi<\alpha\ra\) and \(\la b_\xi:\xi<\alpha\ra\) are two \(\alpha\)-sequences satisfying 1, one shows \(a_\xi=b_\xi\) by
induction on \(\xi\).Thus \(F(\alpha)\) is determined uniquely by 2, and therefore \(F\) is a function.

it follows, again by induction, that for each \(\alpha\) there is an \(\alpha\)-sequence that satisfies 1 (at
limit steps, we use Replacement to get the \(\alpha\)-sequence as the union of all the
\(\xi\)-sequences, \(\xi<\alpha\)). Thus \(F\) is defined for all \(\alpha\in\Ord\). It obviously satisfies
\begin{equation*}
F(\alpha)=G(F\restriction\alpha)
\end{equation*}

If \(F'\) is any function on \(\Ord\) that satisfies
\begin{equation*}
F'(\alpha)=G(F'\restriction\alpha)
\end{equation*}
then it follows by induction that \(F'(\alpha)=F(\alpha)\) for all \(\alpha\)
\end{proof}

\begin{definition}[]
Let \(\alpha>0\) be a limit ordinal and let \(\la\gamma_\xi:\xi<\alpha\ra\) be a \textbf{nondecreasing} sequence of ordinals.
We define the \textbf{limit} of the sequence by
\begin{equation*}
\lim_{\xi\to\alpha}\gamma_\xi=\sup\{\gamma_\xi:\xi<\alpha\}
\end{equation*}

A sequence of ordinals \(\la\gamma_\alpha:\alpha\in\Ord\ra\) is \textbf{normal} if it is increasing and \textbf{continuous}, i.e., for
every limit \(\alpha\),\(\gamma_\alpha=\lim_{\xi\to\alpha}\gamma_\xi\)
\end{definition}

\subsection{Ordinal Arithmetic}
\label{sec:orgfa73516}
\begin{definition}[Addition]
For all ordinal numbers \(\alpha\)
\begin{enumerate}
\item \(\alpha+0=\alpha\)
\item \(\alpha+(\beta+1)=(\alpha+\beta)+1\) for all \(\beta\)
\item \(\alpha+\beta=\lim_{\xi\to\beta}(\alpha+\xi)\) for all limit \(\beta>0\)
\end{enumerate}
\end{definition}

\begin{definition}[Multiplication]
For all ordinal numbers \(\alpha\)
\begin{enumerate}
\item \(\alpha\cdot 0=0\)
\item \(\alpha\cdot(\beta+1)=\alpha\cdot\beta+\alpha\) for all \(\beta\)
\item \(\alpha\cdot\beta=\lim_{\xi\to\beta}\alpha\cdot\xi\) for all limit \(\beta>0\)
\end{enumerate}
\end{definition}

\begin{definition}[Exponentiation]
For all ordinal numbers \(\alpha\)
\begin{enumerate}
\item \(\alpha^0=1\)
\item \(\alpha^{\beta+1}=\alpha^\beta\cdot\alpha\) for all \(\beta\)
\item \(\alpha^\beta=\lim_{\xi\to\beta}\alpha^\xi\) for all limit \(\beta>0\)
\end{enumerate}
\end{definition}

\begin{lemma}[]
For all ordinals \(\alpha\), \(\beta\) and \(\gamma\)
\begin{enumerate}
\item \(\alpha+(\beta+\gamma)=(\alpha+\beta)+\gamma\)
\item \(\alpha\cdot(\beta\cdot\gamma)=(\alpha\cdot\beta)\cdot\gamma\)
\end{enumerate}
\end{lemma}

\begin{proof}
Induction on \(\gamma\)
\end{proof}

Neither + nor \(\cdot\) are commutative:
\begin{equation*}
1+\omega=\omega\neq\omega+1,\quad 2\cdot\omega=\omega\neq\omega\cdot 2=\omega+\omega
\end{equation*}

\begin{definition}[]
Let \((A,<_A)\) and \((B,<_B)\) be disjoint linearly ordered sets. The \textbf{sum} of these linear
orders is the set \(A\cup B\) with the ordering defined as follows: \(x<y\) iff
\begin{enumerate}
\item \(x,y\in A\) and \(x<_Ay\), or
\item \(x,y\in B\) and \(x<_By\), or
\item \(x\in A\) and \(y\in B\)
\end{enumerate}
\end{definition}

\begin{definition}[]
Let \((A,<)\) and \((B,<)\) be linearly ordered sets. The \textbf{product} of these linear orders is the
set \(A\times B\) with the ordering defined by
\begin{equation*}
(a_1,b_1)<(a_2,b_2)\text{ iff either }b_1<b_2\text{ or }(b_1=b_2\text{ and }a_1<a_2)
\end{equation*}
\end{definition}

\begin{lemma}[]
\label{Question2}
For all ordinals \(\alpha\) and \(\beta\), \(\alpha+\beta\) and \(\alpha\cdot\beta\) are isomorphic to the sum and product of \(\alpha\) and \(\beta\)
\end{lemma}

\begin{proof}
We can define \(S(\alpha,\beta)=\{(0,a):a\in\alpha\}\cup\{(1,b)\in\beta\}\)

if \(\beta=0\), then \(S(\alpha,\beta)=\alpha\)

if \(\beta=\eta+1\), then \(S(\alpha,\beta)=S(\alpha,\eta)\cup\{(1,\eta)\}\)
\end{proof}

\begin{lemma}[]
\label{lemma2.25}
\begin{enumerate}
\item if \(\beta<\gamma\) then \(\alpha+\beta<\alpha+\gamma\)
\item if \(\alpha<\beta\) then there exists a unique \(\delta\) s.t. \(\alpha+\delta=\beta\)
\item if \(\beta<\gamma\) and \(\alpha>0\), then \(\alpha\cdot\beta<\alpha\cdot\gamma\)
\item if \(\alpha>0\) and \(\gamma\) is arbitrary, then there exist a unique \(\beta\) and a unique \(\rho<\alpha\) s.t. \(\gamma=\alpha\cdot\beta+\rho\)
\item if \(\beta<\gamma\) and \(\alpha>1\), then \(\alpha^\beta<\alpha^\gamma\)
\end{enumerate}
\end{lemma}

\begin{proof}
\begin{enumerate}
\item induction on \(\gamma\)
\item let \(\delta\) be the order-type of the set \(\{\xi:\alpha\le\xi<\beta\}\); \(\delta\) is unique by 1
\item \(\gamma\)
\item let \(\beta\) be the greatest ordinal s.t. \(\alpha\cdot\beta\le\gamma\)
\item \(\gamma\)
\end{enumerate}
\end{proof}

\begin{theorem}[Cantor's Normal Form Theorem]
Every ordinal \(\alpha>0\) can be represented uniquely in the form
\begin{equation*}
\alpha=\omega^{\beta_1}\cdot k_1+\dots+\omega^{\beta_n}\cdot k_n
\end{equation*}
where \(n\ge 1\), \(\alpha\ge\beta_1>\dots>\beta_n\), and \(k_1,\dots,k_n\) are nonzero natural numbers
\end{theorem}

\begin{proof}
By induction on \(\alpha\). For \(\alpha=1\), we have \(1=\omega^0\cdot 1\); for arbitrary \(\alpha>0\), let \(\beta\) be the
greatest ordinal s.t. \(\omega^\beta\le\alpha\). By Lemma \ref{lemma2.25} (4) there exists a unique \(\delta\) and a
unique \(\rho<\omega^\beta\) s.t. \(\alpha=\omega^\beta\cdot\delta+\rho\); this \(\delta\) must necessarily be finite
\end{proof}


\subsection{Well-Founded Relations}
\label{sec:org7560b82}
A binary relation \(E\) on a set \(P\) is \textbf{well-founded} if every nonempty \(X\subset P\) has an
\textbf{\(E\)-minimal} element, that is \(a\in X\) s.t. there is no \(x\in X\) with \(xEa\)

Given a well-founded relation \(E\) on a set \(P\), we can define the \textbf{height} of \(E\), and
assign to each \(x\in P\) an ordinal number, the \textbf{rank} of \(x\) in \(E\)

\begin{theorem}[]
If \(E\) is a well-founded relation on \(P\), then there exists a unique function \(\rho\) from \(P\)
into the ordinals s.t. for all \(x\in P\)
\begin{equation*}
\rho(x)=\sup\{\rho(y)+1:yEx\}
\end{equation*}
\end{theorem}

The range of \(\rho\) is an initial segment of the ordinals, thus an ordinal number. This ordinal is
called the \textbf{height} of \(E\)

\begin{proof}
By induction, let
\begin{align*}
&P_0=\emptyset\\
&P_{\alpha+1}=\{x\in P:\forall y(yEx\to y\in P_\alpha)\}\\
&P_\alpha=\bigcup_{\xi<\alpha}P_\xi\quad\text{ if $\alpha$ is a limit ordinal}
\end{align*}
Let \(\theta\) be the least ordinal s.t. \(P_{\theta+1}=P_\theta\) (such \(\theta\) exists by Replacement, \(\theta\) is at least
\(\alpha\) i guess).
First \(P_\alpha\subset P_{\alpha+1}\) for each \(\alpha\). Thus \(P_0\subset P_1\subset\dots\subset P_\theta\). We claim that \(P_\theta=P\). Otherwise,
let \(a\) be an \(E\)-minimal element of \(P-P_\theta\). It follows that each \(xEa\) is an \(P_\theta\),
and so \(a\in P_{\theta+1}\), a contradiction. Now we define \(\rho(x)\) as the least \(\alpha\)
s.t. \(x\in P_{\alpha+1}\). The ordinal \(\theta\) is the height of \(E\).

Uniqueness: let \(\rho'\) be another function and consider an \(E\)-minimal element of the
set \(\{x\in P:\rho(x)\neq\rho'(x)\}\).
\end{proof}

\section{Cardinal Numbers}
\label{sec:orge017c8b}

\subsection{Cardinality}
\label{sec:org0addabf}
Two sets \(X,Y\) have the same \textbf{cardinality}
\begin{equation*}
\abs{X}=\abs{Y}
\end{equation*}
if there exists a one-to-one mapping of \(X\) onto \(Y\)

\begin{equation*}
\abs{X}\le\abs{Y}
\end{equation*}
if there exists a one-to-one mapping of \(X\) into \(Y\).

\begin{theorem}[Cantor]
\label{thm3.1}
For every set \(X\), \(\abs{X}<\abs{P(X)}\)
\end{theorem}

\begin{proof}
Let \(f\) be a function from \(X\) into \(P(X)\). The set
\begin{equation*}
Y=\{x\in X:x\not\in f(x)\}
\end{equation*}
is not in the range of \(f\): If \(z\in X\) were such that \(f(z)=Y\), then \(z\in Y\)
iff \(z\not\in Y\). Thus \(f\) is not a function of \(X\) onto \(P(X)\).
Hence \(\abs{P(X)}\neq\abs{X}\)

The function \(f(x)=\{x\}\) is the required one
\end{proof}

\begin{theorem}[Cantor-Bernstein]
\label{thm3.2}
If \(\abs{A}\le\abs{B}\) and \(\abs{B}\le\abs{A}\), then \(\abs{A}=\abs{B}\)
\end{theorem}

\begin{proof}
From a nice \href{https://web.williams.edu/Mathematics/lg5/CanBer.pdf}{note}

We will write \(X\sim Y\) to denote the existence of a bijection from \(X\) to \(Y\).

Given injections \(f:A\to B\) and \(g:B\to A\). Let
\begin{alignat*}{2}
&A_0=A\hspace{2cm}&&B_0=B\\
&A_1=g(B_0)&&B_1=f(A_0)\\
&A_2=g(B_1)&&B_2=f(A_1)\\
&\vdots&&\vdots\\
&A_n=g(B_{n-1})&&B_n=f(A_{n-1})\\
&\vdots&&\vdots
\end{alignat*}
Then
\begin{align*}
&A=A_0\sim B_1\sim A_2\sim B_3\sim A_4\cdots\\
&B=B_0\sim A_1\sim B_2\sim A_3\sim B_4\sim\cdots
\end{align*}
and
\begin{align*}
&A_0\supseteq A_1\supseteq A_2\supseteq\cdots\\
&B_0\supseteq B_1\supseteq B_2\supseteq\cdots
\end{align*}
If \(A_n=A_{n+1}\), then \(A\sim B\), hence we assume \(A_n\supsetneq A_{n+1}\)

Problem here now is that \(X_1\sim Y_1\) and \(X_2\sim Y_2\) do \textbf{not} imply \(X_1\cup X_2\sim Y_1\cup Y_2\) and
therefore \(A\cup A_1\sim B_1\cup B\)
\end{proof}

\begin{lemma}[]
Suppose we have sets \(\{X_i\}\) and \(\{Y_i\}\) satisfying \(X_i\sim Y_i\) for all \(i\). If all
the \(X_i\) are pairwise disjoint, and all the \(Y_i\) are pairwise disjoint, then
\begin{equation*}
\bigcup_iX_i\sim\bigcup_iY_i
\end{equation*}
\end{lemma}

\begin{proof}[Continuation of proof \ref{thm3.2}]
Hence for each \(n\), set \(A_n^*=A_n-A_{n+1}\). By our assumption, all \(A_n^*\) are nonempty,
moreover they are pairwise disjoint. Also we get

\begin{align*}
    &A^*=A^*_0\sim B^*_1\sim A^*_2\sim B^*_3\sim A^*_4\cdots\\
    &B^*=B^*_0\sim A^*_1\sim B^*_2\sim A^*_3\sim B^*_4\sim\cdots
    \end{align*}
Hence we get
\begin{equation*}
\tilde{A}:=\bigcup_{n\ge 0}A_n^*\sim\tilde{B}:=\bigcup_{n\ge 0}B_n^*
\end{equation*}
Let \(\barA=\bigcap_{n\ge 0}A_n\) and \(\barb=\bigcap_{n\ge 0}B_n\)

\textbf{Claim} \(A=\barA\cup\tilde{A}\) is a partition of \(A\), and \(B=\barB\cup\tilde{B}\) is a partition
of \(B\)

Now it remains to show that \(\barA\sim\barB\), which is immediate as \(f(\barA)=\barB\) and \(g(\barB)=\barA\)
\end{proof}

The arithmetic operations on cardinals are defined as follows
\begin{alignat*}{2}
&\kappa+\lambda=\abs{A\cup B}\quad&&\text{ where }\abs{A}=\kappa,\abs{B}=\lambda,\text{ and }A,B\text{ are disjoint}\\
&\kappa\cdot\lambda=\abs{A\times B}&&\text{ where }\abs{A}=\kappa,\abs{B}=\lambda\\
&\kappa^\lambda=\abs{A^B}&&\text{ where }\abs{A}=\kappa,\abs{B}=\lambda
\end{alignat*}

\begin{lemma}[]
if \(\abs{A}=\kappa\), then \(\abs{P(A)}=2^\kappa\)
\end{lemma}

\begin{proof}
For every \(X\subset A\), let \(\chi_X\) be the function
\begin{equation*}
\chi_X(x)=
\begin{cases}
1&\text{ if }x\in X\\
0&\text{ if }x\in A-X
\end{cases}
\end{equation*}
The mapping \(f:X\to\chi_X\) is.a one-to-one correspondence between \(P(A)\) and \(\{0,1\}^A\)
\end{proof}

Thus Cantor's Theorem \ref{thm3.1} can be formulated as
\begin{quoting}
\(\kappa<2^\kappa\) for every cardinal \(\kappa\)
\end{quoting}

Very useful \href{https://math.stackexchange.com/questions/131212/overview-of-basic-results-on-cardinal-arithmetic}{link}

\begin{proposition}[]
\begin{enumerate}
\item + and \(\cdot\) is associative, commutative and distributive
\item \((\kappa\cdot\lambda)^\mu=\kappa^\mu\cdot\lambda^\mu\)
\item \(\kappa^{\lambda+\mu}=\kappa^\lambda\cdot\kappa^\mu\)
\item \((\kappa^\lambda)^\mu=\kappa^{\lambda\cdot\mu}\)
\item if \(\kappa\le\lambda\), then \(\kappa^\mu\le\lambda^\mu\)
\item if \(0<\lambda\le\mu\), then \(\kappa^\lambda\le\kappa^\mu\)
\item \(\kappa^0=1\); \(1^\kappa=1\); \(0^\kappa=0\) if \(\kappa>0\)
\end{enumerate}
\end{proposition}

\begin{proof}
\begin{enumerate}
\item commutativity of + follows from \(A\cup B=B\cup A\), and so is the commutativity of \(\cdot\). Similar
for associativity
\setcounter{enumi}{2}
\item Given \(f:A\cup B\to C\), we get \(f\restriction A\) and \(f\restriction B\). Therefore we
have a map \(f\mapsto(f\restriction A,f\restriction B)\)
\setcounter{enumi}{5}
\item let \(\abs{A}=\kappa\), \(\abs{B}=\lambda\), \(\abs{C}=\mu\). Given injection \(f:B\to C\), for
each \(h:B\to A\) we associate a \(g(y):C\to A\) by \(g(f(x))=h(x)\) if \(y\in f(B)\),
otherwise \(g(y)\) can be anything.
\end{enumerate}
\end{proof}

\subsection{Alephs}
\label{sec:org8ed661b}
An ordinal \(\alpha\) is called a \textbf{cardinal number} if \(\abs{\alpha}\neq\abs{\beta}\) for all \(\beta<\alpha\).

If \(W\)is a well-ordered set, then there exists an ordinal \(\alpha\) s.t. \(\abs{W}=\abs{\alpha}\). Thus we
let
\begin{equation*}
\abs{W}=\text{ the least ordinal s.t. }\abs{W}=\abs{\alpha}
\end{equation*}

Every natural number is a cardinal

The infinite ordinal numbers that are cardinals are called \textbf{alephs}

\begin{lemma}[]
\label{lemma3.4}
\begin{enumerate}
\item For every \(\alpha\) there is a cardinal number greater than \(\alpha\)
\item if \(X\) is a set of cardinals, then \(\sup X\) is a cardinal
\end{enumerate}
\end{lemma}

for every \(\alpha\), let \(\alpha^+\) be the least cardinal number greater than \(\alpha\), the \textbf{cardinal successor} of
\(\alpha\)

\begin{proof}
\begin{enumerate}
\item for any set \(X\), let \(h(X)\)=the least \(\alpha\) s.t. there is no one-to-one function of \(\alpha\)
into \(X\). There is only a set of possible well-orderings of subsets of \(X\).
(But the collection of ordinals is a class)
Hence there is only a set of ordinals for which a one-to-one function of \(\alpha\) into \(X\) exists.
Thus \(h(X)\) exists.

if \(\alpha\) is an ordinal, then \(\abs{\alpha}<\abs{h(\alpha)}\)

\item let \(\alpha=\sup X\). if \(f\) is a one-to-one mapping of \(\alpha\) onto some \(\beta<\alpha\), let \(\kappa\in X\) be
s.t. \(\beta<\kappa\le\alpha\). Then \(\abs{\kappa}=\abs{\{f(\xi):\xi<\kappa\}}\le\beta\), a contradiction. Thus \(\alpha\) is a cardinal
\end{enumerate}
\end{proof}

Use Lemma \ref{lemma3.4} we define the increasing enumeration of all alephs. We usually
use \(\aleph_\alpha\) when referring to the cardinal number, and \(\omega_\alpha\) to denote the order-type
\begin{align*}
&\aleph_0=\omega_0=\omega\\
&\aleph_{\alpha+1}=\omega_{\alpha+1}=\aleph_\alpha^+\\
&\aleph_\alpha=\omega_\alpha=\sup\{\omega_\beta:\beta<\alpha\}\quad\text{ if \(\alpha\) is a limit ordinal}
\end{align*}
\begin{theorem}[]
\label{thm3.5}
\(\aleph_\alpha\cdot\aleph_\alpha=\aleph_\alpha\)
\end{theorem}
\subsection{The Canonical Well-Ordering of \texorpdfstring{\(\alpha\times\alpha\)}{α×α}}
\label{sec:orga6fbf92}
relevant \href{https://planetmath.org/canonicalorderingonpairsofordinals}{reading}

We define a well-ordering of the class \(\Ord\times\Ord\) of ordinal pairs. Under this well-ordering,
each \(\alpha\times\alpha\) is an initial segment of \(\Ord^2\); the induced well-ordering of \(\alpha^2\) is called
the \textbf{canonical well-ordering} of \(\alpha^2\). Moreover, the well-ordered class \(\Ord^2\) is isomorphic
to the class \(\Ord\)

We define
\begin{align*}
(\alpha,\beta)<(\gamma,\delta)\leftrightarrow\;&\text{either }\max\{\alpha,\beta\}<\max\{\gamma,\delta\}\\
&\text{or }\max\{\alpha,\beta\}=\max\{\gamma,\delta\}\text{ and }\alpha<\gamma\\
&\text{or }\max\{\alpha,\beta\}=\max\{\gamma,\delta\},\alpha=\gamma\text{ and }\beta<\delta
\end{align*}
If \(X\subset\Ord\times\Ord\) is nonempty, then \(X\) has a least element. For each \(\alpha\), \(\alpha\times\alpha\) is the
initial segment given by \((0,\alpha)\). If we let
\begin{equation*}
\Gamma(\alpha,\beta)=\text{ the order-type of the set }\{(\xi,\eta):(\xi,\eta)<(\alpha,\beta)\}
\end{equation*}
then \(\Gamma\) is a one-to-one mapping of \(\Ord^2\) onto \(\Ord\), and
\begin{equation*}
(\alpha,\beta)<(\gamma,\delta) \quad\text{ iff }\quad \Gamma(\alpha,\beta)<\Gamma(\gamma,\delta)
\end{equation*}
Note that \(\Gamma(\omega\times\omega)=\omega\) and since \(\gamma(\alpha)=\Gamma(\alpha\times\alpha)\) is an increasing function of \(\alpha\), we
have \(\gamma(\alpha)\ge\alpha\) for every \(\alpha\). However, \(\gamma(\alpha)\) is also continuous, and so \(\Gamma(\alpha\times\alpha)=\alpha\) for
arbitrary large \(\alpha\). \label{Question3}

\begin{proof}[Proof of Theorem \ref{thm3.5}]
We will show that \(\Gamma(\omega_\alpha\times\omega_\alpha)=\omega_\alpha\). This is true for \(\alpha=0\). Thus let \(\alpha\) be the least ordinal
s.t. \(\Gamma(\omega_\alpha\times\omega_\alpha)\neq\omega_\alpha\). Let \(\beta,\gamma<\omega_\alpha\) be s.t. \(\Gamma(\beta,\gamma)=\omega_\alpha\). Pick \(\delta<\omega_\alpha\) s.t. \(\delta>\beta\)
and \(\delta>\gamma\). Since \(\delta\times\delta\) is an initial segment of \(\Ord\times\Ord\) in the canonical well-ordering
and contains \((\beta,\gamma)\), we have \(\Gamma(\delta\times\delta)\supset\omega_\alpha\), and so \(\abs{\delta\times\delta}\ge\aleph_\alpha\).
However \(\abs{\delta\times\delta}=\abs{\delta}\cdot\abs{\delta}\), and by the minimality of \(\alpha\), \(\abs{\delta}\cdot\abs{\delta}=\abs{\delta}<\aleph_\alpha\).
\end{proof}

As a corollary we have
\begin{equation*}
\aleph_\alpha+\aleph_\beta=\aleph_\alpha\cdot\aleph_\beta=\max\{\aleph_\alpha,\aleph_\beta\}
\end{equation*}
\subsection{Cofinality}
\label{sec:org56e0ac6}
Let \(\alpha>0\) be a limit ordinal. We say that an increasing \(\beta\)-sequence \(\la\alpha_\xi:\xi<\beta\ra\), \(\beta\) a limit
ordinal, is \textbf{cofinal} in \(\alpha\) if \(\lim_{\xi\to\beta}\alpha_\xi=\alpha\). Similarly, \(A\subset\alpha\) is \textbf{cofinal} in \(\alpha\)
if \(\sup A=\alpha\). If \(\alpha\) is an infinite limit ordinal, the \textbf{cofinality} of \(\alpha\) is
\begin{align*}
\cf\alpha=\;&\text{the least limit ordinal $\beta$ s.t. there is an increasing}\\
&\beta\text{-sequence }\la\alpha_\xi:\xi<\beta\ra\text{ with }\lim_{\xi\to\beta}\alpha_\xi=\alpha
\end{align*}

\begin{lemma}
\(\cf(\cf\alpha)=\cf\alpha\)
\end{lemma}

\begin{lemma}[]
\label{lemma3.7}
Let \(\alpha>0\) be a limit ordinal
\begin{enumerate}
\item if \(A\subset\alpha\) and \(\sup A=\alpha\), then the order-type of \(A\) is at least \(\cf\alpha\)
\item if \(\beta_0\le\beta_1\le\dots\le\beta_\xi\le\dots\), \(\xi<\gamma\), is a nondecreasing \(\gamma\)-sequence of ordinals in \(\alpha\)
and \(\lim_{\xi\to\gamma}\beta_\xi=\alpha\), then \(\cf\gamma=\cf\alpha\)
\end{enumerate}
\end{lemma}

\begin{proof}
\begin{enumerate}
\item the order-type of \(A\) is the length of the increasing enumeration of \(A\) which is an
increasing sequence with limit \(\alpha\)
\item if \(\gamma=\lim_{\nu\to\cf\gamma}\xi(\nu)\), then \(\alpha=\lim_{\nu\to\cf\gamma}\beta_{\xi(\nu)}\), and the nondecreasing sequence
\(\la\beta_{\xi(\nu)}:\nu<\cf\gamma\ra\) has an increasing subsequence of length \(\le\cf\gamma\), with the same limit.
Thus \(\cf\alpha\le\cf\gamma\)

To show that \(\cf\gamma\le\cf\alpha\), let \(\alpha=\lim_{\nu\to\cf\alpha}\alpha_\nu\). For each \(\nu<\cf\alpha\), let \(\xi(\nu)\) be
the least \(\xi\) greater than all \(\xi(\iota)\), \(\iota<\nu\), s.t. \(\beta_\xi>\alpha_\nu\).
Since \(\lim_{\nu\to\cf\alpha}\beta_{\xi(\nu)}=\alpha\), it follows that \(\lim_{\nu\to\cf\alpha}\xi(\nu)=\gamma\), and so \(\cf\gamma\le\cf\alpha\).
\end{enumerate}
\end{proof}

An infinite cardinal \(\aleph_\alpha\) is \textbf{regular} if \(\cf\omega_\alpha=\omega_\alpha\). It is \textbf{singular} if \(\cf\omega_\alpha<\omega_\alpha\)

\begin{lemma}[]
For every limit ordinal \(\alpha\), \(\cf\alpha\) is a regular cardinal
\end{lemma}

\begin{proof}
if \(\alpha\) is not a cardinal, then by an bijection \(f:\abs{\alpha}\sim\alpha\), we get a cofinal sequence in
\(\alpha\) of length \(\le\abs{\alpha}\), therefore \(\cf\alpha<\alpha\)

if \(\alpha\) is a cardinal, \(\cf\alpha=\alpha\) by Lemma \ref{lemma3.7}
\end{proof}

Let \(\kappa\) be a limit ordinal. A subset \(X\subset\kappa\) is \textbf{bounded} if \(\sup X<\kappa\), and \textbf{unbounded}
if \(\sup X=\kappa\)

\begin{lemma}[]
Let \(\kappa\) be an aleph
\begin{enumerate}
\item If \(X\subset\kappa\) and \(\abs{X}<\cf\kappa\) then \(X\) is bounded
\item If \(\lambda<\cf\kappa\) and \(f:\lambda\to\kappa\) then the range of \(f\) is bounded
\end{enumerate}
\end{lemma}

it follows from 1 that every unbounded subset of a regular cardinal has cardinality \(\kappa\)

\begin{proof}
\begin{enumerate}
\item Lemma \ref{lemma3.7}
\item if \(X=\ran f\), then \(\abs{X}\le\lambda\), then use 1.
\end{enumerate}
\end{proof}

There are arbitrary large singular cardinals. For each \(\alpha\), \(\aleph_{\alpha+\omega}\) is a singular cardinal of
cofinality \(\omega\)

\begin{lemma}[]
An infinite cardinal \(\kappa\) is singular iff there exists a cardinal \(\lambda<\kappa\) and a
family \(\{S_\xi:\xi<\lambda\}\) of subsets of \(\kappa\) s.t. \(\abs{S_\xi}<\kappa\) for each \(\xi<\lambda\),
and \(\kappa=\bigcup_{\xi<\lambda}S_\xi\). The least cardinal \(\lambda\) satisfies the condition is \(\cf\kappa\).
\end{lemma}

\begin{proof}
If \(\kappa\) is singular, then there is an increasing sequence \(\la\alpha_\xi:\xi<\cf\kappa\ra\) with \(\lim_\xi\alpha_\xi=\kappa\).
Let \(\lambda=\cf\kappa\), and \(S_\xi=\alpha_\xi\) for all \(\xi<\lambda\).

If the condition holds, let \(\lambda<\kappa\) be the least cardinal for which there is a
family \(\{S_\xi:\xi<\lambda\}\) s.t. \(\kappa=\bigcup_{\xi<\lambda}S_\xi\) and \(\abs{S_\xi}<\kappa\) for each \(\xi<\lambda\). For
every \(\xi<\lambda\), let \(\beta_\xi\) be the order-type of \(\bigcup_{\nu<\xi}S_\nu\). The sequence \(\la\beta_\xi:\xi<\lambda\ra\) is
nondecreasing, and by the minimality of \(\lambda\), \(\beta_\xi<\kappa\) for all \(\xi<\lambda\). If not, then \(\beta_\xi=\kappa\)
and \(\bigcup_{\nu<\xi}S_\nu=\kappa\). We shall show
that \(\lim_\xi\beta_\xi=\kappa\), thus proving that \(\cf\kappa\le\lambda\).

Let \(\beta=\lim_{\xi\to\lambda}\beta_\xi\). There is a one-to-one mapping \(f\) of \(\kappa=\bigcup_{\xi<\lambda}S_\xi\) into \(\lambda\times\beta\):
if \(\alpha\in\kappa\), let \(f(\alpha)=(\xi,\gamma)\), where \(\xi\) is the least \(\xi\) s.t. \(\alpha\in S_\xi\) and \(\gamma\) is the order type
of \(S_\xi\cap\alpha\). Since \(\lambda<\kappa\) and \(\abs{\lambda\times\beta}=\lambda\cdot\abs{\beta}\), it follows that \(\beta=\kappa\)
\end{proof}

\begin{theorem}[]
If \(\kappa\) is an infinite cardinal, then \(\kappa<\kappa^{\cf\kappa}\)
\end{theorem}

\begin{proof}
Let \(F\) be a collection of \(\kappa\) functions from \(\cf\kappa\) to \(\kappa\): \(F=\{f_\alpha:\alpha<\kappa\}\). It is enough
to find \(f:\cf\kappa\to\kappa\) that is different from all the \(f_\alpha\). Let \(\kappa=\lim_{\xi\to\cf\kappa}\alpha_\xi\).
For \(\xi<\cf\kappa\), let
\begin{equation*}
f(\xi)=\text{ least $\gamma$ s.t. }\gamma\neq f_\alpha(\xi)\text{ for all }\alpha<\alpha_\xi
\end{equation*}
Such \(\gamma\) exists since \(\abs{\{f_\alpha(\xi):\alpha<\alpha_\xi\}}\le\abs{\alpha_\xi}<\kappa\). Obviously, \(f\neq f_\alpha\) for all \(\alpha<\kappa\)
\end{proof}

Consequently \(\kappa^\lambda>\kappa\) whenever \(\lambda\ge\cf\kappa\).

\section{Real Numbers}
\label{sec:orgbfc862a}
\begin{theorem}[Cantor]
The set of all real numbers is uncountable
\end{theorem}

\begin{proof}
Suppose not, let \(c_0,c_1,\dots\) be an enumeration of \(\R\)

Let \(a_0=c_0\) and \(b_0=c_{k_0}\), where \(k_0\) is the least \(k\) s.t. \(a_0<c_k\). For
each \(n\), let \(a_{n+1}=c_{i_n}\) where \(i_n\) is the least \(i\) s.t. \(a_n<c_i<b_n\)
and \(b_{n+1}=c_{k_n}\) where \(k_n\) is the least \(k\) s.t. \(a_{n+1}<c_k<b_n\). If we
let \(a=\sup\{a_n:n\in\N\}\), then \(a\neq c_k\) for all \(k\).
\end{proof}

\subsection{The Cardinality of the Continuum}
\label{sec:org68480f2}
Let \(\fc\) denote the cardinality of \(\R\). As the set \(\Q\) of all rational numbers is dense
in \(\R\), every real number \(r\) is equal to \(\sup\{q\in\Q:q<r\}\) and because \(\Q\) is countable,
it follows that \(\fc\le\abs{P(\Q)}=2^{\aleph_0}\)

\section{Question}
\label{sec:org5353d19}
\ref{Question1}
\ref{Question2}
\ref{Question3}
\end{document}
