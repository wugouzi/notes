% Created 2021-09-07 Tue 12:41
% Intended LaTeX compiler: pdflatex
\documentclass[11pt]{article}
\usepackage[utf8]{inputenc}
\usepackage[T1]{fontenc}
\usepackage{graphicx}
\usepackage{grffile}
\usepackage{longtable}
\usepackage{wrapfig}
\usepackage{rotating}
\usepackage[normalem]{ulem}
\usepackage{amsmath}
\usepackage{textcomp}
\usepackage{amssymb}
\usepackage{capt-of}
\usepackage{hyperref}
\graphicspath{{../../books/}}
% TIPS
% \substack{a\\b} for multiple lines text





% pdfplots will load xolor automatically without option
\usepackage[dvipsnames]{xcolor}

\usepackage{forest}
% two-line text in node by [two \\ lines]
% \begin{forest} qtree, [..] \end{forest}
\forestset{
  qtree/.style={
    baseline,
    for tree={
      parent anchor=south,
      child anchor=north,
      align=center,
      inner sep=1pt,
    }}}
%\usepackage{flexisym}
% load order of mathtools and mathabx, otherwise conflict overbrace

\usepackage{mathtools}
%\usepackage{fourier}
\usepackage{pgfplots}
\usepackage{amsthm, mathabx,  amsmath, commath}
\usepackage{amsfonts}

\usepackage{empheq}
\usepackage{tikz}
\usetikzlibrary{arrows.meta}
\usepackage[most]{tcolorbox}

\newtheorem{theorem}{Theorem}[section]
\newtheorem{definition}{Definition}[section]
\newtheorem{corollary}{Corollary}[section]
\newtheorem{example}{Example}[section]
\newtheorem{lemma}{Lemma}[section]
\newtheorem{proposition}{Proposition}[section]

\newcommand{\bl}[1] {\boldsymbol{#1}}
\newcommand{\Wt}[1] {\stackrel{\sim}{\smash{#1}\rule{0pt}{1.1ex}}}
\newcommand{\wt}[1] {\widetilde{#1}}


%For boxed texts in align, use Aboxed{}
%otherwise use boxed{}

\DeclareMathSymbol{\widehatsym}{\mathord}{largesymbols}{"62}
\newcommand\lowerwidehatsym{%
  \text{\smash{\raisebox{-1.3ex}{%
    $\widehatsym$}}}}
\newcommand\fixwidehat[1]{%
  \mathchoice
    {\accentset{\displaystyle\lowerwidehatsym}{#1}}
    {\accentset{\textstyle\lowerwidehatsym}{#1}}
    {\accentset{\scriptstyle\lowerwidehatsym}{#1}}
    {\accentset{\scriptscriptstyle\lowerwidehatsym}{#1}}
}

\usepackage{graphicx}
    
% text on arrow for xRightarrow
\makeatletter
%\newcommand{\xRightarrow}[2][]{\ext@arrow 0359\Rightarrowfill@{#1}{#2}}
\makeatother


\def \bx {\boldsymbol{x}}
\def \ba {\boldsymbol{a}}
\def \bI {\boldsymbol{I}}
\def \bt {\boldsymbol{t}}
\def \bb {\boldsymbol{b}}
\def \bA {\boldsymbol{A}}
\def \bX {\boldsymbol{X}}
\def \bu {\boldsymbol{u}}
\def \bS {\boldsymbol{S}}
\def \bZ {\boldsymbol{Z}}
\def \bz {\boldsymbol{z}}
\def \by {\boldsymbol{y}}
\def \bw {\boldsymbol{w}}
\def \bT {\boldsymbol{T}}
\def \bS {\boldsymbol{S}}
\def \bm {\boldsymbol{m}}
\def \bW {\boldsymbol{W}}
\def \bY {\boldsymbol{Y}}
\def \bH {\boldsymbol{H}}
\def \blambda {\boldsymbol{\lambda}}
\def \bPhi {\boldsymbol{\Phi}}
\def \btheta {\boldsymbol{\theta}}
\def \bmu {\boldsymbol{\mu}}
\def \bphi {\boldsymbol{\phi}}
\def \bSigma {\boldsymbol{\Sigma}}
\def \lb {\left\{}
\def \rb {\right\}}
\def \caln {\mathcal{N}}
\def \dissum {\displaystyle\Sigma}
\def \dispro {\displaystyle\prod}
\def \E {\mathbb{E}}
\def \Q {\mathbb{Q}}
\def \V {\mathbb{V}}
\def \R {\mathbb{R}}
\def \calq {\mathcal{Q}}
\def \calg {\mathcal{G}}
\def \caln {\mathcal{N}}
\def \calr {\mathcal{R}}
\def \calm {\mathcal{M}}
\def \calc {\mathcal{C}}
\def \bcup {\bigcup}

\makeindex
\author{Thomas Jech}
\date{\today}
\title{Set Theory}
\hypersetup{
 pdfauthor={Thomas Jech},
 pdftitle={Set Theory},
 pdfkeywords={},
 pdfsubject={},
 pdfcreator={Emacs 27.2 (Org mode 9.5)}, 
 pdflang={English}}
\begin{document}

\maketitle
\tableofcontents


\section{Ordinal Numbers}
\label{sec:org0d44a6d}

\subsection{Linear and Partial Ordering}
\label{sec:org020778f}
\begin{definition}[]
A binary relation < on a set \(P\) is a \textbf{partial ordering} of \(P\) if
\begin{enumerate}
\item \(p\not<p\) for any \(p\in P\)
\item if \(p<q\) and \(q<r\) then \(p<r\)
\end{enumerate}


\((P,<)\) is called a \textbf{partially ordered set}. A partial ordering < of \(P\) is a \textbf{linear ordering}
if moreover
\begin{enumerate}
\setcounter{enumi}{2}
\item \(p<q\) or \(p=q\) or \(q<p\) for all \(p,q\in P\)

If < is a partial ordering, then \(\le\) is also a partial ordering
\end{enumerate}
\end{definition}

if \((P,<)\) and \((Q,<)\) are partially ordered sets and \(f:P\to Q\), then \(f\) is
\textbf{order-preserving} if \(x<y\) implies \(f(x)<f(y)\). If \(P\) and \(Q\) are linearly ordered, then
an order-preserving function is also called \textbf{increasing}

\subsection{Well-Ordering}
\label{sec:org9f139bc}
\begin{definition}[]
A linear ordering < of a set \(P\) is a \textbf{well-ordering} if every nonempty subset of \(P\) has a
least element
\end{definition}

\begin{lemma}[]
\label{lemma2.4}
If \((W,<)\) is a well-ordered set and \(f:W\to W\) is an increasing function, then \(f(x)\ge x\)
for each \(x\in W\)
\end{lemma}

\begin{proof}
Assume that the set \(X=\{x\in W:f(x)<x\}\) is nonempty and let \(z\) be the least element of \(X\).
If \(w=f(z)\), then \(f(w)<w\), a contradiction
\end{proof}

\begin{corollary}[]
The only automorphism of a well-ordered set is the identity
\end{corollary}

\begin{proof}
By Lemma \ref{lemma2.4}, \(f(x)\ge x\) for all \(x\), and \(f^{-1}(x)\ge x\) for all \(x\)
\end{proof}

\begin{corollary}[]
If two well-ordered sets \(W_1,W_2\) are isomorphic, then the isomorphism of \(W_1\) onto \(W_2\) is unique
\end{corollary}

if \(W\) is a well-ordered set and \(u\in W\), then \(\{x\in W:x<u\}\) is an \textbf{initial segment} of \(W\)

\begin{lemma}[]
\label{lemma2.7}
No well-ordered set is isomorphic to an initial segment of itself
\end{lemma}

\begin{proof}
If \(\ran(f)=\{x:x<u\}\), then \(f(u)<u\), contrary to Lemma \ref{lemma2.4}
\end{proof}

\begin{theorem}[]
\label{thm2.8}
If \(W_1\) and \(W_2\) are well-ordered sets, then exactly one of the following three cases holds
\begin{enumerate}
\item \(W_1\) is isomorphic to \(W_2\)
\item \(W_1\) is isomorphic to an initial segment of \(W_2\)
\item \(W_2\) is isomorphic to an initial segment of \(W_1\)
\end{enumerate}
\end{theorem}

\begin{proof}
For \(u\in W_i\), (\(i=1,2\)), let \(W_i(u)\) denote the initial segment of \(W_i\) given by \(u\).
Let
\begin{equation*}
f=\{(x,y)\in W_1\times W_2:W_1(x)\text{ is isomorphic to }W_2(y)\}
\end{equation*}
Using Lemma \ref{lemma2.7}, \(f\) is a injective: if \(f(x_1)=f(x_2)=y\),
then \(W_1(x_1)\cong W_2(y)\cong W_1(x_2)\), and \(x_1<x_2\) or \(x_2<x_1\) fail. If \(h\) is an isomorphism
between \(W_1(x)\) and \(W_2(y)\), and \(x'<x\), then \(W_1(x')\) and \(W_2(h(x'))\) are isomorphic.
It follows that \(f\) is order-preserving

If \(\dom(f)=W_1\) and \(\ran(f)=W_2\), then case 1 holds

if \(y_1<y_2\) and \(y_2\in\ran(f)\), then \(y_1\in\ran(f)\). Thus if \(\ran(f)\neq W_2\) and \(y_0\) is the
least element of \(W_2-\ran(f)\), we have \(\ran(f)=W_2(y_0)\). Necessarily, \(\dom(f)=W_1\), for
otherwise we would have \((x_0,y_0)\in f\), where \(x_0\)=the least element of \(W_1-\dom(f)\)
\end{proof}

if \(W_1\) and \(W_2\) are isomorphic, we say that they have the same \textbf{order-type}.

\subsection{Ordinal Numbers}
\label{sec:orgaeea01f}
\begin{definition}[]
A set \(T\) is \textbf{transitive} if every element of \(T\) is a subset of \(T\)
\end{definition}

\begin{definition}[]
A set is an \textbf{ordinal number} (an \textbf{ordinal}) if it is transitive and well-ordered by \(\in\)
\end{definition}

Define
\begin{equation*}
\alpha<\beta \quad\text{ iff }\quad \alpha\in\beta
\end{equation*}

\begin{lemma}[]
\label{lemma2.11}
\begin{enumerate}
\item \(0=\emptyset\) is an ordinal
\item if \(\alpha\) is an ordinal and \(\beta\in\alpha\), then \(\beta\) is an ordinal
\item if \(\alpha\neq\beta\) are ordinals and \(\alpha\subset\beta\), then \(\alpha\in\beta\)
\item if \(\alpha\), \(\beta\) are ordinals, then either \(\alpha\subset\beta\) or \(\beta\subset\alpha\)
\end{enumerate}
\end{lemma}

\begin{proof}
1,2 by definition

\begin{enumerate}
\setcounter{enumi}{2}
\item if \(\alpha\subset\beta\), let \(\gamma\) be the least element of the set \(\beta-\alpha\). Since \(\alpha\) is transitive, it
follows that \(\alpha\) is the initial segment of \(\beta\) given by \(\gamma\): for \(\eta\in\alpha\), \(\eta\neq\gamma\) and \(\gamma\not\in\eta\),
hence \(\eta\in\gamma\) since ordinals are well-ordered by \(\in\). Thus \(\alpha=\{\xi\in\beta:\xi<\gamma\}=\gamma\), and
so \(\alpha\in\beta\).
\item \(\alpha\cap\beta\) is an ordinal, \(\alpha\cap\beta=\gamma\). We have \(\gamma=\alpha\) or \(\gamma=\beta\), for otherwise \(\gamma\in\alpha\)
and \(\gamma\in\beta\), by 3. Then \(\gamma\in\gamma\), which contradicts the definition of an ordinal (namely
that \(\in\) is a \textbf{strict} ordering of \(\alpha\))
\end{enumerate}
\end{proof}

Using Lemma \ref{lemma2.11} one gets the followings
\begin{enumerate}
\item < is a linear ordering of the class \(\Ord\)
\item for each \(\alpha\), \(\alpha=\{\beta:\beta<\alpha\}\) \label{Question1}
\item if \(C\) is a nonempty class of ordinals, then \(\bigcap C\) is an ordinal, \(\bigcap C\in C\)
and \(\bigcap C=\inf C\)
\item if \(X\) is a nonempty set of ordinals, then \(\bigcup X\) is an ordinal, and \(\bigcup X=\sup X\)
\item for every \(\alpha\), \(\alpha\cup\{\alpha\}\) is an ordinal and \(\alpha\cup\{\alpha\}=\inf\{\beta:\beta>\alpha\}\)
\end{enumerate}


We thus define \(\alpha+1=\alpha\cup\{\alpha\}\). In view of 4, the class \(\Ord\) is a proper class; otherwise
consider \(\sup\Ord+1\)

\begin{theorem}[]
Every well-ordered set is isomorphic to a unique ordinal number
\end{theorem}

\begin{proof}
The uniqueness follows from Lemma \ref{lemma2.7}: suppose \(\alpha\cong\beta\) and \(\alpha\neq\beta\)

Given a well-ordered set \(W\),
define \(F(x)=\alpha\) is \(\alpha\) is isomorphic to the initial segment of \(W\) given by \(x\). If such an
\(\alpha\) exists, then it is unique. By the Replacement Axioms, \(F(W)\) is a set. For each \(x\in W\),
such an \(\alpha\) exists (otherwise consider the least \(x\) for which such an \(\alpha\) does not exists). If
\(\gamma\) is the least \(\gamma\not\in F(W)\), then \(F(W)=\gamma\) and we have an isomorphism of \(W\) onto \(\gamma\)
\end{proof}

0 is a limit ordinal and define \(\sup\emptyset=0\)

\begin{definition}[Natural Numbers]
We denote the least nonzero limit ordinal \(\omega\) (or \(\N\)). The ordinals less than \(\omega\) are call
\textbf{finite ordinals}, or \textbf{natural numbers}
\end{definition}

\subsection{Induction and Recursion}
\label{sec:org8edf82d}
\begin{theorem}[Transfinite Induction]
Let \(C\) be a class of ordinals and assume that
\begin{enumerate}
\item \(0\in C\)
\item if \(\alpha\in C\), then \(\alpha+1\in C\)
\item if \(\alpha\) is a nonzero limit ordinal and \(\beta\in C\) for all \(\beta<\alpha\), then \(\alpha\in C\)
\end{enumerate}


Then \(C\) is the class of all ordinals
\end{theorem}

\begin{proof}
Otherwise, let \(\alpha\) be the least \(\alpha\not\in C\) and apply 1,2 and 3.
\end{proof}

A function whose domain is the set \(\N\) is called an (\textbf{infinite}) \textbf{sequence} (A \textbf{sequence in} \(X\)
is a function \(f:\N\to X\)). The standard notation for a sequence is
\begin{equation*}
\la a_n:n<\omega\ra
\end{equation*}
A \textbf{finite sequence} is a function \(s\) s.t. \(\dom(s)=\{i:i<n\}\) for some \(n\in\N\); then \(s\) is a
\textbf{sequence  of length} \(n\)

A \textbf{transfinite sequence} is a function whose domain is an ordinal
\begin{equation*}
\la a_\xi:\xi<\alpha\ra
\end{equation*}
It is also called an \textbf{\(\alpha\)-sequence} or a \textbf{sequence of length} \(\alpha\). We also say that a
sequence \(\la a_\xi:\xi<\alpha\ra\) is an \textbf{enumeration} of its range \(\{a_\xi:\xi<\alpha\}\). If \(s\) is a sequence of
length \(\alpha\), then \(s^\smallfrown x\) or simply \(sx\) denotes the sequence of length \(\alpha+1\) that
extends \(s\) and whose \(\alpha\)th term is \(x\):
\begin{equation*}
s^\smallfrown x=sx=s\cup\{(\alpha,x)\}
\end{equation*}
Sometimes we call a ``sequence''
\begin{equation*}
\la a_\alpha:\alpha\in\Ord\ra
\end{equation*}
a function (a proper class) on \(\Ord\)


``Definition by transfinite recursion'' usually takes the following form: Given a function \(G\)
(on the class of transfinite sequence), then for every \(\theta\) there exists a unique \(\theta\)-sequence
\begin{equation*}
\la a_\alpha:\alpha<\theta\ra
\end{equation*}
s.t.
\begin{equation*}
a_\alpha=G(\la a_\xi:\xi<\alpha\ra)
\end{equation*}
for every \(\alpha<\theta\)

\begin{theorem}[Transfinite Recursion]
Let \(G\) be a function (on \(V\)), then \eqref{2.6} below defines a unique function \(F\)
on \(\Ord\) s.t.
\begin{equation*}
F(\alpha)=G(F\restriction\alpha)
\end{equation*}
for each \(\alpha\)
\end{theorem}

In other words, if we let \(a_\alpha=F(\alpha)\), then for each \(\alpha\)
\begin{equation*}
a_\alpha=G(\la a_\xi:\xi<\alpha\ra)
\end{equation*}
(Note that we tacitly use Replacement: \(F\restriction\alpha\) is a set for each \(\alpha\))

\begin{corollary}[]
Let \(X\) be a set and \(\theta\) an ordinal number. For every function \(G\) on the set of all
transfinite sequences in \(X\) of length \(<\theta\) s.t. \(\ran(G)\subset X\) there exists a unique
\(\theta\)-sequence \(\la a_\alpha:\alpha<\theta\ra\) in \(X\) s.t. \(a_\alpha=G(\la a_\xi:\xi<\alpha\ra)\) for every \(\alpha<\theta\)
\end{corollary}

\begin{proof}
Let
\begin{align}
\label{2.6}
F(\alpha)=x\leftrightarrow&\text{ there is a sequence }\la a_\xi:\xi<\alpha\ra\text{ s.t.:}\\
&1.\; (\forall\xi<\alpha)a_\xi=G(\la a_\eta:\eta<\xi\ra)\nonumber\\
&2.\; x=G(\la a_\xi:\xi<\alpha\ra)\nonumber
\end{align}
For every \(\alpha\), if there is an \(\alpha\)-sequence that satisfies 1, then such a sequence is unique:
if \(\la a_\xi:\xi<\alpha\ra\) and \(\la b_\xi:\xi<\alpha\ra\) are two \(\alpha\)-sequences satisfying 1, one shows \(a_\xi=b_\xi\) by
induction on \(\xi\).Thus \(F(\alpha)\) is determined uniquely by 2, and therefore \(F\) is a function.

it follows, again by induction, that for each \(\alpha\) there is an \(\alpha\)-sequence that satisfies 1 (at
limit steps, we use Replacement to get the \(\alpha\)-sequence as the union of all the
\(\xi\)-sequences, \(\xi<\alpha\)). Thus \(F\) is defined for all \(\alpha\in\Ord\). It obviously satisfies
\begin{equation*}
F(\alpha)=G(F\restriction\alpha)
\end{equation*}

If \(F'\) is any function on \(\Ord\) that satisfies
\begin{equation*}
F'(\alpha)=G(F'\restriction\alpha)
\end{equation*}
then it follows by induction that \(F'(\alpha)=F(\alpha)\) for all \(\alpha\)
\end{proof}

\begin{definition}[]
Let \(\alpha>0\) be a limit ordinal and let \(\la\gamma_\xi:\xi<\alpha\ra\) be a \textbf{nondecreasing} sequence of ordinals.
We define the \textbf{limit} of the sequence by
\begin{equation*}
\lim_{\xi\to\alpha}\gamma_\xi=\sup\{\gamma_\xi:\xi<\alpha\}
\end{equation*}

A sequence of ordinals \(\la\gamma_\alpha:\alpha\in\Ord\ra\) is \textbf{normal} if it is increasing and \textbf{continuous}, i.e., for
every limit \(\alpha\),\(\gamma_\alpha=\lim_{\xi\to\alpha}\gamma_\xi\)
\end{definition}

\subsection{Ordinal Arithmetic}
\label{sec:orgf9f662b}
\begin{definition}[Addition]
For all ordinal numbers \(\alpha\)
\begin{enumerate}
\item \(\alpha+0=\alpha\)
\item \(\alpha+(\beta+1)=(\alpha+\beta)+1\) for all \(\beta\)
\item \(\alpha+\beta=\lim_{\xi\to\beta}(\alpha+\xi)\) for all limit \(\beta>0\)
\end{enumerate}
\end{definition}

\begin{definition}[Multiplication]
For all ordinal numbers \(\alpha\)
\begin{enumerate}
\item \(\alpha\cdot 0=0\)
\item \(\alpha\cdot(\beta+1)=\alpha\cdot\beta+\alpha\) for all \(\beta\)
\item \(\alpha\cdot\beta=\lim_{\xi\to\beta}\alpha\cdot\xi\) for all limit \(\beta>0\)
\end{enumerate}
\end{definition}

\begin{definition}[Exponentiation]
For all ordinal numbers \(\alpha\)
\begin{enumerate}
\item \(\alpha^0=1\)
\item \(\alpha^{\beta+1}=\alpha^\beta\cdot\alpha\) for all \(\beta\)
\item \(\alpha^\beta=\lim_{\xi\to\beta}\alpha^\xi\) for all limit \(\beta>0\)
\end{enumerate}
\end{definition}

\begin{lemma}[]
For all ordinals \(\alpha\), \(\beta\) and \(\gamma\)
\begin{enumerate}
\item \(\alpha+(\beta+\gamma)=(\alpha+\beta)+\gamma\)
\item \(\alpha\cdot(\beta\cdot\gamma)=(\alpha\cdot\beta)\cdot\gamma\)
\end{enumerate}
\end{lemma}

\begin{proof}
Induction on \(\gamma\)
\end{proof}

Neither + nor \(\cdot\) are commutative:
\begin{equation*}
1+\omega=\omega\neq\omega+1,\quad 2\cdot\omega=\omega\neq\omega\cdot 2=\omega+\omega
\end{equation*}

\begin{definition}[]
Let \((A,<_A)\) and \((B,<_B)\) be disjoint linearly ordered sets. The \textbf{sum} of these linear
orders is the set \(A\cup B\) with the ordering defined as follows: \(x<y\) iff
\begin{enumerate}
\item \(x,y\in A\) and \(x<_Ay\), or
\item \(x,y\in B\) and \(x<_By\), or
\item \(x\in A\) and \(y\in B\)
\end{enumerate}
\end{definition}

\begin{definition}[]
Let \((A,<)\) and \((B,<)\) be linearly ordered sets. The \textbf{product} of these linear orders is the
set \(A\times B\) with the ordering defined by
\begin{equation*}
(a_1,b_1)<(a_2,b_2)\text{ iff either }b_1<b_2\text{ or }(b_1=b_2\text{ and }a_1<a_2)
\end{equation*}
\end{definition}

\begin{lemma}[]
\label{Question2}
For all ordinals \(\alpha\) and \(\beta\), \(\alpha+\beta\) and \(\alpha\cdot\beta\) are isomorphic to the sum and product of \(\alpha\) and \(\beta\)
\end{lemma}

\begin{proof}
We can define \(S(\alpha,\beta)=\{(0,a):a\in\alpha\}\cup\{(1,b)\in\beta\}\)

if \(\beta=0\), then \(S(\alpha,\beta)=\alpha\)

if \(\beta=\eta+1\), then \(S(\alpha,\beta)=S(\alpha,\eta)\cup\{(1,\eta)\}\)
\end{proof}

\begin{lemma}[]
\begin{enumerate}
\item if \(\beta<\gamma\) then \(\alpha+\beta<\alpha+\gamma\)
\item if \(\alpha<\beta\) then there exists a unique \(\delta\) s.t. \(\alpha+\delta=\beta\)
\item 
\end{enumerate}
\end{lemma}

\begin{proof}
\begin{enumerate}
\item induction on \(\gamma\)
\item let \(\delta\) be the order-type of the set \(\{\xi:\alpha\le\xi<\beta\}\); \(\delta\) is unique by 1
\end{enumerate}
\end{proof}



\section{Question}
\label{sec:org472873f}
\ref{Question1}
\ref{Question2}
\end{document}
