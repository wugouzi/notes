% Created 2022-05-03 Tue 20:34
% Intended LaTeX compiler: pdflatex
\documentclass[11pt]{article}
\usepackage[utf8]{inputenc}
\usepackage[T1]{fontenc}
\usepackage{graphicx}
\usepackage{longtable}
\usepackage{wrapfig}
\usepackage{rotating}
\usepackage[normalem]{ulem}
\usepackage{amsmath}
\usepackage{amssymb}
\usepackage{capt-of}
\usepackage{hyperref}
\graphicspath{{../../books/}}
% wrong resolution of image
% https://tex.stackexchange.com/questions/21627/image-from-includegraphics-showing-in-wrong-image-size?rq=1

%%%%%%%%%%%%%%%%%%%%%%%%%%%%%%%%%%%%%%
%% TIPS                                 %%
%%%%%%%%%%%%%%%%%%%%%%%%%%%%%%%%%%%%%%
% \substack{a\\b} for multiple lines text
% \usepackage{expl3}
% \expandafter\def\csname ver@l3regex.sty\endcsname{}
% \usepackage{pkgloader}
\usepackage[utf8]{inputenc}

% nfss error
% \usepackage[B1,T1]{fontenc}
\usepackage{fontspec}

% \usepackage[Emoticons]{ucharclasses}
\newfontfamily\DejaSans{DejaVu Sans}
% \setDefaultTransitions{\DejaSans}{}

% pdfplots will load xolor automatically without option
\usepackage[dvipsnames]{xcolor}

%                                                             ┳┳┓   ┓
%                                                             ┃┃┃┏┓╋┣┓
%                                                             ┛ ┗┗┻┗┛┗
% \usepackage{amsmath} mathtools loads the amsmath
\usepackage{amsmath}
\usepackage{mathtools}

\usepackage{amsthm}
\usepackage{amsbsy}

%\usepackage{commath}

\usepackage{amssymb}

\usepackage{mathrsfs}
%\usepackage{mathabx}
\usepackage{stmaryrd}
\usepackage{empheq}

\usepackage{scalerel}
\usepackage{stackengine}
\usepackage{stackrel}



\usepackage{nicematrix}
\usepackage{tensor}
\usepackage{blkarray}
\usepackage{siunitx}
\usepackage[f]{esvect}

% centering \not on a letter
\usepackage{slashed}
\usepackage[makeroom]{cancel}

%\usepackage{merriweather}
\usepackage{unicode-math}
\setmainfont{TeX Gyre Pagella}
% \setmathfont{STIX}
%\setmathfont{texgyrepagella-math.otf}
%\setmathfont{Libertinus Math}
\setmathfont{Latin Modern Math}

 % \setmathfont[range={\smwhtdiamond,\enclosediamond,\varlrtriangle}]{Latin Modern Math}
\setmathfont[range={\rightrightarrows,\twoheadrightarrow,\leftrightsquigarrow,\triangledown,\vartriangle,\precneq,\succneq,\prec,\succ,\preceq,\succeq,\tieconcat}]{XITS Math}
 \setmathfont[range={\int,\setminus}]{Libertinus Math}
 % \setmathfont[range={\mathalpha}]{TeX Gyre Pagella Math}
%\setmathfont[range={\mitA,\mitB,\mitC,\mitD,\mitE,\mitF,\mitG,\mitH,\mitI,\mitJ,\mitK,\mitL,\mitM,\mitN,\mitO,\mitP,\mitQ,\mitR,\mitS,\mitT,\mitU,\mitV,\mitW,\mitX,\mitY,\mitZ,\mita,\mitb,\mitc,\mitd,\mite,\mitf,\mitg,\miti,\mitj,\mitk,\mitl,\mitm,\mitn,\mito,\mitp,\mitq,\mitr,\mits,\mitt,\mitu,\mitv,\mitw,\mitx,\mity,\mitz}]{TeX Gyre Pagella Math}
% unicode is not good at this!
%\let\nmodels\nvDash

 \usepackage{wasysym}

 % for wide hat
 \DeclareSymbolFont{yhlargesymbols}{OMX}{yhex}{m}{n} \DeclareMathAccent{\what}{\mathord}{yhlargesymbols}{"62}

%                                                               ┏┳┓•┓
%                                                                ┃ ┓┃┏┓
%                                                                ┻ ┗┛┗┗

\usepackage{pgfplots}
\pgfplotsset{compat=1.18}
\usepackage{tikz}
\usepackage{tikz-cd}
\tikzcdset{scale cd/.style={every label/.append style={scale=#1},
    cells={nodes={scale=#1}}}}
% TODO: discard qtree and use forest
% \usepackage{tikz-qtree}
\usepackage{forest}

\usetikzlibrary{arrows,positioning,calc,fadings,decorations,matrix,decorations,shapes.misc}
%setting from geogebra
\definecolor{ccqqqq}{rgb}{0.8,0,0}

%                                                          ┳┳┓•    ┓┓
%                                                          ┃┃┃┓┏┏┏┓┃┃┏┓┏┓┏┓┏┓┓┏┏
%                                                          ┛ ┗┗┛┗┗ ┗┗┗┻┛┗┗ ┗┛┗┻┛
%\usepackage{twemojis}
\usepackage[most]{tcolorbox}
\usepackage{threeparttable}
\usepackage{tabularx}

\usepackage{enumitem}
\usepackage[indLines=false]{algpseudocodex}
\usepackage[]{algorithm2e}
% \SetKwComment{Comment}{/* }{ */}
% \algrenewcommand\algorithmicrequire{\textbf{Input:}}
% \algrenewcommand\algorithmicensure{\textbf{Output:}}
% wrong with preview
\usepackage{subcaption}
\usepackage{caption}
% {\aunclfamily\Huge}
\usepackage{auncial}

\usepackage{float}

\usepackage{fancyhdr}

\usepackage{ifthen}
\usepackage{xargs}

\definecolor{mintedbg}{rgb}{0.99,0.99,0.99}
\usepackage[cachedir=\detokenize{~/miscellaneous/trash}]{minted}
\setminted{breaklines,
  mathescape,
  bgcolor=mintedbg,
  fontsize=\footnotesize,
  frame=single,
  linenos}
\usemintedstyle{xcode}
\usepackage{tcolorbox}
\usepackage{etoolbox}



\usepackage{imakeidx}
\usepackage{hyperref}
\usepackage{soul}
\usepackage{framed}

% don't use this for preview
%\usepackage[margin=1.5in]{geometry}
% \usepackage{geometry}
% \geometry{legalpaper, landscape, margin=1in}
\usepackage[font=itshape]{quoting}

%\LoadPackagesNow
%\usepackage[xetex]{preview}
%%%%%%%%%%%%%%%%%%%%%%%%%%%%%%%%%%%%%%%
%% USEPACKAGES end                       %%
%%%%%%%%%%%%%%%%%%%%%%%%%%%%%%%%%%%%%%%

%%%%%%%%%%%%%%%%%%%%%%%%%%%%%%%%%%%%%%%
%% Algorithm environment
%%%%%%%%%%%%%%%%%%%%%%%%%%%%%%%%%%%%%%%
\SetKwIF{Recv}{}{}{upon receiving}{do}{}{}{}
\SetKwBlock{Init}{initially do}{}
\SetKwProg{Function}{Function}{:}{}

% https://github.com/chrmatt/algpseudocodex/issues/3
\algnewcommand\algorithmicswitch{\textbf{switch}}%
\algnewcommand\algorithmiccase{\textbf{case}}
\algnewcommand\algorithmicof{\textbf{of}}
\algnewcommand\algorithmicotherwise{\texttt{otherwise} $\Rightarrow$}

\makeatletter
\algdef{SE}[SWITCH]{Switch}{EndSwitch}[1]{\algpx@startIndent\algpx@startCodeCommand\algorithmicswitch\ #1\ \algorithmicdo}{\algpx@endIndent\algpx@startCodeCommand\algorithmicend\ \algorithmicswitch}%
\algdef{SE}[CASE]{Case}{EndCase}[1]{\algpx@startIndent\algpx@startCodeCommand\algorithmiccase\ #1}{\algpx@endIndent\algpx@startCodeCommand\algorithmicend\ \algorithmiccase}%
\algdef{SE}[CASEOF]{CaseOf}{EndCaseOf}[1]{\algpx@startIndent\algpx@startCodeCommand\algorithmiccase\ #1 \algorithmicof}{\algpx@endIndent\algpx@startCodeCommand\algorithmicend\ \algorithmiccase}
\algdef{SE}[OTHERWISE]{Otherwise}{EndOtherwise}[0]{\algpx@startIndent\algpx@startCodeCommand\algorithmicotherwise}{\algpx@endIndent\algpx@startCodeCommand\algorithmicend\ \algorithmicotherwise}
\ifbool{algpx@noEnd}{%
  \algtext*{EndSwitch}%
  \algtext*{EndCase}%
  \algtext*{EndCaseOf}
  \algtext*{EndOtherwise}
  %
  % end indent line after (not before), to get correct y position for multiline text in last command
  \apptocmd{\EndSwitch}{\algpx@endIndent}{}{}%
  \apptocmd{\EndCase}{\algpx@endIndent}{}{}%
  \apptocmd{\EndCaseOf}{\algpx@endIndent}{}{}
  \apptocmd{\EndOtherwise}{\algpx@endIndent}{}{}
}{}%

\pretocmd{\Switch}{\algpx@endCodeCommand}{}{}
\pretocmd{\Case}{\algpx@endCodeCommand}{}{}
\pretocmd{\CaseOf}{\algpx@endCodeCommand}{}{}
\pretocmd{\Otherwise}{\algpx@endCodeCommand}{}{}

% for end commands that may not be printed, tell endCodeCommand whether we are using noEnd
\ifbool{algpx@noEnd}{%
  \pretocmd{\EndSwitch}{\algpx@endCodeCommand[1]}{}{}%
  \pretocmd{\EndCase}{\algpx@endCodeCommand[1]}{}{}
  \pretocmd{\EndCaseOf}{\algpx@endCodeCommand[1]}{}{}%
  \pretocmd{\EndOtherwise}{\algpx@endCodeCommand[1]}{}{}
}{%
  \pretocmd{\EndSwitch}{\algpx@endCodeCommand[0]}{}{}%
  \pretocmd{\EndCase}{\algpx@endCodeCommand[0]}{}{}%
  \pretocmd{\EndCaseOf}{\algpx@endCodeCommand[0]}{}{}
  \pretocmd{\EndOtherwise}{\algpx@endCodeCommand[0]}{}{}
}%
\makeatother
% % For algpseudocode
% \algnewcommand\algorithmicswitch{\textbf{switch}}
% \algnewcommand\algorithmiccase{\textbf{case}}
% \algnewcommand\algorithmiccaseof{\textbf{case}}
% \algnewcommand\algorithmicof{\textbf{of}}
% % New "environments"
% \algdef{SE}[SWITCH]{Switch}{EndSwitch}[1]{\algorithmicswitch\ #1\ \algorithmicdo}{\algorithmicend\ \algorithmicswitch}%
% \algdef{SE}[CASE]{Case}{EndCase}[1]{\algorithmiccase\ #1}{\algorithmicend\ \algorithmiccase}%
% \algtext*{EndSwitch}%
% \algtext*{EndCase}
% \algdef{SE}[CASEOF]{CaseOf}{EndCaseOf}[1]{\algorithmiccaseof\ #1 \algorithmicof}{\algorithmicend\ \algorithmiccaseof}
% \algtext*{EndCaseOf}



%\pdfcompresslevel0

% quoting from
% https://tex.stackexchange.com/questions/391726/the-quotation-environment
\NewDocumentCommand{\bywhom}{m}{% the Bourbaki trick
  {\nobreak\hfill\penalty50\hskip1em\null\nobreak
   \hfill\mbox{\normalfont(#1)}%
   \parfillskip=0pt \finalhyphendemerits=0 \par}%
}

\NewDocumentEnvironment{pquotation}{m}
  {\begin{quoting}[
     indentfirst=true,
     leftmargin=\parindent,
     rightmargin=\parindent]\itshape}
  {\bywhom{#1}\end{quoting}}

\indexsetup{othercode=\small}
\makeindex[columns=2,options={-s /media/wu/file/stuuudy/notes/index_style.ist},intoc]
\makeatletter
\def\@idxitem{\par\hangindent 0pt}
\makeatother


% \newcounter{dummy} \numberwithin{dummy}{section}
\newtheorem{dummy}{dummy}[section]
\theoremstyle{definition}
\newtheorem{definition}[dummy]{Definition}
\theoremstyle{plain}
\newtheorem{corollary}[dummy]{Corollary}
\newtheorem{lemma}[dummy]{Lemma}
\newtheorem{proposition}[dummy]{Proposition}
\newtheorem{theorem}[dummy]{Theorem}
\newtheorem{notation}[dummy]{Notation}
\newtheorem{conjecture}[dummy]{Conjecture}
\newtheorem{fact}[dummy]{Fact}
\newtheorem{warning}[dummy]{Warning}
\theoremstyle{definition}
\newtheorem{examplle}{Example}[section]
\theoremstyle{remark}
\newtheorem*{remark}{Remark}
\newtheorem{exercise}{Exercise}[subsection]
\newtheorem{problem}{Problem}[subsection]
\newtheorem{observation}{Observation}[section]
\newenvironment{claim}[1]{\par\noindent\textbf{Claim:}\space#1}{}

\makeatletter
\DeclareFontFamily{U}{tipa}{}
\DeclareFontShape{U}{tipa}{m}{n}{<->tipa10}{}
\newcommand{\arc@char}{{\usefont{U}{tipa}{m}{n}\symbol{62}}}%

\newcommand{\arc}[1]{\mathpalette\arc@arc{#1}}

\newcommand{\arc@arc}[2]{%
  \sbox0{$\m@th#1#2$}%
  \vbox{
    \hbox{\resizebox{\wd0}{\height}{\arc@char}}
    \nointerlineskip
    \box0
  }%
}
\makeatother

\setcounter{MaxMatrixCols}{20}
%%%%%%% ABS
\DeclarePairedDelimiter\abss{\lvert}{\rvert}%
\DeclarePairedDelimiter\normm{\lVert}{\rVert}%

% Swap the definition of \abs* and \norm*, so that \abs
% and \norm resizes the size of the brackets, and the
% starred version does not.
\makeatletter
\let\oldabs\abss
%\def\abs{\@ifstar{\oldabs}{\oldabs*}}
\newcommand{\abs}{\@ifstar{\oldabs}{\oldabs*}}
\newcommand{\norm}[1]{\left\lVert#1\right\rVert}
%\let\oldnorm\normm
%\def\norm{\@ifstar{\oldnorm}{\oldnorm*}}
%\renewcommand{norm}{\@ifstar{\oldnorm}{\oldnorm*}}
\makeatother

% \stackMath
% \newcommand\what[1]{%
% \savestack{\tmpbox}{\stretchto{%
%   \scaleto{%
%     \scalerel*[\widthof{\ensuremath{#1}}]{\kern-.6pt\bigwedge\kern-.6pt}%
%     {\rule[-\textheight/2]{1ex}{\textheight}}%WIDTH-LIMITED BIG WEDGE
%   }{\textheight}%
% }{0.5ex}}%
% \stackon[1pt]{#1}{\tmpbox}%
% }

% \newcommand\what[1]{\ThisStyle{%
%     \setbox0=\hbox{$\SavedStyle#1$}%
%     \stackengine{-1.0\ht0+.5pt}{$\SavedStyle#1$}{%
%       \stretchto{\scaleto{\SavedStyle\mkern.15mu\char'136}{2.6\wd0}}{1.4\ht0}%
%     }{O}{c}{F}{T}{S}%
%   }
% }

% \newcommand\wtilde[1]{\ThisStyle{%
%     \setbox0=\hbox{$\SavedStyle#1$}%
%     \stackengine{-.1\LMpt}{$\SavedStyle#1$}{%
%       \stretchto{\scaleto{\SavedStyle\mkern.2mu\AC}{.5150\wd0}}{.6\ht0}%
%     }{O}{c}{F}{T}{S}%
%   }
% }

% \newcommand\wbar[1]{\ThisStyle{%
%     \setbox0=\hbox{$\SavedStyle#1$}%
%     \stackengine{.5pt+\LMpt}{$\SavedStyle#1$}{%
%       \rule{\wd0}{\dimexpr.3\LMpt+.3pt}%
%     }{O}{c}{F}{T}{S}%
%   }
% }

\newcommand{\bl}[1] {\boldsymbol{#1}}
\newcommand{\Wt}[1] {\stackrel{\sim}{\smash{#1}\rule{0pt}{1.1ex}}}
\newcommand{\wt}[1] {\widetilde{#1}}
\newcommand{\tf}[1] {\textbf{#1}}

\newcommand{\wu}[1]{{\color{red} #1}}

%For boxed texts in align, use Aboxed{}
%otherwise use boxed{}

\DeclareMathSymbol{\widehatsym}{\mathord}{largesymbols}{"62}
\newcommand\lowerwidehatsym{%
  \text{\smash{\raisebox{-1.3ex}{%
    $\widehatsym$}}}}
\newcommand\fixwidehat[1]{%
  \mathchoice
    {\accentset{\displaystyle\lowerwidehatsym}{#1}}
    {\accentset{\textstyle\lowerwidehatsym}{#1}}
    {\accentset{\scriptstyle\lowerwidehatsym}{#1}}
    {\accentset{\scriptscriptstyle\lowerwidehatsym}{#1}}
  }


\newcommand{\cupdot}{\mathbin{\dot{\cup}}}
\newcommand{\bigcupdot}{\mathop{\dot{\bigcup}}}

\usepackage{graphicx}

\usepackage[toc,page]{appendix}

% text on arrow for xRightarrow
\makeatletter
%\newcommand{\xRightarrow}[2][]{\ext@arrow 0359\Rightarrowfill@{#1}{#2}}
\makeatother

% Arbitrary long arrow
\newcommand{\Rarrow}[1]{%
\parbox{#1}{\tikz{\draw[->](0,0)--(#1,0);}}
}

\newcommand{\LRarrow}[1]{%
\parbox{#1}{\tikz{\draw[<->](0,0)--(#1,0);}}
}


\makeatletter
\providecommand*{\rmodels}{%
  \mathrel{%
    \mathpalette\@rmodels\models
  }%
}
\newcommand*{\@rmodels}[2]{%
  \reflectbox{$\m@th#1#2$}%
}
\makeatother

% Roman numerals
\makeatletter
\newcommand*{\rom}[1]{\expandafter\@slowromancap\romannumeral #1@}
\makeatother
% \\def \\b\([a-zA-Z]\) {\\boldsymbol{[a-zA-z]}}
% \\DeclareMathOperator{\\b\1}{\\textbf{\1}}

\DeclareMathOperator*{\argmin}{arg\,min}
\DeclareMathOperator*{\argmax}{arg\,max}

\DeclareMathOperator{\bone}{\textbf{1}}
\DeclareMathOperator{\bx}{\textbf{x}}
\DeclareMathOperator{\bz}{\textbf{z}}
\DeclareMathOperator{\bff}{\textbf{f}}
\DeclareMathOperator{\ba}{\textbf{a}}
\DeclareMathOperator{\bk}{\textbf{k}}
\DeclareMathOperator{\bs}{\textbf{s}}
\DeclareMathOperator{\bh}{\textbf{h}}
\DeclareMathOperator{\bc}{\textbf{c}}
\DeclareMathOperator{\br}{\textbf{r}}
\DeclareMathOperator{\bi}{\textbf{i}}
\DeclareMathOperator{\bj}{\textbf{j}}
\DeclareMathOperator{\bn}{\textbf{n}}
\DeclareMathOperator{\be}{\textbf{e}}
\DeclareMathOperator{\bo}{\textbf{o}}
\DeclareMathOperator{\bU}{\textbf{U}}
\DeclareMathOperator{\bL}{\textbf{L}}
\DeclareMathOperator{\bV}{\textbf{V}}
\def \bzero {\mathbf{0}}
\def \bbone {\mathbb{1}}
\def \btwo {\mathbf{2}}
\DeclareMathOperator{\bv}{\textbf{v}}
\DeclareMathOperator{\bp}{\textbf{p}}
\DeclareMathOperator{\bI}{\textbf{I}}
\def \dbI {\dot{\bI}}
\DeclareMathOperator{\bM}{\textbf{M}}
\DeclareMathOperator{\bN}{\textbf{N}}
\DeclareMathOperator{\bK}{\textbf{K}}
\DeclareMathOperator{\bt}{\textbf{t}}
\DeclareMathOperator{\bb}{\textbf{b}}
\DeclareMathOperator{\bA}{\textbf{A}}
\DeclareMathOperator{\bX}{\textbf{X}}
\DeclareMathOperator{\bu}{\textbf{u}}
\DeclareMathOperator{\bS}{\textbf{S}}
\DeclareMathOperator{\bZ}{\textbf{Z}}
\DeclareMathOperator{\bJ}{\textbf{J}}
\DeclareMathOperator{\by}{\textbf{y}}
\DeclareMathOperator{\bw}{\textbf{w}}
\DeclareMathOperator{\bT}{\textbf{T}}
\DeclareMathOperator{\bF}{\textbf{F}}
\DeclareMathOperator{\bmm}{\textbf{m}}
\DeclareMathOperator{\bW}{\textbf{W}}
\DeclareMathOperator{\bR}{\textbf{R}}
\DeclareMathOperator{\bC}{\textbf{C}}
\DeclareMathOperator{\bD}{\textbf{D}}
\DeclareMathOperator{\bE}{\textbf{E}}
\DeclareMathOperator{\bQ}{\textbf{Q}}
\DeclareMathOperator{\bP}{\textbf{P}}
\DeclareMathOperator{\bY}{\textbf{Y}}
\DeclareMathOperator{\bH}{\textbf{H}}
\DeclareMathOperator{\bB}{\textbf{B}}
\DeclareMathOperator{\bG}{\textbf{G}}
\def \blambda {\symbf{\lambda}}
\def \boldeta {\symbf{\eta}}
\def \balpha {\symbf{\alpha}}
\def \btau {\symbf{\tau}}
\def \bbeta {\symbf{\beta}}
\def \bgamma {\symbf{\gamma}}
\def \bxi {\symbf{\xi}}
\def \bLambda {\symbf{\Lambda}}
\def \bGamma {\symbf{\Gamma}}

\newcommand{\bto}{{\boldsymbol{\to}}}
\newcommand{\Ra}{\Rightarrow}
\newcommand{\xrsa}[1]{\overset{#1}{\rightsquigarrow}}
\newcommand{\xlsa}[1]{\overset{#1}{\leftsquigarrow}}
\newcommand\und[1]{\underline{#1}}
\newcommand\ove[1]{\overline{#1}}
%\def \concat {\verb|^|}
\def \bPhi {\mbfPhi}
\def \btheta {\mbftheta}
\def \bTheta {\mbfTheta}
\def \bmu {\mbfmu}
\def \bphi {\mbfphi}
\def \bSigma {\mbfSigma}
\def \la {\langle}
\def \ra {\rangle}

\def \caln {\mathcal{N}}
\def \dissum {\displaystyle\Sigma}
\def \dispro {\displaystyle\prod}

\def \caret {\verb!^!}

\def \A {\mathbb{A}}
\def \B {\mathbb{B}}
\def \C {\mathbb{C}}
\def \D {\mathbb{D}}
\def \E {\mathbb{E}}
\def \F {\mathbb{F}}
\def \G {\mathbb{G}}
\def \H {\mathbb{H}}
\def \I {\mathbb{I}}
\def \J {\mathbb{J}}
\def \K {\mathbb{K}}
\def \L {\mathbb{L}}
\def \M {\mathbb{M}}
\def \N {\mathbb{N}}
\def \O {\mathbb{O}}
\def \P {\mathbb{P}}
\def \Q {\mathbb{Q}}
\def \R {\mathbb{R}}
\def \S {\mathbb{S}}
\def \T {\mathbb{T}}
\def \U {\mathbb{U}}
\def \V {\mathbb{V}}
\def \W {\mathbb{W}}
\def \X {\mathbb{X}}
\def \Y {\mathbb{Y}}
\def \Z {\mathbb{Z}}

\def \cala {\mathcal{A}}
\def \cale {\mathcal{E}}
\def \calb {\mathcal{B}}
\def \calq {\mathcal{Q}}
\def \calp {\mathcal{P}}
\def \cals {\mathcal{S}}
\def \calx {\mathcal{X}}
\def \caly {\mathcal{Y}}
\def \calg {\mathcal{G}}
\def \cald {\mathcal{D}}
\def \caln {\mathcal{N}}
\def \calr {\mathcal{R}}
\def \calt {\mathcal{T}}
\def \calm {\mathcal{M}}
\def \calw {\mathcal{W}}
\def \calc {\mathcal{C}}
\def \calv {\mathcal{V}}
\def \calf {\mathcal{F}}
\def \calk {\mathcal{K}}
\def \call {\mathcal{L}}
\def \calu {\mathcal{U}}
\def \calo {\mathcal{O}}
\def \calh {\mathcal{H}}
\def \cali {\mathcal{I}}
\def \calj {\mathcal{J}}

\def \bcup {\bigcup}

% set theory

\def \zfcc {\textbf{ZFC}^-}
\def \BGC {\textbf{BGC}}
\def \BG {\textbf{BG}}
\def \ac  {\textbf{AC}}
\def \gl  {\textbf{L }}
\def \gll {\textbf{L}}
\newcommand{\zfm}{$\textbf{ZF}^-$}

\def \ZFm {\text{ZF}^-}
\def \ZFCm {\text{ZFC}^-}
\DeclareMathOperator{\WF}{WF}
\DeclareMathOperator{\On}{On}
\def \on {\textbf{On }}
\def \cm {\textbf{M }}
\def \cn {\textbf{N }}
\def \cv {\textbf{V }}
\def \zc {\textbf{ZC }}
\def \zcm {\textbf{ZC}}
\def \zff {\textbf{ZF}}
\def \wfm {\textbf{WF}}
\def \onm {\textbf{On}}
\def \cmm {\textbf{M}}
\def \cnm {\textbf{N}}
\def \cvm {\textbf{V}}

\renewcommand{\restriction}{\mathord{\upharpoonright}}
%% another restriction
\newcommand\restr[2]{{% we make the whole thing an ordinary symbol
  \left.\kern-\nulldelimiterspace % automatically resize the bar with \right
  #1 % the function
  \vphantom{\big|} % pretend it's a little taller at normal size
  \right|_{#2} % this is the delimiter
  }}

\def \pred {\text{pred}}

\def \rank {\text{rank}}
\def \Con {\text{Con}}
\def \deff {\text{Def}}


\def \uin {\underline{\in}}
\def \oin {\overline{\in}}
\def \uR {\underline{R}}
\def \oR {\overline{R}}
\def \uP {\underline{P}}
\def \oP {\overline{P}}

\def \dsum {\displaystyle\sum}

\def \Ra {\Rightarrow}

\def \e {\enspace}

\def \sgn {\operatorname{sgn}}
\def \gen {\operatorname{gen}}
\def \Hom {\operatorname{Hom}}
\def \hom {\operatorname{hom}}
\def \Sub {\operatorname{Sub}}

\def \supp {\operatorname{supp}}

\def \epiarrow {\twoheadarrow}
\def \monoarrow {\rightarrowtail}
\def \rrarrow {\rightrightarrows}

% \def \minus {\text{-}}
% \newcommand{\minus}{\scalebox{0.75}[1.0]{$-$}}
% \DeclareUnicodeCharacter{002D}{\minus}


\def \tril {\triangleleft}

\def \ISigma {\text{I}\Sigma}
\def \IDelta {\text{I}\Delta}
\def \IPi {\text{I}\Pi}
\def \ACF {\textsf{ACF}}
\def \pCF {\textit{p}\text{CF}}
\def \ACVF {\textsf{ACVF}}
\def \HLR {\textsf{HLR}}
\def \OAG {\textsf{OAG}}
\def \RCF {\textsf{RCF}}
\DeclareMathOperator{\GL}{GL}
\DeclareMathOperator{\PGL}{PGL}
\DeclareMathOperator{\SL}{SL}
\DeclareMathOperator{\Inv}{Inv}
\DeclareMathOperator{\res}{res}
\DeclareMathOperator{\Sym}{Sym}
%\DeclareMathOperator{\char}{char}
\def \equal {=}

\def \degree {\text{degree}}
\def \app {\text{App}}
\def \FV {\text{FV}}
\def \conv {\text{conv}}
\def \cont {\text{cont}}
\DeclareMathOperator{\cl}{\text{cl}}
\DeclareMathOperator{\trcl}{\text{trcl}}
\DeclareMathOperator{\sg}{sg}
\DeclareMathOperator{\trdeg}{trdeg}
\def \Ord {\text{Ord}}

\DeclareMathOperator{\cf}{cf}
\DeclareMathOperator{\zfc}{ZFC}

%\DeclareMathOperator{\Th}{Th}
%\def \th {\text{Th}}
% \newcommand{\th}{\text{Th}}
\DeclareMathOperator{\type}{type}
\DeclareMathOperator{\zf}{\textbf{ZF}}
\def \fa {\mathfrak{a}}
\def \fb {\mathfrak{b}}
\def \fc {\mathfrak{c}}
\def \fd {\mathfrak{d}}
\def \fe {\mathfrak{e}}
\def \ff {\mathfrak{f}}
\def \fg {\mathfrak{g}}
\def \fh {\mathfrak{h}}
%\def \fi {\mathfrak{i}}
\def \fj {\mathfrak{j}}
\def \fk {\mathfrak{k}}
\def \fl {\mathfrak{l}}
\def \fm {\mathfrak{m}}
\def \fn {\mathfrak{n}}
\def \fo {\mathfrak{o}}
\def \fp {\mathfrak{p}}
\def \fq {\mathfrak{q}}
\def \fr {\mathfrak{r}}
\def \fs {\mathfrak{s}}
\def \ft {\mathfrak{t}}
\def \fu {\mathfrak{u}}
\def \fv {\mathfrak{v}}
\def \fw {\mathfrak{w}}
\def \fx {\mathfrak{x}}
\def \fy {\mathfrak{y}}
\def \fz {\mathfrak{z}}
\def \fA {\mathfrak{A}}
\def \fB {\mathfrak{B}}
\def \fC {\mathfrak{C}}
\def \fD {\mathfrak{D}}
\def \fE {\mathfrak{E}}
\def \fF {\mathfrak{F}}
\def \fG {\mathfrak{G}}
\def \fH {\mathfrak{H}}
\def \fI {\mathfrak{I}}
\def \fJ {\mathfrak{J}}
\def \fK {\mathfrak{K}}
\def \fL {\mathfrak{L}}
\def \fM {\mathfrak{M}}
\def \fN {\mathfrak{N}}
\def \fO {\mathfrak{O}}
\def \fP {\mathfrak{P}}
\def \fQ {\mathfrak{Q}}
\def \fR {\mathfrak{R}}
\def \fS {\mathfrak{S}}
\def \fT {\mathfrak{T}}
\def \fU {\mathfrak{U}}
\def \fV {\mathfrak{V}}
\def \fW {\mathfrak{W}}
\def \fX {\mathfrak{X}}
\def \fY {\mathfrak{Y}}
\def \fZ {\mathfrak{Z}}

\def \sfA {\textsf{A}}
\def \sfB {\textsf{B}}
\def \sfC {\textsf{C}}
\def \sfD {\textsf{D}}
\def \sfE {\textsf{E}}
\def \sfF {\textsf{F}}
\def \sfG {\textsf{G}}
\def \sfH {\textsf{H}}
\def \sfI {\textsf{I}}
\def \sfJ {\textsf{J}}
\def \sfK {\textsf{K}}
\def \sfL {\textsf{L}}
\def \sfM {\textsf{M}}
\def \sfN {\textsf{N}}
\def \sfO {\textsf{O}}
\def \sfP {\textsf{P}}
\def \sfQ {\textsf{Q}}
\def \sfR {\textsf{R}}
\def \sfS {\textsf{S}}
\def \sfT {\textsf{T}}
\def \sfU {\textsf{U}}
\def \sfV {\textsf{V}}
\def \sfW {\textsf{W}}
\def \sfX {\textsf{X}}
\def \sfY {\textsf{Y}}
\def \sfZ {\textsf{Z}}
\def \sfa {\textsf{a}}
\def \sfb {\textsf{b}}
\def \sfc {\textsf{c}}
\def \sfd {\textsf{d}}
\def \sfe {\textsf{e}}
\def \sff {\textsf{f}}
\def \sfg {\textsf{g}}
\def \sfh {\textsf{h}}
\def \sfi {\textsf{i}}
\def \sfj {\textsf{j}}
\def \sfk {\textsf{k}}
\def \sfl {\textsf{l}}
\def \sfm {\textsf{m}}
\def \sfn {\textsf{n}}
\def \sfo {\textsf{o}}
\def \sfp {\textsf{p}}
\def \sfq {\textsf{q}}
\def \sfr {\textsf{r}}
\def \sfs {\textsf{s}}
\def \sft {\textsf{t}}
\def \sfu {\textsf{u}}
\def \sfv {\textsf{v}}
\def \sfw {\textsf{w}}
\def \sfx {\textsf{x}}
\def \sfy {\textsf{y}}
\def \sfz {\textsf{z}}

\def \ttA {\texttt{A}}
\def \ttB {\texttt{B}}
\def \ttC {\texttt{C}}
\def \ttD {\texttt{D}}
\def \ttE {\texttt{E}}
\def \ttF {\texttt{F}}
\def \ttG {\texttt{G}}
\def \ttH {\texttt{H}}
\def \ttI {\texttt{I}}
\def \ttJ {\texttt{J}}
\def \ttK {\texttt{K}}
\def \ttL {\texttt{L}}
\def \ttM {\texttt{M}}
\def \ttN {\texttt{N}}
\def \ttO {\texttt{O}}
\def \ttP {\texttt{P}}
\def \ttQ {\texttt{Q}}
\def \ttR {\texttt{R}}
\def \ttS {\texttt{S}}
\def \ttT {\texttt{T}}
\def \ttU {\texttt{U}}
\def \ttV {\texttt{V}}
\def \ttW {\texttt{W}}
\def \ttX {\texttt{X}}
\def \ttY {\texttt{Y}}
\def \ttZ {\texttt{Z}}
\def \tta {\texttt{a}}
\def \ttb {\texttt{b}}
\def \ttc {\texttt{c}}
\def \ttd {\texttt{d}}
\def \tte {\texttt{e}}
\def \ttf {\texttt{f}}
\def \ttg {\texttt{g}}
\def \tth {\texttt{h}}
\def \tti {\texttt{i}}
\def \ttj {\texttt{j}}
\def \ttk {\texttt{k}}
\def \ttl {\texttt{l}}
\def \ttm {\texttt{m}}
\def \ttn {\texttt{n}}
\def \tto {\texttt{o}}
\def \ttp {\texttt{p}}
\def \ttq {\texttt{q}}
\def \ttr {\texttt{r}}
\def \tts {\texttt{s}}
\def \ttt {\texttt{t}}
\def \ttu {\texttt{u}}
\def \ttv {\texttt{v}}
\def \ttw {\texttt{w}}
\def \ttx {\texttt{x}}
\def \tty {\texttt{y}}
\def \ttz {\texttt{z}}

\def \bara {\bbar{a}}
\def \barb {\bbar{b}}
\def \barc {\bbar{c}}
\def \bard {\bbar{d}}
\def \bare {\bbar{e}}
\def \barf {\bbar{f}}
\def \barg {\bbar{g}}
\def \barh {\bbar{h}}
\def \bari {\bbar{i}}
\def \barj {\bbar{j}}
\def \bark {\bbar{k}}
\def \barl {\bbar{l}}
\def \barm {\bbar{m}}
\def \barn {\bbar{n}}
\def \baro {\bbar{o}}
\def \barp {\bbar{p}}
\def \barq {\bbar{q}}
\def \barr {\bbar{r}}
\def \bars {\bbar{s}}
\def \bart {\bbar{t}}
\def \baru {\bbar{u}}
\def \barv {\bbar{v}}
\def \barw {\bbar{w}}
\def \barx {\bbar{x}}
\def \bary {\bbar{y}}
\def \barz {\bbar{z}}
\def \barA {\bbar{A}}
\def \barB {\bbar{B}}
\def \barC {\bbar{C}}
\def \barD {\bbar{D}}
\def \barE {\bbar{E}}
\def \barF {\bbar{F}}
\def \barG {\bbar{G}}
\def \barH {\bbar{H}}
\def \barI {\bbar{I}}
\def \barJ {\bbar{J}}
\def \barK {\bbar{K}}
\def \barL {\bbar{L}}
\def \barM {\bbar{M}}
\def \barN {\bbar{N}}
\def \barO {\bbar{O}}
\def \barP {\bbar{P}}
\def \barQ {\bbar{Q}}
\def \barR {\bbar{R}}
\def \barS {\bbar{S}}
\def \barT {\bbar{T}}
\def \barU {\bbar{U}}
\def \barVV {\bbar{V}}
\def \barW {\bbar{W}}
\def \barX {\bbar{X}}
\def \barY {\bbar{Y}}
\def \barZ {\bbar{Z}}

\def \baralpha {\bbar{\alpha}}
\def \bartau {\bbar{\tau}}
\def \barsigma {\bbar{\sigma}}
\def \barzeta {\bbar{\zeta}}

\def \hata {\hat{a}}
\def \hatb {\hat{b}}
\def \hatc {\hat{c}}
\def \hatd {\hat{d}}
\def \hate {\hat{e}}
\def \hatf {\hat{f}}
\def \hatg {\hat{g}}
\def \hath {\hat{h}}
\def \hati {\hat{i}}
\def \hatj {\hat{j}}
\def \hatk {\hat{k}}
\def \hatl {\hat{l}}
\def \hatm {\hat{m}}
\def \hatn {\hat{n}}
\def \hato {\hat{o}}
\def \hatp {\hat{p}}
\def \hatq {\hat{q}}
\def \hatr {\hat{r}}
\def \hats {\hat{s}}
\def \hatt {\hat{t}}
\def \hatu {\hat{u}}
\def \hatv {\hat{v}}
\def \hatw {\hat{w}}
\def \hatx {\hat{x}}
\def \haty {\hat{y}}
\def \hatz {\hat{z}}
\def \hatA {\hat{A}}
\def \hatB {\hat{B}}
\def \hatC {\hat{C}}
\def \hatD {\hat{D}}
\def \hatE {\hat{E}}
\def \hatF {\hat{F}}
\def \hatG {\hat{G}}
\def \hatH {\hat{H}}
\def \hatI {\hat{I}}
\def \hatJ {\hat{J}}
\def \hatK {\hat{K}}
\def \hatL {\hat{L}}
\def \hatM {\hat{M}}
\def \hatN {\hat{N}}
\def \hatO {\hat{O}}
\def \hatP {\hat{P}}
\def \hatQ {\hat{Q}}
\def \hatR {\hat{R}}
\def \hatS {\hat{S}}
\def \hatT {\hat{T}}
\def \hatU {\hat{U}}
\def \hatVV {\hat{V}}
\def \hatW {\hat{W}}
\def \hatX {\hat{X}}
\def \hatY {\hat{Y}}
\def \hatZ {\hat{Z}}

\def \hatphi {\hat{\phi}}

\def \barfM {\bbar{\fM}}
\def \barfN {\bbar{\fN}}

\def \tila {\tilde{a}}
\def \tilb {\tilde{b}}
\def \tilc {\tilde{c}}
\def \tild {\tilde{d}}
\def \tile {\tilde{e}}
\def \tilf {\tilde{f}}
\def \tilg {\tilde{g}}
\def \tilh {\tilde{h}}
\def \tili {\tilde{i}}
\def \tilj {\tilde{j}}
\def \tilk {\tilde{k}}
\def \till {\tilde{l}}
\def \tilm {\tilde{m}}
\def \tiln {\tilde{n}}
\def \tilo {\tilde{o}}
\def \tilp {\tilde{p}}
\def \tilq {\tilde{q}}
\def \tilr {\tilde{r}}
\def \tils {\tilde{s}}
\def \tilt {\tilde{t}}
\def \tilu {\tilde{u}}
\def \tilv {\tilde{v}}
\def \tilw {\tilde{w}}
\def \tilx {\tilde{x}}
\def \tily {\tilde{y}}
\def \tilz {\tilde{z}}
\def \tilA {\tilde{A}}
\def \tilB {\tilde{B}}
\def \tilC {\tilde{C}}
\def \tilD {\tilde{D}}
\def \tilE {\tilde{E}}
\def \tilF {\tilde{F}}
\def \tilG {\tilde{G}}
\def \tilH {\tilde{H}}
\def \tilI {\tilde{I}}
\def \tilJ {\tilde{J}}
\def \tilK {\tilde{K}}
\def \tilL {\tilde{L}}
\def \tilM {\tilde{M}}
\def \tilN {\tilde{N}}
\def \tilO {\tilde{O}}
\def \tilP {\tilde{P}}
\def \tilQ {\tilde{Q}}
\def \tilR {\tilde{R}}
\def \tilS {\tilde{S}}
\def \tilT {\tilde{T}}
\def \tilU {\tilde{U}}
\def \tilVV {\tilde{V}}
\def \tilW {\tilde{W}}
\def \tilX {\tilde{X}}
\def \tilY {\tilde{Y}}
\def \tilZ {\tilde{Z}}

\def \tilalpha {\tilde{\alpha}}
\def \tilPhi {\tilde{\Phi}}

\def \barnu {\bar{\nu}}
\def \barrho {\bar{\rho}}
%\DeclareMathOperator{\ker}{ker}
\DeclareMathOperator{\im}{im}

\DeclareMathOperator{\Inn}{Inn}
\DeclareMathOperator{\rel}{rel}
\def \dote {\stackrel{\cdot}=}
%\DeclareMathOperator{\AC}{\textbf{AC}}
\DeclareMathOperator{\cod}{cod}
\DeclareMathOperator{\dom}{dom}
\DeclareMathOperator{\card}{card}
\DeclareMathOperator{\ran}{ran}
\DeclareMathOperator{\textd}{d}
\DeclareMathOperator{\td}{d}
\DeclareMathOperator{\id}{id}
\DeclareMathOperator{\LT}{LT}
\DeclareMathOperator{\Mat}{Mat}
\DeclareMathOperator{\Eq}{Eq}
\DeclareMathOperator{\irr}{irr}
\DeclareMathOperator{\Fr}{Fr}
\DeclareMathOperator{\Gal}{Gal}
\DeclareMathOperator{\lcm}{lcm}
\DeclareMathOperator{\alg}{\text{alg}}
\DeclareMathOperator{\Th}{Th}
%\DeclareMathOperator{\deg}{deg}


% \varprod
\DeclareSymbolFont{largesymbolsA}{U}{txexa}{m}{n}
\DeclareMathSymbol{\varprod}{\mathop}{largesymbolsA}{16}
% \DeclareMathSymbol{\tonm}{\boldsymbol{\to}\textbf{Nm}}
\def \tonm {\bto\textbf{Nm}}
\def \tohm {\bto\textbf{Hm}}

% Category theory
\DeclareMathOperator{\ob}{ob}
\DeclareMathOperator{\Ab}{\textbf{Ab}}
\DeclareMathOperator{\Alg}{\textbf{Alg}}
\DeclareMathOperator{\Rng}{\textbf{Rng}}
\DeclareMathOperator{\Sets}{\textbf{Sets}}
\DeclareMathOperator{\Set}{\textbf{Set}}
\DeclareMathOperator{\Grp}{\textbf{Grp}}
\DeclareMathOperator{\Met}{\textbf{Met}}
\DeclareMathOperator{\BA}{\textbf{BA}}
\DeclareMathOperator{\Mon}{\textbf{Mon}}
\DeclareMathOperator{\Top}{\textbf{Top}}
\DeclareMathOperator{\hTop}{\textbf{hTop}}
\DeclareMathOperator{\HTop}{\textbf{HTop}}
\DeclareMathOperator{\Aut}{\text{Aut}}
\DeclareMathOperator{\RMod}{R-\textbf{Mod}}
\DeclareMathOperator{\RAlg}{R-\textbf{Alg}}
\DeclareMathOperator{\LF}{LF}
\DeclareMathOperator{\op}{op}
\DeclareMathOperator{\Rings}{\textbf{Rings}}
\DeclareMathOperator{\Ring}{\textbf{Ring}}
\DeclareMathOperator{\Groups}{\textbf{Groups}}
\DeclareMathOperator{\Group}{\textbf{Group}}
\DeclareMathOperator{\ev}{ev}
% Algebraic Topology
\DeclareMathOperator{\obj}{obj}
\DeclareMathOperator{\Spec}{Spec}
\DeclareMathOperator{\spec}{spec}
% Model theory
\DeclareMathOperator*{\ind}{\raise0.2ex\hbox{\ooalign{\hidewidth$\vert$\hidewidth\cr\raise-0.9ex\hbox{$\smile$}}}}
\def\nind{\cancel{\ind}}
\DeclareMathOperator{\acl}{acl}
\DeclareMathOperator{\tspan}{span}
\DeclareMathOperator{\acleq}{acl^{\eq}}
\DeclareMathOperator{\Av}{Av}
\DeclareMathOperator{\ded}{ded}
\DeclareMathOperator{\EM}{EM}
\DeclareMathOperator{\dcl}{dcl}
\DeclareMathOperator{\Ext}{Ext}
\DeclareMathOperator{\eq}{eq}
\DeclareMathOperator{\ER}{ER}
\DeclareMathOperator{\tp}{tp}
\DeclareMathOperator{\stp}{stp}
\DeclareMathOperator{\qftp}{qftp}
\DeclareMathOperator{\Diag}{Diag}
\DeclareMathOperator{\MD}{MD}
\DeclareMathOperator{\MR}{MR}
\DeclareMathOperator{\RM}{RM}
\DeclareMathOperator{\el}{el}
\DeclareMathOperator{\depth}{depth}
\DeclareMathOperator{\ZFC}{ZFC}
\DeclareMathOperator{\GCH}{GCH}
\DeclareMathOperator{\Inf}{Inf}
\DeclareMathOperator{\Pow}{Pow}
\DeclareMathOperator{\ZF}{ZF}
\DeclareMathOperator{\CH}{CH}
\def \FO {\text{FO}}
\DeclareMathOperator{\fin}{fin}
\DeclareMathOperator{\qr}{qr}
\DeclareMathOperator{\Mod}{Mod}
\DeclareMathOperator{\Def}{Def}
\DeclareMathOperator{\TC}{TC}
\DeclareMathOperator{\KH}{KH}
\DeclareMathOperator{\Part}{Part}
\DeclareMathOperator{\Infset}{\textsf{Infset}}
\DeclareMathOperator{\DLO}{\textsf{DLO}}
\DeclareMathOperator{\PA}{\textsf{PA}}
\DeclareMathOperator{\DAG}{\textsf{DAG}}
\DeclareMathOperator{\ODAG}{\textsf{ODAG}}
\DeclareMathOperator{\sfMod}{\textsf{Mod}}
\DeclareMathOperator{\AbG}{\textsf{AbG}}
\DeclareMathOperator{\sfACF}{\textsf{ACF}}
\DeclareMathOperator{\DCF}{\textsf{DCF}}
% Computability Theorem
\DeclareMathOperator{\Tot}{Tot}
\DeclareMathOperator{\graph}{graph}
\DeclareMathOperator{\Fin}{Fin}
\DeclareMathOperator{\Cof}{Cof}
\DeclareMathOperator{\lh}{lh}
% Commutative Algebra
\DeclareMathOperator{\ord}{ord}
\DeclareMathOperator{\Idem}{Idem}
\DeclareMathOperator{\zdiv}{z.div}
\DeclareMathOperator{\Frac}{Frac}
\DeclareMathOperator{\rad}{rad}
\DeclareMathOperator{\nil}{nil}
\DeclareMathOperator{\Ann}{Ann}
\DeclareMathOperator{\End}{End}
\DeclareMathOperator{\coim}{coim}
\DeclareMathOperator{\coker}{coker}
\DeclareMathOperator{\Bil}{Bil}
\DeclareMathOperator{\Tril}{Tril}
\DeclareMathOperator{\tchar}{char}
\DeclareMathOperator{\tbd}{bd}

% Topology
\DeclareMathOperator{\diam}{diam}
\newcommand{\interior}[1]{%
  {\kern0pt#1}^{\mathrm{o}}%
}

\DeclareMathOperator*{\bigdoublewedge}{\bigwedge\mkern-15mu\bigwedge}
\DeclareMathOperator*{\bigdoublevee}{\bigvee\mkern-15mu\bigvee}

% \makeatletter
% \newcommand{\vect}[1]{%
%   \vbox{\m@th \ialign {##\crcr
%   \vectfill\crcr\noalign{\kern-\p@ \nointerlineskip}
%   $\hfil\displaystyle{#1}\hfil$\crcr}}}
% \def\vectfill{%
%   $\m@th\smash-\mkern-7mu%
%   \cleaders\hbox{$\mkern-2mu\smash-\mkern-2mu$}\hfill
%   \mkern-7mu\raisebox{-3.81pt}[\p@][\p@]{$\mathord\mathchar"017E$}$}

% \newcommand{\amsvect}{%
%   \mathpalette {\overarrow@\vectfill@}}
% \def\vectfill@{\arrowfill@\relbar\relbar{\raisebox{-3.81pt}[\p@][\p@]{$\mathord\mathchar"017E$}}}

% \newcommand{\amsvectb}{%
% \newcommand{\vect}{%
%   \mathpalette {\overarrow@\vectfillb@}}
% \newcommand{\vecbar}{%
%   \scalebox{0.8}{$\relbar$}}
% \def\vectfillb@{\arrowfill@\vecbar\vecbar{\raisebox{-4.35pt}[\p@][\p@]{$\mathord\mathchar"017E$}}}
% \makeatother
% \bigtimes

\DeclareFontFamily{U}{mathx}{\hyphenchar\font45}
\DeclareFontShape{U}{mathx}{m}{n}{
      <5> <6> <7> <8> <9> <10>
      <10.95> <12> <14.4> <17.28> <20.74> <24.88>
      mathx10
      }{}
\DeclareSymbolFont{mathx}{U}{mathx}{m}{n}
\DeclareMathSymbol{\bigtimes}{1}{mathx}{"91}
% \odiv
\DeclareFontFamily{U}{matha}{\hyphenchar\font45}
\DeclareFontShape{U}{matha}{m}{n}{
      <5> <6> <7> <8> <9> <10> gen * matha
      <10.95> matha10 <12> <14.4> <17.28> <20.74> <24.88> matha12
      }{}
\DeclareSymbolFont{matha}{U}{matha}{m}{n}
\DeclareMathSymbol{\odiv}         {2}{matha}{"63}


\newcommand\subsetsim{\mathrel{%
  \ooalign{\raise0.2ex\hbox{\scalebox{0.9}{$\subset$}}\cr\hidewidth\raise-0.85ex\hbox{\scalebox{0.9}{$\sim$}}\hidewidth\cr}}}
\newcommand\simsubset{\mathrel{%
  \ooalign{\raise-0.2ex\hbox{\scalebox{0.9}{$\subset$}}\cr\hidewidth\raise0.75ex\hbox{\scalebox{0.9}{$\sim$}}\hidewidth\cr}}}

\newcommand\simsubsetsim{\mathrel{%
  \ooalign{\raise0ex\hbox{\scalebox{0.8}{$\subset$}}\cr\hidewidth\raise1ex\hbox{\scalebox{0.75}{$\sim$}}\hidewidth\cr\raise-0.95ex\hbox{\scalebox{0.8}{$\sim$}}\cr\hidewidth}}}
\newcommand{\stcomp}[1]{{#1}^{\mathsf{c}}}

\setlength{\baselineskip}{0.5in}

\stackMath
\newcommand\yrightarrow[2][]{\mathrel{%
  \setbox2=\hbox{\stackon{\scriptstyle#1}{\scriptstyle#2}}%
  \stackunder[0pt]{%
    \xrightarrow{\makebox[\dimexpr\wd2\relax]{$\scriptstyle#2$}}%
  }{%
   \scriptstyle#1\,%
  }%
}}
\newcommand\yleftarrow[2][]{\mathrel{%
  \setbox2=\hbox{\stackon{\scriptstyle#1}{\scriptstyle#2}}%
  \stackunder[0pt]{%
    \xleftarrow{\makebox[\dimexpr\wd2\relax]{$\scriptstyle#2$}}%
  }{%
   \scriptstyle#1\,%
  }%
}}
\newcommand\yRightarrow[2][]{\mathrel{%
  \setbox2=\hbox{\stackon{\scriptstyle#1}{\scriptstyle#2}}%
  \stackunder[0pt]{%
    \xRightarrow{\makebox[\dimexpr\wd2\relax]{$\scriptstyle#2$}}%
  }{%
   \scriptstyle#1\,%
  }%
}}
\newcommand\yLeftarrow[2][]{\mathrel{%
  \setbox2=\hbox{\stackon{\scriptstyle#1}{\scriptstyle#2}}%
  \stackunder[0pt]{%
    \xLeftarrow{\makebox[\dimexpr\wd2\relax]{$\scriptstyle#2$}}%
  }{%
   \scriptstyle#1\,%
  }%
}}

\newcommand\altxrightarrow[2][0pt]{\mathrel{\ensurestackMath{\stackengine%
  {\dimexpr#1-7.5pt}{\xrightarrow{\phantom{#2}}}{\scriptstyle\!#2\,}%
  {O}{c}{F}{F}{S}}}}
\newcommand\altxleftarrow[2][0pt]{\mathrel{\ensurestackMath{\stackengine%
  {\dimexpr#1-7.5pt}{\xleftarrow{\phantom{#2}}}{\scriptstyle\!#2\,}%
  {O}{c}{F}{F}{S}}}}

\newenvironment{bsm}{% % short for 'bracketed small matrix'
  \left[ \begin{smallmatrix} }{%
  \end{smallmatrix} \right]}

\newenvironment{psm}{% % short for ' small matrix'
  \left( \begin{smallmatrix} }{%
  \end{smallmatrix} \right)}

\newcommand{\bbar}[1]{\mkern 1.5mu\overline{\mkern-1.5mu#1\mkern-1.5mu}\mkern 1.5mu}

\newcommand{\bigzero}{\mbox{\normalfont\Large\bfseries 0}}
\newcommand{\rvline}{\hspace*{-\arraycolsep}\vline\hspace*{-\arraycolsep}}

\font\zallman=Zallman at 40pt
\font\elzevier=Elzevier at 40pt

\newcommand\isoto{\stackrel{\textstyle\sim}{\smash{\longrightarrow}\rule{0pt}{0.4ex}}}
\newcommand\embto{\stackrel{\textstyle\prec}{\smash{\longrightarrow}\rule{0pt}{0.4ex}}}

% from http://www.actual.world/resources/tex/doc/TikZ.pdf

\tikzset{
modal/.style={>=stealth’,shorten >=1pt,shorten <=1pt,auto,node distance=1.5cm,
semithick},
world/.style={circle,draw,minimum size=0.5cm,fill=gray!15},
point/.style={circle,draw,inner sep=0.5mm,fill=black},
reflexive above/.style={->,loop,looseness=7,in=120,out=60},
reflexive below/.style={->,loop,looseness=7,in=240,out=300},
reflexive left/.style={->,loop,looseness=7,in=150,out=210},
reflexive right/.style={->,loop,looseness=7,in=30,out=330}
}


\makeatletter
\newcommand*{\doublerightarrow}[2]{\mathrel{
  \settowidth{\@tempdima}{$\scriptstyle#1$}
  \settowidth{\@tempdimb}{$\scriptstyle#2$}
  \ifdim\@tempdimb>\@tempdima \@tempdima=\@tempdimb\fi
  \mathop{\vcenter{
    \offinterlineskip\ialign{\hbox to\dimexpr\@tempdima+1em{##}\cr
    \rightarrowfill\cr\noalign{\kern.5ex}
    \rightarrowfill\cr}}}\limits^{\!#1}_{\!#2}}}
\newcommand*{\triplerightarrow}[1]{\mathrel{
  \settowidth{\@tempdima}{$\scriptstyle#1$}
  \mathop{\vcenter{
    \offinterlineskip\ialign{\hbox to\dimexpr\@tempdima+1em{##}\cr
    \rightarrowfill\cr\noalign{\kern.5ex}
    \rightarrowfill\cr\noalign{\kern.5ex}
    \rightarrowfill\cr}}}\limits^{\!#1}}}
\makeatother

% $A\doublerightarrow{a}{bcdefgh}B$

% $A\triplerightarrow{d_0,d_1,d_2}B$

\def \uhr {\upharpoonright}
\def \rhu {\rightharpoonup}
\def \uhl {\upharpoonleft}


\newcommand{\floor}[1]{\lfloor #1 \rfloor}
\newcommand{\ceil}[1]{\lceil #1 \rceil}
\newcommand{\lcorner}[1]{\llcorner #1 \lrcorner}
\newcommand{\llb}[1]{\llbracket #1 \rrbracket}
\newcommand{\ucorner}[1]{\ulcorner #1 \urcorner}
\newcommand{\emoji}[1]{{\DejaSans #1}}
\newcommand{\vprec}{\rotatebox[origin=c]{-90}{$\prec$}}

\newcommand{\nat}[6][large]{%
  \begin{tikzcd}[ampersand replacement = \&, column sep=#1]
    #2\ar[bend left=40,""{name=U}]{r}{#4}\ar[bend right=40,',""{name=D}]{r}{#5}\& #3
          \ar[shorten <=10pt,shorten >=10pt,Rightarrow,from=U,to=D]{d}{~#6}
    \end{tikzcd}
}


\providecommand\rightarrowRHD{\relbar\joinrel\mathrel\RHD}
\providecommand\rightarrowrhd{\relbar\joinrel\mathrel\rhd}
\providecommand\longrightarrowRHD{\relbar\joinrel\relbar\joinrel\mathrel\RHD}
\providecommand\longrightarrowrhd{\relbar\joinrel\relbar\joinrel\mathrel\rhd}
\def \lrarhd {\longrightarrowrhd}


\makeatletter
\providecommand*\xrightarrowRHD[2][]{\ext@arrow 0055{\arrowfill@\relbar\relbar\longrightarrowRHD}{#1}{#2}}
\providecommand*\xrightarrowrhd[2][]{\ext@arrow 0055{\arrowfill@\relbar\relbar\longrightarrowrhd}{#1}{#2}}
\makeatother

\newcommand{\metalambda}{%
  \mathop{%
    \rlap{$\lambda$}%
    \mkern3mu
    \raisebox{0ex}{$\lambda$}%
  }%
}

%% https://tex.stackexchange.com/questions/15119/draw-horizontal-line-left-and-right-of-some-text-a-single-line
\newcommand*\ruleline[1]{\par\noindent\raisebox{.8ex}{\makebox[\linewidth]{\hrulefill\hspace{1ex}\raisebox{-.8ex}{#1}\hspace{1ex}\hrulefill}}}

% https://www.dickimaw-books.com/latex/novices/html/newenv.html
\newenvironment{Block}[1]% environment name
{% begin code
  % https://tex.stackexchange.com/questions/19579/horizontal-line-spanning-the-entire-document-in-latex
  \noindent\textcolor[RGB]{128,128,128}{\rule{\linewidth}{1pt}}
  \par\noindent
  {\Large\textbf{#1}}%
  \bigskip\par\noindent\ignorespaces
}%
{% end code
  \par\noindent
  \textcolor[RGB]{128,128,128}{\rule{\linewidth}{1pt}}
  \ignorespacesafterend
}

\mathchardef\mhyphen="2D % Define a "math hyphen"

\def \QQ {\quad}
\def \QW {​\quad}

\makeindex
\usepackage[UTF8]{ctex}
\def \FORM {\text{FORM}}
\def \Rep {\text{Rep}}
\def \ZC {\text{ZC}}
\def \PROOF {\text{PROOF}}
\author{Yao}
\date{\today}
\title{Set Theory2}
\hypersetup{
 pdfauthor={Yao},
 pdftitle={Set Theory2},
 pdfkeywords={},
 pdfsubject={},
 pdfcreator={Emacs 28.0.92 (Org mode 9.6)}, 
 pdflang={English}}
\begin{document}

\maketitle
\tableofcontents

\section{集合的宇宙}
\label{sec:orgc4f98f3}
\subsection{数理逻辑}
\label{sec:org4b77846}
在\(\ZFC\)下证明\(\ZFC\vdash\CH\),希望将``\(\ZFC\vdash\CH\)''表述为一阶句子

一般而言,给定一个\(\call\)-理论\(T\)和一个\(\call\)-句子\(\delta\),``\(T\vdash\sigma\)''不能用一个\(\call\)-句子表示,只能
用元语言表述

我们需要在\(\ZFC\)中编码“元语言”

在\(\ZFC\)中可以定义\(\caln=(\N,+,\times,0,1)\)

即存在集合论语言\(\call=\{\in\}\)中的 \textbf{公式} ,在\(\ZFC\)的任意模型中可以定义 \(\N,+,\times,0,1\),以上公式与模
型无关

用\(\ucorner{0}\),\(\ucorner{1}\),\(\ucorner{2}\)\ldots{} 表示\(\ZFC\)中的“自然数”,以区别元语言中
的自然数

\begin{theorem}[]
如果\(R\subseteq\N^n\)是一个递归关系。\(T\subseteq\Th(\caln)\)是包含数论的适当丰富的理论,则存在公式\(\varphi(x_1,\dots,x_n)\)使
得对任意自然数\(m_1,\dots,m_n\)有
\begin{align*}
&\text{如果}(m_1,\dots,m_n)\in R\text{则}T\vdash\varphi(\ucorner{m_1},\dots,\ucorner{m_n})\\
&\text{如果}(m_1,\dots,m_n)\notin R\text{则}T\vdash\neg\varphi(\ucorner{m_1},\dots,\ucorner{m_n})
\end{align*}
\end{theorem}

\begin{remark}
\begin{enumerate}
\item \(T\subseteq\Th(\caln)\subseteq\ZFC\)
\item \(\varphi\)是语言\(\{+,\times,0,1\}\)上的公式
\item \(\varphi\)可以还原为一个\(\{\in\}\)上的公式
\item \(\varphi(\ucorner{m_1},\dots,\ucorner{m_n})\)是一个闭语句
\end{enumerate}
\end{remark}

\textbf{编码}

编码函数\(f:X\to\N\)

存在解码函数\(g,h\),对\(a=a_0,\dots,a_n\in X\), \(h(f(a))=n+1\), \(g(f(a),k)=a_k\) (分量)

性质:以上三种函数\(f,g,h\)均是递归函数\(\Rightarrow\)都是可表示的

性质:“公式集”的编码集是递归的

性质:如果\(T\subseteq\ZFC\)是可公理化的,则\(T\)的证明集的编码集是递归的

\begin{corollary}[]
存在一个公式 \(\psi\) 和\(\theta\)使得
\begin{align*}
\ZFC\vdash\psi(n)&\Leftrightarrow n\text{ is a formula}\\
\ZFC\vdash\neg\psi(n)&\Leftrightarrow n\text{ is not a formula}\\
\ZFC\vdash\theta(n)&\Leftrightarrow n\text{ is a proof in }\ZFC\\
\ZFC\vdash\neg\theta(n)&\Leftrightarrow n\text{ is not a proof in }\ZFC\\
\end{align*}
称\(\psi\)定义了公式集,\(\theta\)定义了证明集
\end{corollary}

\(\FORM=\{\ucorner{\varphi}\mid\varphi\text{ formula}\}\subseteq\N\)

    如果\(T\subseteq\ZFC\)是可公理化的,则“\(T\)是一致的”是一个一阶表述式
    “不存在一个有穷的证明序列\(D=(\varphi_1,\dots,\varphi_n)\)使得\(\varphi_n\)形如\(\varphi\wedge\neg\varphi\)
,记作\(\Con(T)\)

\begin{theorem}[第二不完全]
如果\(T\)是包含\(\ZFC\)的一个递归公理集,且\(T\)一致,则
\begin{equation*}
T\not\vdash\Con(T)
\end{equation*}
特别地,\(\ZFC\not\vdash\Con(\ZFC)\)
\end{theorem}

\begin{theorem}[]
对任意可公理化的理论\(T\),\(\ZFC\vdash\Con(T)\)当且仅当存在\(M\vDash T\)
\end{theorem}

即不能在\(\ZFC\)里证明\(\ZFC\)有一个模型

需要可公理化来写出\(\Con(T)\),因此因为\(\ZFC\not\vDash\Con(T)\),我们只能假设这么一个模型

集合论的模型跟集合论没什么关系,就是一个集合带一个二元关系,是关于集合论语言的结构

\begin{definition}[]
设\((M,E)\)是集合论模型
\begin{enumerate}
\item 对任意公式\(\varphi(\barx,y)\),定义\(M^n\)上的函数
\begin{equation*}
h_\varphi:M^n\to M
\end{equation*}
满足条件
\begin{equation*}
M\vDash\exists y\varphi(\bara,y)\Rightarrow M\vDash\varphi(\bara,h_\varphi(\bara))
\end{equation*}
称\(h_\varphi\)为\(\varphi\)的Skolem函数(依赖于选择公理,不同的变量选择有不同的函数)
\item 令\(\calh=\{h_\varphi\mid\varphi\text{ formula}\}\)为Skolem函数集合,设\(S\)是\(M\)的任意子集,则\(\calh(S)\)表示包
含\(S\)且对\(\calh\)封闭的最小集合,称之为\(S\)的Skolem壳
\end{enumerate}
\end{definition}

\begin{lemma}[]
令\(N\)是集合论模型,\(S\subseteq N\),如果\(M=\calh(S)\),则\(M\prec N\)
\end{lemma}

\begin{proof}
Induction

对任意\(\bara\in M^n\),有\(M\vDash\varphi(\bara)\Leftrightarrow N\vDash\varphi(\bara)\)
\begin{enumerate}
\item 不含量词,显然成立
\item \(\varphi\)形如\(\exists y\psi(\barx,y)\), \(N\vDash\exists y\psi(\bara,y)\Rightarrow N\vDash\psi(\bara,h_\psi(\bara))\),by
IH, \(M\vDash\psi(\bara,h_\psi(\bara))\Rightarrow M\vDash\exists y\psi(\bara,y)\)
\end{enumerate}
\end{proof}

\begin{theorem}[Löwenheim–Skolem Theorem]

\end{theorem}
\subsection{层垒的谱系}
\label{sec:orge8ecdf8}
工作于\(\ZF^-\):\(\ZF-\)基础公理

\(\alpha\mapsto V_\alpha\)是\(\On\)到\(\WF\)的1-1映射,而\(\On\)是真类

\begin{lemma}[]
For any ordinal \(\alpha\)
\begin{enumerate}
\item \(V_\alpha\) is transitive
\item \(\xi\le\alpha\Rightarrow V_\xi\subseteq V_\alpha\)
\item if \(\kappa\) is inaccessible, then \(\abs{V_\kappa}=\kappa\)
\end{enumerate}
\end{lemma}

\begin{definition}[]
For any \(x\in\WF\), \textbf{rank} of \(x\) is
\begin{equation*}
\rank(x)=\min\{\beta\mid x\in V_{\beta+1}\}
\end{equation*}
\end{definition}

\(\rank(x)=\alpha\Rightarrow x\in V_{\alpha+1}\wedge x\notin V_\alpha\)

\begin{itemize}
\item \(x\in V_{\alpha+1}\Leftrightarrow\rank(x)\le\alpha\)
\item \(x\subseteq V_\alpha\Leftrightarrow\rank(x)\le\alpha\)
\end{itemize}

\begin{lemma}[]
\begin{enumerate}
\item \(V_\alpha=\{x\in\WF\mid\rank(x)<\alpha\}\)
\item \(\WF\) is transitive
\item \(\forall x,y\in\WF\), \(x\in y\Rightarrow\rank(x)<\rank(y)\)
\item \(\forall y\in\WF\), \(\rank(y)=\sup\{\rank(x)+1\mid x\in y\}\)
\end{enumerate}
\end{lemma}

\begin{proof}
\begin{enumerate}
\item by definition, \(x\in V_{\rank(x)+1}\setminus V_{\rank(x)}\), \(\rank(x)<\alpha\Rightarrow x\in V_{\rank(x)+1}\subseteq V_\alpha\)

\(\rank(x)\ge\alpha\Rightarrow x\notin V_\alpha\)

\item \(\WF\) is the ``union'' of transitive sets

\item \(y\in V_{\rank(y)+1}\setminus V_{\rank(y)}\), \(y\subseteq V_{\rank(y)}\), \(x\in y\Rightarrow x\in V_{\rank(y)}\Rightarrow\rank(x)<\rank(y)\)

\item by 3, \(\sup\{\rank(x)+1\mid x\in y\}\le\rank(y)\).

induction on \(\rank(y)\le\sup\{\rank(x)+1\mid x\in y\}\)
\begin{itemize}
\item \(\rank(y)=0\)
\item \(\rank(y)=\beta+1\), \(y\in V_{\beta+2}\setminus V_{\beta+1}\)

\(y\in V_{\beta+2}\Rightarrow y\subseteq V_{\beta+1}\). \(y\notin V_{\beta+1}\Rightarrow y\not\subseteq V_{\beta}\Rightarrow y\setminus V_\beta\) nonempty.
Let \(x\in y\setminus V_\beta\), \(\rank(x)\ge\beta\), \(\sup\{\rank(x)+1\mid x\in y\}\ge\beta+1=\rank(y)\)
\item \(\rank(y)=\gamma\) for some limit, then \(y\subseteq V_\gamma\) and for any \(\xi<\gamma\), \(y\not\subseteq V_\xi\),
let \(X_\xi\in y\setminus V_\xi\), then \(\rank(X_\xi)\ge\xi\), \(\sup\{\rank(x)+1\mid x\in y\}\ge\sup\{\xi+1\mid\xi<\rank(y)\}\ge\rank(y)\)
\end{itemize}
\end{enumerate}
\end{proof}

\begin{itemize}
\item \(\WF\)中的集合按照秩分层
\item 在\(\WF\)中基础公理是成立的:\(\forall y(y\neq\emptyset\to\exists x\in y(x\cap y=\emptyset))\),因为任何序数集都有最小元,挑一个有最
小rank的就好了
\item \(\WF\)类的构造没有用到选择公理
\item \(\On\subseteq\WF\)
\end{itemize}


\begin{lemma}[]
for any ordinal \(\alpha\)
\begin{enumerate}
\item \(\alpha\in\WF\) and \(\rank(\alpha)=\alpha\)
\item \(V_\alpha\cap\On=\alpha\)
\end{enumerate}
\end{lemma}

\begin{proof}
\begin{enumerate}
\item \(0\in V_1\setminus V_0\subset\WF\), \(\rank(0)=0\)

If \(\alpha\in\WF\)
and
\(\rank(\alpha)=\alpha\).
\(\alpha\in V_{\alpha+1}\setminus V_\alpha\), \(\alpha\subseteq V_{\alpha+1}\). \(\alpha+1=\alpha\cup\{\alpha\}\subseteq V_{\alpha+1}\), \(\alpha+1\in V_{\alpha+2}\subset\WF\).
If \(\alpha+1\in V_{\alpha+1}\), then \(\rank(\alpha+1)\le\alpha\), but \(\alpha\in\alpha+1\Rightarrow\rank(\alpha)=\alpha<\rank(\alpha+1)\). A
contradiction

suppose \(\gamma\) is a limit ordinal and for any \(\alpha<\gamma\), \(\alpha\in V_{\alpha+1}\setminus V_\alpha\).
\(\gamma=\bigcup_{\alpha<\gamma}\alpha\subseteq\bigcup_{\alpha<\gamma}V_\alpha=V_\gamma\). Thus \(\gamma\in V_{\gamma+1}\), \(\rank(\gamma)\le\gamma\) and \(\rank(\gamma)\not<\gamma\).
\item suppose \(\beta\in V_\alpha\cap\On\), then \(\beta=\rank(\beta)<\alpha\). If \(\beta\in\alpha\) and \(\rank(\beta)<\alpha\), \(\beta\in V_\alpha\cap\On\)
\end{enumerate}
\end{proof}

\begin{lemma}[]
\begin{enumerate}
\item If \(x\in\WF\), then \(\bigcup x,\calp(x),\{x\}\in\WF\), and their rank \(<\rank(x)+\omega\)
\item If \(x,y\in\WF\), then \(x\times y,x\cup y,x\cap y,\{x,y\},(x,y),x^y\in\WF\), and their
rank \(<\rank(x)+\rank(y)+\omega\)
\item \(\Z,\Q,\R\in V_{\omega+\omega}\)
\item for any set \(x\), \(x\in\WF\Leftrightarrow x\subset\WF\)
\end{enumerate}
\end{lemma}

\begin{proof}
\begin{enumerate}
\item suppose \(\rank(x)=\alpha\). \(x\in V_{\alpha+1}\setminus V_\alpha\) and \(x\subseteq V_\alpha\).

by transitivity, \(\bigcup x\subseteq V_\alpha\Rightarrow \bigcup x\in V_{\alpha+1}\subset\WF\). \(\rank(\bigcup x)\le\alpha\)

suppose
\(y\in\calp(x)\),
\(y\subseteq x\Rightarrow y\subseteq V_\alpha\Rightarrow y\in V_{\alpha+1}\). \(\calp(x)\subseteq V_{\alpha+1}\), \(\calp(x)\in V_{\alpha+2}\), \(\rank(\calp(x))\le\alpha+1\).

\(\{x\}\in\calp(x)\in V_{\alpha+2}\).

\item Suppose \(\rank(x)=\alpha,\rank(y)=\beta\), \(x\subset V_\alpha\), \(y\subset V_\beta\)

\(x\cup y\subset V_\alpha\cup V_\beta=V_{\max(\alpha,\beta)}\), \(\rank(x\cup y)\le\max(\alpha,\beta)\)

\(x\cap y\subset V_{\min(\alpha,\beta)}\)

\(\{x,y\}\subseteq V_{\alpha+1}\cup V_{\beta+1}=V_{\max(\alpha,\beta)+1}\), \(\rank(\{x,y\})=\max(\alpha,\beta)+1\)

\((x,y)=\{\{x\},\{x,y\}\}\subset V_{\max(\alpha,\beta)+2}\). \(\rank((x,y))=\max(\alpha,\beta)+2\)

\(x\times y=\{(a,b)\mid a\in x,b\in y\}\).
\(a\in x\Rightarrow\rank(a)<\alpha\), \(b\in y\Rightarrow\rank(b)<\beta\), \(\rank(a,b)<\max(\alpha,\beta)+2\),
\((a,b)\in V_{\max(\alpha,\beta)+2}\). \(x\times y\subseteq V_{\max(\alpha,\beta)+2}\), \(\rank(x\times y)\le\max(\alpha,\beta)+2\).

\(x^y\subseteq\calp(x\times y)\subseteq V_{\max(\alpha,\beta)+3}\).

\item \(\N=\omega\in V_{\omega+1}\)

\(\Z\): let \(\sim\) be an equivalence relation on \(\omega\times\omega\), \((a,b)\sim(c,d)\Leftrightarrow a+d=b+c\),
then \(\Z=(\omega\times\omega)/\sim\). Hence \(\Z\) is a partition of \(\omega\times\omega\) and
hence \(\Z\subseteq\calp(\omega\times\omega)\). \(\Z\in V_{\omega+3}\)

\(\Q\): let \(\sim\) be an equivalence
on \(\Z\times\Z^+\), \((a,b)\sim(c,d)\Leftrightarrow ad=bc\). \(\Q\subseteq\calp(\Z\times\Z^+)\), \(\Q\in V_{\omega+6}\)

\(\R\): set of dedekind cut on \(\Q\), \(\R\subset\calp(\Q)\), \(\R\in V_{\omega+8}\)

\item \(\Rightarrow\): \(\WF\) is transitive

\(\Leftarrow\): \(x\) is a set and \(x\subset\bigcup_{\alpha\in\On}V_\alpha\).

\textbf{Claim}: there is an ordinal \(\alpha\) s.t. \(x\subset V_\alpha\)

Otherwise, let \(f:\On\to\calp(x)\) s.t. \(f(\alpha)=x\setminus V_\alpha\). Then for any \(y\in\calp(x)\), \(f^{-1}(y)\) is
a set. \(\On=\bigcup_{y\in x}f^{-1}(y)\) and is thus a set, a contradiction
\end{enumerate}
\end{proof}

AC => Any set has cardinality
\begin{lemma}[]
Assume AC (\(V\vDash\ZFC\))
\begin{enumerate}
\item for any group \(G\), there is a group \(G'\) in \(\WF\) s.t. \(G\cong G'\)
\item for any topological space \(T\), there is a topological space \(T'\) in \(\WF\)
s.t. \(T\cong T'\) (homeomorphic)
\end{enumerate}
\end{lemma}

\begin{proof}
\begin{enumerate}
\item suppose \((G,*_G)\) is a group, \(G,*_G\in V\). By AC, there is a cardinal \(\alpha\)
s.t. \(\abs{G}=\alpha\), that is, there is a bijection \(f:\alpha\to G\). Define \(*\): for
any \(x,y,z\in\alpha\), \(x*y=z\Leftrightarrow f(x)*_Gf(y)=f(z)\). Then \((\alpha,​*)\cong(G,​*_G)\), \(​*\subseteq\alpha\times\alpha\)
\end{enumerate}
\end{proof}

\(V\)中的任何结构都可以在\(\WF\)中找到同构象(同构是在\(V\)里看到的)

\begin{definition}[]
任意集合\(A\)上的二元关系<是 \textbf{良基} 的,当且仅当对\(A\)的任意非空子集\(X\),\(X\)有<下的极小元
\end{definition}

\begin{theorem}[]
If \(A\in\WF\), then \(\in\) is a well-founded relation on \(A\)
\end{theorem}

\begin{proof}
suppose \(X\subseteq A\), \(X\neq\emptyset\), \(X\subseteq\WF\), then elements of \(X\) has ranks
and \(x\in y\Rightarrow\rank(x)<\rank(y)\). Let \(x\) having least rank in \(X\), then \(x\) is
the \(\in\)-minimal element in \(X\)
\end{proof}

\begin{lemma}[]
If \(A\) is a transitive set and \(\in\) is a well-founded relation on \(A\), then \(A\in\WF\)
\end{lemma}

\begin{proof}
Just need to prove \(A\subset\WF\). If \(A\not\subset\WF\), \(X=A\setminus\WF\neq\emptyset\). Then \(X\) has a \(\in\)-minimal
element \(x\). Then \(x\neq\emptyset\in\WF\). For any \(y\in x\), \(y\in A\). By the minimality
of \(x\), \(y\in\WF\). Then \(x\subset\WF\), \(x\in\WF\), a contradiction
\end{proof}

\begin{lemma}[]
For any set \(x\), there is a minimal transitive set \(\trcl(x)\) s.t. \(x\subseteq\trcl(x)\)
\end{lemma}

\begin{proof}
For any \(n\in\omega\) define \(x_n\)
\begin{align*}
x_0&=x\\
x_{n+1}&=\bigcup x_n
\end{align*}
let \(\trcl(x)=\bigcup_{n\in\omega}x_n\).
\begin{enumerate}
\item \(\trcl(x)\)is transitive

\(a\in\trcl(x)\Rightarrow a\in x_n\Rightarrow a\subseteq x_{n+1}\subseteq\trcl(x)\)
\item \(\trcl(x)\) is minimal

If \(y\supseteq x\) is transitive, recursively prove for any \(n<\omega\), \(x_n\subseteq y\).
\end{enumerate}
\end{proof}

\(\trcl(x)\) is the \textbf{transitive closure} of \(x\).

\begin{lemma}[]
We can prove the following without axiom of power set
\begin{enumerate}
\item if \(x\) is transitive, \(\trcl(x)=x\)
\item \(y\in x\Rightarrow\trcl(y)\subseteq\trcl(x)\)
\item \(\trcl(x)=x\cup\bigcup\{\trcl(y)\mid y\in x\}\)
\end{enumerate}
\end{lemma}

\begin{proof}
\begin{enumerate}
\setcounter{enumi}{1}
\item \(y\in x\subset\trcl(x)\). \(y\in\trcl(x)\). \(\trcl(y)\subseteq\trcl(x)\).
\item \(x\cup\bigcup\{\trcl(y)\mid y\in x\}\subseteq\trcl(x)\) by (2)

\(\bigcup\{\trcl(y)\mid y\in x\}\) is transitive. For \(y\in x\), \(y\subseteq\trcl(y)\). Thus rhs is transitive
\end{enumerate}
\end{proof}


\begin{theorem}[In \(\ZFm\)]
For any set \(X\), TFAE
\begin{enumerate}
\item \(X\in\WF\)
\item \(\trcl(X)\in\WF\)
\item \(\in\) is a well-founded relation on \(\trcl(X)\)
\end{enumerate}
\end{theorem}

\begin{proof}
\(1\to 2\): \(\WF\) is closed under union
\end{proof}

\begin{theorem}[]
If \(V\vDash\ZFm\), TFAE
\begin{enumerate}
\item axiom of foundation (\(V\vDash\)) axiom of foundation
\item for any set \(X\), \(\in\) is a well-founded relation on \(X\)
\item \(V=\WF\)
\end{enumerate}
\end{theorem}

\(V\vDash\ZF\Rightarrow V=\WF(\WF\vDash\ZF)\)

Goal:  \(V\vDash\ZFm\Rightarrow\WF\vDash\ZFm\)
但是\(\WF\)是一个类,我们并没有定义

我们可以用相对化编码\(\WF\vDash\ZFm\)
\subsection{相对化 relativization}
\label{sec:org310258b}
工作在\(\ZFm\)
\begin{definition}[]
\(M\) class, \(\varphi\) formula, \(\varphi\)对\(M\)的 \textbf{相对化} \(\varphi^{M}\)
\begin{enumerate}
\item \((x=y)^{M}:=x=y\)
\item \((x\in y)^{M}:=x\in y\)
\item \((\varphi\to\psi)^{M}:=\varphi^{M}\to\psi^{M}\)
\item \((\neg\varphi)^{M}:=\neg\varphi^{M}\)
\item \((\forall x\varphi)^{M}:=(\forall x\in M)\varphi^{M}\)
\end{enumerate}
\end{definition}
\(\varphi^{M}\)读作“\(\varphi\)在\(M\)中为真”,表示为\((M,\in)\subseteq(V,\in)\)有\(M\vDash\varphi\),即如果\(V\vDash\varphi^{M}\),那
么\(M\vDash\varphi\),而\(V\)知道得更多一点

重新定义了满足

若\(M\)被公式\(M(u)\)定义,\((\forall x\varphi)^{M}\)是公式\(\forall x(M(x)\to\varphi^{M}(x))\)

\begin{examplle}[]
\(M=\On\),\(\On\vDash\forall x\forall y(x\in y\vee y\in x\vee x= y)\)
\end{examplle}

``\(M\vDash\varphi\)''可以形式化为\(V\vDash\varphi^M\),而\(M\)对应于\(M(u)\),即等价于\(T\vdash\varphi^M\),如果我们工作在某个\(T\)
上

若函数\(f\)被公式\(\varphi(\barx,y)\)定义,则\(V\vDash\forall\barx\exists!y\varphi(\barx,y)\),但相对化后不一定对,因此
\(f^M=\{(\barx,y)\in M:\varphi^M(\barx,y)\}\)不一定是\(M\)上的函数

\begin{definition}[]
for any theory \(T\), any class \(M\), \(M\) is a \textbf{model} of \(T\), \(M\vDash T\), iff for any axiom
\(\varphi\) of \(T\), \(\varphi^M\) holds, i.e., \(V\vDash\varphi^M\)
\end{definition}

\(V\)中定义出语义

\begin{theorem}[]
\(V\vDash\ZFm\Rightarrow\WF\vDash\ZF-\Inf\),

\(V\vDash(\ZF-\Inf)^{\WF}\)
等价的
\(\ZFm\vdash(\ZF-\Inf)^{\WF}\)
\end{theorem}

\begin{itemize}
\item \textbf{存在公理} :\(\exists x\in M(x=x)\)
\item \textbf{外延公理} :\(\Ext^M\)
\begin{equation*}
\forall x\in M\forall y\in M\forall u\in M((u\in X\leftrightarrow u\in Y)\to X=Y)
\end{equation*}
\begin{lemma}[]
If \(M\) is transitive, then \(\Ext^M\) holds
\end{lemma}

\begin{proof}
suppose \(X,Y\in M\), if \(X\neq Y\), then there is \(u\in X\triangle Y\) (by \(\Ext\) in \(V\)), by
transitivity, \(u\in M\)
\end{proof}
\item 分离公理模式:for any \(M\), any formula \(\varphi\), \(S(\varphi)^M\)
\begin{equation*}
\forall x\in M\exists Y\in M\forall u\in M(u\in Y\leftrightarrow u\in X\wedge\varphi^M(u))
\end{equation*}
Therefore, if for any \(X\in M\), \(\{u\in X\mid\varphi^M(x)\}\in M\), then 分离公理模式在\(M\)中为真
\begin{lemma}[]
If \(M\) satisfies \(x\in M\Leftrightarrow x\subset M\), then \(S(\varphi)^M\) holds for any \(M\)
\end{lemma}

\begin{proof}
Suppose \(X\in M\), suffices to find corresponding \(Y\in M\) s.t. \(\forall u\in M(u\in Y\leftrightarrow u\in X\wedge\varphi^M(u))\)

根据\(V\)中的分离公理, \(Y=\{x\in X\mid\varphi^M(u)\}\in V\) and \(Y\subseteq X\subset M\), thus \(Y\in M\) and
\(\forall u(u\in Y\leftrightarrow u\in X\wedge\varphi^M(u))\). But \(x\in Y\Rightarrow x\in M\), thus this is equivalent to
\(\forall u\in M(u\in Y\leftrightarrow u\in X\wedge\varphi^M(u))\)
\end{proof}
\item \textbf{axiom of pairing} Pair
\begin{equation*}
\forall x\in M\forall y\in M\exists z\in M\forall u\in M(u\in z\leftrightarrow u= x\vee u=y)
\end{equation*}
只要\(M\)对对集函数\(x,y\mapsto\{x,y\}\)封闭,则\(Pair^M\)成立
\item 幂集公理 \(Pow\)
\begin{equation*}
\forall X\in M\exists Y\in M\forall u\in M(u\in Y\leftrightarrow\forall a\in M(a\in u\to a\in X))
\end{equation*}
\begin{lemma}[]
If \(M\) satisfies \(x\in M\Leftrightarrow x\subset M\), then \(Pow^M\) holds
\end{lemma}

\begin{proof}
for any \(X\in M\), \(\calp(X)\in M\). and we prove \(\calp(X)\) is the \(Y\), for any \(u\in M\)
\end{proof}
\item \textbf{axiom of infinity} \(Inf\)
\begin{equation*}
\exists X\in M(\emptyset\in X\wedge\forall y\in M(y\in X\to y^+\in X))
\end{equation*}
\(\emptyset:\psi(x):=\forall u(u\in x\to u\neq u)\), \(x=\emptyset\Leftrightarrow\psi(x)\)

\(y^+:\varphi(y,z):\forall u\in z(u=y\vee u\in y)\wedge y\subseteq z\wedge y\in z\)
函数相对化后不一定是函数,所以放到下一节
\item \textbf{axiom of foundation} Fod
\begin{equation*}
\forall x\in M(\exists u\in M(u\in x)\to\exists y\in M(y\in x\wedge\neg\exists u\in M(u\in x\wedge u\in y)))
\end{equation*}
\begin{lemma}[]
If \(M\) is transitive and elements of \(M\) is well-founded under \(\in\), then \(Fod^M\) holds
\end{lemma}

\begin{proof}
suppose \(x_0\in M\) and there is

\(\psi:=\exists u(u\in x)\) and \(\varphi\) is the latter part

\(\psi^M(x_0)\leftrightarrow\exists u(u\in x_0)\) since \(M\) is transitive, \(\varphi^M(x_0)\leftrightarrow\exists y(y\in x_0\wedge\neg\exists u\in M(u\in x\wedge u\in y))\)

在\(V\)中,\(x_0\neq\emptyset\),由条件可知\((x_0,\in)\)是良基的,于是\(\varphi\)在\(V\)中对,那么当然在\(M\)中对
\end{proof}
\item \textbf{替换公理模式} \(Rep(\varphi)\)
\begin{equation*}
\forall A\in M\forall x\in A\cap M\exists !y\in M\varphi^M(x,y)\to\exists B\in M\forall x\in A\exists y\in B\varphi^M(x,y)
\end{equation*}
\(\exists!y\theta(y):\exists y(\theta(y)\wedge\forall y'(\theta(y')\to y=y'))\)
\begin{lemma}[]
if \(M\) satisfied \(x\in M\Leftrightarrow x\subset M\), then \(Rep(\varphi)^M\) holds for any \(\varphi\)
\end{lemma}

\begin{proof}
for any \(A_0\in M\), then \(A_0\cap M=A_0\), thus we have
\(\forall x\in A_0\exists!y(\varphi^M(x,y)\wedge M(y))\).

by \(Rep(\varphi^M(x,y)\wedge M(y))\),
\(\exists B'\forall x\in A_0\exists y\in B'\varphi^M(x,y)\wedge M(y)\)

Let \(B=B'\cap M\), which is what we want
\end{proof}

Thus in \(\ZFm\), we can prove \(\WF\vDash\ZF-\inf\)
\end{itemize}
\subsection{绝对性}
\label{sec:orgdb48752}
\((V,\in)\supseteq(M,\in)\)

对于哪些\(\varphi\),有\(V\vDash\varphi\Leftrightarrow M\vDash\varphi\)

\begin{definition}[]
公式\(\psi(\barx)\),对任意类\(M\subseteq N\),如果
\begin{equation*}
\forall\barx\in M(\psi^M(\barx)\leftrightarrow\psi^N(\barx))
\end{equation*}
就称\(\psi(\barx)\)对于\(M,N\)是 \textbf{绝对的} ,如果\(N=V\),则称\(\psi(\barx)\)对于\(M\)是 \textbf{绝对的}
\end{definition}

\(\bara\in M\), \((M,\in)\vDash\psi(\bara)\Leftrightarrow V\vDash\psi^M(\bara)\)

\(\psi\)相对于\(M,N\)绝对:\(\forall\bara\in M\),有\(M\vDash\psi(\bara)\Leftrightarrow N\vDash\psi(\bara)\)

if \(\forall\varphi(\barx)\in L\),均有\(\varphi\)相对于\(M,N\)绝对的,则\(M\preceq N\)

\begin{lemma}[]
suppose \(M\subseteq N\), \(\varphi,\psi\) formula
\begin{enumerate}
\item 如果\(\varphi,\psi\)相对于\(M,N\)绝对, so are \(\neg\varphi,\varphi\to\psi\)
\item if \(\varphi\) is q.f., then \(\varphi\)对任意\(M\)绝对
\item if \(M\) are transitive, and \(\varphi\)相对于它们绝对, so is \(\forall x\in y\varphi\)
\end{enumerate}
\end{lemma}

\begin{proof}
\begin{enumerate}
\setcounter{enumi}{2}
\item \(\forall x\in y\varphi:=\forall x(x\in y\to\varphi(x,y,\barz))\),故\((\forall x\in y\varphi)^M=\forall x\in M(x\in y\to\varphi^M(x,y,\barz))\),任取
\(y_0,\barz_0\in M\),则由\(M\)的传递性,都有\(x\in y_0\Rightarrow x\in M\)

目标:\(\forall x\in M(x\in y_0\to\varphi^M(x,y_0,\barz_0))\)当且仅当\(\forall x\in N(x\in y_0\to\varphi^N(x,y_0,\barz_0))\)

由\(\varphi\)的绝对性,对每个\(x_0\in M\),有
\begin{equation*}
\varphi^M(x_0,y_0,\barz_0)\leftrightarrow\varphi^N(x_0,y_0,\barz_0)
\end{equation*}
故\(V\vDash\forall x\in M(x\in y_0\to\varphi^M(x,y_0,\barz_0))\),当且仅当
\(V\vDash\forall x(x\in y_0\to\varphi^M(x,y_0,\barz_0))\)当且仅当
\(V\vDash\forall x\in N(x\in y_0\to\varphi^M(x,y_0,\barz_0))\)
\end{enumerate}
\end{proof}

\begin{lemma}[]
令\(M\subseteq N\)且\(M\)传递,\(\psi(\barx)\)是一个公式,则
\begin{enumerate}
\item 如果\(\psi\)是\(\Delta_0\)公式,则他它对\(M,N\)是绝对的
\item 如果\(\psi\)是\(\Sigma_1\)公式,则
\begin{equation*}
\forall\barx\in M(\psi^M(\barx)\to\psi^N(\barx))
\end{equation*}
\item 如果\(\psi\)是\(\Pi_1\)公式,则
\begin{equation*}
\forall\barx\in M^n(\psi^N(\barx)\to\psi^M(\barx))
\end{equation*}
\end{enumerate}
\end{lemma}

\begin{proof}
\begin{enumerate}
\item 对公式的长度进行归纳证明
\item 例子:令\(M=\On\),\(N=\WF\),令\(\psi(y):=\forall x\in y\forall u,v\in x(u\in v\vee v\in u\vee u=v)\),则\(\psi\)是\(\Delta_0\)的,
则
\begin{align*}
&\psi^M(y)=\forall x\in M(x\in y\to\forall u,v\in M(u,v\in x\to(u\in v\vee v\in u\vee u=v)))\\
&\psi^N(y)=\forall x\in N(x\in y\to\forall u,v\in N(u,v\in x\to(u\in v\vee v\in u\vee u=v)))
\end{align*}
任取\(x_0\in\WF\setminus\On\)使得\((x_0,\in)\)不是线序,令\(y_0=\{x_0\}\),则\(\psi^M(y_0)\)的前件假,
\(\psi^M(y_0)\)是真的,\(\psi^N(y_0)\)为假,因此
\begin{equation*}
\forall\barx(\psi^M(\barx)\to\psi^N(\barx))
\end{equation*}
错误

令\(x=\barx,y=\bary\)

设\(\psi:=\exists\varphi(x,y)\),\(\varphi(x,y)\in\Delta_0\), \(\psi^M:=\exists y\in M(\varphi^M(x,y))\), \(\psi^N:=\exists y\in N(\varphi^N(x,y))\),任取
\(a\in M^m\),目标\(\psi^M(a)\to\psi^N(a)\)

若\(\psi^M(a)\)成立,则有\(b\in M^n\)使得\(\psi^M(a,b)\),由\(\Delta_0\)的绝对性,\(\psi^N(a,b)\),因此\(\exists y\psi^N(a,y)\)
\item 设\(\psi:=\forall y\varphi(x,y)\)其中\(\varphi\in\Delta_0\),则\(\psi^M:=\forall y\in M\varphi^M(x,y)\),\(\psi^N:=\forall y\in N\varphi^N(x,y)\),设\(a\in M\)使
得\(\psi^N(a)\)成立,目标\(\psi^M(a)\)成立。

\(\psi^N(a)\Rightarrow\)对所有的\(b\in N\)均有\(\varphi^N(a,b)\)成立,故对一切\(b\in M\)有\(\varphi^N(a,b)\)成立,由\(\varphi\)的绝
对性,\(\forall y\in M\varphi^M(a,y)\)
\end{enumerate}
\end{proof}

\begin{lemma}[]
设\(M\subseteq N\),均是句子集\(\Sigma\)的模型,而\(\Sigma\vdash\forall\barx(\varphi(\barx)\leftrightarrow\psi(\barx))\),则\(\varphi\)对\(M\)与\(N\)绝对
当且仅当\(\psi\)也是
\end{lemma}

\begin{proof}
\(M,N\vDash\forall\barx(\varphi(\barx)\leftrightarrow\psi(\barx))\)

\(\forall\barx\in M^n(\varphi^M(\barx)\leftrightarrow\psi^M(\barx))\),\(\forall\barx\in N^n(\varphi^N(\barx)\leftrightarrow\psi^N(\barx))\)

若\(\varphi\)是绝对的,\(\forall\barx\in M^n(\varphi^M(\barx)\leftrightarrow\varphi^N(\barx))\)

因此\(\forall\barx\in M^n(\psi^M(\barx)\leftrightarrow\psi^N(\barx))\)
\end{proof}

\begin{definition}[]
假设\(M\subseteq N\),\(f(x_1,\dots,x_n)\)是函数(类),设\(f(x_1,\dots,x_n)\)被公式\(\varphi(x_1,\dots,x_n,y)\)定义,称\(f\)相对
于\(M,N\)是绝对的,是指
\begin{enumerate}
\item \(\varphi(x_1,\dots,x_n,y)\)相对于\(M,N\)绝对
\item \(\forall\barx\in M^n\exists!y\in M(\varphi^N(\barx,y))\)
\end{enumerate}
\end{definition}

由上一引理,\(f\)的绝对性与定义\(f\)的公式无关

\(f^M=\{(x_1,\dots,x_n,y)\in M^{n+1}\mid\varphi^M(\barx,y)\}\),\(f\uhr M=\{(x_1,\dots,x_n,y)\in M^{n+1}\mid\varphi(\barx,y)\}\)

\(f\)是绝对的当且仅当\(\forall\barx M\forall y\in M(\varphi(\barx,y)\leftrightarrow\varphi^M(\barx,y))\)当且仅当\(\varphi(M^n,M)=\varphi^M(M^n,M)\),
即\(f\uhr M=f^M\)

即对任意\(\bara\in M^n\),有\(f\uhr M(\bara)=f^M(\bara)\)

\begin{theorem}[]
以下关系和函数可以在\(\ZFm-\Pow-\Inf\)中用公式定义,且在\(\ZFm-\Pow-\Inf\)下等价于一
个\(\Delta_0\)-formula
\begin{enumerate}
\item \(x\in y\)
\item \(x=y\)
\item \(x\subset y\)
\item \(\{x,y\}\)
\item \(\{x\}\)
\item \((x,y)\)
\item \(\emptyset\)
\item \(x\cup y\)
\item \(x\cap y\)
\item \(x-y\)
\item \(x^+=x\cup\{x\}\)
\item \(x\)传递
\item \(\bigcup x\)
\item \(\bigcap x\),且\(\bigcap\emptyset=\emptyset\)
\end{enumerate}
\end{theorem}

\begin{proof}
\begin{enumerate}
\setcounter{enumi}{3}
\item 函数\(z=\{x,y\}\)被公式\(\forall u\in z(u=x\vee u=y)\wedge(x\in z\wedge y\in z)\)
\item \(y=\{x\}\)被公式\(\forall u\in y(u=x)\wedge (x\in y)\)
\item 函数\(z=(x,y)\)被公式\(\forall u\in z(u=x\vee x=\{x,y\})\wedge x\in z\wedge \exists  u\in z(u=\{x,y\})\)
\item \(\forall y\in x(y\neq y)\)
\item 函数\(z=x\cup y\)被公式\(\forall x\subset z\wedge y\subset z\wedge\forall u\in z(u\in x\vee u\in y)\)
\item 函数\(z=x\cap y\)被公式\(z\subset x\wedge z\subset y\wedge\forall u\in x(u\in y\to u\in z)\)
\item 函数\(z=x-y\) \(\forall u\in z(u\in x\wedge u\notin y)\wedge \forall u\in x(u\notin y\to u\in z)\)
\item 函数\(z=x^+\) \(\forall u\in z(u\in x\vee u=x)\wedge x\in z\wedge x\subset z\)
\item \(\forall y\in x(y\subset x)\)
\item 函数\(z=\bigcup x\), \(\forall u\in z\exists y\in x(u\in y)\wedge\forall u\in x(u\subset z)\)
\item 函数\(z=\bigcap x\), \(x=\emptyset\to z=\emptyset\wedge\forall u\in z\forall y\in x(u\in y)\wedge\exists y\in x\forall u\in z(\forall w\in x(u\in w)\to u\in z)\)
\end{enumerate}
\end{proof}

\begin{lemma}[]
如果\(M\)是一个传递类,\(f\)是一个被\(\Delta_0\)公式定义的函数,如果\(f\)在\(M\)上封闭,
即\(f(M^n)\subseteq M\),则\(f\)相对于\(M\)绝对
\end{lemma}

\begin{proof}
设\(f\)被\(\varphi(\barx,y)\)定义,\(\forall\barx,y\in M(\varphi(\barx,y)\leftrightarrow\varphi^M(\barx,y))\),\(\forall\barx\in M\exists!y\in M(\varphi(\barx,y))\)
\end{proof}

\begin{corollary}[]
之前的函数均在\(\ZFm-\Pow-\Inf\)的传递模型\(M\)中绝对的

\(\ZFm-\Pow-\Inf\)能够保证函数封闭,传递性保证定义它们的公式的绝对性
\end{corollary}

\begin{lemma}[]
绝对性对复合运算封闭,即假设\(M\subseteq N\),公式\(\varphi(x_1,\dots,x_n)\)函
数\(f(x_1,\dots,x_n)\), \(g_i(y_1,\dots,y_m)\),\(1\le i\le n\)都相对于\(M,N\)绝对,则
\(\varphi(g_1(y_1,\dots,y_m),\dots,g_n(y_1,\dots,y_m))\)与\(f(g_1(y_1,\dots,y_m),\dots,g_n(y_1,\dots,y_m))\)也相对于\(M,N\)绝对
\end{lemma}

\begin{proof}
不妨设\(m=n=1\)

设\(g(y)=z\)被\(\theta(y,z)\)定义,\(\varphi(g(y)):=\exists z(\theta(y,z)\wedge\varphi(z))\)

\(\varphi^M(g(y)):=\exists z\in M(\theta^M(y,z)\wedge\varphi^M(z))\),\(\varphi^N(g(y)):=\exists z\in N(\theta^M(y,z)\wedge\varphi^M(z))\)

由绝对性
\(\forall z\in M\forall y\in M(\theta^M(y,z)\wedge\varphi^M(z)\leftrightarrow\theta^N(y,z)\wedge\varphi^N(z))\)

任取\(y_0\in M\),设\(\exists z\in N(\theta^N(y_0,z)\wedge\varphi^N(z))\),由函数\(g(y)=z\)的绝对性,
\(\forall y\in M\exists! z\in M(\theta^N(y,z))\),存在唯一的\(z_0\in M\)使得
\(\theta^N(y_0,z_0)\wedge\varphi^N(z_0)\)
\end{proof}

\begin{theorem}[]
以下关系和函数对任意\(\ZFm-\Pow\Inf\)的传递模型\(M\)都是绝对的
\begin{enumerate}
\item \(z\)是有序对
\item \(A\times B\)
\item \(R\)是关系
\item \(\dom(R)\)
\item \(\ran(R)\)
\item \(f\)是函数
\item \(f(x)\)
\item \(f\)是一一函数
\end{enumerate}
\end{theorem}

\begin{proof}
\begin{enumerate}
\item ``\(z\)是有序对'':\(\exists u,v(z=(u,v))\),但是这不是\(\Delta_0\),因此考虑
\(\exists u\in\bigcup z\exists v\in\bigcup z(z=(u,v))\),注意这里不是平常的受囿量词,但是
令\(g_1(z)=\bigcup z\), \(g_2(z)=\bigcup z\), \(g_3(z)=z\), \(\varphi(x_1,x_2,x_3):=\exists u\in x_1\exists v\in x_2(x_3=(u,v))\),则
\(g_1,g_2,g_3,\varphi\)绝对,故\(\varphi(g_1(z),g_2(z),g_3(z))\)绝对
\item 函数\(z=x\times y\): \(\forall u\in z\exists s\in x\exists t\in y(u=(s,t))\wedge\forall s\in x\forall t\in y\exists u\in z(u=(s,t))\)
\item \(R\)是关系\(\Leftrightarrow \forall u\in R(u\text{是有序对})\)
\item 函数,\(D=\dom(R)\): \(\forall x\in D\exists z\in R\exists u\in z\exists y\in u(z=(x,y))\)且
\(\forall z\in R\forall u\in z\forall x\in u\forall y\in u(z=(x,y)\to x\in D)\)
\item 同理
\item \(f\text{是关系}\wedge\forall x\in\dom(f)\exists!y\in\ran(f)\exists u\in f(u=(x,y))\)
\item \(\varphi(f(x))\)表示\(f\)是函数且\(x\in\dom(f)\),则``\(y=f(x)\)''定义为
\(\varphi(f,x)\to\exists u\in f(u=x,y)\vee(\neg\varphi(f,x)\to y\neq\emptyset)\)
\item ``\(f\)是函数''且\(\forall s\in\dom(f)\forall t\in\dom(f)(f(s)=f(t)\to s=t)\)
\end{enumerate}
\end{proof}
\subsection{基础公理的相对一致性}
\label{sec:org7a963b7}
如果\(\ZFm\)一致,则\(\ZF\)一致

目标:\(V\vDash\ZFm\),证明\(\WF\vDash\ZF\),等价于\(\ZFm\vdash(\ZF)^{\WF}\)

\begin{lemma}[]
若传递类\(M\)是\(\ZFm-\Pow-\Inf\)的模型,且\(\omega\in M\),则无穷公理在\(M\)中为真,因此无穷公理
在\(\WF\)中为真(\(\ZFm\vdash(\Inf)^{\WF}\))
\end{lemma}

\begin{proof}
\begin{itemize}
\item 由于\(\emptyset\)与\(x^+\)都被\(\Delta_0\)公式定义
\item 若\(M\vDash\ZFm-\Pow-\Inf\),则\(x^+\)在\(M\)中封闭,且\(\emptyset\in M\)
\item \(\emptyset^M=\emptyset\), \((x^+)^M=x^+\)
\item 无穷公理的相对化为\(\exists x\in M(\emptyset\in x\wedge\forall y\in x(y^+\in x))\)
\item 即\(\exists x\in M(\emptyset\in M\wedge\forall y\in x(y^+\in x))\)
\item 由于\(\omega\in M\),令\(x=\omega\)
\end{itemize}
\end{proof}

结论: \(\WF\vDash\ZF\)


目标:\(Con(\ZFm)\vdash Con(ZF)\)

\begin{theorem}[]
设\(T\) (\(\ZFm\))是集合论的的理论,\(\Sigma\) (\(\ZF\))是一个句子集, 设\(M\)是一个类且非空,如
果\(T\vdash(M\vDash\Sigma)\),即\(T\vdash\Sigma^M\)或者“若\(V\vDash T\),则\(V\vDash\Sigma^M\)”,则
\begin{enumerate}
\item 对集合论语言的任何语句\(\varphi\),如果\(\Sigma\vdash\varphi\),则\(T\vdash\varphi^M\)
\item 如果\(T\)一致,则以\(\Sigma\)为公理的理论也一致
\end{enumerate}
\end{theorem}

\begin{proof}
\begin{enumerate}
\item 设\(\varphi_1,\dots,\varphi_n=\varphi\)是\(\Sigma\)的一个证明,对\(k\le n\),归纳证明\(T\vdash\varphi_k^M\)
\begin{itemize}
\item 若\(\varphi_i\in\Sigma\cup Ax\),\(Ax\)一阶逻辑的公理,\(T\vdash\varphi_i^M\)
\item 若\(i,j<k\)使得\(\varphi_j=\varphi_k\to\varphi_k\),由归纳假设\(T\vdash\varphi_i^M\), \(T\vdash\varphi_i^M\to\varphi_k^M\),因此\(T\vdash\varphi_k^M\)
\end{itemize}
\item 若\(\Sigma\)不一致,则存在\(\varphi\)使得\(\Sigma\vdash\varphi\wedge\neg\varphi\),从而\(T\vdash(\varphi\wedge\neg\varphi)^M\),故\(T\)不一致
\end{enumerate}
\end{proof}

\begin{theorem}[]
基础公理相对于\(\ZFm\)一致,即如果\(\ZFm\)一致,则\(\ZF\)一致
\end{theorem}

\begin{proof}
\begin{itemize}
\item \(\ZFm\vdash(\ZF)^{\WF}\)
\item 故\(\ZFm\)一致能推出\(\ZF\)一致
\end{itemize}
\end{proof}

选择公理:任何非空集合都可被良序化
\(\forall X\exists R(R\text{是$X$上的良序})\)
\begin{enumerate}
\item \(R\subseteq X\times X\)
\item \(R\)是线序
\item \(\forall Y\subseteq X\), \(Y\neq\emptyset\Rightarrow\)Y存在\(R\)-极小元
\end{enumerate}


\begin{lemma}[\(\ZFm\)]
\label{7.5.4}
设\(M\)是\(\ZFm-\Pow-\Inf\)的传递模型,如果\(X,R\in M\)并且\(R\)是\(X\)上的一个良序,
则\((R\text{是$X$的良序})^M\)
\end{lemma}

\begin{proof}
“\(R\)是\(X\)上的线序”被公式\(\varphi(X,R)\)表达
\begin{itemize}
\item \(R\)是关系
\item \(\forall x\in X(\neg R(x,x))\)
\item \(\forall x,y,z\in X(R(x,y)\wedge R(y,z))\to R(x,z)\)
\item \(\forall x,y\in X(R(x,y)\vee R(y,x)\vee x=y)\)

\(R(x,y)\)表示\((x,y)\in R\),\(\exists z\in R(z=(x,y))\)
\end{itemize}


因此\(\varphi(X,R)\)是\(\Delta_0\)-公式

令公式\(\psi(X,Y,R)\)为\(Y\subseteq X\wedge Y\neq\emptyset\to\exists y\in Y\forall x\in Y(\neg R(x,y))\),则\(\psi(X,Y,R)\)是\(\Delta_0\)-公式,
“\(R\)是\(X\)上的良序”可以表达为\(\theta(X,Y)=\varphi(X,R)\wedge\forall Y\psi(X,Y,R)\)

则\(\theta\)是一个\(\Pi_1\)-公式

\(\forall X\in  M\forall R\in M(\theta(X,R)\to\theta^M(X,R))\),任取\(X_0,R_0\in M\)使得\(R_0\)是\(X_0\)上的良序,
则\(\theta(X_0,R_0)\),故\(\theta^M(X_0,R_0)\)也成立,即
\end{proof}

\begin{theorem}[\(\ZFm\)]
\(V_\omega\)是\(\ZFC-\Inf+\neg\Inf\)的模型
\end{theorem}

\begin{proof}
与\(\WF\)类似,\(V_\omega\)是传递的,且关于\(\{x,y\}\),\(\bigcup x\),\(\calp(x)\)封闭,故而是\(\ZF-\Inf\)的模型
(练习)

\(\neg\Inf\): \(\forall x\neg(\emptyset\in X)\wedge\forall y\in x(y^+\in x)\)

\(\neg\Inf^M\): \(\forall x\in M(\emptyset^M\in X\wedge\forall y\in x((y^+)^M\in x))\)

由于\(M=V_\omega\)传递,故\((\neg\Inf)^M\): \(\forall x\in M(\emptyset\in X\wedge\forall y\in x(y^+\in x))\)

由于\(V_\omega\)中没有无穷集,故\((\neg\Inf)^M\)在\(V\)中成立

\(AC^M\):任取\(X\in V_\omega\),若\(X\neq\emptyset\),存在\(R\in V_\omega\)使得\(R\)是\(X\)上的良序

\(\rank(\calp(x\times y))<\max(\rank(x),\rank(y))\),故
\(\calp(x\times x)\in V_\omega\)
\end{proof}

\begin{corollary}[]
\(Con(\ZFm)\vdash Con(\ZFC-\Inf+\neg\Inf)\)
\end{corollary}
\subsection{基于良基关系的归纳与递归}
\label{sec:orgde0bef5}
\begin{definition}[]
类\(R\)(\(\varphi(x,y)\))是类\(X\)(\(\psi(x)\))上的良基关系当且仅当
\begin{equation*}
\forall U\subset X(U\neq\emptyset\to\exists y\in U(\neg\exists z\in U(zRy)))
\end{equation*}
\(U\)是集合
\end{definition}

\begin{examplle}[]
\(\in\)是\(\On\)上的良基关系

如果\(Fud\)成立,则\(\in\)是\(V\)上的良基关系
\end{examplle}

\begin{theorem}[超穷归纳原理]
设\(\varphi(x)\)是一个公式,若\(\forall\alpha\in On\)有\(\forall\beta(\beta<\alpha\to\varphi(\beta))\to\varphi(\alpha)\),则\(\forall\alpha\in\On(\varphi(\alpha))\)
\end{theorem}

\begin{theorem}[超穷递归定理]
设\(G:V\to V\)的函数,则存在唯一的函数\(F:\On\to V\)使得
\begin{equation*}
F(\alpha)=G(F\uhr\alpha)
\end{equation*}
\end{theorem}

\begin{definition}[]
类\(X\)上的关系,类\(R\)是 \textbf{似集合} 的当且仅当对任意\(x\in X\),有\(\{y\in X\mid yRx\}\)是一个集合
\end{definition}

类的元素一定是集合,因为类是集合宇宙的一个子区域

一般称\(\{y\in X\mid yRx\}\)中的元素为\(x\)的前驱,\(\in\)是任何类\(X\)上的似集合关系

\begin{definition}[]
如果\(R\)是\(X\)上的似集合关系,且\(x\in X\),则递归定义
\begin{itemize}
\item \(\pred^0(X,x,R)=\{y\in X\mid yRx\}\)
\item \(\pred^{n+1}(X,x,R)=\bigcup\{\pred(X,y,R)\mid y\in\pred^n(X,x,R)\}\)
\item \(\cl(X,x,R)=\bigcup_{n\in\omega}\pred^n(X,x,R)\)
\end{itemize}
\end{definition}

对每个\(n\),\(\pred^n(X,x,R)\)是集合

故\(\cl(X,x,R)\)是集合

若\(R\)是\(\in\),且\(X\)是传递的,则\(\cl(X,x,R)=x\)


\begin{lemma}[]
如果\(R\)是\(X\)上的似集合关系,则对任意\(y\in\cl(X,x,R)\),都有\(\pred(X,y,R)\subseteq\cl(X,x,R)\)
\end{lemma}

\begin{proof}
设\(y\in\cl(X,x,R)\),则存在\(n\in\omega\)使得\(y\in\pred^n(X,x,R)\),故\(\pred(X,y,R)\subseteq\pred^{n+1}(X,x,R)\)
\end{proof}

\begin{theorem}[]
如果\(R\)是\(X\)上的良基关系, 且是似集合的,则\(X\)的每个非空子类\(Y\)都有\(R\)-极小元
\end{theorem}

\begin{proof}
任取\(x\in Y\),若\(x\)不是\(Y\)的\(R\)-极小元,则\(\pred(X,x,R)\cap Y\)非空,于是\(Y\cap\cl(X,x,R)\)非
空,令\(x_0\in Y\cap\cl(X,x,R)\)为极小元,则\(x_0\)是\(Y\)的极小元,否则\(\pred(X,x_0,R)\cap Y=\emptyset\),任
取\(z_0\in\pred(X,x_0,R)\cap Y\),则\(z_0\in Y\), \(z_0\in\cl(X,x,R)\),于是\(z_0\in Y\cap\cl(X,x,R)\)
且\(z_0Rx_0\),与\(x_0\)的极小性矛盾
\end{proof}

\begin{remark}
假设基础公理,则\(\in\)是\(V\)上的良基关系,若\(V\neq\WF\),则\(V\setminus\WF\)有极小元\(x_0\)非空,但
是\(\forall y\in x_0(y\in\WF)\),于是\(x_0\subset\WF\),矛盾,因此\(V=\WF\)
\end{remark}

\begin{theorem}[]
设\(R\)是\(X\)上的似集合的良基关系,如果\(F:X\times V\to V\)是“函数”,则存在唯一的“函数”\(G:X\to V\)使得
\(\forall x\in X(G(x)=F(x, G\uhr\pred(X,x,R)))\)
\end{theorem}

练习
\begin{proof}
\begin{enumerate}
\item 存在性

令公式\(\theta(x,t)\)表示
\begin{itemize}
\item \(t\)是一个函数(集合)
\item \(\dom(t)=\{x\}\cup\pred(X,x,R)\)
\item \(\forall y\in\dom(t)(t(y)=F(y,t\uhr\pred(X,y,R)))\)
\item \(\forall y\notin\dom(t)(t=\emptyset\)
\end{itemize}
令\(G=\{(x,y)\mid\theta(x,y)\}\),证明\(G\)是函数:
\end{enumerate}


\begin{enumerate}
\item 唯一性

若不唯一,则存在最小的\(x\in X\)使得\(G(x)\neq G(x')\). 但是\(G(x)=F(x,G\uhr(X,x,R))=F(x,G'\uhr(X,x,R))=G'(x)\)
\end{enumerate}
\end{proof}

\begin{definition}[]
如果\(R\)是\(X\)上的似集合关系,设\(x\in X\),则定义
\begin{equation*}
\rank(x,X,R)=\sup\{\rank(y,X,R)+1\mid yRx\wedge y\in X\}
\end{equation*}
(来自超穷递归)
\end{definition}

\begin{definition}[\(\ZFm\)]
如果类\(X\)传递,且\(\in\)是\(X\)上的良基关系,则\(X\subseteq\WF\)且对任意\(x\in X\)有\(\rank(x,X,\in)=\rank(x)\)
\end{definition}

\begin{proof}
若\(X\not\subseteq\WF\), 取极小元\(x_0\in X\setminus\WF\),显然\(x_0\neq\emptyset\)。任取\(z\in x_0\),由传递性,有\(z\in X\cap\WF\),
于是\(x_0\subseteq\WF\),于是\(X\subseteq\WF\)

令\(Y=\{x\in X\mid\rank(x,X,\in)\neq\rank(x)\}\),如果\(Y\)非空,令\(x_0\)为\(Y\)的极小元,根据传递
性,\(x_0\subseteq X\),且\(\forall z\in x_0\),\(\rank(z,X,\in)=\rank(z)\)

\(\rank(x_0,X,\in)=\sup\{\rank(z,X,\in)+1\mid z\in x_0\}\)

\(\rank(x_0)=\sup\{\rank(z)+1\mid z\in x_0\}\)
\end{proof}

\begin{definition}[]
令类\(R\)是\(X\)上似集合的良基关系,则\((X,R)\)上的 \textbf{mostowski函数} \(G\)定义为
\begin{equation*}
G(x)=\{G(y)\mid y\in X\wedge yRx\}
\end{equation*}

\(G\)的值域记作\(M\),称之为\((X,R)\)的 \textbf{Mostowski坍塌}
\end{definition}

\begin{lemma}[]
设\(R\)是\(X\)上的一个似集合的良基关系,\(G\)是其上的Mostowski函数,\(M\)为其Mostowski坍塌,则
\begin{enumerate}
\item \(\forall x,y\in X(xRy\to G(x)\in G(y))\), \(G:(X,R)\to(V,\in)\)同态
\item \(M\)是传递集
\item 如果幂集公理成立,则\(M\subseteq\WF\) (\(\ZFm-\Pow-\Inf\))
\item 如果幂集公理成立,且\(x\in X\),则\(\rank(x,X,R)=\rank(G(x))\)
\end{enumerate}
\end{lemma}

\begin{proof}
\begin{enumerate}
\setcounter{enumi}{2}
\item 断言:\((M,\in)\)是良基的

任取\(Y\subseteq M\)非空,则\(G^{-1}(Y)\subseteq X\)非空,有极小元\(x_0\),若\(G(x_0)\)不是\(Y\)的极小元,
则\(G(x_0)\cap Y\neq\emptyset\)。令\(z\in G(x_0)\cap Y\),则存在\(y\in G^{-1}(Y)\)使得\(G(y)=z\)且\(yRx_0\),
与\(x_0\)极小矛盾
\item 设\(x\in X\), \(\rank(G(x))=\sup\{\rank(v)+1\mid v\in G(x)\}=\sup\{\rank(G(y))+1\mid y\in X\wedge yRx\}\)

设\(x_0\)是使得等式不成立的极小元,则对任意\(y\in X\), \(yRx_0\to\rank(y,X,R)=\rank(G(y))\)

\(\rank(x,X,R)=\sup\{\rank(y,X,R)+1\mid yRx\wedge y\in X\}=\sup\{\rank(G(y))+1\mid yRx\wedge y\in X\}=\rank(G(x))\)
\end{enumerate}
\end{proof}

那么\(G\)在什么条件下是个同构

\begin{definition}[]
\(R\)是\(X\)上的 \textbf{外延} 的关系当且仅当
\begin{equation*}
\forall x,y\in X(\forall z\in X(zRx\leftrightarrow zRy)\to x=y)
\end{equation*}
\end{definition}

\begin{lemma}[]
如果\(X\)是传递的,则\(\in\)在\(X\)上是外延的
\end{lemma}

\begin{proof}
\(\pred(X,x,\in)=x\)
\end{proof}

\begin{lemma}[]
令\(R\)是\(X\)上的似集合良基关系,如果\(R\)在\(X\)上是外延的,则\(G\)是同构
\end{lemma}

\begin{proof}
若\(G\)不是单射,即\(Y=\{x\in X\mid\exists y\in X(x\neq y\wedge G(x)=G(y))\}\)非空,则有极小元\(x_0\),取极小
的\(y_0\in Y\)使得\(x_0\neq y_0\)且\(G(x_0)=G(y_0)\),存在\(z_0\in X\)使得\(\neg(z_0Rx_0\leftrightarrow z_0Ry_0)\)

若\(z_0Rx_0\),\(\neg z_0Ry_0\),则\(G(z_0)\in G(x_0)\), \(G(z_0)\notin G(y_0)\)
\end{proof}

\begin{theorem}[莫斯托夫斯基坍塌定理]
令\(R\)是\(X\)上的似集合良基关系,并且在\(X\)上是外延的,则存在传递类\(M\)和双射\(G\)满
足\(G:X\to M\)满足:\(G\)是\((X,R)\)与\((M,\in)\)之间的同构。另外\(M\)和\(G\)唯一
\end{theorem}

\subsection{基础公理的绝对性}
\label{sec:orgd97f332}
已知\(\ZFm\)一致\(\Rightarrow \ZF\)一致

本节工作于\(\ZF\)中

\begin{theorem}[]
以下关系和函数可以在\(\ZF-\Pow\)中用公式定义,且\(\ZF-\Pow\)可以证明这些公式等价于\(\Delta_0\)公式,
所以它们对任意\(\ZF-\Pow\)的传递模型绝对
\begin{enumerate}
\item \(x\)是序数
\item \(x\)是极限序数
\item \(x\)是后继序数
\item \(\omega\)
\item \(x\)是有穷序数
\item \(0,1,2,\dots,20,\dots\)
\end{enumerate}
\end{theorem}

\begin{proof}
\begin{enumerate}
\item \(\in\)良基

\(x\)是序数\(\Leftrightarrow x\)是传递集且\(\in\)是\(x\)上的线序

即\(\forall y\in x(y\subset x)\wedge\forall y,z\in x(y\in z\vee y=z\vee z\in y)\)

\item 令\(\psi(x)\)为“\(x\)是序数”且\(\forall y\in x\exists z\in x(y\in z)\)

\setcounter{enumi}{3}
\item 令\(\psi(x)\)为“\(x\)是极限序数”且\(\emptyset\in x\)且\(\forall y\in x(y\text{ is limit}\to y=\emptyset)\)

\item 令\(\psi(x)\)为 “\(x\)是序数”且\(x\neq\omega\)且\(\forall y\in x(y\neq\omega)\)

\item 归纳证明:\(\emptyset\): \(\forall y\in x(y\neq y)\) \(\psi_0(x)\)

假设\(n\)被\(\psi_n(x)\)定义,则\(\psi_{n+1}(x):\exists y\in x(\psi_n(y)\wedge x=y^+)\)
\end{enumerate}
\end{proof}

\begin{remark}
令\(\psi_{limit}(x)\)定义极限序数,即使\(V\vDash\neg\Inf\),\(\psi_{limit}(x)\)相对于\(\ZF-\Pow+\neg\Inf\)的传递
模型\(M\)仍然是绝对的,此时,\(V\vDash\forall x(\psi_{limit}(x)\to x=\emptyset)\)

同理定义\(\omega\)的\(\psi_\omega(x)\)也是绝对的,此时
\(V\vDash\neg\exists(\psi_\omega(x))\)

若\(V\)和\(M\)均满足\(\Inf\),则\(\omega\in M\)且\(\psi_\omega(\omega)\leftrightarrow\psi_\omega^M(\omega)\)
\end{remark}

\begin{lemma}[]
如果\(M\)是\(\ZF-\Pow\)的传递模型,则\(M\)的所有有穷子集都是\(M\)的元素
\end{lemma}

\begin{proof}
令\(\sigma_n\)为
\begin{equation*}
\forall x\subset M(\abs{x}=n\to x\in M)
\end{equation*}
\(V\)看到的
\begin{enumerate}
\item \(\sigma_0\),\(V\vDash(\ZF-\Pow)^{M}\),由于\(\ZF-\Pow\vdash\exists x(x=\emptyset)\),故\(V\vDash\exists x\in M(x=\emptyset^M)\),而空集是一个绝对
概念,因此\(V\vDash\exists x\in M(x=\emptyset)\)
\item 假设\(\sigma_n\)成立,任取\(x\subset M\) s.t. \(\abs{x}=n+1\),任取\(y\in x\),则\(y\in M\),
\end{enumerate}
\end{proof}

\begin{theorem}[]
以下概念对\(\ZF-\Pow\)的任何传递模型都是绝对的
\begin{enumerate}
\item \(x\)是有穷的
\item \(X^n\)
\item \(X^{<\omega}\)即\(X\)上所有有穷序列的集合
\end{enumerate}
\end{theorem}

\begin{proof}
\begin{enumerate}
\item 令\(\psi(x,f)\)表示“\(f\)是函数”且\(\dom(f)=x\)且\(\ran(f)\in\omega\)且``\(f\)是一一的''

显然\(\psi(x,f)\)是绝对的,\(x\)有穷\(\Leftrightarrow \exists f\psi(x,f)\)

目标:\(\forall x\in M(x\text{ finite}\leftrightarrow(x\text{ finite})^M)\),即
\(\forall x\in M(\exists f\psi(x,f)\leftrightarrow\exists f\in M\psi(x,f))\)

任取\(x_0\in M\),若存在\(f_0\in M\)使得\(\psi^M(x_0,f_0)\)成立,则\(\psi(x_0,f_0)\)成立,

若存在\(f\)使得\(\psi(x_0,f)\)成立,下面证明\(f_0\in M\)。存在\(n\in\omega\)使得\(f_0:x\to n\)是一一的函数,
\(f_0\subseteq x_0\times n\)是有穷集

\(n\)与\(x_0\)均属于\(M\),故\(x_0\times n\in M\),故\(x_0\times n\subset M\),故\(f_0\subseteq M\)是有穷子集,

\item \(X^n\)是\(n\)到\(X\)的所有函数的集合

令\(f:n\to X\)表示“\(f\)是函数”且\(\dom(f)=n\)且\(\ran(f)\subseteq X\)

\(f\)是绝对的,于是\(\forall f,n,X\in M((f:n\to X)\leftrightarrow(f:n\to X)^M)\)

定义函数
\begin{equation*}
F(X,n)=
\begin{cases}
0&n\notin\omega\\
\{f\mid f:n\to X\}&n\in\omega
\end{cases}
\end{equation*}
\(Z=F(X,n)\)被公式\(\psi(X,n,Z)\)表示:\((n\notin\omega\to Z=0)\wedge(n\in\omega\to Z=\{f\mid f:n\to X\})\)

下面证明\(\psi\)的绝对性,只需证明\(\forall n\in\omega\)以及\(X_0,Z_0\in M\),
\begin{equation*}
\forall y\in Z_0(y:n\to X_0)\wedge\forall f((f:n\to X_0)\to f\in Z_0)
\end{equation*}
有绝对性,唯一的障碍是\(\forall f\),但是因为当\(n,X_0\in M\)且\(f:n\to X_0\),则\(f\)是\(M\)的有穷子集

故\(\psi(X,n,Z)\)是绝对的

下面验证, \(X^n\subseteq\calp(n\times X)\in M\),于是\(F\)有绝对性
\begin{equation*}
V\vDash\forall X\in M\forall n\in M\exists!Z\in M\psi(X,n,Z)
\end{equation*}
任取\(X\in M\),若\(n\notin\omega\),则\(F(X,n)=\emptyset\in M\),若\(n\in\omega\),定义
\(\theta_n(x,y)\)为
\begin{equation*}
\exists a_0\dots a_{n-1}(x=(a_0,\dots,a_{n-1})\wedge y=\{(0,a_0),\dots,(n-1,a_{n-1})\})
\end{equation*}
令\([X^n]\)表示\(n\)次笛卡儿积,显然\([X^n]\in M\)

\(\forall x\in[X^n]\exists!y\in M\theta_n^M(x,y)\)

由于\(M\)满足替换公理,故存在\(z\in M, X^n\subseteq z\)

根据分离公理
\begin{equation*}
V\vDash\exists u\in M\forall f\in M(f\in u\leftrightarrow f\in z\wedge(f:n\to x))
\end{equation*}
故\(u=X^n\in M\)

\item 先证明封闭,再证明绝对

首先证明函数\(Z=X^{<\omega}\)是绝对的

令\(F(X,n)=X^n\),则\(Z=\bigcup\{F(X,0),F(X,1),\dots\}=\bigcup\ran(F(X,-))\uhr\omega\)

由于\(\omega\in M\),于是\(\ran(F(X,-)\uhr\omega)\in M\),由并集公理,\(\bigcup\ran(F(X,-)\uhr\omega)\in M\)

即\(X\in M\Rightarrow X^{<\omega}\in M\)

\(Z=X^{<\omega}\)被公式\(\varphi(x,z)\)定义:\(\forall f(f\in z\leftrightarrow\exists n(n\text{ fintie ordinal}\wedge f\in X^n))\)

验证:\(\forall x\in M\forall z\in M(\varphi(x,z)\leftrightarrow\varphi^M(x,z))\)

\(V\)看到所有有穷序数都在\(M\)中

于是\(\varphi\)绝对,\(\forall x\in M\exists!z\in M\varphi(x,z)\)
\end{enumerate}
\end{proof}

\begin{theorem}[]
以下概念对\(\ZF-\Pow\)的任何传递模型都是绝对的
\begin{enumerate}
\item \(R\)是\(X\)上的良序(集合)
\item \(\type(x,R)\)
\end{enumerate}
\end{theorem}

\begin{proof}
\begin{enumerate}
\item 已证明:\(\forall R\in M\forall x\in M(R\text{是$X$的良序}\to(\text{$R$是$X$的良序})^M)\) (\ref{7.5.4})

另一方面,\(\ZF-\Pow\vdash\forall R\forall X[R\text{是$X$的良序}\to\exists\alpha\exists f(\alpha\text{ ordinal}\wedge f:(\alpha,\in)\cong(X,R)]\)

后面的部分是绝对的

同时这个也有\(M\)的相对化
\((\ZF-\Pow)^M\vdash\forall R\in M\forall X\in M[(R\text{是$X$的良序})^M\to\exists\alpha\in M\exists f\in M(\alpha\text{ ordinal}\wedge f:(\alpha,\in)\cong(X,R)]\)

若\(R_0,X_0\in M\)且\((R_0\text{是$X_0$的良序})^M\),则存在\(\alpha\in M\), \(f\in M\),\(f:(\alpha,\in)\cong(X_0,R_0)\),
因此\(V\vDash R_0\text{是$X_0$的良序}\)
\item 令\(W(X,R)\)表示\(R\)是\(X\)的良序,令\(\chi(X,R,Z)\)表示\(Z\)是序数且\(W(X,R)\)
且\(\exists f:(Z,\in)\cong(X,R)\)

则\(Z=\type(X,R)\Leftrightarrow\chi(X,R,Z)\),而\(\chi\)是绝对(这里的问题是\(\exists f\),要证明\(f\)一定在\(M\)中,参
考良序绝对性的证明,\(f\subset Z\times X\in M\))的
且\(\forall X,R\in M\exists!Z\in M\chi(X,R,Z)\) (练习)
\end{enumerate}
\end{proof}

\begin{theorem}[]
以下概念对\(\ZF-\Pow\)的任何传递模型都是绝对的
\begin{enumerate}
\item \(\alpha+1\)
\item \(\alpha-1\)
\item \(\alpha+\beta\)
\item \(\alpha\cdot\beta\)
\end{enumerate}
\end{theorem}

\begin{proof}
\begin{enumerate}
\setcounter{enumi}{1}
\item \(x=\alpha-1\)被
\begin{equation*}
\alpha\neq 0\wedge((\alpha\text{后继}\wedge\alpha=x+1)\vee(\alpha\text{极限}\wedge\alpha=x))
\end{equation*}
\item 没有递归定义的绝对性

\(\alpha+\beta\)的定义为\(\type(\alpha\oplus\beta)\)

由于\(\type(-,-)\)是绝对的,只需证明\(\alpha\oplus\beta\)是绝对的

令\(F(\alpha,\beta)=W\),其中\(W=\alpha\times\{0\}\cup\beta\times\{1\}\),再令\(G(\alpha,\beta)=R\),其中\(R\subseteq W^2\)且满足
\(\forall x\in\alpha\times\{0\}\forall y\in\beta\times\{1\}(xRy)\)且\(\forall x,y\in\alpha((x,0)R(y,0)\leftrightarrow x\in y)\)
且\(\forall x,y\in\beta((x,1)R(y,1)\leftrightarrow x\in y)\)

显然\(R\)是\(W\)的良序集

\(F\)是绝对的

令\(\psi(\alpha,\beta,R)\)为
\begin{align*}
\forall x\in R[&\exists a\in\alpha\exists b\in\alpha(a\in b\wedge x=((a,0),(b,0)))\\
&\vee\exists a\in\beta\exists b\in\beta(a\in b\wedge x=((a,1),(b,1)))\\
&\vee\exists a\in\alpha\exists b\in\beta(x=((a,0),(b,1)))]\\
\wedge\forall a,b\in\alpha&\exists x\in R(x=((a,0),(b,0)))\\
\wedge\forall a,b\in\beta&\exists x\in R(x=((a,1),(b,1)))\\
\wedge\forall a\in\alpha&\forall b\in\beta\exists x\in R(x=((a,0),(b,1)))
\end{align*}
用\(\theta(\alpha,\beta,x)\)表示方括号,则\(V\vDash\forall z(z\in R\leftrightarrow\theta(\alpha,\beta,z))\)

于是\(G(\alpha,\beta)=R\Leftrightarrow\psi(\alpha,\beta,R)\)

\(\psi,\theta\)是绝对的

若\(\alpha,\beta\in M\),则\(\{x\mid\theta(\alpha,\beta,x)\}=\{x\in M\mid\theta(\alpha,\beta,x)\}=\{x\in M\mid\theta^M(\alpha,\beta,x)\}\subseteq M\),
\(R=\{x\in W^2\mid\theta^M(\alpha,\beta,x)\}\),由分离公理,\(R\in M\)

故\(G(\alpha,\beta)=R\)是绝对的,

\(\alpha+\beta=\type(F(\alpha,\beta),G(\alpha,\beta))\)是绝对的
\item 同理:\(\alpha\cdot\beta=\type(\alpha\otimes\beta)\)是绝对的

令\(F(\alpha,\beta)=W\),其中\(W=\alpha\times\beta\),再令\(G(\alpha,\beta)=R\),其中\(R\subseteq W^2\)满足
\(\forall x,y\in\alpha\forall u,v\in\beta((x,u)R(y,v)\leftrightarrow(x<y\vee(x=y\wedge u<v)))\),于是\(R\)是\(W\)的良序集,\(F\)是绝对的,
同理令\(V\vDash\forall z(z\in R)\leftrightarrow\theta(\alpha,\beta,z)\),\(G(\alpha,\beta)=R\Leftrightarrow\psi(\alpha,\beta,R)\),\(\psi,\theta\)绝对。

若\(\alpha,\beta\in M\),由分离公理,\(R\in M\),因此\(G(\alpha,\beta)=R\)是绝对的,故

\(\alpha\cdot\beta=\type(F(\alpha,\beta),G(\alpha,\beta))\)是绝对的
\end{enumerate}
\end{proof}

设\(X\)是一个类,被公式\(X(x)\)定义,称\(X\)绝对是指\(\forall x\in M(X(x)\leftrightarrow X^M(x))\)

令\(X^M\)表示\(\{x\in M\mid X^M(x)\}\),\(X\)对于\(M\)绝对\(\Leftrightarrow\) \(X^M=X\cap M\)

若\(M\)是\(ZF-\Pow\)的传递模型,则\(\On^M=M\cap\On\)

作为函数的类,\(G:X\to Y\)其中\(X,Y\)是类,是一个公式\(G(x,y)\)满足函数的条件

称\(G\)相对于类\(M\)是绝对的是指
\begin{enumerate}
\item \(\forall x\in X^M\exists!y\in Y^MG^M(x,y)\),即\(G^M:X^M\to Y^M\)
\item \(\forall x\in M\forall y\in M(G^M(x,y)\leftrightarrow G(x,y))\)
\end{enumerate}


\begin{theorem}[]
设\(R\)是\(X\)的似集合的良基关系,\(F:X\times V\to V\),令\(G:X\to V\)如递归定理所定义的:
\begin{equation*}
\forall x\in X(G(x)=F(x,G\uhr\pred(X,x,R)))
\end{equation*}
令\(M\)是\(ZF-\Pow\)的传递模型,且假设
\begin{enumerate}
\item \(F\)相对于\(M\)绝对的
\item \(X,R\)相对于\(M\)是绝对的
\item \((R\text{在$X$上是似集合的})^M\)
\item \(\forall x\in M(\pred(X,x,R)\subseteq M)\)
\end{enumerate}
则\(G\)对\(M\)是绝对的
\end{theorem}

\begin{proof}
阅读书中证明

\(V\vDash(X^M=X\cap M)\)

\(V\vDash(R^M=R\cap(M\times M))\)

\(V\vDash R^M=(X^M)^2\cap R\)

\(V\vDash(X^M,R^M)\)是良基的


\(R\)在\(X\)上是似集合的,\(\forall x\in X\exists z\forall y\in X(y\in z\leftrightarrow yRx)\),它的相对化就是
\(\forall x\in X^M\exists z\in M\forall y\in X^M(y\in z\leftrightarrow yR^Mx)\)

故\((X^M,R^M)\)是似集合的且\(\forall x\in M(\pred(X^M,x,R^M)\in M)\)

由\(X\)与\(R\)的绝对性,\(\pred(X^M,x,R^M)=\pred(X,x,R)\cap M\)

由于\(\forall x\in M(\pred(X,x,R))\subseteq M\),故\(\forall x\in M(\pred(X^M,x,R^M)=\pred(X,x,R))\)

\textbf{断言1} :函数\(y=\pred(X,x,R)\)是绝对的

\(y=\pred(X,x,R)\)被公式\(\psi(x,y)\)表示:
\begin{equation*}
\forall z(z\in y\leftrightarrow z\in X\wedge zRx)
\end{equation*}
则\(\psi^M(x,y)\)为
\begin{equation*}
\forall z\in M(z\in y\leftrightarrow z\in X^M\wedge zR^Mx)
\end{equation*}
若\(x_0,y_n\in M\),有\(z\in y_0\to z\in M\), \(zRx_0\to z\in M\)

故\(\psi\)绝对,由以上分析,若\(x\in M\),则\(\pred(X,x,R)\in M\)。故\(y=\pred(X,x,R)\)是作为函数是绝对的

对任意的\(x\in M\),有\((\pred(X,x,R))^M=\pred(X,x,R)=\pred(X^m,x,R^M)\)

先在\((X^M,R^M)\)是似集合的的良基关系,由绝对性,\(F^M:X^M\times M\to M\),这些都是\(V\)看到的,那么由递归
定理,存在函数\(g:X^M\to V\)满足
\begin{equation*}
\forall x\in X^M(g(x)=F^M(x,g\uhr\pred(X^M,x,R^M)))
\end{equation*}
目标:证明\(g=G^M\)(书本)

问题:递归定理中的``\(G\)''只刻画了\(G\)的性质并非定义(元语言)

回亿:\(G(x)\)的定义
令公式\(\theta(x,t)\)表示
\begin{itemize}
\item \(t\)是一个函数(集合)
\item \(x\in X\)
\item \(\dom(t)=\{x\}\cup\pred(X,x,R)\)
\item \(\forall y\in\dom(t)(t(y)=F(y,t\uhr\pred(X,y,R)))\)
\item \(\forall y\notin\dom(t)(t=\emptyset)\)
\end{itemize}
则\(G(x)=y\Leftrightarrow \exists t(\theta(x,t)\wedge y=t(x))\)

下面证明\(\exists t(\theta(x,t)\wedge y=t(x))\)的绝对性

\textbf{断言2} :\(\theta(x,t)\)是绝对的

只需证明\(t\uhr\pred(X,y,R)\)是绝对的,即若\(x_0\in X^M,y_0\in\pred(X,x_0,R)\),\(t_0\in M\),则
\(t_0\uhr\pred(X,y_0,R)=(t_0\uhr\pred(X,y_0,R))^M\)

函数\(s=t\uhr\pred(X,y,R)\)被公式
\begin{equation*}
\eta(y,t,s):=\forall x\in s\exists u\exists v(uRy\wedge v=t(u)\wedge x=(u,v))\wedge
\forall u\forall v(uRy\wedge v=t(u)\to (u,v)\in s)
\end{equation*}
验证:\(\eta\)是绝对的(练习),但是\(uRy\),因此\(u\in M\),

故\(\theta(x,t)\)是绝对的

\textbf{断言3} :\(\theta(x,t)\)定义了一个类函数,即\(V\vDash\forall x\in X\exists!t\theta(x,t)\)
练习(对\(x\in X\)归纳证明)

下面证明\(\theta\)作为函数是绝对的
\textbf{断言4} :若\(x\in M\),则\(\forall t(\theta(x,t)\to t\in M)\)

否则,存在一个极小的\(x_0\in M\),\(t_0\)使得\(\theta(x_0,t_0)\)且\(t_0\notin M\)

若\(\pred(X,x_0,R)=\emptyset\),则由\(\theta\)的定义,\(t_0=\{(x_0,F(x_0,\emptyset))\}\in M\),矛盾

若\(\pred(X,x_0,R)\neq\emptyset\),令\(t^*=\{y\mid\exists x\in\pred(X,x_0,R)\wedge\theta(x,y)\}\),由极小性,\(t^*\subseteq M\)

\(t^*=\ran(\theta\uhr\pred(X,x_0,R))\)

由归纳假设,\(\forall x\in\pred(X,x_0,R)\exists!y\in M(\theta(x,y))\)

于是\(\forall x\in\pred(X,x_0,R)\exists!y\in M(\theta^M(x,y))\)

因此\(t^*=\ran(\theta^M\uhr\pred(X,x_0,R))\)

由替换公理,\(t^*\in M\),由绝对性
\begin{equation*}
t_0=(\bigcup t^*)\cup\{(x_0,F^M(x_0,\bigcup t^*))\}\in M
\end{equation*}
矛盾

故\(\forall x\in M\exists! t\in M\theta(x,t)\),即\(\theta(x,t)\)作为函数绝对

记\(\phi(x,y):=\exists t(\theta(x,t)\wedge y=t(x))\),则
\begin{equation*}
\phi^M(x,y)=\exists t\in M(\theta(x,t)\wedge y=t(x))
\end{equation*}
但是\(\forall x\in M\forall y\in M\)
\begin{equation*}
(\exists t(\theta(x,t)\wedge y=t(x)))\leftrightarrow\exists t\in M(\theta(x,t)\wedge y=t(x))
\end{equation*}

下面证明\(G(x)\)作为函数绝对,即\(G(x)\)封闭

回亿:\(g:X^M\to M\)满足
\begin{equation*}
\forall x\in X^M(g(x)=F^M(x,g\uhr\pred(X^M,x,R^M)))
\end{equation*}

\textbf{断言5} :\(\forall x\in X^M(G(x)=g(x))\)

否则,存在“极小”的\(x_0\in X^M=X\cap M\)使得\(G(x_0)\neq g(x_0)\)

显然\(\pred(X,x_0,R)=\pred(X^M,x_0,R^M)\neq\emptyset\),否
则\(g(x_0)=F^M(\emptyset,g\uhr\emptyset)=F^M(\emptyset,\emptyset)=F(\emptyset,g\uhr\emptyset)=G(x_0)\)

假设\(\pred(X,x_0,R)=\pred(X^M,x_0,R^M)\neq\emptyset\),由\(x_0\)的极小性,
有\(\forall x\in\pred(X,x_0,R)\cap\pred(X^M,x_0,R^M)\)时,有\(G(x)=g(x)\)

因此\(G\uhr\pred(X,x_0,R)=g\uhr\pred(X^M,x_0,R^M)\)

\(g(x_0)=F^M(x_0,g\uhr\pred(X^M,x_0,R^M))=G(x_0)\),矛盾
\end{proof}

\begin{theorem}[]
一下概念对\(\ZF-\Pow\)的传递模型都是绝对的
\begin{enumerate}
\item \(\alpha^\beta\) (序数)
\item \(\rank(x)\)
\item \(\trcl(x)\)
\end{enumerate}
\end{theorem}

\begin{proof}
\begin{enumerate}
\item 若\(\alpha=0\),则\(\alpha^\beta=0\)

它是递归定义的,因此是绝对的

规定\(\On\times\On\)上的关系\(R\)为
\begin{equation*}
R=\{((\alpha,\beta_1),(\alpha,\beta_2))\mid\beta_1\in\beta_2\}\subseteq\On^2
\end{equation*}
显然\(R\)是良基关系,\(R\)是似集合的,\(\pred(\On^2,(\alpha,\beta),R)=\{\alpha\}\times\beta\)

定义\(F:\On^2\times V\to V\)为
\begin{equation*}
F(\alpha,\beta,x)=
\begin{cases}
0&\alpha=0\vee x\notin\On^3\\
1&\beta=0\wedge\alpha\neq 0\\
\left( \bigcup_{y\in x}\pi_3(y) \right)\cdot\alpha&\text{otherwise},x\in\On^3
\end{cases}
\end{equation*}
有\(M\)的传递性,\(x=(x_1,x_2,x_3)\in M\Rightarrow x_1,x_2,x_3\in M\)

由\((x_1,x_2,x_3)\)的绝对性,\(y=\pi_3(x)\)是绝对的,因为\(y=\pi_3(x)\)为
\begin{equation*}
\exists x_1\exists x_2\exists x_3(x=(x_1,x_2,x_3)\wedge y=x_3)
\end{equation*}

验证\(G(\alpha,\beta)=\alpha^\beta\)是基于\(F\)递归定义的,因此\(G\)是绝对的

\item \(\rank(x)\),即\(\rank(V,x,\in)\)

\(\rank(x)=\sup\{\rank(y)+1\mid y\in x\}\),找\(F\),并证明绝对性,练习

\item \(\trcl(x)=x\cup\bigcup\{\trcl(y)\mid y\in x\}\)练习
\end{enumerate}
\end{proof}

\begin{remark}
\(\alpha+\beta\)也是递归定义的
\end{remark}

\begin{remark}
\begin{itemize}
\item \(\rank(x)\)的定义用到\(V_\alpha\)
\item 当\(M\not\vDash\Pow\),\(V_\alpha^M\)没有意义
\item \(\rank(x)\)仍可递归定义为\(\sup\{\rank(y)+1\mid y\in x\}\)
\item 当\(M\vDash\Pow\),则两种定义等价
\end{itemize}
\end{remark}

定义公式\(\varphi(x,y)\)为
\begin{equation*}
\forall z(z\in y\leftrightarrow z\subseteq x)
\end{equation*}
当\(V\vDash\Pow\),则\(V\vDash\forall x\exists!y\varphi(x,y)\),即\(\varphi(x,y)\)定义了一个函数,记作\(\calp(x)=y\)

若\(M\vDash\Pow\),则
\begin{equation*}
V\vDash\forall x\in M\exists!y\in M\varphi^M(x,y)
\end{equation*}
当\(M\)传递时,\(\subseteq^M\Leftrightarrow\subseteq\),若\(M\vDash\Pow\),则
\begin{equation*}
V\vDash(\varphi^M\text{定义了$M$到$M$的函数})
\end{equation*}
记该函数为\(\calp^M(x)\),即\(\calp^M(x)=\{z\in M\mid z\subseteq^Mx\}\)
当\(M\)传递时,\(\calp^M(x)=\{z\in M\mid z\subseteq x\}=\calp(x)\cap M\)

同理\(V_\alpha^M=\{x\in M\mid(\rank(x)<\alpha)^M\}\)

\begin{lemma}[]
若\(M\)是\(\ZF\)的传递模型,则
\begin{enumerate}
\item 若\(x\in M\),则\(\calp^M(x)=\calp(x)\cap M\)

只需\(\Pow\)加传递
\item 如果\(\alpha\in M\),则\(V_\alpha^M=V_\alpha\cap M\)

只需\(\ZF-\Pow\),若有\(\Pow\),则\(V_\alpha^M\)是由\(\calp^M\)得到的
\end{enumerate}
\end{lemma}

\begin{remark}
\(\calp\)与\(V_\alpha\)作为函数不是绝对的

固定\(x\in M\),则\(\calp(x)\)可以是带参数\(x\)的公式
\begin{equation*}
\calp(x)(y):\forall z(z\in y\leftrightarrow z\in x)
\end{equation*}
此时谓词\(\calp(x)\)是绝对的,\((\calp(x))^M=\calp(x)\cap M\)

固定\(\alpha\in M\cap\On\),则\(V_\alpha\)可以看成带参数\(\alpha\)的谓词,此时\((V_\alpha)^M=V_\alpha\cap M\)是绝对的
\end{remark}
\subsection{不可达基数与ZFC的模型}
\label{sec:org2bbeaa0}

一般来讲,\(V_\alpha\)不是\(\ZF\)的模型,比如\(\ZFm\vdash(V_\omega\vDash\ZFC-\Inf)\)

令\(Z\)表示\(\ZF-\)替换公理模式(Rep)

\(ZC\)表示\(\ZFC-\Rep\)

\begin{theorem}[]
如果\(\gamma>\omega\)是无穷极限序数,则\(\ZF\vdash(V_\gamma\vDash Z)\),\(\ZFC\vdash(V_\gamma\vDash\ZC)\)
\end{theorem}

\begin{proof}
假设\(V\vDash\ZF\)
\begin{itemize}
\item 存在公理:
\item 外延公理:\(\forall x\in V_\gamma\forall y\in V_\gamma\forall u\in V_\gamma((u\in x\leftrightarrow u\in y)\to x=y)\),\(V_\gamma\)传递
\item 分离公理模式:假设\(x\in V_\gamma\),则存在\(\beta<\gamma\)使得\(x\in V_\beta\),故\(x\subseteq V_\beta\),
\(\calp(x)\subseteq\calp(V_\beta)=V_{\beta+1}\subseteq V_\gamma\),则若\(x\in V_\gamma\),则\(X\)的子集均属于\(V_\gamma\)

分离公理
\begin{equation*}
\forall x\in V_\gamma\exists Y\in V_\gamma\forall u\in V_\gamma(u\in Y\leftrightarrow u\in X\wedge\varphi^M(u))
\end{equation*}
在\(V\)里面可以看到这些是\(X\)的子集,且能看到\(X\)的所有子集在\(V_\gamma\)里
\item 对集公理,\(\forall x\forall y\exists z\forall u(u\in z\leftrightarrow u=x\vee u=y)\)

设\(x,y\in V_\gamma\subseteq\WF=V\),有\(\rank(\{x,y\})<\max\{\rank(x),\rank(y)\}+\omega\),故\(\{x,y\}\in V_\gamma\)
\item 并集公理,类似
\item 幂集公理,类似
\item 无穷公理

对于\(\ZFm-\Pow-\Rep\)的传递模型,\(\emptyset\)与后继运算是绝对的

\(\omega\in V_\gamma\),故无穷公理的相对化成立
\item 基础公理

\(\forall x\in V_\gamma((x\neq\emptyset)^{V_\gamma}\to\exists y\in V_\gamma(y\in x\wedge(y\cap x=\emptyset)^{V_\gamma}))\)

对于\(\ZFm-\Pow-\Rep\)的传递模型,\(\emptyset\)与\(\cap\)是绝对的

而\(V_\gamma\subseteq\WF=V\),故Fon成立
\end{itemize}



若\(V\vDash\ZFC\),设\(x\in V_\gamma\),则\(V\vDash\exists R(R\text{是$X$的良序})\)。\(R\subseteq X\times X\Rightarrow R\in V_\gamma\),对
于\(\ZFm-\Pow-\Inf-\Rep\)的传递模型\(V_\gamma\)有
\begin{equation*}
V\vDash(R\text{是$X$的良序})^{V_\gamma}
\end{equation*}
\end{proof}

\begin{exercise}
证明\(V_{\omega+\omega}\)不满足Rep
\end{exercise}

\begin{proof}
\(f:n\to\omega+n\)
\end{proof}

\begin{proposition}[]
\label{7.8.3}
工作在\(\ZFC\)
\begin{itemize}
\item \(\ZF\)不能证明“\(V_\omega\)存在”
\item \(\ZF\)不能证明“对任意\(x\),\(\trcl(x)\)存在”
\end{itemize}
\end{proposition}

\begin{proof}
构造模型否定这两个命题

令\(V\vDash\ZFC\),令\(X_0=\omega\), \(X_{\alpha+1}=\calp(X_\alpha)\), \(X_\gamma=\bigcup_{\beta<\gamma}X_\beta\)(\(\gamma\)极限序数)

显然\(\barX=\bigcup_{\alpha\in\On}X_\alpha=\WF=V\) (练习)

\(X_0\subseteq V_\omega\), \(X_0\in V_{\omega+1}\), \(X_\alpha\subseteq V_{\omega+\alpha}\), \(\barX\subseteq\WF\)

\(V_0\subseteq X_0\), \(V_\alpha\subseteq X_\alpha\), \(\WF\subseteq\barX\)

容易验证以下事实:\(X_\alpha\)传递(归纳),设\(f(x,y)\)表
示\(\{x,y\}\),\((x-y)\),\(x\times y\),\(\bigcup x\),\(\cap x\),\(\calp(x)\),\(\dots\)

若\(x\in X_\alpha\),\(y\in X_\beta\),则\(f(x,y)\in X_{\max\{\alpha,\beta\}+\omega}\)

类似可证\(X_\omega\)是\(\ZC-\Inf\)的传递模型

由于\(\omega\in X_\omega\),故\(X_\omega\vDash\Inf\),即\(X_\omega\vDash\ZC\)

显然\(V_\omega\not\subseteq\omega=X_0\),于是存在\(V_n\not\subseteq X_0\),故\(\calp(V_n)\not\subseteq\calp(X_0)\),即\(\forall k<\omega\),
\(V_{n+k}\not\subseteq X_k\),故\(\forall n<\omega\),都有\(V_\omega\not\subseteq X_n\),故\(V_\omega\notin X_\omega\)

但要严格地说的话得找到一个东西定义\(V_\omega\)然后证明它的相对化在\(X_\omega\)不满足

另一方面,\(V_0\subseteq X_0\Rightarrow V_n\subseteq X_n\),于是\(V_\omega\subseteq X_\omega\)

令\(G:\omega\to\WF\)为\(G(n)=V_\alpha\)

验证\(G\)相对于\(X_\omega\)是绝对的,\(G\)的任何一个片段都是有穷的,因此片段的值域都在\(X_\omega\)中,因
为\(X_\omega\)对于任何有穷集合封闭

注:当\(M\vDash\ZF-\Pow\),我们知道递归函数\(G\)的绝对性,此时\(X_\omega\not\vDash\Rep\),然而\(X_\omega\)的任何有穷
子集都属于\(X_\omega\),故而对任何\(f:\omega\to X_\omega\),有\(f(\{0,\dots,n\})\in X_\omega\),可以证明\(G\)的绝对性(练习)

\(V_\omega\)被公式\(\eta(x):\exists n\in\omega(x\in G(n))\)

(\(V_\omega\)被“\(\alpha\in V_\omega\)”定义,但是\(X_\omega\)不一定认为\(V_\omega\)是集合,必需用\(X_\omega\)认可的方式定义)

\(V_\omega\)存在指的是
\begin{equation*}
\exists y\forall x(x\in y\leftrightarrow\eta(x))
\end{equation*}
由于\(G\)是绝对的,\(\eta(x)\)绝对,因此\(X_\omega\)认为“\(V_\omega\)存在”当且仅
当\(\exists y\in X_\omega\forall x\in X_\omega(x\in y\leftrightarrow\eta(x))\)

由于\(V_\omega\subseteq X_\omega\)且\(X_\omega\)是传递的,以上的公式等价于
\begin{equation*}
\exists y\in X_\omega\forall x(x\in y\leftrightarrow\eta(x))
\end{equation*}
而这样的\(y\)只能是\(V_\omega\),而\(V_\omega\notin X_\omega\),因此以上句子不成立

即\(\ZFC\vdash"X_\omega\vDash\ZC+V_\omega\text{不存在}"\)

证明“\(x\)存在且\(\trcl(x)\)不存在”,假设\(V\vDash\ZFC\),令\(t(u)=\{u\}\),\(x_n=t^n(n)\),
\(\rank(x_n)=2n\),\(x=\{x_n\mid n<\omega\}\),
令\(X_0=x\),\(X_1=\bigcup X_0\),\(\dots\),\(X_{n+1}=\bigcup X_n\),则\(\trcl(x)=\bigcup_{n<\omega}X_n\)

令\(Y_0=\omega\cup X_0\),\(Y_{n+1}=\calp(Y_n)\cup Y_n\cup X_n\),验证\(Y_\omega=\bigcup_{n<\omega}Y_n\)是传递的,验证\(Y_\omega\vDash\ZC\),
验证\(x\in Y_1\subseteq Y\),验证\(\trcl(x)=\bigcup_{n<\omega}X_n\notin Y\),即验证\(\forall n\exists m(X_m\not\subseteq Y_n)\)

后面类似,证明\(Y_\omega\vDash"\trcl(x)\text{不存在}"\)
\end{proof}

\begin{theorem}[]
如果\(\kappa\)是不可达基数,则在\(\ZFm\)中可以证明\(V_\kappa\vDash\ZF\),在\(\ZFCm\)中可以证明\(V_\kappa\vDash\ZFC\)
\end{theorem}

\begin{proof}
已知\(\ZFm\vDash(V_\kappa\vDash Z)\),\(\ZFCm\vDash(V_\kappa\vDash\ZC)\),下面验证替换公理模式
\begin{equation*}
\forall A(\forall x\in A\exists!y\psi(x,y)\to\exists B\forall x\in A\exists y\in B\psi(x,y))
\end{equation*}
相对化
\begin{equation*}
\forall A\in M(\forall x\in A\exists!y\in M\psi^M(x,y)\to\exists B\in M\forall x\in A\exists y\in B\psi(x,y))
\end{equation*}
假设\(A\in V_\kappa\)且\(\forall x\in A\exists!y\in V_\kappa\psi^M(x,y)\)

由于\(\kappa\)是极限序数,故\(A\in V_\kappa\Rightarrow\exists\alpha<\kappa(A\in V_\alpha)\),因此\(A\subseteq V_\alpha\),而\(V\vDash(\psi^M:A\to V_\kappa)\),于
是\(V\vDash\abs{A}\le\abs{V_\alpha}<\kappa\), \(V\vDash\abs{f(A)}<\kappa\),\(V\vDash f(A)\subseteq V_\kappa\),由\(\kappa\)的正则性,所以存
在\(\beta<\kappa\),\(f(A)\subseteq V_\beta\),于是\(f(A)\in V_{\beta+1}\subseteq V_\kappa\),即\(B=f(A)\)即可
\end{proof}

注:\(V_\kappa\)的基数小于\(\kappa\)的子集都是\(V_\kappa\)的元素

若\(M\)的有穷子集都是\(M\)的元素,则\(M\vDash\)有穷Rep

\begin{corollary}[]
\(\ZFC\)中不能证明“存在不可达基数”
\end{corollary}

\begin{proof}
若\(\ZFC\vdash\)“存在不可达基数”,则\(\ZFC\vdash"V_\kappa\vDash\ZFC"\),即\(\ZFC\vdash\exists X(\ZFC)^X\),因为\(V_\kappa\)是个集合,
因此\(\ZFC\vdash\Con(\ZFC)\)

若只能找到一个真类,我们不能得到能证明一致性
\end{proof}

若\(T\)是可公理化的,则
\begin{equation*}
\ZFC\vdash\Con(T)\leftrightarrow\exists M(T)^M
\end{equation*}
(粗略的完全性定理)
取一个适当大的子集\(P\subseteq\ZFC\),有
\begin{equation*}
P\vdash\Con(T)\leftrightarrow\exists M(T)^M
\end{equation*}

已知若\(V\vDash\ZFm\),则\(\WF\vDash\ZF\),\(\ZFm\vdash(\ZF)^{\WF} \not\Rightarrow\) \(\ZFm\vDash Con(\ZF)\),因为\(\WF\)不
是集合

\begin{lemma}[]
设\(\kappa\)是不可达基数(极限序数),则以下概念对\(V_\kappa\)都是绝对的
\begin{enumerate}
\item \(x\)是一个基数
\item \(x\)是正则基数
\item \(x\)是一个不可达基数
\end{enumerate}
\end{lemma}

\begin{proof}
\begin{enumerate}
\item \(x\)是基数被公式\(\varphi(x)\)表示:
“\(x\)是序数”\(\wedge\forall f\forall y\in x((f:y\to x)\to\ran(f)\neq x)\)

三个子公式对\(\ZF-\Pow-\Inf-\Rep\)的传递模型都是绝对的

若\(\kappa\)是极限序数,则
\begin{equation*}
V_\kappa\vDash\ZF-\Pow-\Inf-\Rep
\end{equation*}

由于\(\varphi(x)\)是一个\(\Pi_1\)公式,故
\begin{equation*}
\forall x\in V_\kappa(\varphi(x)\to\varphi^{V_\kappa}(x))
\end{equation*}
另一方面,要证明\(\forall x\in V_\kappa(\varphi^{V_\kappa}(x)\to\varphi(x))\),只需证明若\(x,y\in V_\kappa\)且\(f:y\to x\),
则\(f\in V_\kappa\)

显然若\(f:y\to x\),则\(f\in x^y\),而\(x,y\in V_\alpha\),\(x^y\in V_{\alpha+\omega}\),故\(f\in V_\kappa\),
\(\rank(f)\le\max\{\rank(x),\rank(y)\}+2\)
\item \(x\)是正则基数被公式\(\varphi(x)\)表示:
“\(x\)是基数”\(\wedge\forall f\forall y\in x[(f:y\to x)\to\exists z\in x(\ran(f)\subseteq z)]\)

与1类似
\item \(x\)是不可达基数被公式\(\varphi(x)\)表示:
“\(x\)是正则基数”\(\wedge\forall f\forall y\in x((f:2^y\to x)\to\ran(f)\neq x)\)

\(2^y\)是\(y\)到2的全体函数为绝对概念
\end{enumerate}
\end{proof}

用“\(I\)”表示“存在不可达基数”

\begin{lemma}[]
如果\(\ZFC\)一致,则\(\ZFC+\neg I\)也是一致的,即
\begin{equation*}
\ZFC\vdash\Con(\ZFC)\to\Con(\ZFC+\neg I)
\end{equation*}
\end{lemma}

\begin{proof}
设\(V\vDash\ZFC+\Con(\ZFC)\),\(V\vDash\exists M(ZFC)^M\)

先在\(M\vDash\ZFC\),视\(M\)为集合宇宙,若\(\kappa\)是\(M\)中最小的不可达基数,则\(V_\kappa\vDash\ZFC+\neg I\),
即\(M\vDash(\ZFC+\neg I)^{V_\kappa}\)

存在\(M\)中的元素\(X\)使得\(M\vDash"(X,\in)\vDash\ZFC+\neg I"\),即\(M\vDash(\ZFC+\neg I)^X\),则
\(V\vDash((\ZFC+\neg I)^X)^M\),即\(\forall y\rightsquigarrow\forall y\in X\rightsquigarrow\forall y\in X\cap M\),因此
\(V\vDash(\ZFC+\neg I)^{X\cap M}\) (验证:归纳)

注:\(M\)看到\((X,\in)\)恰好是\(V\)看到的\((X\cap M,\in)\)

因此\(V\vDash\Con(\ZFC+\neg I)\)

若\(M\)中不存在不可达基数,则\(M\vDash\ZFC+\neg I\),因此\(V\vDash(\ZFC+\neg I)^M\)

事实上
\((\N,+,\times,0,1)+AC+\Con(\ZFC)\vdash\Con(\ZFC+\neg I)\)
(完全性要AC)
\end{proof}

以上引理表明:\(\ZFC\not\vdash I\)

以上证明没有使用哥德尔不完全定理

最好的情况是,“\(\ZFC+I\)”一致,即我们希望\(\ZFC\)下构造“\(\ZFC+I\)”的模型

\begin{corollary}[]
在\(\ZFC\)中不能生成“\(ZFC+I\)”的模型,即
\begin{equation*}
\ZFC+\Con(\ZFC)\not\vdash\Con(\ZFC+I)
\end{equation*}
\end{corollary}

\begin{proof}
否则,假设\(\ZFC\)一致,则\(\ZFC+I\)一致,目标\(\ZFC+I\vdash\Con(\ZFC+I)\)

任取\(V\vDash\ZFC+I\),则\(V\vDash(\ZFC)^{V_\kappa}\),由完全性,\(V\vDash\Con(\ZFC)\),因此有了矛盾
\end{proof}

\begin{definition}[]
对任意的无穷基数\(\kappa\)
\begin{equation*}
H_\kappa=\{x\mid\abs{\trcl(x)}<\kappa\}
\end{equation*}
称\(H_\kappa\)的元素为 \textbf{遗传基数\(<\kappa\)的基数}

称\(x\in H_\omega\)为 \textbf{遗传有穷集}
\end{definition}

\begin{lemma}[\(V\vDash\ZFC\)]
对任意的无穷基数\(\kappa\)有
\begin{equation*}
H_\kappa\subseteq V_\kappa
\end{equation*}
\end{lemma}

\begin{proof}
\(V=\WF\),只需验证\(\forall x\in H_\kappa\),有\(\rank(x)<\kappa\)

设\(x\in H_\kappa\),令\(t=\trcl(x)\),令\(s=\{\rank(y)\mid y\in t\}\subseteq\On\),验证\(s\)是序数

假设\(\alpha\)是最小的不属于\(s\)的序数,\(\alpha\subseteq s\),若\(\alpha\neq s\),令\(\beta=\min(s\setminus\alpha)\),因此\(\beta>\alpha\),
令\(y\in t\)使得\(\beta=\rank(y)\),\(\forall z\in y\),\(z\in t\)且\(\rank(z)<\rank(y)\),由\(\beta\)的极小性,
\(\forall z\in y(\rank(z)<\alpha)\),\(\beta=\rank(y)=\sup\{\rank(z)+1\mid z\in y\}\),因此\(\beta\le\alpha\),矛盾

故\(s=\alpha\),且\(\abs{s}\le\abs{t}=\abs{\trcl(x)}<\kappa\),所以\(\alpha<\kappa\),\(x\subseteq\trcl(x)\subseteq V_s\)
\end{proof}

\begin{lemma}[]
如果\(\kappa\)是正则基数,则\(H_\kappa=V_\kappa\)当且仅当\(\kappa\)是不可达基数
\end{lemma}

\begin{proof}
设\(\kappa\)不可达,只需证明\(V_\kappa\subseteq H_\kappa\)

对\(\alpha<\kappa\)进行归纳证明:\(\abs{V_\alpha}<\kappa\) (练习)

设\(x\in V_\kappa\),则存在\(\alpha<\kappa\)使得\(x\in V_\alpha\),\(\trcl(x)\subseteq V_\alpha\),因此\(\abs{\trcl(x)}<\kappa\)

假设\(\kappa\)不是不可达基数,则存在\(\lambda<\kappa\),\(2^\lambda\ge\kappa\),
\(\calp(\lambda)\in V_{\lambda+\omega}\subseteq V_\kappa\),\(\abs{\trcl(P(\lambda))}\ge 2^\lambda\ge\kappa\),因此\(P(\lambda)\in V_\kappa\setminus H_\kappa\)
\end{proof}

\begin{lemma}[]
对于任意无穷基数\(\kappa\)
\begin{enumerate}
\item \(H_\kappa\)传递
\item \(H_\kappa\cap\On=\kappa\)
\item \(x\in H_\kappa\Rightarrow\bigcup x\in H_\kappa\)
\item \(x,y\in H_\kappa\Rightarrow\{x,y\}\in H_\kappa\)
\item \(x\in H_\kappa\)且\(y\subseteq x\),则\(y\in H_\kappa\)
\item 如果\(\kappa\)正则,则
\begin{equation*}
\forall x(x\in H_\kappa\leftrightarrow x\subset H_\kappa\wedge\abs{x}<\kappa)
\end{equation*}
\end{enumerate}
\end{lemma}

\begin{proof}
\begin{enumerate}
\item 设\(x\in y\in H_\kappa\),则\(\abs{\trcl(y)}<\kappa\),而\(\trcl(x)\subset\trcl(y)\),因此\(x\in H_\kappa\)
\item 若\(\alpha<\kappa\),则\(\alpha=\trcl(\alpha)\),因此\(\alpha\in H_\kappa\)

若\(\alpha\in H_\kappa\),则\(\abs{\alpha}<\kappa\),因此\(\alpha<\kappa\)
\item \(\bigcup x\subseteq\trcl(x)\Rightarrow\trcl(\bigcup x)\subseteq\trcl(x)\),故\(x\in H_\kappa\Rightarrow\bigcup x\in H_\kappa\)
\item \(\trcl(\{x,y\})=\{x,y\}\cup\trcl(x)\cup\trcl(y)\)
\item \(\trcl(y)\subseteq\trcl(x)\)
\item 若\(x\in H_\kappa\),由传递性,\(x\subset H_\kappa\),\(\abs{x}\le\abs{\trcl(x)}<\kappa\)

若\(x\subset H_\kappa\),\(\abs{x}<\kappa\),设\(x=\{x_i\mid i<\lambda\}\),则\(\trcl(x)=x\cup\bigcup_{i<\lambda}\trcl(x_i)\),
若\(\abs{\trcl(x)}\ge\kappa\),则\(\forall\alpha<\kappa\),存在\(i<\lambda\)使得\(\abs{\trcl(x_i)}>\alpha\),故\(\lambda\)与\(\kappa\)共尾
\end{enumerate}
\end{proof}

\begin{theorem}[ZFC]
若\(\kappa\)是不可数正则基数,则\(H_\kappa\vDash\ZFC-\Pow\)
\end{theorem}

\begin{proof}
\(H_\kappa\)传递\(\Rightarrow\)外延公理

\(H_\kappa\)非空\(\Rightarrow\)存在公理

由于\(x\in H_\kappa\leftrightarrow x\subset H_\kappa\wedge\abs{x}<\kappa\),故分离公理+替换公理成立

\(H_\kappa\)对\(\bigcup x\)与\(\{x,y\}\)封闭,故对集公理+并集公理成立

\(H_\kappa\)满足以上公理\(\Rightarrow \emptyset,\omega,x^+,x\cap y\)的绝对性

由于\(\omega\in H_\kappa\), \(H_\kappa\vDash\Inf\)

\(\emptyset,x\cap y\)的绝对性,\(H_\kappa\vDash Fud\)

选择公理:\(\forall x\in H_k\exists R\in H_\kappa(R\text{是$X$的良序})^{H_\kappa}\)

已知,若\(x,R\in H_\kappa\),则\(R\)是\(x\)的良序当且仅当\((R\text{是$X$的良序})^{H_\kappa}\) (\(\ZF-\Pow\))

只需验证:如果\(X\in H_\kappa\),则\(\forall R\subseteq X\times X\),有\(R\in H_\kappa\)

显然\(\abs{X}<\kappa\),因此\(\abs{X\times X}<\kappa\),若\(a,b\in X\),
则\(\abs{\trcl((a,b))}<\abs{\trcl(x)}+\aleph_0\),因此\((a,b)\in H_\kappa\),因此\(R\subset H_\kappa\),根据(6),
有\(R\in H_\kappa\)
\end{proof}

\begin{theorem}[ZFC]
如果\(\kappa\)是不可数正则基数,TFAE
\begin{enumerate}
\item \(H_\kappa\vDash\ZFC\)
\item \(H_\kappa=V_\kappa\)
\item \(\kappa\)不可达
\end{enumerate}
\end{theorem}

\begin{proof}
已知\(2\leftrightarrow 3\)

\(1\to 2+3\):若\(\kappa\)不是不可达基数,则存在\(\lambda<\kappa\)使得\(2^\lambda\ge\kappa\),\(\lambda\in H_\kappa\)且\(\calp(\lambda)\notin H_\kappa\),于
是\(H_\kappa\not\vDash\Pow\)

\(V\vDash\forall z\in H_\kappa\forall x\in H_\kappa(x\in z\leftrightarrow x\subseteq\lambda)\)

\(2\to 1\)显然
\end{proof}

以上引理表明,若\(\kappa\)正则且不是不可达的,则
\begin{equation*}
\ZFC\vdash(\ZFC-\Pow+\neg\Pow)^{H_\kappa}
\end{equation*}
故\(\Con(\ZFC)\to\Con(\ZFC-\Pow+\neg\Pow)\),即\(\Pow\)不能由\(\ZFC\)中的其它公理推出

\begin{corollary}[]
\(\Con(\ZFC)\to\Con(\ZFC-\Pow+\forall(x\text{ countable}))\)
\end{corollary}

\begin{proof}
\(H_{\omega_1}\vDash\ZFC-\Pow\)

\(x\)可数:存在\(f\),\((f:x\to\omega)\)是双射,只需要这个\(f\)是属于\(H_{\omega_1}\)就行了,但这是显然的
\(\forall x,y\in H_\kappa\), \(x^y\in H_\kappa\)用性质6

这个可数我们得在\(H_{\omega_1}\)里看到
\end{proof}
\subsection{反映定理}
\label{sec:org747a768}
已知\(V\vDash\ZF\Rightarrow V_\alpha\vDash Z\) \(\alpha>\omega\)

\(V\vDash\ZFC\Rightarrow V_\alpha\vDash\ZC\) (\(\alpha>\omega\))

\(V_\omega\vDash\ZFC-\Inf\)

\(V_\alpha\)不能“反映”\(V\)的全貌,除非\(\alpha\)是不可达基数

对不可达基数\(\kappa\),\(V_\kappa\)能“反映”\(V\)的全貌(不全对)

\(H_\kappa\)也类似,


本节讨论另一个方向:对给定的句子\(\varphi\),若\(\varphi\)在\(V\)中成立,则能否找到\(\alpha\)使得\(V_\alpha\vDash\varphi\)

问:是否存在\(\varphi\),它在\(V\)中成立,但是\(\forall \alpha(V_\alpha\not\vDash\varphi)\) (因为\(\ZC\)少了无穷条Rep)

\begin{theorem}[反映定理]
对于任意有穷\(\varphi_1,\dots,\varphi_n\),存在\(\alpha\)使得
\begin{equation*}
V\vDash\varphi_i\Leftrightarrow V_\alpha\vDash\varphi_i(i=1,\dots,n)
\end{equation*}
即
\begin{equation*}
V\vDash\varphi_i\leftrightarrow\varphi_i^{V_\alpha}
\end{equation*}
\end{theorem}

设\(F\)是一个集合论语言的公式集,如果对每个\(\varphi(x_1,\dots,x_n)\in F\),对每个\(a_1,\dots,a_n\in M\),
有\(M\vDash\varphi[a_1,\dots,a_n]\Leftrightarrow N\vDash\varphi[a_1,\dots,a_n]\),则称\(M\)是\(N\)的相对于\(F\)的初等子模型,记作\(M\prec_FN\)

反映定理是 Löwenheim–Skolem Theorem的有穷“版本”,等价地说\(F\)中的公式相对于\(V_\alpha\)绝对

\begin{lemma}[]
令\(M\subseteq N\)都是类,\(\varphi_1,\dots,\varphi_n\)是对子公式封闭的公式集,则以下命题等价
\begin{enumerate}
\item \(\varphi_1,\dots,\varphi_n\)相对于\(M\)和\(N\)绝对
\item 如果\(\varphi_i\)是形如\(\exists x\varphi_j(x,y_1,\dots,y_m)\)的公式,则
\begin{equation*}
\forall\bary\in M(\exists x\in N\varphi_j^N(x,\bary)\to\exists x\in M\varphi_j^N(x,\bary))
\end{equation*}
\end{enumerate}
\end{lemma}

\begin{proof}
\(1\to 2\): 设\(\varphi_i\)形如这样的形式,由绝对性
\begin{equation*}
\forall\bary\in M(\varphi_i^N(\bary)\leftrightarrow\varphi_i^M(\bary))
\end{equation*}
载有\(\varphi_j\)的绝对性,\(\forall\bary(\exists x\in M\varphi_j^M(x,\bary)\leftrightarrow\exists x\in M\varphi_j^N(x,\bary))\)

\(2\to 1\): 对\(\varphi_i\)的长度归纳证明:
若\(\abs{\varphi_i}\)最小,则\(\varphi_i\)无量词,因此绝对

若长度小于\(\abs{\varphi_i}\)的公式都是绝对的,则\(\varphi_i\)的所有子公式都绝对,而\(\varphi_i\)的形式有以下形式
\begin{enumerate}
\item \(\varphi_j\to\varphi_k\)
\item \(\neg\varphi_j\)
\item \(\exists x\varphi_j(x,\bary)\)
\end{enumerate}


只需验证情形3:任取\(\bary\in M\),由题设条件,
\begin{equation*}
\exists x\in N\varphi_j^N(x,\bary)\to\exists x\in M\varphi_j^N(x,\bary)
\end{equation*}
由\(\varphi_j\)的绝对性,有
\begin{equation*}
\exists x\in N\varphi_j^N(x,\bary)\to\exists x\in M\varphi_j^M(x,\bary)
\end{equation*}
而显然
\begin{equation*}
\exists x\in M\varphi_j^N(x,\bary)\to\exists x\in N\varphi_j^N(x,\bary)
\end{equation*}
\end{proof}

这个证明没有用到有穷性,因此无穷情况也成立

\begin{theorem}[反映定理,ZF]
对于任意有穷公式集\(F=\{\varphi_1,\dots,\varphi_n\}\),对任意\(\alpha\in\On\),存在\(\beta\ge\alpha\)使得
\(F\)对\(V_\beta\)绝对
\end{theorem}

在\(\ZF\)中,\(\WF=V\)

\begin{proof}
由于没有选择公理,无法“构造”\(\calh(V_\alpha)\),\(V_\alpha\)的Skolem hull

本质上,我们只需要找到一个\(V_\beta\)使得每个形如\(\exists x\varphi(x,\bary)\)的公式以及每一组参数\(\barb\in V_\beta\)
有\(V\vDash\exists x\varphi_j(x,\barb)\Leftrightarrow V_\beta\vDash\exists x\varphi_j(x,\barb)\),即系数来自\(V_\beta\)的方程若有解,则有一个解\(\in V_\beta\)

设\(\varphi_i\in F\)且形如\(\exists y\varphi_j(\barx,y)\),定义函数\(h_i\)如下:
\begin{itemize}
\item 任取\(\barx\in V\),令\(U=\{y\mid\varphi_j(\barx,y)\}\)
\item 若\(U=\emptyset\),则\(h_i(\barx)=0\)
\item 若\(U\neq\emptyset\),则存在最小的\(\xi\)使得\(U\cap V_\xi\neq\emptyset\),此时令\(h_i(\barx)=V_\xi\)(用了序数的良序性)
\item 函数\(h_i\)满足
\begin{equation*}
\forall\barx(\exists y\varphi_j(\barx,y)\to\exists y\in h_i(\barx))\varphi_j(\barx,y)
\end{equation*}
\end{itemize}

定义\(h_F\)为:
\begin{equation*}
h_F(x_1,\dots,x_m)=\bigcup\{h_i(x_1,\dots,x_m):i=1,\dots,n\}
\end{equation*}
这里必需要求只能有穷多个,因为\(h_i\)是类

则\(h_F\)满足:对每个形如\(\exists y\varphi_j(\barx,y)\)的公式,有
\begin{equation*}
\forall\barx(\exists y\varphi_j(\barx,y)\to\exists y\in h_F(\barx)\varphi_j(\barx,y))
\end{equation*}
任取\(\alpha\),递归定义\(V_\alpha^i\),\(i\in\omega\)如下:
\begin{itemize}
\item \(V_\alpha^0=V_\alpha\)
\item \(V_\alpha^{i+1}=V_\alpha^i\cup\bigcup\{h_F(\bary)\mid\bary\in V_\alpha^i\}\)
\end{itemize}


令\(V_\beta=\bigcup V_\alpha^i\),相当于\(V_\alpha\)的\(F\)-Skolem hull,若\(\varphi_i\in F\)形如\(\exists y\varphi_j(\barx,y)\)

任取\(\barx\in V_\beta\),则存在\(k<\omega\)使得\(\barx\in V_\alpha^k\),若\(\exists y\varphi_j(\barx,y)\),则
\begin{equation*}
\exists y\in h_F(\barx)\varphi_j(\barx,y)
\end{equation*}
\end{proof}

\begin{corollary}[ZF]
令\(F=\{\sigma_1,\dots,\sigma_n\}\)为\(ZF\)的有穷子集,则
\begin{equation*}
\forall\alpha\exists\beta\ge\alpha(\sigma_1^{V_\beta}\wedge\dots\wedge\sigma_n^{V_\beta})
\end{equation*}
\end{corollary}

\begin{proof}
将\(F\)扩张为\(F'\),有穷且对子公式封闭,于是\(\forall\alpha\exists\beta\ge\alpha\)使得\(F'\)相对于\(V_\beta\)绝对

对于\(F'\)中的句子,有
\begin{equation*}
\ZF\vdash\sigma\leftrightarrow\sigma^{V_\beta}
\end{equation*}
\end{proof}

\begin{corollary}[]
设\(F=\{\sigma_1,\dots,\sigma_n\}\subseteq\ZF\),除非\(\ZF\)不一致,否则``\(F\not\vdash\ZF\)''
\end{corollary}

\begin{proof}
存在\(V_\beta\)使得\(\ZF\vdash(F)^{V_\beta}\),若\(F\vdash\ZF\Rightarrow\ZF\vdash(\ZF)^{V_\beta}\),故\(\ZF\vdash(\ZF)^{V_\beta}\to\Con(\ZF)\) (无需
AC,反过来要),因此\(\ZF\vdash\Con(\ZF)\)
\end{proof}

\begin{remark}
以上推论对\(\ZF\)的任意扩张成立

若\(AC\)成立,则反映定理可以改进为
存在可数\((M,\in)\)使得\(M\prec_FV\) (绝对性强于\(\prec_F\))
\begin{itemize}
\item 若\(F\)含有无穷公理,则\(M\neq V_\omega\)
\item 若\(F\)含有幂集公理,若\(M\)传递,则没有绝对性
\begin{itemize}
\item 令\(\psi(x,y)\)表示\(\forall u(u\in y\leftrightarrow u\subseteq x)\),令\(\Pow:\forall x\exists y\psi(x,y)\),则\(\psi\)与\(\Pow\)不能同时绝对

\(M\)传递时,\(\subseteq\leftrightarrow\subseteq^{\eq}\),若\(\psi\)绝对,则\(V\)看到的幂集跟\(M\)看到的幂集,而\(M\)是可数的
\end{itemize}

\item 若\(F\subseteq_f\ZFC\),由Mostowski collapsing定理,存在传递模型使得\((M,\in)\cong(N,\in)\)

\(F\)相对于\(N\)绝对,但是\(F\)的子公式不一定绝对(比如\(\psi\)与\(\Pow\))
\end{itemize}
\end{remark}

\begin{theorem}[ZFC]
对任意有穷公式集\(F\),对任意集合\(N\),存在集合\(M\)使得
\begin{enumerate}
\item \(N\subseteq M\)
\item \(\varphi_1,\dots,\varphi_n\)相对于\((M,\in)\)绝对
\item \(\abs{M}\le\abs{N}+\aleph_0\)
\item 若\(N\)至多可数,则\(M\)可数
\end{enumerate}
\end{theorem}

\begin{proof}
不妨设\(F\)对于子公式封闭,令\(\calh_F\)为\(F\)对应的Skolem函数集,令\(M=\calh_F(N)\) (练习)
\end{proof}

\begin{corollary}[ZFC]
对任意有穷句子集\(F=\{\varphi_1,\dots,\varphi_n\}\),对任意的传递集\(N\),存在\(M\)满足
\begin{enumerate}
\item \(N\subseteq M\)
\item \(F\)相对于\((M,\in)\)是绝对的
\item \(\abs{M}\le\abs{N}+\aleph_0\)
\item \(M\)传递
\end{enumerate}
\end{corollary}

\begin{proof}
不妨设外延公理\(\in F\),则存在\((M',\in)\)满足\(1-3\)

\((M',\in)\)良基似集合且满足外延公理

故\(G:M'\to V\), \(x\mapsto\{G(y)\mid y\in M'\wedge y\in x\}\)是\(M'\)到\(M=G(M')\)的同构,\(M\)传递,由
\(M'\)的绝对性, \(V\vDash\varphi_i\leftrightarrow\varphi_i^{M'}\)

由同构
\begin{equation*}
\varphi_i^{M'}\Leftrightarrow M'\vDash\varphi_i\Leftrightarrow M\vDash\varphi_i\Leftrightarrow\varphi_i^M
\end{equation*}
故\(F\)相对于\(M\)绝对

设\(N\subseteq M'\)传递,对\(N\)中元素的\(\rank\)归纳证明:\(\forall x\in N(G(x)=x)\),即\(G(N)=N\subseteq M\)
\end{proof}

句子集的绝对性被同构保持,而公式不是这样(例子是幂集公理)

\begin{remark}
若\(\varphi(x_1,\dots,x_n)\)是一个公式,且\((M,\in)\cong(M',\in)\)则\(\varphi\)相对于
\end{remark}


\subsection{Exercise}
\label{sec:org45446a0}
\begin{exercise}
\label{ex7.10.3}
\begin{enumerate}
\item \(V_\alpha=\{x\in\WF\mid\rank(x)<\alpha\}\)
\item \(\WF\) is transitive
\item \(\forall x,y\in\WF\), \(x\in y\Rightarrow\rank(x)<\rank(y)\)
\item \(\forall y\in\WF\), \(\rank(y)=\sup\{\rank(x)+1\mid x\in y\}\)
\end{enumerate}
\end{exercise}

\begin{proof}
\begin{enumerate}
\item by definition, \(x\in V_{\rank(x)+1}\setminus V_{\rank(x)}\), \(\rank(x)<\alpha\Rightarrow x\in V_{\rank(x)+1}\subseteq V_\alpha\)

\(\rank(x)\ge\alpha\Rightarrow x\notin V_\alpha\)

\item \(\WF\) is the ``union'' of transitive sets

\item \(y\in V_{\rank(y)+1}\setminus V_{\rank(y)}\), \(y\subseteq V_{\rank(y)}\), \(x\in y\Rightarrow x\in V_{\rank(y)}\Rightarrow\rank(x)<\rank(y)\)

\item by 3, \(\sup\{\rank(x)+1\mid x\in y\}\le\rank(y)\).

induction on \(\rank(y)\le\sup\{\rank(x)+1\mid x\in y\}\)
\begin{itemize}
\item \(\rank(y)=0\)
\item \(\rank(y)=\beta+1\), \(y\in V_{\beta+2}\setminus V_{\beta+1}\)

\(y\in V_{\beta+2}\Rightarrow y\subseteq V_{\beta+1}\). \(y\notin V_{\beta+1}\Rightarrow y\not\subseteq V_{\beta}\Rightarrow y\setminus V_\beta\) nonempty.
Let \(x\in y\setminus V_\beta\), \(\rank(x)\ge\beta\), \(\sup\{\rank(x)+1\mid x\in y\}\ge\beta+1=\rank(y)\)
\item \(\rank(y)=\gamma\) for some limit, then \(y\subseteq V_\gamma\) and for any \(\xi<\gamma\), \(y\not\subseteq V_\xi\),
let \(X_\xi\in y\setminus V_\xi\), then \(\rank(X_\xi)\ge\xi\), \(\sup\{\rank(x)+1\mid x\in y\}\ge\sup\{\xi+1\mid\xi<\rank(y)\}\ge\rank(y)\)
\end{itemize}
\end{enumerate}
\end{proof}

\begin{exercise}
\(R\)是似集合的,则
\(R\)是外延的当且仅当对任意\(x,y\in X\)
\begin{equation*}
x\neq y\to\pred(X,x,R)\neq\pred(X,y,R)
\end{equation*}
\end{exercise}

\begin{exercise}[7.10.7]
证明莫斯托夫斯基定理中的\(\bM\)和\(\bG\)唯一
\end{exercise}

\begin{proof}
假设\(M,N\)是传递类且\(f:(M,\in)\cong(N,\in)\),\(S=\{x\in M\mid f(x)\neq x\}\)。
因为\(M\neq N\),因此\(S\)非空,取\(S\)的极小元\(x_0\),则对于任意\(y\in x_0\),\(y=f(y)\in f(x_0)\),
于是\(x_0\subset f(x_0)\),又因为\(f\)是双射,同理有\(f(x_0)\subset x_0\),于是\(f(x_0)=x_0\),矛盾。因此\(M=N\)。

若\(f_1:(X,R)\cong(M,\in)\),\(f_2:(X,R)\cong(N,\in)\),则 \(M=N\),于是\(f_1f_2=f_2f_1=\id\),因此\(f_1=f_2\)
\end{proof}

\begin{exercise}[7.10.8]
证明以下概念对任意\(\ZF-\Pow\)的传递模型绝对
\begin{enumerate}
\item \(X^{<\omega}\)
\end{enumerate}
\end{exercise}

\begin{proof}
\begin{enumerate}
\item \(f\in X^{<\omega}\)当且仅当存在有穷序数\(n\)使得\(f\in X^n\)
\end{enumerate}

而任意这样的模型都有有穷序数
\end{proof}

\begin{exercise}[7.10.9]
\(V_\omega\vDash\ZF-\Inf+\neg\Inf\)
\end{exercise}

\begin{proof}

\end{proof}
\end{document}
