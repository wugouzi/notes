% Created 2022-03-10 Thu 13:57
% Intended LaTeX compiler: pdflatex
\documentclass[11pt]{article}
\usepackage[utf8]{inputenc}
\usepackage[T1]{fontenc}
\usepackage{graphicx}
\usepackage{longtable}
\usepackage{wrapfig}
\usepackage{rotating}
\usepackage[normalem]{ulem}
\usepackage{amsmath}
\usepackage{amssymb}
\usepackage{capt-of}
\usepackage{hyperref}
\graphicspath{{../../books/}}
% TIPS
% \substack{a\\b} for multiple lines text





% pdfplots will load xolor automatically without option
\usepackage[dvipsnames]{xcolor}

\usepackage{forest}
% two-line text in node by [two \\ lines]
% \begin{forest} qtree, [..] \end{forest}
\forestset{
  qtree/.style={
    baseline,
    for tree={
      parent anchor=south,
      child anchor=north,
      align=center,
      inner sep=1pt,
    }}}
%\usepackage{flexisym}
% load order of mathtools and mathabx, otherwise conflict overbrace

\usepackage{mathtools}
%\usepackage{fourier}
\usepackage{pgfplots}
\usepackage{amsthm, mathabx,  amsmath, commath}
\usepackage{amsfonts}

\usepackage{empheq}
\usepackage{tikz}
\usetikzlibrary{arrows.meta}
\usepackage[most]{tcolorbox}

\newtheorem{theorem}{Theorem}[section]
\newtheorem{definition}{Definition}[section]
\newtheorem{corollary}{Corollary}[section]
\newtheorem{example}{Example}[section]
\newtheorem{lemma}{Lemma}[section]
\newtheorem{proposition}{Proposition}[section]

\newcommand{\bl}[1] {\boldsymbol{#1}}
\newcommand{\Wt}[1] {\stackrel{\sim}{\smash{#1}\rule{0pt}{1.1ex}}}
\newcommand{\wt}[1] {\widetilde{#1}}


%For boxed texts in align, use Aboxed{}
%otherwise use boxed{}

\DeclareMathSymbol{\widehatsym}{\mathord}{largesymbols}{"62}
\newcommand\lowerwidehatsym{%
  \text{\smash{\raisebox{-1.3ex}{%
    $\widehatsym$}}}}
\newcommand\fixwidehat[1]{%
  \mathchoice
    {\accentset{\displaystyle\lowerwidehatsym}{#1}}
    {\accentset{\textstyle\lowerwidehatsym}{#1}}
    {\accentset{\scriptstyle\lowerwidehatsym}{#1}}
    {\accentset{\scriptscriptstyle\lowerwidehatsym}{#1}}
}

\usepackage{graphicx}
    
% text on arrow for xRightarrow
\makeatletter
%\newcommand{\xRightarrow}[2][]{\ext@arrow 0359\Rightarrowfill@{#1}{#2}}
\makeatother


\def \bx {\boldsymbol{x}}
\def \ba {\boldsymbol{a}}
\def \bI {\boldsymbol{I}}
\def \bt {\boldsymbol{t}}
\def \bb {\boldsymbol{b}}
\def \bA {\boldsymbol{A}}
\def \bX {\boldsymbol{X}}
\def \bu {\boldsymbol{u}}
\def \bS {\boldsymbol{S}}
\def \bZ {\boldsymbol{Z}}
\def \bz {\boldsymbol{z}}
\def \by {\boldsymbol{y}}
\def \bw {\boldsymbol{w}}
\def \bT {\boldsymbol{T}}
\def \bS {\boldsymbol{S}}
\def \bm {\boldsymbol{m}}
\def \bW {\boldsymbol{W}}
\def \bY {\boldsymbol{Y}}
\def \bH {\boldsymbol{H}}
\def \blambda {\boldsymbol{\lambda}}
\def \bPhi {\boldsymbol{\Phi}}
\def \btheta {\boldsymbol{\theta}}
\def \bmu {\boldsymbol{\mu}}
\def \bphi {\boldsymbol{\phi}}
\def \bSigma {\boldsymbol{\Sigma}}
\def \lb {\left\{}
\def \rb {\right\}}
\def \caln {\mathcal{N}}
\def \dissum {\displaystyle\Sigma}
\def \dispro {\displaystyle\prod}
\def \E {\mathbb{E}}
\def \Q {\mathbb{Q}}
\def \V {\mathbb{V}}
\def \R {\mathbb{R}}
\def \calq {\mathcal{Q}}
\def \calg {\mathcal{G}}
\def \caln {\mathcal{N}}
\def \calr {\mathcal{R}}
\def \calm {\mathcal{M}}
\def \calc {\mathcal{C}}
\def \bcup {\bigcup}

\makeindex
\usepackage[UTF8]{ctex}
\def \FORM {\text{FORM}}
\def \PROOF {\text{PROOF}}
\author{Yao}
\date{\today}
\title{Set Theory2}
\hypersetup{
 pdfauthor={Yao},
 pdftitle={Set Theory2},
 pdfkeywords={},
 pdfsubject={},
 pdfcreator={Emacs 28.0.90 (Org mode 9.6)}, 
 pdflang={English}}
\begin{document}

\maketitle
\tableofcontents

\section{集合的宇宙}
\label{sec:org7573365}
\subsection{数理逻辑}
\label{sec:org70c3cea}
在\(\ZFC\)下证明\(\ZFC\vdash\CH\),希望将``\(\ZFC\vdash\CH\)''表述为一阶句子

一般而言,给定一个\(\call\)-理论\(T\)和一个\(\call\)-句子\(\delta\),``\(T\vdash\sigma\)''不能用一个\(\call\)-句子表示,只能
用元语言表述

我们需要在\(\ZFC\)中编码“元语言”

在\(\ZFC\)中可以定义\(\caln=(\N,+,\times,0,1)\)

即存在集合论语言\(\call=\{\in\}\)中的 \textbf{公式} ,在\(\ZFC\)的任意模型中可以定义 \(\N,+,\times,0,1\),以上公式与模
型无关

用\(\ucorner{0}\),\(\ucorner{1}\),\(\ucorner{2}\)\ldots{} 表示\(\ZFC\)中的“自然数”,以区别元语言中
的自然数

\begin{theorem}[]
如果\(R\subseteq\N^n\)是一个递归关系。\(T\subseteq\Th(\caln)\)是包含数论的适当丰富的理论,则存在公式\(\varphi(x_1,\dots,x_n)\)使
得对任意自然数\(m_1,\dots,m_n\)有
\begin{align*}
&\text{如果}(m_1,\dots,m_n)\in R\text{则}T\vdash\varphi(\ucorner{m_1},\dots,\ucorner{m_n})\\
&\text{如果}(m_1,\dots,m_n)\notin R\text{则}T\vdash\neg\varphi(\ucorner{m_1},\dots,\ucorner{m_n})
\end{align*}
\end{theorem}

\begin{remark}
\begin{enumerate}
\item \(T\subseteq\Th(\caln)\subseteq\ZFC\)
\item \(\varphi\)是语言\(\{+,\times,0,1\}\)上的公式
\item \(\varphi\)可以还原为一个\(\{\in\}\)上的公式
\item \(\varphi(\ucorner{m_1},\dots,\ucorner{m_n})\)是一个闭语句
\end{enumerate}
\end{remark}

\textbf{编码}

编码函数\(f:X\to\N\)

存在解码函数\(g,h\),对\(a=a_0,\dots,a_n\in X\), \(h(f(a))=n+1\), \(g(f(a),k)=a_k\) (分量)

性质:以上三种函数\(f,g,h\)均是递归函数\(\Rightarrow\)都是可表示的

性质:“公式集”的编码集是递归的

性质:如果\(T\subseteq\ZFC\)是可公理化的,则\(T\)的证明集的编码集是递归的

\begin{corollary}[]
存在一个公式 \(\psi\) 和\(\theta\)使得
\begin{align*}
\ZFC\vdash\psi(n)&\Leftrightarrow n\text{ is a formula}\\
\ZFC\vdash\neg\psi(n)&\Leftrightarrow n\text{ is not a formula}\\
\ZFC\vdash\theta(n)&\Leftrightarrow n\text{ is a proof in }\ZFC\\
\ZFC\vdash\neg\theta(n)&\Leftrightarrow n\text{ is not a proof in }\ZFC\\
\end{align*}
称\(\psi\)定义了公式集,\(\theta\)定义了证明集
\end{corollary}

\(\FORM=\{\ucorner{\varphi}\mid\varphi\text{ formula}\}\subseteq\N\)

    如果\(T\subseteq\ZFC\)是可公理化的,则“\(T\)是一致的”是一个一阶表述式
    “不存在一个有穷的证明序列\(D=(\varphi_1,\dots,\varphi_n)\)使得\(\varphi_n\)形如\(\varphi\wedge\neg\varphi\)
,记作\(\Con(T)\)

\begin{theorem}[第二不完全]
如果\(T\)是包含\(\ZFC\)的一个递归公理集,且\(T\)一致,则
\begin{equation*}
T\not\vdash\Con(T)
\end{equation*}
特别地,\(\ZFC\not\vdash\Con(\ZFC)\)
\end{theorem}

\begin{theorem}[]
对任意可公理化的理论\(T\),\(\ZFC\vdash\Con(T)\)当且仅当存在\(M\vDash T\)
\end{theorem}

即不能在\(\ZFC\)里证明\(\ZFC\)有一个模型

需要可公理化来写出\(\Con(T)\),因此因为\(\ZFC\not\vDash\Con(T)\),我们只能假设这么一个模型

集合论的模型跟集合论没什么关系,就是一个集合带一个二元关系,是关于集合论语言的结构

\begin{definition}[]
设\((M,E)\)是集合论模型
\begin{enumerate}
\item 对任意公式\(\varphi(\barx,y)\),定义\(M^n\)上的函数
\begin{equation*}
h_\varphi:M^n\to M
\end{equation*}
满足条件
\begin{equation*}
M\vDash\exists y\varphi(\bara,y)\Rightarrow M\vDash\varphi(\bara,h_\varphi(\bara))
\end{equation*}
称\(h_\varphi\)为\(\varphi\)的Skolem函数(依赖于选择公理,不同的变量选择有不同的函数)
\item 令\(\calh=\{h_\varphi\mid\varphi\text{ formula}\}\)为Skolem函数集合,设\(S\)是\(M\)的任意子集,则\(\calh(S)\)表示包
含\(S\)且对\(\calh\)封闭的最小集合,称之为\(S\)的Skolem壳
\end{enumerate}
\end{definition}

\begin{lemma}[]
令\(N\)是集合论模型,\(S\subseteq N\),如果\(M=\calh(S)\),则\(M\prec N\)
\end{lemma}

\begin{proof}
Induction

对任意\(\bara\in M^n\),有\(M\vDash\varphi(\bara)\Leftrightarrow N\vDash\varphi(\bara)\)
\begin{enumerate}
\item 不含量词,显然成立
\item \(\varphi\)形如\(\exists y\psi(\barx,y)\), \(N\vDash\exists y\psi(\bara,y)\Rightarrow N\vDash\psi(\bara,h_\psi(\bara))\),by
IH, \(M\vDash\psi(\bara,h_\psi(\bara))\Rightarrow M\vDash\exists y\psi(\bara,y)\)
\end{enumerate}
\end{proof}

\begin{theorem}[Löwenheim–Skolem Theorem]

\end{theorem}
\subsection{层垒的谱系}
\label{sec:org72923f5}
工作于\(\ZF^-\):\(\ZF-\)基础公理

\(\alpha\mapsto V_\alpha\)是\(\On\)到\(\WF\)的1-1映射,而\(\On\)是真类

\begin{lemma}[]
For any ordinal \(\alpha\)
\begin{enumerate}
\item \(V_\alpha\) is transitive
\item \(\xi\le\alpha\Rightarrow V_\xi\subseteq V_\alpha\)
\item if \(\kappa\) is inaccessible, then \(\abs{V_\kappa}=\kappa\)
\end{enumerate}
\end{lemma}

\begin{definition}[]
For any \(x\in\WF\), \textbf{rank} of \(x\) is
\begin{equation*}
\rank(x)=\min\{\beta\mid x\in V_{\beta+1}\}
\end{equation*}
\end{definition}

\(\rank(x)=\alpha\Rightarrow x\in V_{\alpha+1}\wedge x\notin V_\alpha\)

\begin{itemize}
\item \(x\in V_{\alpha+1}\Leftrightarrow\rank(x)\le\alpha\)
\item \(x\subseteq V_\alpha\Leftrightarrow\rank(x)\le\alpha\)
\end{itemize}

\begin{lemma}[]
\begin{enumerate}
\item \(V_\alpha=\{x\in\WF\mid\rank(x)<\alpha\}\)
\item \(\WF\) is transitive
\item \(\forall x,y\in\WF\), \(x\in y\Rightarrow\rank(x)<\rank(y)\)
\item \(\forall y\in\WF\), \(\rank(y)=\sup\{\rank(x)+1\mid x\in y\}\)
\end{enumerate}
\end{lemma}

\begin{proof}
\begin{enumerate}
\item by definition, \(x\in V_{\rank(x)+1}\setminus V_{\rank(x)}\), \(\rank(x)<\alpha\Rightarrow x\in V_{\rank(x)+1}\subseteq V_\alpha\)

\(\rank(x)\ge\alpha\Rightarrow x\notin V_\alpha\)

\item \(\WF\) is the ``union'' of transitive sets

\item \(y\in V_{\rank(y)+1}\setminus V_{\rank(y)}\), \(y\subseteq V_{\rank(y)}\), \(x\in y\Rightarrow x\in V_{\rank(y)}\Rightarrow\rank(x)<\rank(y)\)

\item by 3, \(\sup\{\rank(x)+1\mid x\in y\}\le\rank(y)\).

induction on \(\rank(y)\le\sup\{\rank(x)+1\mid x\in y\}\)
\begin{itemize}
\item \(\rank(y)=0\)
\item \(\rank(y)=\beta+1\), \(y\in V_{\beta+2}\setminus V_{\beta+1}\)

\(y\in V_{\beta+2}\Rightarrow y\subseteq V_{\beta+1}\). \(y\notin V_{\beta+1}\Rightarrow y\not\subseteq V_{\beta}\Rightarrow y\setminus V_\beta\) nonempty.
Let \(x\in y\setminus V_\beta\), \(\rank(x)\ge\beta\), \(\sup\{\rank(x)+1\mid x\in y\}\ge\beta+1=\rank(y)\)
\item \(\rank(y)=\gamma\) for some limit, then \(y\subseteq V_\gamma\) and for any \(\xi<\gamma\), \(y\not\subseteq V_\xi\),
let \(X_\xi\in y\setminus V_\xi\), then \(\rank(X_\xi)\ge\xi\), \(\sup\{\rank(x)+1\mid x\in y\}\ge\sup\{\xi+1\mid\xi<\rank(y)\}\ge\rank(y)\)
\end{itemize}
\end{enumerate}
\end{proof}

\begin{itemize}
\item \(\WF\)中的集合按照秩分层
\item 在\(\WF\)中基础公理是成立的:\(\forall y(y\neq\emptyset\to\exists x\in y(x\cap y=\emptyset))\),因为任何序数集都有最小元,挑一个有最
小rank的就好了
\item \(\WF\)类的构造没有用到选择公理
\item \(\On\subseteq\WF\)
\end{itemize}


\begin{lemma}[]
for any ordinal \(\alpha\)
\begin{enumerate}
\item \(\alpha\in\WF\) and \(\rank(\alpha)=\alpha\)
\item \(V_\alpha\cap\On=\alpha\)
\end{enumerate}
\end{lemma}

\begin{proof}
\begin{enumerate}
\item \(0\in V_1\setminus V_0\subset\WF\), \(\rank(0)=0\)

If \(\alpha\in\WF\)
and
\(\rank(\alpha)=\alpha\).
\(\alpha\in V_{\alpha+1}\setminus V_\alpha\), \(\alpha\subseteq V_{\alpha+1}\). \(\alpha+1=\alpha\cup\{\alpha\}\subseteq V_{\alpha+1}\), \(\alpha+1\in V_{\alpha+2}\subset\WF\).
If \(\alpha+1\in V_{\alpha+1}\), then \(\rank(\alpha+1)\le\alpha\), but \(\alpha\in\alpha+1\Rightarrow\rank(\alpha)=\alpha<\rank(\alpha+1)\). A
contradiction

suppose \(\gamma\) is a limit ordinal and for any \(\alpha<\gamma\), \(\alpha\in V_{\alpha+1}\setminus V_\alpha\).
\(\gamma=\bigcup_{\alpha<\gamma}\alpha\subseteq\bigcup_{\alpha<\gamma}V_\alpha=V_\gamma\). Thus \(\gamma\in V_{\gamma+1}\), \(\rank(\gamma)\le\gamma\) and \(\rank(\gamma)\not<\gamma\).
\item suppose \(\beta\in V_\alpha\cap\On\), then \(\beta=\rank(\beta)<\alpha\). If \(\beta\in\alpha\) and \(\rank(\beta)<\alpha\), \(\beta\in V_\alpha\cap\On\)
\end{enumerate}
\end{proof}

\begin{lemma}[]
\begin{enumerate}
\item If \(x\in\WF\), then \(\bigcup x,\calp(x),\{x\}\in\WF\), and their rank \(<\rank(x)+\omega\)
\item If \(x,y\in\WF\), then \(x\times y,x\cup y,x\cap y,\{x,y\},(x,y),x^y\in\WF\), and their
rank \(<\rank(x)+\rank(y)+\omega\)
\item \(\Z,\Q,\R\in V_{\omega+\omega}\)
\item for any set \(x\), \(x\in\WF\Leftrightarrow x\subset\WF\)
\end{enumerate}
\end{lemma}

\begin{proof}
\begin{enumerate}
\item suppose \(\rank(x)=\alpha\). \(x\in V_{\alpha+1}\setminus V_\alpha\) and \(x\subseteq V_\alpha\).

by transitivity, \(\bigcup x\subseteq V_\alpha\Rightarrow \bigcup x\in V_{\alpha+1}\subset\WF\). \(\rank(\bigcup x)\le\alpha\)

suppose
\(y\in\calp(x)\),
\(y\subseteq x\Rightarrow y\subseteq V_\alpha\Rightarrow y\in V_{\alpha+1}\). \(\calp(x)\subseteq V_{\alpha+1}\), \(\calp(x)\in V_{\alpha+2}\), \(\rank(\calp(x))\le\alpha+1\).

\(\{x\}\in\calp(x)\in V_{\alpha+2}\).

\item Suppose \(\rank(x)=\alpha,\rank(y)=\beta\), \(x\subset V_\alpha\), \(y\subset V_\beta\)

\(x\cup y\subset V_\alpha\cup V_\beta=V_{\max(\alpha,\beta)}\), \(\rank(x\cup y)\le\max(\alpha,\beta)\)

\(x\cap y\subset V_{\min(\alpha,\beta)}\)

\(\{x,y\}\subseteq V_{\alpha+1}\cup V_{\beta+1}=V_{\max(\alpha,\beta)+1}\), \(\rank(\{x,y\})=\max(\alpha,\beta)+1\)

\((x,y)=\{\{x\},\{x,y\}\}\subset V_{\max(\alpha,\beta)+2}\). \(\rank((x,y))=\max(\alpha,\beta)+2\)

\(x\times y=\{(a,b)\mid a\in x,b\in y\}\).
\(a\in x\Rightarrow\rank(a)<\alpha\), \(b\in y\Rightarrow\rank(b)<\beta\), \(\rank(a,b)<\max(\alpha,\beta)+2\),
\((a,b)\in V_{\max(\alpha,\beta)+2}\). \(x\times y\subseteq V_{\max(\alpha,\beta)+2}\), \(\rank(x\times y)\le\max(\alpha,\beta)+2\).

\(x^y\subseteq\calp(x\times y)\subseteq V_{\max(\alpha,\beta)+3}\).

\item \(\N=\omega\in V_{\omega+1}\)

\(\Z\): let \(\sim\) be an equivalence relation on \(\omega\times\omega\), \((a,b)\sim(c,d)\Leftrightarrow a+d=b+c\),
then \(\Z=(\omega\times\omega)/\sim\). Hence \(\Z\) is a partition of \(\omega\times\omega\) and
hence \(\Z\subseteq\calp(\omega\times\omega)\). \(\Z\in V_{\omega+3}\)

\(\Q\): let \(\sim\) be an equivalence
on \(\Z\times\Z^+\), \((a,b)\sim(c,d)\Leftrightarrow ad=bc\). \(\Q\subseteq\calp(\Z\times\Z^+)\), \(\Q\in V_{\omega+6}\)

\(\R\): set of dedekind cut on \(\Q\), \(\R\subset\calp(\Q)\), \(\R\in V_{\omega+8}\)

\item \(\Rightarrow\): \(\WF\) is transitive

\(\Leftarrow\): \(x\) is a set and \(x\subset\bigcup_{\alpha\in\On}V_\alpha\).

\textbf{Claim}: there is an ordinal \(\alpha\) s.t. \(x\subset V_\alpha\)

Otherwise, let \(f:\On\to\calp(x)\) s.t. \(f(\alpha)=x\setminus V_\alpha\). Then for any \(y\in\calp(x)\), \(f^{-1}(y)\) is
a set. \(\On=\bigcup_{y\in x}f^{-1}(y)\) and is thus a set, a contradiction
\end{enumerate}
\end{proof}

AC => Any set has cardinality
\begin{lemma}[]
Assume AC (\(V\vDash\ZFC\))
\begin{enumerate}
\item for any group \(G\), there is a group \(G'\) in \(\WF\) s.t. \(G\cong G'\)
\item for any topological space \(T\), there is a topological space \(T'\) in \(\WF\)
s.t. \(T\cong T'\) (homeomorphic)
\end{enumerate}
\end{lemma}

\begin{proof}
\begin{enumerate}
\item suppose \((G,*_G)\) is a group, \(G,*_G\in V\). By AC, there is a cardinal \(\alpha\)
s.t. \(\abs{G}=\alpha\), that is, there is a bijection \(f:\alpha\to G\). Define \(*\): for
any \(x,y,z\in\alpha\), \(x*y=z\Leftrightarrow f(x)*_Gf(y)=f(z)\). Then \((\alpha,​*)\cong(G,​*_G)\), \(​*\subseteq\alpha\times\alpha\)
\end{enumerate}
\end{proof}

\(V\)中的任何结构都可以在\(\WF\)中找到同构象(同构是在\(V\)里看到的)

\begin{definition}[]
任意集合\(A\)上的二元关系<是 \textbf{良基} 的,当且仅当对\(A\)的任意非空子集\(X\),\(X\)有<下的极小元
\end{definition}

\begin{theorem}[]
If \(A\in\WF\), then \(\in\) is a well-founded relation on \(A\)
\end{theorem}

\begin{proof}
suppose \(X\subseteq A\), \(X\neq\emptyset\), \(X\subseteq\WF\), then elements of \(X\) has ranks
and \(x\in y\Rightarrow\rank(x)<\rank(y)\). Let \(x\) having least rank in \(X\), then \(x\) is
the \(\in\)-minimal element in \(X\)
\end{proof}

\begin{lemma}[]
If \(A\) is a transitive set and \(\in\) is a well-founded relation on \(A\), then \(A\in\WF\)
\end{lemma}

\begin{proof}
Just need to prove \(A\subset\WF\). If \(A\not\subset\WF\), \(X=A\setminus\WF\neq\emptyset\). Then \(X\) has a \(\in\)-minimal
element \(x\). Then \(x\neq\emptyset\in\WF\). For any \(y\in x\), \(y\in A\). By the minimality
of \(x\), \(y\in\WF\). Then \(x\subset\WF\), \(x\in\WF\), a contradiction
\end{proof}

\begin{lemma}[]
For any set \(x\), there is a minimal transitive set \(\trcl(x)\) s.t. \(x\subseteq\trcl(x)\)
\end{lemma}

\begin{proof}
For any \(n\in\omega\) define \(x_n\)
\begin{align*}
x_0&=x\\
x_{n+1}&=\bigcup x_n
\end{align*}
let \(\trcl(x)=\bigcup_{n\in\omega}x_n\).
\begin{enumerate}
\item \(\trcl(x)\)is transitive

\(a\in\trcl(x)\Rightarrow a\in x_n\Rightarrow a\subseteq x_{n+1}\subseteq\trcl(x)\)
\item \(\trcl(x)\) is minimal

If \(y\supseteq x\) is transitive, recursively prove for any \(n<\omega\), \(x_n\subseteq y\).
\end{enumerate}
\end{proof}

\(\trcl(x)\) is the \textbf{transitive closure} of \(x\).

\begin{lemma}[]
We can prove the following without axiom of power set
\begin{enumerate}
\item if \(x\) is transitive, \(\trcl(x)=x\)
\item \(y\in x\Rightarrow\trcl(y)\subseteq\trcl(x)\)
\item \(\trcl(x)=x\cup\bigcup\{\trcl(y)\mid y\in x\}\)
\end{enumerate}
\end{lemma}

\begin{proof}
\begin{enumerate}
\setcounter{enumi}{1}
\item \(y\in x\subset\trcl(x)\). \(y\in\trcl(x)\). \(\trcl(y)\subseteq\trcl(x)\).
\item \(x\cup\bigcup\{\trcl(y)\mid y\in x\}\subseteq\trcl(x)\) by (2)

\(\bigcup\{\trcl(y)\mid y\in x\}\) is transitive. For \(y\in x\), \(y\subseteq\trcl(y)\). Thus rhs is transitive
\end{enumerate}
\end{proof}


\begin{theorem}[In \(\ZFm\)]
For any set \(X\), TFAE
\begin{enumerate}
\item \(X\in\WF\)
\item \(\trcl(X)\in\WF\)
\item \(\in\) is a well-founded relation on \(\trcl(X)\)
\end{enumerate}
\end{theorem}

\begin{proof}
\(1\to 2\): \(\WF\) is closed under union
\end{proof}

\begin{theorem}[]
If \(V\vDash\ZFm\), TFAE
\begin{enumerate}
\item axiom of foundation (\(V\vDash\)) axiom of foundation
\item for any set \(X\), \(\in\) is a well-founded relation on \(X\)
\item \(V=\WF\)
\end{enumerate}
\end{theorem}

\(V\vDash\ZF\Rightarrow V=\WF(\WF\vDash\ZF)\)

Goal:  \(V\vDash\ZFm\Rightarrow\WF\vDash\ZFm\)
但是\(\WF\)是一个类,我们并没有定义

我们可以用相对化编码\(\WF\vDash\ZFm\)
\subsection{相对化 relativization}
\label{sec:orgfe5d271}
工作在\(\ZFm\)
\begin{definition}[]
\(\bM\) class, \(\varphi\) formula, \(\varphi\)对\(\bM\)的 \textbf{相对化} \(\varphi^{\bM}\)
\begin{enumerate}
\item \((x=y)^{\bM}:=x=y\)
\item \((x\in y)^{\bM}:=x\in y\)
\item \((\varphi\to\psi)^{\bM}:=\varphi^{\bM}\to\psi^{\bM}\)
\item \((\neg\varphi)^{\bM}:=\neg\varphi^{\bM}\)
\item \((\forall x\varphi)^{\bM}:=(\forall x\in\bM)\varphi^{\bM}\)
\end{enumerate}
\end{definition}

\subsection{Exercise}
\label{sec:org7bdbf96}
\begin{exercise}
\label{ex7.10.3}
\begin{enumerate}
\item \(V_\alpha=\{x\in\WF\mid\rank(x)<\alpha\}\)
\item \(\WF\) is transitive
\item \(\forall x,y\in\WF\), \(x\in y\Rightarrow\rank(x)<\rank(y)\)
\item \(\forall y\in\WF\), \(\rank(y)=\sup\{\rank(x)+1\mid x\in y\}\)
\end{enumerate}
\end{exercise}

\begin{proof}
\begin{enumerate}
\item by definition, \(x\in V_{\rank(x)+1}\setminus V_{\rank(x)}\), \(\rank(x)<\alpha\Rightarrow x\in V_{\rank(x)+1}\subseteq V_\alpha\)

\(\rank(x)\ge\alpha\Rightarrow x\notin V_\alpha\)

\item \(\WF\) is the ``union'' of transitive sets

\item \(y\in V_{\rank(y)+1}\setminus V_{\rank(y)}\), \(y\subseteq V_{\rank(y)}\), \(x\in y\Rightarrow x\in V_{\rank(y)}\Rightarrow\rank(x)<\rank(y)\)

\item by 3, \(\sup\{\rank(x)+1\mid x\in y\}\le\rank(y)\).

induction on \(\rank(y)\le\sup\{\rank(x)+1\mid x\in y\}\)
\begin{itemize}
\item \(\rank(y)=0\)
\item \(\rank(y)=\beta+1\), \(y\in V_{\beta+2}\setminus V_{\beta+1}\)

\(y\in V_{\beta+2}\Rightarrow y\subseteq V_{\beta+1}\). \(y\notin V_{\beta+1}\Rightarrow y\not\subseteq V_{\beta}\Rightarrow y\setminus V_\beta\) nonempty.
Let \(x\in y\setminus V_\beta\), \(\rank(x)\ge\beta\), \(\sup\{\rank(x)+1\mid x\in y\}\ge\beta+1=\rank(y)\)
\item \(\rank(y)=\gamma\) for some limit, then \(y\subseteq V_\gamma\) and for any \(\xi<\gamma\), \(y\not\subseteq V_\xi\),
let \(X_\xi\in y\setminus V_\xi\), then \(\rank(X_\xi)\ge\xi\), \(\sup\{\rank(x)+1\mid x\in y\}\ge\sup\{\xi+1\mid\xi<\rank(y)\}\ge\rank(y)\)
\end{itemize}
\end{enumerate}
\end{proof}
\end{document}
