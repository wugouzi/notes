% Created 2021-06-25 Fri 02:46
% Intended LaTeX compiler: pdflatex
\documentclass[11pt]{article}
\usepackage[utf8]{inputenc}
\usepackage[T1]{fontenc}
\usepackage{graphicx}
\usepackage{grffile}
\usepackage{longtable}
\usepackage{wrapfig}
\usepackage{rotating}
\usepackage[normalem]{ulem}
\usepackage{amsmath}
\usepackage{textcomp}
\usepackage{amssymb}
\usepackage{capt-of}
\usepackage{hyperref}
% TIPS
% \substack{a\\b} for multiple lines text





% pdfplots will load xolor automatically without option
\usepackage[dvipsnames]{xcolor}

\usepackage{forest}
% two-line text in node by [two \\ lines]
% \begin{forest} qtree, [..] \end{forest}
\forestset{
  qtree/.style={
    baseline,
    for tree={
      parent anchor=south,
      child anchor=north,
      align=center,
      inner sep=1pt,
    }}}
%\usepackage{flexisym}
% load order of mathtools and mathabx, otherwise conflict overbrace

\usepackage{mathtools}
%\usepackage{fourier}
\usepackage{pgfplots}
\usepackage{amsthm, mathabx,  amsmath, commath}
\usepackage{amsfonts}

\usepackage{empheq}
\usepackage{tikz}
\usetikzlibrary{arrows.meta}
\usepackage[most]{tcolorbox}

\newtheorem{theorem}{Theorem}[section]
\newtheorem{definition}{Definition}[section]
\newtheorem{corollary}{Corollary}[section]
\newtheorem{example}{Example}[section]
\newtheorem{lemma}{Lemma}[section]
\newtheorem{proposition}{Proposition}[section]

\newcommand{\bl}[1] {\boldsymbol{#1}}
\newcommand{\Wt}[1] {\stackrel{\sim}{\smash{#1}\rule{0pt}{1.1ex}}}
\newcommand{\wt}[1] {\widetilde{#1}}


%For boxed texts in align, use Aboxed{}
%otherwise use boxed{}

\DeclareMathSymbol{\widehatsym}{\mathord}{largesymbols}{"62}
\newcommand\lowerwidehatsym{%
  \text{\smash{\raisebox{-1.3ex}{%
    $\widehatsym$}}}}
\newcommand\fixwidehat[1]{%
  \mathchoice
    {\accentset{\displaystyle\lowerwidehatsym}{#1}}
    {\accentset{\textstyle\lowerwidehatsym}{#1}}
    {\accentset{\scriptstyle\lowerwidehatsym}{#1}}
    {\accentset{\scriptscriptstyle\lowerwidehatsym}{#1}}
}

\usepackage{graphicx}
    
% text on arrow for xRightarrow
\makeatletter
%\newcommand{\xRightarrow}[2][]{\ext@arrow 0359\Rightarrowfill@{#1}{#2}}
\makeatother


\def \bx {\boldsymbol{x}}
\def \ba {\boldsymbol{a}}
\def \bI {\boldsymbol{I}}
\def \bt {\boldsymbol{t}}
\def \bb {\boldsymbol{b}}
\def \bA {\boldsymbol{A}}
\def \bX {\boldsymbol{X}}
\def \bu {\boldsymbol{u}}
\def \bS {\boldsymbol{S}}
\def \bZ {\boldsymbol{Z}}
\def \bz {\boldsymbol{z}}
\def \by {\boldsymbol{y}}
\def \bw {\boldsymbol{w}}
\def \bT {\boldsymbol{T}}
\def \bS {\boldsymbol{S}}
\def \bm {\boldsymbol{m}}
\def \bW {\boldsymbol{W}}
\def \bY {\boldsymbol{Y}}
\def \bH {\boldsymbol{H}}
\def \blambda {\boldsymbol{\lambda}}
\def \bPhi {\boldsymbol{\Phi}}
\def \btheta {\boldsymbol{\theta}}
\def \bmu {\boldsymbol{\mu}}
\def \bphi {\boldsymbol{\phi}}
\def \bSigma {\boldsymbol{\Sigma}}
\def \lb {\left\{}
\def \rb {\right\}}
\def \caln {\mathcal{N}}
\def \dissum {\displaystyle\Sigma}
\def \dispro {\displaystyle\prod}
\def \E {\mathbb{E}}
\def \Q {\mathbb{Q}}
\def \V {\mathbb{V}}
\def \R {\mathbb{R}}
\def \calq {\mathcal{Q}}
\def \calg {\mathcal{G}}
\def \caln {\mathcal{N}}
\def \calr {\mathcal{R}}
\def \calm {\mathcal{M}}
\def \calc {\mathcal{C}}
\def \bcup {\bigcup}

\author{Munkres}
\date{\today}
\title{Topology}
\hypersetup{
 pdfauthor={Munkres},
 pdftitle={Topology},
 pdfkeywords={},
 pdfsubject={},
 pdfcreator={Emacs 27.2 (Org mode 9.5)}, 
 pdflang={English}}
\begin{document}

\maketitle
\tableofcontents


\section{Topological Spaces and Continuous Functions}
\label{sec:org33f3cfd}

\subsection{Topological Spaces}
\label{sec:org0daf9dc}
\begin{definition}[]
A \textbf{topology} on a set is a collection \(\calt\) of subsets of \(X\) having the following properties
\begin{enumerate}
\item \(\emptyset\) and \(X\) are in \(\calt\)
\item The union of the elements of any subcollection of \(\calt\) is in \(T\)
\item The intersection of the elements of any finite subcollection of \(\calt\) is in \(\calt\)
\end{enumerate}


A set \(X\) for which a topology \(\calt\) has been specified is called a \textbf{topological space}
\end{definition}

\begin{examplle}[]
Consider \(\bigcap_{n\in\N}(-\frac{1}{n},\frac{1}{n})=\{0\}\). \((-1/n,1/n)\) is open but \(\{0\}\) is not
open in \(\R\).
\end{examplle}

If \(X\)is a topological space with topology \(\calt\), we say that a subset \(U\) of \(X\) is an
\textbf{open set} of \(X\) if \(U\in\calt\)

\begin{examplle}[]
If \(X\) is any set, the collection of all subsets of \(X\) is a topology on \(X\); it is called
the \textbf{discrete topology}. The collection consisting of \(X\) and \(\emptyset\) only is also a topology
on \(X\); we shall call it the \textbf{indiscrete topology}
\end{examplle}

\begin{examplle}[]
Let \(X\) be a set; let \(\calt_f\) be the collection of all subsets \(U\) of \(X\)s.t. \(X-U\)
either is finite or is all of \(X\). Then \(\calt_f\) is a topology on \(X\), called the \textbf{finite
complement topology}. If \(\{U_\alpha\}\) is an indexed family of nonempty elements of \(\calt_f\).
\begin{equation*}
X-\bigcup U_\alpha=\bigcap(X-U_\alpha)
\end{equation*}
\end{examplle}


\begin{definition}[]
Suppose that \(\calt\) and \(\calt'\) are two topology on a given set \(X\). If \(\calt'\supset\calt\) we say
that \(\calt'\) is \textbf{finer} than \(\calt\); if \(\calt'\) properly contains \(\calt\) we say that \(\calt'\)  is
\textbf{strictly finer} than \(\calt\). We say that \(\calt\) is \textbf{coarser} than \(\calt'\) or \textbf{strictly coarser}. We
say \(\calt\) is \textbf{comparable} with \(\calt\) is either \(\calt'\supset\calt\) or \(\calt\supset\calt'\)
\end{definition}

\subsection{Basis for a Topology}
\label{sec:orgc7bf3a7}
\begin{definition}[]
If \(X\) is a set, a \textbf{basis} for a topology on \(X\) is a collection \(\calb\) of subsets of \(X\)
(called \textbf{basis element}) s.t.
\begin{enumerate}
\item for each \(x\in X\), there is at least one basis element \(B\) s.t. \(x\in B\)
\item if \(x\in B_1\cap B_2\), then there is a basis element \(B_3\) s.t. \(x\in B_3\subset B_1\cap B_2\)
\end{enumerate}


If \(\calb\) satisfies these conditions, then we define the \textbf{topology \(\calt\) generated by \(\calb\)} as
follows: A subset \(U\) of \(X\) is said to be open in \(X\) if for each \(x\in U\), there is a
basis \(B\in\calb\) s.t. \(x\in B\subset U\).
\end{definition}

Now we show that \(\calt\) is indeed a topology. Take an indexed family \(\{U_\alpha\}_{\alpha\in J}\) of elements
of \(\calt\), we show that
\begin{equation*}
U=\bigcup_{\alpha\in J}U_\alpha
\end{equation*}
belongs to \(\calt\). Given \(x\in U\), there is an index \(\alpha\) s.t. \(x\in U_\alpha\). Since \(U_\alpha\) is open,
there is a basis element \(B\) s.t. \(x\in B\subset U_\alpha\). Then \(x\in B\) and \(B\subset U\), so \(U\) is open.

If \(U_1,U_2\in \calt\), then given \(x\in U_1\cap U_2\). we choose \(x\in B_1\subset U_1\) and \(x\in B_2\subset U_2\). By the
second condition  for a basis we have \(x\in B_3\subset B_1\cap B_2\). Hence \(x\in B_3\subset U_1\cap U_2\).

\begin{lemma}[]
Let \(X\) be a set; let \(\calb\) be a basis for a topology \(\calt\) on \(X\). Then \(\calt\) equals the
collection of all unions of elements of \(\calb\).
\end{lemma}

\begin{proof}
Given a collection of elements of \(\calb\), they are also elements of \(\calt\). Because \(\calt\) is a
topology, their union is in \(\calt\).

Conversely, given \(U\in\calt\), choose for each \(x\in U\) an element \(B_x\) for \(B\)
s.t. \(x\in B_x\subset U\). Then \(U=\bigcup_{x\in U}B_x\)
\end{proof}
\end{document}
