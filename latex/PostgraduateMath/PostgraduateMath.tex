% Created 2020-09-01 二 18:26
% Intended LaTeX compiler: pdflatex
\documentclass[11pt]{article}
\usepackage[utf8]{inputenc}
\usepackage[T1]{fontenc}
\usepackage{graphicx}
\usepackage{grffile}
\usepackage{longtable}
\usepackage{wrapfig}
\usepackage{rotating}
\usepackage[normalem]{ulem}
\usepackage{amsmath}
\usepackage{textcomp}
\usepackage{amssymb}
\usepackage{capt-of}
\usepackage{hyperref}
\usepackage{minted}
% TIPS
% \substack{a\\b} for multiple lines text





% pdfplots will load xolor automatically without option
\usepackage[dvipsnames]{xcolor}

\usepackage{forest}
% two-line text in node by [two \\ lines]
% \begin{forest} qtree, [..] \end{forest}
\forestset{
  qtree/.style={
    baseline,
    for tree={
      parent anchor=south,
      child anchor=north,
      align=center,
      inner sep=1pt,
    }}}
%\usepackage{flexisym}
% load order of mathtools and mathabx, otherwise conflict overbrace

\usepackage{mathtools}
%\usepackage{fourier}
\usepackage{pgfplots}
\usepackage{amsthm, mathabx,  amsmath, commath}
\usepackage{amsfonts}

\usepackage{empheq}
\usepackage{tikz}
\usetikzlibrary{arrows.meta}
\usepackage[most]{tcolorbox}

\newtheorem{theorem}{Theorem}[section]
\newtheorem{definition}{Definition}[section]
\newtheorem{corollary}{Corollary}[section]
\newtheorem{example}{Example}[section]
\newtheorem{lemma}{Lemma}[section]
\newtheorem{proposition}{Proposition}[section]

\newcommand{\bl}[1] {\boldsymbol{#1}}
\newcommand{\Wt}[1] {\stackrel{\sim}{\smash{#1}\rule{0pt}{1.1ex}}}
\newcommand{\wt}[1] {\widetilde{#1}}


%For boxed texts in align, use Aboxed{}
%otherwise use boxed{}

\DeclareMathSymbol{\widehatsym}{\mathord}{largesymbols}{"62}
\newcommand\lowerwidehatsym{%
  \text{\smash{\raisebox{-1.3ex}{%
    $\widehatsym$}}}}
\newcommand\fixwidehat[1]{%
  \mathchoice
    {\accentset{\displaystyle\lowerwidehatsym}{#1}}
    {\accentset{\textstyle\lowerwidehatsym}{#1}}
    {\accentset{\scriptstyle\lowerwidehatsym}{#1}}
    {\accentset{\scriptscriptstyle\lowerwidehatsym}{#1}}
}

\usepackage{graphicx}
    
% text on arrow for xRightarrow
\makeatletter
%\newcommand{\xRightarrow}[2][]{\ext@arrow 0359\Rightarrowfill@{#1}{#2}}
\makeatother


\def \bx {\boldsymbol{x}}
\def \ba {\boldsymbol{a}}
\def \bI {\boldsymbol{I}}
\def \bt {\boldsymbol{t}}
\def \bb {\boldsymbol{b}}
\def \bA {\boldsymbol{A}}
\def \bX {\boldsymbol{X}}
\def \bu {\boldsymbol{u}}
\def \bS {\boldsymbol{S}}
\def \bZ {\boldsymbol{Z}}
\def \bz {\boldsymbol{z}}
\def \by {\boldsymbol{y}}
\def \bw {\boldsymbol{w}}
\def \bT {\boldsymbol{T}}
\def \bS {\boldsymbol{S}}
\def \bm {\boldsymbol{m}}
\def \bW {\boldsymbol{W}}
\def \bY {\boldsymbol{Y}}
\def \bH {\boldsymbol{H}}
\def \blambda {\boldsymbol{\lambda}}
\def \bPhi {\boldsymbol{\Phi}}
\def \btheta {\boldsymbol{\theta}}
\def \bmu {\boldsymbol{\mu}}
\def \bphi {\boldsymbol{\phi}}
\def \bSigma {\boldsymbol{\Sigma}}
\def \lb {\left\{}
\def \rb {\right\}}
\def \caln {\mathcal{N}}
\def \dissum {\displaystyle\Sigma}
\def \dispro {\displaystyle\prod}
\def \E {\mathbb{E}}
\def \Q {\mathbb{Q}}
\def \V {\mathbb{V}}
\def \R {\mathbb{R}}
\def \calq {\mathcal{Q}}
\def \calg {\mathcal{G}}
\def \caln {\mathcal{N}}
\def \calr {\mathcal{R}}
\def \calm {\mathcal{M}}
\def \calc {\mathcal{C}}
\def \bcup {\bigcup}

\usepackage[UTF8]{ctex}
\author{Qi'ao Chen}
\date{\today}
\title{10天入门高数}
\hypersetup{
 pdfauthor={Qi'ao Chen},
 pdftitle={10天入门高数},
 pdfkeywords={},
 pdfsubject={},
 pdfcreator={Emacs 26.3 (Org mode 9.4)}, 
 pdflang={English}}
\begin{document}

\maketitle
\tableofcontents \clearpage
\section{函数与极限}
\label{sec:org6c69452}
\subsection{映射与函数}
\label{sec:org28238f0}
\begin{proposition}[]
Suppose \(f(x)\)'s domain is \((-l,l)\), then there is odd function
\(f_o(x)\) and even function \(f_e(x)\) on \((-l,l)\) s.t.
\begin{equation*}
f(x)=g(x)+h(x)
\end{equation*}
\end{proposition}

\begin{proof}
\begin{equation*}
f_e(x)=\frac{f(x)+f(-x)}{2}\quad
f_o(x)=\frac{f(x)-f(-x)}{x}
\end{equation*}
\end{proof}

基本初等函数
\begin{itemize}
\item 幂函数: \(y=x^\mu\) (\(\mu\in\R\) is a constant)
\item 指数函数:\(y=a^x\) (\(a\iffalse<\fi>0\) and \(a\neq1\))
\item 对数函数:\(y=\log_ax\) (\(a>0\) and \(a\neq1\))
\item 三角函数:\(y=\sin x,\cos x,\tan x\)
\item 反三角函数:\(y=\arcsin x,\arccos x,\arctan x\)
\end{itemize}
\subsection{数列的极限}
\label{sec:org1900f15}
\begin{definition}[]
suppose \(\{x_n\}\) is a sequence, if there is a constant \(a\) for any
positive \(\epsilon\), there is a positive integer \(N\) s..t if \(n>N\), then
\begin{equation*}
\abs{x_n-a}<\epsilon
\end{equation*}
always holds, then \(a\) is called the limit of \(\{x_n\}\), or \(\{x_n\}\)
converges to \(a\), written as
\begin{equation*}
\lim_{n\to\infty}x_n=a
\end{equation*}
or
\begin{equation*}
x_n\to a(n\to \infty)
\end{equation*}
\end{definition}

\begin{theorem}[极限的唯一性]
如果数列\(\{x_n\}\)收敛,那么它的极限唯一
\end{theorem}

\begin{proof}
假设同时有\(x_n\to a\)及\(x_n\to b\),且\(a<b\),取\(\epsilon=\frac{b-a}{2}\),
因为\(\lim_{n\to\infty}x_n=a\),故存在正整数\(N_1\),当\(n>N_1\)时,
\begin{equation}
\abs{x_n-a}<\frac{b-a}{2}\label{eq2-2}
\end{equation}
同理有当\(n>N_2\)时
\begin{equation}
\abs{x_n-b}<\frac{b-a}{2}\label{eq2-3}
\end{equation}
取\(N=\max\{N_1,N_2\}\),由\eqref{eq2-2} 有\(x_n<\frac{a+b}{2}\),由
\eqref{eq2-3} 有\(x_n>\frac{a+b}{2}\),矛盾
\end{proof}

\begin{theorem}[收敛数列的有界性]
如果数列\(\{x_n\}\)收敛,那么数列\(\{x_n\}\)一定有界
\end{theorem}

\begin{proof}
因为数列\(\{x_n\}\)收敛,设\(\lim_{n\to\infty}x_n=a\),对于\(\epsilon=1\),存
在正整数\(N\),当\(n>N\)时有
\begin{equation*}
\abs{x_n-a}<1
\end{equation*}
于是当\(n>N\)时
\begin{equation*}
\abs{x_n}=\abs{x_n-a+a}\le\abs{x_n-a}+\abs{a}<1+\abs{a}
\end{equation*}
取\(M=\max\{\abs{x_1},\dots,\abs{x_N},1+\abs{a}\}\),那么数列\(\{x_n\}\)中的
一切\(x_n\)都满足不等式
\begin{equation*}
\abs{x_n}\le M
\end{equation*}
\end{proof}

\begin{theorem}[收敛数列的保号性]
如果\(\lim_{n\to\infty}x_n=a\)且\(a>0\)(或\(a<0\)),那么存在正整数\(N\),当
\(n>N\)时,都有\(x_n>0\)(或\(x_n<0\))
\end{theorem}

\begin{proof}
Suppose \(a>0\), let \(\epsilon=\frac{a}{2}>0\), then there is \(N\) for
\(n>N\) s.t.
\begin{equation*}
\abs{x_n-a}<\frac{a}{2}
\end{equation*}
Hence
\begin{equation*}
x_n>a-\frac{a}{2}=\frac{a}{2}>0
\end{equation*}
\end{proof}

\begin{corollary}[]
如果数列\(\{x_n\}\)从某项起有\(x_n\ge0\)(或\(x_n\le0\)),且
\(\lim_{n\to\infty}x_n=a\),那么\(a\ge0\)(或\(a\le0\))
\end{corollary}

在数列\(\{x_n\}\)中任意抽取无限多项并保持这些项在原数列\(\{x_n\}\)中的先后次
序,这样得到的一个数列称为原数列\(\{x_n\}\)的 \textbf{子数列}

\begin{theorem}[收敛数列与其子数列的关系]
如果数列\(\{x_n\}\)收敛于\(a\),那么它的任一子数列也收敛,且极限也是\(a\)
\end{theorem}

\begin{proof}
设数列\(\{x_{n_k}\}\)是数列\(\{x_n\}\)的任一子数列

由于\(\lim_{n\to\infty}x_n=a\),故对任意\(\epsilon>0\),存在正整数\(N\)当\(n>N\)时,
\(\abs{x_n-a}<\epsilon\)

取\(K=N\),则当\(k>K\)时,\(n_k>n_K=n_N\ge N\),于是
\(\abs{x_{n_k}-a}<\epsilon\),因此\(\lim_{k\to\infty}x_{n_k}=a\)
\end{proof}
\subsection{函数的极限}
\label{sec:org960cb1f}
\subsubsection{函数极限的定义}
\label{sec:orgbb84f0d}
\begin{definition}[]
设函数\(f(x)\)在点\(x_0\)的某一去心邻域内有定义,如果存在常数\(A\)对于任一给
定的正数 \(\epsilon\) 总存在正数 \(\delta\) 使得当\(x\) 满足不等式\(0<\abs{x-x_0}<\delta\)时,对
应的函数值 \(f(x)\) 都满足不等式
\begin{equation*}
\abs{f(x)-A}<\epsilon
\end{equation*}
那么常数\(A\)就叫做 \textbf{函数\(f(x)\)当\(x\to x_0\)时的极限} ,记作
\begin{equation*}
\lim_{x\to x_0}f(x)=A \quad\text{ or }\quad
f(x)\to A(\text{when }x\to x_0)
\end{equation*}
\end{definition}

\begin{proposition}[]
\(\lim_{x\to1}(2x-1)=1\)
\end{proposition}

\begin{proof}
Since
\begin{equation*}
\abs{f(x)-A}=\abs{2x-2}=2\abs{x-1}
\end{equation*}
for any \(\epsilon>0\), let \(\delta=\epsilon/2\), then if
\begin{equation*}
0<\abs{x-1}<\delta
\end{equation*}
we have
\begin{equation*}
\abs{f(x)-1}=2\abs{x-1}<\epsilon
\end{equation*}
hence
\begin{equation*}
\lim_{x\to1}(2x-1)=1
\end{equation*}
\end{proof}

将\(0<\abs{x-x_0}<\delta\) 改为\(x_0-\delta<x<x_0\),那么\(A\)就叫做函数
\(f(x)\) 当\(x\to x_0\)时的 \textbf{左极限} ,记作
\begin{equation*}
\lim_{x\to x_0^-}f(x)=A\quad\text{ or }\quad
f(x_0^-)=A
\end{equation*}

函数\(f(x)\)当\(x\to x_0\)时极限存在的充分必要条件时左极限及右极限各自存在且
相等

\begin{definition}[]
设函数\(f(x)\)当\(\abs{x}\)大于某一正数时有定义,如果存在常数\(A\)对于任意给定
的正数 \(\epsilon\) 总存在正数 \(X\) 使得当 \(x\) 满足不等式  \(\abs{x}>X\) 时,对应的函
数值满足
\begin{equation*}
\abs{f(x)-A}<\epsilon
\end{equation*}
那么常数 \(A\) 就叫做 \textbf{函数\(f(x)\)当\(x\to\infty\)时的极限} ,记作
\begin{equation*}
\lim_{x\to\infty}f(x)=A \quad\text{ or }\quad
f(x)\to A(\text{when }x\to\infty)
\end{equation*}
\end{definition}
\subsubsection{函数极限的性质}
\label{sec:org1386f06}
\begin{theorem}[函数极限的唯一性]
如果\(\lim_{x\to x_0}f(x)\)存在,那么这极限唯一
\end{theorem}

\begin{proof}
If \(\lim_{x\to x_0}f(x)=a\) and \(\lim_{x\to x_0}f(x)=b\), let
\(\epsilon=\frac{b-a}{2}\), there is \(\delta_1\) and \(\delta_2\) s.t. for
\(0<\abs{x-x_0}<\delta_1\), \(\abs{f(x)-a}<\frac{b-a}{2}\), and balabala\ldots{}
\end{proof}

\begin{theorem}[函数极限的局部有界性]
如果\(\lim_{x\to x_0}f(x)=A\),那么存在常数\(M>0\)和\(\delta>0\)使得当
\(0<\abs{x-x_0}<\delta\)时,有\(\abs{f(x)}\le M\)
\end{theorem}

\begin{proof}
取\(\epsilon=1\), then there is \(\delta\) for \(0<\abs{x-x_0}<\delta\), we have
\begin{equation*}
\abs{f(x)-A}<1\Rightarrow\abs{f(x)}\le\abs{f(x)-A}+\abs{A}<\abs{A}+1
\end{equation*}
记\(M=\abs{A}+1\)
\end{proof}

\begin{theorem}[函数极限的局部保号性]
如果\(\lim_{x\to x_0}f(x)=A\),且\(A>0\)(或\(A<0\)),那么存在常数\(\delta>0\),使得
当\(0<\abs{x-x_0}<\delta\)时有\(f(x)>0\) (或\(f(x)<0\))
\end{theorem}
\subsection{无穷大与无穷小}
\label{sec:org1fb2184}
\begin{definition}[]
如果函数\(f(x)\)当\(x\to x_0\)(或\(x\to\infty\))时的极限为0,那么称\(f(x)\)
为当\(x\to x_0\)(或\(x\to\infty\))时的无穷小
\end{definition}

\begin{theorem}[]
在自变量的同一变化过程\(x\to x_0\)(或\(x\to\infty\))中,函数\(f(x)\)具有极
限\(A\)的充分必要条件是\(f(x)=A+\alpha\),其中 \(\alpha\) 是无穷小
\end{theorem}

\begin{definition}[]
设函数\(f(x)\)在\(x_0\)的某一去心邻域内有定义(或\(abs{x}\)大于某一正数时有定
义),如果对于任一给定的正数\(M\),总存在正数 \(\delta\) ,如果
\(0<\abs{x-x_0}<\delta\) 则
\(\abs{f(x)}>M\)
那么称函数\(f(x)\)是当\(x\to x_0\)(或\(x\to\infty\))时的无穷大
记作
\begin{equation*}
\lim_{x\to x_0}f(x)=\infty
\end{equation*}
\end{definition}

\begin{theorem}[]
在自变量的同一变化过程中,如果\(f(x)\)为无穷大,那么\(\frac{1}{f(x)}\)为无穷
小;反之亦然
\end{theorem}
\subsection{极限运算法则}
\label{sec:org6c10b91}
\begin{theorem}[]
两个无穷小的和是无穷小
\end{theorem}

\begin{theorem}[]
有界函数与无穷小的乘积是无穷小
\end{theorem}

\begin{corollary}[]
常数与无穷小的乘积时无穷小
\end{corollary}

\begin{corollary}[]
有限个无穷小的乘积是无穷小
\end{corollary}

\begin{theorem}[]
如果\(\lim f(x)=A,\lim g(x)=B\),那么
\begin{enumerate}
\item \(\lim[f(x)\pm g(x)]=\lim f(x)\pm\lim g(x)=A\pm B\)
\item \(\lim[f(x)\cdot g(x)]=\lim f(x)\cdot\lim g(x)=A\cdot B\)
\item 如果\(B\neq0\),则
\begin{equation*}
\lim\frac{f(x)}{g(x)}=\frac{\lim f(x)}{\lim g(x)}=\frac{A}{B}
\end{equation*}
\end{enumerate}
\end{theorem}

\begin{corollary}[]
If \(\lim f(x)\) exists, and \(c\) is a constant, then
\begin{equation*}
\lim[cf(x)]=c\lim f(x)
\end{equation*}
\end{corollary}

\begin{corollary}[]
if \(\lim f(x )\) exists, and \(n\) is a positive integer, then
\begin{equation*}
\lim[f(x)]^n=[\lim f(x)]^n
\end{equation*}
\end{corollary}

\begin{theorem}[]
设有数列\(\{x_n\}\)和\(\{y_n\}\),如果
\begin{equation*}
\lim_{n\to\infty}x_n=A,\quad\lim_{n\to\infty}y_n=B
\end{equation*}
那么
\begin{enumerate}
\item \(lim_{n\to\infty}(x_n\pm y_n)=A\pm B\)
\item \(\lim_{n\to\infty}(x_n\cdot y_n)=A\cdot B\)
\item 当 \(y_n\neq0(n=1,2,\dots)\)且\(B\neq0\)时,\(\lim_{n\to\infty}\frac{x_n}{y_n}=\frac{A}{B}\)
\end{enumerate}
\end{theorem}

\begin{theorem}[]
如果\(\varphi(x)\ge\psi(x)\),而\(\lim\varphi(x)=A,\lim\psi(x)=B\),那么\(A\ge B\)
\end{theorem}

\begin{theorem}[复合函数的极限运算法则]
设函数\(y=f[g(x)]\)是由函数\(u=g(x)\)与函数\(y=f(u)\)复合而成,\(f[g(x)]\)在
点\(x_0\)的某去心邻域内有定义,若\(\lim_{x\to x_0}g(x)=u_0\),\(\lim_{u\to
   u_0}f(u)=A\),且存在\(\delta_0>0\),当\(x\in\interior{U}(x_0,\delta_0)\)时,
有\(g(x)\neq u_0\),则
\begin{equation*}
\lim_{x\to x_0}f[g(x)]=\lim_{u\to u_0}f(u)=A
\end{equation*}
\end{theorem}
\subsection{极限存在准则 两个重要极限}
\label{sec:orgc5eacac}
\begin{proposition}[准则1]
如果数列\(\{x_n\},\{y_n\},\{z_n\}\)满足
\begin{enumerate}
\item 存在\(n_0\in\N\),当\(n>n_0\)时,有
\begin{equation*}
y_n\le x_n\le z_n
\end{equation*}
\item \(\lim_{n\to\infty}y_n=a,\lim_{n\to\infty}x_n=a\)
\end{enumerate}


那么数列\(\{x_n\}\)的极限存在,且\(\lim_{n\to\infty}x_n=a\)
\end{proposition}

\begin{proposition}[]
if
\begin{enumerate}
\item when \(x\in\interior{U}(x_0,r)\) (or \(\abs{x}>M\))
\begin{equation*}
g(x)\le f(x)\le h(x)
\end{equation*}
\item \(\lim_{\substack{x\to x_0\\x\to\infty}}g(x)=A,\lim_{\substack{x\to
      x_0\\x\to \infty}}h(x)=A\)
\end{enumerate}


那么\(\lim_{\substack{x\to x_0\\x\to\infty}}x_n=A\)
\end{proposition}

\begin{proposition}[准则2]
单调有界数列必有极限
\end{proposition}

\begin{corollary}[]
\(\lim_{x\to\infty}(1+\frac{1}{x})^x\)
\end{corollary}

\begin{proof}
let \(x_n=(1+\frac{1}{n})^n\)
\begin{align*}
x_n&=(1+\frac{1}{n})^n\\
&=1+\frac{n}{1!}\cdot\frac{1}{n}+\frac{n(n-1)}{2!}\cdot\frac{1}{n^2}+
\frac{n(n-1)(n-2)}{3!}\cdot\frac{1}{n^3}+\cdots+\\
&\frac{n(n-1)\dots(n-n+1)}{n!}\cdot\frac{1}{n^n}\\
&=1+1+\frac{1}{2!}(1-\frac{1}{n})+\frac{1}{3!}(1-\frac{1}{n})(1-\frac{2}{n})+\cdots+\\
&\frac{1}{n!}(1-\frac{1}{n})(1-\frac{2}{n})\cdots(1-\frac{n-1}{n})
\end{align*}
similarly
\begin{align*}
x_{n+1}&=1+1+\frac{1}{2!}(1-\frac{1}{n+1})+\frac{1}{3!}(1-\frac{1}{n+1})(1-\frac{2}{n+1})+\cdots+\\
&\frac{1}{n!}(1-\frac{1}{n+1})(1-\frac{2}{n+1})\cdots(1-\frac{n-1}{n+1})\\
&\frac{1}{(n+1)!}(1-\frac{1}{n+1})(1-\frac{2}{n+1})\cdots(1-\frac{n}{n+1})
\end{align*}

Hence \(\{x_n\}\) is an increasing sequence and
\begin{align*}
x_n&\le 1+(1+\frac{1}{2!}+\cdots+\frac{1}{n!})\le 1+(1+\frac{1}{2}+\frac{1}{2^2}+\cdots+\frac{1}{2^{n-1}})\\
&=3-\frac{1}{2^{n-1}}<3
\end{align*}
Hence \(\{x_n\}\) 的极限存在

if \(n\le x<n+1\), then
\begin{equation*}
(1+\frac{1}{n+1})^n<(1+\frac{1}{x})^x<(1+\frac{1}{n})^{n+1}
\end{equation*}
因为夹逼准则,我们有
\begin{equation*}
\lim_{x\to+\infty}(1+\frac{1}{x})^x=e
\end{equation*}
\end{proof}

\begin{proposition}[准则2']
设函数\(f(x)\)在点\(x_0\)的某个左邻域内单调并且有界,则\(f(x)\)在\(x_0\)的左
极限\(f(x_0^-)\)存在
\end{proposition}

\begin{proposition}[柯西极限存在准则]
数列\(\{x_n\}\)收敛 iff 对于任意给定的正数 \(\epsilon\) ,存在正整数 \(N\)使得当
\(m,n>N\)时,有
\begin{equation*}
\abs{x_n-x_m}<\epsilon
\end{equation*}
\end{proposition}
\subsection{无穷小的比较}
\label{sec:org1072a0e}
\begin{definition}[]
如果\(\lim\frac{\beta}{\alpha}=0\),那么就说 \(\beta\) 是比 \(\alpha\) \textbf{高阶的无穷小} ,记作
\(\beta=o(\alpha)\)

如果\(\lim\frac{\beta}{\alpha}=\infty\),那么就说 \(\beta\) 是比 \(\alpha\) \textbf{低阶的无穷小}

如果 \(\lim\frac{\beta}{\alpha}=c\neq0\),那么就说 \(\beta\) 与 \(\alpha\) 是 \textbf{同阶无穷小}

如果 \(\lim\frac{\beta}{\alpha^k}=c\neq0,k>0\),那么就说 \(\beta\) 是关于 \(\alpha\) 的 \textbf{\(k\)阶无穷小}

如果 \(\lim\frac{\beta}{\alpha}=1\),那么就说 \(\beta\) 与 \(\alpha\) 是 \textbf{等价无穷小} ,记作 \(\alpha\sim\beta\)
\end{definition}

\begin{proposition}[]
\(\lim_{x\to0}\frac{(1+x)^{\frac{1}{n}}-1}{\frac{1}{n}x}=1\)
\end{proposition}
\begin{proof}
\begin{align*}
\lim_{x\to0}\frac{(1+x)^{1}{n}-1}{\frac{1}{n}x}&=
\frac{1+x-1}{\frac{1}{n}x[\sqrt[n]{(1+x)^{n-1}}+\sqrt[n]{(1+x)^{n-2}}+\dots+1]}\\
&=\lim_{x\to0}\frac{n}{\sqrt[n]{(1+x)^{n-1}}+\dots+1}=1
\end{align*}
\end{proof}

\begin{theorem}[]
\(\beta\) 与 \(\alpha\) 是等价无穷小的充分必要条件是
\begin{equation*}
\beta=\alpha+o(\alpha)
\end{equation*}
\end{theorem}

\begin{proof}
if \(\alpha\sim\beta\), then
\begin{equation*}
\lim\frac{\beta-\alpha}{\alpha}=\lim(\frac{\beta}{\alpha}-1)=\lim\frac{\beta}{\alpha}-1=0
\end{equation*}
Hence \(\beta-\alpha=o(\alpha)\)

If \(\beta=\alpha+o(\alpha)\), then
\begin{equation*}
\lim   \frac{\beta}{\alpha}=\lim\frac{\alpha+o(\alpha)}{\alpha}=1
\end{equation*}
\end{proof}

\begin{theorem}[]
设 \(\alpha\sim\widetilde{\alpha}\), \(\beta\sim\widetilde{\beta}\), 且
\(\lim\frac{\widetilde{\beta}}{\widetilde{\alpha}}\)存在,则
\begin{equation*}
\lim\frac{\beta}{\alpha}=\lim\frac{\widetilde{\beta}}{\widetilde{\alpha}}
\end{equation*}
\end{theorem}

\begin{proof}
\begin{equation*}
\lim\frac{\beta}{\alpha}=\lim(\frac{\beta}{\widetilde{\beta}}\cdot\frac{\widetilde{\beta}}{\widetilde{\alpha}}\cdot\frac{\widetilde{\alpha}}{\alpha})
\end{equation*}
\end{proof}
\subsection{函数的连续性与间断点}
\label{sec:org933afc5}
\index{连续}
\begin{definition}[]
设函数\(y=f(x)\)在点\(x_0\)的某一邻域内有定义,如果
\begin{equation*}
\lim_{\Delta x\to0}\Delta y=\lim_{\Delta x\to0}[f(x_0+\Delta x)-f(x_0)]=0
\end{equation*}
那么就称函数\(y=f(x)\)在点\(x_0\)连续


设函数\(y=f(x)\)在点\(x_0\)的某一邻域内有定义,如果
\begin{equation*}
\lim_{x\to x_0}f(x)=f(x_0)
\end{equation*}
那么就称函数\(f(x)\)在点\(x_0\)连续
\end{definition}

设函数\(f(x)\)在点\(x_0\)的某一去心邻域内有定义,如果有下列三种情况之一
\begin{enumerate}
\item 在\(x=x_0\)没有定义
\item 虽在\(x=x_0\)有定义,但\(\lim_{x\to x_0}f(x)\)不存在
\item 虽在\(x=x_0\)有定义,且\(\lim_{x\to x_0}f(x)\)存在,但\(\lim_{x\to
      x_0}f(x)\neq f(x_0)\)
\end{enumerate}


那么\(f(x)\)在点\(x_0\)不连续,而点\(x_0\)称为函数\(f(x)\)的 \textbf{不连续点} 或 \textbf{间}
\textbf{断点}

如果\(x_0\)时函数\(f(x)\)的间断点,但左极限\(f(x_0^-)\)及右极限\(f(x_0^+)\)都
存在,那么\(x_0\)称为函数\(f(x)\)的 \textbf{第一类间断点} ,其他为 \textbf{第二类间断点}
\subsection{极限函数的运算与初等函数的连续性}
\label{sec:orgcdb8aaa}
\subsubsection{连续函数的和、差、积、商的连续性}
\label{sec:orgd27b1d1}
\begin{theorem}[]
设函数\(f(x)\)和\(g(x)\)在点\(x_0\)连续,则它们的和、差、积、商(当
\(g(x_0)\neq0\)时)都在点\(x_0\)处连续
\end{theorem}
\subsubsection{反函数与复合函数的连续性}
\label{sec:org787c767}
\begin{theorem}[]
如果函数\(y=f(x)\)在区间\(I_x\)上单调增加(或单调减少)且连续,那么它的反函数
\(x=f^{-1}(y)\)也在对应区间\(I_y=\{y\mid y=f(x),x\in I_x\}\)上单调增加(或单
调减少)且连续
\end{theorem}

\begin{theorem}[]
设函数\(y=f[g(x)]\)由函数\(u=g(x)\)与函数\(y=f(u)\)复合而成,
\(\interior{U}(x_0)\susbet D_{f\circ g}\),若\(\lim_{x\to x_0}g(x)=u_0\),而
函数 \(y=f(u)\)在\(u=u_0\)处连续,则
\begin{equation*}
\lim_{x\to x_0}f[g(x)]=\lim_{u\to u_0}f(u)=f(u_0)
\end{equation*}
\end{theorem}

\begin{theorem}[]
设函数\(y=f[g(x)]\)是由函数\(u=g(x)\)与函数\(y=f(u)\)复合而成,
\(U(x_0)\subset D_{f\circ g}\),若函数\(u=g(x)\)在\(x=x_0\)连续,且
\(g(x_0)=u_0\),而函数\(y=f(u)\)在\(u=u_0\)连续,则复合函数\(y=f[g(x)]\)在
\(x=x_0\)也连续
\end{theorem}
\subsubsection{初等函数的连续性}
\label{sec:org46e89de}
\textbf{一切初等函数在其定义区间内都是连续的}

\begin{gather*}
\ln(1+x)\sim x\quad(x\to0)\\
e^x-1\sim x\quad(x\to0)\\
(1+x)^\alpha-1\sim\alpha x\quad(x\to0)
\end{gather*}
\subsection{闭区间上连续函数的性质}
\label{sec:org1db84fb}
\begin{theorem}[有界性与最大值最小值定理]
在闭区间上连续的函数在该区间上有界且一定能取得它的最大值和最小值
\end{theorem}

\begin{theorem}[零点定理]
设函数\(f(x)\)在闭区间\([a,b]\)上连续,且\(f(a)\)与\(f(b)\)异号,则在开区间
\((a,b)\)内至少有一点 \(\xi\) 使
\begin{equation*}
f(\xi)=0
\end{equation*}
\end{theorem}

\begin{theorem}[介值定理]
设函数\(f(x)\)在闭区间\([a,b]\)上连续,且在这区间的端点取不同的函数值
\begin{equation*}
f(a)=A \quad\text{ and }\quad f(b)=B
\end{equation*}
则对于\(A\)与\(B\)之间的任意一个数\(C\),在开区间\((a,b)\)内至少有一点 \(\xi\) ,使
得
\begin{equation*}
f(\xi)=C\quad(a<\xi<b)
\end{equation*}
\end{theorem}

\begin{corollary}[]
在闭区间\([a,b]\)上连续的函数\(f(x)\)的值域为闭区间\([m,M]\),其中\(m\)与
\(M\)依次为\(f(x)\)在\([a,b]\)上的最大值、最小值
\end{corollary}
\section{导数与微分}
\label{sec:org1af2d24}
\subsection{导数概念}
\label{sec:orgc491512}
\begin{definition}[]
设函数\(y=f(x)\)在点\(x_0\)的某个邻域内有定义,当自变量\(x\)在\(x_0\)处取得增
量\(\Delta x\)(点\(x+\Delta x\)仍在该邻域内)时,相应地,因变量取得增量\(\Delta
   y=f(x_0+\Delta x)-f(x_0)\);如果\(\Delta y\)与\(\Delta x\)之比当\(\Delta x\to0\)时的极限存在,
那么称函数\(y=f(x)\)在点\(x_0\)处 \textbf{可导} ,并称这个极限为函数 \(y=f(x)\) 在点
\(x_0\)处的 \textbf{导数} ,记为\(f'(x_0)\),即
\begin{equation*}
f'(x_0)=\lim_{\Delta x\to0}\frac{\Delta y}{\Delta x}=\lim_{\Delta x\to0}\frac{f(x_0+\Delta x)-f(x_0)}{\Delta x}
\end{equation*}
\end{definition}

左导数,右导数
\begin{align*}
f'_-(x_0)&=\lim_{h\to0^-}\frac{f(x_0+h)-f(x_0)}{h}\\
f'_+(x_0)&=\lim_{h\to0^+}\frac{f(x_0+h)-f(x_0)}{h}
\end{align*}

\(f(x)\)在点\(x_0\)处可导的充分必要条件是左导数右导数存在且相等

设函数\(y=f(x)\)在点\(x\)处可导,即
\begin{equation*}
\lim_{\Delta x\to0}\frac{\Delta y}{\Delta x}=f'(x)
\end{equation*}
存在,因此
\begin{equation*}
\frac{\Delta y}{\Delta x}=f'(x)+\alpha
\end{equation*}
其中 \(\alpha\) 为当 \(\Delta x\to0\)时的无穷小。两边同乘 \(\Delta x\),得
\begin{equation*}
\Delta y=f'(x)\Delta x+\alpha \Delta x
\end{equation*}
由此可见,当\(\Delta x\to0\)时,\(\Delta y\to0\),这就是说\(y=f(x)\)在点\(x\)处连续,因
此可导必连续,但连续不一定可导

\begin{examplle}[]
函数\(y=f(x)=\sqrt[3]{x}\)在区间\(-\infty,+\infty\)内连续,但在点\(x=0\)处不
可导,因为
\begin{equation*}
\frac{f(0+h)-f(0)}{h}=\frac{\sqrt[3]{h}-0}{h}=\frac{1}{h^{2/3}}
\end{equation*}
因而\(\lim_{h\to0}\frac{f(0+h)-f(0)}{h}=+\infty\)
\end{examplle}
\subsection{函数的求导法则}
\label{sec:orgdd56464}
\begin{theorem}[]
如果函数\(u=u(x)\)及\(v=v(x)\)都在点\(x\)具有导数,那么它们的和差积商(除分母
为零的点外)都在点\(x\)具有导数,且
\begin{enumerate}
\item \([u(x)\pm v(x)]'=u'(x)\pm v'(x)\)
\item \([u(x)v(x)]'=u'(x)v(x)+u(x)v'(x)\)
\item \([\frac{u(x)}{v(x)}]'=\frac{u'(x)v(x)-u(x)v'(x)}{v^2(x)}(v(x)\neq0)\)
\end{enumerate}
\end{theorem}

\begin{theorem}[]
如果函数\(x=f(y)\)在区间\(I_y\)内单调、可导且\(f'(y)\neq0\),那么它的反函数
\(y=f'(x)\)在区间\(I_x=\{x\mid x=f(y),y\in I_y\}\)内也可导,且
\begin{equation*}
[f^{-1}(x)]'=\frac{1}{f'(y)}
\end{equation*}
\end{theorem}

\begin{theorem}[]
如果\(u=g(x)\)在点\(x\)可导,而\(y=f(u)\)在点\(u=g(x)\)可导,那么复合函数
\(y=f[g(x)]\)在点\(x\)可导,且其导数为
\begin{equation*}
\frac{dy}{dx}=f'(u)\cdot g'(x) \quad\text{ or }\quad
\frac{dy}{dx}=\frac{dy}{du}\cdot\frac{du}{dx}
\end{equation*}
\end{theorem}

\begin{equation*}
(a^x)'=a^x\ln a
\end{equation*}
\subsection{高阶导数}
\label{sec:orgd0491d0}
\subsection{隐函数及由参数方程所确定的函数的导数 相关变化率}
\label{sec:org233c3ee}

若参数方程
\begin{equation*}
\begin{cases}
x=\varphi(t)\\
y=\psi(t)
\end{cases}
\end{equation*}
若函数\(x=\varphi(t)\)具有单调连续反函数\(t=\varphi^{-1}(x)\),且此反函数能与
函数\(y=\psi(t)\)构成复合函数,则
\begin{equation*}
\frac{dy}{dx}=\frac{dy}{dt}\cdot\frac{dt}{dx}=\frac{\psi'(t)}{\varphi'(t)}
\end{equation*}
\subsection{函数的微分}
\label{sec:org60370b2}
\begin{definition}[]
设函数\(y=f(x)\)在某区间内有定义,\(x_0\)及\(x_0+\Delta x\)在这区间内,如果函
数的增量
\begin{equation*}
\Delta y=f(x_0+\Delta x)-f(x_0)
\end{equation*}
可表示为
\begin{equation*}
\Delta y=A\Delta x+o(\Delta x)
\end{equation*}
其中\(A\)是不依赖于\(\Delta x\)的常数,那么称函数\(y=f(x)\)在点\(x_0\)是 \textbf{可微} 的,
而 \(A\Delta x\) 叫做函数\(y=f(x)\)在点\(x_0\)相应与自变量增量\(\Delta x\)的 \textbf{微分}
,记作 \(dy\),即\(dy=A\Delta x\)
\end{definition}

if \(y=f(x)\)在点\(x_0\)可微, then
\begin{equation*}
dy=A\Delta x
\end{equation*}
and
\begin{equation*}
\frac{\Delta y}{\Delta x}=A+\frac{o(\Delta x)}{\Delta x}
\end{equation*}
hence
\begin{equation*}
A=\lim_{\Delta x\to0}\frac{\Delta y}{\Delta x}=f'(x_0)
\end{equation*}

如果\(y=f(x)\)在点\(x_0\)可导
\begin{equation*}
\lim_{\Delta x\to0}\frac{\Delta y}{\Delta x}=f'(x_0)
\end{equation*}
we have
\begin{equation*}
\frac{\Delta y}{\Delta x}=f'(x_0)+\alpha
\end{equation*}
其中 \(\alpha=o(\Delta x)\), hence
\begin{equation*}
\Delta y=f'(x_0)\Delta x+\alpha\Delta x
\end{equation*}
hence \(f(x)\)在点\(x_0\)可微


因此\(f(x)\)在点\(x_0\)可微 iff  \(f(x)\)在点\(x_0\)可导,且当\(f(x)\)在点
\(x_0\)可微时,其微分是
\begin{equation*}
dy=f'(x_0)\Delta x
\end{equation*}
当\(f'(x_0)\neq0\)时,有
\begin{equation*}
\lim_{\Delta x\to0}\frac{\Delta y}{dy}=\lim_{\Delta x\to0}\frac{\Delta y}{f'(x_0)\Delta x}=
\frac{1}{f'(x_0)}\lim_{\Delta x\to0}\frac{\Delta y}{\Delta x}=1
\end{equation*}
Hence
\begin{equation*}
\Delta y=dy+o(dy)
\end{equation*}
即\(dy\)是\(\Delta y\)的 \textbf{主部} ( \textbf{线性主部} )

通常把自变量 \(x\)的增量\(\Delta x\)称为 \textbf{自变量的微分} ,记作\(dx\),即\(dx=\Delta
   x\),于是
\begin{equation*}
dy=f'(x)dx
\end{equation*}
从而
\begin{equation*}
\frac{dy}{dx}=f'(x)
\end{equation*}
\section{微分中值定理与导数的应用}
\label{sec:org89b8851}
\subsection{微分中值定理}
\label{sec:org98dcd8f}
\begin{theorem}[费马定理]
设函数\(f(x)\)在点\(x_0\)的某邻域\(U(x_0)\)内有定义,并且在\(x_0\)处可导,如
果对任意的\(x\in U(x_0)\),有
\begin{equation*}
f(x)\le f(x_0)\quad(\text{or }f(x)\ge f(x_0))
\end{equation*}
那么\(f'(x_0)=0\)
\end{theorem}

\begin{proof}
for any \(x_0+\Delta x\in U(x_0)\), we have
\begin{equation*}
f(x_0+\Delta x)\le f(x_0)
\end{equation*}
when \(\Delta x>0\)
\begin{equation*}
\frac{f(x_0+\Delta x)-f(x_0)}{\Delta}\le0
\end{equation*}
when \(\Delta x<0\)
\begin{equation*}
\frac{f(x_0+\Delta x)-f(x_0)}{\Delta}\ge0
\end{equation*}
Hence
\begin{align*}
&f'(x_0)=f_+'(x_0)=\lim_{\Delta x\to0^+}\frac{f(x_0+\Delta x)-f(x_0)}{\Delta x}\le0\\
&f'(x_0)=f_-'(x_0)=\lim_{\Delta x\to0^-}\frac{f(x_0+\Delta x)-f(x_0)}{\Delta x}\ge0
\end{align*}
Hence \(f'(x_0)=0\)
\end{proof}

通常称导数等于零的点为函数的 \textbf{驻点} ( \textbf{稳定点} , \textbf{临界点}  )
\begin{theorem}[罗尔定理]
如果函数\(f(x)\)满足
\begin{enumerate}
\item 在闭区间\([a,b]\)上连续
\item 在开区间\((a,b)\)上可导
\item 在区间端点处的函数值相等,即\(f(a)=f(b)\)
\end{enumerate}


那么在\((a,b)\)内至少有一点\(\xi(a<\xi<b)\)使得\(f'(\xi)=0\)
\end{theorem}

\begin{proof}
由于\(f(x)\)在闭区间\([a,b]\)上连续,根据闭区间上连续函数的最大值最小值定理,
\(f(x)\)在闭区间\([a,b]\)上必取得它的最大值\(M\)和最小值\(m\)

\begin{enumerate}
\item 若\(M=m\),\(f(x)=M\)
\item 若\(M>m\),因为\(f(a)=f(b)\),所以\(M\)和\(m\)这两个数中至少有一个不等于
\(f(x)\)在\([a,b]\)的端点处的函数值,不妨设\(M\neq f(a)\),那么必存在开区
间\((a,b)\)内有一点\(\xi\)使\(f(\xi)=M\),由费马定理
\end{enumerate}
\end{proof}

\begin{theorem}[拉格朗日中值定理]
如果函数\(f(x)\)满足
\begin{enumerate}
\item 在闭区间\([a,b]\)上连续
\item 在开区间\((a,b)\)内可导
\end{enumerate}


那么在\((a,b)\)内至少有一点\(\xi(a<\xi<b)\)使等式
\begin{equation*}
f(b)-f(a)=f'(\xi)(b-a)
\end{equation*}
成立
\end{theorem}

\begin{theorem}[]
如果函数\(f(x)\)在区间\(I\)上连续,\(I\)内可导且导数恒为0,那么\(f(x)\)在区间
\(I\)上是一个常数
\end{theorem}

\begin{theorem}[柯西中值定理]
如果函数\(f(x)\)及\(F(x)\)满足
\begin{enumerate}
\item 在闭区间\([a,b]\)上连续
\item 在开区间\((a,b)\)内可导
\item 对任一\(x\in(a,b)\), \(F'(x)\neq0\)
\end{enumerate}


那么在\((a,b)\)内至少有一点\(\xi\)使等式
\begin{equation*}
\frac{f(b)-f(a)}{F(b)-F(a)}=\frac{f'(\xi)}{F'(\xi)}
\end{equation*}
\end{theorem}
\subsection{洛必达法则}
\label{sec:org187d40d}
\begin{theorem}[洛必达法则]
设
\begin{enumerate}
\item 当\(x\to a\)时,函数\(f(x)\)及\(F(x)\)都趋于零
\item 在点\(a\)的某去心邻域内\(f'(x)\)及\(F'(x)\)都存在且\(F'(x)\neq0\)
\item \(\lim_{x\to a}\frac{f'(x)}{F'(x)}\)存在(或为无穷大)
\end{enumerate}


则
\begin{equation*}
\lim_{x\to a}\frac{f(x)}{F(x)}=\lim_{x\to a}\frac{f'(x)}{F'(x)}
\end{equation*}
\end{theorem}

\begin{theorem}[]
设
\begin{enumerate}
\item 当\(x\to\infty\)时,函数\(f(x)\)及\(F(x)\)都趋于零
\item 当\(x>\abs{N}\)时\(f'(x)\)及\(F'(x)\)都存在,且\(F'(x)\neq0\)
\item \(\lim_{x\to\infty}\frac{f'(x)}{F'(x)}\)存在(或为无穷大)
\end{enumerate}


则
\begin{equation*}
\lim_{x\to\infty}\frac{f(x)}{F(x)}=\lim_{x\to\infty}\frac{f'(x)}{F'(x)}
\end{equation*}
\end{theorem}
\subsection{泰勒公式}
\label{sec:orge737c53}
\begin{theorem}[泰勒中值定理]
如果函数\(f(x)\)在\(x_0\)处具有\(n\)阶导数,那么存在\(x_0\)的一个邻域,对于该
邻域内的任一\(x\),有
\begin{equation*}
f(x)=f(x_0)+f'(x_0)(x-x_0)+\frac{f''(x_0)}{2!}(x-x_0)^2+\cdots+\frac{f^{(n)}(x_0)}{n!}(x-x_0)^n+R_n(x)
\end{equation*}
其中
\begin{equation*}
R_n(x)=o((x-x_0)^n)
\end{equation*}
\end{theorem}

\begin{proof}
记\(R_n(x)=f(x)-p_n(x)\),则
\begin{equation*}
R_n(x_0)=R'_n(x_0)=R''_n(x_0)=\cdots=R_n^{(n)}(x_0)=0
\end{equation*}
由于\(f(x)\)在\(x_0\)处有\(n\)阶导数,因此\(f(x)\)必在\(x_0\)的某邻域内有
\(n-1\)阶导数,反复洛必达
\begin{align*}
\lim_{x\to x_0}\frac{R_n(x)}{(x-x_0)^n}&=
\lim_{x\to x_0}\frac{R'_n(x)}{n(x-x_0)^{n-1}}=
\lim_{x\to x_0}\frac{R_n''(x)}{n(n-1)(x-x_0)^{n-2}}\\
&=\cdots=\lim_{x\to x_0}\frac{R^{(n-1)}_n(x)}{n!(x-x_0)}\\
&=\frac{1}{n!}\lim_{x\to x_0}\frac{R_n^{(n-1)}(x)-R_n^{(n-1)}(x_0)}{x-x_0}\\
&=\frac{1}{n!}R_n^{(n)}(x_0)=0
\end{align*}
\end{proof}

\begin{theorem}[泰勒中值定理2]
如果函数\(f(x)\)在\(x_0\)的某个邻域\(U(x_0)\)内具有\((n+1)\)阶导数,那么对任
一\(x\in U(x_0)\),有
\begin{align*}
f(x)&=f(x_0)+f'(x_0)(x-x_0)+\frac{f''(x_0)}{2!}(x-x_0)^2+\cdots+\\
&\frac{f^{(n)}(x_0)}{n!}(x-x_0)^n+R_n(x)
\end{align*}
其中
\begin{equation*}
R_n(x)=\frac{f^{(n+1)}(\xi)}{(n+1)!}(x-x_0)^{n+1},\quad (\xi\in U(x_0,\abs{x-x_0}))
\end{equation*}
\end{theorem}

\(R_n(x)\)的表达式称为 \textbf{拉格朗日余项}

\textbf{麦克劳林公式}
\begin{equation*}
f(x)=f(0)+f'(0)x+\cdots+\frac{f^{(n)}(0)}{n!}x^n+o(x^n)
\end{equation*}

\begin{align*}
e^x&=1+x+\frac{x^2}{2!}+\cdots+\frac{x^n}{n!}+\frac{e^{\theta x}}{(n+1)!}x^{n+1},(0<\theta<1)\\
\sin x&=x-\frac{x^3}{3!}+\frac{x^5}{5!}-\cdots+(-1)^{m-1}\frac{x^{2m-1}}{(2m-1)!}+R_{2m}\\
\cos x&=1-\frac{1}{2!}x^2+\frac{1}{4!}x^4-\cdots+(-1)^m\frac{x^{2m}}{(2m)!}+R_{2m+1}
\end{align*}
\subsection{函数的单调性与曲线的凹凸性}
\label{sec:org6170d90}
\begin{theorem}[]
设函数\(y=f(x)\)在\([a,b]\)上连续,在\((a,b)\)内可导
\begin{enumerate}
\item 如果在\((a,b)\)内\(f'(x)\ge0\),且等号仅在有限多个点处成立,那么函数
\(y=f(x)\)在\([a,b]\)上单调增加
\item 如果在\((a,b)\)内\(f'(x)\le0\),且等号仅在有限多个点处成立,那么函数
\(y=f(x)\)在\([a,b]\)上单调减少
\end{enumerate}
\end{theorem}

\begin{definition}[]
设\(f(x)\)在区间\(I\)上连续,如果对\(I\)上任意两点\(x_1,x_2\)恒有
\begin{equation*}
f(\frac{x_1+x_2}{2})<\frac{f(x_1)+f(x_2)}{2}
\end{equation*}
那么称\(f(x)\)在\(I\)上的 \textbf{图形是(向上)凹的(或凹弧)} ;如果恒有
\begin{equation*}
f(\frac{x_1+x_2}{2})>\frac{f(x_1)+f(x_2)}{2}
\end{equation*}
那么称\(f(x)\)在\(I\)上的 \textbf{图形是(向上)凸的(或凸弧)}
\end{definition}

\begin{theorem}[]
设\(f(x)\)在\([a,b]\)上连续,在\((a,b)\)内具有一阶和二阶导数,那么
\begin{enumerate}
\item 若在\((a,b)\)内\(f''(x)>0\),则\(f(x)\)在\([a,b]\)上的图形是凹的
\item 若在\((a,b)\)内\(f''(x)<0\),则\(f(x)\)在\([a,b]\)上的图形是凸的
\end{enumerate}
\end{theorem}

\begin{proof}
if \(f''(x)>0\), suppose \(x_1,x_2\in[a,b]\) and \(x_1<x_2\). Let
\(x_0=\frac{x_1+x_2}{2}\) and \(h=x_2-x_0=x_0-x_1\). Hence we have
\begin{align*}
&f(x_0+h)-f(x_0)=f'(x_0+\theta_1 h)h\\
&f(x_0)-f(x_0-h)=f'(x_0-\theta_2h)h
\end{align*}
where \(0<\theta_1,\theta_2<1\). By substraction
\begin{equation*}
f(x_0+h)+f(x_0-h)-2f(x_0)=[f'(x_0+\theta_1h)-f'(x_0-\theta_2h)]h
\end{equation*}
and
\begin{equation*}
[f'(x_0+theta_1h)-f'(x_0-\theta_2h)]h=f''(\xi)(\theta_1+\theta_2)h^2
\end{equation*}
where \(x_0-\theta_2h<\xi<x_0+\theta_1h\). Since \(f''(\xi)>0\), we have
\begin{equation*}
f(x_0+h)+f(x_0-h)-2f(x_0)>0
\end{equation*}
hence
\begin{equation*}
\frac{f(x_0+h)+f(x_0-h)}{2}>f(x_0)
\end{equation*}
\end{proof}


设\(y=f(x)\)在区间\(I\)上连续,\(x_0\)是\(I\)内的点,如果曲线\(y=f(x)\)在经过
点\((x_0,f(x_0))\)时,曲线的凹凸性改变,那么就称点\((x_0,f(x_0))\)为这曲线的
\textbf{拐点}
\subsection{函数的极值与最大值最小值}
\label{sec:orgdc18214}
\begin{definition}[]
设函数\(f(x)\)在点\(x_0\)的某邻域\(U(x_0)\)内有定义,如果对于去心邻域
\(\interior{U}(x_0)\)内的任一\(x\),有
\begin{equation*}
f(x)<f(x_0)\quad(\text{or } f(x)>f(x_0))
\end{equation*}
那么就称\(f(x_0)\)是函数\(f(x)\)的一个 \textbf{极大值(极小值)}
\end{definition}


\begin{theorem}[必要条件]
设函数\(f(x)\)在\(x_0\)处可导,且在\(x_0\)处取得极值,则\(f'(x_0)=0\)
\end{theorem}

\begin{theorem}[第一充分条件]
设函数\(f(x)\)在\(x_0\)处连续,且在\(x_0\)的某去心邻域\(\interior{U}(x_0,\delta)\)
内可导
\begin{enumerate}
\item 若\(x\in(x_0-\delta,x_0)\)时,\(f'(x)>0\),而\(x\in(x_0,x_0+\delta)\)时,
\(f'(x)<0\),则\(f(x)\)在\(x_0\)处取得极大值
\item 若\(x\in(x_0-\delta,x_0)\)时,\(f'(x)<0\),而\(x\in(x_0,x_0+\delta)\)时,
\(f'(x)>0\),则\(f(x)\)在\(x_0\)处取得极小值
\item 若\(x\in\interior{U}(x_0,\delta)\)时,\(f'(x)\)的符号保持不变,则\(f(x)\)在
\(x_0\)处没有极值
\end{enumerate}
\end{theorem}

\begin{theorem}[第二充分条件]
设函数\(f(x)\)在\(x_0\)处具有二阶导数且\(f'(x_0)=0\),\(f''(x_0)\neq0\),则
\begin{enumerate}
\item 当\(f''(x_0)<0\)时,函数\(f(x)\)在\(x_0\)处取得极大值
\item 当\(f''(x_0)>0\)时,函数\(f(x)\)在\(x_0\)处取得极小值
\end{enumerate}
\end{theorem}
\subsection{曲率}
\label{sec:orgce9de6c}
\begin{center}
\includegraphics[width=.9\linewidth]{/media/wu/file/stuuudy/notes/images/miscellaneous/arc.png}
\end{center}
设函数\(f(x)\)在区间\((a,b)\)内具有连续导数,在曲线\(y=f(x)\)上取固定点
\(M_0(x_0,y_0)\)作为度量弧长的几点,并规定依\(x\)增大的方向作为曲线的争相,对
曲线上任一点\(M(x,y)\),规定有向弧度\(\arc{M_0M}\)的值\(s\)(简称为弧)如下:
\(s\)的绝对值的等于这弧段的长度,当有向弧段\(\arc{M_0M}\)的方向与曲线的正向一
致时,\(s>0\),相反时\(s<0\)

\begin{equation*}
\Delta s=\arc{M_0M'}-\arc{M_0M}=\arc{MM'}
\end{equation*}
于是
\begin{align*}
(\frac{\Delta s}{\Delta x})^2&=(\frac{\arc{MM'}}{\Delta x})^2=
(\frac{\arc{MM'}}{\abs{MM'}})^2\cdot\frac{\abs{MM'}^2}{(\Delta x)^2}\\
&=(\frac{\arc{MM'}}{\abs{MM'}})^2\cdot\frac{(\Delta x)^2+(\Delta y)^2}{(\Delta x)^2}\\
&=(\frac{\arc{MM'}}{\abs{MM'}})^2[1+(\frac{\Delta y}{\Delta x})^2]
\end{align*}
因此
\begin{equation*}
\frac{\Delta s}{\Delta x}=\pm\sqrt{(\frac{\arc{MM'}}{\abs{MM'}})^2\cdot[1+(\frac{\Delta y}{\Delta x})^2]}
\end{equation*}
令\(\Delta x\to0\)取极限,由于\(\Delta x\to0\)时,\(M'\to M\),这时弧的长度与弦的长度之
比的极限等于1,即
\begin{equation*}
\lim_{M'\to M}\frac{\abs{\arc{MM'}}}{\abs{MM'}}=1
\end{equation*}
又
\begin{equation*}
\lim_{\Delta x\to0}\frac{\Delta y}{\Delta x}=y'
\end{equation*}
因此
\begin{equation*}
\frac{ds}{dx}=\pm\sqrt{1+y'^2}
\end{equation*}
由于\(s=s(x)\)是单调增加函数,于是有
\begin{equation*}
ds=\sqrt{1+y'^2}dx
\end{equation*}

\begin{center}
\includegraphics[width=.9\linewidth]{/media/wu/file/stuuudy/notes/images/miscellaneous/ArcDegree.png}
\end{center}

在曲线\(C\)上选定一点\(M_0\)作为度量弧\(s\)的基点,设曲线上点\(M\)对应与弧
\(s\),在点\(M\)处切线的倾角为\(\alpha\),曲线上另外一点\(M'\)对应于弧
\(s+\Delta s\),在点\(M'\)处切线的倾角为\(\alpha+\Delta \alpha\),则弧段
\(\arc{MM'}\)的长度为\(\abs{\Delta s}\)

我们用比值\(\abs{\frac{\Delta \alpha}{\Delta s}}\)来表达弧段\(\arc{MM'}\)的平均弯曲程度,叫
做弧段\(\arc{MM'}\)的 \textbf{平均曲率} ,并记作\(\bbar{K}\)

\begin{center}
\includegraphics[width=.9\linewidth]{/media/wu/file/stuuudy/notes/images/miscellaneous/ArcCircle.png}
\end{center}

设圆的半径是\(a\),\(\angle MDM'=\frac{\Delta s}{a}\),因此
\begin{equation*}
\frac{\Delta\alpha}{\Delta s}=\frac{\frac{\Delta s}{a}}{\Delta s}=\frac{1}{a}
\end{equation*}
从而
\begin{equation*}
K=\abs{\frac{d\alpha}{d s}}=\frac{1}{a}
\end{equation*}

设曲线的直角坐标方程是\(y=f(x)\),且\(f(x)\)具有二阶导数,因为
\(\tan\alpha=y'\),所以
\begin{gather*}
\sec^2\alpha\frac{d\alpha}{dx}=y''\\
\frac{d\alpha}{dx}=\frac{y''}{1+\tan^2\alpha}=\frac{y''}{1+y'^2}
\end{gather*}
于是
\begin{equation*}
d\alpha=\frac{y''}{1+y'^2}dx
\end{equation*}
又因为
\begin{equation*}
ds=\sqrt{1+y'^2}dx
\end{equation*}
因此
\begin{equation*}
K=\frac{\abs{y''}}{(1+y'^2)^{3/2}}
\end{equation*}


\begin{center}
\includegraphics[width=.9\linewidth]{/media/wu/file/stuuudy/notes/images/miscellaneous/CurvatureCircle.png}
\end{center}
设曲线\(y=f(x)\)在点\(M(x,y)\)处的曲率为\(K(K\neq0)\),在点\(M\)处的曲线的法
线上,在凹的一侧取一点\(D\),使\(\abs{DM}=\frac{1}{K}=\rho\),以\(D\)为圆心,
\(\rho\) 为半径作圆,这个圆叫做曲线在点\(M\)处的 \textbf{曲率圆} ,\(D\)为 \textbf{曲率中心} , \(\rho\) 为曲
率半径
\section{不定积分}
\label{sec:orgebe77b0}
\section{Index}
\label{sec:org6707bf0}
\renewcommand{\indexname}{}
\printindex
\end{document}
