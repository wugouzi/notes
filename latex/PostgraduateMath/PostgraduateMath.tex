% Created 2020-08-29 六 21:00
% Intended LaTeX compiler: pdflatex
\documentclass[11pt]{article}
\usepackage[utf8]{inputenc}
\usepackage[T1]{fontenc}
\usepackage{graphicx}
\usepackage{grffile}
\usepackage{longtable}
\usepackage{wrapfig}
\usepackage{rotating}
\usepackage[normalem]{ulem}
\usepackage{amsmath}
\usepackage{textcomp}
\usepackage{amssymb}
\usepackage{capt-of}
\usepackage{hyperref}
\usepackage{minted}
% TIPS
% \substack{a\\b} for multiple lines text





% pdfplots will load xolor automatically without option
\usepackage[dvipsnames]{xcolor}

\usepackage{forest}
% two-line text in node by [two \\ lines]
% \begin{forest} qtree, [..] \end{forest}
\forestset{
  qtree/.style={
    baseline,
    for tree={
      parent anchor=south,
      child anchor=north,
      align=center,
      inner sep=1pt,
    }}}
%\usepackage{flexisym}
% load order of mathtools and mathabx, otherwise conflict overbrace

\usepackage{mathtools}
%\usepackage{fourier}
\usepackage{pgfplots}
\usepackage{amsthm, mathabx,  amsmath, commath}
\usepackage{amsfonts}

\usepackage{empheq}
\usepackage{tikz}
\usetikzlibrary{arrows.meta}
\usepackage[most]{tcolorbox}

\newtheorem{theorem}{Theorem}[section]
\newtheorem{definition}{Definition}[section]
\newtheorem{corollary}{Corollary}[section]
\newtheorem{example}{Example}[section]
\newtheorem{lemma}{Lemma}[section]
\newtheorem{proposition}{Proposition}[section]

\newcommand{\bl}[1] {\boldsymbol{#1}}
\newcommand{\Wt}[1] {\stackrel{\sim}{\smash{#1}\rule{0pt}{1.1ex}}}
\newcommand{\wt}[1] {\widetilde{#1}}


%For boxed texts in align, use Aboxed{}
%otherwise use boxed{}

\DeclareMathSymbol{\widehatsym}{\mathord}{largesymbols}{"62}
\newcommand\lowerwidehatsym{%
  \text{\smash{\raisebox{-1.3ex}{%
    $\widehatsym$}}}}
\newcommand\fixwidehat[1]{%
  \mathchoice
    {\accentset{\displaystyle\lowerwidehatsym}{#1}}
    {\accentset{\textstyle\lowerwidehatsym}{#1}}
    {\accentset{\scriptstyle\lowerwidehatsym}{#1}}
    {\accentset{\scriptscriptstyle\lowerwidehatsym}{#1}}
}

\usepackage{graphicx}
    
% text on arrow for xRightarrow
\makeatletter
%\newcommand{\xRightarrow}[2][]{\ext@arrow 0359\Rightarrowfill@{#1}{#2}}
\makeatother


\def \bx {\boldsymbol{x}}
\def \ba {\boldsymbol{a}}
\def \bI {\boldsymbol{I}}
\def \bt {\boldsymbol{t}}
\def \bb {\boldsymbol{b}}
\def \bA {\boldsymbol{A}}
\def \bX {\boldsymbol{X}}
\def \bu {\boldsymbol{u}}
\def \bS {\boldsymbol{S}}
\def \bZ {\boldsymbol{Z}}
\def \bz {\boldsymbol{z}}
\def \by {\boldsymbol{y}}
\def \bw {\boldsymbol{w}}
\def \bT {\boldsymbol{T}}
\def \bS {\boldsymbol{S}}
\def \bm {\boldsymbol{m}}
\def \bW {\boldsymbol{W}}
\def \bY {\boldsymbol{Y}}
\def \bH {\boldsymbol{H}}
\def \blambda {\boldsymbol{\lambda}}
\def \bPhi {\boldsymbol{\Phi}}
\def \btheta {\boldsymbol{\theta}}
\def \bmu {\boldsymbol{\mu}}
\def \bphi {\boldsymbol{\phi}}
\def \bSigma {\boldsymbol{\Sigma}}
\def \lb {\left\{}
\def \rb {\right\}}
\def \caln {\mathcal{N}}
\def \dissum {\displaystyle\Sigma}
\def \dispro {\displaystyle\prod}
\def \E {\mathbb{E}}
\def \Q {\mathbb{Q}}
\def \V {\mathbb{V}}
\def \R {\mathbb{R}}
\def \calq {\mathcal{Q}}
\def \calg {\mathcal{G}}
\def \caln {\mathcal{N}}
\def \calr {\mathcal{R}}
\def \calm {\mathcal{M}}
\def \calc {\mathcal{C}}
\def \bcup {\bigcup}

\usepackage[UTF8]{ctex}
\author{Qi'ao Chen}
\date{\today}
\title{Postgraduate Math}
\hypersetup{
 pdfauthor={Qi'ao Chen},
 pdftitle={Postgraduate Math},
 pdfkeywords={},
 pdfsubject={},
 pdfcreator={Emacs 26.3 (Org mode 9.4)}, 
 pdflang={English}}
\begin{document}

\maketitle
\tableofcontents \clearpage
\section{函数与极限}
\label{sec:orga117e8a}
\subsection{映射与函数}
\label{sec:orga4f7a06}
\begin{proposition}[]
Suppose \(f(x)\)'s domain is \((-l,l)\), then there is odd function
\(f_o(x)\) and even function \(f_e(x)\) on \((-l,l)\) s.t.
\begin{equation*}
f(x)=g(x)+h(x)
\end{equation*}
\end{proposition}

\begin{proof}
\begin{equation*}
f_e(x)=\frac{f(x)+f(-x)}{2}\quad
f_o(x)=\frac{f(x)-f(-x)}{x}
\end{equation*}
\end{proof}

基本初等函数
\begin{itemize}
\item 幂函数: \(y=x^\mu\) (\(\mu\in\R\) is a constant)
\item 指数函数:\(y=a^x\) (\(a\iffalse<\fi>0\) and \(a\neq1\))
\item 对数函数:\(y=\log_ax\) (\(a>0\) and \(a\neq1\))
\item 三角函数:\(y=\sin x,\cos x,\tan x\)
\item 反三角函数:\(y=\arcsin x,\arccos x,\arctan x\)
\end{itemize}
\subsection{数列的极限}
\label{sec:org16c4bb5}
\begin{definition}[]
suppose \(\{x_n\}\) is a sequence, if there is a constant \(a\) for any
positive \(\epsilon\), there is a positive integer \(N\) s..t if \(n>N\), then
\begin{equation*}
\abs{x_n-a}<\epsilon
\end{equation*}
always holds, then \(a\) is called the limit of \(\{x_n\}\), or \(\{x_n\}\)
converges to \(a\), written as
\begin{equation*}
\lim_{n\to\infty}x_n=a
\end{equation*}
or
\begin{equation*}
x_n\to a(n\to \infty)
\end{equation*}
\end{definition}

\begin{theorem}[极限的唯一性]
如果数列\(\{x_n\}\)收敛,那么它的极限唯一
\end{theorem}

\begin{proof}
假设同时有\(x_n\to a\)及\(x_n\to b\),且\(a<b\),取\(\epsilon=\frac{b-a}{2}\),
因为\(\lim_{n\to\infty}x_n=a\),故存在正整数\(N_1\),当\(n>N_1\)时,
\begin{equation}
\abs{x_n-a}<\frac{b-a}{2}\label{eq2-2}
\end{equation}
同理有当\(n>N_2\)时
\begin{equation}
\abs{x_n-b}<\frac{b-a}{2}\label{eq2-3}
\end{equation}
取\(N=\max\{N_1,N_2\}\),由\eqref{eq2-2} 有\(x_n<\frac{a+b}{2}\),由
\eqref{eq2-3} 有\(x_n>\frac{a+b}{2}\),矛盾
\end{proof}

\begin{theorem}[收敛数列的有界性]
如果数列\(\{x_n\}\)收敛,那么数列\(\{x_n\}\)一定有界
\end{theorem}

\begin{proof}
因为数列\(\{x_n\}\)收敛,设\(\lim_{n\to\infty}x_n=a\),对于\(\epsilon=1\),存
在正整数\(N\),当\(n>N\)时有
\begin{equation*}
\abs{x_n-a}<1
\end{equation*}
于是当\(n>N\)时
\begin{equation*}
\abs{x_n}=\abs{x_n-a+a}\le\abs{x_n-a}+\abs{a}<1+\abs{a}
\end{equation*}
取\(M=\max\{\abs{x_1},\dots,\abs{x_N},1+\abs{a}\}\),那么数列\(\{x_n\}\)中的
一切\(x_n\)都满足不等式
\begin{equation*}
\abs{x_n}\le M
\end{equation*}
\end{proof}

\begin{theorem}[收敛数列的保号性]
如果\(\lim_{n\to\infty}x_n=a\)且\(a>0\)(或\(a<0\)),那么存在正整数\(N\),当
\(n>N\)时,都有\(x_n>0\)(或\(x_n<0\))
\end{theorem}

\begin{proof}
Suppose \(a>0\), let \(\epsilon=\frac{a}{2}>0\), then there is \(N\) for
\(n>N\) s.t.
\begin{equation*}
\abs{x_n-a}<\frac{a}{2}
\end{equation*}
Hence
\begin{equation*}
x_n>a-\frac{a}{2}=\frac{a}{2}>0
\end{equation*}
\end{proof}

\begin{corollary}[]
如果数列\(\{x_n\}\)从某项起有\(x_n\ge0\)(或\(x_n\le0\)),且
\(\lim_{n\to\infty}x_n=a\),那么\(a\ge0\)(或\(a\le0\))
\end{corollary}

在数列\(\{x_n\}\)中任意抽取无限多项并保持这些项在原数列\(\{x_n\}\)中的先后次
序,这样得到的一个数列称为原数列\(\{x_n\}\)的 \textbf{子数列}

\begin{theorem}[收敛数列与其子数列的关系]
如果数列\(\{x_n\}\)收敛于\(a\),那么它的任一子数列也收敛,且极限也是\(a\)
\end{theorem}

\begin{proof}
设数列\(\{x_{n_k}\}\)是数列\(\{x_n\}\)的任一子数列

由于\(\lim_{n\to\infty}x_n=a\),故对任意\(\epsilon>0\),存在正整数\(N\)当\(n>N\)时,
\(\abs{x_n-a}<\epsilon\)

取\(K=N\),则当\(k>K\)时,\(n_k>n_K=n_N\ge N\),于是
\(\abs{x_{n_k}-a}<\epsilon\),因此\(\lim_{k\to\infty}x_{n_k}=a\)
\end{proof}
\subsection{函数的极限}
\label{sec:orgd57f581}
\subsubsection{函数极限的定义}
\label{sec:org6393676}
\begin{definition}[]
设函数\(f(x)\)在点\(x_0\)的某一去心邻域内有定义,如果存在常数\(A\)对于任一给
定的正数 \(\epsilon\) 总存在正数 \(\delta\) 使得当\(x\) 满足不等式\(0<\abs{x-x_0}<\delta\)时,对
应的函数值 \(f(x)\) 都满足不等式
\begin{equation*}
\abs{f(x)-A}<\epsilon
\end{equation*}
那么常数\(A\)就叫做 \textbf{函数\(f(x)\)当\(x\to x_0\)时的极限} ,记作
\begin{equation*}
\lim_{x\to x_0}f(x)=A \quad\text{ or }\quad
f(x)\to A(\text{when }x\to x_0)
\end{equation*}
\end{definition}

\begin{proposition}[]
\(\lim_{x\to1}(2x-1)=1\)
\end{proposition}

\begin{proof}
Since
\begin{equation*}
\abs{f(x)-A}=\abs{2x-2}=2\abs{x-1}
\end{equation*}
for any \(\epsilon>0\), let \(\delta=\epsilon/2\), then if
\begin{equation*}
0<\abs{x-1}<\delta
\end{equation*}
we have
\begin{equation*}
\abs{f(x)-1}=2\abs{x-1}<\epsilon
\end{equation*}
hence
\begin{equation*}
\lim_{x\to1}(2x-1)=1
\end{equation*}
\end{proof}

将\(0<\abs{x-x_0}<\delta\) 改为\(x_0-\delta<x<x_0\),那么\(A\)就叫做函数
\(f(x)\) 当\(x\to x_0\)时的 \textbf{左极限} ,记作
\begin{equation*}
\lim_{x\to x_0^-}f(x)=A\quad\text{ or }\quad
f(x_0^-)=A
\end{equation*}

函数\(f(x)\)当\(x\to x_0\)时极限存在的充分必要条件时左极限及右极限各自存在且
相等

\begin{definition}[]
设函数\(f(x)\)当\(\abs{x}\)大于某一正数时有定义,如果存在常数\(A\)对于任意给定
的正数 \(\epsilon\) 总存在正数 \(X\) 使得当 \(x\) 满足不等式  \(\abs{x}>X\) 时,对应的函
数值满足
\begin{equation*}
\abs{f(x)-A}<\epsilon
\end{equation*}
那么常数 \(A\) 就叫做 \textbf{函数\(f(x)\)当\(x\to\infty\)时的极限} ,记作
\begin{equation*}
\lim_{x\to\infty}f(x)=A \quad\text{ or }\quad
f(x)\to A(\text{when }x\to\infty)
\end{equation*}
\end{definition}
\subsubsection{函数极限的性质}
\label{sec:org09af4b4}
\begin{theorem}[函数极限的唯一性]
如果\(\lim_{x\to x_0}f(x)\)存在,那么这极限唯一
\end{theorem}

\begin{proof}
If \(\lim_{x\to x_0}f(x)=a\) and \(\lim_{x\to x_0}f(x)=b\), let
\(\epsilon=\frac{b-a}{2}\), there is \(\delta_1\) and \(\delta_2\) s.t. for
\(0<\abs{x-x_0}<\delta_1\), \(\abs{f(x)-a}<\frac{b-a}{2}\), and balabala\ldots{}
\end{proof}

\begin{theorem}[函数极限的局部有界性]
如果\(\lim_{x\to x_0}f(x)=A\),那么存在常数\(M>0\)和\(\delta>0\)使得当
\(0<\abs{x-x_0}<\delta\)时,有\(\abs{f(x)}\le M\)
\end{theorem}

\begin{proof}
取\(\epsilon=1\), then there is \(\delta\) for \(0<\abs{x-x_0}<\delta\), we have
\begin{equation*}
\abs{f(x)-A}<1\Rightarrow\abs{f(x)}\le\abs{f(x)-A}+\abs{A}<\abs{A}+1
\end{equation*}
记\(M=\abs{A}+1\)
\end{proof}

\begin{theorem}[函数极限的局部保号性]
如果\(\lim_{x\to x_0}f(x)=A\),且\(A>0\)(或\(A<0\)),那么存在常数\(\delta>0\),使得
当\(0<\abs{x-x_0}<\delta\)时有\(f(x)>0\) (或\(f(x)<0\))
\end{theorem}
\subsection{无穷大与无穷小}
\label{sec:org9cc2d8d}
\begin{definition}[]
如果函数\(f(x)\)当\(x\to x_0\)(或\(x\to\infty\))时的极限为0,那么称\(f(x)\)
为当\(x\to x_0\)(或\(x\to\infty\))时的无穷小
\end{definition}

\begin{theorem}[]
在自变量的同一变化过程\(x\to x_0\)(或\(x\to\infty\))中,函数\(f(x)\)具有极
限\(A\)的充分必要条件是\(f(x)=A+\alpha\),其中 \(\alpha\) 是无穷小
\end{theorem}

\begin{definition}[]
设函数\(f(x)\)在\(x_0\)的某一去心邻域内有定义(或\(abs{x}\)大于某一正数时有定
义),如果对于任一给定的正数\(M\),总存在正数 \(\delta\) ,如果
\(0<\abs{x-x_0}<\delta\) 则
\(\abs{f(x)}>M\)
那么称函数\(f(x)\)是当\(x\to x_0\)(或\(x\to\infty\))时的无穷大
记作
\begin{equation*}
\lim_{x\to x_0}f(x)=\infty
\end{equation*}
\end{definition}

\begin{theorem}[]
在自变量的同一变化过程中,如果\(f(x)\)为无穷大,那么\(\frac{1}{f(x)}\)为无穷
小;反之亦然
\end{theorem}
\subsection{极限运算法则}
\label{sec:orgbb0c199}
\begin{theorem}[]
两个无穷小的和是无穷小
\end{theorem}

\begin{theorem}[]
有界函数与无穷小的乘积是无穷小
\end{theorem}

\begin{corollary}[]
常数与无穷小的乘积时无穷小
\end{corollary}

\begin{corollary}[]
有限个无穷小的乘积是无穷小
\end{corollary}

\begin{theorem}[]
如果\(\lim f(x)=A,\lim g(x)=B\),那么
\begin{enumerate}
\item \(\lim[f(x)\pm g(x)]=\lim f(x)\pm\lim g(x)=A\pm B\)
\item \(\lim[f(x)\cdot g(x)]=\lim f(x)\cdot\lim g(x)=A\cdot B\)
\item 如果\(B\neq0\),则
\begin{equation*}
\lim\frac{f(x)}{g(x)}=\frac{\lim f(x)}{\lim g(x)}=\frac{A}{B}
\end{equation*}
\end{enumerate}
\end{theorem}

\begin{corollary}[]
If \(\lim f(x)\) exists, and \(c\) is a constant, then
\begin{equation*}
\lim[cf(x)]=c\lim f(x)
\end{equation*}
\end{corollary}

\begin{corollary}[]
if \(\lim f(x )\) exists, and \(n\) is a positive integer, then
\begin{equation*}
\lim[f(x)]^n=[\lim f(x)]^n
\end{equation*}
\end{corollary}

\begin{theorem}[]
设有数列\(\{x_n\}\)和\(\{y_n\}\),如果
\begin{equation*}
\lim_{n\to\infty}x_n=A,\quad\lim_{n\to\infty}y_n=B
\end{equation*}
那么
\begin{enumerate}
\item \(lim_{n\to\infty}(x_n\pm y_n)=A\pm B\)
\item \(\lim_{n\to\infty}(x_n\cdot y_n)=A\cdot B\)
\item 当 \(y_n\neq0(n=1,2,\dots)\)且\(B\neq0\)时,\(\lim_{n\to\infty}\frac{x_n}{y_n}=\frac{A}{B}\)
\end{enumerate}
\end{theorem}

\begin{theorem}[]
如果\(\varphi(x)\ge\psi(x)\),而\(\lim\varphi(x)=A,\lim\psi(x)=B\),那么\(A\ge B\)
\end{theorem}

\begin{theorem}[复合函数的极限运算法则]
设函数\(y=f[g(x)]\)是由函数\(u=g(x)\)与函数\(y=f(u)\)复合而成,\(f[g(x)]\)在
点\(x_0\)的某去心邻域内有定义,若\(\lim_{x\to x_0}g(x)=u_0\),\(\lim_{u\to
   u_0}f(u)=A\),且存在\(\delta_0>0\),当\(x\in\interior{U}(x_0,\delta_0)\)时,
有\(g(x)\neq u_0\),则
\begin{equation*}
\lim_{x\to x_0}f[g(x)]=\lim_{u\to u_0}f(u)=A
\end{equation*}
\end{theorem}
\subsection{极限存在准则 两个重要极限}
\label{sec:org151a53a}
\begin{proposition}[]
如果数列\(\{x_n\},\{y_n\},\{z_n\}\)满足
\begin{enumerate}
\item 存在\(n_0\in\N\),当\(n>n_0\)时,有
\begin{equation*}
y_n\le x_n\le z_n
\end{equation*}
\item \(\lim_{n\to\infty}y_n=a,\lim_{n\to\infty}x_n=a\)
\end{enumerate}


那么数列\(\{x_n\}\)的极限存在,且\(\lim_{n\to\infty}x_n=a\)
\end{proposition}

\begin{proposition}[]
if
\begin{enumerate}
\item when \(x\in\interior{U}(x_0,r)\) (or \(\abs{x}>M\))
\begin{equation*}
g(x)\le f(x)\le h(x)
\end{equation*}
\item \(\lim_{\substack{x\to x_0\\x\to\infty}}g(x)=A,\lim_{\substack{x\to
      x_0\\x\to \infty}}h(x)=A\)
\end{enumerate}


那么\(\lim_{\substack{x\to x_0\\x\to\infty}}x_n=A\)
\end{proposition}

\begin{proposition}[]
单调有界数列必有极限
\end{proposition}

\begin{corollary}[]
\(\lim_{x\to\infty}(1+\frac{1}{x})^x\)
\end{corollary}

\begin{proof}
let \(x_n=(1+\frac{1}{n})^n\)
\begin{align*}
x_n&=(1+\frac{1}{n})^n\\
&=1+\frac{n}{1!}\cdot\frac{1}{n}+\frac{n(n-1)}{2!}\cdot\frac{1}{n^2}+
\frac{n(n-1)(n-2)}{3!}\cdot\frac{1}{n^3}+\cdots+\\
&\frac{n(n-1)\dots(n-n+1)}{n!}\cdot\frac{1}{n^n}\\
&=1+1+\frac{1}{2!}(1-\frac{1}{n})+\frac{1}{3!}(1-\frac{1}{n})(1-\frac{2}{n})+\cdots+\\
&\frac{1}{n!}(1-\frac{1}{n})(1-\frac{2}{n})\cdots(1-\frac{n-1}{n})
\end{align*}
\end{proof}
\end{document}
