% Created 2021-07-14 Wed 03:50
% Intended LaTeX compiler: pdflatex
\documentclass[11pt]{article}
\usepackage[utf8]{inputenc}
\usepackage[T1]{fontenc}
\usepackage{graphicx}
\usepackage{grffile}
\usepackage{longtable}
\usepackage{wrapfig}
\usepackage{rotating}
\usepackage[normalem]{ulem}
\usepackage{amsmath}
\usepackage{textcomp}
\usepackage{amssymb}
\usepackage{capt-of}
\usepackage{hyperref}
\graphicspath{{../../books/}}
% TIPS
% \substack{a\\b} for multiple lines text





% pdfplots will load xolor automatically without option
\usepackage[dvipsnames]{xcolor}

\usepackage{forest}
% two-line text in node by [two \\ lines]
% \begin{forest} qtree, [..] \end{forest}
\forestset{
  qtree/.style={
    baseline,
    for tree={
      parent anchor=south,
      child anchor=north,
      align=center,
      inner sep=1pt,
    }}}
%\usepackage{flexisym}
% load order of mathtools and mathabx, otherwise conflict overbrace

\usepackage{mathtools}
%\usepackage{fourier}
\usepackage{pgfplots}
\usepackage{amsthm, mathabx,  amsmath, commath}
\usepackage{amsfonts}

\usepackage{empheq}
\usepackage{tikz}
\usetikzlibrary{arrows.meta}
\usepackage[most]{tcolorbox}

\newtheorem{theorem}{Theorem}[section]
\newtheorem{definition}{Definition}[section]
\newtheorem{corollary}{Corollary}[section]
\newtheorem{example}{Example}[section]
\newtheorem{lemma}{Lemma}[section]
\newtheorem{proposition}{Proposition}[section]

\newcommand{\bl}[1] {\boldsymbol{#1}}
\newcommand{\Wt}[1] {\stackrel{\sim}{\smash{#1}\rule{0pt}{1.1ex}}}
\newcommand{\wt}[1] {\widetilde{#1}}


%For boxed texts in align, use Aboxed{}
%otherwise use boxed{}

\DeclareMathSymbol{\widehatsym}{\mathord}{largesymbols}{"62}
\newcommand\lowerwidehatsym{%
  \text{\smash{\raisebox{-1.3ex}{%
    $\widehatsym$}}}}
\newcommand\fixwidehat[1]{%
  \mathchoice
    {\accentset{\displaystyle\lowerwidehatsym}{#1}}
    {\accentset{\textstyle\lowerwidehatsym}{#1}}
    {\accentset{\scriptstyle\lowerwidehatsym}{#1}}
    {\accentset{\scriptscriptstyle\lowerwidehatsym}{#1}}
}

\usepackage{graphicx}
    
% text on arrow for xRightarrow
\makeatletter
%\newcommand{\xRightarrow}[2][]{\ext@arrow 0359\Rightarrowfill@{#1}{#2}}
\makeatother


\def \bx {\boldsymbol{x}}
\def \ba {\boldsymbol{a}}
\def \bI {\boldsymbol{I}}
\def \bt {\boldsymbol{t}}
\def \bb {\boldsymbol{b}}
\def \bA {\boldsymbol{A}}
\def \bX {\boldsymbol{X}}
\def \bu {\boldsymbol{u}}
\def \bS {\boldsymbol{S}}
\def \bZ {\boldsymbol{Z}}
\def \bz {\boldsymbol{z}}
\def \by {\boldsymbol{y}}
\def \bw {\boldsymbol{w}}
\def \bT {\boldsymbol{T}}
\def \bS {\boldsymbol{S}}
\def \bm {\boldsymbol{m}}
\def \bW {\boldsymbol{W}}
\def \bY {\boldsymbol{Y}}
\def \bH {\boldsymbol{H}}
\def \blambda {\boldsymbol{\lambda}}
\def \bPhi {\boldsymbol{\Phi}}
\def \btheta {\boldsymbol{\theta}}
\def \bmu {\boldsymbol{\mu}}
\def \bphi {\boldsymbol{\phi}}
\def \bSigma {\boldsymbol{\Sigma}}
\def \lb {\left\{}
\def \rb {\right\}}
\def \caln {\mathcal{N}}
\def \dissum {\displaystyle\Sigma}
\def \dispro {\displaystyle\prod}
\def \E {\mathbb{E}}
\def \Q {\mathbb{Q}}
\def \V {\mathbb{V}}
\def \R {\mathbb{R}}
\def \calq {\mathcal{Q}}
\def \calg {\mathcal{G}}
\def \caln {\mathcal{N}}
\def \calr {\mathcal{R}}
\def \calm {\mathcal{M}}
\def \calc {\mathcal{C}}
\def \bcup {\bigcup}

\makeindex
\def \SING {\text{SING}}
\def \tint {\text{int}}
\def \emp {\text{emp}}
\author{Kenneth Kunen}
\date{\today}
\title{Set Theory}
\hypersetup{
 pdfauthor={Kenneth Kunen},
 pdftitle={Set Theory},
 pdfkeywords={},
 pdfsubject={},
 pdfcreator={Emacs 27.2 (Org mode 9.5)}, 
 pdflang={English}}
\begin{document}

\maketitle
\tableofcontents


\section{Background Material}
\label{sec:org887b8f5}
\subsection{The Axioms of Set Theory}
\label{sec:orgfd5cbb7}
We work in predicate logic with \(\call=\{\in\}\).

\textbf{Axiom 1. Extensionality.}
\begin{equation*}
\forall z(z\in x\leftrightarrow z\in y)\to x=y
\end{equation*}
\textbf{Axiom 2. Foundation.}
\begin{equation*}
\exists y(y\in x)\to\exists y(y\in x\wedge\neg\exists z(z\in x\wedge z\in y))
\end{equation*}
\textbf{Axiom 3. Comprehension Scheme}. For each formula, \(\varphi\), without \(y\) free
\begin{equation*}
\exists y\forall x(x\in y\leftrightarrow x\in v\wedge \varphi(x))
\end{equation*}
\textbf{Axiom 4. Pairing.}
\begin{equation*}
\exists z(x\in z\wedge y\in z)
\end{equation*}
\textbf{Axiom 5. Union}
\begin{equation*}
\exists A\forall Y\forall x(x\in Y\wedge Y\in\calf\to x\in A)
\end{equation*}
\textbf{Axiom 6. Replacement Scheme.} For each formula \(\varphi\), without \(B\) free,
\begin{equation*}
\forall x\in A\exists!y\varphi(x,y)\to\exists B\forall x\in A\exists y\in B\varphi(x,y)
\end{equation*}

On the basis of Axioms 1,3,4,5, define \(\subseteq,\emptyset,S,\cap\) and SING(\(x\)) (\(x\) is a singleton) by
\begin{align*}
x\subseteq y&\quad\Leftrightarrow\quad\forall z(z\in x\to z\in y)\\
x=\emptyset&\quad\Leftrightarrow\quad\forall z(z\not\in x)\\
y=S(x)&\quad\Leftrightarrow\quad\forall z(z\in y\Leftrightarrow z\in x\vee z=x)\\
y=v\cap w&\quad\Leftrightarrow\quad\forall x(x\in y\Leftrightarrow x\in v\wedge x\in w)\\
\SING(x)&\quad\Leftrightarrow\quad\exists y\in x\forall z\in x(z=y)
\end{align*}
\textbf{Axiom 7. Infinity.}
\begin{equation*}
\exists x(\emptyset\in x\wedge\forall y\in x(S(y)\in x))
\end{equation*}
\textbf{Axiom 8. Power Set.}
\begin{equation*}
\exists y\forall z(z\subseteq x\to z\in y)
\end{equation*}
\textbf{Axiom 9. Chioce, or AC.}
\begin{equation*}
\emptyset\not\in F\wedge\forall x\in F\forall y\in F(x\neq y\to x\cap y=\emptyset)\to\exists C\forall x\in F(\SING(C\cap x))
\end{equation*}

\begin{itemize}
\item \(ZFC\) = Axioms 1-9. \hspace{1cm} \(ZF\)=Axioms 1-8
\item \(ZC\) and \(Z\) are \(ZFC\) and \(ZF\), respectively, with the Replacement Scheme deleted
\item \(X^{-1}\) denotes \(X\) without Foundation Axiom
\item \(X-P\) denote \(X\) without the Power Set Axiom
\item \(X-inf\) denotes \(X\) without Axiom of Infinity
\end{itemize}
\begin{definition}
BST (``Basic Set Theory'') denotes the axioms of Extensionality, Foundation, Comprehension, Pairing
and Union, plus the disjunction: the Power Set Axiom holds or the Replacement Axioms holds.
\end{definition}

\begin{definition}[]
\(AC^+\) is the statement that every set can be well-ordered
\end{definition}
\subsection{Extensionality, Comprehension, Pairing, Union}
\label{sec:org5ca4434}
\begin{definition}[]
\(\text{int}(v,w,y)\leftrightarrow\forall x(x\in y\leftrightarrow x\in v\wedge x\in w)\)
\end{definition}

Introducing a defined relation such as \(tint(v,w,y)\) requires no justification, although
defining \(v\cap w\) to be \emph{the} \(y\) such that \(\tint(v,w,y)\) \emph{does} require a justification, namely
\begin{lemma}[]
\(\forall v,w\exists!y\tint(v,w,y)\)
\end{lemma}

\begin{proof}
To prove \(\exists y\tint(v,w,y)\), use Comprehension, with \(\varphi\) the formula \(x\in w\):
\begin{equation*}
\forall v,w[\exists y\forall x(x\in y\leftrightarrow x\in v\wedge x\in w)]
\end{equation*}
To prove that \(y\) is unique, observe, from the definition of \(\tint(v,w,y)\), that
\begin{equation*}
\tint(v,w,y_1)\wedge\tint(v,w,y_2)\to\forall x(x\in y_1\leftrightarrow x\in y_2)
\end{equation*}
so that \(y_1=y_2\) by Extensionality
\end{proof}

This justifies:
\begin{definition}[]
\(v\cap w\) is the unique \(y\) s.t. \(\tint(v,w,y)\)
\end{definition}

Before giving a name to an object satisfying some property, we must prove that property really is
held by a unique object

\begin{definition}[]
For any formula \(\varphi(x)\)
\begin{itemize}
\item \(\{x:\varphi(x)\}\) is, informally, called a \textbf{class}
\item if there is a set \(A\) s.t. \(\forall x[x\in A\leftrightarrow\varphi(x)]\), then \(A\) is unique by Extensionality, and we
denote this set by \(\{x:\varphi(x)\}\), and we say that \(\{x:\varphi(x)\}\) exists
\item if there is no such set, then we say that \(\{x:\varphi(x)\}\) doesn't exist, \emph{or} forms a proper class
\item \(\{x\in v:\varphi(x)\}\) abbreviates \(\{x:x\in v\wedge\varphi(x)\}\)
\end{itemize}
\end{definition}

Comprehension asserts that sets of the form \(\{x:x\in v\wedge\varphi(x)\}\) always exists

\begin{definition}[]
Given \(v,w\)
\begin{align*}
&v\cap w:=\{x\in v:x\in w\}\\
&v\setminus w:=\{x\in v:x\not\in w\}
\end{align*}
\end{definition}

\begin{definition}[]
\(\emp(y)\) iff \(\forall x(x\not\in y)\)
\end{definition}

\begin{definition}[]
\(\emptyset\) denotes the (unique) \(y\) s.t. \(\emp(y)\)
\end{definition}

\begin{proof}[\textbf{Justification}]
Fix any \(v\) and form \(y=\{x\in v:x\neq x\}\), which is empty, so \(\exists y\emp(y)\). Then \(y\) is unique
by Extensionality.
\end{proof}

\begin{definition}[]
The \textbf{universe}, \(V:=\{x:x=x\}\)
\end{definition}

\begin{lemma}[]
\(V\) doesn't exists, and neither does \(R:=\{x:x\not\in x\}\)
\end{lemma}

\begin{proof}
For \(R\), we need no axioms at all; if we had an \(R\) s.t. \(\forall x[x\in R\leftrightarrow x\not\in x]\),
then \(R\in R\leftrightarrow R\not\in R\), a contradiction. For \(V\), we use Comprehension; if we had a \(V\)
with \(\forall x[x\in V]\), then we could form \(R\) as \(\{x\in V:x\not\in x\}\)
\end{proof}

\begin{definition}[]
\begin{align*}
&\{x,y\}=\{w:w=x\vee w=y\}\\
&\{x\}=\{x,x\}\\
&\la x,y\ra=(x,y)=\{\{x\},\{x,y\}\}
\end{align*}
\end{definition}

\begin{lemma}[]
\(\la x,y\ra=\la x',y'\ra\to x=x'\wedge y=y'\)
\end{lemma}

\begin{definition}[]
\(0=\emptyset\), \(1=\{0\}=\{\emptyset\}\), \(2=\{0,1\}=\{\emptyset,\{\emptyset\}\}\)
\end{definition}

The axioms so far don't let us construct any sets with more than two elements. To do that we use
the Union Axiom; then we can get \(3=2\cup\{2\}=\{0,1,2\}\). The Union Axiom can be used to justify
infinite unions as well.
\begin{equation*}
\forall\calf\exists A\forall Y\forall x[x\in Y\wedge Y\in\calf\to x\in A]
\end{equation*}

\begin{definition}[]
\(\displaystyle\bigcup\calf=\bigcup_{Y\in\calf}Y=\{x:\exists Y\in\calf(x\in Y)\}\)
\end{definition}
\end{document}
