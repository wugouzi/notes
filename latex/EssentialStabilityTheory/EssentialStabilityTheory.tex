% Created 2021-12-11 Sat 12:14
% Intended LaTeX compiler: pdflatex
\documentclass[11pt]{article}
\usepackage[utf8]{inputenc}
\usepackage[T1]{fontenc}
\usepackage{graphicx}
\usepackage{longtable}
\usepackage{wrapfig}
\usepackage{rotating}
\usepackage[normalem]{ulem}
\usepackage{amsmath}
\usepackage{amssymb}
\usepackage{capt-of}
\usepackage{hyperref}
\graphicspath{{../../books/}}
% TIPS
% \substack{a\\b} for multiple lines text





% pdfplots will load xolor automatically without option
\usepackage[dvipsnames]{xcolor}

\usepackage{forest}
% two-line text in node by [two \\ lines]
% \begin{forest} qtree, [..] \end{forest}
\forestset{
  qtree/.style={
    baseline,
    for tree={
      parent anchor=south,
      child anchor=north,
      align=center,
      inner sep=1pt,
    }}}
%\usepackage{flexisym}
% load order of mathtools and mathabx, otherwise conflict overbrace

\usepackage{mathtools}
%\usepackage{fourier}
\usepackage{pgfplots}
\usepackage{amsthm, mathabx,  amsmath, commath}
\usepackage{amsfonts}

\usepackage{empheq}
\usepackage{tikz}
\usetikzlibrary{arrows.meta}
\usepackage[most]{tcolorbox}

\newtheorem{theorem}{Theorem}[section]
\newtheorem{definition}{Definition}[section]
\newtheorem{corollary}{Corollary}[section]
\newtheorem{example}{Example}[section]
\newtheorem{lemma}{Lemma}[section]
\newtheorem{proposition}{Proposition}[section]

\newcommand{\bl}[1] {\boldsymbol{#1}}
\newcommand{\Wt}[1] {\stackrel{\sim}{\smash{#1}\rule{0pt}{1.1ex}}}
\newcommand{\wt}[1] {\widetilde{#1}}


%For boxed texts in align, use Aboxed{}
%otherwise use boxed{}

\DeclareMathSymbol{\widehatsym}{\mathord}{largesymbols}{"62}
\newcommand\lowerwidehatsym{%
  \text{\smash{\raisebox{-1.3ex}{%
    $\widehatsym$}}}}
\newcommand\fixwidehat[1]{%
  \mathchoice
    {\accentset{\displaystyle\lowerwidehatsym}{#1}}
    {\accentset{\textstyle\lowerwidehatsym}{#1}}
    {\accentset{\scriptstyle\lowerwidehatsym}{#1}}
    {\accentset{\scriptscriptstyle\lowerwidehatsym}{#1}}
}

\usepackage{graphicx}
    
% text on arrow for xRightarrow
\makeatletter
%\newcommand{\xRightarrow}[2][]{\ext@arrow 0359\Rightarrowfill@{#1}{#2}}
\makeatother


\def \bx {\boldsymbol{x}}
\def \ba {\boldsymbol{a}}
\def \bI {\boldsymbol{I}}
\def \bt {\boldsymbol{t}}
\def \bb {\boldsymbol{b}}
\def \bA {\boldsymbol{A}}
\def \bX {\boldsymbol{X}}
\def \bu {\boldsymbol{u}}
\def \bS {\boldsymbol{S}}
\def \bZ {\boldsymbol{Z}}
\def \bz {\boldsymbol{z}}
\def \by {\boldsymbol{y}}
\def \bw {\boldsymbol{w}}
\def \bT {\boldsymbol{T}}
\def \bS {\boldsymbol{S}}
\def \bm {\boldsymbol{m}}
\def \bW {\boldsymbol{W}}
\def \bY {\boldsymbol{Y}}
\def \bH {\boldsymbol{H}}
\def \blambda {\boldsymbol{\lambda}}
\def \bPhi {\boldsymbol{\Phi}}
\def \btheta {\boldsymbol{\theta}}
\def \bmu {\boldsymbol{\mu}}
\def \bphi {\boldsymbol{\phi}}
\def \bSigma {\boldsymbol{\Sigma}}
\def \lb {\left\{}
\def \rb {\right\}}
\def \caln {\mathcal{N}}
\def \dissum {\displaystyle\Sigma}
\def \dispro {\displaystyle\prod}
\def \E {\mathbb{E}}
\def \Q {\mathbb{Q}}
\def \V {\mathbb{V}}
\def \R {\mathbb{R}}
\def \calq {\mathcal{Q}}
\def \calg {\mathcal{G}}
\def \caln {\mathcal{N}}
\def \calr {\mathcal{R}}
\def \calm {\mathcal{M}}
\def \calc {\mathcal{C}}
\def \bcup {\bigcup}

\makeindex
\author{Steven Buechler}
\date{\today}
\title{Essential Stability Theory}
\hypersetup{
 pdfauthor={Steven Buechler},
 pdftitle={Essential Stability Theory},
 pdfkeywords={},
 pdfsubject={},
 pdfcreator={Emacs 27.2 (Org mode 9.6)}, 
 pdflang={English}}
\begin{document}

\maketitle
\tableofcontents

\section{Preliminaries and Notations}
\label{sec:org95f6908}
\begin{exercise}
\label{ex1.1.11}
Let \(\calm\) be a finite model in a language \(L\). Show that
\begin{equation*}
\caln\equiv\calm\Rightarrow\caln\cong\calm
\end{equation*}
\end{exercise}
\section{Constructing Models with Special Properties}
\label{sec:org49cfa31}
\subsection{Prime and Atomic Models}
\label{sec:orgcd10a86}
\begin{proposition}[]
\(T\) complete countable theory
\begin{enumerate}
\item A countable \(\calm\vDash T\) is prime iff \(\calm\) is atomic
\item If \(\calm\) and \(\caln\) are both countable atomic models of \(T\), then \(\calm\cong\caln\)
\end{enumerate}
\end{proposition}

Then our question is: \textbf{does every complete theory have a prime model, or can we find a meaningful}
\textbf{characterization of those which do?}

\begin{examplle}[A countable complete theory with no atomic model]
Let \(L=\{P_i:i<\omega\}\) where each \(P_i\) is a unary relation symbol. Let \(X=2^{<\omega}\). The
theory \(T\) is defined so that for any model \(\calm\vDash T\) and \(s\in X\), the intersection of the
family of sets \(\{P_i(\calm):s(i)=0\}\cup\{M\setminus P_i(\calm):s(i)=1\}\) is nonempty. Let \(P^0_i(v)\) denote the
formula \(P_i(v)\), and \(P_i^1(v)\) the formula \(\neg P_i(v)\).

For \(s\in X\), let \(\varphi_s(v):=\bigwedge_{i<lh(s)}P_i^{s(i)}(v)\) where \(lh\) is the length function, \(\sigma_s:=\exists v\varphi_s(v)\)
and \(T=\{\sigma_s:s\in X\}\). \(T\) is a complete quantifier-eliminable theory.

Thus, if \(\calm\vDash T\) and \(a\in M\), \(\tp(a)\) is implied by \(\{P_i^j(v):\calm\vDash P_i^j(a),i<\omega,j=0,1\}\). We
claim that every complete 1-type in \(T\) is nonisolated. If, to the contrary, \(p\) is an
isolated 1-type, then by the characterization of types just mentioned \(p\) is isolated by
some \(\varphi_s\in p\). However, if \(j=lh(s)\), both \(\exists v(\varphi_s(v)\wedge P_j(v))\)
and \(\exists v(\varphi_s(v)\wedge\neg P_j(v))\) are in \(T\), proving that \(\varphi_s\) does not isolate a complete type
in \(T\). Since \(T\) has no isolated 1-types over \(\emptyset\), no model of \(T\) can be atomic
\end{examplle}

\begin{proposition}[]
Let \(T\) be a countable complete theory. Then \(T\) has a countable atomic model iff the
isolated types of \(T\) are dense
\end{proposition}

\begin{lemma}[]
If \(T\) is a complete theory with \(\abs{S(\emptyset)}<2^{\aleph_0}\) then the isolated types of \(T\) are dense
\end{lemma}

Thus, for a countable complete theory, having fewer than continuum many complete types is
sufficient to guarantee the existence of a prime model.

But this condition is not necessary. Consider \(\Th(\Z,+,1)\). \(\abs{S_1(\emptyset)}=2^{\aleph_0}\). However,
since every element of the model \((\Z,+,1)\) interprets a term of the language, it is an
elementary submodel of any model of \(T\)

\begin{remark}
An algebraic formula is contained in only finitely many complete types in \(T\), each of which
is isolated
\end{remark}

\begin{proof}
\(\varphi\) algebraic and \(\varphi\in p\). Then \(p(\calm)\) is finite

\(p=q\Leftrightarrow p(\calm)=q(\calm)\)
\end{proof}


If \(\calm\) is a model and \(A\subset M\), \(\calm\) is called a \textbf{prime model over} \(A\) if \(\calm_A\) is a prime
model over \(\Th(\calm_A)\). Note that \(\caln\vDash\Th(\calm_A)\) iff \(\caln\equiv\calm\) and there is a elementary
map \(f:A\to\caln\)

\begin{exercise}
\label{ex2.1.2}
Let \(T\) be a complete theory and \(\varphi\) a formula in \(n\) variables which is contained in only
finitely many complete \(n\)-types of \(T\). Show that every complete \(n\)-types containing
\(\varphi\) is isolated
\end{exercise}

\begin{proof}
If there are \(p_1,\dots,p_n\), then there is \(\phi_1,\dots,\phi_n\) s.t. for
any \(\varphi\in q\), \(\phi_i\in q\Leftrightarrow  q=p_i\). Thus \([\phi_i]=\{p_i\}\). Thus for any \(\varphi\), either \([\phi_i\wedge\varphi]\)
or \([\phi_i\vee\neg\varphi]\) is empty. Hence \(\phi_i\) is complete
\end{proof}

\begin{exercise}
\label{ex2.1.3}
Suppose \(\bara\) and \(\barb\) are sequences from a model \(\calm\) which have the same complete
types in \(\calm\) and \(\varphi(v,\bara)\) isolates a complete type over \(\bara\). Show
that \(\varphi(v,\barb)\) isolates a complete type over \(\barb\)
\end{exercise}

\begin{proof}
If \(\varphi(v,\bara)\) isolates \(p(v)=\{\varphi(v,\bara)\}\). First, \(q(v)=\{\varphi(v,\barb)\}\) is a complete
type.

Then \(\varphi(v,\barb)\) isolates it
\end{proof}

\begin{exercise}
\label{ex2.1.4}
Suppose that \(\bara\) and \(\barb\) be finite sequences from the universe of the model \(\calm\).
Prove that \(\tp_{\calm}(\bara\barb)\) is isolated iff \(\tp_{\calm}(\bara/\barb)\) and \(\tp(\barb)\)
are both isolated. Using this fact show that when \(\calm\) is an atomic model and \(\bara\) is a
finite sequence from \(M\), then \(\calm\) is atomic over \(\bara\). Conversely, if \(\calm\) is atomic
over \(\bara\) and \(\tp_{\calm}(\bara)\) is isolated, then \(\calm\) is atomic
\end{exercise}

\begin{proof}
If \(\varphi(\barx,\bary)\) isolates \(\tp_{\calm}(\bara\barb)\), then \(\varphi(\barx,\barb)\)
isolates \(\tp_{\calm}(\bara/\barb)\) and \(\exists\barx\varphi(\barx,\bary)\) isolates \(\tp_{\calm}(\barb)\)

If \(\varphi(\barx,\barb)\) isolates \(\tp_{\calm}(\bara/\barb)\) and \(\psi(\bary)\)
isolates \(\tp(\barb)\). Then \(\psi(\bary)\wedge\varphi(\barx,\bary)\) isolates \(\tp(\bara\barb)\).

For any \(\theta(\barx,\bary)\in\tp(\bara\barb)\). \(\calm\vDash\forall\barx(\varphi(\barx,\barb)\to\theta(\barx,\barb))\).
Hence \(\calm\vDash\forall\bary(\psi(\bary)\to\forall\barx(\varphi(\barx,\bary)\to\theta(\barx,\bary)))\)
\end{proof}

\begin{exercise}
Show that the complete type realized by 1 in \((\Z,+)\) is non-isolated
\end{exercise}

\begin{proof}
\(\tp(1/2)\) is isolated by \(x+x=2\).
\end{proof}

\begin{exercise}
\label{ex2.1.6}
Show that \(\Th(\Z,+e)\) has continuum many complete 1-types over \(\emptyset\)
\end{exercise}

\begin{exercise}
\label{ex2.1.7}
Given an example of a model \(\calm\) containing an element \(a\) which is the only realization
of \(\tp_{\calm}(a)\) in \(\calm\), although this type is not even isolated
\end{exercise}

\begin{proof}
Not isolated means there is no minimum element under \(\subseteq\) in \(\{\varphi(\calm):\varphi\in\tp(a)\}\)
\end{proof}

\begin{exercise}
\label{ex2.1.8}
Let \(\calm\) be a model s.t. the type in \(\calm\) of each tuple from \(M\) is algebraic. Prove
that \(\calm\) is a prime and minimal model of its theory
\end{exercise}

\begin{proof}

\end{proof}
\subsection{Saturated and Homogeneous Models}
\label{sec:orgc93f404}
\begin{proposition}[]
A countable complete theory \(T\) has a saturated countable model iff it is small
\end{proposition}


Let \(T\) be a countable complete theory. We proved that \(T\) has a countable atomic model
when \(\abs{S(\emptyset)}<2^{\aleph_0}\) and \(T\) has a countable saturated model when
when \(S(\emptyset)\) is countable. It is natural to ask \textbf{if there is a countable complete theory}
\textbf{with \(\abs{S(\emptyset)}\) strictly between \(\aleph_0\) and \(2^{\aleph_0}\)}

The Cantor-Bendixson Theorem from point-set topology quickly gives a negative answer: \(S_n(\emptyset)\)
is strictly between \(\aleph_0\) and \(2^{\aleph_0}\)

First, we prove Cantor-Bendixson theorem first from \href{http://ozark.hendrix.edu/\~yorgey/settheory/08-more-real-line.pdf}{here}

\begin{definition}[]
\(a\in X\) is \textbf{isolaed in} \(X\) iff \(\{a\}\) is open. Otherwise \(a\) is a limit point
\end{definition}


\begin{definition}[]
\(X\) is a \textbf{perfect set} iff \(X\) is closed and has no isolated points
\end{definition}

Cantor set is perfect since each point of it is a limit point

\begin{lemma}[]
\label{K1}
If \(P\) is a perfect set and \(I\) is an open interval on \(\R\) s.t. \(I\cap P\neq\emptyset\), then there
exist disjoint closed intervals \(J_0,J_1\subset I\) s.t. \(int(J_0)\cap P\neq\emptyset\) and \(int(J_1)\cap P\neq\emptyset\).
Moreover, we can pick \(J_0\) and \(J_1\) s.t. their lengths are both less than any \(\epsilon>0\)
\end{lemma}

\begin{proof}
Since \(P\) has no isolated points, there must be at least two points \(a_0,a_1\in I\cap P\). Then
pick \(J_0\ni a_0\) and \(J_1\ni a_1\) to be small enough
\end{proof}

\begin{lemma}[]
If \(P\) is a nonempty perfect set, then \(\abs{P}=\fc\)
\end{lemma}

\begin{proof}
We exhibit a one-to-one mapping \(G:2^\omega\to P\)

We build a binary tree. For each \(s\in 2^{<\omega}\), we associate an interval \(I_s\) s.t.
\begin{itemize}
\item \(I_s\) is closed
\item \(I_s\cap P\neq\emptyset\)
\item \(I_{s,b}\subset I_s\)
\item \(I_{s,0}\cap I_{s,1}=\emptyset\)
\item \(\abs{I_s}<1/(\abs{s}+1)\)
\end{itemize}


where \(\abs{I}\) denotes the length of interval \(I\) and \(\abs{s}\) denotes the length of
sequence \(s\)

Let \(\la\ra\) denotes the emptyset sequence, let \(I_{\la\ra}\) be the closure of \(I\cap P\) for some
open interval \(I\) with length at most 1 whose intersection with \(P\) is nonempty. Then by
\ref{K1} choose appropriate \(I_{s,0}\) and \(I_{s,1}\)

Now for all \(f\in 2^{\omega}\), define
\begin{equation*}
G(f)=\bigcap_{i\in\omega}I_{f\uhr i}
\end{equation*}
If we pick elements from each \(I_{f\uhr i}\), then \(G(f)\) is their limit, which is contained
in \(P\) since \(P\) is closed

Suppose \(f,f'\in 2^\omega\) and \(f\neq f'\). Let \(n\in\omega\) be the smallest index s.t. \(f(n)\neq f'(n)\).
Then \(I_{f\uhr n}\cap I_{f'\uhr n}=\emptyset\)
\end{proof}

\begin{theorem}[Cantor-Bendixson]
If \(C\subseteq\R\) is closed and uncountable, then there exists some perfect, nonempty \(P\subseteq C\).
\end{theorem}

\begin{proof}
Let \(C\subseteq\R\) be closed. Define the \textbf{Cantor-Bendixson derivative}
\begin{equation*}
C'=\{a\in C\mid a\text{ is a limit point of }C\}
\end{equation*}
This operation maps closed sets to closed sets, since closed sets in \(\R\) are those which
contain all their limit points, and the derivative is monotone and retains all limit points.
Then define
\begin{align*}
C_0&=C\\
C_{\alpha+1}&=(C_\alpha)'\\
C_\lambda'&=\bigcap_{\beta<\lambda}C_\beta
\end{align*}
Note that \(C_\beta\) is closed for all \(\beta\) by induction

Claim: \(C_\gamma=C_{\gamma+1}\) for some \(\gamma\). For if not, \(C_\alpha\neq C_\beta\) for any \(\alpha\neq\beta\), since \(C\) is
monotone, then \(C_{-}\) would be an injection \(Ord\to\calp(C)\), which is absurd

Note that \(C_\gamma\) is perfect, since it consists solely of limit points and is closed. If \(C_\gamma\neq\emptyset\),
we are done

We claim that \(C_\gamma\) cannot be \(\emptyset\) since this would imply that \(C\)  is countable.
Consider \(C_\beta-C_{\beta+1}\), which contains all the isolated points in \(C_\beta\). That is,
if \(a\in C_\beta-C_{\beta+1}\), there exists an open interval \(I_a\ni a\) s.t. \(C_\beta\cap I_a=\{a\}\). In
particular, we may choose \(I_a\) to be an open interval with rational endpoints

Note that \(I_a\) is distinct; otherwise, at the earliest stage when \(I_a\) arose, it would have
contained more than one point. Therefore we have an injection from \(C\) into the set of
intervals with rational endpoints, which is countable
\end{proof}

\begin{remark}
The above proof shows that every closed set can be decomposed into a perfect subset and a
countable subset.
\end{remark}

\begin{definition}[]
The smallest \(\gamma\) in the above proof for which \(C_\gamma=C_{\gamma+1}\) is called the \textbf{Cantor-Bendixson rank}
of \(C\), and the above proofs shows that \(\gamma<\aleph_1\)
\end{definition}

It can be shown that for every \(\gamma<\aleph_1\), there exists a closed \(C\subseteq\R\) with Cantor-Bendixson
rank \(\gamma\)

\begin{lemma}[]
There are \(2^{\aleph_0}\) perfect sets
\end{lemma}

\begin{proof}
There is an injection from \(\calp(\N)\) to the set of all perfect sets: for each set of naturals,
identify each natural with a small closed interval containing it, and take the union. There are
at most \(2^{\aleph_0}\) perfect sets since there are \(2^{\aleph_0}\) closed sets
\end{proof}

\begin{theorem}[]
There exists a set \(X\) with \(\abs{X}=2^{\aleph_0}=\abs{\R-X}\) s.t. for every perfect
set \(P\), \(P\not\subseteq X\) and \(P\not\subseteq\R-X\)
\end{theorem}

\begin{proof}
Let \((P_\alpha:\alpha<2^{\aleph_0})\) be an enumeration of the perfects sets. Also let \(x_\alpha\) be an
enumeration of \(\R\). Now define \(r_\gamma\)
\end{proof}


Now we come back to book

\begin{definition}[]
Let \(T\) be a complete theory. \(\varphi\) a formula in \(n\) variables
\begin{enumerate}
\item \(CB(\varphi)=-1\) if \(\varphi\) is inconsistent
\item Let \(\Psi_\alpha=\{\psi:CB(\psi)=\beta<\alpha\}\)

\(CB(\varphi)=\alpha\) if \(\{p\in S_n(\emptyset):\varphi\in p\wedge\forall\psi\in\Psi_\alpha(\neg\psi\in p)\}\) is nonempty and finite
\end{enumerate}


For \(p\in S_n(T)\), \(CB(p)\) is
\begin{equation*}
\inf\{CB(\varphi):p\vDash\varphi\}
\end{equation*}
When \(CB(p)=\alpha\) we say that the \textbf{Cantor-Bendixson rank} of \(p\) is \(\alpha\). If there is no such
\(\alpha\), \(CB(p)=\infty\) and say that the Cantor-Bendixson rank of \(p\) does not exist
\end{definition}
\end{document}
