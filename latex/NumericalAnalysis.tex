% Created 2019-03-03 日 13:54
% Intended LaTeX compiler: pdflatex
\documentclass[11pt]{article}
\usepackage[utf8]{inputenc}
\usepackage[T1]{fontenc}
\usepackage{graphicx}
\usepackage{grffile}
\usepackage{longtable}
\usepackage{wrapfig}
\usepackage{rotating}
\usepackage[normalem]{ulem}
\usepackage{amsmath}
\usepackage{textcomp}
\usepackage{amssymb}
\usepackage{capt-of}
\usepackage{hyperref}
\usepackage{xcolor}
\newtheorem{theorem}{Theorem}
\author{gouziwu}
\date{\today}
\title{Numerical Analysis}
\hypersetup{
 pdfauthor={gouziwu},
 pdftitle={Numerical Analysis},
 pdfkeywords={},
 pdfsubject={},
 pdfcreator={Emacs 26.1 (Org mode 9.1.14)}, 
 pdflang={English}}
\begin{document}

\maketitle
\tableofcontents

Analysis


\section{Chap1 Mathematical Preliminaries}
\label{sec:orgbf4170b}
\subsection{1.2 Roundoff Errors and Computer Arithmetic}
\label{sec:org6c0ba58}
\textbf{Truncation Error} : the error involved in using a truncated, or finite, summation to
approximate the sum of an infinite series 

\textbf{Roundoff Error}: the error produced when performing real number calculations.
 It occurs because the arithmetic performed in a machine involves numbers
 with only a finite number of digits. 


Suppose \(y=\textcolor{blue}{0.d_1d_2\dots
    d_k}d_{k+1}d_{k+2}\dots\textcolor{blue}{\times 10^n{}}\) \par
\(fl(y)=\begin{cases} 0.d_1d_2\dots d_k\times 10^n&\quad\text{chopping}\\
    chop(y+5\times 10^{n-(k+1)})=0.\delta_1\delta_2\dots \delta_k\times
    10^n&\quad\text{Rounding}\\\end{cases}\)


\begin{theorem} weafewf \end{theorem}
\end{document}