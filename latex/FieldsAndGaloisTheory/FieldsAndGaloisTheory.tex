% Created 2021-12-28 Tue 21:57
% Intended LaTeX compiler: pdflatex
\documentclass[11pt]{article}
\usepackage[utf8]{inputenc}
\usepackage[T1]{fontenc}
\usepackage{graphicx}
\usepackage{longtable}
\usepackage{wrapfig}
\usepackage{rotating}
\usepackage[normalem]{ulem}
\usepackage{amsmath}
\usepackage{amssymb}
\usepackage{capt-of}
\usepackage{hyperref}
\graphicspath{{../../books/}}
% TIPS
% \substack{a\\b} for multiple lines text





% pdfplots will load xolor automatically without option
\usepackage[dvipsnames]{xcolor}

\usepackage{forest}
% two-line text in node by [two \\ lines]
% \begin{forest} qtree, [..] \end{forest}
\forestset{
  qtree/.style={
    baseline,
    for tree={
      parent anchor=south,
      child anchor=north,
      align=center,
      inner sep=1pt,
    }}}
%\usepackage{flexisym}
% load order of mathtools and mathabx, otherwise conflict overbrace

\usepackage{mathtools}
%\usepackage{fourier}
\usepackage{pgfplots}
\usepackage{amsthm, mathabx,  amsmath, commath}
\usepackage{amsfonts}

\usepackage{empheq}
\usepackage{tikz}
\usetikzlibrary{arrows.meta}
\usepackage[most]{tcolorbox}

\newtheorem{theorem}{Theorem}[section]
\newtheorem{definition}{Definition}[section]
\newtheorem{corollary}{Corollary}[section]
\newtheorem{example}{Example}[section]
\newtheorem{lemma}{Lemma}[section]
\newtheorem{proposition}{Proposition}[section]

\newcommand{\bl}[1] {\boldsymbol{#1}}
\newcommand{\Wt}[1] {\stackrel{\sim}{\smash{#1}\rule{0pt}{1.1ex}}}
\newcommand{\wt}[1] {\widetilde{#1}}


%For boxed texts in align, use Aboxed{}
%otherwise use boxed{}

\DeclareMathSymbol{\widehatsym}{\mathord}{largesymbols}{"62}
\newcommand\lowerwidehatsym{%
  \text{\smash{\raisebox{-1.3ex}{%
    $\widehatsym$}}}}
\newcommand\fixwidehat[1]{%
  \mathchoice
    {\accentset{\displaystyle\lowerwidehatsym}{#1}}
    {\accentset{\textstyle\lowerwidehatsym}{#1}}
    {\accentset{\scriptstyle\lowerwidehatsym}{#1}}
    {\accentset{\scriptscriptstyle\lowerwidehatsym}{#1}}
}

\usepackage{graphicx}
    
% text on arrow for xRightarrow
\makeatletter
%\newcommand{\xRightarrow}[2][]{\ext@arrow 0359\Rightarrowfill@{#1}{#2}}
\makeatother


\def \bx {\boldsymbol{x}}
\def \ba {\boldsymbol{a}}
\def \bI {\boldsymbol{I}}
\def \bt {\boldsymbol{t}}
\def \bb {\boldsymbol{b}}
\def \bA {\boldsymbol{A}}
\def \bX {\boldsymbol{X}}
\def \bu {\boldsymbol{u}}
\def \bS {\boldsymbol{S}}
\def \bZ {\boldsymbol{Z}}
\def \bz {\boldsymbol{z}}
\def \by {\boldsymbol{y}}
\def \bw {\boldsymbol{w}}
\def \bT {\boldsymbol{T}}
\def \bS {\boldsymbol{S}}
\def \bm {\boldsymbol{m}}
\def \bW {\boldsymbol{W}}
\def \bY {\boldsymbol{Y}}
\def \bH {\boldsymbol{H}}
\def \blambda {\boldsymbol{\lambda}}
\def \bPhi {\boldsymbol{\Phi}}
\def \btheta {\boldsymbol{\theta}}
\def \bmu {\boldsymbol{\mu}}
\def \bphi {\boldsymbol{\phi}}
\def \bSigma {\boldsymbol{\Sigma}}
\def \lb {\left\{}
\def \rb {\right\}}
\def \caln {\mathcal{N}}
\def \dissum {\displaystyle\Sigma}
\def \dispro {\displaystyle\prod}
\def \E {\mathbb{E}}
\def \Q {\mathbb{Q}}
\def \V {\mathbb{V}}
\def \R {\mathbb{R}}
\def \calq {\mathcal{Q}}
\def \calg {\mathcal{G}}
\def \caln {\mathcal{N}}
\def \calr {\mathcal{R}}
\def \calm {\mathcal{M}}
\def \calc {\mathcal{C}}
\def \bcup {\bigcup}

\makeindex
\author{J. S. Milne}
\date{\today}
\title{Fields and Galois Theory}
\hypersetup{
 pdfauthor={J. S. Milne},
 pdftitle={Fields and Galois Theory},
 pdfkeywords={},
 pdfsubject={},
 pdfcreator={Emacs 28.0.90 (Org mode 9.6)}, 
 pdflang={English}}
\begin{document}

\maketitle
\tableofcontents

\section{Basic Definitions and Results}
\label{sec:orgb62374b}
\subsection{The characteristic of a field}
\label{sec:org18456c7}
Given a field \(F\) and consider a map
\begin{equation*}
\Z\to F,\quad n\mapsto n\cdot 1_F
\end{equation*}
If the kernel of the map is \(\neq (0)\), so that \(n\cdot 1_F=0\) for some \(n\neq 0\). The smallest
positive such \(n\) will be a prime \(p\) (otherwise \((m\cdot n)\cdot 1_F=(m\cdot 1_F)\cdot (n\cdot 1_F)=0\) there
will be two nonzero elements in \(F\) whose product is zero, but a field is an integral domain)
and \(p\) generates the kernel. Thus the map \(n\mapsto n\cdot 1_F:\Z\to F\) defines an isomorphism
from \(\Z/p\Z\) onto the subring
\begin{equation*}
\{m\cdot 1_F\mid m\in\Z\}
\end{equation*}
of \(F\). In this case, \(F\) contains a copy of \(\F_p\)

A field isomorphic to one of the fields \(\F_2,\F_3,\F_5,\dots,\Q\) is called a \textbf{prime field}. Every field
contains exactly one prime field (as a subfield)

A commutative ring \(R\) is said to have \textbf{characteristic} \(p\) (resp. 0) if it contains a prime
field (as a subring) of characteristic \(p\) (resp. 0). Then the prime field is unique and, by
definition, contains \(1_R\). Thus if \(R\) has characteristic \(p\neq 0\), then \(1_R+\dots+1_R=0\)
(\(p\) terms)

Let \(R\) be a nonzero commutative ring. If \(R\) has characteristic \(p\neq 0\), then
\begin{equation*}
pa:=\underbrace{a+\dots+a}_{p\text{ terms}}=\underbrace{(1_R+\dots+1_R)}_{p\text{ terms}}a=0a=0
\end{equation*}
for all \(a\in R\). Conversely, if \(pa=0\) for all \(a\in R\), then \(R\) has characteristic \(p\)

Let \(R\) be a nonzero commutative ring. The usual proof by induction shows that the binomial
theorem
\begin{equation*}
(a+b)^m=a^m+\binom{m}{1}a^{m-1}b+\binom{m}{2}a^{m-2}b^2+\dots+b^m
\end{equation*}
holds in \(R\). If \(p\) is prime, then it divides
\begin{equation*}
\binom{p}{r}:=\frac{p!}{r!(p-r)!}
\end{equation*}
for all \(r\) with \(1\le r\le p-1\). Therefore, when \(R\) has characteristic \(p\)
\begin{equation*}
(a+b)^p=a^p+b^p\quad\text{ for all }a,b\in R
\end{equation*}
and so the map \(a\mapsto a^p:R\to R\) is a homomorphism of rings (even of \(\F_p\)-algebras). It is
called the \textbf{Frobenius endomorphsim} of \(R\). The map \(a\mapsto a^{p^n}:R\to R\), \(n\ge 1\), is hte
composite of \(n\) copies of the Frobenius endomorphsim, and so it also is a homomorphism.
Therefore
\begin{equation*}
(a_1,\dots,a_m)^{p^n}=a_1^{p^n}+\dots+a_m^{p^n}
\end{equation*}
for all \(a_i\in R\).

When \(F\) is a field, the Frobenius endomorphsim is injective
\subsection{Factoring polynomials}
\label{sec:orgc940b6c}
\begin{proposition}[]
Let \(r\in\Q\) be a root of a polynomial
\begin{equation*}
a_mX^m+a_{m-1}X^{m-1}+\dots+a_0,\quad a_i\in\Z
\end{equation*}
and write \(r=c/d\), \(c,d\in\Z\), \(\gcd(c,d)=1\). Then \(c\mid a_0\) and \(d\mid a_m\)
\end{proposition}

\begin{proof}
\begin{equation*}
a_mc^m+a_{m-1}c^{m-1}d+\dots+a_0d^m=0
\end{equation*}
\(d\mid a_mc^m\) and therefore \(d\mid a_m\). Similarly \(c\mid a_0\)
\end{proof}

\begin{examplle}[]
The polynomial \(f(X)=X^3-3X-1\) is irreducible in \(\Q[X]\) because its only possible roots
are \(\pm 1\) and \(f(1)\neq 0\neq f(-1)\)
\end{examplle}

\begin{proposition}[Gauss's Lemma]
Let \(f(X)\in\Z[X]\). If \(f(X)\) factors nontrivially in \(\Q[X]\), then it factors nontrivially in \(\Z[X]\)
\end{proposition}

\begin{proof}
Let \(f=gh\in\Q[X]\) with \(g,h\) nonconstant. For suitable integers \(m\) and \(n\), \(g_1:=mg\)
and \(h_1:=nh\) have coefficients in \(\Z\), so we have a factoriztion
\begin{equation*}
mnf=g_1\cdot h_1
\end{equation*}
in \(\Z[X]\). If a prime \(p\) divides \(mn\), then looking modulo \(p\), we obtain
\begin{equation*}
0=\bbar{g_1}\cdot\bbar{h_1}\in\F_p[X]
\end{equation*}
Since \(\F_p[X]\) is an integral domain, this implies that \(p\) divides all the coefficients of
at least one of the polynomials \(g_1,h_1\), say \(g_1\), so that \(g_1=pg_2\) for some \(g_2\in\Z[X]\).
Thus we have a factoriztion
\begin{equation*}
(mn/p)f=g_2\cdot h_1\in\Z[X]
\end{equation*}
Continuing in this fashion, we eventually remove all the prime factors of \(mn\).
\end{proof}

\begin{proposition}[]
If \(f\in\Z[X]\) is monic, then every monic factor of \(f\) in \(\Q[X]\) lies in \(\Z[X]\)
\end{proposition}

\begin{proof}
Let \(g\) be a monic factor of \(f\) in \(\Q[X]\), so that \(f=gh\) with \(h\in\Q[X]\) also monic.
Let \(m,n\) be the positive integers with the fewest prime factors s.t. \(mg,nh\in\Z[X]\). As in
the proof of Gauss's Lemma, if a prime \(p\) divides \(mn\), then it divides all the
coefficients of at least one of the polynomials \(mg,nh\), say \(mg\), in which case it
divides \(m\) because \(g\) is monic. Now \(\frac{m}{p}g\in\Z[X]\) which contradicts the definition
of \(m\).
\end{proof}

\begin{proposition}[Eisenstein's Criterion]
Let
\begin{equation*}
f=a_mX^m+\dots+a_0,\quad a_i\in\Z
\end{equation*}
suppose that there is a prime \(p\) s.t.
\begin{enumerate}
\item \(p\nmid a_m\)
\item \(p\mid a_i\)  for \(i=0,\dots,m-1\)
\item \(p^2\nmid a_0\)
\end{enumerate}


Then \(f\) is irreducible in \(\Q[X]\)
\end{proposition}

\begin{proof}
If \(f(X)\) factors nontrivially in \(\Q[X]\), then it factors nontrivially in \(\Z[X]\), say
\begin{equation*}
a_mX^m+\dots+a_0=(b_rX^r+\dots+b_0)(c_sX^s+\dots+c_0)
\end{equation*}
where \(b_i,c_i\in\Z\). Since \(p\), but not \(p^2\), divides \(a_0=b_0c_0\), \(p\) must divide
exactly one of \(b_0,c_0\), say \(b_0\). Now from the equation
\begin{equation*}
a_1=b_0c_1+b_1c_0
\end{equation*}
we see that \(p\mid b_1\), and from the equation
\begin{equation*}
a_2=b_0c_2+b_1c_1+b_2c_0
\end{equation*}
that \(p\mid b_2\). By continuing in this way, we find that \(p\) divides \(b_0,b_1,\dots,b_r\), which
contradicts the condition that \(p\) does not divide \(a_m\)
\end{proof}
\subsection{Extensions}
\label{sec:orgf1e1ff4}
Let \(F\) be a field. A field containing \(F\) is called an \textbf{extension} of \(F\). In other words,
an extension is an \(F\)-algebra whose underlying ring is a field. An extension \(E\) of \(F\)
is, in particular, an \(F\)-vector space, whose dimension is called the \textbf{degree} of \(E\)
over \(F\). It is denoted by \([E:F]\). An extension is \textbf{finite} if its degree is finite.

When \(E\) and \(E'\) are extensions of \(F\), an \textbf{\(F\)-homomorphism} \(E\to E'\) is a
homomorphism \(\varphi:E\to E'\) s.t. \(\varphi(c)=c\) for all \(c\in F\)

\begin{proposition}[Multiplicity of degrees]
Consider fields \(L\supset E\supset F\). Then \(L/F\) is of finite degree iff \(L/E\) and \(E/F\) are both
of finite degree, in which case
\begin{equation*}
[L:F]=[L:E][E:F]
\end{equation*}
\end{proposition}
\subsection{The subring generated by a subset}
\label{sec:orgb143f4b}
Let \(F\) be a subfield of a field \(E\) and let \(S\) be a subset of \(E\). The intersection of
all the subrings of \(E\) containing \(F\) and \(S\) is obviously the smallest subring of \(E\)
containing both \(F\) and \(S\). We call it the subring of \(E\) \textbf{generated by \(F\) and \(S\)}
(\textbf{generated over \(F\) by \(S\)}), and we denote it by \(F[S]\).

\begin{lemma}[]
The ring \(F[S]\) consists of the elements of \(E\) that can be expressed as finite sums of the
form
\begin{equation*}
\sum a_{i_1\cdots i_n}\alpha_1^{i_1}\cdots\alpha_n^{i_n},\quad a_{i_1\cdots i_n}\in F,\quad\alpha_i\in S,\quad i_j\in\N
\end{equation*}
\end{lemma}

\begin{lemma}[]
Let \(R\) be an integral domain containing a subfield \(F\) (as a subring). If \(R\) is
finite-dimensional when regarded as an \(F\)-vector space, then it is a field
\end{lemma}

\begin{proof}
Let \(\alpha\in R\) be nonzero. The map \(h:x\mapsto\alpha x\) is an injective linear map of
finite-dimensional \(F\)-vector spaces, and is therefore surjective. In particular, there is an
element \(\beta\in R\) s.t. \(\alpha\beta=1\)

\(\alpha x=\alpha y\), we need \(R\) to be integral domain to make \(x=y\)

Also for \(f\in R\), we need \(R\) to be a field to make \(\alpha fx=f\alpha x\)

Surjection is trivial
\end{proof}
\subsection{The subfield generated by a subset}
\label{sec:org2173a68}
The intersection of all the subfields of \(E\) containing \(F\) and \(S\) is the smallest
subfield of \(E\) containing both \(F\) and \(S\). We call it the subfield of \(E\) \textbf{generated
by \(F\) and \(S\)}, and we denote it by \(F(S)\), it is the fraction field of \(F[S]\)

An extension \(E\) of \(F\) is \textbf{simple} if \(E=F(\alpha)\) for some \(\alpha\in E\)

Let \(F\) and \(F'\) be subfields of a field \(E\). The intersection of the subfields of \(E\)
containing both \(F\) and \(F'\) is obviously the smallest subfield of \(E\) containing
both \(F\)and \(F\). We call it the \textbf{composite} of \(F\) and \(F'\) in \(E\), and we denote it
by \(F\cdot F'\). It can also be described as the subfield of \(E\) generated over \(F\) by \(F'\),
or the subfield generated over \(F'\) by \(F\)
\begin{equation*}
F(F')=F\cdot F'=F'(F)
\end{equation*}
\subsection{Construction of some extensions}
\label{sec:orgdf95240}
Let \(f(X)\in F(X)\) be a monic polynomial of degree \(m\). Consider the quotient \(F[X]/(f(X))\),
and write \(x\) for the image of \(X\) in \(F[X]/(f(X))\), i.e., \(x=X+(f(X))\)
\begin{enumerate}
\item The map
\begin{equation*}
P(X)\mapsto P(x):F[X]\to F[x]
\end{equation*}
is a homomorphism sending \(f(X)\) to 0, therefore \(f(x)=0\).
\(F[x]=F[X]/(f)\) since for each \(x^n=(X+(f(X))^n)=X^n+(f(X))\).
\item The division algorithm shows that every element \(g\in F[X]/(f)\)  is represented by a unique
polynomial \(r\) of degree \(<m\). Hence each element of \(F[x]\) can be expressed uniquely
as a sum
\begin{equation*}
a_0+a_1x+\dots+a_{m-1}x^{m-1},\quad a_i\in F
\end{equation*}
\item \emph{Now assume that \(f(X)\) is irreducible}. Then every nonzero \(\alpha\in F[x]\) has an inverse, which
can be found as follows. Use 2 to write \(\alpha=g(x)\) with \(g(X)\) a polynomial of
degree \(\le m-1\), and apply Euclid's algorithm in \(F[X]\) to find polynomials \(a(X)\)
and \(b(X)\) s.t.
\begin{equation*}
a(X)f(X)+b(X)g(X)=d(X)
\end{equation*}
with \(d(X)\) the gcd of \(f\) and \(g\). In our case, \(d(X)\) is 1 because \(f(X)\) is
irreducible and \(\deg g(X)<\deg f(X)\). When we replace \(X\) with \(x\), the equality
becomes
\begin{equation*}
b(x)g(x)=1
\end{equation*}
Hence \(b(x)\) is the inverse of \(g(x)\)
\end{enumerate}



We have proved the following statement
\begin{proposition}[]
\label{1.25}
For a monic irreducible polynomial \(f(X)\) of degree \(m\) in \(F[X]\)
\begin{equation*}
F[x]:=F[X]/(f(X))
\end{equation*}
is a field of degree \(m\) over \(F\). Computations in \(F[x]\) come down to computations
in \(F\)
\end{proposition}

Since \(F[x]\) is a field, \(F(x)=F[x]\)


\begin{examplle}[]
Let \(f(X)=X^2+1\in\R[X]\). Then \(\R[x]\) has elements \(a+bx,a,b\in\R\)

We usually write \(i\) for \(x\) and \(\C\) for \(\R[x]\)
\end{examplle}
\subsection{Stem fields}
\label{sec:orgc30db82}
Let \(f\) be a monic irreducible polynomial in \(F[X]\). A pair \((E,\alpha)\) consisting of an
extension \(E\) of \(F\) and an \(\alpha\in E\) is called a \textbf{stem field for} \(f\) if \(E=F[\alpha]\)
and \(f(\alpha)=0\). For example, the pair \((E,\alpha)\) with \(E=F[X]/(f)=F[x]\) and \(\alpha=x\).

Let \((E,\alpha)\) be a stem field, and consider the surjective homomorphism of \(F\)-algebras
\begin{equation*}
g(X)\to g(\alpha):F[X]\to E
\end{equation*}
Its kernel is generated by a nonzero monic polynomial, which divides \(f\), and so must equal
it. Therefore the homomorphism defines an \(F\)-isomorphism
\begin{equation*}
x\mapsto\alpha:F[x]\to E,\quad F[x]=F[X]/(f)
\end{equation*}
In other words, the stem field \((E,\alpha)\) of \(f\) is \(F\)-isomorphic to the standard stem field
\((F[X]/(f),x)\). It follows that every element of a stem field \((E,\alpha)\) for \(f\) can be
written uniquely in the form
\begin{equation*}
a_0+a_1\alpha+\dots+a_{m-1}\alpha^{m-1},\quad a_i\in F,\quad m=\deg(f)
\end{equation*}
and that arithmetic in \(F[\alpha]\) can be performed using the same rules in \(F[x]\).
\subsection{Algebraic and transcendental elements}
\label{sec:org7eec93b}

Let \(F\) be a field. An element \(\alpha\) of an extension \(E\) of \(F\) defines a homomorphism
\begin{equation*}
f(X)\mapsto f(\alpha):F[X]\to E
\end{equation*}
There are two possibilities:
\begin{enumerate}
\item Kernel is \((0)\), so that for \(f\in F[X]\)
\begin{equation*}
f(\alpha)=0\Rightarrow f=0 (\text{in }F[X])
\end{equation*}
In this case we say that \(\alpha\) \textbf{transcendental over} \(F\). The homomorphism \(X\mapsto\alpha\) is an
isomorphism, and it extends to an isomorphism \(F(X)\to F(\alpha)\)
\item The kernel \(\neq(0)\), so that \(g(\alpha)=0\) for some nonzero \(g\in F[X]\). In this case, we say
that \(\alpha\) is \textbf{algebraic over} \(F\).  The polynomials \(g\) s.t. \(g(\alpha)=0\) form a nonzero ideal
in \(F[X]\), which is generated by the monic polynomial \(f\) of least degree
such \(f(\alpha)=0\). We call \(f\) the \textbf{minimal polynomial} of \(\alpha\) over \(F\).

Note that \(F[X]/(f)\cong F[\alpha]\), since the first is a field, so is the second
\end{enumerate}


\begin{examplle}[]
Let \(\alpha\in\C\) be s.t. \(\alpha^3-3\alpha-1=0\). Then \(X^3-3X-1\) is monic, irreducible in \(\Q[X]\) and  has
\(\alpha\) as a root, and so it is the minimal polynomial of \(\alpha\) over \(\Q\). The set \(\{1,\alpha,\alpha^2\}\) is a
basis for \(\Q[\alpha]\) over \(\Q\).
\end{examplle}

An extension \(E\) of \(F\) is \textbf{algebraic} (\(E\) is \textbf{algebraic over} \(F\)) if all elements
of \(E\) are algebraic over \(F\); otherwise it is said to be \textbf{transcendental}

\begin{proposition}[]
Let \(E\supset F\) be fields. If \(E/F\) is finite, then \(E\) is algebraic and finitely generated (as
a field) over \(F\); conversely if \(E\) is generated over \(F\) by a finite set of algebraic
elements, then it is finite over \(F\)
\end{proposition}

\begin{proof}
\(\Rightarrow\). \(\alpha\) of \(E\) is transcendental over \(F\) iff \(1,\alpha,\alpha^2,\dots\) are linearly independent
over \(F\) iff \(F(\alpha)\) is of infinite degree. Thus if \(E\) is finite over \(F\), then every
element of \(E\) is algebraic over \(F\). If \(E\neq F\), then we can pick \(\alpha_1\in E\setminus F\) and
compare \(E\) and \(F[\alpha_1]\)
\end{proof}
\subsection{Algebraically closed fields}
\label{sec:org080c9ce}
Let \(F\) be a field. A polynomial is said to \textbf{split} in \(F[X]\) if it is a product of
polynomials of degree at most 1 in \(F[X]\)

\begin{proposition}[]
\label{1.42}
For a field \(\Omega\), TFAE
\begin{enumerate}
\item Every nonconstant polynomial in \(\Omega[X]\) splits in \(\Omega[X]\)
\item Every nonconstant polynomial in \(\Omega[X]\) has at least one root in \(\Omega\)
\item The irreducible polynomials in \(\Omega[X]\) are those of degree 1
\item Every field of finite degree over \(\Omega\) equals \(\Omega\)
\end{enumerate}
\end{proposition}

\begin{definition}[]
\begin{enumerate}
\item A field \(\Omega\) is \textbf{algebraically closed} if it satisfies the equivalent statements in Proposition \ref{1.42}
\item A field \(\Omega\) is an \textbf{algebraic closure} of a subfield \(F\) if it is algebraically closed and
algebraic over \(F\)
\end{enumerate}
\end{definition}

\begin{proposition}[]
If \(\Omega\) is algebraic over \(F\) and every polynomial \(f\in F[X]\) splits in \(\Omega[X]\), then \(\Omega\) is
algebraically closed
\end{proposition}

\begin{proof}
Let \(f\) be a nonconstant polynomial in \(\Omega[X]\). We know (\ref{1.25}) that \(f\) has a root
\(\alpha\) in some finite extension \(\Omega'\) of \(\Omega\). Set
\begin{equation*}
f=a_nX^n+\dots+a_0,\quad a_i\in\Omega
\end{equation*}
and consider the fields
\begin{equation*}
F\subset F[a_0,\dots,a_n]\subset F[a_0,\dots,a_n,\alpha]
\end{equation*}
Each extension generated by a finite set of algebraic elements, and hence is finite (\ref{1.30})
Therefore \(\alpha\) lies in a finite extension of \(F\) and so is algebraic over \(F\)
\end{proof}
\end{document}
