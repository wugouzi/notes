% Created 2022-01-08 Sat 16:10
% Intended LaTeX compiler: pdflatex
\documentclass[11pt]{article}
\usepackage[utf8]{inputenc}
\usepackage[T1]{fontenc}
\usepackage{graphicx}
\usepackage{longtable}
\usepackage{wrapfig}
\usepackage{rotating}
\usepackage[normalem]{ulem}
\usepackage{amsmath}
\usepackage{amssymb}
\usepackage{capt-of}
\usepackage{hyperref}
\graphicspath{{../../books/}}
% wrong resolution of image
% https://tex.stackexchange.com/questions/21627/image-from-includegraphics-showing-in-wrong-image-size?rq=1

%%%%%%%%%%%%%%%%%%%%%%%%%%%%%%%%%%%%%%
%% TIPS                                 %%
%%%%%%%%%%%%%%%%%%%%%%%%%%%%%%%%%%%%%%
% \substack{a\\b} for multiple lines text
% \usepackage{expl3}
% \expandafter\def\csname ver@l3regex.sty\endcsname{}
% \usepackage{pkgloader}
\usepackage[utf8]{inputenc}

% nfss error
% \usepackage[B1,T1]{fontenc}
\usepackage{fontspec}

% \usepackage[Emoticons]{ucharclasses}
\newfontfamily\DejaSans{DejaVu Sans}
% \setDefaultTransitions{\DejaSans}{}

% pdfplots will load xolor automatically without option
\usepackage[dvipsnames]{xcolor}

%                                                             ┳┳┓   ┓
%                                                             ┃┃┃┏┓╋┣┓
%                                                             ┛ ┗┗┻┗┛┗
% \usepackage{amsmath} mathtools loads the amsmath
\usepackage{amsmath}
\usepackage{mathtools}

\usepackage{amsthm}
\usepackage{amsbsy}

%\usepackage{commath}

\usepackage{amssymb}

\usepackage{mathrsfs}
%\usepackage{mathabx}
\usepackage{stmaryrd}
\usepackage{empheq}

\usepackage{scalerel}
\usepackage{stackengine}
\usepackage{stackrel}



\usepackage{nicematrix}
\usepackage{tensor}
\usepackage{blkarray}
\usepackage{siunitx}
\usepackage[f]{esvect}

% centering \not on a letter
\usepackage{slashed}
\usepackage[makeroom]{cancel}

%\usepackage{merriweather}
\usepackage{unicode-math}
\setmainfont{TeX Gyre Pagella}
% \setmathfont{STIX}
%\setmathfont{texgyrepagella-math.otf}
%\setmathfont{Libertinus Math}
\setmathfont{Latin Modern Math}

 % \setmathfont[range={\smwhtdiamond,\enclosediamond,\varlrtriangle}]{Latin Modern Math}
\setmathfont[range={\rightrightarrows,\twoheadrightarrow,\leftrightsquigarrow,\triangledown,\vartriangle,\precneq,\succneq,\prec,\succ,\preceq,\succeq,\tieconcat}]{XITS Math}
 \setmathfont[range={\int,\setminus}]{Libertinus Math}
 % \setmathfont[range={\mathalpha}]{TeX Gyre Pagella Math}
%\setmathfont[range={\mitA,\mitB,\mitC,\mitD,\mitE,\mitF,\mitG,\mitH,\mitI,\mitJ,\mitK,\mitL,\mitM,\mitN,\mitO,\mitP,\mitQ,\mitR,\mitS,\mitT,\mitU,\mitV,\mitW,\mitX,\mitY,\mitZ,\mita,\mitb,\mitc,\mitd,\mite,\mitf,\mitg,\miti,\mitj,\mitk,\mitl,\mitm,\mitn,\mito,\mitp,\mitq,\mitr,\mits,\mitt,\mitu,\mitv,\mitw,\mitx,\mity,\mitz}]{TeX Gyre Pagella Math}
% unicode is not good at this!
%\let\nmodels\nvDash

 \usepackage{wasysym}

 % for wide hat
 \DeclareSymbolFont{yhlargesymbols}{OMX}{yhex}{m}{n} \DeclareMathAccent{\what}{\mathord}{yhlargesymbols}{"62}

%                                                               ┏┳┓•┓
%                                                                ┃ ┓┃┏┓
%                                                                ┻ ┗┛┗┗

\usepackage{pgfplots}
\pgfplotsset{compat=1.18}
\usepackage{tikz}
\usepackage{tikz-cd}
\tikzcdset{scale cd/.style={every label/.append style={scale=#1},
    cells={nodes={scale=#1}}}}
% TODO: discard qtree and use forest
% \usepackage{tikz-qtree}
\usepackage{forest}

\usetikzlibrary{arrows,positioning,calc,fadings,decorations,matrix,decorations,shapes.misc}
%setting from geogebra
\definecolor{ccqqqq}{rgb}{0.8,0,0}

%                                                          ┳┳┓•    ┓┓
%                                                          ┃┃┃┓┏┏┏┓┃┃┏┓┏┓┏┓┏┓┓┏┏
%                                                          ┛ ┗┗┛┗┗ ┗┗┗┻┛┗┗ ┗┛┗┻┛
%\usepackage{twemojis}
\usepackage[most]{tcolorbox}
\usepackage{threeparttable}
\usepackage{tabularx}

\usepackage{enumitem}
\usepackage[indLines=false]{algpseudocodex}
\usepackage[]{algorithm2e}
% \SetKwComment{Comment}{/* }{ */}
% \algrenewcommand\algorithmicrequire{\textbf{Input:}}
% \algrenewcommand\algorithmicensure{\textbf{Output:}}
% wrong with preview
\usepackage{subcaption}
\usepackage{caption}
% {\aunclfamily\Huge}
\usepackage{auncial}

\usepackage{float}

\usepackage{fancyhdr}

\usepackage{ifthen}
\usepackage{xargs}

\definecolor{mintedbg}{rgb}{0.99,0.99,0.99}
\usepackage[cachedir=\detokenize{~/miscellaneous/trash}]{minted}
\setminted{breaklines,
  mathescape,
  bgcolor=mintedbg,
  fontsize=\footnotesize,
  frame=single,
  linenos}
\usemintedstyle{xcode}
\usepackage{tcolorbox}
\usepackage{etoolbox}



\usepackage{imakeidx}
\usepackage{hyperref}
\usepackage{soul}
\usepackage{framed}

% don't use this for preview
%\usepackage[margin=1.5in]{geometry}
% \usepackage{geometry}
% \geometry{legalpaper, landscape, margin=1in}
\usepackage[font=itshape]{quoting}

%\LoadPackagesNow
%\usepackage[xetex]{preview}
%%%%%%%%%%%%%%%%%%%%%%%%%%%%%%%%%%%%%%%
%% USEPACKAGES end                       %%
%%%%%%%%%%%%%%%%%%%%%%%%%%%%%%%%%%%%%%%

%%%%%%%%%%%%%%%%%%%%%%%%%%%%%%%%%%%%%%%
%% Algorithm environment
%%%%%%%%%%%%%%%%%%%%%%%%%%%%%%%%%%%%%%%
\SetKwIF{Recv}{}{}{upon receiving}{do}{}{}{}
\SetKwBlock{Init}{initially do}{}
\SetKwProg{Function}{Function}{:}{}

% https://github.com/chrmatt/algpseudocodex/issues/3
\algnewcommand\algorithmicswitch{\textbf{switch}}%
\algnewcommand\algorithmiccase{\textbf{case}}
\algnewcommand\algorithmicof{\textbf{of}}
\algnewcommand\algorithmicotherwise{\texttt{otherwise} $\Rightarrow$}

\makeatletter
\algdef{SE}[SWITCH]{Switch}{EndSwitch}[1]{\algpx@startIndent\algpx@startCodeCommand\algorithmicswitch\ #1\ \algorithmicdo}{\algpx@endIndent\algpx@startCodeCommand\algorithmicend\ \algorithmicswitch}%
\algdef{SE}[CASE]{Case}{EndCase}[1]{\algpx@startIndent\algpx@startCodeCommand\algorithmiccase\ #1}{\algpx@endIndent\algpx@startCodeCommand\algorithmicend\ \algorithmiccase}%
\algdef{SE}[CASEOF]{CaseOf}{EndCaseOf}[1]{\algpx@startIndent\algpx@startCodeCommand\algorithmiccase\ #1 \algorithmicof}{\algpx@endIndent\algpx@startCodeCommand\algorithmicend\ \algorithmiccase}
\algdef{SE}[OTHERWISE]{Otherwise}{EndOtherwise}[0]{\algpx@startIndent\algpx@startCodeCommand\algorithmicotherwise}{\algpx@endIndent\algpx@startCodeCommand\algorithmicend\ \algorithmicotherwise}
\ifbool{algpx@noEnd}{%
  \algtext*{EndSwitch}%
  \algtext*{EndCase}%
  \algtext*{EndCaseOf}
  \algtext*{EndOtherwise}
  %
  % end indent line after (not before), to get correct y position for multiline text in last command
  \apptocmd{\EndSwitch}{\algpx@endIndent}{}{}%
  \apptocmd{\EndCase}{\algpx@endIndent}{}{}%
  \apptocmd{\EndCaseOf}{\algpx@endIndent}{}{}
  \apptocmd{\EndOtherwise}{\algpx@endIndent}{}{}
}{}%

\pretocmd{\Switch}{\algpx@endCodeCommand}{}{}
\pretocmd{\Case}{\algpx@endCodeCommand}{}{}
\pretocmd{\CaseOf}{\algpx@endCodeCommand}{}{}
\pretocmd{\Otherwise}{\algpx@endCodeCommand}{}{}

% for end commands that may not be printed, tell endCodeCommand whether we are using noEnd
\ifbool{algpx@noEnd}{%
  \pretocmd{\EndSwitch}{\algpx@endCodeCommand[1]}{}{}%
  \pretocmd{\EndCase}{\algpx@endCodeCommand[1]}{}{}
  \pretocmd{\EndCaseOf}{\algpx@endCodeCommand[1]}{}{}%
  \pretocmd{\EndOtherwise}{\algpx@endCodeCommand[1]}{}{}
}{%
  \pretocmd{\EndSwitch}{\algpx@endCodeCommand[0]}{}{}%
  \pretocmd{\EndCase}{\algpx@endCodeCommand[0]}{}{}%
  \pretocmd{\EndCaseOf}{\algpx@endCodeCommand[0]}{}{}
  \pretocmd{\EndOtherwise}{\algpx@endCodeCommand[0]}{}{}
}%
\makeatother
% % For algpseudocode
% \algnewcommand\algorithmicswitch{\textbf{switch}}
% \algnewcommand\algorithmiccase{\textbf{case}}
% \algnewcommand\algorithmiccaseof{\textbf{case}}
% \algnewcommand\algorithmicof{\textbf{of}}
% % New "environments"
% \algdef{SE}[SWITCH]{Switch}{EndSwitch}[1]{\algorithmicswitch\ #1\ \algorithmicdo}{\algorithmicend\ \algorithmicswitch}%
% \algdef{SE}[CASE]{Case}{EndCase}[1]{\algorithmiccase\ #1}{\algorithmicend\ \algorithmiccase}%
% \algtext*{EndSwitch}%
% \algtext*{EndCase}
% \algdef{SE}[CASEOF]{CaseOf}{EndCaseOf}[1]{\algorithmiccaseof\ #1 \algorithmicof}{\algorithmicend\ \algorithmiccaseof}
% \algtext*{EndCaseOf}



%\pdfcompresslevel0

% quoting from
% https://tex.stackexchange.com/questions/391726/the-quotation-environment
\NewDocumentCommand{\bywhom}{m}{% the Bourbaki trick
  {\nobreak\hfill\penalty50\hskip1em\null\nobreak
   \hfill\mbox{\normalfont(#1)}%
   \parfillskip=0pt \finalhyphendemerits=0 \par}%
}

\NewDocumentEnvironment{pquotation}{m}
  {\begin{quoting}[
     indentfirst=true,
     leftmargin=\parindent,
     rightmargin=\parindent]\itshape}
  {\bywhom{#1}\end{quoting}}

\indexsetup{othercode=\small}
\makeindex[columns=2,options={-s /media/wu/file/stuuudy/notes/index_style.ist},intoc]
\makeatletter
\def\@idxitem{\par\hangindent 0pt}
\makeatother


% \newcounter{dummy} \numberwithin{dummy}{section}
\newtheorem{dummy}{dummy}[section]
\theoremstyle{definition}
\newtheorem{definition}[dummy]{Definition}
\theoremstyle{plain}
\newtheorem{corollary}[dummy]{Corollary}
\newtheorem{lemma}[dummy]{Lemma}
\newtheorem{proposition}[dummy]{Proposition}
\newtheorem{theorem}[dummy]{Theorem}
\newtheorem{notation}[dummy]{Notation}
\newtheorem{conjecture}[dummy]{Conjecture}
\newtheorem{fact}[dummy]{Fact}
\newtheorem{warning}[dummy]{Warning}
\theoremstyle{definition}
\newtheorem{examplle}{Example}[section]
\theoremstyle{remark}
\newtheorem*{remark}{Remark}
\newtheorem{exercise}{Exercise}[subsection]
\newtheorem{problem}{Problem}[subsection]
\newtheorem{observation}{Observation}[section]
\newenvironment{claim}[1]{\par\noindent\textbf{Claim:}\space#1}{}

\makeatletter
\DeclareFontFamily{U}{tipa}{}
\DeclareFontShape{U}{tipa}{m}{n}{<->tipa10}{}
\newcommand{\arc@char}{{\usefont{U}{tipa}{m}{n}\symbol{62}}}%

\newcommand{\arc}[1]{\mathpalette\arc@arc{#1}}

\newcommand{\arc@arc}[2]{%
  \sbox0{$\m@th#1#2$}%
  \vbox{
    \hbox{\resizebox{\wd0}{\height}{\arc@char}}
    \nointerlineskip
    \box0
  }%
}
\makeatother

\setcounter{MaxMatrixCols}{20}
%%%%%%% ABS
\DeclarePairedDelimiter\abss{\lvert}{\rvert}%
\DeclarePairedDelimiter\normm{\lVert}{\rVert}%

% Swap the definition of \abs* and \norm*, so that \abs
% and \norm resizes the size of the brackets, and the
% starred version does not.
\makeatletter
\let\oldabs\abss
%\def\abs{\@ifstar{\oldabs}{\oldabs*}}
\newcommand{\abs}{\@ifstar{\oldabs}{\oldabs*}}
\newcommand{\norm}[1]{\left\lVert#1\right\rVert}
%\let\oldnorm\normm
%\def\norm{\@ifstar{\oldnorm}{\oldnorm*}}
%\renewcommand{norm}{\@ifstar{\oldnorm}{\oldnorm*}}
\makeatother

% \stackMath
% \newcommand\what[1]{%
% \savestack{\tmpbox}{\stretchto{%
%   \scaleto{%
%     \scalerel*[\widthof{\ensuremath{#1}}]{\kern-.6pt\bigwedge\kern-.6pt}%
%     {\rule[-\textheight/2]{1ex}{\textheight}}%WIDTH-LIMITED BIG WEDGE
%   }{\textheight}%
% }{0.5ex}}%
% \stackon[1pt]{#1}{\tmpbox}%
% }

% \newcommand\what[1]{\ThisStyle{%
%     \setbox0=\hbox{$\SavedStyle#1$}%
%     \stackengine{-1.0\ht0+.5pt}{$\SavedStyle#1$}{%
%       \stretchto{\scaleto{\SavedStyle\mkern.15mu\char'136}{2.6\wd0}}{1.4\ht0}%
%     }{O}{c}{F}{T}{S}%
%   }
% }

% \newcommand\wtilde[1]{\ThisStyle{%
%     \setbox0=\hbox{$\SavedStyle#1$}%
%     \stackengine{-.1\LMpt}{$\SavedStyle#1$}{%
%       \stretchto{\scaleto{\SavedStyle\mkern.2mu\AC}{.5150\wd0}}{.6\ht0}%
%     }{O}{c}{F}{T}{S}%
%   }
% }

% \newcommand\wbar[1]{\ThisStyle{%
%     \setbox0=\hbox{$\SavedStyle#1$}%
%     \stackengine{.5pt+\LMpt}{$\SavedStyle#1$}{%
%       \rule{\wd0}{\dimexpr.3\LMpt+.3pt}%
%     }{O}{c}{F}{T}{S}%
%   }
% }

\newcommand{\bl}[1] {\boldsymbol{#1}}
\newcommand{\Wt}[1] {\stackrel{\sim}{\smash{#1}\rule{0pt}{1.1ex}}}
\newcommand{\wt}[1] {\widetilde{#1}}
\newcommand{\tf}[1] {\textbf{#1}}

\newcommand{\wu}[1]{{\color{red} #1}}

%For boxed texts in align, use Aboxed{}
%otherwise use boxed{}

\DeclareMathSymbol{\widehatsym}{\mathord}{largesymbols}{"62}
\newcommand\lowerwidehatsym{%
  \text{\smash{\raisebox{-1.3ex}{%
    $\widehatsym$}}}}
\newcommand\fixwidehat[1]{%
  \mathchoice
    {\accentset{\displaystyle\lowerwidehatsym}{#1}}
    {\accentset{\textstyle\lowerwidehatsym}{#1}}
    {\accentset{\scriptstyle\lowerwidehatsym}{#1}}
    {\accentset{\scriptscriptstyle\lowerwidehatsym}{#1}}
  }


\newcommand{\cupdot}{\mathbin{\dot{\cup}}}
\newcommand{\bigcupdot}{\mathop{\dot{\bigcup}}}

\usepackage{graphicx}

\usepackage[toc,page]{appendix}

% text on arrow for xRightarrow
\makeatletter
%\newcommand{\xRightarrow}[2][]{\ext@arrow 0359\Rightarrowfill@{#1}{#2}}
\makeatother

% Arbitrary long arrow
\newcommand{\Rarrow}[1]{%
\parbox{#1}{\tikz{\draw[->](0,0)--(#1,0);}}
}

\newcommand{\LRarrow}[1]{%
\parbox{#1}{\tikz{\draw[<->](0,0)--(#1,0);}}
}


\makeatletter
\providecommand*{\rmodels}{%
  \mathrel{%
    \mathpalette\@rmodels\models
  }%
}
\newcommand*{\@rmodels}[2]{%
  \reflectbox{$\m@th#1#2$}%
}
\makeatother

% Roman numerals
\makeatletter
\newcommand*{\rom}[1]{\expandafter\@slowromancap\romannumeral #1@}
\makeatother
% \\def \\b\([a-zA-Z]\) {\\boldsymbol{[a-zA-z]}}
% \\DeclareMathOperator{\\b\1}{\\textbf{\1}}

\DeclareMathOperator*{\argmin}{arg\,min}
\DeclareMathOperator*{\argmax}{arg\,max}

\DeclareMathOperator{\bone}{\textbf{1}}
\DeclareMathOperator{\bx}{\textbf{x}}
\DeclareMathOperator{\bz}{\textbf{z}}
\DeclareMathOperator{\bff}{\textbf{f}}
\DeclareMathOperator{\ba}{\textbf{a}}
\DeclareMathOperator{\bk}{\textbf{k}}
\DeclareMathOperator{\bs}{\textbf{s}}
\DeclareMathOperator{\bh}{\textbf{h}}
\DeclareMathOperator{\bc}{\textbf{c}}
\DeclareMathOperator{\br}{\textbf{r}}
\DeclareMathOperator{\bi}{\textbf{i}}
\DeclareMathOperator{\bj}{\textbf{j}}
\DeclareMathOperator{\bn}{\textbf{n}}
\DeclareMathOperator{\be}{\textbf{e}}
\DeclareMathOperator{\bo}{\textbf{o}}
\DeclareMathOperator{\bU}{\textbf{U}}
\DeclareMathOperator{\bL}{\textbf{L}}
\DeclareMathOperator{\bV}{\textbf{V}}
\def \bzero {\mathbf{0}}
\def \bbone {\mathbb{1}}
\def \btwo {\mathbf{2}}
\DeclareMathOperator{\bv}{\textbf{v}}
\DeclareMathOperator{\bp}{\textbf{p}}
\DeclareMathOperator{\bI}{\textbf{I}}
\def \dbI {\dot{\bI}}
\DeclareMathOperator{\bM}{\textbf{M}}
\DeclareMathOperator{\bN}{\textbf{N}}
\DeclareMathOperator{\bK}{\textbf{K}}
\DeclareMathOperator{\bt}{\textbf{t}}
\DeclareMathOperator{\bb}{\textbf{b}}
\DeclareMathOperator{\bA}{\textbf{A}}
\DeclareMathOperator{\bX}{\textbf{X}}
\DeclareMathOperator{\bu}{\textbf{u}}
\DeclareMathOperator{\bS}{\textbf{S}}
\DeclareMathOperator{\bZ}{\textbf{Z}}
\DeclareMathOperator{\bJ}{\textbf{J}}
\DeclareMathOperator{\by}{\textbf{y}}
\DeclareMathOperator{\bw}{\textbf{w}}
\DeclareMathOperator{\bT}{\textbf{T}}
\DeclareMathOperator{\bF}{\textbf{F}}
\DeclareMathOperator{\bmm}{\textbf{m}}
\DeclareMathOperator{\bW}{\textbf{W}}
\DeclareMathOperator{\bR}{\textbf{R}}
\DeclareMathOperator{\bC}{\textbf{C}}
\DeclareMathOperator{\bD}{\textbf{D}}
\DeclareMathOperator{\bE}{\textbf{E}}
\DeclareMathOperator{\bQ}{\textbf{Q}}
\DeclareMathOperator{\bP}{\textbf{P}}
\DeclareMathOperator{\bY}{\textbf{Y}}
\DeclareMathOperator{\bH}{\textbf{H}}
\DeclareMathOperator{\bB}{\textbf{B}}
\DeclareMathOperator{\bG}{\textbf{G}}
\def \blambda {\symbf{\lambda}}
\def \boldeta {\symbf{\eta}}
\def \balpha {\symbf{\alpha}}
\def \btau {\symbf{\tau}}
\def \bbeta {\symbf{\beta}}
\def \bgamma {\symbf{\gamma}}
\def \bxi {\symbf{\xi}}
\def \bLambda {\symbf{\Lambda}}
\def \bGamma {\symbf{\Gamma}}

\newcommand{\bto}{{\boldsymbol{\to}}}
\newcommand{\Ra}{\Rightarrow}
\newcommand{\xrsa}[1]{\overset{#1}{\rightsquigarrow}}
\newcommand{\xlsa}[1]{\overset{#1}{\leftsquigarrow}}
\newcommand\und[1]{\underline{#1}}
\newcommand\ove[1]{\overline{#1}}
%\def \concat {\verb|^|}
\def \bPhi {\mbfPhi}
\def \btheta {\mbftheta}
\def \bTheta {\mbfTheta}
\def \bmu {\mbfmu}
\def \bphi {\mbfphi}
\def \bSigma {\mbfSigma}
\def \la {\langle}
\def \ra {\rangle}

\def \caln {\mathcal{N}}
\def \dissum {\displaystyle\Sigma}
\def \dispro {\displaystyle\prod}

\def \caret {\verb!^!}

\def \A {\mathbb{A}}
\def \B {\mathbb{B}}
\def \C {\mathbb{C}}
\def \D {\mathbb{D}}
\def \E {\mathbb{E}}
\def \F {\mathbb{F}}
\def \G {\mathbb{G}}
\def \H {\mathbb{H}}
\def \I {\mathbb{I}}
\def \J {\mathbb{J}}
\def \K {\mathbb{K}}
\def \L {\mathbb{L}}
\def \M {\mathbb{M}}
\def \N {\mathbb{N}}
\def \O {\mathbb{O}}
\def \P {\mathbb{P}}
\def \Q {\mathbb{Q}}
\def \R {\mathbb{R}}
\def \S {\mathbb{S}}
\def \T {\mathbb{T}}
\def \U {\mathbb{U}}
\def \V {\mathbb{V}}
\def \W {\mathbb{W}}
\def \X {\mathbb{X}}
\def \Y {\mathbb{Y}}
\def \Z {\mathbb{Z}}

\def \cala {\mathcal{A}}
\def \cale {\mathcal{E}}
\def \calb {\mathcal{B}}
\def \calq {\mathcal{Q}}
\def \calp {\mathcal{P}}
\def \cals {\mathcal{S}}
\def \calx {\mathcal{X}}
\def \caly {\mathcal{Y}}
\def \calg {\mathcal{G}}
\def \cald {\mathcal{D}}
\def \caln {\mathcal{N}}
\def \calr {\mathcal{R}}
\def \calt {\mathcal{T}}
\def \calm {\mathcal{M}}
\def \calw {\mathcal{W}}
\def \calc {\mathcal{C}}
\def \calv {\mathcal{V}}
\def \calf {\mathcal{F}}
\def \calk {\mathcal{K}}
\def \call {\mathcal{L}}
\def \calu {\mathcal{U}}
\def \calo {\mathcal{O}}
\def \calh {\mathcal{H}}
\def \cali {\mathcal{I}}
\def \calj {\mathcal{J}}

\def \bcup {\bigcup}

% set theory

\def \zfcc {\textbf{ZFC}^-}
\def \BGC {\textbf{BGC}}
\def \BG {\textbf{BG}}
\def \ac  {\textbf{AC}}
\def \gl  {\textbf{L }}
\def \gll {\textbf{L}}
\newcommand{\zfm}{$\textbf{ZF}^-$}

\def \ZFm {\text{ZF}^-}
\def \ZFCm {\text{ZFC}^-}
\DeclareMathOperator{\WF}{WF}
\DeclareMathOperator{\On}{On}
\def \on {\textbf{On }}
\def \cm {\textbf{M }}
\def \cn {\textbf{N }}
\def \cv {\textbf{V }}
\def \zc {\textbf{ZC }}
\def \zcm {\textbf{ZC}}
\def \zff {\textbf{ZF}}
\def \wfm {\textbf{WF}}
\def \onm {\textbf{On}}
\def \cmm {\textbf{M}}
\def \cnm {\textbf{N}}
\def \cvm {\textbf{V}}

\renewcommand{\restriction}{\mathord{\upharpoonright}}
%% another restriction
\newcommand\restr[2]{{% we make the whole thing an ordinary symbol
  \left.\kern-\nulldelimiterspace % automatically resize the bar with \right
  #1 % the function
  \vphantom{\big|} % pretend it's a little taller at normal size
  \right|_{#2} % this is the delimiter
  }}

\def \pred {\text{pred}}

\def \rank {\text{rank}}
\def \Con {\text{Con}}
\def \deff {\text{Def}}


\def \uin {\underline{\in}}
\def \oin {\overline{\in}}
\def \uR {\underline{R}}
\def \oR {\overline{R}}
\def \uP {\underline{P}}
\def \oP {\overline{P}}

\def \dsum {\displaystyle\sum}

\def \Ra {\Rightarrow}

\def \e {\enspace}

\def \sgn {\operatorname{sgn}}
\def \gen {\operatorname{gen}}
\def \Hom {\operatorname{Hom}}
\def \hom {\operatorname{hom}}
\def \Sub {\operatorname{Sub}}

\def \supp {\operatorname{supp}}

\def \epiarrow {\twoheadarrow}
\def \monoarrow {\rightarrowtail}
\def \rrarrow {\rightrightarrows}

% \def \minus {\text{-}}
% \newcommand{\minus}{\scalebox{0.75}[1.0]{$-$}}
% \DeclareUnicodeCharacter{002D}{\minus}


\def \tril {\triangleleft}

\def \ISigma {\text{I}\Sigma}
\def \IDelta {\text{I}\Delta}
\def \IPi {\text{I}\Pi}
\def \ACF {\textsf{ACF}}
\def \pCF {\textit{p}\text{CF}}
\def \ACVF {\textsf{ACVF}}
\def \HLR {\textsf{HLR}}
\def \OAG {\textsf{OAG}}
\def \RCF {\textsf{RCF}}
\DeclareMathOperator{\GL}{GL}
\DeclareMathOperator{\PGL}{PGL}
\DeclareMathOperator{\SL}{SL}
\DeclareMathOperator{\Inv}{Inv}
\DeclareMathOperator{\res}{res}
\DeclareMathOperator{\Sym}{Sym}
%\DeclareMathOperator{\char}{char}
\def \equal {=}

\def \degree {\text{degree}}
\def \app {\text{App}}
\def \FV {\text{FV}}
\def \conv {\text{conv}}
\def \cont {\text{cont}}
\DeclareMathOperator{\cl}{\text{cl}}
\DeclareMathOperator{\trcl}{\text{trcl}}
\DeclareMathOperator{\sg}{sg}
\DeclareMathOperator{\trdeg}{trdeg}
\def \Ord {\text{Ord}}

\DeclareMathOperator{\cf}{cf}
\DeclareMathOperator{\zfc}{ZFC}

%\DeclareMathOperator{\Th}{Th}
%\def \th {\text{Th}}
% \newcommand{\th}{\text{Th}}
\DeclareMathOperator{\type}{type}
\DeclareMathOperator{\zf}{\textbf{ZF}}
\def \fa {\mathfrak{a}}
\def \fb {\mathfrak{b}}
\def \fc {\mathfrak{c}}
\def \fd {\mathfrak{d}}
\def \fe {\mathfrak{e}}
\def \ff {\mathfrak{f}}
\def \fg {\mathfrak{g}}
\def \fh {\mathfrak{h}}
%\def \fi {\mathfrak{i}}
\def \fj {\mathfrak{j}}
\def \fk {\mathfrak{k}}
\def \fl {\mathfrak{l}}
\def \fm {\mathfrak{m}}
\def \fn {\mathfrak{n}}
\def \fo {\mathfrak{o}}
\def \fp {\mathfrak{p}}
\def \fq {\mathfrak{q}}
\def \fr {\mathfrak{r}}
\def \fs {\mathfrak{s}}
\def \ft {\mathfrak{t}}
\def \fu {\mathfrak{u}}
\def \fv {\mathfrak{v}}
\def \fw {\mathfrak{w}}
\def \fx {\mathfrak{x}}
\def \fy {\mathfrak{y}}
\def \fz {\mathfrak{z}}
\def \fA {\mathfrak{A}}
\def \fB {\mathfrak{B}}
\def \fC {\mathfrak{C}}
\def \fD {\mathfrak{D}}
\def \fE {\mathfrak{E}}
\def \fF {\mathfrak{F}}
\def \fG {\mathfrak{G}}
\def \fH {\mathfrak{H}}
\def \fI {\mathfrak{I}}
\def \fJ {\mathfrak{J}}
\def \fK {\mathfrak{K}}
\def \fL {\mathfrak{L}}
\def \fM {\mathfrak{M}}
\def \fN {\mathfrak{N}}
\def \fO {\mathfrak{O}}
\def \fP {\mathfrak{P}}
\def \fQ {\mathfrak{Q}}
\def \fR {\mathfrak{R}}
\def \fS {\mathfrak{S}}
\def \fT {\mathfrak{T}}
\def \fU {\mathfrak{U}}
\def \fV {\mathfrak{V}}
\def \fW {\mathfrak{W}}
\def \fX {\mathfrak{X}}
\def \fY {\mathfrak{Y}}
\def \fZ {\mathfrak{Z}}

\def \sfA {\textsf{A}}
\def \sfB {\textsf{B}}
\def \sfC {\textsf{C}}
\def \sfD {\textsf{D}}
\def \sfE {\textsf{E}}
\def \sfF {\textsf{F}}
\def \sfG {\textsf{G}}
\def \sfH {\textsf{H}}
\def \sfI {\textsf{I}}
\def \sfJ {\textsf{J}}
\def \sfK {\textsf{K}}
\def \sfL {\textsf{L}}
\def \sfM {\textsf{M}}
\def \sfN {\textsf{N}}
\def \sfO {\textsf{O}}
\def \sfP {\textsf{P}}
\def \sfQ {\textsf{Q}}
\def \sfR {\textsf{R}}
\def \sfS {\textsf{S}}
\def \sfT {\textsf{T}}
\def \sfU {\textsf{U}}
\def \sfV {\textsf{V}}
\def \sfW {\textsf{W}}
\def \sfX {\textsf{X}}
\def \sfY {\textsf{Y}}
\def \sfZ {\textsf{Z}}
\def \sfa {\textsf{a}}
\def \sfb {\textsf{b}}
\def \sfc {\textsf{c}}
\def \sfd {\textsf{d}}
\def \sfe {\textsf{e}}
\def \sff {\textsf{f}}
\def \sfg {\textsf{g}}
\def \sfh {\textsf{h}}
\def \sfi {\textsf{i}}
\def \sfj {\textsf{j}}
\def \sfk {\textsf{k}}
\def \sfl {\textsf{l}}
\def \sfm {\textsf{m}}
\def \sfn {\textsf{n}}
\def \sfo {\textsf{o}}
\def \sfp {\textsf{p}}
\def \sfq {\textsf{q}}
\def \sfr {\textsf{r}}
\def \sfs {\textsf{s}}
\def \sft {\textsf{t}}
\def \sfu {\textsf{u}}
\def \sfv {\textsf{v}}
\def \sfw {\textsf{w}}
\def \sfx {\textsf{x}}
\def \sfy {\textsf{y}}
\def \sfz {\textsf{z}}

\def \ttA {\texttt{A}}
\def \ttB {\texttt{B}}
\def \ttC {\texttt{C}}
\def \ttD {\texttt{D}}
\def \ttE {\texttt{E}}
\def \ttF {\texttt{F}}
\def \ttG {\texttt{G}}
\def \ttH {\texttt{H}}
\def \ttI {\texttt{I}}
\def \ttJ {\texttt{J}}
\def \ttK {\texttt{K}}
\def \ttL {\texttt{L}}
\def \ttM {\texttt{M}}
\def \ttN {\texttt{N}}
\def \ttO {\texttt{O}}
\def \ttP {\texttt{P}}
\def \ttQ {\texttt{Q}}
\def \ttR {\texttt{R}}
\def \ttS {\texttt{S}}
\def \ttT {\texttt{T}}
\def \ttU {\texttt{U}}
\def \ttV {\texttt{V}}
\def \ttW {\texttt{W}}
\def \ttX {\texttt{X}}
\def \ttY {\texttt{Y}}
\def \ttZ {\texttt{Z}}
\def \tta {\texttt{a}}
\def \ttb {\texttt{b}}
\def \ttc {\texttt{c}}
\def \ttd {\texttt{d}}
\def \tte {\texttt{e}}
\def \ttf {\texttt{f}}
\def \ttg {\texttt{g}}
\def \tth {\texttt{h}}
\def \tti {\texttt{i}}
\def \ttj {\texttt{j}}
\def \ttk {\texttt{k}}
\def \ttl {\texttt{l}}
\def \ttm {\texttt{m}}
\def \ttn {\texttt{n}}
\def \tto {\texttt{o}}
\def \ttp {\texttt{p}}
\def \ttq {\texttt{q}}
\def \ttr {\texttt{r}}
\def \tts {\texttt{s}}
\def \ttt {\texttt{t}}
\def \ttu {\texttt{u}}
\def \ttv {\texttt{v}}
\def \ttw {\texttt{w}}
\def \ttx {\texttt{x}}
\def \tty {\texttt{y}}
\def \ttz {\texttt{z}}

\def \bara {\bbar{a}}
\def \barb {\bbar{b}}
\def \barc {\bbar{c}}
\def \bard {\bbar{d}}
\def \bare {\bbar{e}}
\def \barf {\bbar{f}}
\def \barg {\bbar{g}}
\def \barh {\bbar{h}}
\def \bari {\bbar{i}}
\def \barj {\bbar{j}}
\def \bark {\bbar{k}}
\def \barl {\bbar{l}}
\def \barm {\bbar{m}}
\def \barn {\bbar{n}}
\def \baro {\bbar{o}}
\def \barp {\bbar{p}}
\def \barq {\bbar{q}}
\def \barr {\bbar{r}}
\def \bars {\bbar{s}}
\def \bart {\bbar{t}}
\def \baru {\bbar{u}}
\def \barv {\bbar{v}}
\def \barw {\bbar{w}}
\def \barx {\bbar{x}}
\def \bary {\bbar{y}}
\def \barz {\bbar{z}}
\def \barA {\bbar{A}}
\def \barB {\bbar{B}}
\def \barC {\bbar{C}}
\def \barD {\bbar{D}}
\def \barE {\bbar{E}}
\def \barF {\bbar{F}}
\def \barG {\bbar{G}}
\def \barH {\bbar{H}}
\def \barI {\bbar{I}}
\def \barJ {\bbar{J}}
\def \barK {\bbar{K}}
\def \barL {\bbar{L}}
\def \barM {\bbar{M}}
\def \barN {\bbar{N}}
\def \barO {\bbar{O}}
\def \barP {\bbar{P}}
\def \barQ {\bbar{Q}}
\def \barR {\bbar{R}}
\def \barS {\bbar{S}}
\def \barT {\bbar{T}}
\def \barU {\bbar{U}}
\def \barVV {\bbar{V}}
\def \barW {\bbar{W}}
\def \barX {\bbar{X}}
\def \barY {\bbar{Y}}
\def \barZ {\bbar{Z}}

\def \baralpha {\bbar{\alpha}}
\def \bartau {\bbar{\tau}}
\def \barsigma {\bbar{\sigma}}
\def \barzeta {\bbar{\zeta}}

\def \hata {\hat{a}}
\def \hatb {\hat{b}}
\def \hatc {\hat{c}}
\def \hatd {\hat{d}}
\def \hate {\hat{e}}
\def \hatf {\hat{f}}
\def \hatg {\hat{g}}
\def \hath {\hat{h}}
\def \hati {\hat{i}}
\def \hatj {\hat{j}}
\def \hatk {\hat{k}}
\def \hatl {\hat{l}}
\def \hatm {\hat{m}}
\def \hatn {\hat{n}}
\def \hato {\hat{o}}
\def \hatp {\hat{p}}
\def \hatq {\hat{q}}
\def \hatr {\hat{r}}
\def \hats {\hat{s}}
\def \hatt {\hat{t}}
\def \hatu {\hat{u}}
\def \hatv {\hat{v}}
\def \hatw {\hat{w}}
\def \hatx {\hat{x}}
\def \haty {\hat{y}}
\def \hatz {\hat{z}}
\def \hatA {\hat{A}}
\def \hatB {\hat{B}}
\def \hatC {\hat{C}}
\def \hatD {\hat{D}}
\def \hatE {\hat{E}}
\def \hatF {\hat{F}}
\def \hatG {\hat{G}}
\def \hatH {\hat{H}}
\def \hatI {\hat{I}}
\def \hatJ {\hat{J}}
\def \hatK {\hat{K}}
\def \hatL {\hat{L}}
\def \hatM {\hat{M}}
\def \hatN {\hat{N}}
\def \hatO {\hat{O}}
\def \hatP {\hat{P}}
\def \hatQ {\hat{Q}}
\def \hatR {\hat{R}}
\def \hatS {\hat{S}}
\def \hatT {\hat{T}}
\def \hatU {\hat{U}}
\def \hatVV {\hat{V}}
\def \hatW {\hat{W}}
\def \hatX {\hat{X}}
\def \hatY {\hat{Y}}
\def \hatZ {\hat{Z}}

\def \hatphi {\hat{\phi}}

\def \barfM {\bbar{\fM}}
\def \barfN {\bbar{\fN}}

\def \tila {\tilde{a}}
\def \tilb {\tilde{b}}
\def \tilc {\tilde{c}}
\def \tild {\tilde{d}}
\def \tile {\tilde{e}}
\def \tilf {\tilde{f}}
\def \tilg {\tilde{g}}
\def \tilh {\tilde{h}}
\def \tili {\tilde{i}}
\def \tilj {\tilde{j}}
\def \tilk {\tilde{k}}
\def \till {\tilde{l}}
\def \tilm {\tilde{m}}
\def \tiln {\tilde{n}}
\def \tilo {\tilde{o}}
\def \tilp {\tilde{p}}
\def \tilq {\tilde{q}}
\def \tilr {\tilde{r}}
\def \tils {\tilde{s}}
\def \tilt {\tilde{t}}
\def \tilu {\tilde{u}}
\def \tilv {\tilde{v}}
\def \tilw {\tilde{w}}
\def \tilx {\tilde{x}}
\def \tily {\tilde{y}}
\def \tilz {\tilde{z}}
\def \tilA {\tilde{A}}
\def \tilB {\tilde{B}}
\def \tilC {\tilde{C}}
\def \tilD {\tilde{D}}
\def \tilE {\tilde{E}}
\def \tilF {\tilde{F}}
\def \tilG {\tilde{G}}
\def \tilH {\tilde{H}}
\def \tilI {\tilde{I}}
\def \tilJ {\tilde{J}}
\def \tilK {\tilde{K}}
\def \tilL {\tilde{L}}
\def \tilM {\tilde{M}}
\def \tilN {\tilde{N}}
\def \tilO {\tilde{O}}
\def \tilP {\tilde{P}}
\def \tilQ {\tilde{Q}}
\def \tilR {\tilde{R}}
\def \tilS {\tilde{S}}
\def \tilT {\tilde{T}}
\def \tilU {\tilde{U}}
\def \tilVV {\tilde{V}}
\def \tilW {\tilde{W}}
\def \tilX {\tilde{X}}
\def \tilY {\tilde{Y}}
\def \tilZ {\tilde{Z}}

\def \tilalpha {\tilde{\alpha}}
\def \tilPhi {\tilde{\Phi}}

\def \barnu {\bar{\nu}}
\def \barrho {\bar{\rho}}
%\DeclareMathOperator{\ker}{ker}
\DeclareMathOperator{\im}{im}

\DeclareMathOperator{\Inn}{Inn}
\DeclareMathOperator{\rel}{rel}
\def \dote {\stackrel{\cdot}=}
%\DeclareMathOperator{\AC}{\textbf{AC}}
\DeclareMathOperator{\cod}{cod}
\DeclareMathOperator{\dom}{dom}
\DeclareMathOperator{\card}{card}
\DeclareMathOperator{\ran}{ran}
\DeclareMathOperator{\textd}{d}
\DeclareMathOperator{\td}{d}
\DeclareMathOperator{\id}{id}
\DeclareMathOperator{\LT}{LT}
\DeclareMathOperator{\Mat}{Mat}
\DeclareMathOperator{\Eq}{Eq}
\DeclareMathOperator{\irr}{irr}
\DeclareMathOperator{\Fr}{Fr}
\DeclareMathOperator{\Gal}{Gal}
\DeclareMathOperator{\lcm}{lcm}
\DeclareMathOperator{\alg}{\text{alg}}
\DeclareMathOperator{\Th}{Th}
%\DeclareMathOperator{\deg}{deg}


% \varprod
\DeclareSymbolFont{largesymbolsA}{U}{txexa}{m}{n}
\DeclareMathSymbol{\varprod}{\mathop}{largesymbolsA}{16}
% \DeclareMathSymbol{\tonm}{\boldsymbol{\to}\textbf{Nm}}
\def \tonm {\bto\textbf{Nm}}
\def \tohm {\bto\textbf{Hm}}

% Category theory
\DeclareMathOperator{\ob}{ob}
\DeclareMathOperator{\Ab}{\textbf{Ab}}
\DeclareMathOperator{\Alg}{\textbf{Alg}}
\DeclareMathOperator{\Rng}{\textbf{Rng}}
\DeclareMathOperator{\Sets}{\textbf{Sets}}
\DeclareMathOperator{\Set}{\textbf{Set}}
\DeclareMathOperator{\Grp}{\textbf{Grp}}
\DeclareMathOperator{\Met}{\textbf{Met}}
\DeclareMathOperator{\BA}{\textbf{BA}}
\DeclareMathOperator{\Mon}{\textbf{Mon}}
\DeclareMathOperator{\Top}{\textbf{Top}}
\DeclareMathOperator{\hTop}{\textbf{hTop}}
\DeclareMathOperator{\HTop}{\textbf{HTop}}
\DeclareMathOperator{\Aut}{\text{Aut}}
\DeclareMathOperator{\RMod}{R-\textbf{Mod}}
\DeclareMathOperator{\RAlg}{R-\textbf{Alg}}
\DeclareMathOperator{\LF}{LF}
\DeclareMathOperator{\op}{op}
\DeclareMathOperator{\Rings}{\textbf{Rings}}
\DeclareMathOperator{\Ring}{\textbf{Ring}}
\DeclareMathOperator{\Groups}{\textbf{Groups}}
\DeclareMathOperator{\Group}{\textbf{Group}}
\DeclareMathOperator{\ev}{ev}
% Algebraic Topology
\DeclareMathOperator{\obj}{obj}
\DeclareMathOperator{\Spec}{Spec}
\DeclareMathOperator{\spec}{spec}
% Model theory
\DeclareMathOperator*{\ind}{\raise0.2ex\hbox{\ooalign{\hidewidth$\vert$\hidewidth\cr\raise-0.9ex\hbox{$\smile$}}}}
\def\nind{\cancel{\ind}}
\DeclareMathOperator{\acl}{acl}
\DeclareMathOperator{\tspan}{span}
\DeclareMathOperator{\acleq}{acl^{\eq}}
\DeclareMathOperator{\Av}{Av}
\DeclareMathOperator{\ded}{ded}
\DeclareMathOperator{\EM}{EM}
\DeclareMathOperator{\dcl}{dcl}
\DeclareMathOperator{\Ext}{Ext}
\DeclareMathOperator{\eq}{eq}
\DeclareMathOperator{\ER}{ER}
\DeclareMathOperator{\tp}{tp}
\DeclareMathOperator{\stp}{stp}
\DeclareMathOperator{\qftp}{qftp}
\DeclareMathOperator{\Diag}{Diag}
\DeclareMathOperator{\MD}{MD}
\DeclareMathOperator{\MR}{MR}
\DeclareMathOperator{\RM}{RM}
\DeclareMathOperator{\el}{el}
\DeclareMathOperator{\depth}{depth}
\DeclareMathOperator{\ZFC}{ZFC}
\DeclareMathOperator{\GCH}{GCH}
\DeclareMathOperator{\Inf}{Inf}
\DeclareMathOperator{\Pow}{Pow}
\DeclareMathOperator{\ZF}{ZF}
\DeclareMathOperator{\CH}{CH}
\def \FO {\text{FO}}
\DeclareMathOperator{\fin}{fin}
\DeclareMathOperator{\qr}{qr}
\DeclareMathOperator{\Mod}{Mod}
\DeclareMathOperator{\Def}{Def}
\DeclareMathOperator{\TC}{TC}
\DeclareMathOperator{\KH}{KH}
\DeclareMathOperator{\Part}{Part}
\DeclareMathOperator{\Infset}{\textsf{Infset}}
\DeclareMathOperator{\DLO}{\textsf{DLO}}
\DeclareMathOperator{\PA}{\textsf{PA}}
\DeclareMathOperator{\DAG}{\textsf{DAG}}
\DeclareMathOperator{\ODAG}{\textsf{ODAG}}
\DeclareMathOperator{\sfMod}{\textsf{Mod}}
\DeclareMathOperator{\AbG}{\textsf{AbG}}
\DeclareMathOperator{\sfACF}{\textsf{ACF}}
\DeclareMathOperator{\DCF}{\textsf{DCF}}
% Computability Theorem
\DeclareMathOperator{\Tot}{Tot}
\DeclareMathOperator{\graph}{graph}
\DeclareMathOperator{\Fin}{Fin}
\DeclareMathOperator{\Cof}{Cof}
\DeclareMathOperator{\lh}{lh}
% Commutative Algebra
\DeclareMathOperator{\ord}{ord}
\DeclareMathOperator{\Idem}{Idem}
\DeclareMathOperator{\zdiv}{z.div}
\DeclareMathOperator{\Frac}{Frac}
\DeclareMathOperator{\rad}{rad}
\DeclareMathOperator{\nil}{nil}
\DeclareMathOperator{\Ann}{Ann}
\DeclareMathOperator{\End}{End}
\DeclareMathOperator{\coim}{coim}
\DeclareMathOperator{\coker}{coker}
\DeclareMathOperator{\Bil}{Bil}
\DeclareMathOperator{\Tril}{Tril}
\DeclareMathOperator{\tchar}{char}
\DeclareMathOperator{\tbd}{bd}

% Topology
\DeclareMathOperator{\diam}{diam}
\newcommand{\interior}[1]{%
  {\kern0pt#1}^{\mathrm{o}}%
}

\DeclareMathOperator*{\bigdoublewedge}{\bigwedge\mkern-15mu\bigwedge}
\DeclareMathOperator*{\bigdoublevee}{\bigvee\mkern-15mu\bigvee}

% \makeatletter
% \newcommand{\vect}[1]{%
%   \vbox{\m@th \ialign {##\crcr
%   \vectfill\crcr\noalign{\kern-\p@ \nointerlineskip}
%   $\hfil\displaystyle{#1}\hfil$\crcr}}}
% \def\vectfill{%
%   $\m@th\smash-\mkern-7mu%
%   \cleaders\hbox{$\mkern-2mu\smash-\mkern-2mu$}\hfill
%   \mkern-7mu\raisebox{-3.81pt}[\p@][\p@]{$\mathord\mathchar"017E$}$}

% \newcommand{\amsvect}{%
%   \mathpalette {\overarrow@\vectfill@}}
% \def\vectfill@{\arrowfill@\relbar\relbar{\raisebox{-3.81pt}[\p@][\p@]{$\mathord\mathchar"017E$}}}

% \newcommand{\amsvectb}{%
% \newcommand{\vect}{%
%   \mathpalette {\overarrow@\vectfillb@}}
% \newcommand{\vecbar}{%
%   \scalebox{0.8}{$\relbar$}}
% \def\vectfillb@{\arrowfill@\vecbar\vecbar{\raisebox{-4.35pt}[\p@][\p@]{$\mathord\mathchar"017E$}}}
% \makeatother
% \bigtimes

\DeclareFontFamily{U}{mathx}{\hyphenchar\font45}
\DeclareFontShape{U}{mathx}{m}{n}{
      <5> <6> <7> <8> <9> <10>
      <10.95> <12> <14.4> <17.28> <20.74> <24.88>
      mathx10
      }{}
\DeclareSymbolFont{mathx}{U}{mathx}{m}{n}
\DeclareMathSymbol{\bigtimes}{1}{mathx}{"91}
% \odiv
\DeclareFontFamily{U}{matha}{\hyphenchar\font45}
\DeclareFontShape{U}{matha}{m}{n}{
      <5> <6> <7> <8> <9> <10> gen * matha
      <10.95> matha10 <12> <14.4> <17.28> <20.74> <24.88> matha12
      }{}
\DeclareSymbolFont{matha}{U}{matha}{m}{n}
\DeclareMathSymbol{\odiv}         {2}{matha}{"63}


\newcommand\subsetsim{\mathrel{%
  \ooalign{\raise0.2ex\hbox{\scalebox{0.9}{$\subset$}}\cr\hidewidth\raise-0.85ex\hbox{\scalebox{0.9}{$\sim$}}\hidewidth\cr}}}
\newcommand\simsubset{\mathrel{%
  \ooalign{\raise-0.2ex\hbox{\scalebox{0.9}{$\subset$}}\cr\hidewidth\raise0.75ex\hbox{\scalebox{0.9}{$\sim$}}\hidewidth\cr}}}

\newcommand\simsubsetsim{\mathrel{%
  \ooalign{\raise0ex\hbox{\scalebox{0.8}{$\subset$}}\cr\hidewidth\raise1ex\hbox{\scalebox{0.75}{$\sim$}}\hidewidth\cr\raise-0.95ex\hbox{\scalebox{0.8}{$\sim$}}\cr\hidewidth}}}
\newcommand{\stcomp}[1]{{#1}^{\mathsf{c}}}

\setlength{\baselineskip}{0.5in}

\stackMath
\newcommand\yrightarrow[2][]{\mathrel{%
  \setbox2=\hbox{\stackon{\scriptstyle#1}{\scriptstyle#2}}%
  \stackunder[0pt]{%
    \xrightarrow{\makebox[\dimexpr\wd2\relax]{$\scriptstyle#2$}}%
  }{%
   \scriptstyle#1\,%
  }%
}}
\newcommand\yleftarrow[2][]{\mathrel{%
  \setbox2=\hbox{\stackon{\scriptstyle#1}{\scriptstyle#2}}%
  \stackunder[0pt]{%
    \xleftarrow{\makebox[\dimexpr\wd2\relax]{$\scriptstyle#2$}}%
  }{%
   \scriptstyle#1\,%
  }%
}}
\newcommand\yRightarrow[2][]{\mathrel{%
  \setbox2=\hbox{\stackon{\scriptstyle#1}{\scriptstyle#2}}%
  \stackunder[0pt]{%
    \xRightarrow{\makebox[\dimexpr\wd2\relax]{$\scriptstyle#2$}}%
  }{%
   \scriptstyle#1\,%
  }%
}}
\newcommand\yLeftarrow[2][]{\mathrel{%
  \setbox2=\hbox{\stackon{\scriptstyle#1}{\scriptstyle#2}}%
  \stackunder[0pt]{%
    \xLeftarrow{\makebox[\dimexpr\wd2\relax]{$\scriptstyle#2$}}%
  }{%
   \scriptstyle#1\,%
  }%
}}

\newcommand\altxrightarrow[2][0pt]{\mathrel{\ensurestackMath{\stackengine%
  {\dimexpr#1-7.5pt}{\xrightarrow{\phantom{#2}}}{\scriptstyle\!#2\,}%
  {O}{c}{F}{F}{S}}}}
\newcommand\altxleftarrow[2][0pt]{\mathrel{\ensurestackMath{\stackengine%
  {\dimexpr#1-7.5pt}{\xleftarrow{\phantom{#2}}}{\scriptstyle\!#2\,}%
  {O}{c}{F}{F}{S}}}}

\newenvironment{bsm}{% % short for 'bracketed small matrix'
  \left[ \begin{smallmatrix} }{%
  \end{smallmatrix} \right]}

\newenvironment{psm}{% % short for ' small matrix'
  \left( \begin{smallmatrix} }{%
  \end{smallmatrix} \right)}

\newcommand{\bbar}[1]{\mkern 1.5mu\overline{\mkern-1.5mu#1\mkern-1.5mu}\mkern 1.5mu}

\newcommand{\bigzero}{\mbox{\normalfont\Large\bfseries 0}}
\newcommand{\rvline}{\hspace*{-\arraycolsep}\vline\hspace*{-\arraycolsep}}

\font\zallman=Zallman at 40pt
\font\elzevier=Elzevier at 40pt

\newcommand\isoto{\stackrel{\textstyle\sim}{\smash{\longrightarrow}\rule{0pt}{0.4ex}}}
\newcommand\embto{\stackrel{\textstyle\prec}{\smash{\longrightarrow}\rule{0pt}{0.4ex}}}

% from http://www.actual.world/resources/tex/doc/TikZ.pdf

\tikzset{
modal/.style={>=stealth’,shorten >=1pt,shorten <=1pt,auto,node distance=1.5cm,
semithick},
world/.style={circle,draw,minimum size=0.5cm,fill=gray!15},
point/.style={circle,draw,inner sep=0.5mm,fill=black},
reflexive above/.style={->,loop,looseness=7,in=120,out=60},
reflexive below/.style={->,loop,looseness=7,in=240,out=300},
reflexive left/.style={->,loop,looseness=7,in=150,out=210},
reflexive right/.style={->,loop,looseness=7,in=30,out=330}
}


\makeatletter
\newcommand*{\doublerightarrow}[2]{\mathrel{
  \settowidth{\@tempdima}{$\scriptstyle#1$}
  \settowidth{\@tempdimb}{$\scriptstyle#2$}
  \ifdim\@tempdimb>\@tempdima \@tempdima=\@tempdimb\fi
  \mathop{\vcenter{
    \offinterlineskip\ialign{\hbox to\dimexpr\@tempdima+1em{##}\cr
    \rightarrowfill\cr\noalign{\kern.5ex}
    \rightarrowfill\cr}}}\limits^{\!#1}_{\!#2}}}
\newcommand*{\triplerightarrow}[1]{\mathrel{
  \settowidth{\@tempdima}{$\scriptstyle#1$}
  \mathop{\vcenter{
    \offinterlineskip\ialign{\hbox to\dimexpr\@tempdima+1em{##}\cr
    \rightarrowfill\cr\noalign{\kern.5ex}
    \rightarrowfill\cr\noalign{\kern.5ex}
    \rightarrowfill\cr}}}\limits^{\!#1}}}
\makeatother

% $A\doublerightarrow{a}{bcdefgh}B$

% $A\triplerightarrow{d_0,d_1,d_2}B$

\def \uhr {\upharpoonright}
\def \rhu {\rightharpoonup}
\def \uhl {\upharpoonleft}


\newcommand{\floor}[1]{\lfloor #1 \rfloor}
\newcommand{\ceil}[1]{\lceil #1 \rceil}
\newcommand{\lcorner}[1]{\llcorner #1 \lrcorner}
\newcommand{\llb}[1]{\llbracket #1 \rrbracket}
\newcommand{\ucorner}[1]{\ulcorner #1 \urcorner}
\newcommand{\emoji}[1]{{\DejaSans #1}}
\newcommand{\vprec}{\rotatebox[origin=c]{-90}{$\prec$}}

\newcommand{\nat}[6][large]{%
  \begin{tikzcd}[ampersand replacement = \&, column sep=#1]
    #2\ar[bend left=40,""{name=U}]{r}{#4}\ar[bend right=40,',""{name=D}]{r}{#5}\& #3
          \ar[shorten <=10pt,shorten >=10pt,Rightarrow,from=U,to=D]{d}{~#6}
    \end{tikzcd}
}


\providecommand\rightarrowRHD{\relbar\joinrel\mathrel\RHD}
\providecommand\rightarrowrhd{\relbar\joinrel\mathrel\rhd}
\providecommand\longrightarrowRHD{\relbar\joinrel\relbar\joinrel\mathrel\RHD}
\providecommand\longrightarrowrhd{\relbar\joinrel\relbar\joinrel\mathrel\rhd}
\def \lrarhd {\longrightarrowrhd}


\makeatletter
\providecommand*\xrightarrowRHD[2][]{\ext@arrow 0055{\arrowfill@\relbar\relbar\longrightarrowRHD}{#1}{#2}}
\providecommand*\xrightarrowrhd[2][]{\ext@arrow 0055{\arrowfill@\relbar\relbar\longrightarrowrhd}{#1}{#2}}
\makeatother

\newcommand{\metalambda}{%
  \mathop{%
    \rlap{$\lambda$}%
    \mkern3mu
    \raisebox{0ex}{$\lambda$}%
  }%
}

%% https://tex.stackexchange.com/questions/15119/draw-horizontal-line-left-and-right-of-some-text-a-single-line
\newcommand*\ruleline[1]{\par\noindent\raisebox{.8ex}{\makebox[\linewidth]{\hrulefill\hspace{1ex}\raisebox{-.8ex}{#1}\hspace{1ex}\hrulefill}}}

% https://www.dickimaw-books.com/latex/novices/html/newenv.html
\newenvironment{Block}[1]% environment name
{% begin code
  % https://tex.stackexchange.com/questions/19579/horizontal-line-spanning-the-entire-document-in-latex
  \noindent\textcolor[RGB]{128,128,128}{\rule{\linewidth}{1pt}}
  \par\noindent
  {\Large\textbf{#1}}%
  \bigskip\par\noindent\ignorespaces
}%
{% end code
  \par\noindent
  \textcolor[RGB]{128,128,128}{\rule{\linewidth}{1pt}}
  \ignorespacesafterend
}

\mathchardef\mhyphen="2D % Define a "math hyphen"

\def \QQ {\quad}
\def \QW {​\quad}

\makeindex
\author{J. S. Milne}
\date{\today}
\title{Fields and Galois Theory}
\hypersetup{
 pdfauthor={J. S. Milne},
 pdftitle={Fields and Galois Theory},
 pdfkeywords={},
 pdfsubject={},
 pdfcreator={Emacs 28.0.90 (Org mode 9.6)}, 
 pdflang={English}}
\begin{document}

\maketitle
\tableofcontents

\section{Basic Definitions and Results}
\label{sec:orgf368abe}
\subsection{The characteristic of a field}
\label{sec:org9c64896}
Given a field \(F\) and consider a map
\begin{equation*}
\Z\to F,\quad n\mapsto n\cdot 1_F
\end{equation*}
If the kernel of the map is \(\neq (0)\), so that \(n\cdot 1_F=0\) for some \(n\neq 0\). The smallest
positive such \(n\) will be a prime \(p\) (otherwise \((m\cdot n)\cdot 1_F=(m\cdot 1_F)\cdot (n\cdot 1_F)=0\) there
will be two nonzero elements in \(F\) whose product is zero, but a field is an integral domain)
and \(p\) generates the kernel. Thus the map \(n\mapsto n\cdot 1_F:\Z\to F\) defines an isomorphism
from \(\Z/p\Z\) onto the subring
\begin{equation*}
\{m\cdot 1_F\mid m\in\Z\}
\end{equation*}
of \(F\). In this case, \(F\) contains a copy of \(\F_p\)

A field isomorphic to one of the fields \(\F_2,\F_3,\F_5,\dots,\Q\) is called a \textbf{prime field}. Every field
contains exactly one prime field (as a subfield)

A commutative ring \(R\) is said to have \textbf{characteristic} \(p\) (resp. 0) if it contains a prime
field (as a subring) of characteristic \(p\) (resp. 0). Then the prime field is unique and, by
definition, contains \(1_R\). Thus if \(R\) has characteristic \(p\neq 0\), then \(1_R+\dots+1_R=0\)
(\(p\) terms)

Let \(R\) be a nonzero commutative ring. If \(R\) has characteristic \(p\neq 0\), then
\begin{equation*}
pa:=\underbrace{a+\dots+a}_{p\text{ terms}}=\underbrace{(1_R+\dots+1_R)}_{p\text{ terms}}a=0a=0
\end{equation*}
for all \(a\in R\). Conversely, if \(pa=0\) for all \(a\in R\), then \(R\) has characteristic \(p\)

Let \(R\) be a nonzero commutative ring. The usual proof by induction shows that the binomial
theorem
\begin{equation*}
(a+b)^m=a^m+\binom{m}{1}a^{m-1}b+\binom{m}{2}a^{m-2}b^2+\dots+b^m
\end{equation*}
holds in \(R\). If \(p\) is prime, then it divides
\begin{equation*}
\binom{p}{r}:=\frac{p!}{r!(p-r)!}
\end{equation*}
for all \(r\) with \(1\le r\le p-1\). Therefore, when \(R\) has characteristic \(p\)
\begin{equation*}
(a+b)^p=a^p+b^p\quad\text{ for all }a,b\in R
\end{equation*}
and so the map \(a\mapsto a^p:R\to R\) is a homomorphism of rings (even of \(\F_p\)-algebras). It is
called the \textbf{Frobenius endomorphsim} of \(R\). The map \(a\mapsto a^{p^n}:R\to R\), \(n\ge 1\), is hte
composite of \(n\) copies of the Frobenius endomorphsim, and so it also is a homomorphism.
Therefore
\begin{equation*}
(a_1,\dots,a_m)^{p^n}=a_1^{p^n}+\dots+a_m^{p^n}
\end{equation*}
for all \(a_i\in R\).

When \(F\) is a field, the Frobenius endomorphsim is injective
\subsection{Review of polynomial rings}
\label{sec:orgfd465c9}
The \(F\)-algebra \(F[X]\) has the following universal property: for any \(F\)-algebra \(R\) and
element \(r\in R\), \(\exists!\) \(F\)-homomorphism \(\alpha:F[X]\to R\) s.t. \(\alpha(X)=r\)
\subsection{Factoring polynomials}
\label{sec:org6ea6252}
\begin{proposition}[]
Let \(r\in\Q\) be a root of a polynomial
\begin{equation*}
a_mX^m+a_{m-1}X^{m-1}+\dots+a_0,\quad a_i\in\Z
\end{equation*}
and write \(r=c/d\), \(c,d\in\Z\), \(\gcd(c,d)=1\). Then \(c\mid a_0\) and \(d\mid a_m\)
\end{proposition}

\begin{proof}
\begin{equation*}
a_mc^m+a_{m-1}c^{m-1}d+\dots+a_0d^m=0
\end{equation*}
\(d\mid a_mc^m\) and therefore \(d\mid a_m\). Similarly \(c\mid a_0\)
\end{proof}

\begin{examplle}[]
The polynomial \(f(X)=X^3-3X-1\) is irreducible in \(\Q[X]\) because its only possible roots
are \(\pm 1\) and \(f(1)\neq 0\neq f(-1)\)
\end{examplle}

\begin{proposition}[Gauss's Lemma]
Let \(f(X)\in\Z[X]\). If \(f(X)\) factors nontrivially in \(\Q[X]\), then it factors nontrivially in \(\Z[X]\)
\end{proposition}

\begin{proof}
Let \(f=gh\in\Q[X]\) with \(g,h\) nonconstant. For suitable integers \(m\) and \(n\), \(g_1:=mg\)
and \(h_1:=nh\) have coefficients in \(\Z\), so we have a factoriztion
\begin{equation*}
mnf=g_1\cdot h_1
\end{equation*}
in \(\Z[X]\). If a prime \(p\) divides \(mn\), then looking modulo \(p\), we obtain
\begin{equation*}
0=\bbar{g_1}\cdot\bbar{h_1}\in\F_p[X]
\end{equation*}
Since \(\F_p[X]\) is an integral domain, this implies that \(p\) divides all the coefficients of
at least one of the polynomials \(g_1,h_1\), say \(g_1\), so that \(g_1=pg_2\) for some \(g_2\in\Z[X]\).
Thus we have a factoriztion
\begin{equation*}
(mn/p)f=g_2\cdot h_1\in\Z[X]
\end{equation*}
Continuing in this fashion, we eventually remove all the prime factors of \(mn\).
\end{proof}

\begin{proposition}[]
If \(f\in\Z[X]\) is monic, then every monic factor of \(f\) in \(\Q[X]\) lies in \(\Z[X]\)
\end{proposition}

\begin{proof}
Let \(g\) be a monic factor of \(f\) in \(\Q[X]\), so that \(f=gh\) with \(h\in\Q[X]\) also monic.
Let \(m,n\) be the positive integers with the fewest prime factors s.t. \(mg,nh\in\Z[X]\). As in
the proof of Gauss's Lemma, if a prime \(p\) divides \(mn\), then it divides all the
coefficients of at least one of the polynomials \(mg,nh\), say \(mg\), in which case it
divides \(m\) because \(g\) is monic. Now \(\frac{m}{p}g\in\Z[X]\) which contradicts the definition
of \(m\).
\end{proof}

\begin{proposition}[Eisenstein's Criterion]
\label{1.16}
Let
\begin{equation*}
f=a_mX^m+\dots+a_0,\quad a_i\in\Z
\end{equation*}
suppose that there is a prime \(p\) s.t.
\begin{enumerate}
\item \(p\nmid a_m\)
\item \(p\mid a_i\)  for \(i=0,\dots,m-1\)
\item \(p^2\nmid a_0\)
\end{enumerate}


Then \(f\) is irreducible in \(\Q[X]\)
\end{proposition}

\begin{proof}
If \(f(X)\) factors nontrivially in \(\Q[X]\), then it factors nontrivially in \(\Z[X]\), say
\begin{equation*}
a_mX^m+\dots+a_0=(b_rX^r+\dots+b_0)(c_sX^s+\dots+c_0)
\end{equation*}
where \(b_i,c_i\in\Z\). Since \(p\), but not \(p^2\), divides \(a_0=b_0c_0\), \(p\) must divide
exactly one of \(b_0,c_0\), say \(b_0\). Now from the equation
\begin{equation*}
a_1=b_0c_1+b_1c_0
\end{equation*}
we see that \(p\mid b_1\), and from the equation
\begin{equation*}
a_2=b_0c_2+b_1c_1+b_2c_0
\end{equation*}
that \(p\mid b_2\). By continuing in this way, we find that \(p\) divides \(b_0,b_1,\dots,b_r\), which
contradicts the condition that \(p\) does not divide \(a_m\)
\end{proof}
\subsection{Extensions}
\label{sec:orgd67b04c}
Let \(F\) be a field. A field containing \(F\) is called an \textbf{extension} of \(F\). In other words,
an extension is an \(F\)-algebra whose underlying ring is a field. An extension \(E\) of \(F\)
is, in particular, an \(F\)-vector space, whose dimension is called the \textbf{degree} of \(E\)
over \(F\). It is denoted by \([E:F]\). An extension is \textbf{finite} if its degree is finite.

When \(E\) and \(E'\) are extensions of \(F\), an \textbf{\(F\)-homomorphism} \(E\to E'\) is a
homomorphism \(\varphi:E\to E'\) s.t. \(\varphi(c)=c\) for all \(c\in F\)

\begin{proposition}[Multiplicity of degrees]
\label{1.20}
Consider fields \(L\supset E\supset F\). Then \(L/F\) is of finite degree iff \(L/E\) and \(E/F\) are both
of finite degree, in which case
\begin{equation*}
[L:F]=[L:E][E:F]
\end{equation*}
\end{proposition}
\subsection{The subring generated by a subset}
\label{sec:org60bce99}
Let \(F\) be a subfield of a field \(E\) and let \(S\) be a subset of \(E\). The intersection of
all the subrings of \(E\) containing \(F\) and \(S\) is obviously the smallest subring of \(E\)
containing both \(F\) and \(S\). We call it the subring of \(E\) \textbf{generated by \(F\) and \(S\)}
(\textbf{generated over \(F\) by \(S\)}), and we denote it by \(F[S]\).

\begin{lemma}[]
The ring \(F[S]\) consists of the elements of \(E\) that can be expressed as finite sums of the
form
\begin{equation*}
\sum a_{i_1\cdots i_n}\alpha_1^{i_1}\cdots\alpha_n^{i_n},\quad a_{i_1\cdots i_n}\in F,\quad\alpha_i\in S,\quad i_j\in\N
\end{equation*}
\end{lemma}

\begin{lemma}[]
Let \(R\) be an integral domain containing a subfield \(F\) (as a subring). If \(R\) is
finite-dimensional when regarded as an \(F\)-vector space, then it is a field
\end{lemma}

\begin{proof}
Let \(\alpha\in R\) be nonzero. The map \(h:x\mapsto\alpha x\) is an injective linear map of
finite-dimensional \(F\)-vector spaces, and is therefore surjective. In particular, there is an
element \(\beta\in R\) s.t. \(\alpha\beta=1\)

\(\alpha x=\alpha y\), we need \(R\) to be integral domain to make \(x=y\)

Also for \(f\in R\), we need \(R\) to be a field to make \(\alpha fx=f\alpha x\)

Surjection is trivial
\end{proof}
\subsection{The subfield generated by a subset}
\label{sec:org2a9d039}
The intersection of all the subfields of \(E\) containing \(F\) and \(S\) is the smallest
subfield of \(E\) containing both \(F\) and \(S\). We call it the subfield of \(E\) \textbf{generated
by \(F\) and \(S\)}, and we denote it by \(F(S)\), it is the fraction field of \(F[S]\)

An extension \(E\) of \(F\) is \textbf{simple} if \(E=F(\alpha)\) for some \(\alpha\in E\)

Let \(F\) and \(F'\) be subfields of a field \(E\). The intersection of the subfields of \(E\)
containing both \(F\) and \(F'\) is obviously the smallest subfield of \(E\) containing
both \(F\)and \(F\). We call it the \textbf{composite} of \(F\) and \(F'\) in \(E\), and we denote it
by \(F\cdot F'\). It can also be described as the subfield of \(E\) generated over \(F\) by \(F'\),
or the subfield generated over \(F'\) by \(F\)
\begin{equation*}
F(F')=F\cdot F'=F'(F)
\end{equation*}
\subsection{Construction of some extensions}
\label{sec:orgf7ea449}
Let \(f(X)\in F(X)\) be a monic polynomial of degree \(m\). Consider the quotient \(F[X]/(f(X))\),
and write \(x\) for the image of \(X\) in \(F[X]/(f(X))\), i.e., \(x=X+(f(X))\)
\begin{enumerate}
\item The map
\begin{equation*}
P(X)\mapsto P(x):F[X]\to F[x]
\end{equation*}
is a homomorphism sending \(f(X)\) to 0, therefore \(f(x)=0\).
\(F[x]=F[X]/(f)\) since for each \(x^n=(X+(f(X))^n)=X^n+(f(X))\).
\item The division algorithm shows that every element \(g\in F[X]/(f)\)  is represented by a unique
polynomial \(r\) of degree \(<m\). Hence each element of \(F[x]\) can be expressed uniquely
as a sum
\begin{equation*}
a_0+a_1x+\dots+a_{m-1}x^{m-1},\quad a_i\in F
\end{equation*}
\item \emph{Now assume that \(f(X)\) is irreducible}. Then every nonzero \(\alpha\in F[x]\) has an inverse, which
can be found as follows. Use 2 to write \(\alpha=g(x)\) with \(g(X)\) a polynomial of
degree \(\le m-1\), and apply Euclid's algorithm in \(F[X]\) to find polynomials \(a(X)\)
and \(b(X)\) s.t.
\begin{equation*}
a(X)f(X)+b(X)g(X)=d(X)
\end{equation*}
with \(d(X)\) the gcd of \(f\) and \(g\). In our case, \(d(X)\) is 1 because \(f(X)\) is
irreducible and \(\deg g(X)<\deg f(X)\). When we replace \(X\) with \(x\), the equality
becomes
\begin{equation*}
b(x)g(x)=1
\end{equation*}
Hence \(b(x)\) is the inverse of \(g(x)\)
\end{enumerate}



We have proved the following statement
\begin{proposition}[]
\label{1.25}
For a monic irreducible polynomial \(f(X)\) of degree \(m\) in \(F[X]\)
\begin{equation*}
F[x]:=F[X]/(f(X))
\end{equation*}
is a field of degree \(m\) over \(F\). Computations in \(F[x]\) come down to computations
in \(F\)
\end{proposition}

Since \(F[x]\) is a field, \(F(x)=F[x]\)


\begin{examplle}[]
Let \(f(X)=X^2+1\in\R[X]\). Then \(\R[x]\) has elements \(a+bx,a,b\in\R\)

We usually write \(i\) for \(x\) and \(\C\) for \(\R[x]\)
\end{examplle}
\subsection{Stem fields}
\label{sec:orgfdb57cb}
Let \(f\) be a monic irreducible polynomial in \(F[X]\). A pair \((E,\alpha)\) consisting of an
extension \(E\) of \(F\) and an \(\alpha\in E\) is called a \textbf{stem field for} \(f\) if \(E=F[\alpha]\)
and \(f(\alpha)=0\). For example, the pair \((E,\alpha)\) with \(E=F[X]/(f)=F[x]\) and \(\alpha=x\).

Let \((E,\alpha)\) be a stem field, and consider the surjective homomorphism of \(F\)-algebras
\begin{equation*}
g(X)\to g(\alpha):F[X]\to E
\end{equation*}
Its kernel is generated by a nonzero monic polynomial, which divides \(f\), and so must equal
it. Therefore the homomorphism defines an \(F\)-isomorphism
\begin{equation*}
x\mapsto\alpha:F[x]\to E,\quad F[x]=F[X]/(f)
\end{equation*}
In other words, the stem field \((E,\alpha)\) of \(f\) is \(F\)-isomorphic to the standard stem field
\((F[X]/(f),x)\). It follows that every element of a stem field \((E,\alpha)\) for \(f\) can be
written uniquely in the form
\begin{equation*}
a_0+a_1\alpha+\dots+a_{m-1}\alpha^{m-1},\quad a_i\in F,\quad m=\deg(f)
\end{equation*}
and that arithmetic in \(F[\alpha]\) can be performed using the same rules in \(F[x]\).

\subsection{Algebraic and transcendental elements}
\label{sec:org2d8b367}

Let \(F\) be a field. An element \(\alpha\) of an extension \(E\) of \(F\) defines a homomorphism
\begin{equation*}
f(X)\mapsto f(\alpha):F[X]\to E
\end{equation*}
There are two possibilities:
\begin{enumerate}
\item Kernel is \((0)\), so that for \(f\in F[X]\)
\begin{equation*}
f(\alpha)=0\Rightarrow f=0 (\text{in }F[X])
\end{equation*}
In this case we say that \(\alpha\) \textbf{transcendental over} \(F\). The homomorphism \(X\mapsto\alpha\) is an
isomorphism, and it extends to an isppomorphism \(F(X)\to F(\alpha)\)
\item The kernel \(\neq(0)\), so that \(g(\alpha)=0\) for some nonzero \(g\in F[X]\). In this case, we say
that \(\alpha\) is \textbf{algebraic over} \(F\).  The polynomials \(g\) s.t. \(g(\alpha)=0\) form a nonzero ideal
in \(F[X]\), which is generated by the monic polynomial \(f\) of least degree
such \(f(\alpha)=0\). We call \(f\) the \textbf{minimal polynomial} of \(\alpha\) over \(F\).

Note that \(F[X]/(f)\cong F[\alpha]\), since the first is a field, so is the second
\end{enumerate}


\begin{examplle}[]
Let \(\alpha\in\C\) be s.t. \(\alpha^3-3\alpha-1=0\). Then \(X^3-3X-1\) is monic, irreducible in \(\Q[X]\) and  has
\(\alpha\) as a root, and so it is the minimal polynomial of \(\alpha\) over \(\Q\). The set \(\{1,\alpha,\alpha^2\}\) is a
basis for \(\Q[\alpha]\) over \(\Q\).
\end{examplle}

An extension \(E\) of \(F\) is \textbf{algebraic} (\(E\) is \textbf{algebraic over} \(F\)) if all elements
of \(E\) are algebraic over \(F\); otherwise it is said to be \textbf{transcendental}

\begin{proposition}[]
Let \(E\supset F\) be fields. If \(E/F\) is finite, then \(E\) is algebraic and finitely generated (as
a field) over \(F\); conversely if \(E\) is generated over \(F\) by a finite set of algebraic
elements, then it is finite over \(F\)
\end{proposition}

\begin{proof}
\(\Rightarrow\). \(\alpha\) of \(E\) is transcendental over \(F\) iff \(1,\alpha,\alpha^2,\dots\) are linearly independent
over \(F\) iff \(F[\alpha]\) is of infinite degree. Thus if \(E\) is finite over \(F\), then every
element of \(E\) is algebraic over \(F\). If \(E\neq F\), then we can pick \(\alpha_1\in E\setminus F\) and
compare \(E\) and \(F[\alpha_1]\). If \(E\neq F[\alpha_1]\), then there exists an \(\alpha_2\in E\setminus F[\alpha_1]\), and so on.
Since
\begin{equation*}
[F[\alpha_1]:F]<[F[\alpha_1,\alpha_2]:F]<\cdots<[E:F]
\end{equation*}
this process terminates with \(E=F[\alpha_1,\dots,\alpha_n]\)

\(\Leftarrow\): Let \(E=F(\alpha_1,\dots,\alpha_n)\) with \(\alpha_1,\dots,\alpha_n\) algebraic over \(F\). The extension \(F(\alpha_1)/F\)
is finite because \(\alpha_1\) is algebraic over \(F\). And \(F(\alpha_1,\alpha_2)/F\) is finite because \(\alpha_2\) is
algebraic over \(F\) and hence over \(F(\alpha_1)\). Thus by \ref{1.20} \(F(\alpha_1,\alpha_2)\) is finite over \(F\)
\end{proof}

\begin{corollary}[]
\begin{enumerate}
\item If \(E\) is algebraic over \(F\), then every subring \(R\) of \(E\) containing \(F\) is a field
\item Consider fields \(L\supset E\supset F\). If \(L\) is algebraic over \(E\) and \(E\) is algebraic
over \(F\), then \(L\) is algebraic over \(F\)
\end{enumerate}
\end{corollary}

\begin{proof}
\begin{enumerate}
\item If \(\alpha\in R\), then \(F[\alpha]\subset R\). But \(F[\alpha]\) is a field because \(\alpha\) is algebraic, and so \(R\)
contains \(\alpha^{-1}\)
\item By assumption, every \(\alpha\in L\) is a root of a monic polynomial
\begin{equation*}
X^m+a_{m-1}X^{m-1}+\dots+a_0\in E[X]
\end{equation*}
Each of the extensions
\begin{equation*}
F[a_0,\dots,a_{m-1},\alpha]\supset F[a_0,\dots,a_{m-1}]\supset\dots\supset F
\end{equation*}
is finite. Therefore \(F[a_0,\dots,a_{m-1},\alpha]\) is finite over \(F\), which implies that \(\alpha\) is
algebraic over \(F\)
\end{enumerate}
\end{proof}

\subsection{Transcendental numbers}
\label{sec:orgc472f35}
\begin{proposition}[]
The set of algebraic numbers is countable
\end{proposition}

\begin{theorem}[]
The number \(\alpha=\sum\frac{1}{2^{n!}}\) is transcendental
\end{theorem}

\subsection{Constructions with straight-edge and compass}
\label{sec:orga9a72c2}
A real number (length) is \textbf{constructible} if it can be constructed by forming successive intersections of
\begin{itemize}
\item lines drawn through two points already constructed
\item circles with center a point already constructed and radius a constructed length
\end{itemize}


This led them to three famous questions: is it possible to
duplicate the cube, trisect an angle, or square the circle by straight-edge and compass
constructions? We’ll see that the answer to all three is negative.

Let \(F\) be a subfield of \(\R\). For a positive \(a\in F\), The \textbf{\(F\)-plane} is \(F\times F\subset\R\times\R\)

An \textbf{\(F\)-line} is a line in \(\R\times\R\) through two points in the \(F\)-plane. These are the lines
given by equations
\begin{equation*}
ax+by+c=0,\quad a,b,c\in F
\end{equation*}
An \textbf{\(F\)-circle} is a circle in \(\R\times\R\) with center an \(F\)-point and radius an element
of \(F\). These are the circles given by the equations
\begin{equation*}
(x-a)^2+(y-b)^2=c^2,\quad a,b,c\in F
\end{equation*}

\begin{lemma}[]
Let \(L\neq L'\) be \(F\)-lines, and let \(C\neq C'\) be \(F\)-circles
\begin{enumerate}
\item \(L\cap L'=\emptyset\) or consists of a single \(F\)-point
\item \(L\cap C=\emptyset\) or consists of one or two points in the \(F[\sqrt{e}]\)-plane, some \(e\in F\), \(e>0\)
\item \(C\cap C'=\emptyset\) or consists of one or two poitns in the \(F[\sqrt{e}]\)-plane, some \(e\in F\), \(e>0\)
\end{enumerate}
\end{lemma}

\begin{lemma}[]
\begin{enumerate}
\item If \(c\) and \(d\) are constructive, then so also are \(c+d,-c,cd\)
and \(\frac{c}{d}\), \(d\neq 0\)
\item If \(c>0\) is constructible, then so is \(\sqrt{c}\)
\end{enumerate}
\end{lemma}

\begin{proof}
First show that it is possible to construct a line perpendicular to a given line through a given
point (\href{https://qb.zuoyebang.com/xfe-question/question/ccd48609d90b8dd2e9aef5b5abaf3fe2.html}{link}), and then a line parallel to a given line through a given point (\href{https://zhidao.baidu.com/question/239781112?from=\&ssid=\&uid=bd\_1458321853\_37\&pu=sz\%40224\_240\%2Cos\%40\&fr=solved\&step=22\&bd\_page\_type=1\&init=middle}{link}). Hence it is
possible to construct a triangle similar to a given one on a side with given length.


\(\sqrt{c}\) \href{https://zhidao.baidu.com/question/553468349.html}{link}
\end{proof}

\begin{theorem}[]
\begin{enumerate}
\item The set of constructible numbers is a field
\item A number \(\alpha\) is constructible iff it is contained in a subfield of \(\R\) of the form
\begin{equation*}
\Q[\sqrt{a_1},\dots,\sqrt{a_r}],\quad a_i\in\Q[\sqrt{a_1},\dots,\sqrt{a_{i-1}}],\quad a_i>0
\end{equation*}
\end{enumerate}
\end{theorem}

\begin{corollary}[]
If \(\alpha\) is constructible, then \(\alpha\) is algebraic over \(\Q\), and \([\Q[\alpha]:\Q]\) is a power of 2
\end{corollary}

\begin{proof}
\([\Q[\alpha]:\Q]\) divides \([\Q[\sqrt{a_1}]\dots[\sqrt{a_r}]:\Q]\) and \([\Q[\sqrt{a_1},\dots,\sqrt{a_r}]:\Q]\) is a
power of 2
\end{proof}

\begin{corollary}[]
It is impossible to duplicate the cube by straight-edge and compass constructions
\end{corollary}

\begin{proof}
This requires constructing the real root of the polynomial \(X^3-2\). But this polynomial is
irreducible and \([\Q[\sqrt[3]{2}]:\Q]=3\)
\end{proof}

\begin{corollary}[]
In general, it is impossible to trisect an angle by straight-edge and compass constructions
\end{corollary}

\begin{proof}
Knowing an angle is equivalent to knowing the cosine of the angle. Therefore, to trisect \(3\alpha\),
we have to construct a solution to
\begin{equation*}
\cos3\alpha=4\cos^3\alpha-3\cos\alpha
\end{equation*}
For example take \(3\alpha=\ang{60}\). As \(\cos\ang{60}=0.5\), we have to solve \(8x^3-6x-1=0\),
which is irreducible, and so \([\Q[\alpha]:\Q]=3\)
\end{proof}

\begin{corollary}[]
It is impossible to square the circle by straight-edge and compass constructions
\end{corollary}

\begin{proof}
A square with the same area as a circle of radius \(r\) has side \(\sqrt{\pi}r\). Since \(\pi\) is
transcendental, so also is \(\sqrt{\pi}\)
\end{proof}

\(X^p-1=(X-1)(X^{p-1}+X^{p-2}+\dots+1)\)

\begin{lemma}[]
If \(p\) is prime, then \(X^{p-1}+\cdots+1\) is irreducible; hence \(\Q[e^{2\pi i/p}]\) has
degree \(p-1\) over \(\Q\)
\end{lemma}

\begin{proof}
Let \(f(X)=(X^p-1)/(X-1)=X^{p-1}+\dots+1\); then
\begin{equation*}
f(X+1)=\frac{(X+1)^p-1}{X}=X^{p-1}+\dots+a_iX^i+\dots+p
\end{equation*}
with \(a_i=\binom{p}{i+1}\)

\(p\mid a_i\) for \(i=1,\dots,p-2\), and so \(f(X+1)\) is irreducible by Eisenstein's criterion
\ref{1.16}. This implies that \(f(X)\) is irreducible
\end{proof}

\subsection{Algebraically closed fields}
\label{sec:orged8878c}
Let \(F\) be a field. A polynomial is said to \textbf{split} in \(F[X]\) if it is a product of
polynomials of degree at most 1 in \(F[X]\)

\begin{proposition}[]
\label{1.42}
For a field \(\Omega\), TFAE
\begin{enumerate}
\item Every nonconstant polynomial in \(\Omega[X]\) splits in \(\Omega[X]\)
\item Every nonconstant polynomial in \(\Omega[X]\) has at least one root in \(\Omega\)
\item The irreducible polynomials in \(\Omega[X]\) are those of degree 1
\item Every field of finite degree over \(\Omega\) equals \(\Omega\)
\end{enumerate}
\end{proposition}

\begin{proof}
\(3\to 4\): Let \(E\) be a finite extension of \(\Omega\), and let \(\alpha\in E\). The minimal polynomial of \(\alpha\),
being irreducible, has degree 1, and so \(\alpha\in\Omega\)

\(4\to 3\): Let \(f\) be an irreducible polynomial of \(\Omega\), then \(\Omega[X]/(f)\) is an extension of
\(\Omega\) of degree \(\deg(f)\), and so \(\deg(f)=1\)
\end{proof}

\begin{definition}[]
\begin{enumerate}
\item A field \(\Omega\) is \textbf{algebraically closed} if it satisfies the equivalent statements in Proposition \ref{1.42}
\item A field \(\Omega\) is an \textbf{algebraic closure} of a subfield \(F\) if it is algebraically closed and
algebraic over \(F\)
\end{enumerate}
\end{definition}

\begin{proposition}[]
\label{1.44}
If \(\Omega\) is algebraic over \(F\) and every polynomial \(f\in F[X]\) splits in \(\Omega[X]\), then \(\Omega\) is
algebraically closed
\end{proposition}

\begin{proof}
Let \(f\) be a nonconstant polynomial in \(\Omega[X]\). We know (\ref{1.25}) that \(f\) has a root
\(\alpha\) in some finite extension \(\Omega'\) of \(\Omega\). Set
\begin{equation*}
f=a_nX^n+\dots+a_0,\quad a_i\in\Omega
\end{equation*}
and consider the fields
\begin{equation*}
F\subset F[a_0,\dots,a_n]\subset F[a_0,\dots,a_n,\alpha]
\end{equation*}
Each extension generated by a finite set of algebraic elements, and hence is finite (\ref{1.30})
Therefore \(\alpha\) lies in a finite extension of \(F\) and so is algebraic over \(F\) - it is a root of
a polynomial \(g\) with coefficients in \(F\). By assumption, \(g\) splits in \(\Omega[X]\), and so
the root of \(g\) in \(\Omega'\) all lie in \(\Omega\). In particular, \(\alpha\in\Omega\)
\end{proof}

\begin{proposition}[]
Let \(\Omega\supset F\), then
\begin{equation*}
\{\alpha\in\Omega\mid\alpha\text{ algebraic over }F\}
\end{equation*}
is a field
\end{proposition}

\begin{proof}
If \(\alpha\) and \(\beta\) are algebraic over \(F\), then \(F[\alpha,\beta]\) is a field of finite degree over \(F\).
Thus every element of \(F[\alpha,\beta]\) is algebraic over \(F\), in particular \(\alpha\pm\beta\), \(\alpha/\beta\)
and \(\alpha\beta\) are algebraic over \(F\)
\end{proof}

The field constructed in the proposition is called the \textbf{algebraic closure of \(F\) in \(\Omega\)}

\begin{corollary}[]
\(\Omega\vDash\ACF\), for any subfield \(F\) of \(\Omega\), the algebraic closure \(E\) of \(F\)  in \(\Omega\) is an
algebraic closure of \(F\)
\end{corollary}

\begin{proof}
It is algebraic over \(F\) by definition. Every polynomial in \(F[X]\) splits in \(\Omega[X]\) and
has its roots in \(E\), and so splits in \(E[X]\). Now apply Proposition \ref{1.44}
\end{proof}

\subsection{Exercises}
\label{sec:org9e514ff}
\begin{enumerate}
\item \(f(x)=x^3-\alpha^2+\alpha+2\), \(f(x)\) is irreducible in \(\Q[x]\). Thus \(\Q[\alpha]\cong\Q[x]/(f)\), which is a
field

\((\alpha-1)^{-1}=-\frac{1}{3}(\alpha^2+1)\)

\item 4

\item ​
\begin{enumerate}
\item \(f(X)-f(a)=q(X)(X-a)+r(X)\) and \(\deg r<1\), hence \(\deg r=0\)
\item obvious
\item obvious
\end{enumerate}
\setcounter{enumi}{4}
\item Let \(g\) be the irreducible factor in \(E[X]\) and let \((L,\alpha)\) be a stem field
for \(g\) over \(E\). Then \(L=E[\alpha]\cong E/(f)\). Then \(m\mid[E[\alpha]:F]\).
Since \(f(\alpha)=0\). \([F[\alpha]:F]=n\). Now \(n\mid[L:F]\). We deduce that \([L:F]=mn\)
and \([L:E]=n\). But \([E[\alpha]:E]=\deg(g)\). Hence \(\deg(g)=\deg(f)\)
\begin{center}\begin{tikzcd}
E[\alpha]\ar[r,dash,"\le n"]\ar[d,dash]&E\ar[r,dash,"m"]&F\\
F[\alpha]\ar[d,dash,"n"]\\
F
\end{tikzcd}\end{center}
\item The polynomials \(f(X)-1\) and \(f(X)+1\) have only finitely many roots, and so there
is \(n\in\Z\) s.t. \(f(n)\neq\pm 1\), then there is prime \(p\) s.t. \(p\mid f(n)\). Hence \(f(x)\) is
reducible in \(\F_p[x]\)
\item Let \(f(x)=x^3-2\), then \(R\cong\Q[x]/(f)\).
\end{enumerate}




\section{Splitting Fields; Multiple Roots}
\label{sec:org9dd4329}
\subsection{Homomorphisms from simple extensions}
\label{sec:org2dc6c09}
Let \(F\) be a field and \(E,E'\) fields containing \(F\). Recall that an \(F\)-homomorphism is
a homomorphim \(\varphi:E\to E'\) s.t. \(\varphi(a)=a\) for all \(a\in F\). Thus an \(F\)-homomorphism \(\varphi\) maps a
polynomial
\begin{equation*}
\sum a_{i_1\dots i_m}\alpha^{i_1}_1\dots\alpha_m^{i_m},\quad a_{i_1\dots i_m}\in F,\quad \alpha_i\in E
\end{equation*}
to
\begin{equation*}
\sum a_{i_1\dots i_m}\varphi(\alpha_1)^{i_1}\dots\varphi(\alpha_m)^{i_m}
\end{equation*}
An \textbf{\(F\)-isomorphism} is a bijective \(F\)-homomorphism

An \(F\)-homomorphism \(E\to E'\) of fields is, in particular, an injective \(F\)-linear map
of \(F\)-vector spaces, and so it is an \(F\)-isomorphism if \(E\) and \(E'\) have the same
finite degree over \(F\)

\begin{proposition}[]
\label{2.1}
Let \(F(\alpha)\) be a simple extension of \(F\) and \(\Omega\) a second extension of \(F\)
\begin{enumerate}
\item Let \(\alpha\) be transcendental over \(F\). For every \(F\)-homomorphism \(\varphi:F(\alpha)\to\Omega\), \(\varphi(\alpha)\) is
transcendental over \(F\), and the map \(\varphi\mapsto\varphi(\alpha)\) defines a one-to-one correspondence
\begin{equation*}
\{F\text{-homomorphisms } F(\alpha)\to\Omega\}\leftrightarrow\{\text{elements of $\Omega$ transcendental over }F\}
\end{equation*}
\item Let \(\alpha\) be algebraic over \(F\) with minimal polynomial \(f(X)\). For
every \(F\)-homomorphism \(\varphi:F[\alpha]\to\Omega\), \(\varphi(\alpha)\) is a root of \(f(X)\) in \(\Omega\), and the
map \(\varphi\mapsto\varphi(\alpha)\) defines a one-to-one correspondence
\begin{equation*}
\{F\text{-homomorphisms }\varphi:F[\alpha]\to\Omega\}\leftrightarrow\{\text{roots of $f$ in }\Omega\}
\end{equation*}
In particular, the number of such maps is the number of distinct roots of \(f\) in \(\Omega\)
\end{enumerate}
\end{proposition}

\begin{proof}
\begin{enumerate}
\item To say that \(\alpha\) is transcendental over \(F\) means that \(F[\alpha]\) is isomorphic to the
polynomial ring in the symbol \(\alpha\). Therefore for every \(\gamma\in\Omega\), there is a
unique \(F\)-homomorphism \(\varphi:F[\alpha]\to\Omega\) s.t. \(\varphi(\alpha)=\gamma\). This \(\varphi\) extends (uniquely) to the
field of fractions \(F(\alpha)\) iff nonzero elements of \(F[\alpha]\) are sent to nonzero elements of
\(\Omega\), which is the case iff \(\gamma\) is transcendental over \(F\). Thus there is a one-to-one
correspondence between
\begin{enumerate}
\item \(F(\alpha)\to\Omega\)
\item \(\varphi:F[\alpha]\to\Omega\) s.t. \(\varphi(\alpha)\) is transcendental
\item the transcendental elements of \(\Omega\)
\end{enumerate}
\item If \(\gamma\in\Omega\) is a root of \(f(X)\), then the map \(F[X]\to\Omega\), \(g(X)\mapsto g(\gamma)\), factor
through \(F[X]/(f(X))\). When composed with the inverse of the canonical
isomorphism \(F[\alpha]\to F[X]/(f(X))\), this becomes a homomorphism \(F[\alpha]\to\Omega\) sending \(\alpha\) to \(\gamma\)
\end{enumerate}
\end{proof}

\begin{proposition}[]
\label{2.2}
Let \(F(\alpha)\) be a simple extension of \(F\) and \(\varphi_0:F\to\Omega\) a homomorphism from \(F\) into a
second field \(\Omega\)
\begin{center}\begin{tikzcd}
F(\alpha)\ar[rr,dash]&&\Omega\\
F\ar[u,dash]\ar[rd,dash]\ar[rr,dash]&&\varphi_0(F)\ar[ld,dash]\ar[u,dash]\\
&F
\end{tikzcd}\end{center}

\begin{enumerate}
\item if \(\alpha\) is transcendental over \(F\), then the map \(\varphi\mapsto\varphi(\alpha)\) defines a one-to-one
correspondence
\begin{equation*}
\{\text{extensions }\varphi:F(\alpha)\to\Omega\text{ of }\varphi_0\}\leftrightarrow\{\text{elements of $\Omega$ transcendental over $\varphi_0(F)$}\}
\end{equation*}
\item If \(\alpha\) is algebraic over \(F\), with minimal polynomial \(f(X)\), then the map \(\varphi\mapsto\varphi(\alpha)\)
defines a one-to-one correspondence
\begin{equation*}
\{\text{extensions }\varphi:F[\alpha]\to\Omega\text{ of }\varphi_0\}\leftrightarrow\{\text{roots of }\varphi_0f\text{ in }\Omega\}
\end{equation*}
\end{enumerate}
\end{proposition}
\subsection{Splitting fields}
\label{sec:org6852841}
Let \(f\) be a polynomial with coefficients in \(F\). A field \(E\supseteq F\) is said to \textbf{split} \(f\)
if \(f\) splits in \(E[X]\), i.e.,
\begin{equation*}
f(X)=a\prod_{i=1}^m(X-\alpha_i),\quad\alpha_i\in E
\end{equation*}
If \(E\) splits \(f\) and is generated by the roots of \(f\)
\begin{equation*}
E=F[\alpha_1,\dots,\alpha_m]
\end{equation*}
then it is called a \textbf{splitting} or \textbf{root field} for \(f\)

\begin{proposition}[]
\label{2.4}
Every polynomial \(f\in F[X]\) has a splitting field \(E_f\), and
\begin{equation*}
[E_f:F]\le(\deg f)!
\end{equation*}
\end{proposition}

\begin{proof}
Let \(F_1=F[\alpha_1]\) be a stem field for some monic irreducible factor of \(f\) in \(F[X]\).
Then \(f(\alpha_1)=0\), and we let \(F_2=F_1[\alpha_2]\) be a stem field for some monic irreducible factor
of \(f(X)/(X-\alpha_1)\) in \(F_1[X]\). Continuing in this fashion, we arrive at a splitting
field \(E_f\). Let \(n=\deg f\). Then \([F_1:F]=\deg g_1\le n\), \([F_2:F_1]\le n-1\), and
so \([E_f:F]\le n!\)
\end{proof}

\begin{examplle}[]
\label{2.6}
\begin{enumerate}
\item Let \(f(X)=(X^p-1)/(X-1)\in\Q[X]\), \(p\) prime. If \(\xi\) is one root of \(f\), then the remaining
roots are \(\xi^2,\xi^3,\dots,\xi^{p-1}\), and so the splitting field of \(f\) is \(\Q[\xi]\)
\item Let \(F\) have characteristic \(p\neq 0\), and let \(f=X^p-X-a\in F[X]\). If \(\alpha\) is one root of \(f\)
in some extension of \(F\), then the remaining roots are \(\alpha+1,\dots,\alpha+p-1\), and so the
splitting field of \(f\) is \(F[\alpha]\)
\item If \(\alpha\) is one root of \(X^n-a\), then the remaining roots are all of the form \(\xi\alpha\),
where \(\xi^n=1\). Therefore \(F[\alpha]\) is a splitting field for \(X^n-a\) iff \(F\) contains all
the \(n\)th roots of 1. Note that if \(p\) is the characteristic of \(F\),
then \(X^p-1=(X-1)^p\), and so \(F\) automatically contains all the \(p\)th roots of 1
\end{enumerate}
\end{examplle}

\begin{proposition}[]
\label{2.7}
Let \(f\in F[X]\). Let \(E\) be the extension of \(F\) generated by the roots of \(f\) in \(E\),
and let \(\Omega\) be an extension of \(F\) splitting \(f\)
\begin{enumerate}
\item There exists an \(F\)-homomorphism \(\varphi:E\to\Omega\); the number of such homomorphisms is at
most \([E:F]\), and equals \([E:F]\) if \(f\) has distinct roots in \(\Omega\)
\item If \(E\) and \(\Omega\) are both splitting fields for \(f\), then every \(F\)-homomorphism \(E\to\Omega\) is
an isomorphism. In particular, any two splitting fields for \(f\) are \(F\)-isomorphic
\end{enumerate}
\end{proposition}

\begin{proof}
We may assume that \(f\) is monic

Let \(F,f,\Omega\) be as in the statement of the proposition, let \(L\) be a subfield of \(\Omega\)
containing \(F\), and let \(g\) be a monic factor of \(f\) in \(L[X]\); as \(g\) divides \(f\)
in \(\Omega[X]\), it is a product of certain number of the factors \(X-\beta_i\) of \(f\) in \(\Omega[X]\); in
particular, we see that \(g\) splits in \(\Omega\), and that it has distinct roots in \(\Omega\) if \(f\) does

\begin{enumerate}
\item \(E=F[\alpha_1,\dots,\alpha_m]\), each \(\alpha_i\) a root of \(f(X)\) in \(E\). The minimal polynomial of \(\alpha_1\)
is an irreducible polynomial \(f_1\) dividing \(f\). From the initial observation
with \(L=F\), we see that \(f_1\) splits in \(\Omega\), and that its roots are distinct if the roots
of \(f\) are distinct. According to Proposition \ref{2.1}, there exists
an \(F\)-homomorphism \(\varphi_1:F[\alpha_1]\to\Omega\) and the number of such homomorphisms is at
most \([F[\alpha_1]:F]\), with equality holding when \(f\) has distinct roots in \(\Omega\)

The minimal polynomial of \(\alpha_2\) over \(F[\alpha_1]\) is an irreducible factor \(f_2\) of \(f\)
in \(F[\alpha_1][X]\). On applying the initial observation with \(L=\varphi_1F[\alpha_1]\) and \(g=\varphi_1f_2\) we see
that \(\varphi_1f_2\) splits in \(\Omega\). According to Proposition \ref{2.2}, each \(\varphi_1\) extends to a
homomorphism \(\varphi_2:F[\alpha_1,\alpha_2]\to\Omega\), and the number of extensions is at most \([F[\alpha_1,\alpha_2]:F[\alpha_1]]\),
with equality holding when \(f\) has distinct roots in \(\Omega\)

On combining these statements we conclude that there exists an \(F\)-homomorphism
\begin{equation*}
\varphi:F[\alpha_1,\alpha_2]\to\Omega
\end{equation*}
and that the number of such homomorphisms is at most \([F[\alpha_1,\alpha_2]:F]\), with equality holding
if \(f\) has distinct roots in \(\Omega\)

\item Every \(F\)-homomorphism \(E\to\Omega\) is injective
\wu{
if \(\alpha_1\neq\alpha_2\), then \(\alpha_1\) is not a root of \(f_2\), otherwise \(f_2\) is not minimal
in \(F[\alpha_1][X]\). Thus \(f_2(\varphi_2\alpha_2)=0\neq f_2(\varphi_2\alpha_1)\), and so \(\varphi_2\alpha_2\neq\varphi_2\alpha_1\). Thus
every \(F\)-homomorphism is injective.
}
And so, if there exists such a homomorphism,
then \([E:F]\le[\Omega:F]\). If \(E\) and \(\Omega\) are both splitting fields for \(f\), then 1 shows that
there exist homomorphism \(E\leftrightarrows\Omega\), and so \([E:F]=[\Omega:F]\)
\end{enumerate}
\end{proof}

\begin{corollary}[]
\label{2.8}
Let \(E\) and \(L\) be extension of \(F\), with \(E\) finite over \(F\)
\begin{enumerate}
\item The number of \(F\)-homomorphisms \(E\to L\) is at most \([E:F]\)
\item There exists a finite extension \(\Omega/L\) and an \(F\)-homomorphism \(E\to\Omega\)
\end{enumerate}
\end{corollary}

\begin{proof}
Write \(E=F[\alpha_1,\dots,\alpha_m]\), and let \(f\in F[X]\) be the product of the minimal polynomials of
the \(\alpha_i\); thus \(E\) is generated over \(F\) by roots of \(f\). Let \(\Omega\) be a splitting field
for \(f\) regarded as an element of \(L[X]\). The proposition shows that there exists
an \(F\)-homomorphism \(E\to\Omega\), and the number of such homomorphisms is \(\le[E:F]\). This proves
(2). And since an \(F\)-homomorphism \(E\to L\) can be regarded as an \(F\)-homomorphism \(E\to\Omega\),
it also proves (1)
\end{proof}

\begin{remark}
\begin{enumerate}
\item Let \(E_1,\dots,E_m\) be finite extensions of \(F\), and let \(L\) be an extension of \(F\). From
the corollary we see that there exists a finite extension \(L_1/L\) s.t. \(L_1\) contains an
isomorphic image of \(E_1\); then there exists a finite extension \(L_2/L_1\) s.t. \(L_2\)
contains an isomorphic image of \(E_2\). Finally we can find a finite extension \(\Omega/L\) s.t.
\(\Omega\) contains an isomorphic copy of each \(E_i\)
\item 
\end{enumerate}
\end{remark}
\subsection{Multiple roots}
\label{sec:org7bdb721}
Even when polynomials in \(F[X]\) have no common factor in \(F[X]\), one might expect that they
could acquire a common factor in \(\Omega[X]\) for some \(\Omega\supset F\). In fact, this doesn't happen

\begin{proposition}[]
\label{2.10}
Let \(f\) and \(g\) be polynomials in \(F[X]\), and let \(\Omega\) be an extension of \(F\). If \(r(X)\)
is the gcd of \(f\) and \(g\) computed in \(F[X]\), then it is also the gcd of \(f\) and \(g\)
in \(\Omega[X]\). In particular, distinct monic irreducible polynomials in \(F[X]\) do not acquire a
common root in any extension of \(F\)
\end{proposition}

\begin{proof}
Let \(r_F(X)\) and \(r_\Omega(X)\) be the greatest common divisors of \(f\) and \(g\) in \(F[X]\)
and \(\Omega[X]\) respectively. Certainly \(r_F(X)\mid r_\Omega(X)\) in \(\Omega[X]\), but Euclid's algorithm
shows that there are polynomials \(a\) and \(b\) in \(F[X]\) s.t.
\begin{equation*}
a(X)f(X)+b(X)g(X)=r_F(X)
\end{equation*}
and so \(r_\Omega(X)\) divides \(r_F(X)\) in \(\Omega[X]\)
\end{proof}

The proposition allows us to speak of the gcd of \(f\) and \(g\) without reference to a field

Let \(f\in F[X]\), then \(f\) splits into linear factors
\begin{equation*}
f(X)=a\prod_{i=1}^r(X-\alpha_i)^{m_i},\alpha_i\text{ distinct}, m_i\ge 1,\sum_{i=1}^rm_i=\deg(f)
\end{equation*}
in \(E[X]\) for some extension \(E\) of \(F\) (\ref{2.4}). We say that \(\alpha_i\) is a root of \(f\)
of \textbf{multiplicity} \(m_i\) in \(E\). If \(m_i>1\), then \(\alpha_i\) is said to be a \textbf{multiple root}
of \(f\), and otherwise it is a \textbf{simple root}

Let \(E\) and \(E'\) be splitting fields for \(F\), and suppose
that \(f(X)=a\prod_{i=1}^r(X-\alpha_i)^{m_i}\) in \(E[X]\) and \(f(X)=a'\prod_{i=1}^{r'}(X-\alpha_i')^{m_i'}\)
in \(E'[X]\). Let \(\varphi:E\to E'\) be an \(F\)-isomorphism, which exists by \ref{2.7}, and extend it to
an isomorphism \(E[X]\to E'[X]\) by sending \(X\) to \(X\). Then \(\varphi\) maps the factorization of \(f\)
in \(E[X]\) onto a factorization
\begin{equation*}
f(X)=\varphi(a)\prod_{i=1}^r(X-\varphi(\alpha_i))^{m_i}
\end{equation*}
in \(E'[X]\). By unique factorization, this coincides with the earlier factorization
in \(E'[X]\) up to a renumbering of the \(\alpha_i\). Therefore \(r=r'\) and
\begin{equation*}
\{m_1,\dots,m_r\}=\{m_1',\dots,m_r'\}
\end{equation*}

\(f\) \textbf{has a multiple root} when at least one of the \(m_i>1\), and that \(f\) has \textbf{only simple
roots} when all \(m_i=1\). Thus ``\(f\) has a multiple root'' means ``\(f\) has a multiple root in
one, hence every, extension of \(F\) splitting \(f\)'', and similarly for ``\(f\) has only simple roots''

When will an irreducible polynomial has a multiple root

\begin{examplle}[]
\label{2.11}
Let \(F\) be of characteristic \(p\neq 0\), and assume that \(F\) contains an element \(a\) that is
not a \(p\)th-power, \(a=T\) in the field \(\F_p(T)\). Then \(X^p-a\) is irreducible,
but \(X^p-a=(X-\alpha)^p\) in its splitting field. Thus an irreducible polynomial can have multiple roots
\end{examplle}

The derivative of a polynomial \(f(X)=\sum a_iX^i\) is defined to be \(f'(X)=\sum ia_iX^{i-1}\).

\begin{proposition}[]
\label{2.12}
For a nonconstant irreducible polynomial \(f\) in \(F[X]\), TFAE
\begin{enumerate}
\item \(f\) has a multiple root
\item \(\gcd(f,f')\neq 1\)
\item \(F\) has nonzero characteristic \(p\) and \(f\) is a polynomial in \(X^p\)
\item all the roots of \(f\) are multiple
\end{enumerate}
\end{proposition}

\begin{proof}
\(2\to 3\): as \(f\) is irreducible and \(\deg(f')<\deg(f)\), \(f'=0\) in \(F\).

\(3\to 4\). \(f(X)=g(X^p)\). Suppose \(g(X)=\prod_i(X-a_i)^{m_i}\) in some extension field. Then
\(f(X)=g(X^p)=\prod_i(X^p-a_i)=\prod_i(X-a_i)^{pm_i}\)
\end{proof}

\begin{proposition}[]
For a nonzero polynomial \(f\in F[X]\), TFAE
\label{2.13}
\begin{enumerate}
\item \(\gcd(f,f')=1\) in \(F[X]\)
\item \(f\) only has simple roots
\end{enumerate}
\end{proposition}

\begin{proof}
Let \(\Omega\) be an extension of \(F\) splitting \(f\). If a root \(\alpha\) of \(f\) in \(\Omega\) is multiple iff it is
also a root of \(f'\)
\end{proof}

\begin{definition}[]
A polynomial is \textbf{separable} if it is nonzero and satisfied the equivalent conditions in \ref{2.13}
\end{definition}

\begin{definition}[]
A field \(F\) is \textbf{perfect} if it has characteristic zero  or it has characteristic \(p\) and every
element of \(F\) is a \(p\)th power
\end{definition}

Thus \(F\) is perfect iff \(F=F^p\)

\begin{proposition}[]
A field \(F\) is perfect iff every irreducible polynomial in \(F[X]\) is separable
\end{proposition}

\begin{proof}
If \(F\) has characteristic 0, the statement is obvious. If \(F\) has characteristic \(p\neq 0\).
If \(F\) contains an element \(a\) that is not a \(p\)th power, then \(X^p-a\) is irreducible
in \(F[X]\) but not separable

If \(F\) is perfect and \(f\) is not separable, then \(f\) is a polynomial in \(X^p\). Then \(f\)
can't be irreducible

If every element of \(F\) is a \(p\)th power, then every polynomial in \(X^p\) with coefficients
in \(F\) is a \(p\)th power in \(F[X]\)
\begin{equation*}
\sum a_iX^{ip}=(\sum b_iX^i)^p,\quad a_i=b_i^p
\end{equation*}
and so it is not irreducible
\end{proof}

\begin{examplle}[]
\label{Problem1}
\begin{enumerate}
\item A finite field \(F\) is perfect, because the Frobenius endomorphism \(a\mapsto a^p:F\to F\) is
injective and therefore surjective
\item A field that can be written as a union of perfect fields is perfect. Therefore, every field
algebraic over \(\F_p\) is perfect
\item Every algebraically closed field is perfect
\item If \(F_0\) has characteristic \(p\neq 0\), then \(F=F_0(X)\) is not perfect, because \(X\) is
not a \(p\)th power
\end{enumerate}
\end{examplle}
\subsection{Exercises}
\label{sec:org9e7e451}
\begin{exercise}
\label{ex2.1}
Let \(F\) be a field of characteristic \(\neq 2\)
\begin{enumerate}
\item Let \(E\) be a quadratic extension of \(F\); show that
\begin{equation*}
S(E)=\{a\in F^{\times}\mid a\text{ is a square in }E\}
\end{equation*}
is a subgroup of \(F^{\times}\) containing \(F^{\times 2}\)
\item Let \(E\) and \(E'\) be quadratic extension of \(F\); show that there exists
an \(F\)-isomorphism \(\varphi:E\to E'\) iff \(S(E)=S(E')\)
\item Show that there is an infinite sequence of fields \(E_1,E_2,\dots\) with \(E_i\) a quadratic
extension of \(\Q\) s.t. \(E_i\) is not isomorphic to \(E_j\) for \(i\neq j\)
\item Let \(p\) be an odd prime. Show that, up to isomorphism, there is exactly one field
with \(p^2\) elements
\end{enumerate}
\end{exercise}

\begin{exercise}
\label{ex2.3}
Construct a splitting field for \(X^5-2\) over \(\Q\). What is its degree over \(\Q\)

\ref{2.6}
\end{exercise}

\begin{exercise}
\label{ex2.2}
\begin{enumerate}
\item Let \(F\) be a field of characteristic \(p\). Show that if \(X^p-X-a\) is reducible
in \(F[X]\), then it splits into distinct factors in \(F[X]\)
\item For every prime \(p\), show that \(X^p-X-1\) is irreducible in \(\Q[X]\)
\end{enumerate}
\end{exercise}

\begin{proof}
\(x^5-2\) is irreducible in \(\Q\)

Let \(\xi^5=1\), and \(\alpha=\sqrt[5]{2}\), then the five solutions are \(\alpha,\xi\alpha,\xi^2\alpha,\xi^3\alpha,\xi^4\alpha\). Note
that \([\Q[\alpha]:\Q]=5\) and \([\Q[\xi]:\Q]=4\). Then \([\Q[\alpha,\xi]:\Q[\alpha]]\le 4\). Hence \([\Q[\alpha,\xi]:\Q]=20\)
\end{proof}

\begin{exercise}
\label{ex2.4}
Find a splitting field of \(X^{p^m}-1\in\F_p[X]\). What is its degree over \(\F_p\)
\end{exercise}

\begin{exercise}
\label{2.5}
Let \(f\in F[X]\), where \(F\) is a field of characteristic 0. Let \(d(X)=\gcd(f,f')\). Show
that \(g(X)=f(X)d(X)^{-1}\) has the same roots as \(f(X)\), and these are all simple roots of \(g(X)\)
\end{exercise}

\begin{exercise}
\label{2.6}
Let \(f(X)\) be an irreducible polynomial in \(F[X]\), where \(F\) has characteristic \(p\).
Show that \(f(X)\) can be written \(g(X)=g(X^{p^e})\) where \(g(X)\) is irreducible and
separable. Deduce that every root of \(f(X)\) has the same multiplicity \(p^e\) in any splitting field
\end{exercise}

\begin{proof}
If \(f\) is not separable, then \(f\) is a polynomial in \(X^p\), say \(f(X)=g(X^p)\). If \(g\) is
not separable, then \(g(X^p)=h(X^{2p})\). This process will end since each polynomial has finite degree.
\end{proof}
\section{The Fundamental Theorem of Galois Theory}
\label{sec:org36de633}
\subsection{Groups of automorphism of fields}
\label{sec:org59bdb7b}
Consider fields \(E\supset F\). An \(F\)-isomorphism \(E\to E\) is called an \textbf{\(F\)-automorphism}
of \(E\). The \(F\)-automorphisms of \(E\) form a group, which we denote \(\Aut(E/F)\)

\begin{examplle}[]
Let \(E=\C(X)\). A \(\C\)-automorphism of \(E\) sends \(X\) to another generator of \(E\)
over \(\C\). It follows from \ref{9.24} below that these are exactly the
elements \(\frac{aX+b}{cX+d}\), \(ad-bc\neq 0\). Therefore \(\Aut(E/\C)\) consists of the
maps \(f(X)\mapsto f\left( \frac{aX+b}{cX+d} \right)\), \(ad-bc\neq 0\), and so
\begin{equation*}
\Aut(E/\C)\cong\PGL_2(\C)
\end{equation*}
the group of invertible \(2\times 2\) matrices with complex coefficients modulo its centre.
\end{examplle}

\begin{proposition}[]
Let \(E\) be a splitting field of a separable polynomial \(f\) in \(F[X]\); then \(\Aut(E/F)\)
has order \([E:F]\)
\end{proposition}

\begin{proof}
As \(f\) is separable, it has \(\deg f\) different roots in \(E\). Therefore Proposition \ref{2.7}
shows that the number of \(F\)-homomorphisms \(E\to E\) is \([E:F]\). Because \(E\) is finite
over \(F\), all such homomorphisms are isomorphisms
\end{proof}

When \(G\) is a group of automorphisms of a field \(E\), we set
\begin{equation*}
E^G=\Inv(G)=\{\alpha\in E\mid\sigma\alpha=\alpha,\forall\alpha\in G\}
\end{equation*}
It is a subfield of \(E\), called the subfield of \textbf{\(G\)-invariants} of \(E\) or the \textbf{fixed field}
of \(G\)

\begin{theorem}[E. Artin]
Let \(G\) be a finite group of automorphisms of a field \(E\), then
\begin{equation*}
[E:E^G]\le(G:1)
\end{equation*}
\end{theorem}

\begin{proof}
Let \(F=E^G\), and let \(G=\{\sigma_1,\dots,\sigma_m\}\) with \(\sigma_1\) the identity map. It suffices to show that
every set \(\{\alpha_1,\dots,\alpha_n\}\) of elements of \(E\) with \(n>m\) is linearly dependent over \(F\). For
such a set, consider the system of linear equations
\begin{align*}
\sigma_1(\alpha_1)X_1+\dots+\sigma_1(\alpha_n)X_n&=0\\
&\vdots\\
\sigma_m(\alpha_1)X_1+\dots+\sigma_m(\alpha_n)X_n&=0
\end{align*}
with coefficients in \(E\). There are \(m\) equations and \(n>m\) unknowns, and hence there are
nontrivial solutions in \(E\). We choose one \((c_1,\dots,c_n)\) having the fewest possible nonzero
elements. After renumbering the \(\alpha_i\), we may choose that \(c_1\neq 0\), and then, after
multiplying by a scalar, that \(c_1\in F\)
\wu{
Let \(d_i=-(\sigma_i(\alpha_1^{-1}\alpha_2)c_2+\dots+\sigma_i(\alpha_1^{-1}\alpha_n)c_n)\)
Then \(c_1=d_i\) for \(i=1,\dots,n\), for any \(i\in\{1,\dots,n\}\), \(\sigma_i(c_1)=\sigma_i(d_1)=d_i=c_1\). Thus \(c_1\in F\)
}
With these normalizations, we'll show that
all \(c_i\in F\), and so the first equation
\begin{equation*}
\alpha_1c_1+\dots+\alpha_nc_n=0
\end{equation*}
is a linear relation on the \(\alpha_i\)

If not all \(c_i\) are in \(F\), then \(\sigma_k(c_i)\neq c_i\) for some \(k\neq 1\) and \(i\neq 1\). On
applying \(\sigma_k\) to the system of linear equations
\begin{align*}
\sigma_1(\alpha_1)c_1+\dots+\sigma_1(\alpha_n)c_n&=0\\
&\vdots\\
\sigma_m(\alpha_1)c_1+\dots+\sigma_m(\alpha_n)c_n&=0
\end{align*}
and using that \(\{\sigma_k\sigma_1,\dots,\sigma_k\sigma_m\}=\{\sigma_1,\dots,\sigma_m\}\), we find that
\begin{equation*}
(c_1,\sigma_k(c_2),\dots,\sigma_k(c_n))
\end{equation*}
is also a solution to the system of equations. On subtracting it from the first solution, we
obtain a solution \((0,\dots,c_i-\sigma_k(c_i),\dots)\), which is nonzero, but has more zeros than the first
solutions - contradiction
\wu{
If \(c_i=0\), then \(\sigma_k(c_i)=0\) since this is an automorphism
}
\end{proof}

\begin{corollary}[]
\label{3.5}
Let \(G\) be a finite group of automorphisms of a field \(E\); then
\begin{equation*}
G=\Aut(E/E^G)
\end{equation*}
\end{corollary}

\begin{proof}
As \(G\subset\Aut(E/E^G)\), we have inequalities
\begin{equation*}
[E:E^G]\le(G:1)\le(\Aut(E/E^G):1)\le[E:E^G]
\end{equation*}
last inequality by \ref{2.8} (1)
\end{proof}
\subsection{Separable, normal, and Galois extensions}
\label{sec:org69b19f2}
\begin{definition}[]
An algebraic extension \(E/F\) is \textbf{separable} if the minimal polynomial of every element of \(E\)
is separable; otherwise it is \textbf{inseparable}
\end{definition}

Thus, an algebraic extension \(E/F\) is separable if every irreducible polynomial in \(F[X]\)
having at least one root in \(E\) is separable, and it is inseparable if
\begin{itemize}
\item \(F\) is nonperfect, and in particular has characteristic \(p\neq 0\), and
\item there is an element \(\alpha\in E\) whose minimal polynomial is of the form \(g(X^p)\), \(g\in F[X]\)
\end{itemize}


\(\F_p(T)/\F_p(T^p)\) is inseparable extension because \(T\) has minimal polynomial \(X^p-T^p\)

\begin{definition}[]
An extension \(E/F\) is \textbf{normal} if it is algebraic and the minimal polynomial of every element
of \(E\) splits in \(E[X]\)
\end{definition}

an algebraic extension \(E/F\) \(\Leftrightarrow\) every irreducible polynomial \(f\in F[X]\) having at least
one root in \(E\) splits in \(E[X]\)

Let \(f\) be a monic irreducible polynomial of degree \(m\) in \(F[X]\), and let \(E\) be an
algebraic extension of \(F\). If \(f\) has a root in \(E\), so that it is the minimal polynomial
of an element of \(E\), then
\begin{equation*}
\left.
\begin{alignat*}{3}
&E/F\text{ separable}\quad&&\Rightarrow\quad&&f\text{ has only simple roots}\\
&E/F\text{ normal}&&\Rightarrow&&f\text{ splits in }E
\end{alignat*}
\right\}\quad\Rightarrow\quad f\text{ has $m$ distinct roots in }E
\end{equation*}



\section{Problem}
\label{sec:orgda7aacd}
\begin{center}
\begin{tabular}{l}
\ref{Problem1}\\
\end{tabular}
\end{center}
\end{document}
