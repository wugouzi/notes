% Created 2022-09-26 Mon 10:32
% Intended LaTeX compiler: pdflatex
\documentclass[11pt]{article}
\usepackage[utf8]{inputenc}
\usepackage[T1]{fontenc}
\usepackage{graphicx}
\usepackage{longtable}
\usepackage{wrapfig}
\usepackage{rotating}
\usepackage[normalem]{ulem}
\usepackage{amsmath}
\usepackage{amssymb}
\usepackage{capt-of}
\usepackage{hyperref}
\graphicspath{{../../books/}}
% TIPS
% \substack{a\\b} for multiple lines text





% pdfplots will load xolor automatically without option
\usepackage[dvipsnames]{xcolor}

\usepackage{forest}
% two-line text in node by [two \\ lines]
% \begin{forest} qtree, [..] \end{forest}
\forestset{
  qtree/.style={
    baseline,
    for tree={
      parent anchor=south,
      child anchor=north,
      align=center,
      inner sep=1pt,
    }}}
%\usepackage{flexisym}
% load order of mathtools and mathabx, otherwise conflict overbrace

\usepackage{mathtools}
%\usepackage{fourier}
\usepackage{pgfplots}
\usepackage{amsthm, mathabx,  amsmath, commath}
\usepackage{amsfonts}

\usepackage{empheq}
\usepackage{tikz}
\usetikzlibrary{arrows.meta}
\usepackage[most]{tcolorbox}

\newtheorem{theorem}{Theorem}[section]
\newtheorem{definition}{Definition}[section]
\newtheorem{corollary}{Corollary}[section]
\newtheorem{example}{Example}[section]
\newtheorem{lemma}{Lemma}[section]
\newtheorem{proposition}{Proposition}[section]

\newcommand{\bl}[1] {\boldsymbol{#1}}
\newcommand{\Wt}[1] {\stackrel{\sim}{\smash{#1}\rule{0pt}{1.1ex}}}
\newcommand{\wt}[1] {\widetilde{#1}}


%For boxed texts in align, use Aboxed{}
%otherwise use boxed{}

\DeclareMathSymbol{\widehatsym}{\mathord}{largesymbols}{"62}
\newcommand\lowerwidehatsym{%
  \text{\smash{\raisebox{-1.3ex}{%
    $\widehatsym$}}}}
\newcommand\fixwidehat[1]{%
  \mathchoice
    {\accentset{\displaystyle\lowerwidehatsym}{#1}}
    {\accentset{\textstyle\lowerwidehatsym}{#1}}
    {\accentset{\scriptstyle\lowerwidehatsym}{#1}}
    {\accentset{\scriptscriptstyle\lowerwidehatsym}{#1}}
}

\usepackage{graphicx}
    
% text on arrow for xRightarrow
\makeatletter
%\newcommand{\xRightarrow}[2][]{\ext@arrow 0359\Rightarrowfill@{#1}{#2}}
\makeatother


\def \bx {\boldsymbol{x}}
\def \ba {\boldsymbol{a}}
\def \bI {\boldsymbol{I}}
\def \bt {\boldsymbol{t}}
\def \bb {\boldsymbol{b}}
\def \bA {\boldsymbol{A}}
\def \bX {\boldsymbol{X}}
\def \bu {\boldsymbol{u}}
\def \bS {\boldsymbol{S}}
\def \bZ {\boldsymbol{Z}}
\def \bz {\boldsymbol{z}}
\def \by {\boldsymbol{y}}
\def \bw {\boldsymbol{w}}
\def \bT {\boldsymbol{T}}
\def \bS {\boldsymbol{S}}
\def \bm {\boldsymbol{m}}
\def \bW {\boldsymbol{W}}
\def \bY {\boldsymbol{Y}}
\def \bH {\boldsymbol{H}}
\def \blambda {\boldsymbol{\lambda}}
\def \bPhi {\boldsymbol{\Phi}}
\def \btheta {\boldsymbol{\theta}}
\def \bmu {\boldsymbol{\mu}}
\def \bphi {\boldsymbol{\phi}}
\def \bSigma {\boldsymbol{\Sigma}}
\def \lb {\left\{}
\def \rb {\right\}}
\def \caln {\mathcal{N}}
\def \dissum {\displaystyle\Sigma}
\def \dispro {\displaystyle\prod}
\def \E {\mathbb{E}}
\def \Q {\mathbb{Q}}
\def \V {\mathbb{V}}
\def \R {\mathbb{R}}
\def \calq {\mathcal{Q}}
\def \calg {\mathcal{G}}
\def \caln {\mathcal{N}}
\def \calr {\mathcal{R}}
\def \calm {\mathcal{M}}
\def \calc {\mathcal{C}}
\def \bcup {\bigcup}

\makeindex
\usepackage{minted}
\renewcommand*{\proofname}{Solution}
\author{wu}
\date{\today}
\title{Competitive programming}
\hypersetup{
 pdfauthor={wu},
 pdftitle={Competitive programming},
 pdfkeywords={},
 pdfsubject={},
 pdfcreator={Emacs 28.0.92 (Org mode 9.6)}, 
 pdflang={English}}
\begin{document}

\maketitle

\section{Dynamic Programming}
\label{sec:org5a5fa17}
\subsection{Digit DP}
\label{sec:org91a95a5}
\begin{problem}[LeetCode 788: Rotated Digits]
An integer x is a \textbf{good} if after rotating each digit individually by 180 degrees, we get a valid
number that is different  from x. Each digit must be rotated - we cannot choose to leave it alone.

A number is valid if each digit remains a digit after rotation. For example:
\begin{itemize}
\item 0, 1, and 8 rotate to themselves,
\item 2 and 5 rotate to each other (in this case they are rotated in a different direction, in other
words, 2 or 5 gets mirrored)
\item 6 and 9 rotate to each other, and
\item the rest of the numbers do not rotate to any other number and become invalid.
\end{itemize}

Given an integer n, return the number of good integers in the range [1, n].
\end{problem}

\begin{proof}
Given \(n\).
Let \(f(pos,bound,diff)\) be the number of good numbers satisfying
\begin{enumerate}
\item Only consider \(pos\)th digit and \(pos\) starts from left, which means 0th digit is the
highest digit. And we assume the first \(pos-1\)
digits are fixed
\item If digits in \([0,pos-1]\) are first \(pos\) digits of \(n\), then \(bound\) is \texttt{true}
\item If digits in \([0,pos-1]\) has at least one 2/5/6/9, then \(diff\) is \texttt{true}
\end{enumerate}

Therefore the answer is \(f(0, true, false)\), and the transition formula is
\begin{equation*}
f(pos,bound,diff)=\sum f(pos+1,bound',diff')
\end{equation*}
\begin{itemize}
\item \(bound'\) is true iff \(bound\) is true and the digit we choose is the \(pos\)th digit
of \(n\)
\item \(diff'\) is true iff \(diff\) is true or we chose 2/5/6/9
\end{itemize}
\end{proof}
\section{Trick and Bit}
\label{sec:org86f5b04}
\subsection{Bit operation}
\label{sec:org228ec7c}
\begin{problem}[\href{https://leetcode.cn/problems/missing-two-lcci/}{Leetcode: Missing Two LCCI}]
You are given an array with all the numbers from 1 to N appearing exactly once, except for two
number that is missing. How can you find the missing number in \(O(N)\) time and \(O(1)\) space?

You can return the missing numbers in any order.

\begin{center}
\begin{tabular}{ll}
Input & Output\\
\hline
\texttt{[1]} & \texttt{[2,3]}\\
\texttt{[2,3]} & \texttt{[1,4]}\\
\end{tabular}
\end{center}


\texttt{nums.length <= 30000}
\end{problem}

\begin{proof}
Suppose the missing two numbers are \(x_1\) and \(x_2\), and if we add \(1,\dots,N\) to the end of the
array \(A\), then \(x=\bigoplus A=x_1\oplus x_2\).

By \texttt{x\&-x} we can get the lowest bit of \(x\), assume it's in \(l\)th bit. Then we can
assume \(x_1\)'s \(l\)th bit is 0, and \(x_2\)'s \(l\)th bit is 1, and we can partition \(A\) into
\(A_1\) and \(A_2\) by whether the elements' \(l\)th bit is 1, then \(\bigoplus A_1=x_1\) and \(\bigoplus A_2=x_2\)
\end{proof}
\end{document}
