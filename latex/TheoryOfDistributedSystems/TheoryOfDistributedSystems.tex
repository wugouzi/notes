% Created 2024-03-02 Sat 14:19
% Intended LaTeX compiler: xelatex
\documentclass[11pt]{article}
\usepackage{hyperref}
% TIPS
% \substack{a\\b} for multiple lines text





% pdfplots will load xolor automatically without option
\usepackage[dvipsnames]{xcolor}

\usepackage{forest}
% two-line text in node by [two \\ lines]
% \begin{forest} qtree, [..] \end{forest}
\forestset{
  qtree/.style={
    baseline,
    for tree={
      parent anchor=south,
      child anchor=north,
      align=center,
      inner sep=1pt,
    }}}
%\usepackage{flexisym}
% load order of mathtools and mathabx, otherwise conflict overbrace

\usepackage{mathtools}
%\usepackage{fourier}
\usepackage{pgfplots}
\usepackage{amsthm, mathabx,  amsmath, commath}
\usepackage{amsfonts}

\usepackage{empheq}
\usepackage{tikz}
\usetikzlibrary{arrows.meta}
\usepackage[most]{tcolorbox}

\newtheorem{theorem}{Theorem}[section]
\newtheorem{definition}{Definition}[section]
\newtheorem{corollary}{Corollary}[section]
\newtheorem{example}{Example}[section]
\newtheorem{lemma}{Lemma}[section]
\newtheorem{proposition}{Proposition}[section]

\newcommand{\bl}[1] {\boldsymbol{#1}}
\newcommand{\Wt}[1] {\stackrel{\sim}{\smash{#1}\rule{0pt}{1.1ex}}}
\newcommand{\wt}[1] {\widetilde{#1}}


%For boxed texts in align, use Aboxed{}
%otherwise use boxed{}

\DeclareMathSymbol{\widehatsym}{\mathord}{largesymbols}{"62}
\newcommand\lowerwidehatsym{%
  \text{\smash{\raisebox{-1.3ex}{%
    $\widehatsym$}}}}
\newcommand\fixwidehat[1]{%
  \mathchoice
    {\accentset{\displaystyle\lowerwidehatsym}{#1}}
    {\accentset{\textstyle\lowerwidehatsym}{#1}}
    {\accentset{\scriptstyle\lowerwidehatsym}{#1}}
    {\accentset{\scriptscriptstyle\lowerwidehatsym}{#1}}
}

\usepackage{graphicx}
    
% text on arrow for xRightarrow
\makeatletter
%\newcommand{\xRightarrow}[2][]{\ext@arrow 0359\Rightarrowfill@{#1}{#2}}
\makeatother


\def \bx {\boldsymbol{x}}
\def \ba {\boldsymbol{a}}
\def \bI {\boldsymbol{I}}
\def \bt {\boldsymbol{t}}
\def \bb {\boldsymbol{b}}
\def \bA {\boldsymbol{A}}
\def \bX {\boldsymbol{X}}
\def \bu {\boldsymbol{u}}
\def \bS {\boldsymbol{S}}
\def \bZ {\boldsymbol{Z}}
\def \bz {\boldsymbol{z}}
\def \by {\boldsymbol{y}}
\def \bw {\boldsymbol{w}}
\def \bT {\boldsymbol{T}}
\def \bS {\boldsymbol{S}}
\def \bm {\boldsymbol{m}}
\def \bW {\boldsymbol{W}}
\def \bY {\boldsymbol{Y}}
\def \bH {\boldsymbol{H}}
\def \blambda {\boldsymbol{\lambda}}
\def \bPhi {\boldsymbol{\Phi}}
\def \btheta {\boldsymbol{\theta}}
\def \bmu {\boldsymbol{\mu}}
\def \bphi {\boldsymbol{\phi}}
\def \bSigma {\boldsymbol{\Sigma}}
\def \lb {\left\{}
\def \rb {\right\}}
\def \caln {\mathcal{N}}
\def \dissum {\displaystyle\Sigma}
\def \dispro {\displaystyle\prod}
\def \E {\mathbb{E}}
\def \Q {\mathbb{Q}}
\def \V {\mathbb{V}}
\def \R {\mathbb{R}}
\def \calq {\mathcal{Q}}
\def \calg {\mathcal{G}}
\def \caln {\mathcal{N}}
\def \calr {\mathcal{R}}
\def \calm {\mathcal{M}}
\def \calc {\mathcal{C}}
\def \bcup {\bigcup}

\graphicspath{{../../books/}}
\makeindex
\DeclareMathOperator{\state}{\textsf{state}}
\DeclareMathOperator{\buffer}{\textsf{buffer}}
\DeclareMathOperator{\del}{\textsf{del}}
\DeclareMathOperator{\comp}{\textsf{comp}}
\DeclareMathOperator{\true}{\textbf{true}}
\DeclareMathOperator{\false}{\textbf{false}}
\DeclareMathOperator{\pid}{\textsf{pid}}
\DeclareMathOperator{\troot}{\textsf{root}}
\DeclareMathOperator{\seenmessage}{\textsf{seen-message}}
\DeclareMathOperator{\parent}{\textsf{parent}}
\DeclareMathOperator{\nonChildren}{\textsf{nonChildren}}
\DeclareMathOperator{\children}{\textsf{children}}
\DeclareMathOperator{\ack}{\textsf{ack}}
\DeclareMathOperator{\nack}{\textsf{nack}}
\DeclareMathOperator{\tinput}{\textsf{input}}
\DeclareMathOperator{\initiator}{\textsf{initiator}}
\DeclareMathOperator{\tpid}{\textsf{pid}}
\DeclareMathOperator{\distance}{\textsf{distance}}
\DeclareMathOperator{\bound}{\textsf{bound}}
\DeclareMathOperator{\exactly}{\textsf{exactly}}
\DeclareMathOperator{\morethan}{\textsf{more-than}}
\DeclareMathOperator{\leader}{\textsf{leader}}
\DeclareMathOperator{\maxId}{\textsf{maxId}}
\DeclareMathOperator{\tid}{\textsf{id}}
\DeclareMathOperator{\phase}{\textsf{phase}}
\DeclareMathOperator{\candidate}{\texttt{candidate}}
\DeclareMathOperator{\relay}{\texttt{relay}}
\DeclareMathOperator{\probe}{\texttt{probe}}
\DeclareMathOperator{\current}{\textsf{current}}

%% ox-latex features:
%   !announce-start, !guess-pollyglossia, !guess-babel, !guess-inputenc, caption,
%   !announce-end.

\usepackage{capt-of}

%% end ox-latex features


\author{James Aspnes}
\date{\today}
\title{Theory Of Distributed Systems}
\hypersetup{
 pdfauthor={James Aspnes},
 pdftitle={Theory Of Distributed Systems},
 pdfkeywords={},
 pdfsubject={},
 pdfcreator={Emacs 30.0.50 (Org mode 9.7-pre)}, 
 pdflang={English}}
\begin{document}

\maketitle
\tableofcontents

\section{Model}
\label{sec:org0c30293}
\subsection{Basic message-passing model}
\label{sec:org23b584a}
We have a collection of \(n\) \textbf{processes} \(p_1,\dots,p_n\), each of which has a \textbf{state} consisting of a state
from state set \(Q_i\). We think of these processes as nodes in a directed \textbf{communication graph} or
\textbf{network}. The edges in this graph are a collection of point-to-point \textbf{channels} or \textbf{buffers} \(b_{ij}\),
one for each pair of adjacent processes \(i\) and \(j\), representing messages that have been sent but
that have not yet been delivered.

A \textbf{configuration} of the system consists of a vector of states, one for each process and channel. The
configuration of the system is updated by an \textbf{event}, where
\begin{enumerate}
\item zero or more messages in channels \(b_{ij}\) are delivered to process \(p_j\), removing them from
\(b_{ij}\);
\item \(p_j\) updates its state in response;
\item zero or more messages are added by \(p_j\) to outgoing channels \(b_{ji}\).
\end{enumerate}

An \textbf{execution segment} is a sequence of alternating configurations and events \(C_0,\phi_1,C_1,\phi_2,\dots\), where
each triple \(C_i\phi_{i+1}C_{i+1}\) is consistent with the transition rules for the event \(\phi_{i+1}\) and
the last element of the sequence is a configuration. If the first configuration \(C_0\) is an \textbf{initial
configuration} of the system, we have an \textbf{execution}. A \textbf{schedule} is an execution with the configurations
removed.
\subsubsection{Formal Details}
\label{sec:org7935ca2}
Let \(P\) be the set of processes, \(Q\) the set of process states, and \(M\) the set of possible
messages.

Each process \(p_i\) has a state \(\state_i\in Q\). Each channel \(b_{ij}\) has a state \(\buffer_{ij}\in\calp(M)\).
We assume each process has a \textbf{transition function} \(\delta:Q\times\calp(M)\to Q\times\calp(P\times M)\) that maps tuples consisting
of a state and a set of incoming messages a new state and a set of recipients and messages to be sent.
A delivery event \(\del(i,A)\) where \(A=\{(j_k,m_k)\}\) removes each message \(m_k\) from \(b_{ji}\),
updates \(\state_i\) according to \(\delta(\state_i,A)\) to the appropriate channels. A computation event \(\comp(i)\)
does the same thing, except that it applies \(\delta(\state_j,\emptyset)\).
\subsection{Asynchronous systems}
\label{sec:orgdfc976e}
In an \textbf{asynchronous} model, only minimal restrictions are placed on when messages are delivered and when
local computation occurs. A schedule is \textbf{admissible} if
\begin{enumerate}
\item there are infinitely many computation steps for each process,
\item every message is eventually delivered
\end{enumerate}
These are \textbf{fairness} conditions. Condition (a) assumes that processes do not explicitly terminate.
\subsection{Synchronous systems}
\label{sec:org2e96f46}
A \textbf{synchronous message-passing} system is exactly like an asynchronous system, except we insist that the
schedule consists of alternating phases where
\begin{enumerate}
\item every process executes a computation step,
\item all messages are delivered while none are sent
\end{enumerate}
The combination of a computation phase and a delivery phase is called a \textbf{round}.
\subsection{Drawing message-passing executions}
\label{sec:org9a4faf0}
\section{Broadcast and convergecast}
\label{sec:org3e168ba}
\subsection{Flooding}
\label{sec:orgc3d9ede}
\subsubsection{Basic algorithm}
\label{sec:org1a653a4}
\begin{algorithm}
\caption{Basic flooding algorithm}
\Init{
        \If{\(\textsf{tpid}=\textsf{root}\)}{
                \(\textsf{seen-message}\gets\textbf{true}\)\;
                send \(M\) to all neighbors\;
        }
        \Else{
                \(\textsf{seen-message}\gets\textbf{false}\)\;
        }
}
\Recv{\(M\)}{
        \If{\(\seenmessage=\false\)}{
                \(\seenmessage\gets\true\)\;
                send \(M\) to all neighbors\;
        }
}
\end{algorithm}

\begin{theorem}[]
Every process receives \(M\) after at most \(D\) time and at most \(\abs{E}\) messages, where \(D\) is
the diameter of the network and \(E\) is the set of (directed) edges in the network
\end{theorem}

We can optimize the algorithm slightly by not sending M back to the node it came from; this will
slightly reduce the message complexity in many cases but makes the proof a sentence or two longer.
\subsubsection{Adding parent pointers}
\label{sec:orgef350b6}
\begin{algorithm}
\label{3.2}
\caption{Flooding with parent pointers}
\Init{
        \If{\(\textsf{tpid}=\textsf{root}\)}{
                \(\parent\gets\troot\)\;
                send \(M\) to all neighbors\;
        }
        \Else{
                \(\parent\gets\bot\)\;
        }
}
\Recv{\(M\) from \(p\)}{
        \If{\(\parent=\bot\)}{
                \(\parent\gets p\)\;
                send \(M\) to all neighbors\;
        }
}
\end{algorithm}

\begin{lemma}[]
At any time during the execution of Algorithm \ref{3.2}, the following invariant holds:
\begin{enumerate}
\item If \(u.\parent\neq\bot\) then \(u.\parent.\parent\neq\bot\) and following parent pointers gives a
path from \(u\) to \(\troot\)
\item If there is a message \(M\) in transit from \(u\) to \(v\), then \(u.\parent\neq\bot\)
\end{enumerate}
\end{lemma}

Though we get a spanning tree at the end, we may not get a very good spanning tree.
\subsubsection{Identifying children}
\label{sec:org2e540b4}
\begin{algorithm}
\label{3.3}
\caption{Flooding tracking children}
\Init{
        \(\nonChildren=\emptyset\)\;
        \If{\(\textsf{tpid}=\textsf{root}\)}{
                \(\parent\gets\troot\)\;
                \(\children\gets\{\troot\}\)\;
                send \(M\) to all neighbors\;
        }
        \Else{
                \(\parent\gets\bot\)\;
                \(\children\gets\emptyset\)\;
        }
}
\Recv{\(M\) from \(p\)}{
        \If{\(\parent=\bot\)}{
                \(\parent\gets p\)\;
                send \(\ack\) to \(p\)\;
                send \(M\) to all neighbors\;
        }
        \Else{
                send \(\nack\) to \(p\)\;
        }
}
\Recv{\(\ack\) from \(p\)}{
        \(\children\gets\children\cup\{p\}\)
}
\Recv{\(\nack\)}{
        \(\nonChildren=\nonChildren\cup\{p\}\)
}
\end{algorithm}

Properties
\begin{enumerate}
\item (safety) If \(p_j\in p_i.\children\), then \(p_j.\parent=p_i\)
\item (safety) If \(p_j\in p_i.\nonChildren\), then \(p_j.\parent\not\in\{p_i,\bot\}\)
\item (liveness) Eventually, every neighbor of \(p_i\) appears in \(p_i.\children\cup p_i.\nonChildren\)
\end{enumerate}
\subsubsection{Convergecast}
\label{sec:orgcd64fd5}
A \textbf{convergecast} is the inverse of broadcast: data is collected from outlying nodes to the root.
\begin{algorithm}
\Init{
        \If{I am a leaf}{
                send \(\tinput\) to \(\parent\)\;
        }
}
\Recv{\(M\) from \(c\)}{
        append \((c,M)\) to \(\buffer\)\;
        \If{\(\buffer\) contains messages from all my children}{
                \(v\gets f(\buffer,\tinput)\)\;
                \eIf{\(\tpid=\troot\)}{
                        \Return{\(v\)}
                }{
                        send \(v\) to \(\parent\)\;
                }
        }
}
\end{algorithm}

Running time is bounded by the depth of the tree: we can prove by induction that any node at height h
(height is length of the longest path from this node to some leaf) sends a message by time \(h\) at the latest.
Message complexity is exactly \(n − 1\), where n is the number of nodes;
\subsubsection{Flooding and convergecast together}
\label{sec:org0c6a5ec}
\section{Distributed breadth-first search}
\label{sec:org40e97f4}

\subsection{Using explicit distances}
\label{sec:orgd4159c9}
\begin{algorithm}
\caption{AsynchBFS algorithm}
\Init{
        \eIf{\(\tpid=\initiator\)}{
                \(\distance\gets 0\)\;
                send \(\distance\) to all neighbors
        }{
                \(\distance\gets\infty\)\;
        }
}
\Recv{\(d\) from \(p\)}{
        \If{\(d+1<\distance\)}{
                \(\distance\gets d+1\)\;
                \(\parent\gets p\)\;
                send \(\distance\) to all neighbors\;
        }
}
\end{algorithm}

The claim is that after at most \(O(VE)\) messages and \(O(D)\) time, all distance values are equal to
the length of the shortest path from the initiator.
\begin{lemma}[]
The variable \(\distance_p\) is always the length of some path from initiator to \(p\), and any
message sent by \(p\) is also the length of some path from \(\initiator\) to \(p\)
\end{lemma}

\begin{proof}
Induction
\end{proof}

A liveness property: \(\distance_p=d(\initiator,p)\) no later than time \(d(\initiator, p)\)
\subsection{Using layering}
\label{sec:org06e55ee}
Here we run a sequence of up to \(\abs{V}\) instances of the simple algorithm with a distance bound on
each: instead of sending out just 0, the initiator sends out \((0,\bound)\) where \(\bound\) is
initially 1 and increases at each phase. A process only sends out its improved distance if it is less
than \(\bound\).

Each phase of the algorithm constructs a partial BFS tree that contains only those nodes within
distance \(\bound\) of the root.

With some effort, it is possible to prove that in a bidirectional network that this approach
guarantees that each edge is only probed once with a new distance, and the \(\bound\)-update and
acknowledgment messages contribute at most \(\abs{V}\) messages per phase. So we get \(O(E+VD)\) total
messages. But the time complexity is bad: \(O(D^2)\) in the worst case.

\label{Problem BFS} TODO: figure out
\subsection{Using local synchronization}
\label{sec:org6fcf6c3}
The reason the layering algorithm takes so long is that at each phase we have to phone all the way
back up the tree to the initiator to get permission to go on to the next phase.

We'll require each node at distance \(d\) to delay sending out a recruiting message until it has confirmed
that none of its neighbors will be sending it a smaller distance. We do this by having two classes of
messages:
\begin{itemize}
\item \(\exactly(d)\): ``I know that my distance is \(d\)''
\item \(\morethan(d)\): ``I know that my distance is \(>d\)''
\end{itemize}
The rules for sending these messages for a non-initiator are:
\begin{enumerate}
\item I can send \(\exactly(d)\) as soon as I have received \(\exactly(d-1)\) from at least one neighbor
and \(\morethan(d-2)\) from all neighbors.
\item I can send \(\morethan(d)\) if \(d=0\) or as soon as I have received \(\morethan(d-1)\) from all
neighbors.
\end{enumerate}

The initiator sends \(\exactly(0)\) to all neighbors at the start of the protocol.

\begin{proposition}[]
Under the assumption that local computation takes zero time and message delivery takes at most 1 time
unit, we'll show that if \(d(\initiator,p)=d\):
\begin{enumerate}
\item \(p\) sends \(\morethan(d')\) for any \(d'<d\) by time \(d'\)
\item \(p\) sends \(\exactly(d)\) by time \(d\)
\item \(p\) never sends \(\morethan(d')\) for any \(d'\ge d\)
\item \(p\) never sends \(\exactly(d')\) for any \(d'\neq d\)
\end{enumerate}
\end{proposition}

\begin{proof}
For (3) and (4). The base case is that the initiator never sends any \(\morethan\) messages at all,
and any non-initiator never sends \(\exactly(0)\). For larger \(d'\), observe that if a non-initiator
\(p\) sends \(\morethan(d')\) for \(d'\ge d\), it must first have received \(\morethan(d'-1)\) from all
neighbors, including some neighbor \(p'\) at distance \(d-1\). But the induction hypothesis tells us
that \(p'\) can't send \(\morethan(d'-1)\) for \(d'-1\ge d-1\). Similarly, to send \(\exactly(d')\) for
\(d'>d\), \(p\) must first receive \(\morethan(d'-2)\) from this closer neighbor \(p'\), but then
\(d'-2>d-2\ge d-1\) so \(\morethan(d'-2)\) is not sent by \(p'\).

For (1) and (2). The base case is that the initiator sends \(\exactly(0)\) to all nodes at time 0,
giving (1), and there is no \(\morethan(d')\) with \(d'<0\) for it to send, giving (2).

Message complexity: A node at distance \(d\) sends \(\morethan(d')\) for all \(0<d'<d\) and
\(\exactly(d)\) and no other messages. So we have message complexity bounded by \(\abs{E}\cdot D\).

Time complexity: \(D\)
\end{proof}
\section{Leader election}
\label{sec:org9427c3f}
\subsection{Symmetry}
\label{sec:org1637d12}
A system exhibits \textbf{symmetry} if we can permute the nodes without changing the behaviour of the system.
More formally, we can define a symmetry as an \textbf{equivalence relation} on processes, where we have the
additional properties that all processes in the same equivalence class run the same code; and whenever
\(p\) is equivalent to \(p'\), each neighbor \(q\) of \(p\) is equivalent to a corresponding neighbor
\(q'\) of \(p'\).

Symmetries are convenient for proving impossibility results, as observed by Angluin. The underlying
theme is that without some mechanism for  \textbf{symmetry breaking}, a message-passing system escape from a
symmetric initial configuration. The following lemma holds for \textbf{deterministic} systems, basically those
in which processes can’t flip coins:
\begin{lemma}[]
\label{5.1.1}
    A symmetric deterministic message-passing system that starts in an initial configuration in which
    equivalent processes have the same state has a synchronous execution in which equivalent processes
    continue to have the same state.
\end{lemma}

\begin{proof}
Easy induction on rounds: if in some round \(p\) and \(p'\) are equivalent and have the same state, and all
their neighbors are equivalent and have the same state, then p and p0 receive the same messages from
their neighbors and can proceed to the same state (including outgoing messages) in the next round.
\end{proof}

An immediate corollary is that you can’t do leader election in an anonymous system with a symmetry
that puts each node in a non-trivial equivalence class, because as soon as I stick my hand up to
declare I’m the leader, so do all my equivalence-class buddies.

A more direct way to break symmetry is to assume that all processes have identities; now processes can
break symmetry by just declaring that the one with the smaller or larger identity wins.
\subsection{Leader election in rings}
\label{sec:org9f322d0}
\subsubsection{The Le Lann-Chang-Roberts algorithm}
\label{sec:org192ddba}
This algorithm works in a \textbf{unidirectional ring}, where messages can only travel clockwise.
\begin{algorithm}
\caption{LCR leader election}
        \Init{
            \(\leader\gets 0\)\;
            \(\maxId\gets \tid_i\)\;
            send \(\id_i\) to clockwise neighbor\;
        }
\Recv{\(j\)}{
    \If{\(j=\tid_i\)}{
                \(\leader\gets 1\)\;
    }
        \If{\(j>\maxId\)}{
                \(\maxId\gets j\)\;
                send \(j\) to clockwise neighbor\;
        }
}
\end{algorithm}
Protocol works because whichever process \(p_{max}\) holds the maximum ID \(\tid_{max}\) will
\begin{enumerate}
\item refuse to forward any smaller ID
\item eventually have its value forwarded through all of the other processes, causing it to eventually
set its \(\leader\) bit to 1.
\end{enumerate}
\subsubsection{The Hirschberg-Sinclair algorithm}
\label{sec:orga2418cb}
Nancy's book is better.

This algorithm improves on Le Lann-Chang-Roberts by reducing the message complexity. The idea is that
instead of having each process send a message all the way around a ring, each process will first probe
locally to see if it has the largest ID within a short distance. If it wins among its immediate
neighbors, it doubles the size of the neighborhood it checks, and continues as long as it has a
winning ID. This means that most nodes drop out quickly, giving a total message complexity of \(O(n
        log n)\). The running time is a constant factor worse than LCR, but still \(O(n)\).
\subsubsection{Peterson's algorithm for the unidirectional ring}
\label{sec:org1936f83}
Assume an asynchronous unidirectional ring. It gets \(O(n\log n)\) message complexity.

\begin{algorithm}
\caption{Peterson's leader-election algorithm}
\Function{\texttt{candidate}\(()\)}{
        \(\phase\gets 0\)\;
        \(\current\gets\tpid\)\;
        \While{\(\true\)}{
                send \(\texttt{probe}(\phase,\current)\)\;
                wait for \(\probe(\phase,x)\)\;
                \(\tid_2\gets x\)\;
                send \(\probe(\phase+1/2,\tid_2)\)\;
                wait for \(\probe(\phase+1/2,x)\)\;
                \(\tid_3\gets x\)\;
                \uIf{\(\tid_2=\current\)}{
                        I am the leader\;
                        \Return{}\;
                }\uElseIf{\(\tid_2>\current\wedge\tid_2>\tid_3\)}{
                        \(\current\gets\tid_2\)\;
                        \(\phase\gets\phase+1\)\;
                }\Else{
                        switch to \(\relay()\)\;
                }
        }
}
\Function{\(\relay()\)}{
        \Recv{\(\probe(p,i)\)}{
            send \(\probe(p,i)\)\;
        }
}
\end{algorithm}
\section{Problems}
\label{sec:orgad03eca}
\begin{center}
\begin{tabular}{lll}
Link & Problems & \\
\hline
\label{Problem BFS} & proof of the complexity & false\\
\end{tabular}
\end{center}
\end{document}
