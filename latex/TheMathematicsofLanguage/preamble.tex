%%%%%%%%%%%%%%%%%%%%%%%%%%%%%%%%%%%%%%
%% TIPS                                 %%
%%%%%%%%%%%%%%%%%%%%%%%%%%%%%%%%%%%%%%
% \substack{a\\b} for multiple lines text
% \usepackage{expl3}
% \expandafter\def\csname ver@l3regex.sty\endcsname{}
% \usepackage{pkgloader}
\usepackage[utf8]{inputenc}

\usepackage[B1,T1]{fontenc}
\usepackage{fontspec}

% \usepackage[Emoticons]{ucharclasses}
\newfontfamily\DejaSans{DejaVu Sans}
% \setDefaultTransitions{\DejaSans}{}

% pdfplots will load xolor automatically without option
\usepackage[dvipsnames]{xcolor}
%%%%%%%%%%%%%%%%%%%%%%%%%%%%%%%%%%%%%%%
%% MATH related pacakge                  %%
%%%%%%%%%%%%%%%%%%%%%%%%%%%%%%%%%%%%%%%
% \usepackage{amsmath} mathtools loads the amsmath
\usepackage{amsmath}
\usepackage{mathtools}


\usepackage{amsthm}
\usepackage{amsbsy}

%\usepackage{commath}

\usepackage{amssymb}

\usepackage{mathrsfs}
%\usepackage{mathabx}
\usepackage{stmaryrd}
\usepackage{empheq}

\usepackage{scalerel}
\usepackage{stackengine}
\usepackage{stackrel}

\usepackage{nicematrix}
\usepackage{tensor}
\usepackage{blkarray}
\usepackage{siunitx}
\usepackage[f]{esvect}

% centering \not on a letter
\usepackage{slashed}

%\usepackage{merriweather}
\usepackage{unicode-math}
\setmainfont{TeX Gyre Pagella}
% \setmathfont{STIX}
%\setmathfont{texgyrepagella-math.otf}
%\setmathfont{Libertinus Math}
\setmathfont{Latin Modern Math}

 
% \setmathfont[range={\smwhtdiamond,\enclosediamond,\varlrtriangle}]{Latin Modern Math}
\setmathfont[range={\rightrightarrows,\twoheadrightarrow,\leftrightsquigarrow,\triangledown,\vartriangle}]{XITS Math}
 \setmathfont[range={\int,\setminus}]{Libertinus Math}
 % \setmathfont[range={\mathalpha}]{TeX Gyre Pagella Math}
%\setmathfont[range={\mitA,\mitB,\mitC,\mitD,\mitE,\mitF,\mitG,\mitH,\mitI,\mitJ,\mitK,\mitL,\mitM,\mitN,\mitO,\mitP,\mitQ,\mitR,\mitS,\mitT,\mitU,\mitV,\mitW,\mitX,\mitY,\mitZ,\mita,\mitb,\mitc,\mitd,\mite,\mitf,\mitg,\miti,\mitj,\mitk,\mitl,\mitm,\mitn,\mito,\mitp,\mitq,\mitr,\mits,\mitt,\mitu,\mitv,\mitw,\mitx,\mity,\mitz}]{TeX Gyre Pagella Math}
% unicode is not good at this!
%\let\nmodels\nvDash

\usepackage{wasysym}
%%%%%%%%%%%%%%%%%%%%%%%%%%%%%%%%%%%%%%%
%% TIKZ related packages                 %%
%%%%%%%%%%%%%%%%%%%%%%%%%%%%%%%%%%%%%%%

\usepackage{pgfplots}
\pgfplotsset{compat=1.15}
\usepackage{tikz}
\usepackage{tikz-cd}
\usepackage{tikz-qtree}

\usetikzlibrary{arrows,positioning,calc,fadings,decorations,matrix,decorations,shapes.misc}
%setting from geogebra
\definecolor{ccqqqq}{rgb}{0.8,0,0}


%%%%%%%%%%%%%%%%%%%%%%%%%%%%%%%%%%%%%%%
%% MISCLELLANEOUS packages               %%
%%%%%%%%%%%%%%%%%%%%%%%%%%%%%%%%%%%%%%%
%\usepackage{twemojis}
\usepackage[most]{tcolorbox}
\usepackage{threeparttable}
\usepackage{tabularx}

\usepackage{enumitem}

% wrong with preview
\usepackage{subcaption}
\usepackage{caption}
% {\aunclfamily\Huge}
\usepackage{auncial}

\usepackage{float}

\usepackage{fancyhdr}

\usepackage{ifthen}
\usepackage{xargs}


\usepackage{imakeidx}
\usepackage{hyperref}
\usepackage{soul}
\usepackage{framed}

% don't use this for preview
%\usepackage[margin=1.5in]{geometry}
% \usepackage{geometry}
% \geometry{legalpaper, landscape, margin=1in}
\usepackage[font=itshape]{quoting}

%\LoadPackagesNow
%\usepackage[xetex]{preview}
%%%%%%%%%%%%%%%%%%%%%%%%%%%%%%%%%%%%%%%
%% USEPACKAGES end                       %%
%%%%%%%%%%%%%%%%%%%%%%%%%%%%%%%%%%%%%%%
%\pdfcompresslevel0

% \setlist{nosep}
% \numberwithin{equation}{subsection}
% \fancyhead{} % Clear the headers
% \renewcommand{\headrulewidth}{0pt} % Width of line at top of page
% \fancyhead[R]{\slshape\leftmark} % Mark right [R] of page with Chapter name [\leftmark]
% \pagestyle{fancy} % Set default style for all content pages (not TOC, etc)


% \newlength\shlength
% \newcommand\vect[2][0]{\setlength\shlength{#1pt}%
%   \stackengine{-5.6pt}{$#2$}{\smash{$\kern\shlength%
%     \stackengine{7.55pt}{$\mathchar"017E$}%
%       {\rule{\widthof{$#2$}}{.57pt}\kern.4pt}{O}{r}{F}{F}{L}\kern-\shlength$}}%
%       {O}{c}{F}{T}{S}}


\indexsetup{othercode=\small}
\makeindex[columns=2,options={-s /media/wu/file/stuuudy/notes/index_style.ist},intoc]
\makeatletter
\def\@idxitem{\par\hangindent 0pt}
\makeatother


%\newcounter{dummy} \numberwithin{dummy}{section}
\newtheorem{dummy}{dummy}[section]
\theoremstyle{definition}
\newtheorem{definition}[dummy]{Definition}
\theoremstyle{plain}
\newtheorem{corollary}[dummy]{Corollary}
\newtheorem{lemma}[dummy]{Lemma}
\newtheorem{proposition}[dummy]{Proposition}
\newtheorem{theorem}[dummy]{Theorem}
\newtheorem{notation}[dummy]{Notation}
\theoremstyle{definition}
\newtheorem{examplle}{Example}[section]
\theoremstyle{remark}
\newtheorem*{remark}{Remark}
\newtheorem{exercise}{Exercise}[subsection]
\newtheorem{observation}{Observation}[section]


\newenvironment{claim}[1]{\par\noindent\textbf{Claim:}\space#1}{}

\makeatletter
\DeclareFontFamily{U}{tipa}{}
\DeclareFontShape{U}{tipa}{m}{n}{<->tipa10}{}
\newcommand{\arc@char}{{\usefont{U}{tipa}{m}{n}\symbol{62}}}%

\newcommand{\arc}[1]{\mathpalette\arc@arc{#1}}

\newcommand{\arc@arc}[2]{%
  \sbox0{$\m@th#1#2$}%
  \vbox{
    \hbox{\resizebox{\wd0}{\height}{\arc@char}}
    \nointerlineskip
    \box0
  }%
}
\makeatother

\setcounter{MaxMatrixCols}{20}
%%%%%%% ABS
\DeclarePairedDelimiter\abss{\lvert}{\rvert}%
\DeclarePairedDelimiter\normm{\lVert}{\rVert}%

% Swap the definition of \abs* and \norm*, so that \abs
% and \norm resizes the size of the brackets, and the
% starred version does not.
\makeatletter
\let\oldabs\abss
%\def\abs{\@ifstar{\oldabs}{\oldabs*}}
\newcommand{\abs}{\@ifstar{\oldabs}{\oldabs*}}
\newcommand{\norm}[1]{\left\lVert#1\right\rVert}
%\let\oldnorm\normm
%\def\norm{\@ifstar{\oldnorm}{\oldnorm*}}
%\renewcommand{norm}{\@ifstar{\oldnorm}{\oldnorm*}}
\makeatother

% \newcommand\what[1]{\ThisStyle{%
%     \setbox0=\hbox{$\SavedStyle#1$}%
%     \stackengine{-1.0\ht0+.5pt}{$\SavedStyle#1$}{%
%       \stretchto{\scaleto{\SavedStyle\mkern.15mu\char'136}{2.6\wd0}}{1.4\ht0}%
%     }{O}{c}{F}{T}{S}%
%   }
% }

% \newcommand\wtilde[1]{\ThisStyle{%
%     \setbox0=\hbox{$\SavedStyle#1$}%
%     \stackengine{-.1\LMpt}{$\SavedStyle#1$}{%
%       \stretchto{\scaleto{\SavedStyle\mkern.2mu\AC}{.5150\wd0}}{.6\ht0}%
%     }{O}{c}{F}{T}{S}%
%   }
% }

% \newcommand\wbar[1]{\ThisStyle{%
%     \setbox0=\hbox{$\SavedStyle#1$}%
%     \stackengine{.5pt+\LMpt}{$\SavedStyle#1$}{%
%       \rule{\wd0}{\dimexpr.3\LMpt+.3pt}%
%     }{O}{c}{F}{T}{S}%
%   }
% }

\newcommand{\bl}[1] {\boldsymbol{#1}}
\newcommand{\Wt}[1] {\stackrel{\sim}{\smash{#1}\rule{0pt}{1.1ex}}}
\newcommand{\wt}[1] {\widetilde{#1}}
\newcommand{\tf}[1] {\textbf{#1}}

\newcommand{\wu}[1]{{\color{red} #1}}



%For boxed texts in align, use Aboxed{}
%otherwise use boxed{}

\DeclareMathSymbol{\widehatsym}{\mathord}{largesymbols}{"62}
\newcommand\lowerwidehatsym{%
  \text{\smash{\raisebox{-1.3ex}{%
    $\widehatsym$}}}}
\newcommand\fixwidehat[1]{%
  \mathchoice
    {\accentset{\displaystyle\lowerwidehatsym}{#1}}
    {\accentset{\textstyle\lowerwidehatsym}{#1}}
    {\accentset{\scriptstyle\lowerwidehatsym}{#1}}
    {\accentset{\scriptscriptstyle\lowerwidehatsym}{#1}}
  }


\newcommand{\cupdot}{\mathbin{\dot{\cup}}}
\newcommand{\bigcupdot}{\mathop{\dot{\bigcup}}}

\usepackage{graphicx}

\usepackage[toc,page]{appendix}

% text on arrow for xRightarrow
\makeatletter
%\newcommand{\xRightarrow}[2][]{\ext@arrow 0359\Rightarrowfill@{#1}{#2}}
\makeatother

% Arbitrary long arrow
\newcommand{\Rarrow}[1]{%
\parbox{#1}{\tikz{\draw[->](0,0)--(#1,0);}}
}

\newcommand{\LRarrow}[1]{%
\parbox{#1}{\tikz{\draw[<->](0,0)--(#1,0);}}
}


\makeatletter
\providecommand*{\rmodels}{%
  \mathrel{%
    \mathpalette\@rmodels\models
  }%
}
\newcommand*{\@rmodels}[2]{%
  \reflectbox{$\m@th#1#2$}%
}
\makeatother







\newcommand{\trcl}[1]{%
  \mathrm{trcl}{(#1)}
}



% Roman numerals
\makeatletter
\newcommand*{\rom}[1]{\expandafter\@slowromancap\romannumeral #1@}
\makeatother
% \\def \\b\([a-zA-Z]\) {\\boldsymbol{[a-zA-z]}}
% \\DeclareMathOperator{\\b\1}{\\textbf{\1}}


\DeclareMathOperator{\bx}{\textbf{x}}
\DeclareMathOperator{\bz}{\textbf{z}}
\DeclareMathOperator{\bff}{\textbf{f}}
\DeclareMathOperator{\ba}{\textbf{a}}
\DeclareMathOperator{\bk}{\textbf{k}}
\DeclareMathOperator{\bs}{\textbf{s}}
\DeclareMathOperator{\bh}{\textbf{h}}
\DeclareMathOperator{\bc}{\textbf{c}}
\DeclareMathOperator{\br}{\textbf{r}}
\DeclareMathOperator{\bi}{\textbf{i}}
\DeclareMathOperator{\bj}{\textbf{j}}
\DeclareMathOperator{\bn}{\textbf{n}}
\DeclareMathOperator{\be}{\textbf{e}}
\DeclareMathOperator{\bo}{\textbf{o}}
\DeclareMathOperator{\bU}{\textbf{U}}
\DeclareMathOperator{\bL}{\textbf{L}}
\DeclareMathOperator{\bV}{\textbf{V}}
\def \bzero {\mathbf{0}}
\def \btwo {\mathbf{2}}
\DeclareMathOperator{\bv}{\textbf{v}}
\DeclareMathOperator{\bp}{\textbf{p}}
\DeclareMathOperator{\bI}{\textbf{I}}
\def \dbI {\dot{\bI}}
\DeclareMathOperator{\bM}{\textbf{M}}
\DeclareMathOperator{\bN}{\textbf{N}}
\DeclareMathOperator{\bK}{\textbf{K}}
\DeclareMathOperator{\bt}{\textbf{t}}
\DeclareMathOperator{\bb}{\textbf{b}}
\DeclareMathOperator{\bA}{\textbf{A}}
\DeclareMathOperator{\bX}{\textbf{X}}
\DeclareMathOperator{\bu}{\textbf{u}}
\DeclareMathOperator{\bS}{\textbf{S}}
\DeclareMathOperator{\bZ}{\textbf{Z}}
\DeclareMathOperator{\bJ}{\textbf{J}}
\DeclareMathOperator{\by}{\textbf{y}}
\DeclareMathOperator{\bw}{\textbf{w}}
\DeclareMathOperator{\bT}{\textbf{T}}
\DeclareMathOperator{\bF}{\textbf{F}}
\DeclareMathOperator{\bmm}{\textbf{m}}
\DeclareMathOperator{\bW}{\textbf{W}}
\DeclareMathOperator{\bR}{\textbf{R}}
\DeclareMathOperator{\bC}{\textbf{C}}
\DeclareMathOperator{\bD}{\textbf{D}}
\DeclareMathOperator{\bE}{\textbf{E}}
\DeclareMathOperator{\bQ}{\textbf{Q}}
\DeclareMathOperator{\bP}{\textbf{P}}
\DeclareMathOperator{\bY}{\textbf{Y}}
\DeclareMathOperator{\bH}{\textbf{H}}
\DeclareMathOperator{\bB}{\textbf{B}}
\DeclareMathOperator{\bG}{\textbf{G}}
\def \blambda {\symbf{\lambda}}
\def \boldeta {\symbf{\eta}}
\def \balpha {\symbf{\alpha}}
\def \bbeta {\symbf{\beta}}
\def \bgamma {\symbf{\gamma}}
\def \bxi {\symbf{\xi}}
\def \bLambda {\symbf{\Lambda}}
\def \bGamma {\symbf{\Gamma}}

\newcommand{\bto}{{\boldsymbol{\to}}}
\newcommand{\Ra}{\Rightarrow}
\newcommand\und[1]{\underline{#1}}
\newcommand\ove[1]{\overline{#1}}
\def \bPhi {\mbfPhi}
\def \btheta {\mbftheta}
\def \bTheta {\mbfTheta}
\def \bmu {\mbfmu}
\def \bphi {\mbfphi}
\def \bSigma {\mbfSigma}
\def \lb {\left\{}
\def \rb {\right\}}
\def \la {\langle}
\def \ra {\rangle}
\def \caln {\mathcal{N}}
\def \dissum {\displaystyle\Sigma}
\def \dispro {\displaystyle\prod}

\def \A {\mathbb{A}}
\def \B {\mathbb{B}}
\def \C {\mathbb{C}}
\def \D {\mathbb{D}}
\def \E {\mathbb{E}}
\def \F {\mathbb{F}}
\def \G {\mathbb{G}}
\def \H {\mathbb{H}}
\def \I {\mathbb{I}}
\def \J {\mathbb{J}}
\def \K {\mathbb{K}}
\def \L {\mathbb{L}}
\def \M {\mathbb{M}}
\def \N {\mathbb{N}}
\def \O {\mathbb{O}}
\def \P {\mathbb{P}}
\def \Q {\mathbb{Q}}
\def \R {\mathbb{R}}
\def \S {\mathbb{S}}
\def \T {\mathbb{T}}
\def \U {\mathbb{U}}
\def \V {\mathbb{V}}
\def \W {\mathbb{W}}
\def \X {\mathbb{X}}
\def \Y {\mathbb{Y}}
\def \Z {\mathbb{Z}}

\def \bbone {\mathbb{1}}

\def \cala {\mathcal{A}}
\def \cale {\mathcal{E}}
\def \calb {\mathcal{B}}
\def \calq {\mathcal{Q}}
\def \calp {\mathcal{P}}
\def \cals {\mathcal{S}}
\def \calx {\mathcal{X}}
\def \caly {\mathcal{Y}}
\def \calg {\mathcal{G}}
\def \cald {\mathcal{D}}
\def \caln {\mathcal{N}}
\def \calr {\mathcal{R}}
\def \calt {\mathcal{T}}
\def \calm {\mathcal{M}}
\def \calw {\mathcal{W}}
\def \calc {\mathcal{C}}
\def \calv {\mathcal{V}}
\def \calf {\mathcal{F}}
\def \calk {\mathcal{K}}
\def \call {\mathcal{L}}
\def \calu {\mathcal{U}}
\def \calo {\mathcal{O}}
\def \calh {\mathcal{H}}
\def \cali {\mathcal{I}}
\def \calj {\mathcal{J}}

\def \bcup {\bigcup}

% set theory

\def \zfcc {\textbf{ZFC}^-}
\def \ac  {\textbf{AC}}
\def \gl  {\textbf{L }}
\def \gll {\textbf{L}}
\newcommand{\zfm}{$\textbf{ZF}^-$}

%\def \zfm {$\textbf{ZF}^-$}
\def \zfmm {\textbf{ZF}^-}
\def \wf {\textbf{WF }}
\def \on {\textbf{On }}
\def \cm {\textbf{M }}
\def \cn {\textbf{N }}
\def \cv {\textbf{V }}
\def \zc {\textbf{ZC }}
\def \zcm {\textbf{ZC}}
\def \zff {\textbf{ZF}}
\def \wfm {\textbf{WF}}
\def \onm {\textbf{On}}
\def \cmm {\textbf{M}}
\def \cnm {\textbf{N}}
\def \cvm {\textbf{V}}
\def \gchh {\textbf{GCH}}
\renewcommand{\restriction}{\mathord{\upharpoonright}}
%% another restriction
\newcommand\restr[2]{{% we make the whole thing an ordinary symbol
  \left.\kern-\nulldelimiterspace % automatically resize the bar with \right
  #1 % the function
  \vphantom{\big|} % pretend it's a little taller at normal size
  \right|_{#2} % this is the delimiter
  }}

\def \pred {\text{pred}}

\def \rank {\text{rank}}
\def \con {\text{Con}}
\def \deff {\text{Def}}


\def \uin {\underline{\in}}
\def \oin {\overline{\in}}
\def \uR {\underline{R}}
\def \oR {\overline{R}}
\def \uP {\underline{P}}
\def \oP {\overline{P}}

\def \dsum {\displaystyle\sum}

\def \Ra {\Rightarrow}

\def \e {\enspace}

\def \sgn {\operatorname{sgn}}
\def \gen {\operatorname{gen}}
\def \Hom {\operatorname{Hom}}
\def \hom {\operatorname{hom}}
\def \Sub {\operatorname{Sub}}

\def \supp {\operatorname{supp}}

\def \epiarrow {\twoheadarrow}
\def \monoarrow {\rightarrowtail}
\def \rrarrow {\rightrightarrows}

% \def \minus {\text{-}}
% \newcommand{\minus}{\scalebox{0.75}[1.0]{$-$}}
% \DeclareUnicodeCharacter{002D}{\minus}


\def \tril {\triangleleft}

\def \ISigma {\text{I}\Sigma}
\def \IDelta {\text{I}\Delta}
\def \IPi {\text{I}\Pi}
\def \ACF {\textsf{ACF}}
\def \RCF {\textsf{RCF}}
\def \GL {\text{GL}}
\def \PGL {\text{PGL}}
\def \equal {=}

\def \degree {\text{degree}}
\def \app {\text{App}}
\def \FV {\text{FV}}
\def \conv {\text{conv}}
\def \cont {\text{cont}}
\DeclareMathOperator{\cl}{\text{cl}}
\DeclareMathOperator{\sg}{sg}
\DeclareMathOperator{\trdeg}{trdeg}
\def \Ord {\text{Ord}}

\DeclareMathOperator{\cf}{cf}
\DeclareMathOperator{\zfc}{ZFC}

%\DeclareMathOperator{\Th}{Th}
%\def \th {\text{Th}}
% \newcommand{\th}{\text{Th}}
\DeclareMathOperator{\type}{type}
\DeclareMathOperator{\zf}{\textbf{ZF}}
\def \fa {\mathfrak{a}}
\def \fb {\mathfrak{b}}
\def \fc {\mathfrak{c}}
\def \fd {\mathfrak{d}}
\def \fe {\mathfrak{e}}
\def \ff {\mathfrak{f}}
\def \fg {\mathfrak{g}}
\def \fh {\mathfrak{h}}
%\def \fi {\mathfrak{i}}
\def \fj {\mathfrak{j}}
\def \fk {\mathfrak{k}}
\def \fl {\mathfrak{l}}
\def \fm {\mathfrak{m}}
\def \fn {\mathfrak{n}}
\def \fo {\mathfrak{o}}
\def \fp {\mathfrak{p}}
\def \fq {\mathfrak{q}}
\def \fr {\mathfrak{r}}
\def \fs {\mathfrak{s}}
\def \ft {\mathfrak{t}}
\def \fu {\mathfrak{u}}
\def \fv {\mathfrak{v}}
\def \fw {\mathfrak{w}}
\def \fx {\mathfrak{x}}
\def \fy {\mathfrak{y}}
\def \fz {\mathfrak{z}}
\def \fA {\mathfrak{A}}
\def \fB {\mathfrak{B}}
\def \fC {\mathfrak{C}}
\def \fD {\mathfrak{D}}
\def \fE {\mathfrak{E}}
\def \fF {\mathfrak{F}}
\def \fG {\mathfrak{G}}
\def \fH {\mathfrak{H}}
\def \fI {\mathfrak{I}}
\def \fJ {\mathfrak{J}}
\def \fK {\mathfrak{K}}
\def \fL {\mathfrak{L}}
\def \fM {\mathfrak{M}}
\def \fN {\mathfrak{N}}
\def \fO {\mathfrak{O}}
\def \fP {\mathfrak{P}}
\def \fQ {\mathfrak{Q}}
\def \fR {\mathfrak{R}}
\def \fS {\mathfrak{S}}
\def \fT {\mathfrak{T}}
\def \fU {\mathfrak{U}}
\def \fV {\mathfrak{V}}
\def \fW {\mathfrak{W}}
\def \fX {\mathfrak{X}}
\def \fY {\mathfrak{Y}}
\def \fZ {\mathfrak{Z}}

\def \sfA {\textsf{A}}
\def \sfB {\textsf{B}}
\def \sfC {\textsf{C}}
\def \sfD {\textsf{D}}
\def \sfE {\textsf{E}}
\def \sfF {\textsf{F}}
\def \sfG {\textsf{G}}
\def \sfH {\textsf{H}}
\def \sfI {\textsf{I}}
\def \sfJ {\textsf{J}}
\def \sfK {\textsf{K}}
\def \sfL {\textsf{L}}
\def \sfM {\textsf{M}}
\def \sfN {\textsf{N}}
\def \sfO {\textsf{O}}
\def \sfP {\textsf{P}}
\def \sfQ {\textsf{Q}}
\def \sfR {\textsf{R}}
\def \sfS {\textsf{S}}
\def \sfT {\textsf{T}}
\def \sfU {\textsf{U}}
\def \sfV {\textsf{V}}
\def \sfW {\textsf{W}}
\def \sfX {\textsf{X}}
\def \sfY {\textsf{Y}}
\def \sfZ {\textsf{Z}}
\def \sfa {\textsf{a}}
\def \sfb {\textsf{b}}
\def \sfc {\textsf{c}}
\def \sfd {\textsf{d}}
\def \sfe {\textsf{e}}
\def \sff {\textsf{f}}
\def \sfg {\textsf{g}}
\def \sfh {\textsf{h}}
\def \sfi {\textsf{i}}
\def \sfj {\textsf{j}}
\def \sfk {\textsf{k}}
\def \sfl {\textsf{l}}
\def \sfm {\textsf{m}}
\def \sfn {\textsf{n}}
\def \sfo {\textsf{o}}
\def \sfp {\textsf{p}}
\def \sfq {\textsf{q}}
\def \sfr {\textsf{r}}
\def \sfs {\textsf{s}}
\def \sft {\textsf{t}}
\def \sfu {\textsf{u}}
\def \sfv {\textsf{v}}
\def \sfw {\textsf{w}}
\def \sfx {\textsf{x}}
\def \sfy {\textsf{y}}
\def \sfz {\textsf{z}}

\def \bara {\bbar{a}}
\def \barb {\bbar{b}}
\def \barc {\bbar{c}}
\def \bard {\bbar{d}}
\def \bare {\bbar{e}}
\def \barf {\bbar{f}}
\def \barg {\bbar{g}}
\def \barh {\bbar{h}}
\def \bari {\bbar{i}}
\def \barj {\bbar{j}}
\def \bark {\bbar{k}}
\def \barl {\bbar{l}}
\def \barm {\bbar{m}}
\def \barn {\bbar{n}}
\def \baro {\bbar{o}}
\def \barp {\bbar{p}}
\def \barq {\bbar{q}}
\def \barr {\bbar{r}}
\def \bars {\bbar{s}}
\def \bart {\bbar{t}}
\def \baru {\bbar{u}}
\def \barv {\bbar{v}}
\def \barw {\bbar{w}}
\def \barx {\bbar{x}}
\def \bary {\bbar{y}}
\def \barz {\bbar{z}}
\def \barA {\bbar{A}}
\def \barB {\bbar{B}}
\def \barC {\bbar{C}}
\def \barD {\bbar{D}}
\def \barE {\bbar{E}}
\def \barF {\bbar{F}}
\def \barG {\bbar{G}}
\def \barH {\bbar{H}}
\def \barI {\bbar{I}}
\def \barJ {\bbar{J}}
\def \barK {\bbar{K}}
\def \barL {\bbar{L}}
\def \barM {\bbar{M}}
\def \barN {\bbar{N}}
\def \barO {\bbar{O}}
\def \barP {\bbar{P}}
\def \barQ {\bbar{Q}}
\def \barR {\bbar{R}}
\def \barS {\bbar{S}}
\def \barT {\bbar{T}}
\def \barU {\bbar{U}}
\def \barVV {\bbar{V}}
\def \barW {\bbar{W}}
\def \barX {\bbar{X}}
\def \barY {\bbar{Y}}
\def \barZ {\bbar{Z}}

\def \hata {\hat{a}}
\def \hatb {\hat{b}}
\def \hatc {\hat{c}}
\def \hatd {\hat{d}}
\def \hate {\hat{e}}
\def \hatf {\hat{f}}
\def \hatg {\hat{g}}
\def \hath {\hat{h}}
\def \hati {\hat{i}}
\def \hatj {\hat{j}}
\def \hatk {\hat{k}}
\def \hatl {\hat{l}}
\def \hatm {\hat{m}}
\def \hatn {\hat{n}}
\def \hato {\hat{o}}
\def \hatp {\hat{p}}
\def \hatq {\hat{q}}
\def \hatr {\hat{r}}
\def \hats {\hat{s}}
\def \hatt {\hat{t}}
\def \hatu {\hat{u}}
\def \hatv {\hat{v}}
\def \hatw {\hat{w}}
\def \hatx {\hat{x}}
\def \haty {\hat{y}}
\def \hatz {\hat{z}}
\def \hatA {\hat{A}}
\def \hatB {\hat{B}}
\def \hatC {\hat{C}}
\def \hatD {\hat{D}}
\def \hatE {\hat{E}}
\def \hatF {\hat{F}}
\def \hatG {\hat{G}}
\def \hatH {\hat{H}}
\def \hatI {\hat{I}}
\def \hatJ {\hat{J}}
\def \hatK {\hat{K}}
\def \hatL {\hat{L}}
\def \hatM {\hat{M}}
\def \hatN {\hat{N}}
\def \hatO {\hat{O}}
\def \hatP {\hat{P}}
\def \hatQ {\hat{Q}}
\def \hatR {\hat{R}}
\def \hatS {\hat{S}}
\def \hatT {\hat{T}}
\def \hatU {\hat{U}}
\def \hatVV {\hat{V}}
\def \hatW {\hat{W}}
\def \hatX {\hat{X}}
\def \hatY {\hat{Y}}
\def \hatZ {\hat{Z}}

\def \barfM {\bbar{\fM}}
\def \barfN {\bbar{\fN}}

\def \tila {\tilde{a}}
\def \tilb {\tilde{b}}
\def \tilc {\tilde{c}}
\def \tild {\tilde{d}}
\def \tile {\tilde{e}}
\def \tilf {\tilde{f}}
\def \tilg {\tilde{g}}
\def \tilh {\tilde{h}}
\def \tili {\tilde{i}}
\def \tilj {\tilde{j}}
\def \tilk {\tilde{k}}
\def \till {\tilde{l}}
\def \tilm {\tilde{m}}
\def \tiln {\tilde{n}}
\def \tilo {\tilde{o}}
\def \tilp {\tilde{p}}
\def \tilq {\tilde{q}}
\def \tilr {\tilde{r}}
\def \tils {\tilde{s}}
\def \tilt {\tilde{t}}
\def \tilu {\tilde{u}}
\def \tilv {\tilde{v}}
\def \tilw {\tilde{w}}
\def \tilx {\tilde{x}}
\def \tily {\tilde{y}}
\def \tilz {\tilde{z}}
\def \tilA {\tilde{A}}
\def \tilB {\tilde{B}}
\def \tilC {\tilde{C}}
\def \tilD {\tilde{D}}
\def \tilE {\tilde{E}}
\def \tilF {\tilde{F}}
\def \tilG {\tilde{G}}
\def \tilH {\tilde{H}}
\def \tilI {\tilde{I}}
\def \tilJ {\tilde{J}}
\def \tilK {\tilde{K}}
\def \tilL {\tilde{L}}
\def \tilM {\tilde{M}}
\def \tilN {\tilde{N}}
\def \tilO {\tilde{O}}
\def \tilP {\tilde{P}}
\def \tilQ {\tilde{Q}}
\def \tilR {\tilde{R}}
\def \tilS {\tilde{S}}
\def \tilT {\tilde{T}}
\def \tilU {\tilde{U}}
\def \tilVV {\tilde{V}}
\def \tilW {\tilde{W}}
\def \tilX {\tilde{X}}
\def \tilY {\tilde{Y}}
\def \tilZ {\tilde{Z}}



\def \barnu {\bar{\nu}}
\def \barrho {\bar{\rho}}
%\DeclareMathOperator{\ker}{ker}
\DeclareMathOperator{\im}{im}

\DeclareMathOperator{\inn}{Inn}
\DeclareMathOperator{\rel}{rel}
%\DeclareMathOperator{\AC}{\textbf{AC}}
\DeclareMathOperator{\cod}{cod}
\DeclareMathOperator{\dom}{dom}
\DeclareMathOperator{\card}{card}
\DeclareMathOperator{\ran}{ran}
\DeclareMathOperator{\textd}{d}
\DeclareMathOperator{\td}{d}
\DeclareMathOperator{\id}{id}
\DeclareMathOperator{\LT}{LT}
\DeclareMathOperator{\Mat}{Mat}
\DeclareMathOperator{\Eq}{Eq}
\DeclareMathOperator{\irr}{irr}
\DeclareMathOperator{\Fr}{Fr}
\DeclareMathOperator{\Gal}{Gal}
\DeclareMathOperator{\lcm}{lcm}
\DeclareMathOperator{\alg}{\text{alg}}
\DeclareMathOperator{\Th}{Th}
%\DeclareMathOperator{\deg}{deg}


% \varprod
\DeclareSymbolFont{largesymbolsA}{U}{txexa}{m}{n}
\DeclareMathSymbol{\varprod}{\mathop}{largesymbolsA}{16}
% \DeclareMathSymbol{\tonm}{\boldsymbol{\to}\textbf{Nm}}
\def \tonm {\bto\textbf{Nm}}
\def \tohm {\bto\textbf{Hm}}

% Category theory
\DeclareMathOperator{\ob}{ob}
\DeclareMathOperator{\Ab}{\textbf{Ab}}
\DeclareMathOperator{\Alg}{\textbf{Alg}}
\DeclareMathOperator{\Rng}{\textbf{Rng}}
\DeclareMathOperator{\Sets}{\textbf{Sets}}
\DeclareMathOperator{\Met}{\textbf{Met}}
\DeclareMathOperator{\BA}{\textbf{BA}}
\DeclareMathOperator{\Mon}{\textbf{Mon}}
\DeclareMathOperator{\Top}{\textbf{Top}}
\DeclareMathOperator{\hTop}{\textbf{hTop}}
\DeclareMathOperator{\Aut}{\textbf{Aut}}
\DeclareMathOperator{\RMod}{R-\textbf{Mod}}
\DeclareMathOperator{\RAlg}{R-\textbf{Alg}}
\DeclareMathOperator{\LF}{LF}
\DeclareMathOperator{\op}{op}
\DeclareMathOperator{\Rings}{\textbf{Rings}}
\DeclareMathOperator{\Groups}{\textbf{Groups}}
% Algebraic Topology
\DeclareMathOperator{\obj}{obj}
% Model theory
\def\ind{\mathrel{\raise0.2ex\hbox{\ooalign{\hidewidth$\vert$\hidewidth\cr\raise-0.9ex\hbox{$\smile$}}}}}
\DeclareMathOperator{\acl}{acl}
\DeclareMathOperator{\dcl}{dcl}
\DeclareMathOperator{\tp}{tp}
\DeclareMathOperator{\Diag}{Diag}
\DeclareMathOperator{\el}{el}
\DeclareMathOperator{\depth}{depth}
\def \FO {\text{FO}}
\DeclareMathOperator{\fin}{fin}
\DeclareMathOperator{\qr}{qr}
\DeclareMathOperator{\Mod}{Mod}
\DeclareMathOperator{\Def}{Def}
\DeclareMathOperator{\TC}{TC}
\DeclareMathOperator{\KH}{KH}
\DeclareMathOperator{\Part}{Part}
\DeclareMathOperator{\Infset}{\textsf{Infset}}
\DeclareMathOperator{\DLO}{\textsf{DLO}}
\DeclareMathOperator{\PA}{\textsf{PA}}
\DeclareMathOperator{\DAG}{\textsf{DAG}}
\DeclareMathOperator{\ODAG}{\textsf{ODAG}}
\DeclareMathOperator{\sfMod}{\textsf{Mod}}
\DeclareMathOperator{\AbG}{\textsf{AbG}}
\DeclareMathOperator{\sfACF}{\textsf{ACF}}
\DeclareMathOperator{\DCF}{\textsf{DCF}}
% Computability Theorem
\DeclareMathOperator{\Tot}{Tot}
\DeclareMathOperator{\graph}{graph}
\DeclareMathOperator{\Fin}{Fin}
\DeclareMathOperator{\Cof}{Cof}
\DeclareMathOperator{\lh}{lh}
% Commutative Algebra
\DeclareMathOperator{\ord}{ord}
\DeclareMathOperator{\Idem}{Idem}
\DeclareMathOperator{\zdiv}{z.div}
\DeclareMathOperator{\Frac}{Frac}
\DeclareMathOperator{\rad}{rad}
\DeclareMathOperator{\nil}{nil}
\DeclareMathOperator{\Ann}{Ann}
\DeclareMathOperator{\End}{End}
\DeclareMathOperator{\coim}{coim}
\DeclareMathOperator{\coker}{coker}
\DeclareMathOperator{\Bil}{Bil}
\DeclareMathOperator{\Tril}{Tril}
% Topology
\DeclareMathOperator{\diam}{diam}
\newcommand{\interior}[1]{%
  {\kern0pt#1}^{\mathrm{o}}%
}

\DeclareMathOperator*{\bigdoublewedge}{\bigwedge\mkern-15mu\bigwedge}
\DeclareMathOperator*{\bigdoublevee}{\bigvee\mkern-15mu\bigvee}

% \makeatletter
% \newcommand{\vect}[1]{%
%   \vbox{\m@th \ialign {##\crcr
%   \vectfill\crcr\noalign{\kern-\p@ \nointerlineskip}
%   $\hfil\displaystyle{#1}\hfil$\crcr}}}
% \def\vectfill{%
%   $\m@th\smash-\mkern-7mu%
%   \cleaders\hbox{$\mkern-2mu\smash-\mkern-2mu$}\hfill
%   \mkern-7mu\raisebox{-3.81pt}[\p@][\p@]{$\mathord\mathchar"017E$}$}

% \newcommand{\amsvect}{%
%   \mathpalette {\overarrow@\vectfill@}}
% \def\vectfill@{\arrowfill@\relbar\relbar{\raisebox{-3.81pt}[\p@][\p@]{$\mathord\mathchar"017E$}}}

% \newcommand{\amsvectb}{%
% \newcommand{\vect}{%
%   \mathpalette {\overarrow@\vectfillb@}}
% \newcommand{\vecbar}{%
%   \scalebox{0.8}{$\relbar$}}
% \def\vectfillb@{\arrowfill@\vecbar\vecbar{\raisebox{-4.35pt}[\p@][\p@]{$\mathord\mathchar"017E$}}}
% \makeatother
% \bigtimes

\DeclareFontFamily{U}{mathx}{\hyphenchar\font45}
\DeclareFontShape{U}{mathx}{m}{n}{
      <5> <6> <7> <8> <9> <10>
      <10.95> <12> <14.4> <17.28> <20.74> <24.88>
      mathx10
      }{}
\DeclareSymbolFont{mathx}{U}{mathx}{m}{n}
\DeclareMathSymbol{\bigtimes}{1}{mathx}{"91}
% \odiv
\DeclareFontFamily{U}{matha}{\hyphenchar\font45}
\DeclareFontShape{U}{matha}{m}{n}{
      <5> <6> <7> <8> <9> <10> gen * matha
      <10.95> matha10 <12> <14.4> <17.28> <20.74> <24.88> matha12
      }{}
\DeclareSymbolFont{matha}{U}{matha}{m}{n}
\DeclareMathSymbol{\odiv}         {2}{matha}{"63}


\newcommand\subsetsim{\mathrel{%
  \ooalign{\raise0.2ex\hbox{\scalebox{0.9}{$\subset$}}\cr\hidewidth\raise-0.85ex\hbox{\scalebox{0.9}{$\sim$}}\hidewidth\cr}}}
\newcommand\simsubset{\mathrel{%
  \ooalign{\raise-0.2ex\hbox{\scalebox{0.9}{$\subset$}}\cr\hidewidth\raise0.75ex\hbox{\scalebox{0.9}{$\sim$}}\hidewidth\cr}}}

\newcommand\simsubsetsim{\mathrel{%
  \ooalign{\raise0ex\hbox{\scalebox{0.8}{$\subset$}}\cr\hidewidth\raise1ex\hbox{\scalebox{0.75}{$\sim$}}\hidewidth\cr\raise-0.95ex\hbox{\scalebox{0.8}{$\sim$}}\cr\hidewidth}}}
\newcommand{\stcomp}[1]{{#1}^{\mathsf{c}}}

\setlength{\baselineskip}{0.8in}

\stackMath
\newcommand\yrightarrow[2][]{\mathrel{%
  \setbox2=\hbox{\stackon{\scriptstyle#1}{\scriptstyle#2}}%
  \stackunder[0pt]{%
    \xrightarrow{\makebox[\dimexpr\wd2\relax]{$\scriptstyle#2$}}%
  }{%
   \scriptstyle#1\,%
  }%
}}
\newcommand\yleftarrow[2][]{\mathrel{%
  \setbox2=\hbox{\stackon{\scriptstyle#1}{\scriptstyle#2}}%
  \stackunder[0pt]{%
    \xleftarrow{\makebox[\dimexpr\wd2\relax]{$\scriptstyle#2$}}%
  }{%
   \scriptstyle#1\,%
  }%
}}
\newcommand\yRightarrow[2][]{\mathrel{%
  \setbox2=\hbox{\stackon{\scriptstyle#1}{\scriptstyle#2}}%
  \stackunder[0pt]{%
    \xRightarrow{\makebox[\dimexpr\wd2\relax]{$\scriptstyle#2$}}%
  }{%
   \scriptstyle#1\,%
  }%
}}
\newcommand\yLeftarrow[2][]{\mathrel{%
  \setbox2=\hbox{\stackon{\scriptstyle#1}{\scriptstyle#2}}%
  \stackunder[0pt]{%
    \xLeftarrow{\makebox[\dimexpr\wd2\relax]{$\scriptstyle#2$}}%
  }{%
   \scriptstyle#1\,%
  }%
}}

\newcommand\altxrightarrow[2][0pt]{\mathrel{\ensurestackMath{\stackengine%
  {\dimexpr#1-7.5pt}{\xrightarrow{\phantom{#2}}}{\scriptstyle\!#2\,}%
  {O}{c}{F}{F}{S}}}}
\newcommand\altxleftarrow[2][0pt]{\mathrel{\ensurestackMath{\stackengine%
  {\dimexpr#1-7.5pt}{\xleftarrow{\phantom{#2}}}{\scriptstyle\!#2\,}%
  {O}{c}{F}{F}{S}}}}

\newenvironment{bsm}{% % short for 'bracketed small matrix'
  \left[ \begin{smallmatrix} }{%
  \end{smallmatrix} \right]}

\newenvironment{psm}{% % short for ' small matrix'
  \left( \begin{smallmatrix} }{%
  \end{smallmatrix} \right)}

\newcommand{\bbar}[1]{\mkern 1.5mu\overline{\mkern-1.5mu#1\mkern-1.5mu}\mkern 1.5mu}

\newcommand{\bigzero}{\mbox{\normalfont\Large\bfseries 0}}
\newcommand{\rvline}{\hspace*{-\arraycolsep}\vline\hspace*{-\arraycolsep}}

\font\zallman=Zallman at 40pt
\font\elzevier=Elzevier at 40pt

\newcommand\isoto{\stackrel{\textstyle\sim}{\smash{\longrightarrow}\rule{0pt}{0.4ex}}}
\newcommand\embto{\stackrel{\textstyle\prec}{\smash{\longrightarrow}\rule{0pt}{0.4ex}}}

% from http://www.actual.world/resources/tex/doc/TikZ.pdf

\tikzset{
modal/.style={>=stealth’,shorten >=1pt,shorten <=1pt,auto,node distance=1.5cm,
semithick},
world/.style={circle,draw,minimum size=0.5cm,fill=gray!15},
point/.style={circle,draw,inner sep=0.5mm,fill=black},
reflexive above/.style={->,loop,looseness=7,in=120,out=60},
reflexive below/.style={->,loop,looseness=7,in=240,out=300},
reflexive left/.style={->,loop,looseness=7,in=150,out=210},
reflexive right/.style={->,loop,looseness=7,in=30,out=330}
}


\makeatletter
\newcommand*{\doublerightarrow}[2]{\mathrel{
  \settowidth{\@tempdima}{$\scriptstyle#1$}
  \settowidth{\@tempdimb}{$\scriptstyle#2$}
  \ifdim\@tempdimb>\@tempdima \@tempdima=\@tempdimb\fi
  \mathop{\vcenter{
    \offinterlineskip\ialign{\hbox to\dimexpr\@tempdima+1em{##}\cr
    \rightarrowfill\cr\noalign{\kern.5ex}
    \rightarrowfill\cr}}}\limits^{\!#1}_{\!#2}}}
\newcommand*{\triplerightarrow}[1]{\mathrel{
  \settowidth{\@tempdima}{$\scriptstyle#1$}
  \mathop{\vcenter{
    \offinterlineskip\ialign{\hbox to\dimexpr\@tempdima+1em{##}\cr
    \rightarrowfill\cr\noalign{\kern.5ex}
    \rightarrowfill\cr\noalign{\kern.5ex}
    \rightarrowfill\cr}}}\limits^{\!#1}}}
\makeatother

% $A\doublerightarrow{a}{bcdefgh}B$

% $A\triplerightarrow{d_0,d_1,d_2}B$

\newcommand{\floor}[1]{\lfloor #1 \rfloor}
\newcommand{\ceil}[1]{\lceil #1 \rceil}
\newcommand{\lcorner}[1]{\llcorner #1 \lrcorner}
\newcommand{\ucorner}[1]{\ulcorner #1 \urcorner}
\newcommand{\emoji}[1]{{\DejaSans #1}}
\newcommand{\vprec}{\rotatebox[origin=c]{-90}{$\prec$}}
