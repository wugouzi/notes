% Created 2021-12-31 Fri 14:47
% Intended LaTeX compiler: pdflatex
\documentclass[11pt]{article}
\usepackage[utf8]{inputenc}
\usepackage[T1]{fontenc}
\usepackage{graphicx}
\usepackage{longtable}
\usepackage{wrapfig}
\usepackage{rotating}
\usepackage[normalem]{ulem}
\usepackage{amsmath}
\usepackage{amssymb}
\usepackage{capt-of}
\usepackage{hyperref}
\graphicspath{{../../books/}}
% TIPS
% \substack{a\\b} for multiple lines text





% pdfplots will load xolor automatically without option
\usepackage[dvipsnames]{xcolor}

\usepackage{forest}
% two-line text in node by [two \\ lines]
% \begin{forest} qtree, [..] \end{forest}
\forestset{
  qtree/.style={
    baseline,
    for tree={
      parent anchor=south,
      child anchor=north,
      align=center,
      inner sep=1pt,
    }}}
%\usepackage{flexisym}
% load order of mathtools and mathabx, otherwise conflict overbrace

\usepackage{mathtools}
%\usepackage{fourier}
\usepackage{pgfplots}
\usepackage{amsthm, mathabx,  amsmath, commath}
\usepackage{amsfonts}

\usepackage{empheq}
\usepackage{tikz}
\usetikzlibrary{arrows.meta}
\usepackage[most]{tcolorbox}

\newtheorem{theorem}{Theorem}[section]
\newtheorem{definition}{Definition}[section]
\newtheorem{corollary}{Corollary}[section]
\newtheorem{example}{Example}[section]
\newtheorem{lemma}{Lemma}[section]
\newtheorem{proposition}{Proposition}[section]

\newcommand{\bl}[1] {\boldsymbol{#1}}
\newcommand{\Wt}[1] {\stackrel{\sim}{\smash{#1}\rule{0pt}{1.1ex}}}
\newcommand{\wt}[1] {\widetilde{#1}}


%For boxed texts in align, use Aboxed{}
%otherwise use boxed{}

\DeclareMathSymbol{\widehatsym}{\mathord}{largesymbols}{"62}
\newcommand\lowerwidehatsym{%
  \text{\smash{\raisebox{-1.3ex}{%
    $\widehatsym$}}}}
\newcommand\fixwidehat[1]{%
  \mathchoice
    {\accentset{\displaystyle\lowerwidehatsym}{#1}}
    {\accentset{\textstyle\lowerwidehatsym}{#1}}
    {\accentset{\scriptstyle\lowerwidehatsym}{#1}}
    {\accentset{\scriptscriptstyle\lowerwidehatsym}{#1}}
}

\usepackage{graphicx}
    
% text on arrow for xRightarrow
\makeatletter
%\newcommand{\xRightarrow}[2][]{\ext@arrow 0359\Rightarrowfill@{#1}{#2}}
\makeatother


\def \bx {\boldsymbol{x}}
\def \ba {\boldsymbol{a}}
\def \bI {\boldsymbol{I}}
\def \bt {\boldsymbol{t}}
\def \bb {\boldsymbol{b}}
\def \bA {\boldsymbol{A}}
\def \bX {\boldsymbol{X}}
\def \bu {\boldsymbol{u}}
\def \bS {\boldsymbol{S}}
\def \bZ {\boldsymbol{Z}}
\def \bz {\boldsymbol{z}}
\def \by {\boldsymbol{y}}
\def \bw {\boldsymbol{w}}
\def \bT {\boldsymbol{T}}
\def \bS {\boldsymbol{S}}
\def \bm {\boldsymbol{m}}
\def \bW {\boldsymbol{W}}
\def \bY {\boldsymbol{Y}}
\def \bH {\boldsymbol{H}}
\def \blambda {\boldsymbol{\lambda}}
\def \bPhi {\boldsymbol{\Phi}}
\def \btheta {\boldsymbol{\theta}}
\def \bmu {\boldsymbol{\mu}}
\def \bphi {\boldsymbol{\phi}}
\def \bSigma {\boldsymbol{\Sigma}}
\def \lb {\left\{}
\def \rb {\right\}}
\def \caln {\mathcal{N}}
\def \dissum {\displaystyle\Sigma}
\def \dispro {\displaystyle\prod}
\def \E {\mathbb{E}}
\def \Q {\mathbb{Q}}
\def \V {\mathbb{V}}
\def \R {\mathbb{R}}
\def \calq {\mathcal{Q}}
\def \calg {\mathcal{G}}
\def \caln {\mathcal{N}}
\def \calr {\mathcal{R}}
\def \calm {\mathcal{M}}
\def \calc {\mathcal{C}}
\def \bcup {\bigcup}

\makeindex
\usepackage[UTF8]{ctex}
\def \pred {\text{pred}}
\def \quo {\text{quo}}
\def \rem {\text{rem}}
\author{wu}
\date{\today}
\title{First Order Logic}
\hypersetup{
 pdfauthor={wu},
 pdftitle={First Order Logic},
 pdfkeywords={},
 pdfsubject={},
 pdfcreator={Emacs 28.0.90 (Org mode 9.6)}, 
 pdflang={English}}
\begin{document}

\maketitle
\tableofcontents

\section{递归论基础}
\label{sec:org316feae}
\subsection{primitive recursive}
\label{sec:org1d6c484}
\begin{definition}[]
初始函数
\begin{enumerate}
\item 零函数\(Z(x)=0\)
\item 后继函数\(S(x)=x+1\)
\item 投射函数\(\pi_i^n(x_1,\dots,x_i,\dots,x_n)=x_i\)
\end{enumerate}
\end{definition}

\begin{definition}[]
设\(g:\N^n\to\N\)与\(h:\N^{n+2}\to\N\),称\(f:\N^{n+1}\to\N\)是从\(g\)和\(h\) \textbf{经原始递归得到的} ,如果
\begin{enumerate}
\item \(f(\barx,0)=g(\barx)\)
\item \(f(\barx,n+1)=h(\barx,f(\barx,n),n)\)
\end{enumerate}
\end{definition}

\begin{definition}[]
全体原始递归函数的集合\(C\)是最小的满足以下条件的自然数上的函数集合
\begin{enumerate}
\item 初始函数\(\subseteq C\)
\item 复合封闭
\item 原始递归封闭
\end{enumerate}


称\(C\)中的元素为原始递归函数
\end{definition}

\begin{lemma}[]
以下为原始递归函数
\begin{enumerate}
\item 加法
\item \(C_k^n(x_1,\dots,x_n)=k\)
\item \(x\cdot y\), \(x^y\),\(x!\)
\item 非零检测和零检测
\begin{equation*}
\sigma(x)=
\begin{cases}
0&x=0\\
1
\end{cases}\quad
\delta(x)=
\begin{cases}
1&x=0\\
0
\end{cases}
\end{equation*}
\item 前驱函数\(\pred(x)\)
\item 截断减法
\begin{equation*}
x\dot-y=
\begin{cases}
0&x<y\\
x-y&x\ge y
\end{cases}
\end{equation*}
\end{enumerate}
\end{lemma}

\begin{proof}
\(\sigma(0)=0\),\(\sigma(n+1)=C_1^2(n,\sigma(n))\)

\(\pred(0)=0\),\(\pred(n+1)=\pi_1^2(n,\pred(n))\)
\end{proof}

\begin{lemma}[]
\(f:\N^k\to\N\) p.r., \(g:\N^r\to\N\)
\begin{equation*}
g(x_1,\dots,x_r)=f(y_1,\dots,y_k)
\end{equation*}
\(y_j\) is either \(x_i\) or a constant, then \(g\) is p.r.
\end{lemma}

\begin{proof}
\(h_1,\dots,h_k:\N^r\to\N\)
\begin{itemize}
\item if \(y_j\) is \(x_i\), then \(h_j(x_1,\dots,x_r)=\pi_i^r(x_1,\dots,x_r)\)
\item if \(y_j\) is a constant \(k\in\N\), then \(h_j(x_1,\dots,x_r)=C_k^r(x_1,\dots,x_r)\)
\end{itemize}


\begin{equation*}
g(x_1,\dots,x_r)=f(h_1(x_1,\dots,x_r),\dots,(h_k(x_1,\dots,x_r)))
\end{equation*}
\end{proof}

\begin{definition}[]
\(A\subseteq\N^k\) is \textbf{primitive recursive} if its characteristic function is p.r.
\end{definition}

\begin{lemma}[]
\begin{enumerate}
\item If \(A,B\subseteq\N^k\) is p.r., then \(\N^k\setminus A\), \(A\cup B\), \(A\cap B\) is p.r.
\item if \(P,Q\) is p.r. predicate, then \(\neg P\), \(P\vee Q\), \(P\wedge Q\) is p.r.
\end{enumerate}
\end{lemma}

\begin{proof}
\(1\dot-\chi_A(x)\), \(\sigma(\chi_A(x)+\chi_B(x))\), \(\chi_A(x)\cdot\chi_B(x)\)
\end{proof}

if \(f:\N^k\to\N\) is p.r., then
\begin{align*}
&\{x\in\N^k\mid f(x)=0\}\\
&\{x\in\N^k\mid f(x)>0\}
\end{align*}
is p.r.

\begin{lemma}[]
If \(f_1,f_2\) is \(k\)-ary p.r., \(P\) p.r. predicate, then
\begin{equation*}
f(\barx)=
\begin{cases}
f_1(\barx)&P(\barx)\\
f_2(\barx)
\end{cases}
\end{equation*}
is p.r.
\end{lemma}

\begin{proof}
\(f(x)=\chi_P(x)f_1(x)+(1\dot-\chi_P(x))f_2(x)\)
\end{proof}

\begin{lemma}[]
\(\quo(x,y)\) and \(\rem(x,y)\) are p.r.
\end{lemma}

\begin{proof}
Intuition
\begin{align*}
&\rem(x,y+1)=
\begin{cases}
\rem(x,y)+1&\rem(x,y)+1<x\\
0
\end{cases}\\
&\quo(x,y+1)=
\begin{cases}
\quo(x,y)&\rem(x,y)+1<x\\
\quo(x,y)+1
\end{cases}
\end{align*}
solution
\begin{align*}
&\rem(x,0)=0\\
&\rem(x,y+1)=(\rem(x,y)+1)\sigma(x-\rem(x,y)-1)\\
&\quo(x,0)=0\\
&\quo(x,y+1)=\quo(x,y)\sigma(x-\rem(x,y)-1)+(\quo(x,y)+1)\delta(x-\rem(x,y)-1)
\end{align*}
\end{proof}

\begin{definition}[]
\begin{enumerate}
\item \((\exists x<a)\phi(x):=\exists x(x<a\wedge\phi(x))\)
\item \((\forall x<a)\phi(x):=\forall x(x<a\to\phi(x))\)
\end{enumerate}


\textbf{bounded quantifier}
\end{definition}

\begin{lemma}[]
If \(P(\barx,y)\) is a p.r. predicate
\begin{enumerate}
\item predicate
\begin{align*}
&E(\barx,y):=(\exists z\le y)P(\barx,z)\\
&A(\barx,y):=(\forall z\le y)P(\barx,z)
\end{align*}
are p.r.
\item function
\begin{equation*}
f(\barx,y):=(\mu z\le y)P(\barx,z)
\end{equation*}
is p.r.
\end{enumerate}
\end{lemma}

\begin{lemma}[]
\begin{enumerate}
\item predicate ``\(x\) divides \(y\)'' is p.r.
\item ``\(x\) is not prime'' ``\(x\) is prime'' are p.r.
\item \(p:\N\to\N\), \(n\mapsto n\text{th prime}\) is p.r.
\end{enumerate}
\end{lemma}

\begin{proof}
\(p(0)=2\). \(p(n+1)=(\mu z\le y)(z>p(n)\wedge z\text{ prime}\wedge y=p(n)!+1)\)
\end{proof}

\begin{itemize}
\item \(\la a_0,\dots,a_n\ra:=p_0^{a_0+1}\dots p_n^{a_{n}+1}\)  is the Gödel number of \((a_0,\dots,a_n)\)
\item \(\la\ra=1\)
\item \(\lh:\N\to\N\) is \(\lh(a)=\mu k\le a(p_k\nmid a)\)
\item \((a)_i:\N^2\to\N\) is \((a)_i=(\mu k\le a)(p_i^{k+2}\nmid a)\)
\item for any \(a=\la a_0,\dots,a_n\ra\), \((a)_i=a_i\)
\item concatenation function \(\tieconcat:\N^2\to\N\)
\begin{equation*}
a\tieconcat b=a\cdot\prod_{i<\lh(b)}p_{\lh(a)+i}^{(b)_i+1}
\end{equation*}
\end{itemize}



\begin{lemma}[]
\begin{enumerate}
\item Set of Gödel numbers are p.r.
\item \(\lh(a)\) and \((a)_i\) is p.r.
\item \(a\tieconcat b\) is p.r. and
\begin{equation*}
\la a_0,\dots,a_n\ra\tieconcat\la b_0,\dots,b_m\ra=\la a_0,\dots,a_n,b_0,\dots,b_m\ra
\end{equation*}
\end{enumerate}
\end{lemma}

\begin{proof}
  \begin{equation*}
\exists n\le x\left( \forall i\le n(p_i\mid x)\wedge\forall j\le x(j>n\to p_j\nmid x) \right)
  \end{equation*}
\end{proof}

function \(f(\barx,y)\),
  \begin{equation*}
F(\barx,n)=p_0^{f(\barx,0)+1}\dots p_n^{f(\barx,n)+1}
  \end{equation*}

\begin{definition}[]
function \(g(\barx)\) and \(h(\barx,y,z)\), \(f(\barx,y)\)是从\(g\)与\(h\)经 \textbf{强递归} 得到的如果
  \begin{align*}
f(\barx,0)&=g(\barx)\\
f(\barx,n+1)&=h(\barx,n,F(\barx,n))
  \end{align*}
\end{definition}

\begin{lemma}[]
如果\(f(\barx,y)\)是从\(g\)与\(h\)经强递归得到,and  \(g,h\) p.r., then \(f\) is p.r.
\end{lemma}

\begin{proof}
  \begin{align*}
F(\barx,0)&=2^{f(\barx,0)+1}=2^{g(\barx)+1}\\
F(\barx,n+1)&=F(\barx,n)p_{n+1}^{f(\barx,n+1)+1}=F(\barx,n)p_{n+1}^{h(\barx,n,F(\barx,n))+1}
  \end{align*}
Hence \(F(\barx,y)\) is p.r., so \(f(\barx,y)=(F(\barx,y))_y\) is p.r.
\end{proof}
\subsection{recursive function}
\label{sec:orge6e5fa9}
\begin{itemize}
\item 假设有一个程序可以枚举所有的原始递归函数
\item 设\(g_0,g_1,g_2,\dots\)是所有原始递归函数的枚举
\item 令\(F:\N\to\N\)为\(F(n)=g_n(n)+1\)
\item 虽然\(F\)在直观上可计算,但不属于原始递归函数
\end{itemize}


\begin{definition}[]
total function \(f:\N^{n+1}\to\N\),\(g(\barx)\)是从\(f\)通过正则极小化或正则\(\mu\)-算子得到的如果
\begin{itemize}
\item \(\forall\barx\exists yf(\barx,y)=0\)
\item \(g(\barx)\)是使得\(f(\barx,y)=0\)最小的\(y\)
\end{itemize}


记作\(g(\barx)=\mu y(f(\barx,y)=0)\)
\end{definition}

\begin{definition}[]
\begin{enumerate}
\item 全体递归函数的集合为最小的包含所有初始函数,并且对复合、原始递归、正则极小化封闭的函数集合
\item \(A\subseteq\N^k\)是递归集如果\(\chi_A\)是递归函数
\end{enumerate}
\end{definition}

\begin{definition}[]
partial function \(f\), \(g\)是从\(f\)通过极小化或者由\(\mu\)-算子得到的如果
\begin{equation*}
g(\barx)=\mu y(\forall z\le y(f(x,z)\downarrow)\wedge f(x,y)=0)
\end{equation*}
\end{definition}

\begin{definition}[]
全体部分递归函数的集合为最小的包含所有初始函数、并且怼复合、原始递归、极小化封闭的函数集合
\end{definition}

\begin{lemma}[]
Ackermann function is partial recursive
\begin{gather*}
A(0,y)=y+1,\quad A(x+1,0)=A(x,1)\\
A(x+1,y+1)=A(x,A(x+1,y))
\end{gather*}
\end{lemma}
\subsection{Turing Machine}
\label{sec:org1697bab}
规定输入向量为\((x_1,\dots,x_n)\)时,初始格局为
\begin{equation*}
q_s1^{x_1+1}01^{x_2+1}0\dots 01^{x_k+1}
\end{equation*}

输出时,格局为\(q_h1^y\),表示输出值为\(y\)

\begin{definition}[]
一个部分函数\(f:\N^k\to\N\)是被图灵机\(M\)所计算的,或者说图灵机\(M\)计算函数\(f\),如果
\begin{equation*}
f(x)=
\begin{cases}
y&\text{如果$M$对输入\(x\)的输出为\(y\)}\\
\text{没有定义}&\text{如果计算过程无限或没有终止格局}
\end{cases}
\end{equation*}
称部分函数\(f\)为图灵可计算的,如果存在一个图灵机\(M\)计算它
\end{definition}
\subsection{turing computability and partial recursive function}
\label{sec:org3a10e46}
\begin{theorem}[]
一个函数是图灵可计算的当且仅当它是部分递归的
\end{theorem}

\begin{lemma}[]
每个初始函数都是图灵可计算的
\end{lemma}

\begin{lemma}[]
任何一台标准图灵机都可以被一台单向无穷纸带图灵机模拟
\end{lemma}

\begin{corollary}[]
任何图灵可计算函数\(h\)都可以被一台加了如下限制的图灵机计算
\begin{enumerate}
\item 在初始格局中,纸带中有一个不在字母表中的新字符\$,可以在任何实现给定的位置,只要不混在输入字符
串中见
\item 计算完成后,\$左边的内容不变
\item 输出字符串的位置起始于\$右边一格
\end{enumerate}
\end{corollary}

\begin{lemma}[]
图灵可计算对复合封闭
\end{lemma}

\begin{definition}[]
\(T(e,x,z)\)表示\(z\)是图灵机\(e\)对输入\(x\)的计算过程(格局序列)的编码,称为Kleene谓词
\end{definition}

\begin{lemma}[]
Kleene predicate is p.r.
\end{lemma}

\begin{theorem}[]
存在原始递归函数\(U:\N\to\N\)和原始递归谓词\(T(e,x,z)\)使得对任意的部分递归函数\(f:\N\to\N\)都存在自然
数\(e\)使得\(f(x)=U(\mu zT(e,x,z))\)
\end{theorem}

\begin{corollary}[]
一个函数是递归的当且仅当它是部分递归的全函数
\end{corollary}

\begin{proof}
\(\Leftarrow\). 部分递归的全函数\(f(x)=U(\mu zT(e,x,z))\)满足正则性
\end{proof}

\begin{theorem}[通用函数定理]
存在一个通用的部分递归函数;即存在二元函数\(\Phi:\N^2\to\N\)使得对任何一元部分递归函数\(f:\N\to\N\)都存在一
个自然数\(e\)使得对所有\(x\)有\(f(x)=\Phi(e,x)\)
\end{theorem}

令\(e_0,e_1,\dots\)是图灵机的一个枚举,则\(\phi_0(x),\phi_1(x),\dots\)是对应的对全体部分递归函数的枚举,
即\(\phi_i(x)=\Phi(e_i,x)\)

\begin{theorem}[]
对递归函数来说,不存在通用函数,即不存在递归函数\(T:\N^2\to\N\)使得对任何一元递归函数\(f:\N\to\N\)都存在
一个自然数\(e\)使得对所有\(x\)有\(f(x)=T(e,x)\)
\end{theorem}

存在一个部分函数\(f\)使得对任何递归全函数\(g\),都存在\(n\in\dom(f)\)使得\(f(n)\neq g(n)\)

\(f(n)=\Phi(n,n)+1\), \(g(x)=\Phi(m,x)\),\(f(m)=\Phi(m,m)+1\neq g(x)\)
\subsection{递归可枚举}
\label{sec:org943acbf}
\begin{definition}[]
\(A\subseteq\N\) is recursively enumerable (r.e.) if \(A=\emptyset\) or \(A=\im(f)\) for some recursive \(f\)
\end{definition}

\begin{lemma}[]
\(A\subseteq\N\), TFAE
\begin{enumerate}
\item \(A\) r.e.
\item \(A=\emptyset\) or \(A=\im(f)\) for some p.r. \(f\)
\item \(A=\emptyset\) or \(A=\im(f)\) for some partial recursive \(f\)
\item \(\chi_A\) is partial recursive
\item \(A=\dom(f)\) for some partial recursive \(f\)
\item there is a recursive/primitive recursive predicate \(R(x,y)\) s.t.
\begin{equation*}
A=\{x\mid\exists yR(x,y)\}
\end{equation*}
\end{enumerate}
\end{lemma}

\begin{proof}
\(1\to 2\). Suppose \(A=\im(f)\) where \(f=U(\mu zT(e,x,z))\), for any \(a_0\in A\)
\begin{equation*}
F(x,n)=
\begin{cases}
U(\mu\le nT(e,x,n))&\exists y\le nT(e,x,y)\\
a_0
\end{cases}
\end{equation*}
Then \(F(\N^2)=f(\N)\)

\(2\to 4\). \(A=f(\N)\)
\begin{equation*}
\chi_A(y)=C_1^1(\mu xf(x)=y)
\end{equation*}

\(5\to 6\). \(f(x)=U(\mu zT(e,x,z))\)
\begin{equation*}
\dom(f)=\{x\mid\exists zT(e,x,z)\}
\end{equation*}

\(6\to 1\).
\begin{equation*}
A=\{x\mid\exists yR(x,y)\},g(y)=x\cdot C_1^1(\mu xR(x,y))
\end{equation*}
\end{proof}

\begin{theorem}[]
一个自然数的集合\(A\)是递归的当且仅当\(A\)和它的补集\(\N\setminus A\)都是递归可枚举的
\end{theorem}

\begin{proof}
设\(A\)是\(f_1:2\N\to\N\)的值域,\(\N\setminus A\)是\(f_2:2\N+1\to\N\)的值域

\(R_i(x,y)\Leftrightarrow y=f_i(x)\)
\begin{equation*}
h(y)=\mu x(R_1(x,y)\vee R_2(x,y))
\end{equation*}
\end{proof}

\begin{definition}[]
\(A,B\subseteq\N^k\) r.e., then
\begin{enumerate}
\item \(A\cup B\),\(A\cap B\) r.e.
\item \(\{x\in\N^{k-1}\mid\exists y(x,y)\in A\}\) r.e.
\end{enumerate}
\end{definition}

\begin{theorem}[]
\(K=\{e\in\N\mid\phi(e,e)\downarrow\}\) is r.e., but not recursive
\end{theorem}

\begin{proof}
\(K=\dom(\Phi(x,x))\), thus is r.e.

If \(K\) is recursive, then \(\N\setminus K\) is recursive. Thus \(x\in K\) and \(x\notin K\) are recursive
predicates. Then function
\begin{equation*}
f(x)=
\begin{cases}
\Phi(x,x)+1&x\in K\\
0
\end{cases}
\end{equation*}
is recursive. Thus there is a natural number \(e\) s.t. \(f(x)=\Phi(e,x)\). If \(e\in K\),
then \(f(e)=\Phi(e,e)+1\), a contradiction. If \(e\notin K\), then \(\Phi(e,e)\uparrow\), but \(f(e)=0\), contradiction
\end{proof}
\section{自然数的模型}
\label{sec:org48014f3}
\begin{definition}[皮亚诺公理系统]
语言\(L_{ar}=\{0,S,+,\times\}\),则皮亚诺公理系统\(\PA\)由下列公式的全称概括组成
\begin{enumerate}
\item \(Sx\neq 0\)
\item \(Sx=Sy\to x=y\)
\item \(x+0=x\)
\item \(x+Sy=S(x+y)\)
\item \(x\times Sy=x\times y+x\)
\item 对每个一阶公式\(\phi\),都有\(\phi\)的归纳公理
\begin{equation*}
(\phi(0)\wedge\forall(\phi(x)\to\phi(S(x))))\to\forall x\phi(x)
\end{equation*}
\end{enumerate}
\end{definition}
\subsection{可判定的理论}
\label{sec:orgdf4f4ba}
\begin{definition}[]
理论\(T\)可公理化如果存在一个可判定的闭语句集\(\Sigma\)使得
\begin{equation*}
T=\{\sigma\mid\Sigma\vDash\sigma\}
\end{equation*}
如果\(\Sigma\)有穷,则称\(T\)是有穷公理化的
\end{definition}

\begin{definition}[]
理论\(T\)是可判定的,如果存在一个算法,使得对任何闭语句\(\sigma\),该算法都能告诉我们\(\sigma\)是否在\(T\)
中
\end{definition}

\begin{proof}
\(T\) is decidable iff
\begin{equation*}
\# T=\{\#\sigma\mid\sigma\in T\}
\end{equation*}
is a recursive set
\end{proof}

\begin{lemma}[]
complete axiomatizable theory is decidable
\end{lemma}

\begin{proof}


A set is recursive iff itself and its complement is r.e.. \(T=\{\sigma\mid\Sigma\vDash\sigma\}=\{\sigma\mid\Sigma\vdash\sigma\}\).

\(\Sigma\) the axiom set. \(\Sigma\) is decidable, there is a recursive function \(f:\N\to\#T\), for any sentence
\(\tau\), check whether \(\#\tau\) or \(\#\tau\) is in \(f(\N)\)

\(\Sigma\)可判定,\(\chi_\Sigma\)递归
\end{proof}

\begin{theorem}[Łoś-Vaught test]
\(T\) is a theory on countable language, if
\begin{enumerate}
\item \(T\) is \(\lambda\)-categorical for some cardinal \(\lambda\)
\item \(T\) doesn't have finite model
\end{enumerate}


Then \(T\) is complete
\end{theorem}

\begin{proof}
Suppose \(T\) is not complete, then there is \(\sigma\) s.t. \(T\cup\{\sigma\}\) and \(T\cup\{\neg\sigma\}\) is consistent.

Let \(\fM_1\vDash T\cup\{\sigma\}\), \(\fM\vDash T\cup\{\not\sigma\}\). \(\fM_1\) and \(\fM_2\) are infinite

By LST, since \(T\) is at most countable, there is \(\fM_1'\) and \(\fM_2'\) of cardinality \(\lambda\) s.t.
\begin{equation*}
\fM_1'\vDash T\cup\{\sigma\},\quad\fM_2'\vDash T\cup\{\neg\sigma\}
\end{equation*}
By categoricity, \(\fM_1'\cong\fM_2'\)
\end{proof}
\subsection{只含后继的自然数模型}
\label{sec:org6252fea}
\begin{definition}[]
结构\(\fN_S=(\N,0,S)\),语言\(L_S=\{0,S\}\),公理集
\begin{enumerate}
\item \(0\neq Sx\)
\item \(Sx=Sy\to x=y\)
\item \(x\neq 0\to\exists y(x=s(y))\)
\item \(\bigwedge_{i<n}(Sx_i=x_{i+1})\to x_0\neq x_n\)
\end{enumerate}


令\(T_S\)为以上公式的全称概括的逻辑后承的集合
\end{definition}

\begin{lemma}[]
\(T_S\)是不可数范畴的理论,从而是完备的
\end{lemma}

\begin{theorem}[]
\(\Th(\fN_S)\) has quantifier elimination
\end{theorem}

\section{哥德尔不完备性定理}
\label{sec:org6b7dc5e}

\subsection{鲁宾逊算数理论Q}
\label{sec:orgdc2319d}
设\(T\)是一个包含\(Q\)的理论
\begin{definition}[]
称一个自然数上的\(k\)-元关系\(P\)在\(T\)中 \textbf{数码逐点可表示的} (简称可表示的),如果存在公式\(\rho(x)\),
称为\(P\)的一个表示公式,使得
\begin{align*}
&(n_1,\dots,n_k)\in P\Rightarrow T\vdash\rho(n_1,\dots,n_k)\\
&(n_1,\dots,n_k)\notin P\Rightarrow T\vdash\neg\rho(n_1,\dots,n_k
\end{align*}
\end{definition}

\begin{lemma}[]
如果\(T\)可公理化,则\(T\)是递归可枚举的
\end{lemma}

\begin{proof}
\(T\)可公理化\(\Leftrightarrow\)存在可判定的\(\Sigma\)使得
\begin{equation*}
T=\{\sigma\mid T\vdash\sigma\}
\end{equation*}

\(\Sigma\)可判定:\(\sharp\Sigma=\{\sharp\sigma\mid\sigma\in\Sigma\}\subseteq\N\)可判定(递归)集合

\(\Sigma\)的证明集合\(P_\Sigma\)可判定(递归):
\begin{itemize}
\item 公式序列\((\sharp\sigma_1,\dots,\sharp\sigma_n)\mapsto p\in\N\)
\item \(p\in P_\Sigma\Leftrightarrow\forall i<\ln(p)\)
\begin{itemize}
\item \(p_i\in\Sigma\cup A\)或者
\item \(\exists j,k<\ln(p)(\alpha_k:=\alpha_j\to\alpha_i)\),\(\sharp\alpha_{ijk}=p_{ijk}\)
\end{itemize}
\item \(P_\Sigma\)递归
\item \(\sigma\in T\Leftrightarrow\exists p(p\in P_\Sigma\wedge\exists i<\ln(p)(p_i=\sharp\sigma))\)
\item \(T\) (\(\sharp T\))是递归可枚举的
\item \(\sharp T\)递归函数的值域
\end{itemize}
\end{proof}

\begin{lemma}[]
\begin{enumerate}
\item 自然数上的等同关系\(\{(n,n)\mid n\in\N\}\)被公式\(x=x\)表示
\item \(\le\)关系被\(x\le y\)表示
\item 如果\(P\)是可表示的,则\(P\)是递归的
\item 可表示的关系在布尔运算下封闭
\item 如果\(P\)在\(Q\)中被\(\rho\)表示,则\(P\)在\(Q\)的任何一致扩张中都被\(\rho\)表示
\item \(P\)在\(\Th(\fN)\)中被\(\rho\)表示当且仅当\(P\)在结构\(\fN\)中被\(\rho\)表示
\end{enumerate}
\end{lemma}

\begin{proof}
\begin{enumerate}
\setcounter{enumi}{2}
\item \(P\)是可表示的使得肯定能枚举出\(\rho(n_1,\dots,n_k)\)或者\(\neg\rho(n_1,\dots,n_k)\),

对于枚举函数\(f\),\(\#\rho\in\im(f)\)或者\(\#\neg\rho\in\im(f)\),不管怎么说肯定存在一个自然数对应它们,并
且自然数是有限的
\end{enumerate}
\end{proof}

\begin{corollary}[]
\(P\)在\(Q\)中可表示,则\(P\)在\(\fN\)中可定义
\end{corollary}

\begin{proof}
\(\Th(\fN)\)是\(Q\)的一致扩张
\end{proof}

\begin{itemize}
\item 称一个\(L_{ar}\)公式是\(\Delta_0\)的,如果它只包含有界量词
\item 如果一个公式\(\phi\)(\(Q\)下)等价于\(\exists x_1\dots\exists x_n\theta\),其中\(\theta\)是\(\Delta_0\)的,则称\(\phi\)是\(\Sigma_1\)的公式
\item 如果一个公式\(\phi\)等价于\(\forall x_1\dots\forall x_n\theta\),其中\(\theta\)是\(\Delta_0\)的,则\(\phi\)是\(\Pi_1\)公式
\item 如果一个公式既等价于\(\Sigma_1\),又等价于\(\Pi_1\),则它是\(\Delta_1\)的
\end{itemize}


\begin{theorem}[\(\Sigma_1\)-完备性]
对任何一个\(\Sigma_1\)-闭语句,我们有
\begin{equation*}
\fN\vDash\tau\Leftrightarrow Q\vdash\tau
\end{equation*}
\end{theorem}

\begin{proof}
\(\Rightarrow\): 对任何\(\Delta_0\)-闭语句\(\sigma\),对任何\(\fM\vDash Q\),有
\begin{equation*}
\fM\vDash\sigma\Leftrightarrow\fN\vDash\sigma
\end{equation*}

设\(\sigma\)为\((\forall x\le t)\psi\)且\(\psi\)是一个\(\Delta_0\)公式,\(t\)是一个闭项,于是存在\(n\in\N\)使得\(Q\vdash t=n\)。
如果\(\fM\vDash(\forall x\le t)\psi(x,t)\),则\(\fM\vDash(\forall x\le n)\psi(x,n)\)

若\(\sigma:=\exists\barx\psi(\barx)\),\(\psi\)是\(\Delta_0\)公式,设\(\fN\vDash\sigma\),则存在\(\barm\in\N^n\)使得\(\fN\vDash\psi(\barm)\)。
\(Q\vdash\psi(\barm)\Rightarrow Q\vdash\exists\barx\psi(\barx)\)
\end{proof}

\begin{definition}[]
称一个函数\(f:\N^k\to\N\)在\(T\)中可表示,如果存在\(L_{ar}\)公式\(\phi(x_1,\dots,x_k,y)\)使得对所
有\((n_1,\dots,n_k)\in\N^k\)有
\begin{equation*}
T\vdash\forall y\left( \phi(n_1,\dots,n_k,y)\leftrightarrow y=f(n_1,\dots,n_k) \right)
\end{equation*}
此时称\(\phi\)作为一个函数表示\(f\)
\end{definition}

表示函数图象的公式不能表示函数

设\(f\)是一个函数,\(G_f=\{(\barx,y)\mid y=f(\barx)\}\),设\(\phi(\barx,y)\)表示\(G_f\),于是对任
意\(\bara\in\N^k\),\(b\in\N\),有
\begin{equation*}
f(a)=b\Rightarrow T\vdash\phi(\bara,b),\quad f(a)\neq b\Rightarrow T\vdash\neg\phi(\bara,b)
\end{equation*}
若\(\fM\vDash T\),对于非标准\(y\in M\setminus\N\),
\begin{equation*}
\fM\vDash y\neq f(\bara)\to\neg\phi(\bara,y)
\end{equation*}
不一定成立

例如\(Z(x)=0\),于是\(\phi(x,y):=y+y=y\)表示\(G_Z\),\(Q\)不能证明\(y+y=y\to y=0\),考虑\(\N\cup\{\infty\}\)

\begin{lemma}[]
令\(t\)为\(L_{ar}\)的项,则\(t\)诱导出来的函数是可表示的
\end{lemma}

\begin{theorem}[]
可表示函数类关于复合封闭
\end{theorem}

\begin{lemma}[]
可表示函数类关于极小算子封闭
\begin{enumerate}
\item 设\(P\subseteq\N^{k+1}\)被\(\alpha(\barx,y)\)表示
\item 令\(\phi(\barx,y)\)味\(\alpha(\barx,y)\wedge(\forall z<y)\neg\alpha(\barx,z)\)
\item 则\(f:\bara\mapsto\mu b[P(\bara,b)]\)被\(\phi(\barx,y)\)表示
\end{enumerate}
\end{lemma}

\begin{corollary}[]
函数\(f\)可表示当且仅当\(G_f\)可表示
\end{corollary}

\begin{proof}
\(a\mapsto\mu b[G_f(a,b)]\)是可表示函数,它是\(f\)自身
\end{proof}

\begin{corollary}[]
加定函数\(g(x,y)\)可表示,则函数
\begin{equation*}
f(x):=\mu y(g(x,y)=0)
\end{equation*}
也可表示
\end{corollary}

目标:可表示函数类关于原始递归封闭,


\begin{definition}[]
哥德尔函数\(\beta:\N^3\to\N^{<\omega}\)定义为:对任意\(u,v,w\),\(\beta(u,v,w)\)是一个长度为\(w\)的序
列\(a_0,\dots,a_{w-1}\),其中
\begin{equation*}
u=d((i+1)v+1)+a_i
\end{equation*}
\end{definition}

定义\(\alpha(u,v,i)\)为\(\frac{u}{v(i+1)+1}\)的余数,即\(\beta\)函数的坐标分量函数,则\(\alpha(u,v,i)\)是可表
示的

\begin{theorem}[]
\(\beta\)是满射
\end{theorem}

\begin{lemma}[欧几里得引理]
设\(a,b\in\N\)互素,则存在\(x,y\in\Z\)使得\(ax+by=1\)
\end{lemma}

\begin{proof}
令\(X=\{ax+by\mid x,y\in\Z\}\cap\N\),则\(X\)有最小元\(x_0\),若\(x_0\)不能整除\(a\),则\(a=cx_0+r\),因
此\(x_0\)是最小公倍数

consider \(a\N\) and \(b\N\), then \(a\N\cap b\N=c\N\) for some \(c\in\N\) since \(\N\) is a PID
\end{proof}

\begin{theorem}[中国剩余定理]
设\(d_0,\dots,d_n\)是两两互素的自然数,\(a_0,\dots,a_n\)为满足\(a_i<d_i\)的自然数,则存在\(c\in\N\)使
得\(c\equiv a_i\mod d_i\) for all \(i\)
\end{theorem}

\begin{lemma}[]
对任意\(n\),存在\(n+1\)个数两两互素
\begin{equation*}
1+n!,1+2\cdot n!,\dots,1+(n+1)\cdot n!
\end{equation*}
\end{lemma}

\begin{theorem}[]
\(\beta\)是满射
\end{theorem}

\begin{proof}
设\(a_0,\dots,a_{w-1}\)是一个自然数的序列,令\(n=\max\{a_0,\dots,a_{w-1},w\}\),令\(v=n!\),
则\(\{v(i+1)+1\mid i=0,\dots,w-1\}\)两两互素且\(a_i<v(i+1)+1\),根据中国剩余定理,存在\(u\in\N\)使得
\(u\equiv_{v(i+1)+1}a_i\)
\end{proof}

\begin{lemma}[]
\(\alpha(u,v,i)=\frac{u}{v(i+1)+1}\)的余数是可表示的
\end{lemma}

\begin{proof}
\(P(c,d,i,r,q):=(c=q(1+(i+1)d)+r)\)可表示

\(R(c,d,i,r):=\exists q\le c\;P(c,d,i,r,q)\)可表示:

\(\mu r(R(c,d,i,r))\)可表示
\end{proof}

\begin{theorem}[]
递归函数都是可表示的
\end{theorem}

\begin{proof}
只需证明\(\fN\)中的可表示函数类包含初始函数,且对复合、极小化、原始递归封闭
\end{proof}



\section{作业}
\label{sec:org2020d74}
7.5.3 (2)/

7.5.5 \(h[A]\)递归吗

7.1 7.2 7.3

\begin{exercise}
\label{ex8.1.1}
令\(\Sigma_1\)和\(\Sigma_2\)两个语句集,并且没有模型能同时满足\(\Sigma_1\)和\(\Sigma_2\).证明存在一个语句\(\sigma\)使
得\(\Mod\Sigma_1\subseteq\Mod\tau\)并且\(\Mod\Sigma_2\subseteq\Mod\neg\tau\)
\end{exercise}

\begin{proof}
\(\Mod\Sigma_1\cap\Mod\Sigma_2=\emptyset\). \(\Mod\Sigma_1\subseteq\Mod\tau\Leftrightarrow\Sigma_1\vDash\tau\)

suppose for all \(\tau\), \(\Sigma_1\not\vDash\tau\) or \(\Sigma_2\not\vDash\tau\)

Then for all \(\tau\in\Sigma_1\), \(\Sigma_2\not\vDash\neg\tau\) and hence \(\Sigma_2\cup\{\tau\}\) is satisfiable. Thus \(\Sigma_1\cup\Sigma_2\) is
satisfiable, a contradiction
\end{proof}

\begin{exercise}
\label{8.1.2}
\(\vdash_{\PA}x<y\leftrightarrow Sx\le y\) and \(\vdash_{\PA}x\le y\vee y\le x\)
\end{exercise}

\begin{proof}
\(x<y\Leftrightarrow\exists z(\neg z\approx 0\wedge x+z=y)\).
\(\neg z\approx 0\Leftrightarrow \exists m(z\approx S(m))\).
\(x+S(m)\approx S(x+m)\approx S(x)+m\approx y\).
\end{proof}

\begin{exercise}
\label{8.2.3}
证明有端点的稠密线序理论\(\Th(\Q\cap[0,1),<)\),\(\Th(\Q\cap[0,1],<)\),\(\Th(\Q\cap(0,1],<)\)都分别
是\(\aleph_0\)-categorical,因而是完全的。再验证它们和\(\Th(\Q,<)\)是稠密线序理论仅有的四个完全扩张
\end{exercise}

\begin{exercise}
\label{8.2.4}
\(\ACF_0\) is not finitely axiomatizable
\end{exercise}

\begin{proof}
\href{https://math.stackexchange.com/questions/3453682/fields-with-characteristic-0-are-not-finitely-axiomatizable-in-fopc}{proof}
\end{proof}

\begin{exercise}
\label{8.3.2}
证明:理论\(T_S\)被下列公理公理化:(S1) (S2)加上对语言\(\call_S=\{0,S\}\)的归纳公理模式
\begin{equation*}
[\varphi(0)\wedge\forall x(\varphi(x)\to\varphi(Sx))]\to\forall x\varphi(x)
\end{equation*}
其中\(\varphi\)是任意的语言\(\call_S\)上的公式
\end{exercise}

\begin{exercise}
\label{8.3.3}
\(T_S\)不能被有穷公理化
\end{exercise}

\begin{proof}
如果\(T_S\)能被有穷公理化
\end{proof}
\end{document}
