% Created 2021-12-31 Fri 14:47
% Intended LaTeX compiler: pdflatex
\documentclass[11pt]{article}
\usepackage[utf8]{inputenc}
\usepackage[T1]{fontenc}
\usepackage{graphicx}
\usepackage{longtable}
\usepackage{wrapfig}
\usepackage{rotating}
\usepackage[normalem]{ulem}
\usepackage{amsmath}
\usepackage{amssymb}
\usepackage{capt-of}
\usepackage{hyperref}
\graphicspath{{../../books/}}
% wrong resolution of image
% https://tex.stackexchange.com/questions/21627/image-from-includegraphics-showing-in-wrong-image-size?rq=1

%%%%%%%%%%%%%%%%%%%%%%%%%%%%%%%%%%%%%%
%% TIPS                                 %%
%%%%%%%%%%%%%%%%%%%%%%%%%%%%%%%%%%%%%%
% \substack{a\\b} for multiple lines text
% \usepackage{expl3}
% \expandafter\def\csname ver@l3regex.sty\endcsname{}
% \usepackage{pkgloader}
\usepackage[utf8]{inputenc}

% nfss error
% \usepackage[B1,T1]{fontenc}
\usepackage{fontspec}

% \usepackage[Emoticons]{ucharclasses}
\newfontfamily\DejaSans{DejaVu Sans}
% \setDefaultTransitions{\DejaSans}{}

% pdfplots will load xolor automatically without option
\usepackage[dvipsnames]{xcolor}

%                                                             ┳┳┓   ┓
%                                                             ┃┃┃┏┓╋┣┓
%                                                             ┛ ┗┗┻┗┛┗
% \usepackage{amsmath} mathtools loads the amsmath
\usepackage{amsmath}
\usepackage{mathtools}

\usepackage{amsthm}
\usepackage{amsbsy}

%\usepackage{commath}

\usepackage{amssymb}

\usepackage{mathrsfs}
%\usepackage{mathabx}
\usepackage{stmaryrd}
\usepackage{empheq}

\usepackage{scalerel}
\usepackage{stackengine}
\usepackage{stackrel}



\usepackage{nicematrix}
\usepackage{tensor}
\usepackage{blkarray}
\usepackage{siunitx}
\usepackage[f]{esvect}

% centering \not on a letter
\usepackage{slashed}
\usepackage[makeroom]{cancel}

%\usepackage{merriweather}
\usepackage{unicode-math}
\setmainfont{TeX Gyre Pagella}
% \setmathfont{STIX}
%\setmathfont{texgyrepagella-math.otf}
%\setmathfont{Libertinus Math}
\setmathfont{Latin Modern Math}

 % \setmathfont[range={\smwhtdiamond,\enclosediamond,\varlrtriangle}]{Latin Modern Math}
\setmathfont[range={\rightrightarrows,\twoheadrightarrow,\leftrightsquigarrow,\triangledown,\vartriangle,\precneq,\succneq,\prec,\succ,\preceq,\succeq,\tieconcat}]{XITS Math}
 \setmathfont[range={\int,\setminus}]{Libertinus Math}
 % \setmathfont[range={\mathalpha}]{TeX Gyre Pagella Math}
%\setmathfont[range={\mitA,\mitB,\mitC,\mitD,\mitE,\mitF,\mitG,\mitH,\mitI,\mitJ,\mitK,\mitL,\mitM,\mitN,\mitO,\mitP,\mitQ,\mitR,\mitS,\mitT,\mitU,\mitV,\mitW,\mitX,\mitY,\mitZ,\mita,\mitb,\mitc,\mitd,\mite,\mitf,\mitg,\miti,\mitj,\mitk,\mitl,\mitm,\mitn,\mito,\mitp,\mitq,\mitr,\mits,\mitt,\mitu,\mitv,\mitw,\mitx,\mity,\mitz}]{TeX Gyre Pagella Math}
% unicode is not good at this!
%\let\nmodels\nvDash

 \usepackage{wasysym}

 % for wide hat
 \DeclareSymbolFont{yhlargesymbols}{OMX}{yhex}{m}{n} \DeclareMathAccent{\what}{\mathord}{yhlargesymbols}{"62}

%                                                               ┏┳┓•┓
%                                                                ┃ ┓┃┏┓
%                                                                ┻ ┗┛┗┗

\usepackage{pgfplots}
\pgfplotsset{compat=1.18}
\usepackage{tikz}
\usepackage{tikz-cd}
\tikzcdset{scale cd/.style={every label/.append style={scale=#1},
    cells={nodes={scale=#1}}}}
% TODO: discard qtree and use forest
% \usepackage{tikz-qtree}
\usepackage{forest}

\usetikzlibrary{arrows,positioning,calc,fadings,decorations,matrix,decorations,shapes.misc}
%setting from geogebra
\definecolor{ccqqqq}{rgb}{0.8,0,0}

%                                                          ┳┳┓•    ┓┓
%                                                          ┃┃┃┓┏┏┏┓┃┃┏┓┏┓┏┓┏┓┓┏┏
%                                                          ┛ ┗┗┛┗┗ ┗┗┗┻┛┗┗ ┗┛┗┻┛
%\usepackage{twemojis}
\usepackage[most]{tcolorbox}
\usepackage{threeparttable}
\usepackage{tabularx}

\usepackage{enumitem}
\usepackage[indLines=false]{algpseudocodex}
\usepackage[]{algorithm2e}
% \SetKwComment{Comment}{/* }{ */}
% \algrenewcommand\algorithmicrequire{\textbf{Input:}}
% \algrenewcommand\algorithmicensure{\textbf{Output:}}
% wrong with preview
\usepackage{subcaption}
\usepackage{caption}
% {\aunclfamily\Huge}
\usepackage{auncial}

\usepackage{float}

\usepackage{fancyhdr}

\usepackage{ifthen}
\usepackage{xargs}

\definecolor{mintedbg}{rgb}{0.99,0.99,0.99}
\usepackage[cachedir=\detokenize{~/miscellaneous/trash}]{minted}
\setminted{breaklines,
  mathescape,
  bgcolor=mintedbg,
  fontsize=\footnotesize,
  frame=single,
  linenos}
\usemintedstyle{xcode}
\usepackage{tcolorbox}
\usepackage{etoolbox}



\usepackage{imakeidx}
\usepackage{hyperref}
\usepackage{soul}
\usepackage{framed}

% don't use this for preview
%\usepackage[margin=1.5in]{geometry}
% \usepackage{geometry}
% \geometry{legalpaper, landscape, margin=1in}
\usepackage[font=itshape]{quoting}

%\LoadPackagesNow
%\usepackage[xetex]{preview}
%%%%%%%%%%%%%%%%%%%%%%%%%%%%%%%%%%%%%%%
%% USEPACKAGES end                       %%
%%%%%%%%%%%%%%%%%%%%%%%%%%%%%%%%%%%%%%%

%%%%%%%%%%%%%%%%%%%%%%%%%%%%%%%%%%%%%%%
%% Algorithm environment
%%%%%%%%%%%%%%%%%%%%%%%%%%%%%%%%%%%%%%%
\SetKwIF{Recv}{}{}{upon receiving}{do}{}{}{}
\SetKwBlock{Init}{initially do}{}
\SetKwProg{Function}{Function}{:}{}

% https://github.com/chrmatt/algpseudocodex/issues/3
\algnewcommand\algorithmicswitch{\textbf{switch}}%
\algnewcommand\algorithmiccase{\textbf{case}}
\algnewcommand\algorithmicof{\textbf{of}}
\algnewcommand\algorithmicotherwise{\texttt{otherwise} $\Rightarrow$}

\makeatletter
\algdef{SE}[SWITCH]{Switch}{EndSwitch}[1]{\algpx@startIndent\algpx@startCodeCommand\algorithmicswitch\ #1\ \algorithmicdo}{\algpx@endIndent\algpx@startCodeCommand\algorithmicend\ \algorithmicswitch}%
\algdef{SE}[CASE]{Case}{EndCase}[1]{\algpx@startIndent\algpx@startCodeCommand\algorithmiccase\ #1}{\algpx@endIndent\algpx@startCodeCommand\algorithmicend\ \algorithmiccase}%
\algdef{SE}[CASEOF]{CaseOf}{EndCaseOf}[1]{\algpx@startIndent\algpx@startCodeCommand\algorithmiccase\ #1 \algorithmicof}{\algpx@endIndent\algpx@startCodeCommand\algorithmicend\ \algorithmiccase}
\algdef{SE}[OTHERWISE]{Otherwise}{EndOtherwise}[0]{\algpx@startIndent\algpx@startCodeCommand\algorithmicotherwise}{\algpx@endIndent\algpx@startCodeCommand\algorithmicend\ \algorithmicotherwise}
\ifbool{algpx@noEnd}{%
  \algtext*{EndSwitch}%
  \algtext*{EndCase}%
  \algtext*{EndCaseOf}
  \algtext*{EndOtherwise}
  %
  % end indent line after (not before), to get correct y position for multiline text in last command
  \apptocmd{\EndSwitch}{\algpx@endIndent}{}{}%
  \apptocmd{\EndCase}{\algpx@endIndent}{}{}%
  \apptocmd{\EndCaseOf}{\algpx@endIndent}{}{}
  \apptocmd{\EndOtherwise}{\algpx@endIndent}{}{}
}{}%

\pretocmd{\Switch}{\algpx@endCodeCommand}{}{}
\pretocmd{\Case}{\algpx@endCodeCommand}{}{}
\pretocmd{\CaseOf}{\algpx@endCodeCommand}{}{}
\pretocmd{\Otherwise}{\algpx@endCodeCommand}{}{}

% for end commands that may not be printed, tell endCodeCommand whether we are using noEnd
\ifbool{algpx@noEnd}{%
  \pretocmd{\EndSwitch}{\algpx@endCodeCommand[1]}{}{}%
  \pretocmd{\EndCase}{\algpx@endCodeCommand[1]}{}{}
  \pretocmd{\EndCaseOf}{\algpx@endCodeCommand[1]}{}{}%
  \pretocmd{\EndOtherwise}{\algpx@endCodeCommand[1]}{}{}
}{%
  \pretocmd{\EndSwitch}{\algpx@endCodeCommand[0]}{}{}%
  \pretocmd{\EndCase}{\algpx@endCodeCommand[0]}{}{}%
  \pretocmd{\EndCaseOf}{\algpx@endCodeCommand[0]}{}{}
  \pretocmd{\EndOtherwise}{\algpx@endCodeCommand[0]}{}{}
}%
\makeatother
% % For algpseudocode
% \algnewcommand\algorithmicswitch{\textbf{switch}}
% \algnewcommand\algorithmiccase{\textbf{case}}
% \algnewcommand\algorithmiccaseof{\textbf{case}}
% \algnewcommand\algorithmicof{\textbf{of}}
% % New "environments"
% \algdef{SE}[SWITCH]{Switch}{EndSwitch}[1]{\algorithmicswitch\ #1\ \algorithmicdo}{\algorithmicend\ \algorithmicswitch}%
% \algdef{SE}[CASE]{Case}{EndCase}[1]{\algorithmiccase\ #1}{\algorithmicend\ \algorithmiccase}%
% \algtext*{EndSwitch}%
% \algtext*{EndCase}
% \algdef{SE}[CASEOF]{CaseOf}{EndCaseOf}[1]{\algorithmiccaseof\ #1 \algorithmicof}{\algorithmicend\ \algorithmiccaseof}
% \algtext*{EndCaseOf}



%\pdfcompresslevel0

% quoting from
% https://tex.stackexchange.com/questions/391726/the-quotation-environment
\NewDocumentCommand{\bywhom}{m}{% the Bourbaki trick
  {\nobreak\hfill\penalty50\hskip1em\null\nobreak
   \hfill\mbox{\normalfont(#1)}%
   \parfillskip=0pt \finalhyphendemerits=0 \par}%
}

\NewDocumentEnvironment{pquotation}{m}
  {\begin{quoting}[
     indentfirst=true,
     leftmargin=\parindent,
     rightmargin=\parindent]\itshape}
  {\bywhom{#1}\end{quoting}}

\indexsetup{othercode=\small}
\makeindex[columns=2,options={-s /media/wu/file/stuuudy/notes/index_style.ist},intoc]
\makeatletter
\def\@idxitem{\par\hangindent 0pt}
\makeatother


% \newcounter{dummy} \numberwithin{dummy}{section}
\newtheorem{dummy}{dummy}[section]
\theoremstyle{definition}
\newtheorem{definition}[dummy]{Definition}
\theoremstyle{plain}
\newtheorem{corollary}[dummy]{Corollary}
\newtheorem{lemma}[dummy]{Lemma}
\newtheorem{proposition}[dummy]{Proposition}
\newtheorem{theorem}[dummy]{Theorem}
\newtheorem{notation}[dummy]{Notation}
\newtheorem{conjecture}[dummy]{Conjecture}
\newtheorem{fact}[dummy]{Fact}
\newtheorem{warning}[dummy]{Warning}
\theoremstyle{definition}
\newtheorem{examplle}{Example}[section]
\theoremstyle{remark}
\newtheorem*{remark}{Remark}
\newtheorem{exercise}{Exercise}[subsection]
\newtheorem{problem}{Problem}[subsection]
\newtheorem{observation}{Observation}[section]
\newenvironment{claim}[1]{\par\noindent\textbf{Claim:}\space#1}{}

\makeatletter
\DeclareFontFamily{U}{tipa}{}
\DeclareFontShape{U}{tipa}{m}{n}{<->tipa10}{}
\newcommand{\arc@char}{{\usefont{U}{tipa}{m}{n}\symbol{62}}}%

\newcommand{\arc}[1]{\mathpalette\arc@arc{#1}}

\newcommand{\arc@arc}[2]{%
  \sbox0{$\m@th#1#2$}%
  \vbox{
    \hbox{\resizebox{\wd0}{\height}{\arc@char}}
    \nointerlineskip
    \box0
  }%
}
\makeatother

\setcounter{MaxMatrixCols}{20}
%%%%%%% ABS
\DeclarePairedDelimiter\abss{\lvert}{\rvert}%
\DeclarePairedDelimiter\normm{\lVert}{\rVert}%

% Swap the definition of \abs* and \norm*, so that \abs
% and \norm resizes the size of the brackets, and the
% starred version does not.
\makeatletter
\let\oldabs\abss
%\def\abs{\@ifstar{\oldabs}{\oldabs*}}
\newcommand{\abs}{\@ifstar{\oldabs}{\oldabs*}}
\newcommand{\norm}[1]{\left\lVert#1\right\rVert}
%\let\oldnorm\normm
%\def\norm{\@ifstar{\oldnorm}{\oldnorm*}}
%\renewcommand{norm}{\@ifstar{\oldnorm}{\oldnorm*}}
\makeatother

% \stackMath
% \newcommand\what[1]{%
% \savestack{\tmpbox}{\stretchto{%
%   \scaleto{%
%     \scalerel*[\widthof{\ensuremath{#1}}]{\kern-.6pt\bigwedge\kern-.6pt}%
%     {\rule[-\textheight/2]{1ex}{\textheight}}%WIDTH-LIMITED BIG WEDGE
%   }{\textheight}%
% }{0.5ex}}%
% \stackon[1pt]{#1}{\tmpbox}%
% }

% \newcommand\what[1]{\ThisStyle{%
%     \setbox0=\hbox{$\SavedStyle#1$}%
%     \stackengine{-1.0\ht0+.5pt}{$\SavedStyle#1$}{%
%       \stretchto{\scaleto{\SavedStyle\mkern.15mu\char'136}{2.6\wd0}}{1.4\ht0}%
%     }{O}{c}{F}{T}{S}%
%   }
% }

% \newcommand\wtilde[1]{\ThisStyle{%
%     \setbox0=\hbox{$\SavedStyle#1$}%
%     \stackengine{-.1\LMpt}{$\SavedStyle#1$}{%
%       \stretchto{\scaleto{\SavedStyle\mkern.2mu\AC}{.5150\wd0}}{.6\ht0}%
%     }{O}{c}{F}{T}{S}%
%   }
% }

% \newcommand\wbar[1]{\ThisStyle{%
%     \setbox0=\hbox{$\SavedStyle#1$}%
%     \stackengine{.5pt+\LMpt}{$\SavedStyle#1$}{%
%       \rule{\wd0}{\dimexpr.3\LMpt+.3pt}%
%     }{O}{c}{F}{T}{S}%
%   }
% }

\newcommand{\bl}[1] {\boldsymbol{#1}}
\newcommand{\Wt}[1] {\stackrel{\sim}{\smash{#1}\rule{0pt}{1.1ex}}}
\newcommand{\wt}[1] {\widetilde{#1}}
\newcommand{\tf}[1] {\textbf{#1}}

\newcommand{\wu}[1]{{\color{red} #1}}

%For boxed texts in align, use Aboxed{}
%otherwise use boxed{}

\DeclareMathSymbol{\widehatsym}{\mathord}{largesymbols}{"62}
\newcommand\lowerwidehatsym{%
  \text{\smash{\raisebox{-1.3ex}{%
    $\widehatsym$}}}}
\newcommand\fixwidehat[1]{%
  \mathchoice
    {\accentset{\displaystyle\lowerwidehatsym}{#1}}
    {\accentset{\textstyle\lowerwidehatsym}{#1}}
    {\accentset{\scriptstyle\lowerwidehatsym}{#1}}
    {\accentset{\scriptscriptstyle\lowerwidehatsym}{#1}}
  }


\newcommand{\cupdot}{\mathbin{\dot{\cup}}}
\newcommand{\bigcupdot}{\mathop{\dot{\bigcup}}}

\usepackage{graphicx}

\usepackage[toc,page]{appendix}

% text on arrow for xRightarrow
\makeatletter
%\newcommand{\xRightarrow}[2][]{\ext@arrow 0359\Rightarrowfill@{#1}{#2}}
\makeatother

% Arbitrary long arrow
\newcommand{\Rarrow}[1]{%
\parbox{#1}{\tikz{\draw[->](0,0)--(#1,0);}}
}

\newcommand{\LRarrow}[1]{%
\parbox{#1}{\tikz{\draw[<->](0,0)--(#1,0);}}
}


\makeatletter
\providecommand*{\rmodels}{%
  \mathrel{%
    \mathpalette\@rmodels\models
  }%
}
\newcommand*{\@rmodels}[2]{%
  \reflectbox{$\m@th#1#2$}%
}
\makeatother

% Roman numerals
\makeatletter
\newcommand*{\rom}[1]{\expandafter\@slowromancap\romannumeral #1@}
\makeatother
% \\def \\b\([a-zA-Z]\) {\\boldsymbol{[a-zA-z]}}
% \\DeclareMathOperator{\\b\1}{\\textbf{\1}}

\DeclareMathOperator*{\argmin}{arg\,min}
\DeclareMathOperator*{\argmax}{arg\,max}

\DeclareMathOperator{\bone}{\textbf{1}}
\DeclareMathOperator{\bx}{\textbf{x}}
\DeclareMathOperator{\bz}{\textbf{z}}
\DeclareMathOperator{\bff}{\textbf{f}}
\DeclareMathOperator{\ba}{\textbf{a}}
\DeclareMathOperator{\bk}{\textbf{k}}
\DeclareMathOperator{\bs}{\textbf{s}}
\DeclareMathOperator{\bh}{\textbf{h}}
\DeclareMathOperator{\bc}{\textbf{c}}
\DeclareMathOperator{\br}{\textbf{r}}
\DeclareMathOperator{\bi}{\textbf{i}}
\DeclareMathOperator{\bj}{\textbf{j}}
\DeclareMathOperator{\bn}{\textbf{n}}
\DeclareMathOperator{\be}{\textbf{e}}
\DeclareMathOperator{\bo}{\textbf{o}}
\DeclareMathOperator{\bU}{\textbf{U}}
\DeclareMathOperator{\bL}{\textbf{L}}
\DeclareMathOperator{\bV}{\textbf{V}}
\def \bzero {\mathbf{0}}
\def \bbone {\mathbb{1}}
\def \btwo {\mathbf{2}}
\DeclareMathOperator{\bv}{\textbf{v}}
\DeclareMathOperator{\bp}{\textbf{p}}
\DeclareMathOperator{\bI}{\textbf{I}}
\def \dbI {\dot{\bI}}
\DeclareMathOperator{\bM}{\textbf{M}}
\DeclareMathOperator{\bN}{\textbf{N}}
\DeclareMathOperator{\bK}{\textbf{K}}
\DeclareMathOperator{\bt}{\textbf{t}}
\DeclareMathOperator{\bb}{\textbf{b}}
\DeclareMathOperator{\bA}{\textbf{A}}
\DeclareMathOperator{\bX}{\textbf{X}}
\DeclareMathOperator{\bu}{\textbf{u}}
\DeclareMathOperator{\bS}{\textbf{S}}
\DeclareMathOperator{\bZ}{\textbf{Z}}
\DeclareMathOperator{\bJ}{\textbf{J}}
\DeclareMathOperator{\by}{\textbf{y}}
\DeclareMathOperator{\bw}{\textbf{w}}
\DeclareMathOperator{\bT}{\textbf{T}}
\DeclareMathOperator{\bF}{\textbf{F}}
\DeclareMathOperator{\bmm}{\textbf{m}}
\DeclareMathOperator{\bW}{\textbf{W}}
\DeclareMathOperator{\bR}{\textbf{R}}
\DeclareMathOperator{\bC}{\textbf{C}}
\DeclareMathOperator{\bD}{\textbf{D}}
\DeclareMathOperator{\bE}{\textbf{E}}
\DeclareMathOperator{\bQ}{\textbf{Q}}
\DeclareMathOperator{\bP}{\textbf{P}}
\DeclareMathOperator{\bY}{\textbf{Y}}
\DeclareMathOperator{\bH}{\textbf{H}}
\DeclareMathOperator{\bB}{\textbf{B}}
\DeclareMathOperator{\bG}{\textbf{G}}
\def \blambda {\symbf{\lambda}}
\def \boldeta {\symbf{\eta}}
\def \balpha {\symbf{\alpha}}
\def \btau {\symbf{\tau}}
\def \bbeta {\symbf{\beta}}
\def \bgamma {\symbf{\gamma}}
\def \bxi {\symbf{\xi}}
\def \bLambda {\symbf{\Lambda}}
\def \bGamma {\symbf{\Gamma}}

\newcommand{\bto}{{\boldsymbol{\to}}}
\newcommand{\Ra}{\Rightarrow}
\newcommand{\xrsa}[1]{\overset{#1}{\rightsquigarrow}}
\newcommand{\xlsa}[1]{\overset{#1}{\leftsquigarrow}}
\newcommand\und[1]{\underline{#1}}
\newcommand\ove[1]{\overline{#1}}
%\def \concat {\verb|^|}
\def \bPhi {\mbfPhi}
\def \btheta {\mbftheta}
\def \bTheta {\mbfTheta}
\def \bmu {\mbfmu}
\def \bphi {\mbfphi}
\def \bSigma {\mbfSigma}
\def \la {\langle}
\def \ra {\rangle}

\def \caln {\mathcal{N}}
\def \dissum {\displaystyle\Sigma}
\def \dispro {\displaystyle\prod}

\def \caret {\verb!^!}

\def \A {\mathbb{A}}
\def \B {\mathbb{B}}
\def \C {\mathbb{C}}
\def \D {\mathbb{D}}
\def \E {\mathbb{E}}
\def \F {\mathbb{F}}
\def \G {\mathbb{G}}
\def \H {\mathbb{H}}
\def \I {\mathbb{I}}
\def \J {\mathbb{J}}
\def \K {\mathbb{K}}
\def \L {\mathbb{L}}
\def \M {\mathbb{M}}
\def \N {\mathbb{N}}
\def \O {\mathbb{O}}
\def \P {\mathbb{P}}
\def \Q {\mathbb{Q}}
\def \R {\mathbb{R}}
\def \S {\mathbb{S}}
\def \T {\mathbb{T}}
\def \U {\mathbb{U}}
\def \V {\mathbb{V}}
\def \W {\mathbb{W}}
\def \X {\mathbb{X}}
\def \Y {\mathbb{Y}}
\def \Z {\mathbb{Z}}

\def \cala {\mathcal{A}}
\def \cale {\mathcal{E}}
\def \calb {\mathcal{B}}
\def \calq {\mathcal{Q}}
\def \calp {\mathcal{P}}
\def \cals {\mathcal{S}}
\def \calx {\mathcal{X}}
\def \caly {\mathcal{Y}}
\def \calg {\mathcal{G}}
\def \cald {\mathcal{D}}
\def \caln {\mathcal{N}}
\def \calr {\mathcal{R}}
\def \calt {\mathcal{T}}
\def \calm {\mathcal{M}}
\def \calw {\mathcal{W}}
\def \calc {\mathcal{C}}
\def \calv {\mathcal{V}}
\def \calf {\mathcal{F}}
\def \calk {\mathcal{K}}
\def \call {\mathcal{L}}
\def \calu {\mathcal{U}}
\def \calo {\mathcal{O}}
\def \calh {\mathcal{H}}
\def \cali {\mathcal{I}}
\def \calj {\mathcal{J}}

\def \bcup {\bigcup}

% set theory

\def \zfcc {\textbf{ZFC}^-}
\def \BGC {\textbf{BGC}}
\def \BG {\textbf{BG}}
\def \ac  {\textbf{AC}}
\def \gl  {\textbf{L }}
\def \gll {\textbf{L}}
\newcommand{\zfm}{$\textbf{ZF}^-$}

\def \ZFm {\text{ZF}^-}
\def \ZFCm {\text{ZFC}^-}
\DeclareMathOperator{\WF}{WF}
\DeclareMathOperator{\On}{On}
\def \on {\textbf{On }}
\def \cm {\textbf{M }}
\def \cn {\textbf{N }}
\def \cv {\textbf{V }}
\def \zc {\textbf{ZC }}
\def \zcm {\textbf{ZC}}
\def \zff {\textbf{ZF}}
\def \wfm {\textbf{WF}}
\def \onm {\textbf{On}}
\def \cmm {\textbf{M}}
\def \cnm {\textbf{N}}
\def \cvm {\textbf{V}}

\renewcommand{\restriction}{\mathord{\upharpoonright}}
%% another restriction
\newcommand\restr[2]{{% we make the whole thing an ordinary symbol
  \left.\kern-\nulldelimiterspace % automatically resize the bar with \right
  #1 % the function
  \vphantom{\big|} % pretend it's a little taller at normal size
  \right|_{#2} % this is the delimiter
  }}

\def \pred {\text{pred}}

\def \rank {\text{rank}}
\def \Con {\text{Con}}
\def \deff {\text{Def}}


\def \uin {\underline{\in}}
\def \oin {\overline{\in}}
\def \uR {\underline{R}}
\def \oR {\overline{R}}
\def \uP {\underline{P}}
\def \oP {\overline{P}}

\def \dsum {\displaystyle\sum}

\def \Ra {\Rightarrow}

\def \e {\enspace}

\def \sgn {\operatorname{sgn}}
\def \gen {\operatorname{gen}}
\def \Hom {\operatorname{Hom}}
\def \hom {\operatorname{hom}}
\def \Sub {\operatorname{Sub}}

\def \supp {\operatorname{supp}}

\def \epiarrow {\twoheadarrow}
\def \monoarrow {\rightarrowtail}
\def \rrarrow {\rightrightarrows}

% \def \minus {\text{-}}
% \newcommand{\minus}{\scalebox{0.75}[1.0]{$-$}}
% \DeclareUnicodeCharacter{002D}{\minus}


\def \tril {\triangleleft}

\def \ISigma {\text{I}\Sigma}
\def \IDelta {\text{I}\Delta}
\def \IPi {\text{I}\Pi}
\def \ACF {\textsf{ACF}}
\def \pCF {\textit{p}\text{CF}}
\def \ACVF {\textsf{ACVF}}
\def \HLR {\textsf{HLR}}
\def \OAG {\textsf{OAG}}
\def \RCF {\textsf{RCF}}
\DeclareMathOperator{\GL}{GL}
\DeclareMathOperator{\PGL}{PGL}
\DeclareMathOperator{\SL}{SL}
\DeclareMathOperator{\Inv}{Inv}
\DeclareMathOperator{\res}{res}
\DeclareMathOperator{\Sym}{Sym}
%\DeclareMathOperator{\char}{char}
\def \equal {=}

\def \degree {\text{degree}}
\def \app {\text{App}}
\def \FV {\text{FV}}
\def \conv {\text{conv}}
\def \cont {\text{cont}}
\DeclareMathOperator{\cl}{\text{cl}}
\DeclareMathOperator{\trcl}{\text{trcl}}
\DeclareMathOperator{\sg}{sg}
\DeclareMathOperator{\trdeg}{trdeg}
\def \Ord {\text{Ord}}

\DeclareMathOperator{\cf}{cf}
\DeclareMathOperator{\zfc}{ZFC}

%\DeclareMathOperator{\Th}{Th}
%\def \th {\text{Th}}
% \newcommand{\th}{\text{Th}}
\DeclareMathOperator{\type}{type}
\DeclareMathOperator{\zf}{\textbf{ZF}}
\def \fa {\mathfrak{a}}
\def \fb {\mathfrak{b}}
\def \fc {\mathfrak{c}}
\def \fd {\mathfrak{d}}
\def \fe {\mathfrak{e}}
\def \ff {\mathfrak{f}}
\def \fg {\mathfrak{g}}
\def \fh {\mathfrak{h}}
%\def \fi {\mathfrak{i}}
\def \fj {\mathfrak{j}}
\def \fk {\mathfrak{k}}
\def \fl {\mathfrak{l}}
\def \fm {\mathfrak{m}}
\def \fn {\mathfrak{n}}
\def \fo {\mathfrak{o}}
\def \fp {\mathfrak{p}}
\def \fq {\mathfrak{q}}
\def \fr {\mathfrak{r}}
\def \fs {\mathfrak{s}}
\def \ft {\mathfrak{t}}
\def \fu {\mathfrak{u}}
\def \fv {\mathfrak{v}}
\def \fw {\mathfrak{w}}
\def \fx {\mathfrak{x}}
\def \fy {\mathfrak{y}}
\def \fz {\mathfrak{z}}
\def \fA {\mathfrak{A}}
\def \fB {\mathfrak{B}}
\def \fC {\mathfrak{C}}
\def \fD {\mathfrak{D}}
\def \fE {\mathfrak{E}}
\def \fF {\mathfrak{F}}
\def \fG {\mathfrak{G}}
\def \fH {\mathfrak{H}}
\def \fI {\mathfrak{I}}
\def \fJ {\mathfrak{J}}
\def \fK {\mathfrak{K}}
\def \fL {\mathfrak{L}}
\def \fM {\mathfrak{M}}
\def \fN {\mathfrak{N}}
\def \fO {\mathfrak{O}}
\def \fP {\mathfrak{P}}
\def \fQ {\mathfrak{Q}}
\def \fR {\mathfrak{R}}
\def \fS {\mathfrak{S}}
\def \fT {\mathfrak{T}}
\def \fU {\mathfrak{U}}
\def \fV {\mathfrak{V}}
\def \fW {\mathfrak{W}}
\def \fX {\mathfrak{X}}
\def \fY {\mathfrak{Y}}
\def \fZ {\mathfrak{Z}}

\def \sfA {\textsf{A}}
\def \sfB {\textsf{B}}
\def \sfC {\textsf{C}}
\def \sfD {\textsf{D}}
\def \sfE {\textsf{E}}
\def \sfF {\textsf{F}}
\def \sfG {\textsf{G}}
\def \sfH {\textsf{H}}
\def \sfI {\textsf{I}}
\def \sfJ {\textsf{J}}
\def \sfK {\textsf{K}}
\def \sfL {\textsf{L}}
\def \sfM {\textsf{M}}
\def \sfN {\textsf{N}}
\def \sfO {\textsf{O}}
\def \sfP {\textsf{P}}
\def \sfQ {\textsf{Q}}
\def \sfR {\textsf{R}}
\def \sfS {\textsf{S}}
\def \sfT {\textsf{T}}
\def \sfU {\textsf{U}}
\def \sfV {\textsf{V}}
\def \sfW {\textsf{W}}
\def \sfX {\textsf{X}}
\def \sfY {\textsf{Y}}
\def \sfZ {\textsf{Z}}
\def \sfa {\textsf{a}}
\def \sfb {\textsf{b}}
\def \sfc {\textsf{c}}
\def \sfd {\textsf{d}}
\def \sfe {\textsf{e}}
\def \sff {\textsf{f}}
\def \sfg {\textsf{g}}
\def \sfh {\textsf{h}}
\def \sfi {\textsf{i}}
\def \sfj {\textsf{j}}
\def \sfk {\textsf{k}}
\def \sfl {\textsf{l}}
\def \sfm {\textsf{m}}
\def \sfn {\textsf{n}}
\def \sfo {\textsf{o}}
\def \sfp {\textsf{p}}
\def \sfq {\textsf{q}}
\def \sfr {\textsf{r}}
\def \sfs {\textsf{s}}
\def \sft {\textsf{t}}
\def \sfu {\textsf{u}}
\def \sfv {\textsf{v}}
\def \sfw {\textsf{w}}
\def \sfx {\textsf{x}}
\def \sfy {\textsf{y}}
\def \sfz {\textsf{z}}

\def \ttA {\texttt{A}}
\def \ttB {\texttt{B}}
\def \ttC {\texttt{C}}
\def \ttD {\texttt{D}}
\def \ttE {\texttt{E}}
\def \ttF {\texttt{F}}
\def \ttG {\texttt{G}}
\def \ttH {\texttt{H}}
\def \ttI {\texttt{I}}
\def \ttJ {\texttt{J}}
\def \ttK {\texttt{K}}
\def \ttL {\texttt{L}}
\def \ttM {\texttt{M}}
\def \ttN {\texttt{N}}
\def \ttO {\texttt{O}}
\def \ttP {\texttt{P}}
\def \ttQ {\texttt{Q}}
\def \ttR {\texttt{R}}
\def \ttS {\texttt{S}}
\def \ttT {\texttt{T}}
\def \ttU {\texttt{U}}
\def \ttV {\texttt{V}}
\def \ttW {\texttt{W}}
\def \ttX {\texttt{X}}
\def \ttY {\texttt{Y}}
\def \ttZ {\texttt{Z}}
\def \tta {\texttt{a}}
\def \ttb {\texttt{b}}
\def \ttc {\texttt{c}}
\def \ttd {\texttt{d}}
\def \tte {\texttt{e}}
\def \ttf {\texttt{f}}
\def \ttg {\texttt{g}}
\def \tth {\texttt{h}}
\def \tti {\texttt{i}}
\def \ttj {\texttt{j}}
\def \ttk {\texttt{k}}
\def \ttl {\texttt{l}}
\def \ttm {\texttt{m}}
\def \ttn {\texttt{n}}
\def \tto {\texttt{o}}
\def \ttp {\texttt{p}}
\def \ttq {\texttt{q}}
\def \ttr {\texttt{r}}
\def \tts {\texttt{s}}
\def \ttt {\texttt{t}}
\def \ttu {\texttt{u}}
\def \ttv {\texttt{v}}
\def \ttw {\texttt{w}}
\def \ttx {\texttt{x}}
\def \tty {\texttt{y}}
\def \ttz {\texttt{z}}

\def \bara {\bbar{a}}
\def \barb {\bbar{b}}
\def \barc {\bbar{c}}
\def \bard {\bbar{d}}
\def \bare {\bbar{e}}
\def \barf {\bbar{f}}
\def \barg {\bbar{g}}
\def \barh {\bbar{h}}
\def \bari {\bbar{i}}
\def \barj {\bbar{j}}
\def \bark {\bbar{k}}
\def \barl {\bbar{l}}
\def \barm {\bbar{m}}
\def \barn {\bbar{n}}
\def \baro {\bbar{o}}
\def \barp {\bbar{p}}
\def \barq {\bbar{q}}
\def \barr {\bbar{r}}
\def \bars {\bbar{s}}
\def \bart {\bbar{t}}
\def \baru {\bbar{u}}
\def \barv {\bbar{v}}
\def \barw {\bbar{w}}
\def \barx {\bbar{x}}
\def \bary {\bbar{y}}
\def \barz {\bbar{z}}
\def \barA {\bbar{A}}
\def \barB {\bbar{B}}
\def \barC {\bbar{C}}
\def \barD {\bbar{D}}
\def \barE {\bbar{E}}
\def \barF {\bbar{F}}
\def \barG {\bbar{G}}
\def \barH {\bbar{H}}
\def \barI {\bbar{I}}
\def \barJ {\bbar{J}}
\def \barK {\bbar{K}}
\def \barL {\bbar{L}}
\def \barM {\bbar{M}}
\def \barN {\bbar{N}}
\def \barO {\bbar{O}}
\def \barP {\bbar{P}}
\def \barQ {\bbar{Q}}
\def \barR {\bbar{R}}
\def \barS {\bbar{S}}
\def \barT {\bbar{T}}
\def \barU {\bbar{U}}
\def \barVV {\bbar{V}}
\def \barW {\bbar{W}}
\def \barX {\bbar{X}}
\def \barY {\bbar{Y}}
\def \barZ {\bbar{Z}}

\def \baralpha {\bbar{\alpha}}
\def \bartau {\bbar{\tau}}
\def \barsigma {\bbar{\sigma}}
\def \barzeta {\bbar{\zeta}}

\def \hata {\hat{a}}
\def \hatb {\hat{b}}
\def \hatc {\hat{c}}
\def \hatd {\hat{d}}
\def \hate {\hat{e}}
\def \hatf {\hat{f}}
\def \hatg {\hat{g}}
\def \hath {\hat{h}}
\def \hati {\hat{i}}
\def \hatj {\hat{j}}
\def \hatk {\hat{k}}
\def \hatl {\hat{l}}
\def \hatm {\hat{m}}
\def \hatn {\hat{n}}
\def \hato {\hat{o}}
\def \hatp {\hat{p}}
\def \hatq {\hat{q}}
\def \hatr {\hat{r}}
\def \hats {\hat{s}}
\def \hatt {\hat{t}}
\def \hatu {\hat{u}}
\def \hatv {\hat{v}}
\def \hatw {\hat{w}}
\def \hatx {\hat{x}}
\def \haty {\hat{y}}
\def \hatz {\hat{z}}
\def \hatA {\hat{A}}
\def \hatB {\hat{B}}
\def \hatC {\hat{C}}
\def \hatD {\hat{D}}
\def \hatE {\hat{E}}
\def \hatF {\hat{F}}
\def \hatG {\hat{G}}
\def \hatH {\hat{H}}
\def \hatI {\hat{I}}
\def \hatJ {\hat{J}}
\def \hatK {\hat{K}}
\def \hatL {\hat{L}}
\def \hatM {\hat{M}}
\def \hatN {\hat{N}}
\def \hatO {\hat{O}}
\def \hatP {\hat{P}}
\def \hatQ {\hat{Q}}
\def \hatR {\hat{R}}
\def \hatS {\hat{S}}
\def \hatT {\hat{T}}
\def \hatU {\hat{U}}
\def \hatVV {\hat{V}}
\def \hatW {\hat{W}}
\def \hatX {\hat{X}}
\def \hatY {\hat{Y}}
\def \hatZ {\hat{Z}}

\def \hatphi {\hat{\phi}}

\def \barfM {\bbar{\fM}}
\def \barfN {\bbar{\fN}}

\def \tila {\tilde{a}}
\def \tilb {\tilde{b}}
\def \tilc {\tilde{c}}
\def \tild {\tilde{d}}
\def \tile {\tilde{e}}
\def \tilf {\tilde{f}}
\def \tilg {\tilde{g}}
\def \tilh {\tilde{h}}
\def \tili {\tilde{i}}
\def \tilj {\tilde{j}}
\def \tilk {\tilde{k}}
\def \till {\tilde{l}}
\def \tilm {\tilde{m}}
\def \tiln {\tilde{n}}
\def \tilo {\tilde{o}}
\def \tilp {\tilde{p}}
\def \tilq {\tilde{q}}
\def \tilr {\tilde{r}}
\def \tils {\tilde{s}}
\def \tilt {\tilde{t}}
\def \tilu {\tilde{u}}
\def \tilv {\tilde{v}}
\def \tilw {\tilde{w}}
\def \tilx {\tilde{x}}
\def \tily {\tilde{y}}
\def \tilz {\tilde{z}}
\def \tilA {\tilde{A}}
\def \tilB {\tilde{B}}
\def \tilC {\tilde{C}}
\def \tilD {\tilde{D}}
\def \tilE {\tilde{E}}
\def \tilF {\tilde{F}}
\def \tilG {\tilde{G}}
\def \tilH {\tilde{H}}
\def \tilI {\tilde{I}}
\def \tilJ {\tilde{J}}
\def \tilK {\tilde{K}}
\def \tilL {\tilde{L}}
\def \tilM {\tilde{M}}
\def \tilN {\tilde{N}}
\def \tilO {\tilde{O}}
\def \tilP {\tilde{P}}
\def \tilQ {\tilde{Q}}
\def \tilR {\tilde{R}}
\def \tilS {\tilde{S}}
\def \tilT {\tilde{T}}
\def \tilU {\tilde{U}}
\def \tilVV {\tilde{V}}
\def \tilW {\tilde{W}}
\def \tilX {\tilde{X}}
\def \tilY {\tilde{Y}}
\def \tilZ {\tilde{Z}}

\def \tilalpha {\tilde{\alpha}}
\def \tilPhi {\tilde{\Phi}}

\def \barnu {\bar{\nu}}
\def \barrho {\bar{\rho}}
%\DeclareMathOperator{\ker}{ker}
\DeclareMathOperator{\im}{im}

\DeclareMathOperator{\Inn}{Inn}
\DeclareMathOperator{\rel}{rel}
\def \dote {\stackrel{\cdot}=}
%\DeclareMathOperator{\AC}{\textbf{AC}}
\DeclareMathOperator{\cod}{cod}
\DeclareMathOperator{\dom}{dom}
\DeclareMathOperator{\card}{card}
\DeclareMathOperator{\ran}{ran}
\DeclareMathOperator{\textd}{d}
\DeclareMathOperator{\td}{d}
\DeclareMathOperator{\id}{id}
\DeclareMathOperator{\LT}{LT}
\DeclareMathOperator{\Mat}{Mat}
\DeclareMathOperator{\Eq}{Eq}
\DeclareMathOperator{\irr}{irr}
\DeclareMathOperator{\Fr}{Fr}
\DeclareMathOperator{\Gal}{Gal}
\DeclareMathOperator{\lcm}{lcm}
\DeclareMathOperator{\alg}{\text{alg}}
\DeclareMathOperator{\Th}{Th}
%\DeclareMathOperator{\deg}{deg}


% \varprod
\DeclareSymbolFont{largesymbolsA}{U}{txexa}{m}{n}
\DeclareMathSymbol{\varprod}{\mathop}{largesymbolsA}{16}
% \DeclareMathSymbol{\tonm}{\boldsymbol{\to}\textbf{Nm}}
\def \tonm {\bto\textbf{Nm}}
\def \tohm {\bto\textbf{Hm}}

% Category theory
\DeclareMathOperator{\ob}{ob}
\DeclareMathOperator{\Ab}{\textbf{Ab}}
\DeclareMathOperator{\Alg}{\textbf{Alg}}
\DeclareMathOperator{\Rng}{\textbf{Rng}}
\DeclareMathOperator{\Sets}{\textbf{Sets}}
\DeclareMathOperator{\Set}{\textbf{Set}}
\DeclareMathOperator{\Grp}{\textbf{Grp}}
\DeclareMathOperator{\Met}{\textbf{Met}}
\DeclareMathOperator{\BA}{\textbf{BA}}
\DeclareMathOperator{\Mon}{\textbf{Mon}}
\DeclareMathOperator{\Top}{\textbf{Top}}
\DeclareMathOperator{\hTop}{\textbf{hTop}}
\DeclareMathOperator{\HTop}{\textbf{HTop}}
\DeclareMathOperator{\Aut}{\text{Aut}}
\DeclareMathOperator{\RMod}{R-\textbf{Mod}}
\DeclareMathOperator{\RAlg}{R-\textbf{Alg}}
\DeclareMathOperator{\LF}{LF}
\DeclareMathOperator{\op}{op}
\DeclareMathOperator{\Rings}{\textbf{Rings}}
\DeclareMathOperator{\Ring}{\textbf{Ring}}
\DeclareMathOperator{\Groups}{\textbf{Groups}}
\DeclareMathOperator{\Group}{\textbf{Group}}
\DeclareMathOperator{\ev}{ev}
% Algebraic Topology
\DeclareMathOperator{\obj}{obj}
\DeclareMathOperator{\Spec}{Spec}
\DeclareMathOperator{\spec}{spec}
% Model theory
\DeclareMathOperator*{\ind}{\raise0.2ex\hbox{\ooalign{\hidewidth$\vert$\hidewidth\cr\raise-0.9ex\hbox{$\smile$}}}}
\def\nind{\cancel{\ind}}
\DeclareMathOperator{\acl}{acl}
\DeclareMathOperator{\tspan}{span}
\DeclareMathOperator{\acleq}{acl^{\eq}}
\DeclareMathOperator{\Av}{Av}
\DeclareMathOperator{\ded}{ded}
\DeclareMathOperator{\EM}{EM}
\DeclareMathOperator{\dcl}{dcl}
\DeclareMathOperator{\Ext}{Ext}
\DeclareMathOperator{\eq}{eq}
\DeclareMathOperator{\ER}{ER}
\DeclareMathOperator{\tp}{tp}
\DeclareMathOperator{\stp}{stp}
\DeclareMathOperator{\qftp}{qftp}
\DeclareMathOperator{\Diag}{Diag}
\DeclareMathOperator{\MD}{MD}
\DeclareMathOperator{\MR}{MR}
\DeclareMathOperator{\RM}{RM}
\DeclareMathOperator{\el}{el}
\DeclareMathOperator{\depth}{depth}
\DeclareMathOperator{\ZFC}{ZFC}
\DeclareMathOperator{\GCH}{GCH}
\DeclareMathOperator{\Inf}{Inf}
\DeclareMathOperator{\Pow}{Pow}
\DeclareMathOperator{\ZF}{ZF}
\DeclareMathOperator{\CH}{CH}
\def \FO {\text{FO}}
\DeclareMathOperator{\fin}{fin}
\DeclareMathOperator{\qr}{qr}
\DeclareMathOperator{\Mod}{Mod}
\DeclareMathOperator{\Def}{Def}
\DeclareMathOperator{\TC}{TC}
\DeclareMathOperator{\KH}{KH}
\DeclareMathOperator{\Part}{Part}
\DeclareMathOperator{\Infset}{\textsf{Infset}}
\DeclareMathOperator{\DLO}{\textsf{DLO}}
\DeclareMathOperator{\PA}{\textsf{PA}}
\DeclareMathOperator{\DAG}{\textsf{DAG}}
\DeclareMathOperator{\ODAG}{\textsf{ODAG}}
\DeclareMathOperator{\sfMod}{\textsf{Mod}}
\DeclareMathOperator{\AbG}{\textsf{AbG}}
\DeclareMathOperator{\sfACF}{\textsf{ACF}}
\DeclareMathOperator{\DCF}{\textsf{DCF}}
% Computability Theorem
\DeclareMathOperator{\Tot}{Tot}
\DeclareMathOperator{\graph}{graph}
\DeclareMathOperator{\Fin}{Fin}
\DeclareMathOperator{\Cof}{Cof}
\DeclareMathOperator{\lh}{lh}
% Commutative Algebra
\DeclareMathOperator{\ord}{ord}
\DeclareMathOperator{\Idem}{Idem}
\DeclareMathOperator{\zdiv}{z.div}
\DeclareMathOperator{\Frac}{Frac}
\DeclareMathOperator{\rad}{rad}
\DeclareMathOperator{\nil}{nil}
\DeclareMathOperator{\Ann}{Ann}
\DeclareMathOperator{\End}{End}
\DeclareMathOperator{\coim}{coim}
\DeclareMathOperator{\coker}{coker}
\DeclareMathOperator{\Bil}{Bil}
\DeclareMathOperator{\Tril}{Tril}
\DeclareMathOperator{\tchar}{char}
\DeclareMathOperator{\tbd}{bd}

% Topology
\DeclareMathOperator{\diam}{diam}
\newcommand{\interior}[1]{%
  {\kern0pt#1}^{\mathrm{o}}%
}

\DeclareMathOperator*{\bigdoublewedge}{\bigwedge\mkern-15mu\bigwedge}
\DeclareMathOperator*{\bigdoublevee}{\bigvee\mkern-15mu\bigvee}

% \makeatletter
% \newcommand{\vect}[1]{%
%   \vbox{\m@th \ialign {##\crcr
%   \vectfill\crcr\noalign{\kern-\p@ \nointerlineskip}
%   $\hfil\displaystyle{#1}\hfil$\crcr}}}
% \def\vectfill{%
%   $\m@th\smash-\mkern-7mu%
%   \cleaders\hbox{$\mkern-2mu\smash-\mkern-2mu$}\hfill
%   \mkern-7mu\raisebox{-3.81pt}[\p@][\p@]{$\mathord\mathchar"017E$}$}

% \newcommand{\amsvect}{%
%   \mathpalette {\overarrow@\vectfill@}}
% \def\vectfill@{\arrowfill@\relbar\relbar{\raisebox{-3.81pt}[\p@][\p@]{$\mathord\mathchar"017E$}}}

% \newcommand{\amsvectb}{%
% \newcommand{\vect}{%
%   \mathpalette {\overarrow@\vectfillb@}}
% \newcommand{\vecbar}{%
%   \scalebox{0.8}{$\relbar$}}
% \def\vectfillb@{\arrowfill@\vecbar\vecbar{\raisebox{-4.35pt}[\p@][\p@]{$\mathord\mathchar"017E$}}}
% \makeatother
% \bigtimes

\DeclareFontFamily{U}{mathx}{\hyphenchar\font45}
\DeclareFontShape{U}{mathx}{m}{n}{
      <5> <6> <7> <8> <9> <10>
      <10.95> <12> <14.4> <17.28> <20.74> <24.88>
      mathx10
      }{}
\DeclareSymbolFont{mathx}{U}{mathx}{m}{n}
\DeclareMathSymbol{\bigtimes}{1}{mathx}{"91}
% \odiv
\DeclareFontFamily{U}{matha}{\hyphenchar\font45}
\DeclareFontShape{U}{matha}{m}{n}{
      <5> <6> <7> <8> <9> <10> gen * matha
      <10.95> matha10 <12> <14.4> <17.28> <20.74> <24.88> matha12
      }{}
\DeclareSymbolFont{matha}{U}{matha}{m}{n}
\DeclareMathSymbol{\odiv}         {2}{matha}{"63}


\newcommand\subsetsim{\mathrel{%
  \ooalign{\raise0.2ex\hbox{\scalebox{0.9}{$\subset$}}\cr\hidewidth\raise-0.85ex\hbox{\scalebox{0.9}{$\sim$}}\hidewidth\cr}}}
\newcommand\simsubset{\mathrel{%
  \ooalign{\raise-0.2ex\hbox{\scalebox{0.9}{$\subset$}}\cr\hidewidth\raise0.75ex\hbox{\scalebox{0.9}{$\sim$}}\hidewidth\cr}}}

\newcommand\simsubsetsim{\mathrel{%
  \ooalign{\raise0ex\hbox{\scalebox{0.8}{$\subset$}}\cr\hidewidth\raise1ex\hbox{\scalebox{0.75}{$\sim$}}\hidewidth\cr\raise-0.95ex\hbox{\scalebox{0.8}{$\sim$}}\cr\hidewidth}}}
\newcommand{\stcomp}[1]{{#1}^{\mathsf{c}}}

\setlength{\baselineskip}{0.5in}

\stackMath
\newcommand\yrightarrow[2][]{\mathrel{%
  \setbox2=\hbox{\stackon{\scriptstyle#1}{\scriptstyle#2}}%
  \stackunder[0pt]{%
    \xrightarrow{\makebox[\dimexpr\wd2\relax]{$\scriptstyle#2$}}%
  }{%
   \scriptstyle#1\,%
  }%
}}
\newcommand\yleftarrow[2][]{\mathrel{%
  \setbox2=\hbox{\stackon{\scriptstyle#1}{\scriptstyle#2}}%
  \stackunder[0pt]{%
    \xleftarrow{\makebox[\dimexpr\wd2\relax]{$\scriptstyle#2$}}%
  }{%
   \scriptstyle#1\,%
  }%
}}
\newcommand\yRightarrow[2][]{\mathrel{%
  \setbox2=\hbox{\stackon{\scriptstyle#1}{\scriptstyle#2}}%
  \stackunder[0pt]{%
    \xRightarrow{\makebox[\dimexpr\wd2\relax]{$\scriptstyle#2$}}%
  }{%
   \scriptstyle#1\,%
  }%
}}
\newcommand\yLeftarrow[2][]{\mathrel{%
  \setbox2=\hbox{\stackon{\scriptstyle#1}{\scriptstyle#2}}%
  \stackunder[0pt]{%
    \xLeftarrow{\makebox[\dimexpr\wd2\relax]{$\scriptstyle#2$}}%
  }{%
   \scriptstyle#1\,%
  }%
}}

\newcommand\altxrightarrow[2][0pt]{\mathrel{\ensurestackMath{\stackengine%
  {\dimexpr#1-7.5pt}{\xrightarrow{\phantom{#2}}}{\scriptstyle\!#2\,}%
  {O}{c}{F}{F}{S}}}}
\newcommand\altxleftarrow[2][0pt]{\mathrel{\ensurestackMath{\stackengine%
  {\dimexpr#1-7.5pt}{\xleftarrow{\phantom{#2}}}{\scriptstyle\!#2\,}%
  {O}{c}{F}{F}{S}}}}

\newenvironment{bsm}{% % short for 'bracketed small matrix'
  \left[ \begin{smallmatrix} }{%
  \end{smallmatrix} \right]}

\newenvironment{psm}{% % short for ' small matrix'
  \left( \begin{smallmatrix} }{%
  \end{smallmatrix} \right)}

\newcommand{\bbar}[1]{\mkern 1.5mu\overline{\mkern-1.5mu#1\mkern-1.5mu}\mkern 1.5mu}

\newcommand{\bigzero}{\mbox{\normalfont\Large\bfseries 0}}
\newcommand{\rvline}{\hspace*{-\arraycolsep}\vline\hspace*{-\arraycolsep}}

\font\zallman=Zallman at 40pt
\font\elzevier=Elzevier at 40pt

\newcommand\isoto{\stackrel{\textstyle\sim}{\smash{\longrightarrow}\rule{0pt}{0.4ex}}}
\newcommand\embto{\stackrel{\textstyle\prec}{\smash{\longrightarrow}\rule{0pt}{0.4ex}}}

% from http://www.actual.world/resources/tex/doc/TikZ.pdf

\tikzset{
modal/.style={>=stealth’,shorten >=1pt,shorten <=1pt,auto,node distance=1.5cm,
semithick},
world/.style={circle,draw,minimum size=0.5cm,fill=gray!15},
point/.style={circle,draw,inner sep=0.5mm,fill=black},
reflexive above/.style={->,loop,looseness=7,in=120,out=60},
reflexive below/.style={->,loop,looseness=7,in=240,out=300},
reflexive left/.style={->,loop,looseness=7,in=150,out=210},
reflexive right/.style={->,loop,looseness=7,in=30,out=330}
}


\makeatletter
\newcommand*{\doublerightarrow}[2]{\mathrel{
  \settowidth{\@tempdima}{$\scriptstyle#1$}
  \settowidth{\@tempdimb}{$\scriptstyle#2$}
  \ifdim\@tempdimb>\@tempdima \@tempdima=\@tempdimb\fi
  \mathop{\vcenter{
    \offinterlineskip\ialign{\hbox to\dimexpr\@tempdima+1em{##}\cr
    \rightarrowfill\cr\noalign{\kern.5ex}
    \rightarrowfill\cr}}}\limits^{\!#1}_{\!#2}}}
\newcommand*{\triplerightarrow}[1]{\mathrel{
  \settowidth{\@tempdima}{$\scriptstyle#1$}
  \mathop{\vcenter{
    \offinterlineskip\ialign{\hbox to\dimexpr\@tempdima+1em{##}\cr
    \rightarrowfill\cr\noalign{\kern.5ex}
    \rightarrowfill\cr\noalign{\kern.5ex}
    \rightarrowfill\cr}}}\limits^{\!#1}}}
\makeatother

% $A\doublerightarrow{a}{bcdefgh}B$

% $A\triplerightarrow{d_0,d_1,d_2}B$

\def \uhr {\upharpoonright}
\def \rhu {\rightharpoonup}
\def \uhl {\upharpoonleft}


\newcommand{\floor}[1]{\lfloor #1 \rfloor}
\newcommand{\ceil}[1]{\lceil #1 \rceil}
\newcommand{\lcorner}[1]{\llcorner #1 \lrcorner}
\newcommand{\llb}[1]{\llbracket #1 \rrbracket}
\newcommand{\ucorner}[1]{\ulcorner #1 \urcorner}
\newcommand{\emoji}[1]{{\DejaSans #1}}
\newcommand{\vprec}{\rotatebox[origin=c]{-90}{$\prec$}}

\newcommand{\nat}[6][large]{%
  \begin{tikzcd}[ampersand replacement = \&, column sep=#1]
    #2\ar[bend left=40,""{name=U}]{r}{#4}\ar[bend right=40,',""{name=D}]{r}{#5}\& #3
          \ar[shorten <=10pt,shorten >=10pt,Rightarrow,from=U,to=D]{d}{~#6}
    \end{tikzcd}
}


\providecommand\rightarrowRHD{\relbar\joinrel\mathrel\RHD}
\providecommand\rightarrowrhd{\relbar\joinrel\mathrel\rhd}
\providecommand\longrightarrowRHD{\relbar\joinrel\relbar\joinrel\mathrel\RHD}
\providecommand\longrightarrowrhd{\relbar\joinrel\relbar\joinrel\mathrel\rhd}
\def \lrarhd {\longrightarrowrhd}


\makeatletter
\providecommand*\xrightarrowRHD[2][]{\ext@arrow 0055{\arrowfill@\relbar\relbar\longrightarrowRHD}{#1}{#2}}
\providecommand*\xrightarrowrhd[2][]{\ext@arrow 0055{\arrowfill@\relbar\relbar\longrightarrowrhd}{#1}{#2}}
\makeatother

\newcommand{\metalambda}{%
  \mathop{%
    \rlap{$\lambda$}%
    \mkern3mu
    \raisebox{0ex}{$\lambda$}%
  }%
}

%% https://tex.stackexchange.com/questions/15119/draw-horizontal-line-left-and-right-of-some-text-a-single-line
\newcommand*\ruleline[1]{\par\noindent\raisebox{.8ex}{\makebox[\linewidth]{\hrulefill\hspace{1ex}\raisebox{-.8ex}{#1}\hspace{1ex}\hrulefill}}}

% https://www.dickimaw-books.com/latex/novices/html/newenv.html
\newenvironment{Block}[1]% environment name
{% begin code
  % https://tex.stackexchange.com/questions/19579/horizontal-line-spanning-the-entire-document-in-latex
  \noindent\textcolor[RGB]{128,128,128}{\rule{\linewidth}{1pt}}
  \par\noindent
  {\Large\textbf{#1}}%
  \bigskip\par\noindent\ignorespaces
}%
{% end code
  \par\noindent
  \textcolor[RGB]{128,128,128}{\rule{\linewidth}{1pt}}
  \ignorespacesafterend
}

\mathchardef\mhyphen="2D % Define a "math hyphen"

\def \QQ {\quad}
\def \QW {​\quad}

\makeindex
\usepackage[UTF8]{ctex}
\def \pred {\text{pred}}
\def \quo {\text{quo}}
\def \rem {\text{rem}}
\author{wu}
\date{\today}
\title{First Order Logic}
\hypersetup{
 pdfauthor={wu},
 pdftitle={First Order Logic},
 pdfkeywords={},
 pdfsubject={},
 pdfcreator={Emacs 28.0.90 (Org mode 9.6)}, 
 pdflang={English}}
\begin{document}

\maketitle
\tableofcontents

\section{递归论基础}
\label{sec:org316feae}
\subsection{primitive recursive}
\label{sec:org1d6c484}
\begin{definition}[]
初始函数
\begin{enumerate}
\item 零函数\(Z(x)=0\)
\item 后继函数\(S(x)=x+1\)
\item 投射函数\(\pi_i^n(x_1,\dots,x_i,\dots,x_n)=x_i\)
\end{enumerate}
\end{definition}

\begin{definition}[]
设\(g:\N^n\to\N\)与\(h:\N^{n+2}\to\N\),称\(f:\N^{n+1}\to\N\)是从\(g\)和\(h\) \textbf{经原始递归得到的} ,如果
\begin{enumerate}
\item \(f(\barx,0)=g(\barx)\)
\item \(f(\barx,n+1)=h(\barx,f(\barx,n),n)\)
\end{enumerate}
\end{definition}

\begin{definition}[]
全体原始递归函数的集合\(C\)是最小的满足以下条件的自然数上的函数集合
\begin{enumerate}
\item 初始函数\(\subseteq C\)
\item 复合封闭
\item 原始递归封闭
\end{enumerate}


称\(C\)中的元素为原始递归函数
\end{definition}

\begin{lemma}[]
以下为原始递归函数
\begin{enumerate}
\item 加法
\item \(C_k^n(x_1,\dots,x_n)=k\)
\item \(x\cdot y\), \(x^y\),\(x!\)
\item 非零检测和零检测
\begin{equation*}
\sigma(x)=
\begin{cases}
0&x=0\\
1
\end{cases}\quad
\delta(x)=
\begin{cases}
1&x=0\\
0
\end{cases}
\end{equation*}
\item 前驱函数\(\pred(x)\)
\item 截断减法
\begin{equation*}
x\dot-y=
\begin{cases}
0&x<y\\
x-y&x\ge y
\end{cases}
\end{equation*}
\end{enumerate}
\end{lemma}

\begin{proof}
\(\sigma(0)=0\),\(\sigma(n+1)=C_1^2(n,\sigma(n))\)

\(\pred(0)=0\),\(\pred(n+1)=\pi_1^2(n,\pred(n))\)
\end{proof}

\begin{lemma}[]
\(f:\N^k\to\N\) p.r., \(g:\N^r\to\N\)
\begin{equation*}
g(x_1,\dots,x_r)=f(y_1,\dots,y_k)
\end{equation*}
\(y_j\) is either \(x_i\) or a constant, then \(g\) is p.r.
\end{lemma}

\begin{proof}
\(h_1,\dots,h_k:\N^r\to\N\)
\begin{itemize}
\item if \(y_j\) is \(x_i\), then \(h_j(x_1,\dots,x_r)=\pi_i^r(x_1,\dots,x_r)\)
\item if \(y_j\) is a constant \(k\in\N\), then \(h_j(x_1,\dots,x_r)=C_k^r(x_1,\dots,x_r)\)
\end{itemize}


\begin{equation*}
g(x_1,\dots,x_r)=f(h_1(x_1,\dots,x_r),\dots,(h_k(x_1,\dots,x_r)))
\end{equation*}
\end{proof}

\begin{definition}[]
\(A\subseteq\N^k\) is \textbf{primitive recursive} if its characteristic function is p.r.
\end{definition}

\begin{lemma}[]
\begin{enumerate}
\item If \(A,B\subseteq\N^k\) is p.r., then \(\N^k\setminus A\), \(A\cup B\), \(A\cap B\) is p.r.
\item if \(P,Q\) is p.r. predicate, then \(\neg P\), \(P\vee Q\), \(P\wedge Q\) is p.r.
\end{enumerate}
\end{lemma}

\begin{proof}
\(1\dot-\chi_A(x)\), \(\sigma(\chi_A(x)+\chi_B(x))\), \(\chi_A(x)\cdot\chi_B(x)\)
\end{proof}

if \(f:\N^k\to\N\) is p.r., then
\begin{align*}
&\{x\in\N^k\mid f(x)=0\}\\
&\{x\in\N^k\mid f(x)>0\}
\end{align*}
is p.r.

\begin{lemma}[]
If \(f_1,f_2\) is \(k\)-ary p.r., \(P\) p.r. predicate, then
\begin{equation*}
f(\barx)=
\begin{cases}
f_1(\barx)&P(\barx)\\
f_2(\barx)
\end{cases}
\end{equation*}
is p.r.
\end{lemma}

\begin{proof}
\(f(x)=\chi_P(x)f_1(x)+(1\dot-\chi_P(x))f_2(x)\)
\end{proof}

\begin{lemma}[]
\(\quo(x,y)\) and \(\rem(x,y)\) are p.r.
\end{lemma}

\begin{proof}
Intuition
\begin{align*}
&\rem(x,y+1)=
\begin{cases}
\rem(x,y)+1&\rem(x,y)+1<x\\
0
\end{cases}\\
&\quo(x,y+1)=
\begin{cases}
\quo(x,y)&\rem(x,y)+1<x\\
\quo(x,y)+1
\end{cases}
\end{align*}
solution
\begin{align*}
&\rem(x,0)=0\\
&\rem(x,y+1)=(\rem(x,y)+1)\sigma(x-\rem(x,y)-1)\\
&\quo(x,0)=0\\
&\quo(x,y+1)=\quo(x,y)\sigma(x-\rem(x,y)-1)+(\quo(x,y)+1)\delta(x-\rem(x,y)-1)
\end{align*}
\end{proof}

\begin{definition}[]
\begin{enumerate}
\item \((\exists x<a)\phi(x):=\exists x(x<a\wedge\phi(x))\)
\item \((\forall x<a)\phi(x):=\forall x(x<a\to\phi(x))\)
\end{enumerate}


\textbf{bounded quantifier}
\end{definition}

\begin{lemma}[]
If \(P(\barx,y)\) is a p.r. predicate
\begin{enumerate}
\item predicate
\begin{align*}
&E(\barx,y):=(\exists z\le y)P(\barx,z)\\
&A(\barx,y):=(\forall z\le y)P(\barx,z)
\end{align*}
are p.r.
\item function
\begin{equation*}
f(\barx,y):=(\mu z\le y)P(\barx,z)
\end{equation*}
is p.r.
\end{enumerate}
\end{lemma}

\begin{lemma}[]
\begin{enumerate}
\item predicate ``\(x\) divides \(y\)'' is p.r.
\item ``\(x\) is not prime'' ``\(x\) is prime'' are p.r.
\item \(p:\N\to\N\), \(n\mapsto n\text{th prime}\) is p.r.
\end{enumerate}
\end{lemma}

\begin{proof}
\(p(0)=2\). \(p(n+1)=(\mu z\le y)(z>p(n)\wedge z\text{ prime}\wedge y=p(n)!+1)\)
\end{proof}

\begin{itemize}
\item \(\la a_0,\dots,a_n\ra:=p_0^{a_0+1}\dots p_n^{a_{n}+1}\)  is the Gödel number of \((a_0,\dots,a_n)\)
\item \(\la\ra=1\)
\item \(\lh:\N\to\N\) is \(\lh(a)=\mu k\le a(p_k\nmid a)\)
\item \((a)_i:\N^2\to\N\) is \((a)_i=(\mu k\le a)(p_i^{k+2}\nmid a)\)
\item for any \(a=\la a_0,\dots,a_n\ra\), \((a)_i=a_i\)
\item concatenation function \(\tieconcat:\N^2\to\N\)
\begin{equation*}
a\tieconcat b=a\cdot\prod_{i<\lh(b)}p_{\lh(a)+i}^{(b)_i+1}
\end{equation*}
\end{itemize}



\begin{lemma}[]
\begin{enumerate}
\item Set of Gödel numbers are p.r.
\item \(\lh(a)\) and \((a)_i\) is p.r.
\item \(a\tieconcat b\) is p.r. and
\begin{equation*}
\la a_0,\dots,a_n\ra\tieconcat\la b_0,\dots,b_m\ra=\la a_0,\dots,a_n,b_0,\dots,b_m\ra
\end{equation*}
\end{enumerate}
\end{lemma}

\begin{proof}
  \begin{equation*}
\exists n\le x\left( \forall i\le n(p_i\mid x)\wedge\forall j\le x(j>n\to p_j\nmid x) \right)
  \end{equation*}
\end{proof}

function \(f(\barx,y)\),
  \begin{equation*}
F(\barx,n)=p_0^{f(\barx,0)+1}\dots p_n^{f(\barx,n)+1}
  \end{equation*}

\begin{definition}[]
function \(g(\barx)\) and \(h(\barx,y,z)\), \(f(\barx,y)\)是从\(g\)与\(h\)经 \textbf{强递归} 得到的如果
  \begin{align*}
f(\barx,0)&=g(\barx)\\
f(\barx,n+1)&=h(\barx,n,F(\barx,n))
  \end{align*}
\end{definition}

\begin{lemma}[]
如果\(f(\barx,y)\)是从\(g\)与\(h\)经强递归得到,and  \(g,h\) p.r., then \(f\) is p.r.
\end{lemma}

\begin{proof}
  \begin{align*}
F(\barx,0)&=2^{f(\barx,0)+1}=2^{g(\barx)+1}\\
F(\barx,n+1)&=F(\barx,n)p_{n+1}^{f(\barx,n+1)+1}=F(\barx,n)p_{n+1}^{h(\barx,n,F(\barx,n))+1}
  \end{align*}
Hence \(F(\barx,y)\) is p.r., so \(f(\barx,y)=(F(\barx,y))_y\) is p.r.
\end{proof}
\subsection{recursive function}
\label{sec:orge6e5fa9}
\begin{itemize}
\item 假设有一个程序可以枚举所有的原始递归函数
\item 设\(g_0,g_1,g_2,\dots\)是所有原始递归函数的枚举
\item 令\(F:\N\to\N\)为\(F(n)=g_n(n)+1\)
\item 虽然\(F\)在直观上可计算,但不属于原始递归函数
\end{itemize}


\begin{definition}[]
total function \(f:\N^{n+1}\to\N\),\(g(\barx)\)是从\(f\)通过正则极小化或正则\(\mu\)-算子得到的如果
\begin{itemize}
\item \(\forall\barx\exists yf(\barx,y)=0\)
\item \(g(\barx)\)是使得\(f(\barx,y)=0\)最小的\(y\)
\end{itemize}


记作\(g(\barx)=\mu y(f(\barx,y)=0)\)
\end{definition}

\begin{definition}[]
\begin{enumerate}
\item 全体递归函数的集合为最小的包含所有初始函数,并且对复合、原始递归、正则极小化封闭的函数集合
\item \(A\subseteq\N^k\)是递归集如果\(\chi_A\)是递归函数
\end{enumerate}
\end{definition}

\begin{definition}[]
partial function \(f\), \(g\)是从\(f\)通过极小化或者由\(\mu\)-算子得到的如果
\begin{equation*}
g(\barx)=\mu y(\forall z\le y(f(x,z)\downarrow)\wedge f(x,y)=0)
\end{equation*}
\end{definition}

\begin{definition}[]
全体部分递归函数的集合为最小的包含所有初始函数、并且怼复合、原始递归、极小化封闭的函数集合
\end{definition}

\begin{lemma}[]
Ackermann function is partial recursive
\begin{gather*}
A(0,y)=y+1,\quad A(x+1,0)=A(x,1)\\
A(x+1,y+1)=A(x,A(x+1,y))
\end{gather*}
\end{lemma}
\subsection{Turing Machine}
\label{sec:org1697bab}
规定输入向量为\((x_1,\dots,x_n)\)时,初始格局为
\begin{equation*}
q_s1^{x_1+1}01^{x_2+1}0\dots 01^{x_k+1}
\end{equation*}

输出时,格局为\(q_h1^y\),表示输出值为\(y\)

\begin{definition}[]
一个部分函数\(f:\N^k\to\N\)是被图灵机\(M\)所计算的,或者说图灵机\(M\)计算函数\(f\),如果
\begin{equation*}
f(x)=
\begin{cases}
y&\text{如果$M$对输入\(x\)的输出为\(y\)}\\
\text{没有定义}&\text{如果计算过程无限或没有终止格局}
\end{cases}
\end{equation*}
称部分函数\(f\)为图灵可计算的,如果存在一个图灵机\(M\)计算它
\end{definition}
\subsection{turing computability and partial recursive function}
\label{sec:org3a10e46}
\begin{theorem}[]
一个函数是图灵可计算的当且仅当它是部分递归的
\end{theorem}

\begin{lemma}[]
每个初始函数都是图灵可计算的
\end{lemma}

\begin{lemma}[]
任何一台标准图灵机都可以被一台单向无穷纸带图灵机模拟
\end{lemma}

\begin{corollary}[]
任何图灵可计算函数\(h\)都可以被一台加了如下限制的图灵机计算
\begin{enumerate}
\item 在初始格局中,纸带中有一个不在字母表中的新字符\$,可以在任何实现给定的位置,只要不混在输入字符
串中见
\item 计算完成后,\$左边的内容不变
\item 输出字符串的位置起始于\$右边一格
\end{enumerate}
\end{corollary}

\begin{lemma}[]
图灵可计算对复合封闭
\end{lemma}

\begin{definition}[]
\(T(e,x,z)\)表示\(z\)是图灵机\(e\)对输入\(x\)的计算过程(格局序列)的编码,称为Kleene谓词
\end{definition}

\begin{lemma}[]
Kleene predicate is p.r.
\end{lemma}

\begin{theorem}[]
存在原始递归函数\(U:\N\to\N\)和原始递归谓词\(T(e,x,z)\)使得对任意的部分递归函数\(f:\N\to\N\)都存在自然
数\(e\)使得\(f(x)=U(\mu zT(e,x,z))\)
\end{theorem}

\begin{corollary}[]
一个函数是递归的当且仅当它是部分递归的全函数
\end{corollary}

\begin{proof}
\(\Leftarrow\). 部分递归的全函数\(f(x)=U(\mu zT(e,x,z))\)满足正则性
\end{proof}

\begin{theorem}[通用函数定理]
存在一个通用的部分递归函数;即存在二元函数\(\Phi:\N^2\to\N\)使得对任何一元部分递归函数\(f:\N\to\N\)都存在一
个自然数\(e\)使得对所有\(x\)有\(f(x)=\Phi(e,x)\)
\end{theorem}

令\(e_0,e_1,\dots\)是图灵机的一个枚举,则\(\phi_0(x),\phi_1(x),\dots\)是对应的对全体部分递归函数的枚举,
即\(\phi_i(x)=\Phi(e_i,x)\)

\begin{theorem}[]
对递归函数来说,不存在通用函数,即不存在递归函数\(T:\N^2\to\N\)使得对任何一元递归函数\(f:\N\to\N\)都存在
一个自然数\(e\)使得对所有\(x\)有\(f(x)=T(e,x)\)
\end{theorem}

存在一个部分函数\(f\)使得对任何递归全函数\(g\),都存在\(n\in\dom(f)\)使得\(f(n)\neq g(n)\)

\(f(n)=\Phi(n,n)+1\), \(g(x)=\Phi(m,x)\),\(f(m)=\Phi(m,m)+1\neq g(x)\)
\subsection{递归可枚举}
\label{sec:org943acbf}
\begin{definition}[]
\(A\subseteq\N\) is recursively enumerable (r.e.) if \(A=\emptyset\) or \(A=\im(f)\) for some recursive \(f\)
\end{definition}

\begin{lemma}[]
\(A\subseteq\N\), TFAE
\begin{enumerate}
\item \(A\) r.e.
\item \(A=\emptyset\) or \(A=\im(f)\) for some p.r. \(f\)
\item \(A=\emptyset\) or \(A=\im(f)\) for some partial recursive \(f\)
\item \(\chi_A\) is partial recursive
\item \(A=\dom(f)\) for some partial recursive \(f\)
\item there is a recursive/primitive recursive predicate \(R(x,y)\) s.t.
\begin{equation*}
A=\{x\mid\exists yR(x,y)\}
\end{equation*}
\end{enumerate}
\end{lemma}

\begin{proof}
\(1\to 2\). Suppose \(A=\im(f)\) where \(f=U(\mu zT(e,x,z))\), for any \(a_0\in A\)
\begin{equation*}
F(x,n)=
\begin{cases}
U(\mu\le nT(e,x,n))&\exists y\le nT(e,x,y)\\
a_0
\end{cases}
\end{equation*}
Then \(F(\N^2)=f(\N)\)

\(2\to 4\). \(A=f(\N)\)
\begin{equation*}
\chi_A(y)=C_1^1(\mu xf(x)=y)
\end{equation*}

\(5\to 6\). \(f(x)=U(\mu zT(e,x,z))\)
\begin{equation*}
\dom(f)=\{x\mid\exists zT(e,x,z)\}
\end{equation*}

\(6\to 1\).
\begin{equation*}
A=\{x\mid\exists yR(x,y)\},g(y)=x\cdot C_1^1(\mu xR(x,y))
\end{equation*}
\end{proof}

\begin{theorem}[]
一个自然数的集合\(A\)是递归的当且仅当\(A\)和它的补集\(\N\setminus A\)都是递归可枚举的
\end{theorem}

\begin{proof}
设\(A\)是\(f_1:2\N\to\N\)的值域,\(\N\setminus A\)是\(f_2:2\N+1\to\N\)的值域

\(R_i(x,y)\Leftrightarrow y=f_i(x)\)
\begin{equation*}
h(y)=\mu x(R_1(x,y)\vee R_2(x,y))
\end{equation*}
\end{proof}

\begin{definition}[]
\(A,B\subseteq\N^k\) r.e., then
\begin{enumerate}
\item \(A\cup B\),\(A\cap B\) r.e.
\item \(\{x\in\N^{k-1}\mid\exists y(x,y)\in A\}\) r.e.
\end{enumerate}
\end{definition}

\begin{theorem}[]
\(K=\{e\in\N\mid\phi(e,e)\downarrow\}\) is r.e., but not recursive
\end{theorem}

\begin{proof}
\(K=\dom(\Phi(x,x))\), thus is r.e.

If \(K\) is recursive, then \(\N\setminus K\) is recursive. Thus \(x\in K\) and \(x\notin K\) are recursive
predicates. Then function
\begin{equation*}
f(x)=
\begin{cases}
\Phi(x,x)+1&x\in K\\
0
\end{cases}
\end{equation*}
is recursive. Thus there is a natural number \(e\) s.t. \(f(x)=\Phi(e,x)\). If \(e\in K\),
then \(f(e)=\Phi(e,e)+1\), a contradiction. If \(e\notin K\), then \(\Phi(e,e)\uparrow\), but \(f(e)=0\), contradiction
\end{proof}
\section{自然数的模型}
\label{sec:org48014f3}
\begin{definition}[皮亚诺公理系统]
语言\(L_{ar}=\{0,S,+,\times\}\),则皮亚诺公理系统\(\PA\)由下列公式的全称概括组成
\begin{enumerate}
\item \(Sx\neq 0\)
\item \(Sx=Sy\to x=y\)
\item \(x+0=x\)
\item \(x+Sy=S(x+y)\)
\item \(x\times Sy=x\times y+x\)
\item 对每个一阶公式\(\phi\),都有\(\phi\)的归纳公理
\begin{equation*}
(\phi(0)\wedge\forall(\phi(x)\to\phi(S(x))))\to\forall x\phi(x)
\end{equation*}
\end{enumerate}
\end{definition}
\subsection{可判定的理论}
\label{sec:orgdf4f4ba}
\begin{definition}[]
理论\(T\)可公理化如果存在一个可判定的闭语句集\(\Sigma\)使得
\begin{equation*}
T=\{\sigma\mid\Sigma\vDash\sigma\}
\end{equation*}
如果\(\Sigma\)有穷,则称\(T\)是有穷公理化的
\end{definition}

\begin{definition}[]
理论\(T\)是可判定的,如果存在一个算法,使得对任何闭语句\(\sigma\),该算法都能告诉我们\(\sigma\)是否在\(T\)
中
\end{definition}

\begin{proof}
\(T\) is decidable iff
\begin{equation*}
\# T=\{\#\sigma\mid\sigma\in T\}
\end{equation*}
is a recursive set
\end{proof}

\begin{lemma}[]
complete axiomatizable theory is decidable
\end{lemma}

\begin{proof}


A set is recursive iff itself and its complement is r.e.. \(T=\{\sigma\mid\Sigma\vDash\sigma\}=\{\sigma\mid\Sigma\vdash\sigma\}\).

\(\Sigma\) the axiom set. \(\Sigma\) is decidable, there is a recursive function \(f:\N\to\#T\), for any sentence
\(\tau\), check whether \(\#\tau\) or \(\#\tau\) is in \(f(\N)\)

\(\Sigma\)可判定,\(\chi_\Sigma\)递归
\end{proof}

\begin{theorem}[Łoś-Vaught test]
\(T\) is a theory on countable language, if
\begin{enumerate}
\item \(T\) is \(\lambda\)-categorical for some cardinal \(\lambda\)
\item \(T\) doesn't have finite model
\end{enumerate}


Then \(T\) is complete
\end{theorem}

\begin{proof}
Suppose \(T\) is not complete, then there is \(\sigma\) s.t. \(T\cup\{\sigma\}\) and \(T\cup\{\neg\sigma\}\) is consistent.

Let \(\fM_1\vDash T\cup\{\sigma\}\), \(\fM\vDash T\cup\{\not\sigma\}\). \(\fM_1\) and \(\fM_2\) are infinite

By LST, since \(T\) is at most countable, there is \(\fM_1'\) and \(\fM_2'\) of cardinality \(\lambda\) s.t.
\begin{equation*}
\fM_1'\vDash T\cup\{\sigma\},\quad\fM_2'\vDash T\cup\{\neg\sigma\}
\end{equation*}
By categoricity, \(\fM_1'\cong\fM_2'\)
\end{proof}
\subsection{只含后继的自然数模型}
\label{sec:org6252fea}
\begin{definition}[]
结构\(\fN_S=(\N,0,S)\),语言\(L_S=\{0,S\}\),公理集
\begin{enumerate}
\item \(0\neq Sx\)
\item \(Sx=Sy\to x=y\)
\item \(x\neq 0\to\exists y(x=s(y))\)
\item \(\bigwedge_{i<n}(Sx_i=x_{i+1})\to x_0\neq x_n\)
\end{enumerate}


令\(T_S\)为以上公式的全称概括的逻辑后承的集合
\end{definition}

\begin{lemma}[]
\(T_S\)是不可数范畴的理论,从而是完备的
\end{lemma}

\begin{theorem}[]
\(\Th(\fN_S)\) has quantifier elimination
\end{theorem}

\section{哥德尔不完备性定理}
\label{sec:org6b7dc5e}

\subsection{鲁宾逊算数理论Q}
\label{sec:orgdc2319d}
设\(T\)是一个包含\(Q\)的理论
\begin{definition}[]
称一个自然数上的\(k\)-元关系\(P\)在\(T\)中 \textbf{数码逐点可表示的} (简称可表示的),如果存在公式\(\rho(x)\),
称为\(P\)的一个表示公式,使得
\begin{align*}
&(n_1,\dots,n_k)\in P\Rightarrow T\vdash\rho(n_1,\dots,n_k)\\
&(n_1,\dots,n_k)\notin P\Rightarrow T\vdash\neg\rho(n_1,\dots,n_k
\end{align*}
\end{definition}

\begin{lemma}[]
如果\(T\)可公理化,则\(T\)是递归可枚举的
\end{lemma}

\begin{proof}
\(T\)可公理化\(\Leftrightarrow\)存在可判定的\(\Sigma\)使得
\begin{equation*}
T=\{\sigma\mid T\vdash\sigma\}
\end{equation*}

\(\Sigma\)可判定:\(\sharp\Sigma=\{\sharp\sigma\mid\sigma\in\Sigma\}\subseteq\N\)可判定(递归)集合

\(\Sigma\)的证明集合\(P_\Sigma\)可判定(递归):
\begin{itemize}
\item 公式序列\((\sharp\sigma_1,\dots,\sharp\sigma_n)\mapsto p\in\N\)
\item \(p\in P_\Sigma\Leftrightarrow\forall i<\ln(p)\)
\begin{itemize}
\item \(p_i\in\Sigma\cup A\)或者
\item \(\exists j,k<\ln(p)(\alpha_k:=\alpha_j\to\alpha_i)\),\(\sharp\alpha_{ijk}=p_{ijk}\)
\end{itemize}
\item \(P_\Sigma\)递归
\item \(\sigma\in T\Leftrightarrow\exists p(p\in P_\Sigma\wedge\exists i<\ln(p)(p_i=\sharp\sigma))\)
\item \(T\) (\(\sharp T\))是递归可枚举的
\item \(\sharp T\)递归函数的值域
\end{itemize}
\end{proof}

\begin{lemma}[]
\begin{enumerate}
\item 自然数上的等同关系\(\{(n,n)\mid n\in\N\}\)被公式\(x=x\)表示
\item \(\le\)关系被\(x\le y\)表示
\item 如果\(P\)是可表示的,则\(P\)是递归的
\item 可表示的关系在布尔运算下封闭
\item 如果\(P\)在\(Q\)中被\(\rho\)表示,则\(P\)在\(Q\)的任何一致扩张中都被\(\rho\)表示
\item \(P\)在\(\Th(\fN)\)中被\(\rho\)表示当且仅当\(P\)在结构\(\fN\)中被\(\rho\)表示
\end{enumerate}
\end{lemma}

\begin{proof}
\begin{enumerate}
\setcounter{enumi}{2}
\item \(P\)是可表示的使得肯定能枚举出\(\rho(n_1,\dots,n_k)\)或者\(\neg\rho(n_1,\dots,n_k)\),

对于枚举函数\(f\),\(\#\rho\in\im(f)\)或者\(\#\neg\rho\in\im(f)\),不管怎么说肯定存在一个自然数对应它们,并
且自然数是有限的
\end{enumerate}
\end{proof}

\begin{corollary}[]
\(P\)在\(Q\)中可表示,则\(P\)在\(\fN\)中可定义
\end{corollary}

\begin{proof}
\(\Th(\fN)\)是\(Q\)的一致扩张
\end{proof}

\begin{itemize}
\item 称一个\(L_{ar}\)公式是\(\Delta_0\)的,如果它只包含有界量词
\item 如果一个公式\(\phi\)(\(Q\)下)等价于\(\exists x_1\dots\exists x_n\theta\),其中\(\theta\)是\(\Delta_0\)的,则称\(\phi\)是\(\Sigma_1\)的公式
\item 如果一个公式\(\phi\)等价于\(\forall x_1\dots\forall x_n\theta\),其中\(\theta\)是\(\Delta_0\)的,则\(\phi\)是\(\Pi_1\)公式
\item 如果一个公式既等价于\(\Sigma_1\),又等价于\(\Pi_1\),则它是\(\Delta_1\)的
\end{itemize}


\begin{theorem}[\(\Sigma_1\)-完备性]
对任何一个\(\Sigma_1\)-闭语句,我们有
\begin{equation*}
\fN\vDash\tau\Leftrightarrow Q\vdash\tau
\end{equation*}
\end{theorem}

\begin{proof}
\(\Rightarrow\): 对任何\(\Delta_0\)-闭语句\(\sigma\),对任何\(\fM\vDash Q\),有
\begin{equation*}
\fM\vDash\sigma\Leftrightarrow\fN\vDash\sigma
\end{equation*}

设\(\sigma\)为\((\forall x\le t)\psi\)且\(\psi\)是一个\(\Delta_0\)公式,\(t\)是一个闭项,于是存在\(n\in\N\)使得\(Q\vdash t=n\)。
如果\(\fM\vDash(\forall x\le t)\psi(x,t)\),则\(\fM\vDash(\forall x\le n)\psi(x,n)\)

若\(\sigma:=\exists\barx\psi(\barx)\),\(\psi\)是\(\Delta_0\)公式,设\(\fN\vDash\sigma\),则存在\(\barm\in\N^n\)使得\(\fN\vDash\psi(\barm)\)。
\(Q\vdash\psi(\barm)\Rightarrow Q\vdash\exists\barx\psi(\barx)\)
\end{proof}

\begin{definition}[]
称一个函数\(f:\N^k\to\N\)在\(T\)中可表示,如果存在\(L_{ar}\)公式\(\phi(x_1,\dots,x_k,y)\)使得对所
有\((n_1,\dots,n_k)\in\N^k\)有
\begin{equation*}
T\vdash\forall y\left( \phi(n_1,\dots,n_k,y)\leftrightarrow y=f(n_1,\dots,n_k) \right)
\end{equation*}
此时称\(\phi\)作为一个函数表示\(f\)
\end{definition}

表示函数图象的公式不能表示函数

设\(f\)是一个函数,\(G_f=\{(\barx,y)\mid y=f(\barx)\}\),设\(\phi(\barx,y)\)表示\(G_f\),于是对任
意\(\bara\in\N^k\),\(b\in\N\),有
\begin{equation*}
f(a)=b\Rightarrow T\vdash\phi(\bara,b),\quad f(a)\neq b\Rightarrow T\vdash\neg\phi(\bara,b)
\end{equation*}
若\(\fM\vDash T\),对于非标准\(y\in M\setminus\N\),
\begin{equation*}
\fM\vDash y\neq f(\bara)\to\neg\phi(\bara,y)
\end{equation*}
不一定成立

例如\(Z(x)=0\),于是\(\phi(x,y):=y+y=y\)表示\(G_Z\),\(Q\)不能证明\(y+y=y\to y=0\),考虑\(\N\cup\{\infty\}\)

\begin{lemma}[]
令\(t\)为\(L_{ar}\)的项,则\(t\)诱导出来的函数是可表示的
\end{lemma}

\begin{theorem}[]
可表示函数类关于复合封闭
\end{theorem}

\begin{lemma}[]
可表示函数类关于极小算子封闭
\begin{enumerate}
\item 设\(P\subseteq\N^{k+1}\)被\(\alpha(\barx,y)\)表示
\item 令\(\phi(\barx,y)\)味\(\alpha(\barx,y)\wedge(\forall z<y)\neg\alpha(\barx,z)\)
\item 则\(f:\bara\mapsto\mu b[P(\bara,b)]\)被\(\phi(\barx,y)\)表示
\end{enumerate}
\end{lemma}

\begin{corollary}[]
函数\(f\)可表示当且仅当\(G_f\)可表示
\end{corollary}

\begin{proof}
\(a\mapsto\mu b[G_f(a,b)]\)是可表示函数,它是\(f\)自身
\end{proof}

\begin{corollary}[]
加定函数\(g(x,y)\)可表示,则函数
\begin{equation*}
f(x):=\mu y(g(x,y)=0)
\end{equation*}
也可表示
\end{corollary}

目标:可表示函数类关于原始递归封闭,


\begin{definition}[]
哥德尔函数\(\beta:\N^3\to\N^{<\omega}\)定义为:对任意\(u,v,w\),\(\beta(u,v,w)\)是一个长度为\(w\)的序
列\(a_0,\dots,a_{w-1}\),其中
\begin{equation*}
u=d((i+1)v+1)+a_i
\end{equation*}
\end{definition}

定义\(\alpha(u,v,i)\)为\(\frac{u}{v(i+1)+1}\)的余数,即\(\beta\)函数的坐标分量函数,则\(\alpha(u,v,i)\)是可表
示的

\begin{theorem}[]
\(\beta\)是满射
\end{theorem}

\begin{lemma}[欧几里得引理]
设\(a,b\in\N\)互素,则存在\(x,y\in\Z\)使得\(ax+by=1\)
\end{lemma}

\begin{proof}
令\(X=\{ax+by\mid x,y\in\Z\}\cap\N\),则\(X\)有最小元\(x_0\),若\(x_0\)不能整除\(a\),则\(a=cx_0+r\),因
此\(x_0\)是最小公倍数

consider \(a\N\) and \(b\N\), then \(a\N\cap b\N=c\N\) for some \(c\in\N\) since \(\N\) is a PID
\end{proof}

\begin{theorem}[中国剩余定理]
设\(d_0,\dots,d_n\)是两两互素的自然数,\(a_0,\dots,a_n\)为满足\(a_i<d_i\)的自然数,则存在\(c\in\N\)使
得\(c\equiv a_i\mod d_i\) for all \(i\)
\end{theorem}

\begin{lemma}[]
对任意\(n\),存在\(n+1\)个数两两互素
\begin{equation*}
1+n!,1+2\cdot n!,\dots,1+(n+1)\cdot n!
\end{equation*}
\end{lemma}

\begin{theorem}[]
\(\beta\)是满射
\end{theorem}

\begin{proof}
设\(a_0,\dots,a_{w-1}\)是一个自然数的序列,令\(n=\max\{a_0,\dots,a_{w-1},w\}\),令\(v=n!\),
则\(\{v(i+1)+1\mid i=0,\dots,w-1\}\)两两互素且\(a_i<v(i+1)+1\),根据中国剩余定理,存在\(u\in\N\)使得
\(u\equiv_{v(i+1)+1}a_i\)
\end{proof}

\begin{lemma}[]
\(\alpha(u,v,i)=\frac{u}{v(i+1)+1}\)的余数是可表示的
\end{lemma}

\begin{proof}
\(P(c,d,i,r,q):=(c=q(1+(i+1)d)+r)\)可表示

\(R(c,d,i,r):=\exists q\le c\;P(c,d,i,r,q)\)可表示:

\(\mu r(R(c,d,i,r))\)可表示
\end{proof}

\begin{theorem}[]
递归函数都是可表示的
\end{theorem}

\begin{proof}
只需证明\(\fN\)中的可表示函数类包含初始函数,且对复合、极小化、原始递归封闭
\end{proof}



\section{作业}
\label{sec:org2020d74}
7.5.3 (2)/

7.5.5 \(h[A]\)递归吗

7.1 7.2 7.3

\begin{exercise}
\label{ex8.1.1}
令\(\Sigma_1\)和\(\Sigma_2\)两个语句集,并且没有模型能同时满足\(\Sigma_1\)和\(\Sigma_2\).证明存在一个语句\(\sigma\)使
得\(\Mod\Sigma_1\subseteq\Mod\tau\)并且\(\Mod\Sigma_2\subseteq\Mod\neg\tau\)
\end{exercise}

\begin{proof}
\(\Mod\Sigma_1\cap\Mod\Sigma_2=\emptyset\). \(\Mod\Sigma_1\subseteq\Mod\tau\Leftrightarrow\Sigma_1\vDash\tau\)

suppose for all \(\tau\), \(\Sigma_1\not\vDash\tau\) or \(\Sigma_2\not\vDash\tau\)

Then for all \(\tau\in\Sigma_1\), \(\Sigma_2\not\vDash\neg\tau\) and hence \(\Sigma_2\cup\{\tau\}\) is satisfiable. Thus \(\Sigma_1\cup\Sigma_2\) is
satisfiable, a contradiction
\end{proof}

\begin{exercise}
\label{8.1.2}
\(\vdash_{\PA}x<y\leftrightarrow Sx\le y\) and \(\vdash_{\PA}x\le y\vee y\le x\)
\end{exercise}

\begin{proof}
\(x<y\Leftrightarrow\exists z(\neg z\approx 0\wedge x+z=y)\).
\(\neg z\approx 0\Leftrightarrow \exists m(z\approx S(m))\).
\(x+S(m)\approx S(x+m)\approx S(x)+m\approx y\).
\end{proof}

\begin{exercise}
\label{8.2.3}
证明有端点的稠密线序理论\(\Th(\Q\cap[0,1),<)\),\(\Th(\Q\cap[0,1],<)\),\(\Th(\Q\cap(0,1],<)\)都分别
是\(\aleph_0\)-categorical,因而是完全的。再验证它们和\(\Th(\Q,<)\)是稠密线序理论仅有的四个完全扩张
\end{exercise}

\begin{exercise}
\label{8.2.4}
\(\ACF_0\) is not finitely axiomatizable
\end{exercise}

\begin{proof}
\href{https://math.stackexchange.com/questions/3453682/fields-with-characteristic-0-are-not-finitely-axiomatizable-in-fopc}{proof}
\end{proof}

\begin{exercise}
\label{8.3.2}
证明:理论\(T_S\)被下列公理公理化:(S1) (S2)加上对语言\(\call_S=\{0,S\}\)的归纳公理模式
\begin{equation*}
[\varphi(0)\wedge\forall x(\varphi(x)\to\varphi(Sx))]\to\forall x\varphi(x)
\end{equation*}
其中\(\varphi\)是任意的语言\(\call_S\)上的公式
\end{exercise}

\begin{exercise}
\label{8.3.3}
\(T_S\)不能被有穷公理化
\end{exercise}

\begin{proof}
如果\(T_S\)能被有穷公理化
\end{proof}
\end{document}
