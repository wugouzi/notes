% Created 2020-06-30 二 00:44
% Intended LaTeX compiler: pdflatex
\documentclass[11pt]{article}
\usepackage[utf8]{inputenc}
\usepackage[T1]{fontenc}
\usepackage{graphicx}
\usepackage{grffile}
\usepackage{longtable}
\usepackage{wrapfig}
\usepackage{rotating}
\usepackage[normalem]{ulem}
\usepackage{amsmath}
\usepackage{textcomp}
\usepackage{amssymb}
\usepackage{capt-of}
\usepackage{hyperref}
\usepackage{minted}
% TIPS
% \substack{a\\b} for multiple lines text





% pdfplots will load xolor automatically without option
\usepackage[dvipsnames]{xcolor}

\usepackage{forest}
% two-line text in node by [two \\ lines]
% \begin{forest} qtree, [..] \end{forest}
\forestset{
  qtree/.style={
    baseline,
    for tree={
      parent anchor=south,
      child anchor=north,
      align=center,
      inner sep=1pt,
    }}}
%\usepackage{flexisym}
% load order of mathtools and mathabx, otherwise conflict overbrace

\usepackage{mathtools}
%\usepackage{fourier}
\usepackage{pgfplots}
\usepackage{amsthm, mathabx,  amsmath, commath}
\usepackage{amsfonts}

\usepackage{empheq}
\usepackage{tikz}
\usetikzlibrary{arrows.meta}
\usepackage[most]{tcolorbox}

\newtheorem{theorem}{Theorem}[section]
\newtheorem{definition}{Definition}[section]
\newtheorem{corollary}{Corollary}[section]
\newtheorem{example}{Example}[section]
\newtheorem{lemma}{Lemma}[section]
\newtheorem{proposition}{Proposition}[section]

\newcommand{\bl}[1] {\boldsymbol{#1}}
\newcommand{\Wt}[1] {\stackrel{\sim}{\smash{#1}\rule{0pt}{1.1ex}}}
\newcommand{\wt}[1] {\widetilde{#1}}


%For boxed texts in align, use Aboxed{}
%otherwise use boxed{}

\DeclareMathSymbol{\widehatsym}{\mathord}{largesymbols}{"62}
\newcommand\lowerwidehatsym{%
  \text{\smash{\raisebox{-1.3ex}{%
    $\widehatsym$}}}}
\newcommand\fixwidehat[1]{%
  \mathchoice
    {\accentset{\displaystyle\lowerwidehatsym}{#1}}
    {\accentset{\textstyle\lowerwidehatsym}{#1}}
    {\accentset{\scriptstyle\lowerwidehatsym}{#1}}
    {\accentset{\scriptscriptstyle\lowerwidehatsym}{#1}}
}

\usepackage{graphicx}
    
% text on arrow for xRightarrow
\makeatletter
%\newcommand{\xRightarrow}[2][]{\ext@arrow 0359\Rightarrowfill@{#1}{#2}}
\makeatother


\def \bx {\boldsymbol{x}}
\def \ba {\boldsymbol{a}}
\def \bI {\boldsymbol{I}}
\def \bt {\boldsymbol{t}}
\def \bb {\boldsymbol{b}}
\def \bA {\boldsymbol{A}}
\def \bX {\boldsymbol{X}}
\def \bu {\boldsymbol{u}}
\def \bS {\boldsymbol{S}}
\def \bZ {\boldsymbol{Z}}
\def \bz {\boldsymbol{z}}
\def \by {\boldsymbol{y}}
\def \bw {\boldsymbol{w}}
\def \bT {\boldsymbol{T}}
\def \bS {\boldsymbol{S}}
\def \bm {\boldsymbol{m}}
\def \bW {\boldsymbol{W}}
\def \bY {\boldsymbol{Y}}
\def \bH {\boldsymbol{H}}
\def \blambda {\boldsymbol{\lambda}}
\def \bPhi {\boldsymbol{\Phi}}
\def \btheta {\boldsymbol{\theta}}
\def \bmu {\boldsymbol{\mu}}
\def \bphi {\boldsymbol{\phi}}
\def \bSigma {\boldsymbol{\Sigma}}
\def \lb {\left\{}
\def \rb {\right\}}
\def \caln {\mathcal{N}}
\def \dissum {\displaystyle\Sigma}
\def \dispro {\displaystyle\prod}
\def \E {\mathbb{E}}
\def \Q {\mathbb{Q}}
\def \V {\mathbb{V}}
\def \R {\mathbb{R}}
\def \calq {\mathcal{Q}}
\def \calg {\mathcal{G}}
\def \caln {\mathcal{N}}
\def \calr {\mathcal{R}}
\def \calm {\mathcal{M}}
\def \calc {\mathcal{C}}
\def \bcup {\bigcup}

\setcounter{secnumdepth}{1}
\author{Atiyah \& Macdonald}
\date{\today}
\title{Introduction To Commutative Algebra}
\hypersetup{
 pdfauthor={Atiyah \& Macdonald},
 pdftitle={Introduction To Commutative Algebra},
 pdfkeywords={},
 pdfsubject={},
 pdfcreator={Emacs 26.3 (Org mode 9.4)}, 
 pdflang={English}}
\begin{document}

\maketitle
\tableofcontents \clearpage\section{Rings and Ideals}
\label{sec:orgf21a49f}
A \textbf{unit} is an element \(u\) with a \textbf{reciprocal} \(1/u\) or the
\textbf{multiplicative inverse}. The units form a multiplicative group, denoted
\(R^\times\)

A ring \textbf{homomorphism}, or simply a \textbf{ring map}, \(\varphi:R\to R'\) is a map
preserving sum, products and 1

If there is an unspecified isomorphism between rings \(R\) and \(R'\), then we
write \(R=R'\) when it is \textbf{canonical}; that is, it does not depend on any
artificial choices.

A subset \(R''\subset R\) is a \textbf{subring} if \(R''\) is a ring and the
inclusion \(R''\hookrightarrow R\) is a ring map. In this case, we call \(R\)
a \textbf{(ring) extension}.

An \textbf{\(R\)-algebra} is a ring \(R'\) that comes equipped with a ring map
\(\varphi:R\to R'\), called the \textbf{structure map}, denoted by \(R'/R\). For
example, every ring is canonically a \(\Z\)-algebra. An
\textbf{\(R\)-algebra homomorphism}, or \textbf{\(R\)-map}, \(R'\to R''\) is a ring map
between \(R\)-algebras.

A group \(G\) is said to \textbf{act} on \(R\) if there is a homomorphism given from
\(G\) into the group of automorphism of \(R\). The \textbf{ring of invariants}
\(R^G\) is the subring defined by
\begin{equation*}
R^G:=\{x\in R\mid gx=g\text{ for all }g\in G\}
\end{equation*}

Similarly a group \(G\) is said to \textbf{act} on \(R'/R\) if \(G\) acts on \(R'\)
and each \(g\in G\) is an \(R\)-map. Note that \(R'^G\) is an \(R\)-subalgebra

\subsection*{Boolean rings \label{sec1.2}}
\label{sec:org60899b9}
The simplest nonzero ring has two elements, 0 and 1. It's denoted \(\F_2\)


Given any ring \(R\) and any set \(X\), let \(R^X\) denote the set of
functions \(f:X\to R\). Then \(R^X\) is a ring.

For example, take \(R:=\F_2\). Given \(f:X\to R\), put \(S:=f^{-1}\{1\}\).
Then \(f(x)=1\) if \(x\in S\). In other words, \(f\) is the \textbf{characteristic
function} \(\chi_S\). Thus \emph{the characteristic functions form a ring, namely}, \(\F_2^X\)

Given \(T\subset X\), clearly \(\chi_S\cdot\chi_T=\chi_{S\cap T}\).
\(\chi_S+\chi_T=\chi_{S\triangle T}\), where \(S\triangle T\) is the
\textbf{symmetric difference}:
\begin{equation*}
S\triangle T:=(S\cup T)-(S\cap T)
\end{equation*}
Thus \emph{the subsets of \(X\) form a ring: sum is symmetric difference, and}
\emph{product is intersection. This ring is canonically isomorphic to \(\F_2^X\)}

A ring \(B\) is called \textbf{Boolean} if \(f^2=f\) for all \(f\in B\). If so, then
\(2f=0\) as \(2f=(f+f)^2=f^2+2f+f^2=4f\)

Suppose \(X\) is a topological space, and give \(\F_2\) the \textbf{discrete}
topology; that is, every subset is both open and closed. Consider the
continuous functions \(f:X\to\F_2\). Clearly, they are just the \(\chi_S\)
where \(S\) is both open and closed.

\subsection*{Polynomial rings}
\label{sec:org804b424}
Let \(R\) be a ring, \(P:=R[X_1,\dots,X_n]\). \(P\) has this \textbf{Universal
Mapping Property} (UMP): \emph{given a ring map \(\varphi:R\to R'\) and given an}
\emph{element \(x_i\) of \(R'\) for each \(i\), there is a unique ring map}
\emph{\(\pi:P\to R'\) with \(\pi|R=\varphi\) and \(\pi(X_i)=x_i\).} In fact, since
\(\pi\) is a ring map, necessarily \(\pi\) is given by the formula:
\begin{equation}
\pi(\sum a_{(i_1,\dots,i_n)}X_1^{i_1}\dots X_n^{i_n})=\sum
\varphi(a_{(i_1,\dots,i_n)})x_1^{i_1}\dots x_n^{i_n}\label{eq1.3.1}
\end{equation}
In other words, \(P\) is universal among \(R\)-algebras equipped with a list
of \(n\) elements

Similarly let \(\calx:=\{X_\lambda\}_{\lambda\in\Lambda}\) be any set of
variables. Set \(P':=R[\calx]\); the elements of \(P'\) are the polynomials
in any finitely many of the \(X_\lambda\). \(P'\) has essentially the same
UMP as \(P\)

\subsection*{Ideals}
\label{sec:orgf2e60d5}
Let \(R\) be a ring. A subset \(\fa\) is called an \textbf{ideal} if
\begin{enumerate}
\item \(0\in\fa\)
\item whenever \(a,b\in\fa\), also \(a+b\in\fa\)
\item whenever \(x\in R\) and \(a\in\fa\) also \(xa\in\fa\)
\end{enumerate}


Given a subset \(\fa\subset R\), by the ideal \(\la\fa\ra\) that \(\fa\)
\textbf{generates}, we mean the smallest ideal containing \(\fa\)

All ideal containing all the \(a_\lambda\) contains any (finite) \textbf{linear
combination} \(\sum x_\lambda a_\lambda\) with \(x_\lambda\in R\) and almost
all 0.

Given a single element \(a\), we say that the ideal \(\la a\ra\) is
\textbf{principal}

Given a number of ideals \(\fa_\lambda\), by their \textbf{sum} \(\sum\fa_\lambda\)
we mean the set of all finite linear combinations \(\sum x_\lambda
   a_\lambda\) with \(x_\lambda\in R\) and \(a_\lambda\in\fa_\lambda\)

Given two ideals \(\fa\) and \(\fb\), by the \textbf{transporter} of \(\fb\) into
\(\fa\) we mean the set
\begin{equation*}
(\fa:\fb):=\{x\in R\mid x\fb\subset\fa\}
\end{equation*}
\((\fa:\fb)\)  is an ideal. Plainly,
\begin{equation*}
\fa\fb\subset\fa\cap\fb\subset\fa+\fb,\quad\fa,\fb\subset\fa+\fb,\quad
\fa\subset(\fa:\fb)
\end{equation*}
Further, for any ideal \(\fc\), the distributive law holds:
\(\fa(\fb+\fc)=\fa\fb+\fa\fc\)

Given an ideal \(fa\), notice \(\fa=R\) \emph{if and only if} \(1\in\fa\). It
follows that \(\fa=R\) iff \(\fa\) contains a unit.

Given a ring map \(\varphi:R\to R'\), denote by \(\fa R'\) or \(\fa^e\) the
ideal of \(R'\) generated by the set \(\varphi(\fa)\). We call it the
\textbf{extension} of \(\fa\)

Given an ideal \(\fa'\) of \(R'\), its preimage \(\varphi^{-1}(\fa')\) is an
ideal of \(R\). We call \(\varphi^{-1}(\fa')\) the \textbf{contraction} of \(\fa'\)
and sometimes denote it by \(\fa'^c\)

\subsection*{Residue rings}
\label{sec:org6c8fe09}
\textbf{kernel} \(\ker(\varphi)\) is defined to be the ideal \(\varphi^{-1}(0)\) of
\(R\)

Let \(\fa\) be an ideal of \(R\). Form the set of cosets of \(\fa\)
\begin{equation*}
R/\fa:=\{x+\fa\mid x\in R\}
\end{equation*}
\(R/\fa\) is called the \textbf{residure ring} or \textbf{quotient ring} or \textbf{factor ring} of
\(R\) \textbf{modulo} \(\fa\). From the \textbf{quotient map}
\begin{equation*}
\kappa:R\to R/\fa\quad\text{ by }\kappa x:=x+\fa
\end{equation*}
The element \(\kappa x\in R/\fa\) is called the \textbf{residue} of \(x\).

If \(\ker(\varphi)\supset\fa\), \emph{then there is a ring map} \(\psi:R/\fa\to R'\)
\emph{with} \(\psi\kappa=\varphi\); that is, the following diagram is commutative

\begin{center}
\begin{tikzcd}
R\arrow[r,"\kappa"]\arrow[dr,"\varphi"]&R/\fa\arrow[d,"\psi"]\\
&R'
\end{tikzcd}
\end{center}
by \(\psi(x\fa)=\varphi(x)\). Then we only need to verify that \(\psi\) is a map

Conversely, \emph{if \(\psi\) exists, then} \(\ker(\varphi)\supset\fa\), \emph{or}
\(\varphi\fa=0\), \emph{or} \(\fa R'=0\), since \(\kappa\fa=0\)

Further, \emph{if \(\psi\) exists, then \(\psi\) is unique} as \(\kappa\) is surjective

Finally, as \(\kappa\) is surjective, \emph{if \(\psi\) exists, then \(\psi\) is surjective}
\emph{iff \(\psi\) is so}. In addition, \(\psi\) \emph{is injective iff \(\fa=\ker(\varphi)\)}.
Hence \(\psi\) \emph{is an isomorphism iff \(\varphi\) is surjective and}
\(\fa=\ker(\varphi)\). Therefore,
\begin{equation*}
R/\ker(\varphi)\xrightarrow{\sim}\im(\varphi)
\end{equation*}

\(R/\fa\) has UMP: \(\kappa(\fa)=0\), and given \(\varphi:R\to R'\) s.t.
\(\varphi:R\to R'\) s.t. \(\varphi(\fa)=0\), there is a unique ring map
\(\psi:R/\fa\to R'\) s.t. \(\psi\kappa=\varphi\). In other words, \(R/\fa\)
is universal among \(R\)-algebras \(R'\) s.t. \(\fa R'=0\)

If \(\fa\) is the ideal generated by elements \(a_\lambda\),then the UMP can
be usefully rephrased as follows: \(\kappa(a_\lambda)=0\) for all \(\lambda\),
and given \(\varphi:R\to R'\) s.t. \(\varphi(a_\lambda)=0\) for all \(\lambda\),
there is a unique ring map \(\psi:R/\fa\to R'\) s.t. \(\psi\kappa=\varphi\)

\emph{The UMP serves to determine} \(R/\fa\) \emph{up to unique isomorphism}.
Say \(R'\), equipped with \(\varphi:R\to R'\) has the UMP too.
\(\kappa(\fa)=0\) so there is a unique \(\psi':R'\to R/\fa\) with
\(\psi'\varphi=\kappa\). Then \(\psi'\psi\kappa=\kappa\). Hence
\(\psi'\psi=1\) by uniqueness. Thus \(\psi\) and \(\psi'\) are inverse isomorphism
\begin{center}
\begin{tikzcd}
&&R/\fa\arrow[dd,"1"]\arrow[dl,"\psi"]\\
R\arrow[urr,"\kappa"]\arrow[r,"\varphi"]\arrow[drr,"\kappa"]&
R'\arrow[dr,"\psi'"]&\\
&&R/\fa
\end{tikzcd}
\end{center}

\begin{proposition}[]
Let \(R\) be a ring, \(P:=R[X]\), \(a\in R\) and \(\pi:P\to R\) the
\(R\)-algebra map defined by \(\pi(X):=a\). Then
\begin{enumerate}
\item \(\ker(\pi)=\{F(X)\in P\mid F(a)=0\}=\la X-a\ra\)
\item \(R/\la X-a\ra\simeq R\)
\end{enumerate}
\end{proposition}

\begin{proof}
Set \(G:=X-a\). Given \(F\in P\), let's show \(F=GH+r\) with \(H\in P\) and
\(r\in R\). By linearity, we may assume \(F:=X^n\). If \(n\ge1\), then
\(F=(G+a)X^{n-1}\), so \(F=GH+aX^{n-1}\) with \(H:=X^{n-1}\).

Then \(\pi(F)=\pi(G)\pi(H)+\pi(r)=r\). Hence \(F\in\ker(\pi)\) iff \(F=GH\). But
\(\pi(F)=F(a)\) by \ref{eq1.3.1}
\end{proof}


\subsection*{Degree of a polynomial}
\label{sec:org020dd71}
Let \(R\) be a ring, \(P\) the polynomial ring in any number of variables.
If \(F\) is a monomial \(\bM\), then its degree \(\deg(\bM)\) is the sum of
its exponents; in general, \(\deg(F)\) is the largest \(\deg(\bM)\) of all
monomials \(\bM\) in \(F\)

Given any \(G\in P\) with \(FG\) nonzero, notice that
\begin{equation*}
\deg(FG)\le\deg(F)+\deg(G)
\end{equation*}

\subsection*{Order of a polynomial}
\label{sec:orgd9fbe77}
Let \(R\) be a ring, \(P\) the polynomial ring in variable \(X_\lambda\) for
\(\lambda\in\Lambda\), and \((x_\lambda)\in R^\Lambda\) a vector. Let
\(\varphi_{(x_\lambda)}:P\to P\) denote the \(R\)-algebra map defined by
\(\varphi_{(x_\lambda)}X_\mu:=X_\mu+x_\mu\) for all \(\mu\in\Lambda\). Fix a
nonzero \(F\in P\)

The \textbf{order} of \(F\) at the zero vector \((0)\), denoted \(\ord_{(0)}F\), is
defined as the smallest \(\deg(\bM)\) of all the monomials \(\bM\) in \(F\).
In general, the \textbf{order} of \(F\) at the vector \((x_\lambda)\), denoted
\(\ord_{(x_\lambda)}F\) is defined by the formula: \(\ord_{(x_\lambda)}F:=\ord_{(0)}(\varphi_{(x_\lambda)}F)\)

Notice that \(\ord_{(x_\lambda)}F=0\) iff \(F(x_\lambda)\neq0\) as \((\varphi_{x_\lambda}F)(0)=F(x_\lambda)\)

Given \(\mu\) and \(x\in R\), form \(F_{\mu,x}\) by substituting \(x\) for \(X_\mu\)
in \(F\). If \(F_{\mu,x_\mu}\neq0\) , then
\begin{equation*}
\ord_{(x_\lambda)}F\le\ord_{(x_\lambda)}F_{\mu,x_\mu}\label{eq1.8.1}
\end{equation*}
If \(x_\mu=0\), then \(F_{\mu,x_\mu}\) is the sum of the terms without
\(x_\mu\) in \(F\). Hence if \((x_\lambda)=(0)\), then \ref{eq1.8.1} holds. But
substituting 0 for \(X_\mu\) in \(\varphi_{(x_\lambda)}F\) is the same as
substituting \(x_\mu\) for \(X_\mu\) in \(F\) and then applying
\(\varphi_{(x_\lambda)}\) to the result; that is,
\((\varphi_{(x_\mu)}F)_{\mu,0}=\varphi_{(x_\lambda)}F_{\mu,x_\mu}\)

Given any \(G\in P\) with \(FG\) nonzero,
\begin{equation*}
\ord_{(x_\lambda)}FG\ge\ord_{(x_\lambda)}F+\ord_{(x_\lambda)}G
\end{equation*}

\subsection*{Nested ideals \label{1.9}}
\label{sec:orgee18195}
Let \(R\) be a ring, \(\fa\) an ideal, and \(\kappa:R\to R/\fa\) the quotient map.
Given an ideal \(\fb\supset\fa\), form the corresponding set of cosets of
\(\fa\)
\begin{equation*}
\fb/\fa:=\{b+\fa\mid b\in\fb\}=\kappa(\fb)
\end{equation*}
Clearly, \(\fb/\fa\) is an ideal of \(R/\fa\). Also \(\fb/\fa=\fb(R/\fa)\)

\emph{The operation} \(\fb\mapsto\fb/\fa\) \emph{and} \(\fb'\mapsto\kappa^{-1}(\fb')\) \emph{are}
\emph{inverse to each other, and establish a bijective correspondence between the}
\emph{set of ideals \(\fb\) of \(R\) containing \(\fa\) and the set of all ideals}
\(\fb'\) \emph{of} \(R/\fa\). \emph{Moreover, this correspondence preserves inclusions}

Given an ideal \(\fb\supset\fa\), form the composition of the quotient maps
\begin{equation*}
\varphi:R\to R/\fa\to (R/\fa)/(\fb/\fa)
\end{equation*}
\(\varphi\) is surjective and \(\ker(\varphi)=\fb\). Hence \(\varphi\) factors
\begin{center}
\begin{tikzcd}
R\arrow[r]\arrow[d]&R/\fb\arrow[d,"\psi","\simeq"']\\
R/\fa\arrow[r]&(R/\fa)/(\fb/\fa)
\end{tikzcd}
\end{center}

\subsection*{Idempotents}
\label{sec:orgfde77ce}
Let \(R\) be a ring. Let \(e\in R\) be an \textbf{idempotent}; that is, \(e^2=e\).
Then \(Re\) is a ring with \(e\) as 1.

Set \(e':=1-e\). Then \(e'\) is idempotent and \(e\cdot e'=0\). We call \(e\)
and \(e'\)  \textbf{complementary idempotents}. Conversely, if two elements
\(e_1,e_2\in R\) satisfy \(e_1+e_2=1\) and \(e_1e_2=0\), then they are
complementary idempotents, as for each \(i\),
\begin{equation*}
e_i=e_i\cdot 1=e_i(e_1+e_2)=e_i^2
\end{equation*}
We denote the set of all idempotents by \(\Idem(R)\). Let \(\varphi:R\to R'\)
be a ring map. Then \(\varphi(e)\) is idempotent. So the restriction of
\(\varphi\) to \(\Idem(R)\) is a map
\begin{equation*}
\Idem(\varphi):\Idem(R)\to\Idem(R')
\end{equation*}
\begin{examplle}[]
Let \(R:=R'\times R''\) be a \textbf{product} of two rings. Set \(e':=(1,0)\) and
\(e'':=(0,1)\). Then \(e'\) and \(e''\) are complementary idempotents.
\end{examplle}

\begin{proposition}[]
Let \(R\) be a ring, and \(e',e''\) complementary idempotents. Set
\(R':=Re'\) and \(R'':=Re''\). Define \(\varphi:R\to R'\times R''\) by
\(\varphi(x):=(xe',xe'')\). Then \(\varphi\) is a ring isomorphism. Moreover, \(R'=R/Re''\)
and \(R''=R/Re'\)
\end{proposition}

\begin{proof}
Define a surjection \(\varphi':R\to R'\) by \(\varphi'(x):=xe'\). Then
\(\varphi'\) is a ring map, since \(xye'=xye'^2=(xe')(ye')\). Moreover,
\(\ker(\varphi')=Re''\) since \(x=x\cdot 1=xe'+xe''=xe''\). Thus
\(R'=R/Re''\)

Since \(\varphi\) is a ring map. It's surjective since \((xe',x'e'')=\varphi(xe'+x'e'')\)
\end{proof}

\subsection*{Exercise}
\label{sec:org61d7bb1}
\begin{exercise}
\label{ex1.13}
Let \(\varphi:R\to R'\) be a map of rings, \(\fa_1,\fa_2,\fa\) ideals of \(R\),
\(\fb_1,\fb_2,\fb\) ideals of \(R'\). Prove
\begin{enumerate}
\item \((\fa_1+\fa_2)^e=\fa_1^e+\fa_2^e\)
\item \((\fb_1+\fb_2)^c\supset\fb_1^c+\fb_2^c\)
\item \((\fa_1\cap\fa_2)^e\subset\fa_1^e\cap\fa_2^e\)
\item \((\fb_1\cap\fb_2)^c=\fb_1^c\cap\fb_2^c\)
\item \((\fa_1\fa_2)^e=\fa_1^e\fa_2^e\)
\item \((\fb_1\fb_2)^c\supset\fb_1^c\fb_2^c\)
\item \((\fa_1:\fa_2)^e\subset(\fa_1^e:\fa_2^e)\)
\item \((\fb_1:\fb_2)^c\subset(\fb_1^c:\fb_2^c)\)
\end{enumerate}
\end{exercise}

\begin{exercise}
\label{ex1.14}
Let \(\varphi:R\to R'\) be a map of rings, \(\fa\) an ideal of \(R\), and \(\fb\)
an ideal of \(R'\). Prove the following statements:
\begin{enumerate}
\item \(\fa^{ec}\supset\fa\) and \(\fb^{ce}\subset\fb\)
\item \(\fa^{ece}=\fa^e\) and \(\fb^{cec}=\fb^c\)
\item If \(\fb\) is an extension, then \(\fb^c\) is the largest ideal of \(R\)
with extension \(\fb\)
\item If two extensions have the same contraction, then they are equal
\end{enumerate}
\end{exercise}

\begin{exercise}
\label{1.15}
Let \(R\) be a ring, \(\fa\) an ideal, \(\calx\) a set of variables. Prove:
\begin{enumerate}
\item The extension \(\fa(R[\calx])\) is the set \(\fa[\calx]\)
\item \(\fa(R[\calx])\cap R=\fa\)
\end{enumerate}
\end{exercise}

\begin{exercise}
\label{1.16}
Let \(R\) be a ring, \(\fa\) an ideal, and \(\calx\) a set of variables. Set
\(P:=R[\calx]\). Prove \(P/\fa P=(R/\fa)[\calx]\)
\end{exercise}

\begin{exercise}
\label{1.17}
Let \(R\) be a ring, \(P:=R[\{X_\lambda\}]\) the polynomial ring in variables
\(X_\lambda\) for \(\lambda\in\Lambda\) a vector. Let
\(\pi_{(x_\lambda)}:P\to R\) denote the \(R\)-algebra map defined by
\(\pi_{(x_\lambda)}X_\mu:=x_\mu\) for all \(\mu\in\Lambda\). Show:
\begin{enumerate}
\item Any \(F\in P\) has the form \(F=\sum
      a_{(i_1,\dots,i_n)}(X_{\lambda_1}^{i_1}-x_{\lambda_1})\dots
      (X_{\lambda_n}-x_{\lambda_n})^{i_n}\) for unique \(a_{(i_1,\dots,i_n)}\in R\)
\item \(\ker(\pi_{(x_\lambda)})=\{F\in P\mid F((x_\lambda))=0\}=\la\{X_\lambda-x_\lambda\}\ra\)
\item \(\pi\) induces an isomorphism \(P/\la\{X_\lambda-x_\lambda\}\ra\simeq R\)
\item Given \(F\in P\), its residue in \(P/\la\{X_\lambda-x_\lambda\}\ra\) is
equal to \(F((x_\lambda))\)
\item Let \(\caly\) be a second set of variables. Then
\(P[\caly]/\la\{X_\lambda-x_\lambda\}\ra\simeq R[\caly]\)
\end{enumerate}
\end{exercise}

\begin{proof}
\begin{enumerate}
\item Let \(\varphi_{(x_\lambda)}\) be the \(R\)-automorphism of \(P\). Say
\(\varphi_{(x_\lambda)}F=\sum a_{(i_1,\dots,i_n)}X_{\lambda_1}^{i_1}\dots
      X_{\lambda_n}^{i_n}\) . And \(\varphi_{(x_\lambda)}^{-1}\varphi_{(x_\lambda)}F=F\)
\item Note that \(\pi_{(x_\lambda)}F=F((x_\lambda))\). Hence
\(F\in\ker(\pi_{(x_\lambda)})\) iff \(F((x_\lambda))=0\). If
\(F((x_\lambda))=0\), then \(a_{(0,\dots,0)}=0\), and so \(F\in\la\{X_\lambda-x_\lambda\}\ra\)
\setcounter{enumi}{4}
\item Set \(R':=R[\caly]\)
\end{enumerate}
\end{proof}

\begin{exercise}
\label{ex.1.18}
Let \(R\) be a ring, \(P:=R[X_1,\dots,X_n]\) the polynomial ring in variables
\(X_i\). Given \(F=\sum a_{(i_1,\dots,i_n)}X_1^{i_1}\dots X_n^{i_n}\in P\),
formally set
\begin{equation*}
\partial F/\partial X_j:=\sum i_ja_{(i_1,\dots,i_n)}
X_1^{i_i}\dots X_n^{i_n}/X_j\in P
\end{equation*}
Given \((x_1,\dots,x_n)\in R^n\), set \(\bx:=(x_1,\dots,x_n)\), set
\(a_j:=(\partial F/\partial X_j)(\bx)\), and set
\(\fM:=\la X_1-x_1,\dots,X_n-x_n\ra\). Show \(F=F(\bx)+\sum a_j(X_j-x_j)+G\)
with \(G\in\fM^2\). First show that if
\(F=(X_1-x_1)^{i_1}\dots(X_n-x_n)^{i_n}\), then \(\partial F/\partial X_j=i_jF/(X_j-x_j)\)
\end{exercise}

\begin{proof}
\((\partial F/\partial X_j)(\bx)=b_{(\delta_{1j},\dots,\delta_{nj})}\) where
\(\delta_{ij}\) is the Kronecker delta
\end{proof}

\begin{exercise}
\label{ex1.19}
Let \(R\) be a ring, \(X\) a variable, \(F\in P:=R[x]\), and \(a\in R\). Set
\(F':=\partial F/\partial X\). We call \(a\) a \textbf{root} of \(F\) if \(F(a)=0\), a
\textbf{simple root} if also \(F'(a)\neq0\), and a \textbf{supersimple root} if also \(F'(a)\)
is a unit.

Show that \(a\) is a root of \(F\) iff \(F=(X-a)G\) for some \(G\in P\), and
if so, then \(G\) is unique; that \(a\) is a simple root iff also
\(G(a)\neq0\); and that \(a\) is a supersimple root iff also \(G(a)\) is a unit
\end{exercise}

\begin{exercise}
\label{ex1.20}
Let \(R\) be a ring, \(P:=R[X_1,\dots,X_n]\), \(F\in P\) of degree \(d\) and
\(F_i:=X_i^{d_i}+a_1X_i^{d_i-1}+\dots\) a monic polynomial in \(X_i\) aloen
for all \(i\). Find \(G,G_i\in P\) s.t. \(F=\sum_{i=1}^nF_iG_i+G\) where
\(G_i=0\) or \(\deg(G_i)\le d-d_i\) and where the highest power of \(X_i\) in
\(G\) is less than \(d_i\)
\end{exercise}

\begin{proof}
By linearity, we may assume \(F:=X_1^{m_1}\dots X_n^{m_n}\). If \(m_i<d_i\)
for all \(i\), set \(G_i:=0\) and \(G:=F\) and we're done. Else, fix \(i\)
with \(m_i\ge d_i\), and set \(G_i:=F/X_i^{d_i}\) and \(G:=(-a_1X_i^{d_i-1}-\dots)G_i\)
\end{proof}

\begin{exercise}[Chinese Remainder Theorem]
\label{ex1.21}
Let \(R\) be a ring
\begin{enumerate}
\item Let \(\fa\) and \(\fb\) be \textbf{comaximal} ideals; that is, \(\fa+\fb=R\). Show
\begin{enumerate}
\item \(\fa\fb=\fa\cap\fb\)
\item \(R/\fa\fb=(R/\fa)\times(R/\fb)\)
\end{enumerate}
\item Let \(\fa\) be comaximal to both \(\fb\) and \(\fb'\). Show \(\fa\) is
also comaximal to \(\fb\fb'\)
\item Given \(m,n\ge1\), show \(\fa\) and \(\fb\) are comaximal iff \(\fa^m\)
and \(\fb^n\) are.
\item Let \(\fa_1,\dots,\fa_n\) be pairwise comaximal. Show
\begin{enumerate}
\item \(\fa_1\) and \(\fa_2\dots\fa_n\) are comaximal
\item \(\fa_1\cap\dots\cap\fa_n=\fa_1\dots\fa_n\)
\item \(R/(\fa_1\dots\fa_n)\simeq\prod(R/\fa_i)\)
\end{enumerate}
\item Find an example where \(\fa\) and \(\fb\) satisfy 1.1 but aren't comaximal
\end{enumerate}
\end{exercise}

\begin{proof}
\begin{enumerate}
\item \(\fa+\fb=R\) implies \(x+y=1\) with \(x\in\fa\) and \(y\in\fb\). So given
\(z\in\fa\cap\fb\), we have \(z=xz+yz\in\fa\fb\)
\item \(R=(\fa+\fb)(\fa+\fb')=(\fa^2+\fb\fa+\fa\fb')+\fb\fb'\subseteq\fa+\fb\fb'\subseteq
      R\)
\item Build with \(\fa+\fb^2=R\). Conversely, note that \(\fa^n\subset\fa\)
\item Induction
\item Let \(k\) be a field. Take \(R:=k[X,Y]\) and \(\fa:=\la X\ra\) and
\(\fb:=\la Y\ra\). Given \(f\in\la X\ra\cap\la Y\ra\), note that every
monomial of \(f\) contains both \(X\) and \(Y\), and so \(f\in\la X\ra\la
      Y\ra\). But \(\la X\ra\) and \(\la Y\ra\) are not comaximal
\end{enumerate}
\end{proof}

\begin{exercise}
\label{ex1.22}
First given a prime number \(p\) and a \(k\ge1\), find the idempotents in
\(\Z/\la p^k\ra\). Second, find the idempotents in \(\Z/\la12\ra\). Third,
find the number of idempotents in \(\Z/\la n\ra\) where
\(n=\prod_{i=1}^Np_i^{n_i}\) with \(p_i\) distinct prime numbers
\end{exercise}

\begin{proof}
\(x=0,1\)

Since \(-3+4=1\), the Chinese Remainder Theorem yields
\begin{equation*}
\Z/\la 12\ra=\Z/\la3\ra\times\Z/\la4\ra
\end{equation*}
\(m\) is idempotent in \(\Z/\la12\ra\) iff it's idempotent in \(\Z/\la3\ra\)
and \(\Z/\la4\ra\)

\(p_i^{n_i}\) has a linear combination equal to 1. Hence \(2^N\)
\end{proof}

\begin{exercise}
\label{ex1.23}
Let \(R:=R'\times R''\) be a product of rings, \(\fa\subset R\) an ideal.
Show \(\fa=\fa'\times\fa''\) with \(\fa'\subset R\) and \(\fa''\subset R''\)
ideals. Show \(R/\fa=(R'/\fa')\times(R''/\fa'')\)
\end{exercise}

\begin{exercise}
\label{ex1.24}
Let \(R\) be a ring; \(e,e'\) idempotents. Show
\begin{enumerate}
\item Set \(\fa:=\la e\ra\). Then \(\fa\) is idempotent; that is, \(\fa^2=\fa\)
\item Let \(\fa\) be a principal idempotent ideal. Then \(\fa=\la f\ra\) with
\(f\) idempotent
\item Set \(e'':=e+e'-ee'\). Then \(\la e,e'\ra=\la e''\ra\) and \(e''\) is
idempotent
\item Let \(e_1,\dots,e_r\) be idempotents. Then \(\la e_1,\dots,e_r\ra=\la
      f\ra\) with \(f\) idempotent
\item Assume \(R\) is Boolean. Then every finitely generated ideal is principal
\end{enumerate}
\end{exercise}

\begin{proof}
\begin{enumerate}
\setcounter{enumi}{2}
\item \(ee''=e^2=e\)
\end{enumerate}
\end{proof}

\begin{exercise}
\label{ex1.25}
Let \(L\) be a \textbf{lattice}, that is, a partially ordered set in which every pair
\(x,y\in L\) has a sup \(x\vee y\) and an inf \(x\wedge y\). Assume \(L\) is
\textbf{Boolean}; that is:
\begin{enumerate}
\item \(L\) has a least element 0 and a greatest element 1
\item The operations \(\vee\) and \(\wedge\) \textbf{distribute} over each other
\begin{equation*}
x\wedge(y\vee z)=(x\wedge y)\vee(x\wedge z)
\quad\text{ and }\quad
x\vee(y\wedge z)=(x\vee y)\wedge(x\vee z)
\end{equation*}
\item Each \(x\in L\) has a unique \textbf{complement} \(x'\); that is, \(x\wedge x'=0\)
and \(x\vee x'=1\) .
\end{enumerate}


Show that the following six laws obeyed
\begin{center}
\begin{tabular}{rclr}
\(x\wedge x=x\) & and & \(x\vee x=x\) & \textbf{(idempotent)}\\
\(x\wedge0=0,x\wedge1=x\) & and & \(x\vee1=1,x\vee0=x\) & \textbf{(unitary)}\\
\(x\wedge y=y\wedge x\) & and & \(x\vee y=y\vee x\) & \textbf{(commutative)}\\
\(x\wedge(y\wedge z)=(x \wedge y)\wedge z\) & and & \(x\vee(y\vee z)=(x\vee y)\vee z\) & \textbf{(associative)}\\
\(x''=x\) & and & \(0'=1,1'=0\) & \textbf{(involutory)}\\
\((x\wedge y)'=x'\vee y'\) & and & \((x\vee y)'=x'\wedge y'\) & \textbf{(De Morgan's)}\\
\end{tabular}
\end{center}

Moreover, show that \(x\le y\) iff \(x=x\wedge y\)
\end{exercise}

\begin{exercise}
\label{ex1.26}
Let \(L\) be a Boolean lattice. For all \(x,y\in L\), set
\begin{equation*}
x+y:=(x\wedge y')\vee(x'\wedge y)\quad\text{ and }\quad
xy:=x\wedge y
\end{equation*}
Show
\begin{enumerate}
\item \(x+y=(x\vee y)(x'\vee y')\)
\item \((x+y)'=(x'y')\vee(xy)\)
\item \(L\) is a Boolean ring
\end{enumerate}
\end{exercise}

\begin{exercise}
\label{ex1.27}
Given a Boolean ring \(R\), order \(R\) by \(x\le y\) if \(x=xy\). Show \(R\)
is thus a Boolean lattice. Viewing this construction as a map \(\rho\) from the set
of Boolean-ring structures on the set \(R\) to the set of Boolean-lattice
structures on \(R\), show \(\rho\) is bijective with inverse the map \(\lambda\) associated to
the construction in \ref{ex1.26}
\end{exercise}

\begin{proof}
First check \(R\) is partially ordered.

Given \(x,y\in R\), set \(x\vee y:=x+y+xy\) and \(x\wedge y:=xy\). Then
\(x\le x\vee y\) as \(x(x+y+xy)=x^2+xy+x^2y=x+2xy=x\). If \(z\le x\) and
\(z\le y\), then \(z=zx\) and \(z=zy\), and so \(z(x\vee y)=z\); thus \(z\le
   x\vee y\)
\end{proof}

\begin{exercise}
\label{ex1.28}
Let \(X\) be a set, and \(L\) the set of all subsets of \(X\), partially
ordered by inclusion. Show that \(L\) is a Boolean lattice and that the ring
structure on \(L\) constructed in \ref{sec1.2} coincides with that constructed
in \ref{ex1.26}

Assume \(X\) is a topological space, and let \(M\) be the set of all its open
and closed subsets. Show that \(M\) is a sublattice of \(L\), and that the
subring structure on \(M\) of \ref{sec1.2} coincides with the ring structure of
\ref{ex1.26} with \(M\) for \(L\)
\end{exercise}

\section{Prime Ideals}
\label{sec:org9ed5af3}

\subsection*{Zerodivisors}
\label{sec:org703271b}
Let \(R\) be a ring. An element \(x\) is called a \textbf{zerodivisor} if there is a
nonzero \(y\) with \(xy=0\); otherwise \(x\) is called a \textbf{nonzerodivisor}.
Denote the set of zerodivisors by \(\zdiv(R)\) and the set of nonzerodivisor
by \(S_0\)

\subsection*{Multiplicative subsets, prime ideals}
\label{sec:org3ec7164}
Let \(R\) be a ring. A subset \(S\) is called \textbf{multiplicative} if \(1\in S\)
and if \(x,y\in S\) implies \(xy\in S\)

An ideal \(\fp\) is called \textbf{prime} if its complement \(R-\fp\) is
multiplicative, or equivalently, if \(1\not\in\fp\) and if \(xy\in\fp\)
implies \(x\in\fp\) or \(y\in\fp\)

\subsection*{Fields, domains}
\label{sec:org26a29c3}
A ring is called a \textbf{field} if \(1\neq0\) and if every nonzero element is a
unit.

A ring is called an \textbf{integral domain}, or simply a \textbf{domain}, if \(\la0\ra\) is
prime, or equivalently, if \(R\) is nonzero and has no nonzero zerodivisors.

Every domain \(R\) is a subring of its \textbf{fraction field} \(\Frac(R)\).
Conversely, any subring \(R\) of a field \(K\), including \(K\) itself, is a
domain. Further, \(\Frac(R)\) has this UMP: the inclusion of \(R\) into any
field \(L\) extends uniquely to an inclusion of \(\Frac(R)\) into \(L\).


\subsection*{Polynomials over a domain}
\label{sec:org20f9360}
Let \(R\) be a domain, \(\calx:=\{X_\lambda\}_{\lambda\in\Lambda}\) a set of
variables. Set \(P:=R[\calx]\). Then \(P\) is a domain too. In fact, given
nonzero \(F,G\in P\), not only is their product \(FG\) nonzero, but also given
a well ordering of the variables, the grlex leading term of \(FG\) is the
product of the grlex leading terms of \(F\) and \(G\), and
\begin{equation*}
\deg(FG)=\deg(F)+\deg(G)\label{eq2.4.1}
\end{equation*}
Using the given ordering of the variables, well order all the monomials
\(\bM\) of the same degree via the lexicographic order on exponents. Among
the \(\bM\) in \(F\) with \(\deg(\bM)=\deg(F)\), the largest is called the
\textbf{grlex leading monomial} (graded lexicographic) of \(F\). Its \textbf{grlex leading
term} is the product \(a\bM\) whre \(a\in R\) is the coefficient of \(\bM\) in
\(F\), and \(a\) is called the \textbf{grlex leading coefficient}

\emph{The grlex leading term of \(FG\) is the product of those \(a\bM\) and}
\emph{\(b\bN\) of \(F\) and \(G\)}. and \ref{eq2.4.1} holds, for the following
reasons. First, \(ab\neq0\) as \(R\) is domain. Second
\begin{equation*}
\deg(\bM\bN)=\deg(\bM)+\deg(\bN)=\deg(F)+\deg(G)
\end{equation*}
Third, \(\deg(\bM\bN)\ge\deg(\bM'\bN')\) for every pair of monomials \(\bM'\)
and \(\bN'\) in \(F\) and \(G\).

\emph{The grlex hind term of \(FG\) is the product of the grlex hind terms of}
\emph{\(F\) and \(G\).} Further, given a vector \((x_\lambda)\in R^\Lambda\), then
\begin{equation*}
\ord_{(x_\lambda)}FG=\ord_{(x_\lambda)}F+\ord_{(x_\lambda)}G
\end{equation*}
Among the monomials \(\bM\) in \(F\) with \(\ord(\bM)=\ord(F)\), the smallest
is called the \textbf{grlex hind monomial} of \(F\). The \textbf{grlex hind term} of \(F\) os
the product \(a\bM\) where \(a\in R\) is the coefficient of \(\bM\) in \(F\)

The fraction field \(\Frac(P)\) is called the field of \textbf{rational functions},
and is also denoted by \(K(\calx)\) where \(K:=\Frac(R)\)

\subsection*{Unique factorization}
\label{sec:orgcefff53}
Let \(R\) be a domain, \(p\) a nonzero nonunit. We call \(p\) \textbf{prime} if
whenever \(p\mid xy\), either \(p\mid x\) or \(p\mid y\).
\emph{\(p\) is prime iff \(\la p\ra\) is prime}

We call \(p\) \textbf{irreducible} if whenever \(p=yz\), either \(y\) or \(z\) is a
unit. We call \(R\) a \textbf{Unique Factorization Domain} (UFD) if
\begin{enumerate}
\item every nonzero nonunit factors into a product of irreducibles
\item the factorization is unique up to order and units.
\end{enumerate}


If \(R\) is a UFD, then \(\gcd(x,y)\) always exists

\begin{lemma}[]
Let \(\varphi:R\to R'\) be a ring map, and \(T\subset R'\) a subset. If \(T\) is
multiplicative, then \(\varphi^{-1}T\) is multiplicative; the converse holds if \(\varphi\)
is surjective
\end{lemma}

\begin{proposition}[]
\label{prop2.7}
Let \(\varphi:R\to R'\) be a ring map, and \(\fq\subset R'\) an ideal. Set
\(\fp:=\varphi^{-1}\fq\). If \(\fq\) is prime, then \(\fp\) is prime; the
converse holds if \(\varphi\) is surjective
\end{proposition}

\begin{corollary}[]
Let \(R\) be a ring, \(\fp\) an ideal. Then \(\fp\) is prime iff \(R/\fp\) is
a domain
\end{corollary}

\begin{proof}
By Proposition \ref{prop2.7}, \(\fp\) is prime iff \(\la0\ra\subset R/\fp\) is
\end{proof}

\begin{exercise}
\label{ex2.9}
Let \(R\) be a ring, \(P:=R[\calx,\caly]\) the polynomial ring in two sets of
variables \(\calx\) and \(\caly\). Set \(\fp:=\la\calx\ra\). Show \(\fp\) is
prime iff \(R\) is a domain
\end{exercise}

\begin{proof}
\(\fp\) is prime iff \(R[\caly]\) is a domain
\end{proof}

\begin{definition}[]
Let \(R\) be a ring. An ideal \(\fm\) is said to be \textbf{maximal} if \(\fm\) is
proper and if there is no proper ideal \(\fa\) with \(\fm\subsetneq\fa\)
\end{definition}

\begin{examplle}[]
Let \(R\) be a domain, \(R[X,Y]\) the polynomial ring. Then \(\la X\ra\) is
prime. However, \(\la X\ra\) is not maximal since \(\la X\ra\subsetneq\la X,Y\ra\)
\end{examplle}

\begin{proposition}[]
\label{2.12}
A ring \(R\) is a field iff \(\la0\ra\) is a maximal ideal
\end{proposition}

\begin{proof}
If \(\la0\ra\) is  maximal. Take \(x\neq0\), then \(\la x\ra\neq0\). So \(\la
   x\ra=R\) and \(x\) is a unit.
\end{proof}

\begin{corollary}[]
\label{2.13}
Let \(R\) be a ring, \(\fm\) an ideal. Then \(\fm\) is maximal iff \(R/\fm\)
is a field.
\end{corollary}

\begin{proof}
\(\fm\) is maximal iff \(\la0\ra\) is maximal in \(R/\fm\) by Correspondence Theorem.
\end{proof}

\begin{examplle}[]
Let \(R\) be a ring, \(P\) the polynomial ring in variable \(X_\lambda\), and
\(x_\lambda\in R\) for all \(\lambda\). Set \(\fm:=\la\{X_\lambda-x_\lambda\}\ra\).
Then \(P/\fm=R\) by Exercise \ref{ex1.17}. Thus \(\fm\) is maximal iff \(R\) is a field
\end{examplle}

\begin{corollary}[]
In a ring, every maximal ideal is prime
\end{corollary}

\subsection*{Coprime elements \label{2.16}}
\label{sec:orgce5693b}
Let \(R\) be a ring and \(x,y\in R\). We say \(x\) and \(y\) are \textbf{(strictly)
coprime} if their ideals \(\la x\ra\) and \(\la y\ra\) are comaximal

Plainly, \(x\) and \(y\) are coprime iff there are \(a,b\in R\) s.t.
\(ax+by=1\)

Plainly, \(x\) and \(y\) are coprime iff there is \(b\in R\) with
\(by\equiv1\mod\la x\ra\) iff the residue of \(y\) is a unit in \(R/\la x\ra\)

Fix \(m,n\ge1\). By Exercise \ref{ex1.21}, \(x\) and \(y\) are coprim eiff
\(x^m\) and \(x^n\) are.

If \(x\) and \(y\) are coprime, then their images in algebra \(R'\) too.

\subsection*{PIDs}
\label{sec:orgeb87912}
A domain \(R\) is called a \textbf{Principal Ideal Domain} (PID) if every ideal is
principal. A PID is a UFD

Let \(R\) be a PID, \(\fp\) a nonzero prime ideal. Say \(\fp=\la p\ra\). Then
\(p\) is prime, so irreducible. Now let \(q\in R\) be irreducible. Then \(\la
   q\ra\) is maximal for: if \(\la q\ra\subsetneq\la x\ra\), then \(q=xy\) for
some nonunit \(y\); so \(x\) must be a unit as \(q\) is irreducible. So
\(R/\la q\ra\) is a field. Also \(\la q\ra\) is prime; so \(q\) is prime
Thus every irreducible element is prime, and every
nonzero prime ideal is maximal

\begin{exercise}
\label{2.18}
Show that, in a PID, nonzero elements \(x\) and \(y\) are \textbf{relatively prime}
(share no prime factor) iff they are coprime
\end{exercise}

\begin{proof}
Say \(\la x\ra+\la y\ra=\la d\ra\). Then \(d=\gcd(x,y)\)
\end{proof}

\begin{examplle}[]
Let \(R\) be a PID, and \(p\in R\) a prime. Set \(k:=R/\la p\ra\). Let \(X\)
be a variable, and set \(P:=R[X]\). Take \(G\in P\); let \(G'\) be its image
in \(k[X]\); assume \(G'\) is irreducible. Set \(\fm:=\la p,G\ra\). Then
\(P/\fm\simeq k[X]/\la G'\ra\) by \ref{ex1.16} and \ref{1.9} and \(k[X]/\la
   G'\ra\) is a field; hence
\(\fm\) is maximal
\end{examplle}

\begin{theorem}[]
\label{2.20}
Let \(R\) be a PID. Let \(P:=R[X]\) and \(\fp\) a nonzero prime ideal of
\(P\)
\begin{enumerate}
\item \(\fp=\la F\ra\) with \(F\) prime or \(\fp\) is maximal
\item Assume \(\fp\) is maximal. Then either \(\fp=\la F\ra\) with \(F\) prime,
or \(\fp=\la p,G\ra\) with \(p\in R\) prime, \(pR=\fp\cap R\) and \(G\in
      P\) prime with image \(G'\in(R/pR)[X]\) prime
\end{enumerate}
\end{theorem}

\begin{proof}
\(P\) is a UFD.

If \(\fp=\la F\ra\) for some \(F\in P\), then \(F\) is prime. Assume \(\fp\)
isn't principal

Take a nonzero \(F_1\in\fp\). Since \(\fp\) is prime, \(\fp\) contains a
prime factor \(F_1'\) of \(F_1\). Replace \(F_1\) by \(F_1'\). As \(\fp\)
isn't principal, \(\fp\neq\la F_1\ra\). So there is a prime \(F_2\in\fp-\la
   F_1\ra\).
Set \(K:=\Frac(R)\), Gauss's lemma implies that \(F_1\) and \(F_2\) are also
prime in \(K[X]\). So \(F_1\) and \(F_2\) are relatively prime in \(K[X]\).
So \ref{2.18} yield \(G_1,G_2\in P\) and \(c\in P\) with
\((G_1/c)F_1+(G_2/c)F_2=1\). So \(c=G_1F_1+G_2F_2\in R\cap\fp\). Hence
\(R\cap\fp\neq0\). But \(R\cap\fp\) is prime, and \(R\) is a PID; so
\(R\cap\fp=pR\) where \(p\) is prime. Also \(pR\) is maximal.

Set \(k:=R/pR\). Then \(k\) is a field. Set \(\fq:=\fp/pR\subset k[X]\). Then
\(k[X]/\fq=P/\fp\) by \ref{1.9}. But \(\fp\) is prime, so \(P/\fp\) is a
domain. So \(k[X]/\fq\) is a domain too. So \(\fq\) is prime. So \(\fq\) is
maximal. So \(\fp\) is maximal.

Since \(k[X]\) is a PID and \(\fq\) is prime, \(\fq=\la G'\ra\) where \(G'\)
is prime in \(k[X]\). Take \(G\in\fp\)  with image \(G'\)
\end{proof}

\begin{theorem}[]
Every proper ideal \(\fa\) is contained in some maximal ideal
\end{theorem}

\begin{proof}
Set \(\cals:=\{\text{ideals }\fb\mid\fb\supset\fa\text{ and }\fb\not\ni1\}\).
Then \(\fa\in\cals\) and \(\cals\) is partially ordered by inclusion. By
Zorn's Lemma
\end{proof}

\begin{corollary}[]
\label{2.22}
Let \(R\) be a ring, \(x\in R\). Then \(x\) is a unit iff \(x\) belongs to no
maximal ideal
\end{corollary}


\subsection*{Exercise}
\label{sec:orgad0987b}
\begin{exercise}
\label{2.23}
Let \(\fa\) and \(\fb\) be ideals, and \(\fp\) a prime ideal. Prove that
these conditions are equivalent
\begin{enumerate}
\item \(\fa\subset\fp\) or \(\fb\subset\fp\)
\item \(\fa\cap\fb\subset\fp\)
\item \(\fa\fb\subset\fp\)
\end{enumerate}
\end{exercise}

\begin{exercise}
\label{2.24}
Let \(R\)  be a  ring, \(\fp\) a prime ideal, and \(\fm_1,\dots,\fm_n\)
maximal ideals. Assume \(\fm_1\dots\fm_n=0\). Show \(\fp=\fm_i\) for some \(i\)
\end{exercise}

\begin{proof}
Note \(\fp\supset0=\fm_1\dots\fm_n\). So \(\fp\supset\fm_1\) or
\(\fp\supset\fm_2\dots\fm_n\) by \ref{2.23}
\end{proof}

\begin{exercise}
\label{2.25}
Let \(R\) be a ring, and \(\fp,\fa_1,\dots,\fa_n\) ideals with \(\fp\) prime
\begin{enumerate}
\item Assume \(\fp\supset\bigcap_{i=1}^n\fa_i\). Show \(\fp\supset\fa_j\) for
some \(j\)
\item Assume \(\fp=\bigcap_{i=1}^n\fa_i\). Show \(\fp=\fa_j\) for some \(j\)
\end{enumerate}
\end{exercise}

\begin{exercise}
\label{2.26}
Let \(R\) be a ring, \(\cals\) the set of all ideals that consist entirely of
zerodivisors. Show that \(\cals\) has maximal elements and they're prime.
Conclude that \(\zdiv(R)\) is a union of primes.
\end{exercise}

\begin{proof}
Order \(\cals\) by inclusion. \(\cals\) is not empty. \(\cals\) consists of a
maximal element \(\fp\).

Given \(x,x'\in R\) with \(xx'\in\fp\), but \(x,x'\not\in\fp\). Hence \(\la
   x\ra+\fp,\la x'\ra+\fp\not\in\cals\). So there are \(a,a'\in R\) and
\(p,p'\in\fp\)  s.t. \(y:=ax+p\) and \(y':=a'x'+p'\) are not zerodivisors. Then
\(yy'\in\fp\). So \(yy'\in\zdiv(R)\), a contradiction. Thus \(\fp\) is prime.

Given \(x\in\zdiv(R)\), note \(\la x\ra\in\cals\). So \(\la x\ra\) lies in a
maximal element \(\fp\) of \(\cals\). Thus \(x\in\fp\) and \(\fp\) is prime
\end{proof}

\begin{exercise}
\label{2.27}
Given a prime number \(p\) and an integer \(n\ge2\), prove that the residue
ring \(\Z/\la p^n\ra\) does not contain a domain as a subring
\end{exercise}

\begin{proof}
Any subring of \(\Z/\la p^n\ra\) must contain 1, and \(1\) generates \(\Z/\la
   p^n\ra\) as an Abelian group. So \(\Z/\la p^n\ra\) contains no proper subrings.
\end{proof}

\begin{exercise}
\label{2.28}
Let \(R:=R'\times R''\) be a product of two rings. Show that \(R\) is a
domain if and only if either \(R'\) or \(R''\) is a domain and the other 0
\end{exercise}

\begin{proof}
Assume \(R\) is a domain. As \((1,0)\cdot(0,1)=(0,0)\), either \(R'\) or
\(R''\) is 0.
\end{proof}

\begin{exercise}
\label{2.29}
Let \(R:=R'\times R''\) be a product of rings, \(\fp\subset R\) an ideal.
Show \(\fp\) is prime iff either \(\fp=\fp'\times R''\) with \(\fp'\subset
   R'\) prime or \(\fp=R'\times\fp''\) with \(\fp''\subset R''\) prime
\end{exercise}

\begin{proof}
\(1\in\fp\). \((1,0)(0,1)\in\fp\). Hence \((1,0)\in\fp\) or \((0,1)\in\fp\).
\end{proof}

\begin{exercise}
\label{2.30}
Let \(R\) be a domain, and \(x,y\in R\). Assume \(\la x\ra=\la y\ra\). Show
\(x=uy\) for some unit \(u\)
\end{exercise}

\begin{proof}
\((1-tu)y=0\) and domain
\end{proof}

\begin{exercise}
\label{2.31}
Let \(k\) be a field, \(R\) a nonzero ring, \(\varphi:k\to R\) a ring map. Prove \(\varphi\)
is injective
\end{exercise}

\begin{proof}
Since \(1\neq0\), \(\ker(\varphi)\neq k\). And by \ref{2.12}, \(\ker(\varphi)=0\) and hence
\(\varphi\) is injective
\end{proof}

\begin{exercise}
\label{2.32}
Let \(R\) be a ring, \(\fp\) a prime, \(\calx\) a set of variables. Let
\(\fp[\calx]\) denote the set of polynomials with coefficients in \(\fp\).
Prove
\begin{enumerate}
\item \(\fp R[\calx]\) and \(\fp[\calx]\) and \(\fp R[\calx]+\la\calx\ra\) are
primes of \(R[\calx]\), which contract to \(\fp\)
\item Assume \(\fp\) is maximal. Then \(\fp R[\calx]+\la\calx\ra\) is maximal
\end{enumerate}
\end{exercise}

\begin{proof}
\begin{enumerate}
\item \(R/\fp\) is a domain. \(\fp R[\calx]=\fp[\calx]\) by \ref{1.15}.

\((\fp R[\calx]+\la\calx\ra/\fp R[\calx])\) is equal to
\(\la\calx\ra\subset(R/\fp)[\calx]\). \((R/\fp)\la\calx\ra/\la\calx\ra\)
is equal to \(R/\fp\). Hence
\(R[X]/(\fp R[\calx]+\la\calx\ra)=(R[x]/\fp R[X])/((\fp
      R[\calx]+\la\calx\ra)/\fp R[X])=R/\fp\)

Since the canonical map \(R/\fp\to R[\calx]/(\fp R[\calx]+\la\calx\ra)\)
is bijective, it's injective.

\item \(R/\fp\simeq R[\calx]/(\fp R[\calx]+\la\calx\ra)\)
\end{enumerate}
\end{proof}

\begin{exercise}
\label{2.33}
Let \(R\) be a ring, \(X\) a variable, \(H\in P:=R[X]\) and \(a\in R.\)Given
\(n\ge1\), show \((X-a)^n\) and \(H\) are coprime iff \(H(a)\) is a unit.
\end{exercise}

\begin{proof}
\((X-a)^n\) and \(H\) are coprime iff \(X-a\) and \(H\) are coprime.
\(R[x]/\la X-a\ra=\la H\ra/\la X-a\ra\), which implies the residue of \(H\)
modulo \(X-a\) is a unit. Hence \(H(a)\) is a unit.
\end{proof}

\begin{exercise}
\label{2.34}
Let \(R\) be a ring, \(X\) a variable, \(F\in P:=R[X]\), and \(a\in R\). Set
\(F':=\partial F/\partial X\). Show the following statements are equivalent
\begin{enumerate}
\item \(a\) is a supersimple root of \(F\)
\item \(a\) is a root of \(F\), and \(X-a\) and \(F'\) are coprime
\item \(F=(X-a)G\) for some \(G\) in \(P\) coprime to \(X-a\)
\end{enumerate}


Show that if (3) holds, then \(G\) is unique
\end{exercise}

\begin{exercise}
\label{2.35}
Let \(R\) be a ring, \(\fp\) a prime; \(\calx\) a set of variables; \(F,G\in
   R[\calx]\). Let \(c(F)\), \(c(G)\), \(c(FG)\) be the ideals of \(R\)
generated by the coefficients of \(F,G,FG\)
\begin{enumerate}
\item Assume \(\fp\) doesn't contain either \(c(F)\) or \(c(G)\). Show \(\fp\)
doesn't contain \(c(FG)\)
\item Assume \(c(F)=R\) and \(c(G)=R\). Show \(c(FG)=R\)
\end{enumerate}
\end{exercise}

\begin{proof}
\begin{enumerate}
\item Denote the residues of \(F,G,FG\)  in \((R/\fp)[\calx]\) by
\(\overbar{F}\), \(\overbar{G}\) and \(\overbar{FG}\). Since
\(\fp\not\supset c(F),c(G)\), \(\overbar{F},\overbar{G}\neq0\). Since
\(R/\fp\) is a domain, so is \((R/\fp)[\calx]\) and we have
\(\overbar{F}\overbar{G}\neq0\). Note that
\(\overbar{F}\overbar{G}=\overbar{FG}\), we have \(\overbar{FG}\neq0\).
\item Assume \(c(F)=c(G)=R\), since \(\fp\not\supset c(F),c(G)\) we have
\(\fp\not\supset c(FG)\) for any prime ideals \(\fp\). Hence \(c(FG)=R\).

If \(c(FG)=R\), \(c(FG)\subset c(F)\)
\end{enumerate}
\end{proof}

\begin{exercise}
\label{2.36}
Let \(B\) be a Boolean ring. Show that every prime \(\fp\) is maximal, and
that \(B/\fp=\F_2\)
\end{exercise}

\begin{proof}
\(x(x-1)=0\) in \(B/\fp\). Since \(B/\fp\) is a domain, \(x=0\) or \(x=1\).
\end{proof}

\begin{exercise}
\label{2.37}
Let \(R\) be a ring. Assume that, given any \(x\in R\), there is an
\(n\ge2\) with \(x^n=x\). Show that every prime \(\fp\) is maximal
\end{exercise}

\begin{proof}
Same. Every element has an inverse
\end{proof}

\begin{exercise}
\label{2.38}
Prove the following statements or give a counterexample
\begin{enumerate}
\item The complement of a multiplicative subset is a prime ideal
\item Given two prime ideals, their intersection is prime
\item Given two prime ideals, their sum is prime
\item Given a ring map \(\varphi:R\to R'\), the operation \(\varphi^{-1}\) carries maximal
ideals of \(R'\) to maximal ideals of \(R\)
\item An ideal \(\fm'\subset R/\fa\) is maximal iff \(\kappa^{-1}\fm'\subset R\) is
maximal in \ref{1.9}
\end{enumerate}
\end{exercise}

\begin{proof}
\begin{enumerate}
\item 0 can be belongs to the multiplicative subset
\item False. In \(\Z\), \(\la2\ra\cap\la3\ra=\la6\ra\)
\item False. In \(\Z\), \(\la2\ra+\la3\ra=\Z\)
\item False. Consider \(\varphi:\Z\to\Q\). \(\varphi^{-1}(\la0\ra)=\la0\ra\)
\item 
\end{enumerate}
\end{proof}

\section{Radicals}
\label{sec:org275d869}
\begin{definition}[]
Let \(R\) be a ring. Its (Jacobson) \textbf{radical} \(\rad(R)\) is defined to be the
intersection of all its maximal ideals
\end{definition}

\begin{proposition}[]
\label{3.2}
Let \(R\) be a ring, \(\fa\) an ideal, \(x\in R,u\in R^\times\). Then
\(x\in\rad(R)\) iff \(u-xy\in R^\times\) for all \(x\in R\). In particular,
the sum of an element of \(\rad(R)\) and a unit is a unit, and
\(\fa\subset\rad(R)\) if \(1-\fa\in R^\times\)
\end{proposition}

\begin{proof}
Assume \(x\in\rad(R)\). Given a maximal ideal \(\fm\), suppose \(u-xy\in\fm\).
Since \(x\in\fm\) too, also \(u\in\fm\), a contradiction. Thus \(u-xy\) is a
unit by \ref{2.22}. In particular, tkaing \(y:=-1\) yields \(u+x\in R^\times\)

Conversely, assume \(x\not\in\rad(R)\). Then there is a maximal ideal \(\fm\)
with \(x\not\in\fm\). So \(\la x\ra+\fm=R\). Hence there exists \(y\in R\) and
\(m\in\fm\) s.t. \(xy+m=u\). Then \(u-xy=m\in\fm\). A contradiction

In particular, given \(y\in R\), set \(a:=u^{-1}xy\). Then \(u-xy=u(1-a)\in
  R^\times\) if \(1-a\in R^\times\)
\end{proof}

\begin{corollary}[]
\label{3.3}
Let \(R\) be a ring, \(\fa\) an ideal, \(\kappa:R\to R/\fa\) the quotient map.
Assume \(\fa\subset\rad(R)\). Then \(\Idem(\kappa)\) is injective
\end{corollary}

\begin{proof}
Given \(e,e'\in\Idem(R)\) with \(\kappa(e)=\kappa(e')\), set \(x:=e-e'\). Then
\begin{equation*}
x^3=e-e'=x
\end{equation*}
Hence \(x(1-x^2)=0\). But \(\kappa(x)=0\); so \(x\in\fa\). But
\(\fa\subset\rad(R)\). Hence \(1-x^2\) is a unit by \ref{3.2}. Thus \(x=0\).
Thus \(\Idem(\kappa)\) is injective
\end{proof}

\begin{definition}[]
A ring is called \textbf{local} if it has exactly one maximal ideal, and \textbf{semilocal} if
it has at least one and at most finitely many

By the \textbf{residue field} of a local ring \(A\), we mean the field \(A/\fm\) where
\(\fm\) is the maximal ideal of \(A\)
\end{definition}

\begin{lemma}[Nonunit Criterion]
\label{3.5}
Let \(A\) be a ring, \(\fn\) the set of nonunits. Then \(A\) is local iff
\(\fn\) is an ideal; if so, then \(\fn\) is the maximal ideal
\end{lemma}

\begin{proof}
Assume \(A\) is local with maximal ideal \(\fm\). Then \(A-\fn=A-\fm\) by
\ref{2.22}. Thus \(\fn\) is an ideal
\end{proof}

\begin{examplle}[]
The product ring \(R'\times R''\) is not local by \ref{3.5} if both \(R'\) and
\(R''\) are nonzero. \((1,0)\) and \((0,1)\) are nonunits, but their sum is a unit.
\end{examplle}

\begin{examplle}[]
Let \(R\) be a ring. A \textbf{formal power series} in the \(n\) variables
\(X_1,\dots,X_n\) is a formal \emph{infinite} sum of the form \(\sum
  a_{(i)}X_1^{i_1}\dots X_n^{i_n}\) where \(a_{(i)}\in R\) and where
\((i):=(i_1,\dots,i_n)\) with each \(i_j\ge0\). The term \(a_{(0)}\) where
\((0):=(0,\dots,0)\) is called the \textbf{constant term}. Addition and multiplication
are performed as for polynomials; with these operations, these series form a
ring \(R[[X_1,\dots,X_n]]\)

Set \(P:=R[[X_1,\dots,X_n]]\) and \(\fa:=\la X_1,\dots,X_n\ra\). Then \(\sum
  a_{(i)}X_1^{i_1}\dots X_n^{i_n}\mapsto a_{(0)}\) is a canonical surjective
ring map \(P\to R\) with kernel \(\fa\); hence \(P/\fa=R\)

Given an ideal \(\fm\subset R\), set \(\fn:=\fa+\fm P\). Then \ref{1.9} yields
\(P/\fn=R/\fm\)

A power series \(F\) is a unit iff its constant term is a unit. If \(a_{(0)}\)
is a unit, then \(F=a_{(0)}(1-G)\) with \(G\in\fa\). Set
\(F':=a_{(0)}^{-1}(1+G+G^2+\dots)\);

Suppose \(R\) is a local ring with maximal ideal \(\fm\). Given a power series
\(F\not\in\fn\), its constant term lies outside \(\fm\), so is a unit. So
\(F\) is itself a unit. Hence the nonunits constitutes \(\fn\). Thus \(P\) is
local.
\end{examplle}

\begin{examplle}[]
Let \(k\) be a ring, and \(A:=k[[X]]\) the formal power series ring in one
variables. A \textbf{formal Laurent series} is a formal sum of the form
\(\sum_{i=-m}^\infty a_iX^i\) with \(a_i\in k\) and \(m\in\Z\). Plainly, these
seires form a ring \(k\{\{X\}\}\). Set \(K:=k\{\{X\}\}\)

Set \(F:=\sum_{i=-m}^\infty a_iX^i\). If \(a_{-m}\in k^\times\), then \(F\in
  K^\times\); indeed, \(F=a_{-m}X^{-m}(1-G)\) where \(G\in A\) and

Assume \(k\) is a field. If \(F\neq0\), then \(F=X^{-m}H\) with
\(H:=a_{-m}(1-G)\in A^\times\). Let \(\fa\subset A\) be a nonzero ideal.
Suppose \(F\in\fa\). Then \(X^{-m}\in\fa\). Let \(n\) be the smallest integer
s.t. \(X^n\in\fa\). Then \(-m\ge n\). Set \(E:=X^{-m-n}H\). Then \(E\in A\)
and \(F=X^nE\). Hence \(\fa=\la X^n\ra\). Thus \(A\) \textbf{is a} PID

Further, \(K\) is a field. In fact, \(K=\Frac(A)\).

Let \(A[Y]\) be the polynomial ring in one variable, and \(\iota:A\hookrightarrow
  K\) the inclusion.
Define \(\varphi:A[Y]\to K\) by \(\varphi|A=\iota\) and \(\varphi(Y)=X^{-1}\). Then \(\varphi\) is
surjective. Set \(\fm:=\ker(\varphi)\). Then \(\fm\) is maximal. So by \ref{2.20}
\(\fm\) has the form \(\la F\ra\) with \(F\) irreducible, or the form \(\la
  p,G\ra\) with \(p\in A\) irreducible and \(G\in A[Y]\). But \(\fm\cap
  A=\la0\ra\) as \(\iota\) is injective. So \(\fm=\la F\ra\). But \(XY-1\) belongs to
\(\fm\), and is clearly irreducible; hence \(XY-1=FH\) with \(H\) a unit. Thus
\(\la XY-1\ra\) is maximal

In addition, \(\la X,Y\ra\) is maximal. Indeed, \(A[Y]/\la X,Y\ra=A/\la
  X\ra=k\). Howevery ,\(\la X,Y\ra\) is not principal, as no nonunit of \(A[Y]\)
divides both \(X\) and \(Y\). Thus \(A[Y]\) \emph{has both principal and
nonprincipal maximal ideals, two types allows by \ref{2.20}}
\end{examplle}

\begin{proposition}[]
\label{3.9}
Let \(R\) be a ring, \(S\) a multiplicative subset, and \(\fa\) an ideal with
\(\fa\cap S=\emptyset\). Set \(\cals:=\{\text{ideals
  }\fb\mid\fb\supset\fa\text{ and }\fb\cap S=\emptyset\}\). Then \(\cals\) has a
maximal element \(\fp\), and every such \(\fp\) is prime
\end{proposition}

\begin{proof}
Take \(x,y\in R-\fp\). Then \(\fp+\la x\ra\) and \(\fp+\la y\ra\) are strictly
larger than \(\fp\). So there are \(p,q\in\fp\) and \(a,b\in R\) with
\(p+ax,q+by\in S\). Hence \(pq+pby+qax+abxy\in S\). But \(pq+pby+qax\in\fp\),
so \(xy\not\in\fp\). Thus \(\fp\) is prime
\end{proof}

\begin{exercise}
\label{3.10}
Let \(\varphi:R\to R'\) be a ring map, \(\fp\) an ideal of \(R\). Show
\begin{enumerate}
\item there is an ideal \(\fq\) of \(R'\) with \(\varphi^{-1}(\fq)=\fp\) iff
\(\varphi^{-1}(\fp R')=\fp\)
\item if \(\fp\) is prime with \(\varphi^{-1}(\fp R')=\fp\), then there is a prime
\(\fq\) of \(R'\) with \(\varphi^{-1}(\fq)=\fp\)
\end{enumerate}
\end{exercise}

\subsection*{Saturated multiplicative subsets}
\label{sec:org2860a0c}
Let \(R\) be a ring, and \(S\) a multiplicative subset. We say \(S\) is
\textbf{saturated} if given \(x,y\in R\) with \(xy\in S\), necessarily \(x,y\in S\)

\begin{lemma}[Prime Avoidance]
Let \(R\) be a ring, \(\fa\) a subset of \(R\) that is stable under addition
and multiplication, and \(\fp_1,\dots,\fp_n\) ideals s.t.
\(\fp_3,\dots,\fp_n\) are prime. If \(\fa\not\subset\fp_j\) for all \(j\),
then there is an \(x\in\fa\) s.t. \(x\not\in\fp_j\) for all \(j\); or
equivalently, if \(\fa\subset\bigcup_{i=1}^n\fp_i\), then \(\fa\subset\fp_i\)
for some \(i\)
\end{lemma}

\begin{proof}
Assume there is an \(x_i\in\fa\) s.t. \(x_i\not\in\fp_j\) for all
 \(i\neq j\)and \(x_i\in\fp_i\) for every \(i\). If \(n=2\) then clearly
 \(x_1+x_2\not\in\fp_j\) for \(j=1,2\). If \(n\ge3\), then \((x_1\dots
   x_{n-1})+x_n\not\in\fp_j\) for all \(j\) as, if \(j=n\), then \(x_n\in\fp_n\)
 and \(\fp_n\) is prime.
\end{proof}

\subsection*{Other radicals}
\label{sec:orgc337ced}
Let \(R\) be a ring, \(\fa\) a subset. Its \textbf{radical} \(\sqrt{\fa}\) is the set
\begin{equation*}
\sqrt{\fa}:=\{x\in R\mid x^n\in\fa\text{ for some }n\ge1\}
\end{equation*}
If \(\fa\) is an ideal and \(\fa=\sqrt{\fa}\), then \(\fa\) is said to be
\textbf{radical}. For example, suppose \(\fa=\bigcap\fp_\lambda\) with all
\(\fp_\lambda\) prime. If \(x^n\in\fa\) for some \(n\ge1\), then
\(x\in\fp_\lambda\). Thus \(\fa\) is radical. Hence two radicals coincide

We call \(\sqrt{\la0\ra}\) the \textbf{nilradical}, and sometimes denote it by
\(\nil(R)\). We call an element \(x\in R\) \textbf{nilpotent} if \(x\) belongs to
\(\sqrt{\la0\ra}\). We call an ideal \(\fa\) \textbf{nilpotent} if \(\fa^n=0\) for
some \(n\ge1\)

\(\la0\ra\subset\rad(R)\). So \(\sqrt{\la0\ra}\subset\sqrt{\rad(R)}\). Thus
\begin{equation*}
\nil(R)\subset\rad(R)
\end{equation*}
We call \(R\) \textbf{reduced} if \(\nil(R)=\la0\ra\)

\begin{theorem}[Scheinnullstellensatz]
\label{3.14}
Let \(R\) be a ring, \(\fa\) an ideal. Then
\begin{equation*}
\sqrt{\fa}=\bigcap_{\fp\supset\fa}\fp
\end{equation*}
where \(\fp\) runs through all the prime ideals containing \(\fa\). (By
convention, the empty intersection is equal to \(R\))
\end{theorem}

\begin{proof}
Take \(x\not\in\sqrt{\fa}\). Set \(S:=\{1,x,x^2,\dots\}\). Then \(S\) is
multiplicative, and \(\fa\cap S=\emptyset\). By \ref{3.9} there is a
\(\fp\supset\fa\), but \(x\not\in\fp\), but \(x\not\in\fp\). So
\(x\not\in\bigcap_{\fp\supset\fa}\fp\). Thus
\(\sqrt{\fa}\supset\bigcap_{\fp\supset\fa}\fp\).
\end{proof}

\begin{proposition}[]
\label{3.15}
Let \(R\) be a ring, \(\fa\) an ideal. Then \(\sqrt{\fa}\) is an ideal
\end{proposition}

\begin{proof}
Assume \(x^n,y^m\in\fa\). Then
\begin{equation*}
(x+y)^{m+n-1}=\sum_{i+j=m+n-1}\binom{n+m-1}{j}x^iy^j
\end{equation*}
Thus \(x+y\in\fa\)

Alternatively by \ref{3.14}
\end{proof}

\begin{exercise}
\label{3.16}
Use Zorn's lemma to prove that any prime ideal \(\fp\) contains a prime ideal
\(\fq\) that is minimal containing any given subset \(\fs\subset\fp\)
\end{exercise}

\subsection*{Minimal primes \label{3.17}}
\label{sec:orgec65511}
Let \(R\) be a ring, \(\fa\) an ideal, \(\fp\) a prime. We call \(\fp\) a
\textbf{minimal prime} of \(\fa\), or over \(\fa\), if \(\fp\) is minimal in the set
of primes containing \(\fa\). We call \(\fp\) a \textbf{minimal prime} of \(R\) if
\(\fp\) is a minimal prime of \(\la0\ra\)

Owing to \ref{3.16}, every prime of \(R\) containing \(\fa\) contains a minimal
prime of \(\fa\). So owing to the Scheinnullstellensatz \ref{3.14}, the radical
\(\sqrt{\fa}\) is the intersection of all the minimal primes of \(\fa\).

\begin{proposition}[]
A ring \(R\) is reduced and has only one minimal prime if and only if \(R\)
is a domain
\end{proposition}

\begin{proof}
\ref{3.17} implies \(\la0\ra=\fq\)
\end{proof}

\begin{exercise}
\label{3.19}
Let \(R\) be a ring, \(\fa\) an ideal, \(X\) a variable, \(R[[X]]\) the formal
power series ring, \(\fM\subset R[[X]]\) an ideal, \(F:=\sum a_nX_n\in R[[X]]\). Set
\(\fm:=\fM\cap R\) and \(\fA:=\{\sum b_nX^n\mid b_n\in\fa\}\). Prove the
following statements:
\begin{enumerate}
\item If \(F\) is a nilpotent, then \(a_n\) is nilpotent for all \(n\). The
converse is false
\item \(F\in\rad(R[[X]])\) iff \(a_0\in\rad(R)\)
\item Assume \(X\in\fM\). Then \(X\) and \(\fm\) generate \(\fM\)
\item Assume \(\fM\) is maximal. Then \(X\in\fM\) and \(\fm\) is maximal
\item If \(\fa\) is finitely generated, then \(\fa R[[X]]=\fA\). However, there's an
example of an \(R\) with a prime ideal \(\fa\) s.t. \(\fa R[[X]]\neq\fA\)
\end{enumerate}
\end{exercise}

\begin{proof}
\begin{enumerate}
\item Assume \(F\) and \(a_i\) for \(i<n\) nilpotent. Set \(G:=\sum_{i\ge
      n}a_iX^i\). Then \(G=F-\sum_{i<n}a_iX^i\). So \(G\) is nilpotent by
\ref{3.15}; say \(G^m=0\) for some \(m\ge1\). Then \(a^m_n=0\)

Set \(P:=\Z[X_2,X_3,\dots]\). Set \(R:=P/\la X_2^2,X_3^3,\dots\ra\). Let
\(a_n\) be the residue of \(X_n\). Then \(a^n_n=0\), but \(\sum a_nX^n\)
is not nilpotent.

\item By \ref{3.2}, suppose \(G=\sum b_iX^i\)
\begin{equation*}
F\in\rad(R[[X]])\Longleftrightarrow 1+FG\in R[[X]]^\times
\Longleftrightarrow 1+a_0b_0\in R^\times\Longleftrightarrow
a_0\in\rad(R)
\end{equation*}

\setcounter{enumi}{4}
\item Take \(R:=\Z[a_1,a_2,\dots]\) and \(\fa:=\la a_1,\dots\ra\). Then
\(R/\fa=\Z\) and \(\fa\) is prime.

Given \(G\in\fa R[[X]]\), say \(G=\sum_{i=1}^mb_iG_i\) with \(b_i\in\fa\) and
\(G_i=\sum_{n\ge0}b_{in}X^n\) and \(F\neq G\) for any \(m\)
\end{enumerate}
\end{proof}

\begin{examplle}[]
Let \(R\) be a ring, \(R[[X]]\) the formal power series ring. Then every prime
\(\fp\) of \(R\) is the contraction of a prime of \(R[[X]]\). Indeed \(\fp R[[X]]\cap
   R=\fp\). So by \ref{3.10} there is a prime \(\fq\) of \(R[[X]]\) with \(\fq\cap
   R=\fp\). In fact ,a specific choice for \(\fq\) is the set of series \(\sum
   a_nX^n\) with \(a_n\in\fq\). Indeed, the canonical map \(R\to R/\fp\) induces
a surjection \(R[[X]]\to(R/\fp)[[X]]\) with kernel \(\fq\); so \(R[[X]]/\fq=(R/\fp)[[X]]\).
But \ref{3.19} shows \(\fq\) may not be equal to \(\fp R[[X]]\)
\end{examplle}

\subsection*{Exercise}
\label{sec:org1f39957}
\begin{exercise}
\label{3.21}
Let \(R\) be a ring, \(\fa\subset\rad(R)\) an ideal, \(w\in R\) and \(w'\in
   R/\fa\) its residue. Prove that \(w\in R^\times\) iff
\(w'\in(R/\fa)^\times\). What if \(\fa\not\subset\rad(R)\)?
\end{exercise}

\begin{proof}
Assume \(\fa\subset\rad(R)\). \(\fm\mapsto\fm/\fa\) is a bijection for
maximal ideal \(\fm\). So \(w\) belongs to a maximal ideal of \(R\) iff \(w'\) belongs
to one of \(R/\fa\)

Assume \(\fa\not\subset\rad(R)\), then there is a maximal ideal \(\fm\) s.t.
\(\fa\not\subset\fm\). So \(\fa+\fm=R\). So there are \(a\in\fa\) and
\(v\in\fm\) s.t. \(a+v=w\). Then \(v\not\in R^\times\) but the residue of
\(v\) is \(w'\), even if \(w'\in(R/\fa)^\times\). For example, take \(R:=\Z\)
and \(\fa=\la2\ra\) and \(w:=3\). Then \(w\not\in R^\times\) but the residue
of \(w\) is \(1\in(R/\fa)^\times\)
\end{proof}

\begin{exercise}
\label{3.22}
Let \(A\) be a local ring, \(e\) an idempotent. Show \(e=1\) or \(e=0\)
\end{exercise}

\begin{proof}
\(1-e+e=1\). Since \(1\not\in\fm\), at least one of \(1-e\) and \(e\) doesn't
belong to \(\fm\)
\end{proof}

\begin{exercise}
\label{3.23}
Let \(A\) be a ring, \(\fm\) a maximal ideal s.t. \(1+m\) is a unit for every
\(m\in\fm\). Prove \(A\) is local. Is this assertion still true if \(\fm\) is
not maximal?
\end{exercise}

\begin{proof}
Let \(y\in A-\fm\). Then \(\la y\ra+\fm=A\) and there is a \(x\in A\) s.t.
\(xy+m=1\). Hence \(xy\) is a unit and \(\la xy\ra=\la y\ra\). \(y\) is a unit.
\end{proof}

\begin{exercise}
\label{3.24}
Let \(R\) be a ring, and \(S\) a subset. Show that \(S\) is saturated
multiplicative iff \(R-S\) is a union of primes.
\end{exercise}

\begin{proof}
Assume \(S\) is saturated multiplicative. Take \(x\in R-S\). Then \(xy\not\in
   S\) for all \(y\in R\); in other words, \(\la x\ra\cap S=\emptyset\). Then
\ref{3.9} gives a prime \(\fp\supset\la x\ra\) with \(\fp\cap S=\emptyset\).
Thus \(R-S\) is a union of primes.
\end{proof}

\begin{exercise}
\label{3.25}
Let \(R\) be a ring, and \(S\) a multiplicative subset. Define its \textbf{saturation}
to be the subset
\begin{equation*}
\overbar{S}:=\{x\in R\mid\text{there is }y\in R\text{ with }xy\in S\}
\end{equation*}
\begin{enumerate}
\item Show that \(\overbar{S}\supset S\) and that \(\overbar{S}\) is saturated
multiplicative and that any saturated multiplicative subset \(T\)
containing \(S\) also contains \(\overbar{S}\)
\item Set \(U:=\bigcup_{\fp\cap S=\emptyset}\fp\). Show that \(R-\overbar{S}=U\)
\item Let \(\fa\) an ideal; assume \(S=1+\fa\); set
\(W:=\bigcup_{\fp\supset\fa}\fp\). Show \(R-\overbar{S}=W\)
\item Given \(f,g\in R\), show that \(\overbar{S_f}\subset\overbar{S_g}\) iff
\(\sqrt{\la f\ra}\supset\sqrt{\la g\ra}\), where \(S_f=\{f^n\mid n\ge0\}\)
\end{enumerate}
\end{exercise}

\begin{proof}
\begin{enumerate}
\setcounter{enumi}{2}
\item First take a prime \(\fp\) with \(\fp\cap S=\emptyset\). Then
\(1\not\in\fp+\fa\); else, \(1=p+a\) and \(p=1-a\in\fp\cap S\). So
\(\fp+\fa\) lies in a maximal ideal \(\fm\). Then \(\fa\subset\fm\); so
\(\fm\subset W\). But also \(\fp\subset W\). So \(U\subset W\)

Conversely, take \(\fp\supset\fa\). Then \(1+\fp\supset 1+\fa=S\). But
\(\fp\cap(1+\fp)=\emptyset\). So \(\fp\cap S=\emptyset\). Thus \(U\subset
      W\). Thus \(U=W\). Thus \(2\) implies (3)

\item \(\overbar{S_f}\subset\overbar{S_g}\) iff \(f\in\overbar{S_g}\) iff
\(hf=g^n\) iff \(g\in\sqrt{\la f\ra}\) iff \(\sqrt{\la
      g\ra}\subset\sqrt{\la f\ra}\)
\end{enumerate}
\end{proof}

\begin{exercise}
\label{3.26}
Let \(R\) be a nonzero ring, \(S\) a subset. Show \(S\) is maximal in the
\(\fS\) of multiplicative subsets \(T\) of \(R\) with \(0\not\in T\) iff
\(R-S\) is a minimal prime
\end{exercise}

\begin{proof}
First assume \(S\) is maximal. Then \(S=\overbar{S}\). So \(R-S\) is a union
of primes \(\fp\). Fix a \(\fp\). Then \ref{3.16} yields in \(\fp\) a minimal
prime ideal \(\fq\). Then \(S\subset R-\fq\). But \(R-\fq\in\fS\).
\(S=R-\fq\)

If \(R-S\) is a minimal prime. Then \(S\in\fS\). Given \(T\in\fS\) with
\(S\subset T\), note \(R-\overbar{T}=\bigcup\fp\) with \(\fp\) prime. Fix a
\(\fp\), then \(S\subset T\subset\overbar{T}\). So \(\fq\supset\fp\). But
\(\fq\) is minimal and hence \(\fq=\fp\). Hence \(\fq=R-\overbar{T}\). So \(S=\overbar{T}\)
\end{proof}

\begin{exercise}
\label{3.27}
Let \(k\) be a field, \(X_\lambda\) for \(\lambda\in\Lambda\) variables, and
\(\Lambda_\pi\) for \(\pi\in\Pi\) disjoint subsets of \(\Lambda\). Set
\(P:=k[\{X_\lambda\}_{\lambda\in\Lambda}]\) and
\(\fp_\pi:=\la\{X_\lambda\}_{\lambda\in\Lambda_\pi}\ra\)for all
\(\pi\in\Pi\). Let \(F,G\in P\) be nonzero, and \(\fa\subset P\) a nonzero
ideal. Set \(U:=\bigcup_{\pi\in\Pi}\fp_\pi\). Show
\begin{enumerate}
\item Assume \(F\in\fp_\pi\) for some \(\pi\in\Pi\), then every monomial of
\(F\) is in \(\fp_\pi\)
\item Assume there are \(\pi,\rho\in\Pi\) s.t. \(F+G\in\fp_\pi\) and
\(G\in\fp_\rho\) but \(\fp_\rho\) contains no monomial of \(F\). Then
\(\fp_\pi\) contains every monomial of \(F\) and of \(G\)
\item Assume \(\fa\subset U\). Then \(\fa\subset\fp_\pi\) for some \(\pi\in\Pi\)
\end{enumerate}
\end{exercise}

\section{Modules}
\label{sec:orgc90ba4e}

\subsection*{Modules}
\label{sec:org9b72e0a}
Let \(R\) be a ring. Recall that an \textbf{\(R\)-module} \(M\) is an abelian group,
written additively, with a \textbf{scalar multiplication}, \(R\times M\to M\), written
\((x,m)\mapsto xm\), which is
\begin{enumerate}
\item \textbf{distributive}, \(x(m+n)=xm+xn\) and \((x+y)m=xm+xn\)
\item \textbf{associative}, \(x(ym)=(xy)m\)
\item \textbf{unitary}, \(1\cdot m=m\)
\end{enumerate}


For example, if \(R\) is a field, then an \(R\)-module is a vector space. A
\(\Z\)-module is just an abelian group

A \textbf{submodule} \(N\) of \(M\) is a subgroup that is closed under
multiplication.; that is, \(xn\in N\) for all \(x\in R\) and \(n\in N\). For
example, the ring \(R\) is itself an \(R\)-module, and the submodules are
just the ideals. Given an ideal \(\fa\), let \(\fa N\) denote the smallest
submodule containing all products \(an\) with \(a\in\fa\) and \(n\in N\).
\(\fa N\) is equal to the set of finite sums \(\sum a_in_i\).

Given \(m\in M\), we call the set of \(x\in R\) with \(xm=0\) the \textbf{annihilator}
of \(m\), and denote it \(\Ann(m)\). We call the set of \(x\in R\) with
\(xm=0\) for all \(m\in M\) the \textbf{annihilator} of \(M\), and denote it \(\Ann(M)\)

\subsection*{Homomorphisms}
\label{sec:orga1c34ee}
Let \(R\) be a ring, \(M\) and \(N\) modules. A \textbf{homomorphism}, or \textbf{module map}
is a map \(\alpha:M\to N\) that is \textbf{\(R\)-linear}:
\begin{equation*}
\alpha(xm+yn)=x(\alpha m)+y(\alpha n)
\end{equation*}

Note that \(f\) is injective iff it has a left inverse. \(f\) is surjective
iff it has a right inverse

A homomorphism \(\alpha\) is an isomorphism iff there is a set map \(\beta:N\to M\) s.t.
\(\beta\alpha=1_M\) and \(\alpha\beta=1_N\), and then \(\beta=\alpha^{-1}\).

The set of homomorphisms \(\alpha\) is denoted by \(\Hom_R(M,N)\) or simply
\(\Hom(M,N)\). It is an \(R\)-module with addition and scalar multiplication
defined by
\begin{equation*}
(\alpha+\beta)m:=\alpha m+\beta m \quad\text{ and }\quad
(x\alpha)m:=x(\alpha m)=\alpha(xm)
\end{equation*}

Homomorphisms \(\alpha:L\to M\) and \(\beta:N\to P\) induce, via composition, a map
\begin{equation*}
\Hom(\alpha,\beta):\Hom(M,N)\to\Hom(L,P)
\end{equation*}

When \(\alpha\) is the identity map \(1_M\), we write \(\Hom(M,\beta)\) for
\(\Hom(1_M,\beta)\)

\begin{exercise}
\label{4.3}
Let \(R\) be a ring, \(M\) a module. Consider the map
\begin{equation*}
\theta:\Hom(R,M)\to M\quad\text{ defined by }\quad\theta(\rho):=\rho(1)
\end{equation*}
Show that \(\theta\) is an isomorphism, and describe its inverse
\end{exercise}

\begin{proof}
First, \(\theta\) is \(R\)-linear. Set \(H:=\Hom(R,M)\). Define \(\eta:M\to H\) by
\(\eta(m)(x):=xm\). It is easy to check that \(\eta\theta=1_H\) and
\(\theta\eta=1_M\). Thus \(\theta\) and \(\eta\) are inverse isomorphism
\end{proof}

\subsection*{Endomorphisms}
\label{sec:orgfaef9d5}
Let \(R\) be a ring, \(M\) a module. An \textbf{endomorphism} of \(M\) is a
homomorphism \(\alpha:M\to M\). The module of endomorphism \(\Hom(M,M)\) is also
denoted \(\End_R(M)\). \uline{Further}, \(\End_R(M)\) \uline{is a subring of} \(\End_{\Z}(M)\)

Given \(x\in R\), let \(\mu_x:M\to M\) denote the map of \textbf{multiplication} by
\(x\), defined by \(\mu_x(m):=xm\). It is an endomorphism. Further,
\(x\mapsto\mu_x\) is a ring map
\begin{equation*}
\mu_R:R\to\End_R(M)\subset\End_{\Z}(M)
\end{equation*}
(Thus we may view \(\mu_R\) as representing \(R\) as a ring of operators on
the abelian gorup). Note that \(\ker(\mu_R)=\Ann(M)\)

Conversely, given an abelian group \(N\) and a ring map
\begin{equation*}
\nu:R\to\End_{\Z}(N)
\end{equation*}
we obtain a module structure on \(N\) by setting \(xn:=(\nu x)(n)\). Then \(\mu_R=\nu\)

We call \(M\) \textbf{faithful} if \(\mu_R:R\to\End_R(M)\) is injective, or
\(\Ann(M)=0\). For example, \(R\) is a faithful \(R\)-module for \(x\cdot
   1=0\) implies


\subsection*{Algebras}
\label{sec:orgd266dc8}
Fix two rings \(R\) and \(R'\). Suppose \(R'\) is an \(R\)-algebra with
structure map \(\varphi\). Let \(M'\) be an \(R'\)-module.  Then \(M'\) is
also an \(R\)-module by \textbf{restriction on scalars}: \(xm:=\varphi(x)m\). In other
words, the \(R\)-module structure on \(M'\) corresponds to the composition
\begin{equation*}
R\xrightarrow{\varphi}R'\xrightarrow{\mu_{R'}}\End_{\Z}(M')
\end{equation*}

In particular, \(R'\) is an \(R\)-module; further, for all \(x\in R\) and
\(y,z\in R'\)
\begin{equation*}
(xy)z=x(yz)
\end{equation*}
by restriction on scalars

Conversely, suppose \(R'\) is an \(R\)-module s.t. \((xy)z=x(yz)\). Then
\(R'\) has an \(R\)-algebra structure that is compatible with the given
\(R\)-module structure.. Indeed, define \(\varphi:R\to R'\) by \(\varphi(x):=x\cdot1\).
Then \(\varphi(x)z=xz\) as \((x\cdot1)z=x(1\cdot z)\). So the composition
\(\mu_{R'}\varphi:R\to R'\to\End_{\Z}(R')\) is equal to \(\mu_R\). Hence \(\varphi\)
is a ring map. Thus \(R'\) is an \(R\)-algebra, and restriction of scalars
recovers its given \(R\)-module structure

Suppose that \(R'=R/\fa\) for some ideal \(\fa\). Then an \(R\)-module \(M\)
has a compatible \(R'\)-module structure iff \(\fa M=0\); if so, then the
\(R'\)-structure is unique. Indeed, the ring map \(\mu_R:R\to\End_{\Z}(M)\)
factors through \(R'\) iff \(\mu_R(\fa)=0\), so iff \(\fa M=0\)

Again suppose \(R'\) is an arbitrary \(R\)-algebra with structure map \(\varphi\). A
\textbf{subalgebra} \(R''\) of \(R'\) is a subring s.t. \(\varphi\) maps into \(R''\). The
subalgebra \textbf{generated} by \(x_1,\dots,x_n\in R'\) is the smallest
\(R\)-subalgebra that contains them. We denote it by \(R[x_1,\dots,x_n]\).

We say \(R'\) is a \textbf{finitely generated \(R\)-subalgebra} or is \textbf{algrbra finite}
\textbf{over \(R\)} if there exist \(x_1,\dots,x_n\in R'\) s.t. \(R'=R[x_1,\dots,x_n]\)

\subsection*{Residue modules}
\label{sec:org81dc87d}
Let \(R\) be a ring, \(M\) a module, \(M'\subset M\) a submodule. Form the
set of cosets
\begin{equation*}
M/M':=\{m+M'\mid m\in M\}
\end{equation*}
 \(M/M'\) inherits a module structure, and is called the \textbf{residue module} or
\textbf{quotient of \(M\) modulo \(M'\)}. Form the \textbf{quotient map}
\begin{equation*}
\kappa:M\to M/M'\quad\text{by}\quad
\kappa(m):=m+M'
\end{equation*}
Clearly \(\kappa\) is surjective, \(\kappa\) is linear, and \(\kappa\) has kernel \(M'\)

Let \(\alpha:M\to N\) be linear. Note that \(\ker(\alpha')\supset M'\) iff
\(\alpha(M')=0\)

If \(\ker(\alpha)\supset M'\), then there exists a homomorphism \(\beta:M/M'\to N\)
s.t. \(\beta\kappa=\alpha\)
\begin{center}
\begin{tikzcd}
M\arrow[r,"\kappa"]\arrow[rd,"\alpha"]&M/M'\arrow[d,"\beta"]\\
&N
\end{tikzcd}
\end{center}
Always
\begin{equation*}
M/\ker(\alpha)\similarrightarrow\im(\alpha)
\end{equation*}

\(M/M'\) has the following UMP: \(\kappa(M')=0\), and given \(\alpha:M\to N\) s.t.
\(\alpha(M')=0\), there is a unique homomorphism \(\beta:M/M'\to N\) s.t. \(\beta\kappa=\alpha\)

\subsection*{Cyclic modules}
\label{sec:org5dc93d1}
Let \(R\) be a ring. A module \(M\) is said to be \textbf{cyclic} if there exists
\(m\in  M\) s.t. \(M=Rm\). If so, form \(\alpha:R\to M\) by \(x\mapsto xm\); then
\(\alpha\) induces an isomorphism \(R/\Ann(m)\similarrightarrow M\). Note that
\(\Ann(m)=\Ann(M)\). Conversely, given any ideal \(\fa\), the \(R\)-module
\(R/\fa\) is cyclic, generated by the coset of 1, and \(\Ann(R/\fa)=\fa\)

\subsection*{Noether Isomorphisms}
\label{sec:orgb57d01c}
Let \(R\) be a ring, \(N\) a module, and \(L\) and \(M\) submodules.

First, assume \(L\subset M\subset N\). Form the following composition of
quotient maps:
\begin{equation*}
\alpha:N\to N/L\to (N/L)/(M/L)
\end{equation*}
\(\alpha\) is surjective and \(\ker(\alpha)=M\). Hence
\begin{center}
\begin{tikzcd}
N\arrow[r]\arrow[d]&N/M\arrow[d,"\beta","\simeq"']\\
N/L\arrow[r]&(N/L)/(M/L)
\end{tikzcd}
\end{center}

Second, let \(L+M\) denote the set of all sums \(l+m\) with \(l\in L\) and
\(m\in M\). Clearly \(L+M\) is a submodule of \(N\). It is called the \textbf{sum} of
\(L\) and \(M\)

Form the composition \(\alpha'\) of the inclusion map \(L\to L+M\) and the
quotient map \(L+M\to(L+M)/M\). Clearly \(\alpha'\) is surjective and
\(\ker(\alpha')=L\cap M\). Hence
\begin{center}
\begin{tikzcd}
L\arrow[r]\arrow[d]&L/(L\cap M)\arrow[d,"\beta'","\simeq"']\\
L+M\arrow[r]&(L+M)/M
\end{tikzcd}
\end{center}

\subsection*{Cokernels, coimages \label{4.9}}
\label{sec:orge497465}
Let \(R\) be a ring, \(\alpha:M\to N\) a linear map. Associated to \(\alpha\) are
its \textbf{cokernel} and its \textbf{coimage}
\begin{equation*}
\coker(\alpha):=N/\im(\alpha)\quad\text{ and }\quad
\coim(\alpha):=M/\ker(\alpha)
\end{equation*}
they are quotient modules, and their quotient maps are both denoted by \(\kappa\).

UMP of the cokernel: \(\kappa\alpha=0\) and given a map \(\beta:N\to P\) with
\(\beta:N\to P\) with \(\beta\alpha=0\), there is a unique map \(\gamma:\coker(\alpha)\to
   P\) with \(\gamma\kappa=\beta\)
\begin{center}
\begin{tikzcd}
M\arrow[r,"\alpha"]\arrow[rd]&N\arrow[d,"\beta"]\arrow[r,"\kappa"]
&\coker(\alpha)\arrow[ld,"\gamma"]\\
&P&
\end{tikzcd}
\end{center}

Further, \(\coim(\alpha)\similarrightarrow\im(\alpha)\)

\subsection*{Free modules}
\label{sec:org39b376c}
Let \(R\) be a ring, \(\Lambda\) a set, \(M\) a module. Given elements \(m_\lambda\in
   M\) for \(\lambda\in\Lambda\), by the submodule they \textbf{generate}, we mean the
smallest submodule that contains then all. Clearly, any submodule that
contains them all contains any (finite) linear combination \(\sum x_\lambda
   m_\lambda\) with \(x_\lambda\in R\)

\(m_\lambda\) are said to be \textbf{free} or \textbf{linearly independent} if whenever \(\sum
   x_\lambda m_\lambda=0\), also \(x_\lambda=0\) for all \(\lambda\). Finally, the
\(m_\lambda\) are said to form a \textbf{free basis} of \(M\) if they are free and
generate \(M\); if so, then we say \(M\) is \textbf{free} on the \(m_\lambda\)

We say \(M\) is \textbf{free} if it has a free basis. Any two free bases have the same
number \(l\) of elements, and we say \(M\) is \textbf{free of rank} \(l\)

For example, form the set of \textbf{restricted vectors}
\begin{equation*}
R^{\oplus\Lambda}:=\{(x_\lambda)\mid x_\lambda\in R\text{ with }x_\lambda=0
\text{ for almost all }\lambda\}
\end{equation*}
It's a module under componentwise addition and scalar multiplication. It has
a \textbf{standard basis}, which consists of the vectors \(e_\mu\) whose \(\lambda\)th
component is the value of the \textbf{Kronecker delta function}

If \(\Lambda\) has a finite number \(l\) of elements, then \(R^{\oplus\Lambda}\) is
often written \(R^l\) and called the \textbf{direct sum of \(l\) copies} of \(R\)

The free module \(R^{\oplus\Lambda}\) has the following UMP: given a module
\(M\) and elements \(m_\lambda\in M\) for \(\lambda\in\Lambda\), there is a
unique homomorphism
\begin{equation*}
\alpha:R^{\oplus\Lambda}\to M\text{ with }\alpha(e_\lambda)=m_\lambda
\text{ for each }\lambda\in\Lambda
\end{equation*}
namely, \(\alpha((x_\lambda))=\alpha(\sum x_\lambda e_\lambda)=sum x_\lambda
   m_\lambda\). Note the following obvious statements:
\begin{enumerate}
\item \(\alpha\) is surjective iff \(m_\lambda\) generate \(M\)
\item \(\alpha\) is injective iff \(m_\lambda\) are linearly independent
\item \(\alpha\) is an isomorphism iff \(m_\lambda\) for a free basis
\end{enumerate}


Thus \(M\) is free of rank \(l\) iff \(M\simeq R^l\)

\begin{exercise}
\label{4.12}
Take \(R:=\Z\) and \(M:=\Q\). Then any two \(x,y\in M\) are not free. Aso \(M\)
is not finitely generated. Indeed, given any \(m_1/n_1,\dots,m_r/n_r\in M\),
let \(d\) be a common multiple of \(n_1,\dots,n_r\). Then \((1/d)\Z\)
contains every linear combination but \((1/d)\Z\neq\Q\)
\end{exercise}

\begin{exercise}
\label{4.13}
Let \(R\) be a domain, and \(x\in R\) nonzero. Let \(M\) be the submodule of
\(\Frac(R)\) generated by \(1,x^{-1},x^{-2},\dots\). Suppose that \(M\) is
finitely generated. Prove that \(x^{-1}\in R\) and conclude that \(M=R\)
\end{exercise}

\begin{proof}
Suppose \(M\) is generated by \(m_1,\dots,m_k\). Say
\(m_i=\sum_{j=0}^{n_i}a_{ij}x^{-j}\) for some \(n_i\) and \(a_{ij}\in R\).
Set \(n:=\max\{n_i\}\). Then \(1,x^{-1},\dots,x^{-n}\) generate \(M\). So
\begin{equation*}
x^{-n+1}=a_nx^{-n}+\dots+a_0
\end{equation*}
Thus
\begin{equation*}
x^{-1}=a_n+\dots+a_0x^n
\end{equation*}
\end{proof}

\subsection*{Direct Products, Direct Sums}
\label{sec:orgb058ef3}
Let \(R\) be a ring, \(\Gamma\) a set, \(M_\lambda\) a module for
\(\lambda\in\Lambda\). The \textbf{direct product} of the \(M_\lambda\) is the set of
arbitrary vectors:
\begin{equation*}
\prod M_\lambda:=\{(m_\lambda)\mid m_\lambda\in M_\lambda\}
\end{equation*}
The \textbf{direct sum} of the \(M_\lambda\) is the subset of \textbf{restricted vectors}:
\begin{equation*}
\bigoplus M_\lambda:=\{(m_\lambda)\mid m_\lambda=0\text{ for almost all }\lambda\}
\subset\prod M_\lambda
\end{equation*}

The direct product comes equipped with projections
\begin{equation*}
\pi_\kappa:\prod M_\lambda\to M_\kappa\quad\text{given by}\quad
\pi_\kappa((m_\lambda)):=m_\kappa
\end{equation*}
\(\prod M_\lambda\) has UMP: given homomorphisms \(\alpha_\kappa:N\to
   M_\kappa\), there is a unique homomorphism \(\alpha:N\to\prod M_\lambda\)
satisfying \(\pi_\kappa\alpha=\alpha_\kappa\) for all \(\kappa\in\Lambda\);
namely \(\alpha(n)=(\alpha_\lambda(n))\). Often \(\alpha\) is denoted \((\alpha_\lambda)\).
In other words, the \(\pi_\lambda\) induce a bijection of sets
\begin{equation*}
\Hom(N,\prod M_\lambda)\similarrightarrow\prod\Hom(N,M_\lambda)
\end{equation*}

Similarly, the direct sum comes equipped with injections
\begin{equation*}
\iota_\kappa:M_\kappa\to\bigoplus M_\lambda\quad\text{given by}\quad
\iota_\kappa(m):=(m_\lambda)\text{ where }m_\lambda:=
\begin{cases}
m&\lambda=\kappa\\
0
\end{cases}
\end{equation*}
UMP: given homomorphisms \(\beta_\kappa:M_\kappa\to N\), there is a unique
homomorphism
\(\beta:\bigoplus M_\lambda\to N\) satisfying
\(\beta\iota_\kappa=\beta_\kappa\) for all \(\kappa\in\Lambda\) for all
\(\kappa\in\Lambda\); namely,
\(\beta((m_\lambda))=\sum\beta_\lambda(m_\lambda)\). Often \(\beta\) is denoted
\(\sum\beta_\lambda\); often \((\beta_\lambda)\). In other words, the
\(\iota_\kappa\) induce this bijection of sets:
\begin{equation*}
\Hom(\bigoplus M_\lambda,N)\similarrightarrow\prod\Hom(M_\lambda,N)
\end{equation*}

For example, if \(M_\lambda=R\) for all \(\lambda\), then \(\bigoplus
   M_\lambda=R^{\oplus\Lambda}\)

\begin{exercise}
\label{4.14}
Let \(\Lambda\) be an infinite set, \(R_\lambda\) a ring for \(\lambda\in\Lambda\).
Endow \(\prod R_\lambda\) and \(\bigoplus R_\lambda\) with componentwise
addition and multiplication. Show that \(\prod R_\lambda\) has a
multiplicative identity (so is a ring), but \(\bigoplus R_\lambda\) does not
(so is not a ring)
\end{exercise}

\begin{exercise}
\label{4.15}
Let \(L,M,N\) be modules. Consider a diagram
\begin{center}
\begin{tikzcd}
L\arrow[r,"\alpha",yshift=0.7ex]
&M\arrow[r,"\beta",yshift=0.7ex]\arrow[l,"\rho",yshift=-0.7ex]
&N\arrow[l,"\sigma",yshift=-0.7ex]
\end{tikzcd}
\end{center}
where \(\alpha\), \(\beta\), \(\rho\) and \(\sigma\) are homomorphisms. Prove that
\begin{equation*}
M=L\oplus N \quad\text{ and }\quad\alpha=\iota_L,\beta=\pi_N,\sigma=\iota_N,\rho=\pi_L
\end{equation*}
iff the following relations holds
\begin{equation*}
\beta\alpha=0,\beta\sigma=1,\rho\sigma=0,\rho\alpha=1,\alpha\rho+\sigma\beta=1
\end{equation*}
\end{exercise}

\begin{proof}
Consider the map \(\varphi:M\to L\oplus N\) and \(\theta:L\oplus N\to M\) given by
\(\varphi m:=(\rho m,\rho m)\) and \(\theta(l,n):=\alpha l+\sigma n\). They are inverse
isomorphism since
\begin{equation*}
\varphi\theta(l,n)=(\rho\alpha l+\rho\sigma n,\beta\alpha l+\beta\sigma n)=(l,n)
\quad\text{ and }\quad
\theta\varphi m=\alpha\rho m+\sigma\beta m=m
\end{equation*}
\end{proof}

\begin{exercise}
\label{4.16}
Let \(N\) be a module, \(\Lambda\) a nonempty set, \(M_\lambda\) a module for
\(\lambda\in\Lambda\). Prove that the injections
\(\iota_\kappa:M_\kappa\to\bigoplus M_\lambda\) induce an injection
\begin{equation*}
\bigoplus\Hom(N,M_\lambda)\hookrightarrow\Hom(N,\bigoplus M_\lambda)
\end{equation*}
and that it is an isomorphism if \(N\) is finitely generated
\end{exercise}

\begin{proof}
For \((\beta_\kappa)\in\bigoplus\Hom(N,M_\lambda)\)
\begin{equation*}
\beta(n)=
\begin{cases}
\iota_\kappa\beta_\kappa&\text{if }\beta_\kappa\neq0\\
0&\beta_\kappa=0
\end{cases}
\in\Hom(N,\bigoplus M_\lambda)
\end{equation*}
If \(N\) is fintitely generated, suppose \(a_1,\dots,a_n\) generates \(N\)
and \(\beta(a_i)=b_i\in\bigoplus M_\lambda\), which means \(\beta(N)\) is a finite
direct subsum of \(\bigoplus M_\lambda\).
then we have
\(\beta_\kappa=\pi_\kappa\beta\) and almost
\end{proof}

\begin{exercise}
\label{4.17}
Let \(\fa\) be an ideal, \(\Lambda\) a nonempty set, \(M_\lambda\) a module for
\(\lambda\in\Lambda\). Prove \(\fa(\bigoplus M_\lambda)=\bigoplus\fa
   M_\lambda\). Prove \(\fa(\prod M_\lambda)=\prod\fa M_\lambda\) if \(\fa\) is
finitely generated
\end{exercise}


\section{Exact Sequence}
\label{sec:org5b9646b}
\begin{definition}[]
A (finite or infinite) sequence of module homomorphisms
\begin{equation*}
\cdots\to M_{i-1}\yrightarrow{\alpha_{i-1}}M_i\yrightarrow{\alpha_i}M_{i+1}\to\cdots
\end{equation*}
is said to be \textbf{exact at} \(M_i\) if \(\ker(\alpha_i)=\im(\alpha_{i-1})\).. The
sequence is said to be \textbf{exact} if it is exact at every \(M_i\), except an
initial source or final target
\end{definition}

\begin{examplle}[]
\label{5.2}
\begin{enumerate}
\item A sequence \(0\to L\yrightarrow{\alpha} M\) is exact iff \(\alpha\) is injective. If
so, then we often identify \(L\) with its image \(\alpha(L)\)

\textbf{Dually} - a sequence \(M\yrightarrow{\beta}N\to0\) is exact iff \(\beta\) is surjective

\item A sequence \(0\to L\yrightarrow{\alpha}M\yrightarrow{\beta}N\) is exact iff
\(L=\ker(\beta)\), where '\(\equal\)' means ``canocially isomorphic''. Dually, a sequence
\(L\yrightarrow{\alpha}M\yrightarrow{\beta}N\to0\) is exact iff \(N=\coker(\alpha)\)
\end{enumerate}
\end{examplle}
\subsection*{Short exact sequences}
\label{sec:org735e005}
A sequence \(0\to L\yrightarrow{\alpha}M\xrightarrow{\beta}N\to0\) is exact iff
\(\alpha\) is injective and \(N=\coker(\alpha)\), or dually, iff \(\beta\) is surjective and
\(L=\ker(\beta)\). If so, then the sequence is called \textbf{short exact}, and often we
regard \(L\) as a submodule of \(M\), and \(N\) as the quotient \(M/L\)

For example, the following sequence is shor t exact
\begin{equation*}
0\to L\yrightarrow{\iota_L}L\oplus N\yrightarrow{\pi_N}N\to0
\end{equation*}

\begin{proposition}[]
\label{5.4}
For \(\lambda\in\Lambda\), let \(M'_\lambda\to M_\lambda\to M_\lambda''\) be a
sequence  of module homomorphisms. If every sequence is exact, then so are
the two induced sequences
\begin{equation*}
\bigoplus M'_\lambda\to\bigoplus M_\lambda\to\bigoplus M_\lambda''
\quad\text{ and }\quad
\prod M'_\lambda\to \prod M_\lambda\to\prod M_\lambda''
\end{equation*}
Conversely, if either induced sequence is exact then so is every original one
\end{proposition}

\begin{exercise}
\label{5.5}
Let \(M'\) and \(M''\) be modules, \(N\subset M'\) a submodule. Set
\(M:=M'\oplus M''\). Prove \(M/N=M'/N\oplus M''\)
\end{exercise}

\begin{proof}
\(N=N\oplus 0\)

The two sequence \(0\to M''\to M''\to0\) and \(0\to N\to M'\to M'/N\to 0\)
are exact. So by \ref{5.4}, the sequence
\begin{equation*}
0\to N\to M'\oplus M''\to(M'/N)\oplus M''\to 0
\end{equation*}
is exact
\end{proof}

\begin{exercise}
Let \(0\to M'\to M\to M''\to 0\) be a short exact sequence. Prove the if
\(M'\) and \(M''\) are finitely generated, then so is \(M\)
\end{exercise}

\begin{lemma}[]
Let \(0\to M'\yrightarrow{\alpha}M\yrightarrow{\beta}M''\to0\) be a short exact
sequence, and \(N\subset M\) a submodule. Set \(N':=\alpha^{-1}\) and
\(N'':=\beta(N)\). Then the induced sequence \(0\to N'\to N\to N''\to 0\) is
short exact
\end{lemma}

\begin{definition}[]
We say that a short exact sequence
\begin{equation*}
0\to M'\yrightarrow{\alpha}M\yrightarrow{\beta}M''\to0
\end{equation*}
\textbf{splits} if there is an isomorphism \(\varphi:M\similarrightarrow M'\oplus M''\) with
 \(\varphi\alpha=\iota_{M'}\) and \(\beta=\pi_{M''}\varphi\)

We call a homomorphism \(\rho:M\to M'\) a \textbf{retraction} of \(\alpha\) if
\(\rho\alpha=1_{M'}\)

Dually, we call a homomorphism \(\sigma:M''\to M\) a \textbf{section} of \(\beta\) if \(\beta\sigma=1_{M''}\)
\end{definition}

\begin{proposition}[]
\label{5.9}
Let \(0\to M'\yrightarrow{\alpha}M\yrightarrow{\beta}M''\to0\) be a short exact
sequence. Then the following conditions are equivalent
\begin{enumerate}
\item The sequence splits
\item There exists a retraction
\item There exists a section
\end{enumerate}
\end{proposition}

\begin{proof}
Assume (2). Set \(\sigma':=1_M-\alpha\rho\). Then \(\sigma'\alpha=0\). So
there exists \(\sigma:M''\to M\) with \(\sigma\beta=\sigma'\) by \ref{5.2} and UMP.
So \(1_M=\alpha\rho+\sigma\beta\). Since \(\beta\sigma\beta=\beta\) and \(\beta\) is
surjective, \(\beta\sigma=1_{M''}\). Hence \(\alpha\rho\sigma=0\). Since \(\alpha\) is
injective, \(\rho\sigma=0\). Thus \ref{4.15} yields (1) and also (3)
\end{proof}

\begin{exercise}
\label{5.10}
Let \(M',M''\) be modules, and set \(M:=M'\oplus M''\). Let \(N\) be a
submodule of \(M\) containing \(M'\), and set \(N'':=N\cap M''\). Prove
\(N=M'\oplus N''\)
\end{exercise}

\begin{proof}
Form the sequence \(0\to M'\to N\to\pi_{M''}N\to0\). It splits by \ref{5.9}  as
\((\pi_{M'}|N)\circ\iota_{M'}=1_{M'}\). Finally if \((m',m'')\in N\), then
\((0,m'')\in N\) as \(M'\subset N\); hence \(\pi_{M''}N=N''\)
\end{proof}

\begin{exercise}
\label{5.11} Criticize the following misstatement of \ref{5.9}: given a short
exactg sequence \(0\to M'\xrightarrow{\alpha}M\xrightarrow{\beta}M''\to0\), there is
an isomorphism \(M\simeq M'\oplus M''\) iff there is a section \(\sigma:M''\to M\)
of \(\beta\)
\end{exercise}

\begin{proof}
We have \(\alpha:M'\to M\) and \(\iota_{M'}:M'\oplus M''\), but \ref{5.9} requires
that they be compatible with the isomorphism \(M\simeq M'\oplus M''\).

Let's construct a counterexample. For each integer \(n\ge2\), let \(M_n\) be
the direct sum of countably many copies of \(\Z/\la n\ra\). Set
\(M:=\bigoplus M_n\)

First let us check these two statements:
\begin{enumerate}
\item For any finite abelian group \(G\), we have \(G\oplus M\simeq M\)
\item For any finite abelian subgroup \(G\subset M\), we have \(M/G\simeq M\)
\end{enumerate}


Statement (1) holds since \(G\) is isomorphic to a direct sum of copies of
\(\Z/\la n\ra\)

To prove (2), write \(M=B\oplus M'\), where \(B\) contains \(G\) and involes
only finitely many components of \(M\). Then \(M'\simeq M\). Therefore,
\ref{5.10} yields
\begin{equation*}
M/G\simeq(B/G)\oplus M'\simeq M
\end{equation*}

To construct the counterexample, let \(p\) be a prime number. Take one of the
\(\Z/\la p^2\ra\) components of \(M\), and let \(M'\subset\Z/\la p^2\ra\) be
the cyclic subgroup of order \(p\). There is no retraction \(\Z/\la p^2\ra\to
   M'\), so there is no traction \(M\to M'\) either, since the latter would
induce the former. Finally take \(M'':=M/M'\). Then (1) and (2) yield
\(M\simeq M'\oplus M''\)
\end{proof}

\begin{lemma}[Snake]
Consider this commutative diagram with exact rows:
\begin{center}
\begin{tikzcd}
&M'\arrow[r,"\alpha"]\arrow[d,"\gamma'"]&M\arrow[r,"\beta"]
\arrow[d,"\gamma"]&M''\arrow[r]\arrow[d,"\gamma''"]&0\\
0\arrow[r]&N'\arrow[r,"\alpha'"]&N\arrow[r,"\beta'"]&N''
\end{tikzcd}
\end{center}
It yields the following exact sequence
\begin{equation*}
\ker(\gamma')\xrightarrow{\varphi}\ker(\gamma)\xrightarrow{\psi}\ker(\gamma'')
\xrightarrow{\partial}\coker(\gamma')\xrightarrow{\varphi'}\coker(\gamma)
\xrightarrow{\psi'}\coker(\gamma'')
\end{equation*}
Moreover, if \(\alpha\) is injective, then so is \(\varphi\); dually, if \(\beta'\) is
surjective, then so is \(\psi'\)
\end{lemma}

\begin{proof}
Clearly, \(\alpha\) yields a unique compatible homomorphism
\(\ker(\gamma')\to\ker(\gamma)\) since \(\gamma\alpha(\ker(\gamma'))=0\). By the
UMP in \ref{4.9}, \(\alpha'\) yields a unique compatible homomorphism \(\varphi'\) because
\(M'\) goes to 0 in \(\coker(\gamma)\).
\begin{center}
\begin{tikzcd}
M'\arrow[r,"\gamma'"]&N'\arrow[r]\arrow[ld,"\alpha'"]\arrow[d]&
\coker(\gamma')\arrow[ld]\\
N\arrow[r]&\coker(\gamma)
\end{tikzcd}
\end{center}
Similarly, \(\beta\) and \(\beta'\) induce corresponding homomorphisms \(\psi\) and
\(\psi'\)

To define \(\partial\), \textbf{chase} an \(m''\in\ker(\gamma'')\) through the diagram
\end{proof}
\end{document}
