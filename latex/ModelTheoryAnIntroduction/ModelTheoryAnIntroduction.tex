% Created 2019-12-18 三 20:55
% Intended LaTeX compiler: pdflatex
\documentclass[11pt]{article}
\usepackage[utf8]{inputenc}
\usepackage[T1]{fontenc}
\usepackage{graphicx}
\usepackage{grffile}
\usepackage{longtable}
\usepackage{wrapfig}
\usepackage{rotating}
\usepackage[normalem]{ulem}
\usepackage{amsmath}
\usepackage{textcomp}
\usepackage{amssymb}
\usepackage{capt-of}
\usepackage{hyperref}
\usepackage{minted}
% TIPS
% \substack{a\\b} for multiple lines text





% pdfplots will load xolor automatically without option
\usepackage[dvipsnames]{xcolor}

\usepackage{forest}
% two-line text in node by [two \\ lines]
% \begin{forest} qtree, [..] \end{forest}
\forestset{
  qtree/.style={
    baseline,
    for tree={
      parent anchor=south,
      child anchor=north,
      align=center,
      inner sep=1pt,
    }}}
%\usepackage{flexisym}
% load order of mathtools and mathabx, otherwise conflict overbrace

\usepackage{mathtools}
%\usepackage{fourier}
\usepackage{pgfplots}
\usepackage{amsthm, mathabx,  amsmath, commath}
\usepackage{amsfonts}

\usepackage{empheq}
\usepackage{tikz}
\usetikzlibrary{arrows.meta}
\usepackage[most]{tcolorbox}

\newtheorem{theorem}{Theorem}[section]
\newtheorem{definition}{Definition}[section]
\newtheorem{corollary}{Corollary}[section]
\newtheorem{example}{Example}[section]
\newtheorem{lemma}{Lemma}[section]
\newtheorem{proposition}{Proposition}[section]

\newcommand{\bl}[1] {\boldsymbol{#1}}
\newcommand{\Wt}[1] {\stackrel{\sim}{\smash{#1}\rule{0pt}{1.1ex}}}
\newcommand{\wt}[1] {\widetilde{#1}}


%For boxed texts in align, use Aboxed{}
%otherwise use boxed{}

\DeclareMathSymbol{\widehatsym}{\mathord}{largesymbols}{"62}
\newcommand\lowerwidehatsym{%
  \text{\smash{\raisebox{-1.3ex}{%
    $\widehatsym$}}}}
\newcommand\fixwidehat[1]{%
  \mathchoice
    {\accentset{\displaystyle\lowerwidehatsym}{#1}}
    {\accentset{\textstyle\lowerwidehatsym}{#1}}
    {\accentset{\scriptstyle\lowerwidehatsym}{#1}}
    {\accentset{\scriptscriptstyle\lowerwidehatsym}{#1}}
}

\usepackage{graphicx}
    
% text on arrow for xRightarrow
\makeatletter
%\newcommand{\xRightarrow}[2][]{\ext@arrow 0359\Rightarrowfill@{#1}{#2}}
\makeatother


\def \bx {\boldsymbol{x}}
\def \ba {\boldsymbol{a}}
\def \bI {\boldsymbol{I}}
\def \bt {\boldsymbol{t}}
\def \bb {\boldsymbol{b}}
\def \bA {\boldsymbol{A}}
\def \bX {\boldsymbol{X}}
\def \bu {\boldsymbol{u}}
\def \bS {\boldsymbol{S}}
\def \bZ {\boldsymbol{Z}}
\def \bz {\boldsymbol{z}}
\def \by {\boldsymbol{y}}
\def \bw {\boldsymbol{w}}
\def \bT {\boldsymbol{T}}
\def \bS {\boldsymbol{S}}
\def \bm {\boldsymbol{m}}
\def \bW {\boldsymbol{W}}
\def \bY {\boldsymbol{Y}}
\def \bH {\boldsymbol{H}}
\def \blambda {\boldsymbol{\lambda}}
\def \bPhi {\boldsymbol{\Phi}}
\def \btheta {\boldsymbol{\theta}}
\def \bmu {\boldsymbol{\mu}}
\def \bphi {\boldsymbol{\phi}}
\def \bSigma {\boldsymbol{\Sigma}}
\def \lb {\left\{}
\def \rb {\right\}}
\def \caln {\mathcal{N}}
\def \dissum {\displaystyle\Sigma}
\def \dispro {\displaystyle\prod}
\def \E {\mathbb{E}}
\def \Q {\mathbb{Q}}
\def \V {\mathbb{V}}
\def \R {\mathbb{R}}
\def \calq {\mathcal{Q}}
\def \calg {\mathcal{G}}
\def \caln {\mathcal{N}}
\def \calr {\mathcal{R}}
\def \calm {\mathcal{M}}
\def \calc {\mathcal{C}}
\def \bcup {\bigcup}

\author{David Marker}
\date{\today}
\title{Model Theory: An Introduction}
\hypersetup{
 pdfauthor={David Marker},
 pdftitle={Model Theory: An Introduction},
 pdfkeywords={},
 pdfsubject={},
 pdfcreator={Emacs 26.3 (Org mode 9.3)}, 
 pdflang={English}}
\begin{document}

\maketitle
\tableofcontents \clearpage\section{Structures and Theories}
\label{sec:org2193cdd}
\subsection{Languages and Structures}
\label{sec:org20dc365}
\begin{definition}[]
A language \(\call\) is given by specifying the following data
\begin{enumerate}
\item A set of function symbols \(\calf\) and positive integers \(n_f\) for each
\(f\in\calf\)
\item a set of relation symbols \(\calr\) and positive integers \(n_R\) for each
\(R\in\calr\)
\item a set of constant symbols \(\calc\)
\end{enumerate}
\end{definition}

\begin{definition}[]
An \(\call\)-structure \(\calm\) is given by the following data
\begin{enumerate}
\item a nonempty set \(M\) called the \tf{universe}, \tf{domain} or \tf{underlying set}
of \(\calm\)
\item a function \(f^\calm:M^{n_f}\to M\) for each \(f\in\calf\)
\item a set \(R^\calm\subseteq M^{n_R}\) for each \(R\in\calr\)
\item an element \(c^\calm\in M\) for each \(c\in\calc\)
\end{enumerate}
\end{definition}

We refer to \(f^\calm,R^\calm,c^\calm\) as the \tf{interpretations} of the
symbols \(f,R\) and \(c\). We often write the structure as
\(\calm=(M,f^\calm,R^\calm,c^\calm:f\in\calf,R\in\calr,c\in\calc)\) 

\begin{definition}[]
Suppose that \(\calm\) and \(\caln\) are \(\call\)-structures with universes \(M\)
and \(N\) respectively. An \(\call\)-\tf{embedding} \(\eta:\calm\to\caln\) is a
one-to-one map \(\eta:M\to N\) that
\begin{enumerate}
\item \(\eta(f^\calm(a_1,\dots,a_{n_f}))=f^\caln(\eta(a_1),\dots,\eta(a_{n_f}))\)
for all \(f\in\calf\) and \(a_1,\dots,a_{n_f}\in M\)
\item \((a_1,\dots,a_{m_R})\in R^\calm\) if and only if
\((\eta(a_1),\dots,\eta(a_{m_R}))\in R^\caln\) for all \(R\in\calr\) and
\(a_1,\dots,a_{m_R}\in M\)
\item \(\eta(c^\calm)=c^\caln\) for \(c\in\calc\)
\end{enumerate}
\end{definition}

A bijective \(\call\)-embedding is called an \(\call\)-\tf{isomorphism}. If
\(M\subseteq N\) and the inclusion map is an \(\call\)-embedding, we say either
\(\calm\) is a \tf{substrcture} of \(\caln\) or that \(\caln\) is an \tf{extension}
of \(\calm\)

The \tf{cardinality} of \(\calm\) is \(\abs{M}\)

\begin{definition}[]
The set of \(\call\)-\tf{terms} is the smallest set \(\calt\) s.t.
\begin{enumerate}
\item \(c\in\calt\) for each constant symbol \(c\in\calc\)
\item each variable symbol \(v_i\in\calt\) for \(i=1,2,\dots\)
\item if \(t_1,\dots,t_{n_f}\in\calt\) and \(f\in\calf\) then
\(f(t_1,\dots,n_{n_f})\in\calt\)
\end{enumerate}
\end{definition}


Suppose that \(\calm\) is an \(\call\)-structure and that \(t\) is a term built
using variables from \(\bar{v}=(v_{i_1},\dots,v_{i_m})\). We want to interpret
\(t\) as a function \(t^\calm:M^m\to M\). For \(s\) a subterm of \(t\) and
\(\bar{a}=(a_{i_1},\dots,a_{i_m})\in M\), we inductively define
\(s^\calm(\bar{a})\) as follows.
\begin{enumerate}
\item If \(s\) is a constant symbol \(c\), then \(s^\calm(\bar{a})=c^\calm\)
\item If \(s\) is the variable \(v_{i_j}\), then \(s^\calm(\bar{a})=a_{i_j}\)
\item If \(s\) is the term \(f(t_1,\dots,t_{n_f})\), where \(f\) is a function symbol
of \(\call\) and \(t_1,\dots,t_{n_f}\) are terms, then 
\(s^\calm(\bar{a})=f^\calm(t^\calm_1(\bar{a}),\dots,t_{n_f}^\calm(\bar{a}))\)
\end{enumerate}


The function \(t^\calm\) is defined by \(\bar{a}\mapsto t^\calm(\bar{a})\)


\begin{definition}[]
\(\phi\) is an \tf{atomic} \(\call\)-\tf{formula} if \(\phi\) is either
\begin{enumerate}
\item \(t_1=t_2\) where \(t_1\) and \(t_2\) are terms
\item \(R(t_1,\dots,t_{n_R})\)
\end{enumerate}


The set of \(\call\)-\tf{formulas} is the smallest set \(\calw\) containing the
atomic formulas s.t.
\begin{enumerate}
\item if \(\phi\in\calw\), then \(\neg\phi\in\calw\)
\item if \(\phi,\psi\in\calw\), then \((\phi\wedge\psi),(\phi\vee\psi)\in\calw\)
\item if \(\phi\in\calw\), then \(\exists v_i\phi,\forall v_i\phi\in\calw\)
\end{enumerate}
\end{definition}

We say a variable \(v\) \tf{occurs freely} in a formula \(\phi\) if it is not
inside a \(\exists v\) or \(\forall v\) quantifier; otherwise we say that it's
\tf{bound}. We call a formula a \tf{sentence} if it has no free variables. We
often write \(\phi(v_1,\dots,v_n)\) to make explicit the free variables in \(\phi\)

\begin{definition}[]
Let \(\phi\) be a formula with free variables from
\(\bar{v}=(v_{i_1,\dots,v_{i_m}})\) and let \(\bar{a}=(a_{i_1},\dots,a_{i_m})\in
   M^m\). We inductively define \(\calm\models\phi\bar{a}\) as follows
\begin{enumerate}
\item If \(\phi\) is \(t_1=t_2\), then \(\calm\models\phi(\bar{a})\) if
\(t_1^\calm(\bar{a})=t_2^\calm(\bar{a})\)
\item If \(\phi\) is \(R(t_1,\dots,t_{m_R})\) then \(\calm\models\phi(\bar{a})\) if
\((t_1^\calm(\bar{a}),\dots,t_{m_R}^\calm(\bar{a}))\in R^\calm\)
\item If \(\phi\) is \(\neg\psi\) then \(\calm\models\phi(\bar{a})\) if
\(\calm\not\models\psi(\bar{a})\)
\item If \(\phi\) is \((\psi\wedge\theta)\) then \(\calm\models\phi(\bar{a})\) if
\(\calm\models\psi(\bar{a})\) and
\(\calm\models\theta(\bar{a})\)
\item If \(\phi\) is \((\psi\vee\theta)\) then \(\calm\models\phi(\bar{a})\) if
\(\calm\models\psi(\bar{a})\) or
\(\calm\models\theta(\bar{a})\)
\item If \(\phi\) is \(\exists v_j\psi(\bar{v},v_j)\) then \(\calm\models\phi(\bar{a})\)
if there is \(b\in M\) s.t. \(\calm\models\psi(\bar{a},b)\)
\item If \(\phi\) is \(\forall v_j\psi(\bar{v},v_j)\) then \(\calm\models\phi(\bar{a})\)
if \(\calm\models\psi(\bar{a},b)\) for all \(b\in M\)
\end{enumerate}
\end{definition}


If \(\calm\models\phi(\bar{a})\) we say that \(\calm\) \tf{satisfies}
\(\phi(\bar{a})\) or \(\phi(\bar{a})\) is \tf{true} in \(\calm\)

\begin{proposition}[]
Suppose that \(\calm\) is a substrcture of \(\caln\), \(\bar{a}\in M\) and
\(\phi(\bar{v})\) is a quantifier-free formula. Then
\(\calm\models\phi(\bar{a})\) if and only if \(\caln\models\psi(\bar{a})\)
\end{proposition}

\begin{proof}
\tf{Claim} If \(t(\bar{v})\) is a term and \(\bar{b}\in M\) then
\(t^\calm(\bar{b})=t^\caln(\bar{b})\). 
\end{proof}

\begin{definition}[]
We say that two \(\call\)-strctures \(\calm\) and \(\caln\) are \tf\{elementarily
equivalent\} and write \(\calm\equiv\caln\) if
\begin{equation*}
 \calm\models\phi\text{ if and only if } \caln\models\phi
\end{equation*}
for all \(\call\)-sentences \(\phi\)
\end{definition}

We let \(\th(\calm)\), the \tf{full theory} of \(\calm\) be the set of
\(\call\)-sentences \(\phi\) s.t. \(\calm\models\phi\)

\begin{theorem}[]
Suppose that \(j:\calm\to\caln\) is an isomorphism. Then \(\calm\equiv\caln\)
\end{theorem}

\begin{proof}
Show by induction on formulas that \(\calm\models\phi(a_1,\dots,a_n)\) if and
only if \(\caln\models\phi(j(a_1),\dots,j(a_n))\) for all formulas \(\phi\)
\end{proof}
\subsection{Theories}
\label{sec:org122036b}
Let \(\call\) be a language. An \(\call\)-\tf{theory} \(T\) is a set of
\(\call\)-sentences. We say that \(\calm\) is a \tf{model} of \(T\) and write
\(\calm\models T\) if \(\calm\models\phi\) for all sentences \(\phi\in T\). A
theory is \tf{satisfiable} if it has a model.

A class of \(\call\)-structures \(\calk\) is an \tf{elementary class} if there
is an \(\call\)-theory \(T\) s.t. \(\calk=\{\calm:\calm\models T\}\)

\begin{example}[Groups]
Let $\call=\{\cdot,e\}$ where $\cdot$ is a binary function symbol and $e$ is a 
constant symbol. The class of groups is axiomatized by
\begin{align*}
&\forall x\;e\cdot x=x\cdot e=x\\
&\forall x\forall y\forall z\;x\cdot(y\cdot z)=(x\cdot y)\cdot z\\
&\forall x\exists y\;x\cdot y=y\cdot x= e
\end{align*}
\end{example}

\begin{example}[Rings and Fields]
Let $\call_r$ be the language of rings $\{+,-,\cdot,0,1\}$, where $ +,-$ and $\cdot$
are binary function symbols and $0$ and $1$ are constants. The axioms for rings are given 
by
\begin{align*}
&\forall x\forall y\forall z\;(x-y=z\leftrightarrow x=y+z)\\
&\forall x\;x\cdot 0=0\\
&\forall x\forall y\forall z\;x\cdot(y\cdot z)=(x\cdot y)\cdot z\\
&\forall x\;x\cdot 1=1\cdot x=x\\
&\forall x\forall y\forall z\;x\cdot(y+z)=(x\cdot y)+(x\cdot z)\\
&\forall x\forall y\forall z\;(x+y)\cdot z=(x\cdot z)+(y\cdot z)
\end{align*}
We axiomatize the class of fields by adding
\begin{align*}
&\forall x\forall y\;x\cdot y=y\cdot x\\
&\forall x\;(x\neq 0\to\exists y\;x\cdot y=1)
\end{align*}
We axiomatize the class of algebraically closed fields by adding to the field axioms the sentences
\begin{equation*}
\forall a_0\dots\forall a_{n-1}\exists x\;x^n+\displaystyle\sum_{i=1}^{n-1}
a_ix^i=0
\end{equation*}
for $n=1,2,\dots$. Let \acf\; be the axioms for algebraically closed fields.

Let $\psi_p$ be the \(\call_r\)-sentence $\forall x\;
\underbrace{x+\dots+x}_{p\text{-times}}=0$, which asserts that a field has characteristic
$p$. For $p>0$ a prime, let $\acf_p=\acf\cup\{\psi_p\}$ and
$\acf_0=\acf\cup\{\neg\psi_p:p>0\}$ be the theories of algebraically
closed fields of characteristic $p$ and zero respectively
\end{example}

\begin{definition}[]
Let \(T\) be an \(\call\)-theory and \(\phi\) an \(\call\)-sentence. We say that
\(\phi\) is a \tf{logical consequence} of \(T\) and write \(T\models\phi\) if
\(\calm\models\phi\) whenever \(\calm\models T\)
\end{definition}

\begin{proposition}[]
\begin{enumerate}
\item Let \(\call=\{+,<,0\}\) and let \(T\) be the theory of ordered abelian groups.
Then \(\forall x(x\neq 0\to x+x\neq 0)\) is a logical consequence of \(T\)
\item Let \(T\) be the theory of groups where every element has order 2. Then\par
\(T\not\models\exists x_1\exists x_2\exists x_3(x_1\neq x_2\wedge
      x_2\neq x_3\wedge x_1\neq x_3)\)
\end{enumerate}
\end{proposition}

\begin{proof}
\begin{enumerate}
\item \(\Z/2\Z\models T\wedge\neg\exists x_1\exists x_2\exists x_3(x_1\neq x_2\wedge
      x_2\neq x_3\wedge x_1\neq x_3)\)
\end{enumerate}
\end{proof}
\subsection{Definable Sets and Interpretability}
\label{sec:org348d9e9}
\begin{definition}[]
Let \(\calm=(M,\dots)\) be an \(\call\)-structure. We say that \(X\subseteq M^n\)
is \tf{definable} if and only if there is an \(\call\)-formula 
\(\phi(v_1,\dots,v_n,w_1,\dots,w_m)\) and \(\bar{b}\in M^b\) s.t. 
\(X=\{\bar{a}\in M^n:\calm\models\phi(\bar{a},\bar{b})\}\). We say that
\(\phi(\bar{v},\bar{b})\) \tf{defines} \(X\). We say that \(X\) is
\(A\)-\tf{definable} or \tf{definable over}  \(A\) if there is a formula 
\(\psi(\bar{v},w_1,\dots,w_l)\) and \(\bar{b}\in A^l\) s.t.
\(\psi(\bar{v},\bar{b})\) defines \(X\)
\end{definition}

A number of examples using \(\call_r\), the language of rings
\begin{itemize}
\item Let \(\calm=(R,+,-,\cdot,0,1)\) be a ring. Let \(p(X)\in R[X]\). Then 
\(Y=\{x\in R:p(x)=0\}\) is definable. Suppose that
\(p(X)=\displaystyle\sum_{i=0}^ma_iX^i\). Let \(\phi(v,w_0,\dots,w_n)\) be the
formula
\begin{equation*}
w_n\cdot\underbrace{v\cdots v}_{n\text{-times}}+\dots+w_1\cdot v+w_0=0
\end{equation*}
Then \(\phi(v,a_0,\dots,a_n)\) defines \(Y\). Indeed, \(Y\) is \(A\)-definable
for any \(A\supseteq\{a_0,\dots,a_n\}\)
\item Let \(\calm=(\R,+,-,\cdot,0,1)\) be the field of real numbers. Let
\(\phi(x,y)\) be the formula 
\begin{equation*}
\exists z(z\neq 0\wedge y=x+z^2)
\end{equation*}
Because \(a<b\) if and only if \(\calm\models\phi(a,b)\), the ordering is
\(\emptyset\)-definable
\item Consider the natural numbers \(\N\) as an \(\call=\{+,\cdot,0,1\}\) structure.
There is an \(\call\)-formula \(T(e,x,s)\) s.t. \(\N\models T(e,x,s)\) if and
only if the Turing machine with program coded by \(e\) halts on input \(x\) in
at most \(s\) steops. Thus the Turing machine with program \(e\) halts on input
\(x\) if and only if
\end{itemize}
\(\N\models\exists s\;T(e,x,s)\). So the halting
    computations is definable


\begin{proposition}[]
Let \(\calm\) be an \(\call\)-structure. Suppose that \(D_n\) is a collection of
subsets of \(M^n\) for all \(n\ge 1\) and \(\cald=(D_n:n\ge 1)\) is the smallest
collection s.t. 
\begin{enumerate}
\item \(M^n\in D_n\)
\item for all \(n\)-ary function symbols \(f\) of \(\call\), the graph of \(f^\calm\)
is in \(D_{n+1}\)
\item for all \(n\)-ary relation symbols \(R\) of \(\call\), \(R^\calm\in D_n\)
\item for all \(i,j\le n\), \(\{(x_1,\dots,x_n)\in M^n:x_i=x_j\}\in D_n\)
\item if \(X\in D_n\), then \(M\times X\in D_{n+1}\)
\item each \(D_n\) is cloed under complement, union and intersection
\item if \(X\in D_{n+1}\) and \(\pi:M^{n+1}\to M^n\) is the projection 
\((x_1,\dots,x_{n+1})\mapsto(x_1,\dots,x_n)\), then \(\pi(X)\in D_n\)
\item if \(X\in D_{n+m}\) and \(b\in M^m\), then \(\{a\in M^n:(a,b)\in X\}\in D_n\)
\end{enumerate}


Thus \(X\subseteq M^n\) is definable if and only if \(X\in D_n\)
\end{proposition}

\begin{proposition}[]
Let \(\calm\) be an \(\call\)-structure. If \(X\subset M^n\) is \(A\)-definable,
then every \(\call\)-automorphism of \(\calm\) that fixes \(A\) pointwise fixes
\(X\) setwise(that is, if \(\sigma\) is an automorphism of \(M\) and \(\sigma(a)=a\)
for all \(a\in A\), then \(\sigma(X)=X\))
\end{proposition}

\begin{proof}
\begin{equation*}
\calm\models\psi(\bar{v},\bar{a})\leftrightarrow
\calm\models\psi(\sigma(\bar{v}),\sigma(\bar{a}))\leftrightarrow
\calm\models\psi(\sigma(\bar{v}),\bar{a})
\end{equation*}
In other words, \(\bar{b}\in X\) if and only if \(\sigma(\bar{b})\in X\)
\end{proof}

\begin{definition}[]
A subset \(S\) of a field \(L\) is \tf{algebraically independent} over a
subfield \(K\) if the elements of 
\(S\) do not satisfy any non-trivial polynomial equation with
coefficients in \(K\) 
\end{definition}


\begin{corollary}[]
The set of real numbers is not definable in the field of complex numbers
\end{corollary}

\begin{proof}
If \(\R\) where definable, then it would be definable over a finite
\(A\subset\C\). Let \(r,s\in\C\) be algebraically independent over \(A\) with
\(r\in\R\) and \(s\not\in\R\). There is an automorphism \(\sigma\) of \(\C\) s.t.
\(\sigma|A\) is the identity and \(\sigma(r)=s\). Thus \(\sigma(\R)\neq\R\) and
\(\R\) is not definable over \(A\)
\end{proof}

We say that an \(\call_0\)-structure \(\caln\) is \tf{definably interpreted} in
an \(\call\)-structure \(\calm\) if and only if we can find a definable
\(X\subseteq M^n\) for some \(n\) and we can interpret the symbols of \(\call_0\)
as definable subsets and functions on \(X\) so that the resulting
\(\call_0\)-structure is isomorphic to \(\calm\)
\end{document}