% Created 2021-09-16 Thu 13:18
% Intended LaTeX compiler: pdflatex
\documentclass[11pt]{article}
\usepackage[utf8]{inputenc}
\usepackage[T1]{fontenc}
\usepackage{graphicx}
\usepackage{grffile}
\usepackage{longtable}
\usepackage{wrapfig}
\usepackage{rotating}
\usepackage[normalem]{ulem}
\usepackage{amsmath}
\usepackage{textcomp}
\usepackage{amssymb}
\usepackage{capt-of}
\usepackage{hyperref}
\graphicspath{{../../books/}}
% TIPS
% \substack{a\\b} for multiple lines text





% pdfplots will load xolor automatically without option
\usepackage[dvipsnames]{xcolor}

\usepackage{forest}
% two-line text in node by [two \\ lines]
% \begin{forest} qtree, [..] \end{forest}
\forestset{
  qtree/.style={
    baseline,
    for tree={
      parent anchor=south,
      child anchor=north,
      align=center,
      inner sep=1pt,
    }}}
%\usepackage{flexisym}
% load order of mathtools and mathabx, otherwise conflict overbrace

\usepackage{mathtools}
%\usepackage{fourier}
\usepackage{pgfplots}
\usepackage{amsthm, mathabx,  amsmath, commath}
\usepackage{amsfonts}

\usepackage{empheq}
\usepackage{tikz}
\usetikzlibrary{arrows.meta}
\usepackage[most]{tcolorbox}

\newtheorem{theorem}{Theorem}[section]
\newtheorem{definition}{Definition}[section]
\newtheorem{corollary}{Corollary}[section]
\newtheorem{example}{Example}[section]
\newtheorem{lemma}{Lemma}[section]
\newtheorem{proposition}{Proposition}[section]

\newcommand{\bl}[1] {\boldsymbol{#1}}
\newcommand{\Wt}[1] {\stackrel{\sim}{\smash{#1}\rule{0pt}{1.1ex}}}
\newcommand{\wt}[1] {\widetilde{#1}}


%For boxed texts in align, use Aboxed{}
%otherwise use boxed{}

\DeclareMathSymbol{\widehatsym}{\mathord}{largesymbols}{"62}
\newcommand\lowerwidehatsym{%
  \text{\smash{\raisebox{-1.3ex}{%
    $\widehatsym$}}}}
\newcommand\fixwidehat[1]{%
  \mathchoice
    {\accentset{\displaystyle\lowerwidehatsym}{#1}}
    {\accentset{\textstyle\lowerwidehatsym}{#1}}
    {\accentset{\scriptstyle\lowerwidehatsym}{#1}}
    {\accentset{\scriptscriptstyle\lowerwidehatsym}{#1}}
}

\usepackage{graphicx}
    
% text on arrow for xRightarrow
\makeatletter
%\newcommand{\xRightarrow}[2][]{\ext@arrow 0359\Rightarrowfill@{#1}{#2}}
\makeatother


\def \bx {\boldsymbol{x}}
\def \ba {\boldsymbol{a}}
\def \bI {\boldsymbol{I}}
\def \bt {\boldsymbol{t}}
\def \bb {\boldsymbol{b}}
\def \bA {\boldsymbol{A}}
\def \bX {\boldsymbol{X}}
\def \bu {\boldsymbol{u}}
\def \bS {\boldsymbol{S}}
\def \bZ {\boldsymbol{Z}}
\def \bz {\boldsymbol{z}}
\def \by {\boldsymbol{y}}
\def \bw {\boldsymbol{w}}
\def \bT {\boldsymbol{T}}
\def \bS {\boldsymbol{S}}
\def \bm {\boldsymbol{m}}
\def \bW {\boldsymbol{W}}
\def \bY {\boldsymbol{Y}}
\def \bH {\boldsymbol{H}}
\def \blambda {\boldsymbol{\lambda}}
\def \bPhi {\boldsymbol{\Phi}}
\def \btheta {\boldsymbol{\theta}}
\def \bmu {\boldsymbol{\mu}}
\def \bphi {\boldsymbol{\phi}}
\def \bSigma {\boldsymbol{\Sigma}}
\def \lb {\left\{}
\def \rb {\right\}}
\def \caln {\mathcal{N}}
\def \dissum {\displaystyle\Sigma}
\def \dispro {\displaystyle\prod}
\def \E {\mathbb{E}}
\def \Q {\mathbb{Q}}
\def \V {\mathbb{V}}
\def \R {\mathbb{R}}
\def \calq {\mathcal{Q}}
\def \calg {\mathcal{G}}
\def \caln {\mathcal{N}}
\def \calr {\mathcal{R}}
\def \calm {\mathcal{M}}
\def \calc {\mathcal{C}}
\def \bcup {\bigcup}

\makeindex
\def \EF {\text{EF}}
\def \tint {\text{int}}
\author{Will Johnson}
\date{\today}
\title{Introduction To Model Theory}
\hypersetup{
 pdfauthor={Will Johnson},
 pdftitle={Introduction To Model Theory},
 pdfkeywords={},
 pdfsubject={},
 pdfcreator={Emacs 27.2 (Org mode 9.5)}, 
 pdflang={English}}
\begin{document}

\maketitle
\tableofcontents

\section{Back-and-forth Equivalence}
\label{sec:orgbb3db5e}
Convention: Relations and functions are sets of pairs \((x,y)\)

\begin{definition}[]
A \textbf{binary relation} is a pair \((E,R)\) where \(E\) is a set and \(R\subseteq E^2\). We call \(E\) the
\textbf{universe} of the relation. For \(a,b\in E\), write \(aEb\) if \((a,b)\in R\)
\end{definition}

We abbreviate \((E,R)\) as \(R\) or \(E\), if \(E\) or \(R\) is clear

\begin{examplle}[]
\((\R,<)\), \((\R,=)\), \((\R,\ge)\),\((\Z,<)\)
\end{examplle}

\begin{definition}[]
A binary relation \(R\) is said to be
\begin{itemize}
\item \textbf{reflexive} if \(aRa\) (\(\forall a\in E\))
\item \textbf{symmetric} if \(aRb\Rightarrow bRa\) (\(\forall a,b\in E\))
\item \textbf{transitive} if \(aRb\wedge bRc\Rightarrow aRc\) (\(\forall a,b,c\in E\))
\item \textbf{antisymmetric} if \(aRb\wedge bRa\Rightarrow a=b\) (\(\forall a,b\in E\))
\item \textbf{total} if \(aRb\vee bRa\) (\(\forall a,b\in E\))
\item an \textbf{equivalence relation} if it's reflexive, symmetric and transitive
\item a \textbf{partial order} if it's reflexive, antisymmetric and transitive
\item a \textbf{linear order}  if it's a total partial order
\end{itemize}
\end{definition}

\begin{examplle}[]
\(=\) is an equivalence relation

\(\subseteq\) is a partial order

\(\le\) is a linear order
\end{examplle}

\begin{definition}[]
An \textbf{isomorphism} from \((E,R)\) to \((E',R')\) is a bijection \(f:E\to E'\) s.t. for any \(a,b\in E\),
\(aRb\Leftrightarrow f(a)R'f(b)\). Two binary relations \((E,R)\) and \((E',R')\) are \textbf{isomorphic} (\(\cong\)) if
there is an isomorphism between them
\end{definition}

\begin{examplle}[]
\(f:(\Z,<)\to(2\Z,>)\) and \(f(x)=-2x\) is an isomorphism. \(x<y\Leftrightarrow -2x>-2y\)
\end{examplle}

\(\cong\) is an equivalence relation

\begin{definition}[]
A \textbf{local isomorphism} from \(R\) to \(R'\) is an isomorphism from a finite restriction of \(R\) to
a finite restriction of \(R'\). The set of local isomorphisms from \(R\) to \(R'\) is
denoted \(S_0(R,R')\). For \(f\in S_0(R,R')\), \(\dom(f)\) and \(\im(f)\) denote the domain and
range of \(f\)
\end{definition}

\begin{examplle}[]
\((\Z,<)\) is a restriction of \((\R,<)\)
\end{examplle}

\begin{examplle}[]
Suppose \(R=R'=(\Z,<)\), there is \(f\in S_0(R,R')\) given by \(\dom(f)​=\{1,2,3\}\)
and \(\im(f)​=\{10,20,30\}\) and \(f(1)=10\),\(f(2)=20\), \(f(3)=30\)
\end{examplle}

\begin{definition}[]
Let \(f,g\) be local isomorphisms from \(R\) to \(R'\). Then \(f\) is a \textbf{restriction} of \(g\)
if \(f\subseteq g\) and \(f\) is an \textbf{extension} of \(g\) if \(f\supseteq g\).
\end{definition}

\begin{examplle}[]
\(g:\{0,1,2,3\}\to\{5,10,20,30\}\), \(g\) extends \(f\) in the previous example
\end{examplle}

\begin{definition}[]
Let \(R,R'\) be binary relations with universe \(E,E'\). A \textbf{Karpian family} for \((R,R')\) is a
set \(K\subseteq S_0(R,R')\) satisfying the following two conditions for any \(f\in K\)
\begin{enumerate}
\item (\textbf{forth}) if \(a\in E\) then there is \(g\in K\) with \(g\supseteq f\) and \(a\in\dom(g)\)
\item (\textbf{back}) if \(b\in E'\) then there is \(g\in K\) with \(g\supseteq f\) and \(b\in\im(g)\)
\end{enumerate}


\(R\) and \(R'\) are \textbf{\(\infty\)-equivalent}, write \(R\sim_\infty R'\), if there is a non-empty
Karpian family
\end{definition}

\begin{proposition}[]
If \(f:(E,R)\to(E',R')\) an isomorphism and \(K=\{g\subseteq f:g\text{ is finite}\}\), then \(K\) is Karpian
and \(R\sim_\infty R'\)
\end{proposition}

\begin{proof}
Suppose \(g\in K\)
\begin{itemize}
\item (forth) Suppose \(a\in E\), take \(b=f(a)\) and let \(h=g\cup\{(a,b)\}\). Then \(h\subseteq f\),
so \(h\in K\), \(h\supseteq g\), \(a\in\dom(h)\)
\item (back) similarly
\end{itemize}
\end{proof}

\begin{proposition}[]
If \((E,R)\) and \((E',R')\) are countable and \(R\sim_\infty R'\), then \(R\cong R'\)
\end{proposition}

\begin{proof}
Let \(K\subseteq S_0(R,R')\) be Karpian, \(K\neq\emptyset\), \(E=\{e_1,e_2,e_3,\dots\}\), \(E'=\{e_1',e_2',e_3',\dots\}\)

Recursively build \(f_1\subseteq f_2\subseteq\cdots\), \(f_i\in K\)

Let \(f_1\) be anything in \(K\) as \(K\) is non-empty.

\(f_{2i}\) some extension of \(f_{2i-1}\) with \(e_i\in\dom(f_{2i})\)

\(f_{2i+1}\) some extension of \(f_{2i}\) with \(e_i'\in\im(f_{2i+1})\)

Now let \(g=\bigcup_{i=1}^\infty f_i\), then \(g\) is an isomorphism
\end{proof}

\begin{definition}[]
A \textbf{dense linear order without endpoints} (DLO) is a linear order \((C,\le)\) satisfying
\begin{enumerate}
\item \(C\neq\emptyset\)
\item \(\forall x,y\in C\), \(x<y\Rightarrow\exists z\in C\; x<z<y\)
\item \(\forall x\in C\), \(\exists y,z\in C\;y<x<z\)
\end{enumerate}
\end{definition}

\begin{examplle}[]
\((\Q,\le)\), \((\R,\le)\)

non-example: \((\Z,\le)\), \(([0,1],\le)\)
\end{examplle}

\begin{proposition}[]
Let \((C,\le)\) and \((C',\le)\) be DLO's. Then \(S_0(C,C')\) is Karpian. So \(C\sim_\infty C'\)
\end{proposition}

\begin{proof}
Let \(f\in S_0(C,C')\), \(\dom(f)=\{a_1,\dots,a_n\}\), \(a_1<\dots<a_n\)
and \(\im(f)=b_1,\dots,b_n\), \(b_1<\dots<b_n\). Since \(f\) is a local isomorphism, \(f(a_i)=b_i\)
\begin{itemize}
\item (forth) Suppose \(a\in C\). We want \(b\in C'\) s.t. \(f\cup\{(a,b)\}\in S_0(C,C')\).
\begin{itemize}
\item if \(a_i<a<a_{i+1}\). We take \(b\in C'\) s.t. \(b_i<b<b_{i+1}\) since dense
\item if \(a<a_1\). We take \(b\in C'\) s.t. \(b<b_1\) since no endpoints
\item if \(a>a_n\), take \(b\in C'\) s.t. \(b>b_n\)
\item if \(a=a_i\), take \(b=b_i\)
\end{itemize}
\item (back) similar
\end{itemize}
\end{proof}

\begin{proposition}[]
If \((C,\le)\) and \((C',\le)\) are countable DLOs, then \(C\sim_\infty C'\), so \(C\cong C'\)
\end{proposition}
Hence
\begin{align*}
(\Q,\le)&\cong(\Q\setminus\{0\},\le)\\
&\cong(\Q\cup\{\sqrt{2}\},\le)\\
&\cong(\Q\cap(0,1),\le)
\end{align*}
\begin{definition}[]
Let \(R,R'\) be binary relations with universe \(E,E'\)
\begin{itemize}
\item A \textbf{0-isomorphism} from \(R\) to \(R'\) is a local isomorphism from \(R\) to \(R'\)
\item For \(p>0\), a \textbf{\(p\)-isomorphism} from \(R\) to \(R'\) is a local isomorphism \(f\) from \(R\)
to \(R'\) satisfying the following two conditions
\begin{enumerate}
\item (\textbf{forth}) For any \(a\in E\), there is a \((p-1)\)-isomorphism \(g\supseteq f\) with \(a\in\dom(g)\)
\item (\textbf{back}) For any \(b\in E'\), there is a \((p-1)\)-isomorphism \(g\supseteq f\) with \(b\in\im(g)\)
\end{enumerate}
\item An \textbf{\(\omega\)-isomorphism} from \(R\) to \(R'\) is a local isomorphism \(f\) from \(R\) to \(R'\)
s.t. \(f\) is a \(p\)-isomorphism for all \(p<\omega\)
\end{itemize}


The set of \(p\)-isomorphisms from \(R\) to \(R'\) is denoted \(S_p(R,R')\)
\end{definition}

\begin{examplle}[]
Suppose \(R=R'=(\Z,<)\), \(f:\{2,4\}\to\{1,2\}\) is a local isomorphism with \(f(2)=1\) and \(f(4)=2\).
Then \(f\notin S_1(\Z,\Z)\) (forth) fails. For \(a=3\), there is no \(b\) s.t. \(1<b<2\)

\(g:\{2,4\}\to\{1,5\}\) is a 1-isomorphism but not a 2-isomorphism
\end{examplle}

\begin{proposition}[]
If \(f\in S_p(R,R')\) and \(g\subseteq f\), then \(g\in S_p(R,R')\)
\end{proposition}

\begin{proof}
if \(p=0\) easy

if \(p>0\) (forward), \(\forall a\in E\), \(\exists h\in S_{p-1}(R,R')\) has \(a\in\dom(h)\) and \(h\supseteq f\supseteq g\)
\end{proof}

\begin{proposition}[]
\(S_p(R,R')\neq\emptyset\) iff \(\emptyset\in S_p(R,R')\)
\end{proposition}

\begin{proof}
\(\Leftarrow\) immediate

\(\Rightarrow\). Suppose \(f\in S_p(R,R')\). Then \(\emptyset\subseteq f\). Hence \(\emptyset\in S_p(R,R')\).
\end{proof}

\begin{definition}[]
\(R\) and \(R'\) are \textbf{\(p\)-equivalent}, written \(R\sim_p R'\), if there is a \(p\)-isomorphism
from \(R\to R'\)
\end{definition}

\(R\) and \(R'\) are \textbf{\(\omega\)-equivalent} or \textbf{elementarily equivalent}, written \(R\sim_\omega R'\)
or \(R\equiv R'\), if there is an \(\omega\)-isomorphism from \(R\) to \(R'\)

Note: \(R\sim_\omega R'\) iff \(S_\omega(R,R')\neq\emptyset\) iff \(\emptyset\in S_\omega(R,R')\) iff \(\forall p\;\emptyset\in S_p(R,R')\)
iff \(\forall p\; R\sim_pR'\)

\begin{definition}[]
Let \(R,R'\) be binary relations with universe \(E,E'\). The Ehfrenfeucht-Fraïssé game of
length \(n\), denoted \(\EF_n(R,R')\) is played as follows
\begin{itemize}
\item There are two players, the Duplicator and Spoiler
\item There are \(n\) rounds
\item In the \(i\)th round, the Spoiler chooses either an \(a_i\in E\) or a \(b_i\in E'\)
\item The Duplicator responds with a \(b_i\in E'\) or an \(a_i\in E\) respectively
\item At the ends of the game, the Duplicator wins
\begin{equation*}
\{(a_i,b_i),\dots,(a_n,b_n)\}
\end{equation*}
is a local isomorphism from \(R\) to \(R'\)
\item Otherwise, the Spoiler wins
\end{itemize}
\end{definition}

\begin{examplle}[]
For \(\EF_3(\Q,\R)\)
\begin{center}
\begin{tabular}{ll}
\(\Q\) & \(\R\)\\
\hline
S:\(a_1=7\) & D:\(b_1=7\)\\
D:\(a_2=1.4\) & S:\(b_2=\sqrt{2}\)\\
D:\(a_3=-10\) & S:\(b_3=1.41\)\\
\end{tabular}
\end{center}

So \(D\) wins
\end{examplle}

\begin{examplle}[]
\(\EF_3(\R,\Z)\)

\begin{center}
\begin{tabular}{ll}
\(\R\) & \(\Z\)\\
D:\(a_1=1\) & S:\(b_1=1\)\\
D:\(a_2=1.1\) & S:\(b_2=2\)\\
S:\(a_3=1.01\) & \\
\end{tabular}
\end{center}
D fails
\end{examplle}

\begin{proposition}[]
\(\EF_n(R,R')\) is a win for Duplicator iff \(R\sim_nR'\)
\end{proposition}

\begin{proposition}[]
In \(\EF_n(R,R')\) if moves so far are \(a_1,b_1,\dots,a_i,b_i\), \(p=n-1\), \(f=\{(a_1,b_1),\dots,(a_i,b_i)\}\).
Then Duplicator wins iff \(f\in S_p(R,R')\)
\end{proposition}

\appendix
\section{Metric Spaces}
\label{sec:org65b4cf5}
\(\R_{\ge 0}\) denotes \([0,+\infty]=\{x\in\R:x\ge 0\}\)
\begin{definition}[]
A \textbf{metric} on a set \(M\) is a function \(d:M\times M\to\R_{\ge 0}\) satisfying the following properties
\begin{enumerate}
\item \(d(x,y)=0\Leftrightarrow x=y\)
\item \(d(x,y)=d(y,x)\)
\item \(d(x,z)\le d(x,y)+d(y,z)\)
\end{enumerate}
\end{definition}

\begin{examplle}[]
\(M=\R^2\), \(d(x,y)=\)(the distance from \(x\) to \(y\))
\begin{equation*}
  d(x_1,x_2;y_1,y_2)=\sqrt{(x_1-y_1)^2+(x_2-y_2)^2}
\end{equation*}
\end{examplle}

\begin{examplle}[]
The \textbf{Manhattan metric} on \(\R^2\) is given by
\begin{equation*}
  d(x_1,x_2;y_1,y_2)=\abs{x_1-y_1}+\abs{x_2-y_2}
\end{equation*}

measure distances in a city grid
\end{examplle}

\begin{examplle}[]
Let \(M\) be the set of strings. The \textbf{edit distance} from \(x\) to \(y\) is the minimum number of
intersections, deletions, and substitutions to go from \(x\) to \(y\)
\begin{gather*}
  d(drip,rope)=3\\
  drip\mapsto drop\mapsto rop\mapsto rope
\end{gather*}
Edit distance is a metric on \(M\)
\end{examplle}

\begin{definition}[]
A \textbf{metric space} is a pair \((M,d)\) where \(M\) is a set and \(d\) is a metric space
\end{definition}

\begin{itemize}
\item \((\R^n,d_{Euclidean})\) where \(d_{Euclidean}\) is the usual Euclidean distance
\item \((\R^2,d_{Manhattan})\) where \(d_{Manhattan}\) is the Manhattan distance
\end{itemize}


Often we abbreviate \((M,d)\) as \(M\), when \(d\) is clear

Fix a metric space \((M,d)\)

\begin{definition}[]
If \(p\in M\) and \(\epsilon>0\), then
\begin{align*}
&B_\epsilon(p)=\{x\in M:d(x,p)<\epsilon\}\\
&\barB_\epsilon(p)=\{x\in M:d(x,p)\le\epsilon\}\\
\end{align*}
\(B_\epsilon(p)\) and \(\barB_\epsilon(p)\) are called the \textbf{open} and \textbf{closed} balls of radius \(\epsilon\) around \(p\)
\end{definition}

\begin{examplle}[]
In \(\R^2\) with the Euclidean metric, the open ball of radius 2 around \((0,0)\) the open disk
\begin{equation*}
\{(x,y)\in\R^2:x^2+y^2<2^2\}
\end{equation*}
\end{examplle}

\begin{examplle}[]
In \(\R^2\) with the Manhattan metric, the open ball of radius 1 around \((0,0)\) the open disk
\begin{equation*}
\{(x,y)\in\R^2:\abs{x}+\abs{y}\le 1\}
\end{equation*}
\end{examplle}

Suppose \(p\in M\) and \(X\subseteq M\)
\begin{definition}[]
\(p\) is an \textbf{interior point} of \(X\) if \(X\) contains an open ball of positive radius around \(p\)
\end{definition}

In particular, \(p\) must be an element of \(X\)


\begin{examplle}[]
If \(X=[-1,1]\times[-1,1]\), then \((0,0)\) is an interior point of \(X\), but \((1,0)\)
and \((0,2)\) are not
\end{examplle}

\begin{definition}[]
The \textbf{interior} \(\tint(X)\) is the set of interior points
\end{definition}

Warning: There are metric spaces where the interior of \(\barB_\epsilon(p)\) isn't \(B_\epsilon(p)\)

\begin{definition}[]
A set \(X\subseteq M\) is \textbf{open} if \(X=\tint(X)\), i.e., every point of \(X\) is an interior point of \(X\)
\end{definition}

\begin{examplle}[in \(\R\)]
The set \((-1,2)\) is open. The sets \([-1,2]\) and \([-1,2)\) are not; they have interior \((-1,2)\)
\end{examplle}

Fact: the interior \(\tint(X)\) is the unique largest open set contained in \(X\)

Let \(a_1,a_2,\dots\) be a sequence in a metric space \((M,d)\) and let \(p\) be a point
\begin{definition}[]
``\(\lim_{i\to\infty}a_i=p\)'' if for every \(\epsilon>0\), there is \(n\) s.t.
\begin{equation*}
\{a_n,a_{n+1},a_{n+2},\dots\}\subseteq B_\epsilon(p)
\end{equation*}
\end{definition}

\begin{examplle}[]
Work in \(\R\) with the usual distance. Let \(a_n=1/n\). Then \(\lim_{n\to\infty}a_n=0\)
but \(\lim_{n\to\infty}a_n\neq 1\)
\end{examplle}

Fact: For any sequence \(a_1,a_2,a_3,\cdots\) in \((M,d)\), there is at most one point \(p\)
s.t. \(\lim_{i\to\infty}a_i=p\)

If such a \(p\) exists, it is called the \textbf{limit}, and written \(\lim_{i\to\infty}a_i\)

let \(X\) be a set and \(p\) be a point in a metric space \((M,d)\)

\begin{definition}[]
\(p\) is an \textbf{accumulation point} of \(X\) if \(p=\lim_{n\to\infty}a_n\) for some sequence \(a_n\) in \(X\)
\end{definition}

Equivalently
\begin{definition}[]
\(p\) is an accumulation point of \(X\) if for every \(\epsilon>0\), we have \(B_\epsilon(p)\cap X\neq\emptyset\)
\end{definition}

\begin{definition}[]
The \textbf{closure} of \(X\), written \(\cl(X)\) or \(\barX\), is the set of accumulation points
\end{definition}

\begin{definition}[]
A set \(X\subseteq M\) is \textbf{closed} if \(X=\cl(X)\)
\end{definition}


Fact: The closure \(\cl(X)\) is the unique smallest closed set containing \(X\)

\begin{examplle}[]
Work in \(\R\) with the distance \(d(x,y)=\abs{x-y}\)

\(\Q\) is neither closed nor open

\(\R\) is both closed and open, so is \(emptyset\)
\end{examplle}

Let \(X^c\) denote the completement \(M\setminus X\)

Fact: \(X\) is closed iff \(X^c\) is open

Fact: \(\tint(X)=\cl(X^c)^c\) and \(\cl(X)=\tint(X^c)^c\)

Let \((M,d)\) and \((M',d)\) be metric spaces. Let \(f:M\to M'\) be a function
\begin{definition}[]
\(f\) is \textbf{continuous} if
\begin{equation*}
\lim_{n\to\infty}a_n=p\Rightarrow\lim_{n\to\infty}f(a_n)=f(p)
\end{equation*}
for \(a_1,a_2,a_3,\dots,p\in M\)
\end{definition}

idea: \(f\) is continuous iff \(f\) preserves limits

\begin{examplle}[]
Let \(f:\R\to\R\) be given by
\begin{equation*}
f(x)=
\begin{cases}
1&\text{if }x>0\\
-1&\text{if }x\le 0
\end{cases}
\end{equation*}
Then \(\lim_{n\to\infty}1/n=0\), but
\begin{equation*}
\lim_{n\to\infty}f(1/n)=\lim_{n\to\infty}1=1\neq-1=f(0)
\end{equation*}
\end{examplle}

\begin{proposition}[]
Fix \(f:(M,d)\to(M',d)\). The following are equivalent
\begin{enumerate}
\item \(f\) is continuous
\item For every open set \(U\subseteq M'\), the preimage \(f^{-1}(U)\) is open
\item For every \(p\in M\), for every \(\epsilon>0\), there is \(\delta>0\) s.t. for every \(x\in M\),
 \begin{equation*}
d(x,p)<\delta\Rightarrow d(f(x),f(p))<\epsilon
 \end{equation*}
\end{enumerate}
\end{proposition}

Fact: The functions \(\sin\), \(\cos\), \(\exp\), \(\sqrt[3]{-}\) and polynomials are continuous

\begin{proposition}[]
If \(f,g:\R\to\R\) are continuous, then \(f+g,f\cdot g,f-g,f\circ g\) are continuous
\end{proposition}

\begin{proposition}[]
If \(f:\R\to\R\) is continuous and \(f(x)\neq 0\) for all \(x\), then \(1/f(x)\) is continuous.
If \(f(x)\ge 0\) for all \(x\), then \(\sqrt{f(x)}\) is continuous
\end{proposition}

\begin{examplle}[]
This function is continuous
\begin{equation*}
h(x)=\exp\left( \frac{1}{1+x^2} \right)-\frac{1}{17+\sin(\sqrt[3]{x})}
\end{equation*}
\end{examplle}

\begin{definition}[]
A function \(f:M\to M'\) is \textbf{Lipschitz continuous} if there is \(c\in\R\) s.t. for any \(x,y\in M\)
\begin{equation*}
d(f(x),f(y))\le c\cdot d(x,y)
\end{equation*}
\end{definition}

\begin{examplle}[In \(\R\)]
The function \(f(x)=\abs{x}+\abs{x-1}\) is Lipschitz continuous with \(c=2\)
\end{examplle}

\begin{proposition}[]
If \(f\) is Lipschitz continuous, then \(f\) is continuous
\end{proposition}

\begin{examplle}[]
The function \(f(x)=x^2\) is continuous but not Lipschitz continuous
\end{examplle}

\begin{definition}[]
Let \((M,d)\)be a metric space and \(S\subseteq M\) be a set. Then \((S,d')\) is a metric space,
where \(d'(x,y)=d(x,y)\) for \(x,y\in S\)
\begin{itemize}
\item \(d'\) is the restriction of \(d\) to \(S\times S\)
\item We say that \((S,d')\) is a \textbf{subspace} of \((M,d)\)
\end{itemize}
\end{definition}

Let \((M,d)\), \((M',d)\) be metric spaces, \(S\subseteq M\) and \(f:S\to M'\) be a function

\begin{definition}[]
\(f\)  is \textbf{continuous} if \(f\) is continuous as a map from the subspace \((S,d')\) to \((M',d)\)
\end{definition}

\begin{examplle}[in \(\R\)]
Let \(f:(-\infty,0)\cup(0,\infty)\to\R\)  be given by \(f(x)=1/x\). Then \(f\) is continuous
\end{examplle}

\begin{definition}[]
An \textbf{isometry} or \textbf{isomorphism} from \((M,d)\) to \((M',d')\) is a bijection \(f:M\to M'\) s.t. for
any \(x,y\in M\)
\begin{equation*}
d(x,y)=d'(f(x),f(y))
\end{equation*}
\end{definition}

\begin{examplle}[in \(\R^2\)]
The map \((x,y)\mapsto(x+1,y-7)\) is an isometry

So is the map \((x,y)\mapsto(3/5x+4/5y,-4/5x+3/5y)\)
\end{examplle}
\end{document}
