% Created 2020-04-23 四 14:55
% Intended LaTeX compiler: pdflatex
\documentclass[11pt]{article}
\usepackage[utf8]{inputenc}
\usepackage[T1]{fontenc}
\usepackage{graphicx}
\usepackage{grffile}
\usepackage{longtable}
\usepackage{wrapfig}
\usepackage{rotating}
\usepackage[normalem]{ulem}
\usepackage{amsmath}
\usepackage{textcomp}
\usepackage{amssymb}
\usepackage{capt-of}
\usepackage{imakeidx}
\usepackage{hyperref}
\usepackage{minted}
% TIPS
% \substack{a\\b} for multiple lines text





% pdfplots will load xolor automatically without option
\usepackage[dvipsnames]{xcolor}

\usepackage{forest}
% two-line text in node by [two \\ lines]
% \begin{forest} qtree, [..] \end{forest}
\forestset{
  qtree/.style={
    baseline,
    for tree={
      parent anchor=south,
      child anchor=north,
      align=center,
      inner sep=1pt,
    }}}
%\usepackage{flexisym}
% load order of mathtools and mathabx, otherwise conflict overbrace

\usepackage{mathtools}
%\usepackage{fourier}
\usepackage{pgfplots}
\usepackage{amsthm, mathabx,  amsmath, commath}
\usepackage{amsfonts}

\usepackage{empheq}
\usepackage{tikz}
\usetikzlibrary{arrows.meta}
\usepackage[most]{tcolorbox}

\newtheorem{theorem}{Theorem}[section]
\newtheorem{definition}{Definition}[section]
\newtheorem{corollary}{Corollary}[section]
\newtheorem{example}{Example}[section]
\newtheorem{lemma}{Lemma}[section]
\newtheorem{proposition}{Proposition}[section]

\newcommand{\bl}[1] {\boldsymbol{#1}}
\newcommand{\Wt}[1] {\stackrel{\sim}{\smash{#1}\rule{0pt}{1.1ex}}}
\newcommand{\wt}[1] {\widetilde{#1}}


%For boxed texts in align, use Aboxed{}
%otherwise use boxed{}

\DeclareMathSymbol{\widehatsym}{\mathord}{largesymbols}{"62}
\newcommand\lowerwidehatsym{%
  \text{\smash{\raisebox{-1.3ex}{%
    $\widehatsym$}}}}
\newcommand\fixwidehat[1]{%
  \mathchoice
    {\accentset{\displaystyle\lowerwidehatsym}{#1}}
    {\accentset{\textstyle\lowerwidehatsym}{#1}}
    {\accentset{\scriptstyle\lowerwidehatsym}{#1}}
    {\accentset{\scriptscriptstyle\lowerwidehatsym}{#1}}
}

\usepackage{graphicx}
    
% text on arrow for xRightarrow
\makeatletter
%\newcommand{\xRightarrow}[2][]{\ext@arrow 0359\Rightarrowfill@{#1}{#2}}
\makeatother


\def \bx {\boldsymbol{x}}
\def \ba {\boldsymbol{a}}
\def \bI {\boldsymbol{I}}
\def \bt {\boldsymbol{t}}
\def \bb {\boldsymbol{b}}
\def \bA {\boldsymbol{A}}
\def \bX {\boldsymbol{X}}
\def \bu {\boldsymbol{u}}
\def \bS {\boldsymbol{S}}
\def \bZ {\boldsymbol{Z}}
\def \bz {\boldsymbol{z}}
\def \by {\boldsymbol{y}}
\def \bw {\boldsymbol{w}}
\def \bT {\boldsymbol{T}}
\def \bS {\boldsymbol{S}}
\def \bm {\boldsymbol{m}}
\def \bW {\boldsymbol{W}}
\def \bY {\boldsymbol{Y}}
\def \bH {\boldsymbol{H}}
\def \blambda {\boldsymbol{\lambda}}
\def \bPhi {\boldsymbol{\Phi}}
\def \btheta {\boldsymbol{\theta}}
\def \bmu {\boldsymbol{\mu}}
\def \bphi {\boldsymbol{\phi}}
\def \bSigma {\boldsymbol{\Sigma}}
\def \lb {\left\{}
\def \rb {\right\}}
\def \caln {\mathcal{N}}
\def \dissum {\displaystyle\Sigma}
\def \dispro {\displaystyle\prod}
\def \E {\mathbb{E}}
\def \Q {\mathbb{Q}}
\def \V {\mathbb{V}}
\def \R {\mathbb{R}}
\def \calq {\mathcal{Q}}
\def \calg {\mathcal{G}}
\def \caln {\mathcal{N}}
\def \calr {\mathcal{R}}
\def \calm {\mathcal{M}}
\def \calc {\mathcal{C}}
\def \bcup {\bigcup}

\author{Reter J. Brockwell \& Richard A. Davis}
\date{\today}
\title{\aunclfamily\Huge Time Series:\\ Theory and Methods}
\hypersetup{
 pdfauthor={Reter J. Brockwell \& Richard A. Davis},
 pdftitle={\aunclfamily\Huge Time Series:\\ Theory and Methods},
 pdfkeywords={},
 pdfsubject={},
 pdfcreator={Emacs 26.3 (Org mode 9.3.6)}, 
 pdflang={English}}
\begin{document}

\maketitle \clearpage
\tableofcontents \clearpage
\section{Stationary Time Series}
\label{sec:org2b75ac1}

\subsection{Stochastic Processes}
\label{sec:orgf5949fa}
\begin{definition}[]
A \textbf{stochastic process} is a family of random variables \(\{X_t,t\in T\}\) defined
on a probability space \((\Omega,\calf,P)\)
\end{definition}

A \textbf{probability space} or a \textbf{probability triple} \((\Omega,\calf,P)\) consists of three
elements
\begin{enumerate}
\item The sample space \(\Omega\) - an arbitrary non-empty set
\item The \(\sigma\)-algebra \(\calf\in 2^{\Omega}\) - called events, s.t.
\begin{itemize}
\item \(\calf\) contains the sample space: \(\Omega\in\calf\)
\item \(\calf\) is closed under complements
\item \(\calf\) is closed under countable unions
\end{itemize}
\item The probability measure \(P:\calf\to[0,1]\) - a function on \(\calf\) s.t.
\begin{itemize}
\item \(P\) is countably additive: if \(\{A_i\}_{i=1}^\infty\subseteq\calf\) is a
countable collection of pairwise disjoint sets, then
\(P(\bigcup_{i=1}^\infty A_i)=\sum_{i=1}^\infty P(A_i)\)
\item the measure of entire sample space is equal to one
\end{itemize}
\end{enumerate}


A \textbf{random variable} is a measurable function \(X:\Omega\to E\) from a set of
possible outcomes \(\Omega\) to a measurable space \(E\). The probability that \(X\) takes
on a value in a measurable set \(S\subseteq E\) is written as
\begin{equation*}
P(X\in S)=P({\omega\in\Omega\mid X(\omega)\in S})
\end{equation*}
\begin{remark}
In time series analysis, the index set \(T\) is a set of time points, very
often
\(\{0,\pm 1,\pm 2,\dots\}\),\(\{1,2,3,\dots\}\),\([0,\infty)\) or \((-\infty,\infty)\)
\end{remark}

\begin{definition}[Realizations of a Stochastic Process]
The functions \(\{X(\omega),\omega\in\Omega\}\) on \(T\) are known as the
\textbf{realizations} or \textbf{sample-paths} of the process \(\{X_t,t\in T\}\)
\end{definition}

\begin{examplle}[Sinusoid with Random Phase and Amplitude]
Let \(A\) and \(\Theta\) be independent random variable with \(A\ge0\) and \(\Theta\) distributed
uniformly on \([0,2\pi)\). A stochastic process \(\{X(t),t\in\R\}\) can then be
defined in terms of \(A\) and \(\Theta\) for any given \(\nu\ge0\) and \(r>0\) by
\begin{equation*}
X_t=r^{-1}A\cos(\nu t+\Theta)
\end{equation*}
or more explicitly
\begin{equation*}
X_t(\omega)=r^{-1}A(\omega)\cos(\nu t+\Theta(\omega))
\end{equation*}
\end{examplle}
\end{document}