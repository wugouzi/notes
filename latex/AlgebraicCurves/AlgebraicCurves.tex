% Created 2021-12-24 Fri 10:33
% Intended LaTeX compiler: pdflatex
\documentclass[11pt]{article}
\usepackage[utf8]{inputenc}
\usepackage[T1]{fontenc}
\usepackage{graphicx}
\usepackage{longtable}
\usepackage{wrapfig}
\usepackage{rotating}
\usepackage[normalem]{ulem}
\usepackage{amsmath}
\usepackage{amssymb}
\usepackage{capt-of}
\usepackage{hyperref}
\graphicspath{{../../books/}}
% TIPS
% \substack{a\\b} for multiple lines text





% pdfplots will load xolor automatically without option
\usepackage[dvipsnames]{xcolor}

\usepackage{forest}
% two-line text in node by [two \\ lines]
% \begin{forest} qtree, [..] \end{forest}
\forestset{
  qtree/.style={
    baseline,
    for tree={
      parent anchor=south,
      child anchor=north,
      align=center,
      inner sep=1pt,
    }}}
%\usepackage{flexisym}
% load order of mathtools and mathabx, otherwise conflict overbrace

\usepackage{mathtools}
%\usepackage{fourier}
\usepackage{pgfplots}
\usepackage{amsthm, mathabx,  amsmath, commath}
\usepackage{amsfonts}

\usepackage{empheq}
\usepackage{tikz}
\usetikzlibrary{arrows.meta}
\usepackage[most]{tcolorbox}

\newtheorem{theorem}{Theorem}[section]
\newtheorem{definition}{Definition}[section]
\newtheorem{corollary}{Corollary}[section]
\newtheorem{example}{Example}[section]
\newtheorem{lemma}{Lemma}[section]
\newtheorem{proposition}{Proposition}[section]

\newcommand{\bl}[1] {\boldsymbol{#1}}
\newcommand{\Wt}[1] {\stackrel{\sim}{\smash{#1}\rule{0pt}{1.1ex}}}
\newcommand{\wt}[1] {\widetilde{#1}}


%For boxed texts in align, use Aboxed{}
%otherwise use boxed{}

\DeclareMathSymbol{\widehatsym}{\mathord}{largesymbols}{"62}
\newcommand\lowerwidehatsym{%
  \text{\smash{\raisebox{-1.3ex}{%
    $\widehatsym$}}}}
\newcommand\fixwidehat[1]{%
  \mathchoice
    {\accentset{\displaystyle\lowerwidehatsym}{#1}}
    {\accentset{\textstyle\lowerwidehatsym}{#1}}
    {\accentset{\scriptstyle\lowerwidehatsym}{#1}}
    {\accentset{\scriptscriptstyle\lowerwidehatsym}{#1}}
}

\usepackage{graphicx}
    
% text on arrow for xRightarrow
\makeatletter
%\newcommand{\xRightarrow}[2][]{\ext@arrow 0359\Rightarrowfill@{#1}{#2}}
\makeatother


\def \bx {\boldsymbol{x}}
\def \ba {\boldsymbol{a}}
\def \bI {\boldsymbol{I}}
\def \bt {\boldsymbol{t}}
\def \bb {\boldsymbol{b}}
\def \bA {\boldsymbol{A}}
\def \bX {\boldsymbol{X}}
\def \bu {\boldsymbol{u}}
\def \bS {\boldsymbol{S}}
\def \bZ {\boldsymbol{Z}}
\def \bz {\boldsymbol{z}}
\def \by {\boldsymbol{y}}
\def \bw {\boldsymbol{w}}
\def \bT {\boldsymbol{T}}
\def \bS {\boldsymbol{S}}
\def \bm {\boldsymbol{m}}
\def \bW {\boldsymbol{W}}
\def \bY {\boldsymbol{Y}}
\def \bH {\boldsymbol{H}}
\def \blambda {\boldsymbol{\lambda}}
\def \bPhi {\boldsymbol{\Phi}}
\def \btheta {\boldsymbol{\theta}}
\def \bmu {\boldsymbol{\mu}}
\def \bphi {\boldsymbol{\phi}}
\def \bSigma {\boldsymbol{\Sigma}}
\def \lb {\left\{}
\def \rb {\right\}}
\def \caln {\mathcal{N}}
\def \dissum {\displaystyle\Sigma}
\def \dispro {\displaystyle\prod}
\def \E {\mathbb{E}}
\def \Q {\mathbb{Q}}
\def \V {\mathbb{V}}
\def \R {\mathbb{R}}
\def \calq {\mathcal{Q}}
\def \calg {\mathcal{G}}
\def \caln {\mathcal{N}}
\def \calr {\mathcal{R}}
\def \calm {\mathcal{M}}
\def \calc {\mathcal{C}}
\def \bcup {\bigcup}

\makeindex
\DeclareMathOperator{\Rad}{\text{Rad}}
\author{William Fulton}
\date{\today}
\title{Algebraic Curves}
\hypersetup{
 pdfauthor={William Fulton},
 pdftitle={Algebraic Curves},
 pdfkeywords={},
 pdfsubject={},
 pdfcreator={Emacs 27.2 (Org mode 9.6)}, 
 pdflang={English}}
\begin{document}

\maketitle
\tableofcontents

This \href{https://ziyuzhang.github.io/ma40188/Lecture\_Notes\_Long.pdf}{notes} seems a good companion
\section{Affine Algebraic Sets}
\label{sec:org91aa9c0}
\subsection{Affine Space and Algebraic Sets}
\label{sec:org2b0fd8c}
Let \(k\) be any field. By \(\A^n(k)\), or simply \(\A^n\), we shall mean the Cartesian product
of \(k\) with itself \(n\) times. We call \(\A^n(k)\) \textbf{affine \(n\)-space} over \(k\); its elements
will be called \textbf{points}. In particular, \(\A^1(k)\) is the \textbf{affine line}, \(\A^2(k)\) is the \textbf{affine
space}

If \(F\in k[X_1,\dots,X_n]\), a point \(P=(a_1,\dots,a_n)\in\A^n(k)\) is called a \textbf{zero} of \(F\) if \(F(P)=0\).
If \(F\) is not a constant, the set of zeros of \(F\) is called the \textbf{hypersurface} defined
by \(F\), and is denoted by \(V(F)\). A hypersurface in \(\A^2(k)\) is called an \textbf{affine plane
curve}. If \(F\) is a polynomial of degree one, \(V(F)\) is called a \textbf{hyperplane} in \(\A^n(k)\);
if \(n=2\), it is a \textbf{line}

More generally, if \(S\) is any set of polynomials in \(k[X_1,\dots,X_n]\), we
let \(V(S)=\{P\in\A^n\mid F(P)=0\text{ for all }F\in S\}\), \(V(S)=\bigcap_{F\in S}V(F)\). If \(S=\{F_1,\dots,F_r\}\), we
usually write \(V(F_1,\dots,F_r)\). A subset \(X\subseteq\A^n(k)\) is an \textbf{affine algebraic set}, or simply an
\textbf{algebraic set}, if \(X=V(S)\) for some \(S\)

\begin{enumerate}
\item If \(I\) is the ideal in \(k[X_1,\dots,X_n]\) generated by \(S\), then \(V(S)=V(I)\); so every
algebraic set \(V(I)\) is equal to some ideal \(I\)
\item If \(\{I_\alpha\}\) is any collection of ideals, then \(V(\bigcup_\alpha I_\alpha)=\bigcap_\alpha V(I_\alpha)\).
\item If \(I\subset J\), then \(V(I)\supset V(J)\)
\item \(V(FG)=V(F)\cup V(G)\)

\(x\in V(FG)\Leftrightarrow FG(x)\Leftrightarrow F(x)=0\vee G(x)=0\) since \(k\) is a field and \(k[X_1,\dots,X_n]\) is a domain

\(V(I)\cup V(J)=V(\{FG\mid F\in I,G\in J\})\)
\item \(V(0)=\A^n(k)\), \(V(1)=\emptyset\), \(V(X_1-a_1,\dots,X_n-a_n)=\{(a_1,\dots,a_n)\}\) for \(a_i\in k\). So any finite
subset of \(\A^n(k)\) is an algebraic set
\end{enumerate}


\begin{exercise}
\label{ex1.13}
Show that each of the following sets is not algebraic
\begin{enumerate}
\item \(A=\{(x,y)\in\A^2(\R)\mid y=\sin x\}\)
\end{enumerate}
\end{exercise}

\begin{proof}
\begin{enumerate}
\item Suppose \(f\in I(A)\) and fix a \(a\in\R\), then \(f(x,a)\in\R[x]\) but has infinitely many
solutions, a contradiction
\end{enumerate}
\end{proof}
\subsection{The Ideal of a Set of Points}
\label{sec:org59a5c74}
For any subset \(X\subseteq\A^n(k)\), we consider those polynomials that vanish on \(X\); they form an
ideal in \(k[X_1,\dots,X_n]\), called the \textbf{ideal} of \(X\), and
written \(I(X)\), \(I(X)=\{F\in k[X_1,\dots,X_n]\mid F(a_1,\dots,a_n)=0\text{ for all }(a_1,\dots,a_n)\in X\}\).
\begin{enumerate}
\item If \(X\subset Y\), then \(I(X)\supset I(Y)\)
\item \(I(\emptyset)=k[X_1,\dots,X_n]\) \(I(\A^n(k))=(0)\) if \(k\) is an infinite
field; \(I(\{(a_1,\dots,a_n)\})=(X_1-a_1,\dots,X_n-a_n)\) for \(a_1,\dots,a_n\in k\)
\item \(I(V(S))\supset S\) for any set \(S\) of polynomials; \(V(I(X))\supset X\) for any set \(X\) of points
\item \(V(I(V(S)))=V(S)\) for any set \(S\) of polynomials; \(I(V(I(X)))=I(X)\) for any set \(X\)
of points. So if \(X\) is an algebraic set, \(X=V(I(X))\); and if \(J\) is an ideal of an
algebraic set, \(I(V(J))=J\)
\end{enumerate}


An ideal that is the ideal of an algebraic set has a property not shared with all ideals:
if \(J=I(X)\) and \(F^n\in I\) for some integer \(n>0\), then \(F\in I\). If \(I\) is any ideal in a
ring \(R\), we define the \textbf{radical} of \(I\), written \(\Rad(I)\), to
be \(\{a\in R\mid a^n\in I\text{ for some integer }n>0\}\). Then \(\Rad(I)\) is an ideal containing \(I\).
An ideal \(I\) is called a \textbf{radical ideal} if \(I=\Rad(I)\). So we have
\begin{enumerate}
\setcounter{enumi}{4}
\item \(I(X)\) is a radical ring for any \(X\subset\A^n(k)\)
\end{enumerate}


\begin{exercise}
\label{ex1.14}
Let \(F\) be a nonconstant polynomial in \(k[X_1,\dots,X_n]\), \(k\) algebraically closed. Show
that \(\A^n(k)\setminus V(F)\) is infinite if \(n\ge 1\) and \(V(F)\) is infinite if \(n\ge 2\)
\end{exercise}

\begin{proof}
\(\A^1(k)\setminus V(F)\) is infinite. Now if \(\A^n(k)\setminus V(F)\) is infinite, then for
each \((a_1,\dots,a_n,a_{n+1})\in V(F)\), \((\A^n(k)\setminus V(F))\times\{a_{n+1}\}\) is infinite.

\(V(F)=\bigcup_{a_1\in k}\cdots\bigcup_{a_{n-1}\in k}V(F(a_1,\dots,a_{n-1},X_n))\)
\end{proof}
\subsection{The Hilbert Basis Theorem}
\label{sec:orga638798}
\begin{theorem}[]
Every algebraic set is the intersection of a finite number of hypersurface
\end{theorem}

\begin{proof}
Let the algebraic set be \(V(I)\) for some ideal \(I\subset k[X_1,\dots,X_n]\). It is enough to show
that \(I\) is finitely generated, for if \(I=(F_1,\dots,F_r)\), then \(V(I)=V(F_1)\cap\dots\cap V(F_r)\). To prove
this we need some algebra:
\end{proof}

A ring is \textbf{Noetherian} if every ideal in the ring is finitely generated. Fields and PID's are
Noetherian rings. Theorem, therefore, is a consequence of

\begin{theorem}[Hilbert Basis Theorem]
If \(R\) is a Noetherian ring, then \(R[X_1,\dots,X_n]\) is a Noetherian ring
\end{theorem}

\begin{proof}
Since \(R[X_1,\dots,X_n]\cong R[X_1,\dots,X_{n-1}][X_n]\) , the theorem will follow by induction if we can
prove that \(R[X]\) is Noetherian whenever \(R\) is Noetherian. Let \(I\) be an ideal
in \(R[X]\). We must find a finite set of generators for \(I\)

If \(F=\sum_{i=0}^da_iX^i\in R[X]\), \(a_d\neq 0\), we call \(a_d\) the leading coefficient of \(F\).
Let \(J\) be the set of leading coefficients of all polynomials in \(I\). It is easy to check
that \(J\) is an ideal in \(R\), so there are polynomials \(F_1,\dots,F_r\in I\) whose leading
coefficients generate \(J\). Take an integer \(N\) larger than the degree of each \(F_i\). For
each \(m\le N\), let \(J_m\) be the ideal in \(R\) consisting of all leading coefficients of all
polynomials \(F\in I\) s.t. \(\deg(F)\le m\). Let \(\{F_{m,j}\}\) be a finite set of polynomials
in \(I\) of degree \(\le m\) whose leading coefficients generate \(J_m\). Let \(I'\) be the ideal
generated by the \(F_i\)'s and all the \(F_{m,j}\)'s. It suffices to show that \(I=I'\)

Suppose \(I'\) were smaller than \(I\); let \(G\) be an element of \(I\) of lowest degree that
is not in \(I'\). If \(\deg(G)>N\), we can find polynomials \(Q_i\) s.t. \(\sum Q_iF_i\) and \(G\)
have the same leading term. But then \(\deg(G-\sum Q_iF_i)<\deg G\) so \(G-\sum Q_iF_i\in I'\) and
so \(G\in I'\).  Similarly if \(\deg(G)=m\le N\), we can lower the degree by subtracting
off \(\sum Q_jF_{m,j}\) for some \(Q_j\). This proves the theorem
\end{proof}

\begin{corollary}[]
\(k[X_1,\dots,X_n]\) is Noetherian for any field \(k\).
\end{corollary}

\begin{exercise}
\label{ex1.22}
Let \(I\) be an ideal in a ring \(R\), \(\pi:R\to R/I\) the natural homomorphism
\begin{enumerate}
\item Show that for every ideal \(J'\) of \(R/I\), \(\pi^{-1}(J')=J\) is an ideal of \(R\)
containing \(I\). And for every ideal \(J\) of \(R\) containing \(I\), \(\pi(J)=J'\) is an
ideal of \(R/I\).

This sets up a natural one-to-one correspondence between ideals of \(R/I\) and ideals
of \(R\) that contains \(I\)

\item Show that \(J'\) is a radical ideal iff \(J\) is radical. Similarly for prime and maximal ideals

\item Show that \(J'\) is finitely generated if \(J\) is. Conclude that \(R/I\) is Noetherian
if \(R\) is Noetherian. Any ring of the form \(k[X_1,\dots,X_n]/I\) is Noetherian
\end{enumerate}
\end{exercise}
\subsection{Irreducible Components of an Algebraic Set}
\label{sec:orgf4dfcae}
An algebraic set \(V\subset\A^n\) is \textbf{reducible} if \(V=V_1\cup V_2\) where \(V_1\) and \(V_2\) are algebraic sets
in \(\A^n\) and \(V_i\neq V\) for \(i=1,2\). Otherwise \(V\) is reducible

\begin{proposition}[]
\label{prop1.1}
An algebraic set \(V\) is irreducible iff \(I(V)\) is prime
\end{proposition}

\begin{proof}
If \(I(V)\) is not prime and suppose \(F_1F_2\in I(V)\), \(F_1,F_2\notin I(V)\).
Then \(V=(V\cap V(F_1))\cup(V\cap V(F_2))\) and \(V\cap V(F_i)\subsetneq V\), so \(V\) is reducible

If \(V=V_1\cup V_2\) and \(V_i\subsetneq V\), then \(I(V_i)\supsetneq I(V)\); let \(F_i\in I(V_i)\setminus I(V)\).
Then \(F_1F_2\in I(V)\) , so \(I(V)\) is not prime
\end{proof}

We want to show that an algebraic set is the union of a finite number of irreducible algebraic
set.s

\begin{lemma}[]
Let \(\cali\) be any nonempty collection of ideals in a Noetherian ring \(R\). Then \(\cali\) has a
maximal member
\end{lemma}

\begin{proof}
Choose (using the axiom of choice) an ideal from each subset of \(\cali\). Let \(I_0\) be the chosen
ideal for \(\cali\) itself. Let \(\cali_1=\{I\in\cali\mid I\supsetneq I_0\}\), and let \(I_1\) be the chosen ideal
of \(\cali\), etc. It suffices to show that some \(\cali_n\) is empty. If not let \(I=\bigcup_{i=0}^\infty I_i\),
an ideal of \(R\). Let \(F_1,\dots,F_r\) generate \(I\); each \(F_i\in I_n\) if \(n\) is chosen
sufficiently large. But then \(I_n=I\), so \(I_{n+1}=I_n\), a contradiction
\end{proof}

It follows that any collection of algebraic sets in \(\A^n(k)\) has a minimal member. For
if \(\{V_\alpha\}\) is such a collection, take a maximal member \(I(V_{\alpha_0})\) from \(\{I(V_\alpha)\}\),
then \(V_{\alpha_0}\) is the minimal

\begin{theorem}[]
Let \(V\) be an algebraic set in \(\A^n(k)\). Then there are unique irreducible algebraic
sets \(V_1,\dots,V_m\) s.t. \(V=V_1\cup\cdots\cup V_m\) and \(V_i\not\subset V_j\) for all \(i\neq j\)
\end{theorem}

\begin{proof}
Let
\(\cali=\{\text{algebraic sets }V\subset\A^n(k)\mid V\text{ is not the union of a finite number of irreducible algebraic sets}\}\).
We want to show that \(\cali\) is empty. If not, let \(V\) be a minimal member of \(\cali\).
Since \(V\in\cali\), \(V\) is not irreducible, so \(V=V_1\cup V_2\), \(V_i\subsetneq V\). Then \(V_i\notin\cali\),
so \(V_i=V_{i1}\cup\dots\cup V_{im_i}\), \(V_{ij}\) irreducible. But then \(V=\bigcup_{i,j}V_{ij}\) , a
contradiction.

So any algebraic set \(V\) may be written as \(V=V_1\cup\dots\cup V_m\), \(V_i\) irreducible. We can throw
away any \(V_i\) s.t. \(V_i\subset V_j\) for some \(i\neq j\). To show uniqueness, let \(V=W_1\cup\dots\cup W_m\).
Then \(V_i=\bigcup_j(W_j\cap V_i)\), so \(V_i\subset W_{j(i)}\) for some \(j(i)\) since \(V_i\) is irreducible. Similarly \(V_{j(i)}\subset V_k\) for
some \(k\).
\end{proof}

The \(V_i\) are called the \textbf{irreducible components} of \(V\); \(V=V_1\cup\cdots\cup V_m\) is the \textbf{decomposition}
of \(V\) into irreducible components

\begin{exercise}
\label{ex1.25}
\begin{enumerate}
\item Show that \(V(Y-X^2)\subset\A^2(\C)\) is irreducible; in fact, \(I(V(Y-X^2))=(Y-X^2)\)
\item Decompose \(V(Y^4-X^2,Y^4-X^2Y^2+XY^2-X^3)\subset\A^2(\C)\) into irreducible components
\end{enumerate}
\end{exercise}

\begin{proof}
\begin{enumerate}
\item Consider \(h:\C[X,Y]\to\C[X]\) by \(h(f(x,y))=f(x,x^2)\). This is a homomorphism and
thus \(\C[X,Y]/(Y-X^2)\cong\C[X]\). Thus \((Y-X^2)\) is prime
\item Solution is finite
\end{enumerate}
\end{proof}

\begin{exercise}
\label{ex1.28}
If \(V=V_1\cup\dots\cup V_r\) is the decomposition of an algebraic set into irreducible components, show
that \(V_i\not\subset\bigcup_{j\neq i}V_j\)
\end{exercise}

\begin{proof}
suppose \(V_i\subset\bigcup_{j\neq i}V_j\), then \(V_i=\bigcup_{j\neq i}(V_j\cap V_i)\)
\end{proof}

\begin{exercise}
\label{ex1.29}
Show that \(\A^n(k)\) is irreducible if \(k\) is infinite
\end{exercise}

\begin{proof}
\(\A^1(k)\) is irreducible

For each \(a\in k\), \(\A^n(k)\times\{a\}\) is irreducible
\end{proof}
\subsection{Algebraic Subsets of the Plane}
\label{sec:org047750a}
\begin{proposition}[]
\label{prop1.2}
Let \(F\) and \(G\) be polynomials in \(k[X,Y]\) with no common factors.
Then \(V(F,G)=V(F)\cap V(G)\) is a finite set of points
\end{proposition}

\begin{proof}
\(F\) and \(G\) have no common factors in \(k[X][Y]\), so they also have no common factors
in \(k(X)[Y]\). Since \(k(X)[Y]\) is a PID, \((F,G)=(1)\) in \(k(X)[Y]\), so \(RF+SG=1\) for
some \(R,S\in k(X)[Y]\). There is a nonzero \(D\in k{X}\) s.t. \(DR=A,DS=B\in k[X,Y]\).
Therefore \(AF+BG=D\). If \((a,b)\in V(F,G)\) then \(D(a)=0\). But \(D\) has only a finite number
of zeros, this shows that a finite number of \(X\)-coordinates appear among the points
of \(V(F,G)\). Since the same reasoning applies to the \(Y\)-coordinates, there can be only a
finite number of points
\end{proof}

\begin{corollary}[]
If \(F\) is an irreducible polynomials in \(k[X,Y]\) s.t. \(V(F)\) is infinite,
then \(I(V(F))=(F)\) and \(V(F)\) is irreducible
\end{corollary}

\begin{proof}
If \(G\in I(V(F))\), then \(V(F,G)\) is infinite, so \(F\) divides \(G\) by the proposition,
i.e., \(G\in(F)\). \(V(F)\) is irreducible follows from Proposition \ref{prop1.1}.
\end{proof}

\begin{corollary}[]
Suppose \(k\) is infinite. Then the irreducible algebraic subsets of \(\A^2(k)\)
are: \(\A^2(k)\), \(\emptyset\), points, and irreducible plane curves \(V(F)\) where \(F\) is an
irreducible polynomial and \(V(F)\) is infinite
\end{corollary}

\begin{proof}
Let \(V\) be an irreducible algebraic set in \(\A^2(k)\). If \(V\) is finite
or \(I(V)=(0)\), \(V\) is of the required type. Otherwise \(I(V)\) contains a nonconstant
polynomial \(F\); since \(I(V)\) is prime, some irreducible polynomial factor of \(F\) belongs
to \(I(V)\), so we may assume \(F\) is irreducible. Then \(I(V)=(F)\); for
if \(G\in I(V)\), \(G\notin(F)\), then \(V\subset V(F,G)\) is finite.
\end{proof}

\begin{corollary}[]
Assume \(k\) is algebraically closed, \(F\) a nonconstant polynomial in \(k[X,Y]\).
Let \(F=F_1^{n_1}\dots F_r^{n_r}\) be the decomposition of \(F\) into irreducible factors.
Then \(V(F)=V(F_1)\cup\dots\cup V(F_r)\) is the decomposition of \(V(F)\) into irreducible components,
and \(I(V(F))=(F_1,\dots,F_r)\)
\end{corollary}

\begin{proof}
No \(F_i\) divides any \(F_j\), \(j\neq i\), so there are no inclusion relations among
the \(V(F_i)\). And \(I(\bigcup_iV(F_i))=\bigcap_iI(V(F_i))=\bigcap_i(F_i)\). Since any polynomial divisible by
each \(F_i\) is also divisible by \(F_1\cdots F_r\), \(\bigcap_i(F_i)=(F_1\cdots F_r)\). Note that the \(V(F_i)\) are
infinite since \(k\) is algebraically closed
\end{proof}
\subsection{Hilbert's Nullstellensatz}
\label{sec:orgb811f17}
Assume \(k\) is algebraically closed

\begin{theorem}[Weak Nullstellensatz]
If \(I\) is a proper ideal in \(k[X_1,\dots,X_n]\), then \(V(I)\neq\emptyset\)
\end{theorem}

\begin{proof}
We may assume that \(I\) is a maximal ideal, for there is a maximal ideal \(J\) containing \(I\)
and \(V(J)\subset V(I)\). So \(L=k[X_1,\dots,X_n]/I\) is a field, and \(k\) may be regared as a subfield
of \(L\)

Suppose we knew that \(k=L\), then for each \(i\) there is an \(a_i\in k\) s.t. the \(I\)-residue
of \(X_i\) is \(a_i\), or \(X_i-a_i\in I\). But \((X_1-a_1,\dots,X_n-a_n)\) is a maximal ideal,
so \(I=(X_1-a_1,\dots,X_n-a_n)\) and \(V(I)=\{(a_1,\dots,a_n)\}\neq\emptyset\)

Thus we have reduced problem to showing:

\textbf{Claim ($\backslash$(}$\backslash$))*: If an algebraically closed field \(k\) is a subfield of a field \(L\), and there is a
 ring homomorphism from \(k[X_1,\dots,X_n]\) onto \(L\) (identity on \(k\)), then \(k=L\)

This will be proved later
\end{proof}

\begin{theorem}[Hilbert's Nullstellensatz]
Let \(I\) be an ideal in \(k[X_1,\dots,X_n]\), then \(I(V(I))=\Rad(I)\)
\end{theorem}

This says the following: if \(F_1,\dots,F_r\) and \(G\) are in \(k[X_1,\dots,X_n]\) and \(G\) vanishes
whenever \(F_1,\dots,F_r\) vanish, then there is an equation \(G^N=A_1F_1+A_2F_2+\dots+A_rF^r\) for some \(N>0\)
and some \(A_i\in k[X_1,\dots,X_n]\)

\begin{proof}
\(\Rad(I)\subset I(V(I))\) is easy. Suppose \(G\in I(V(F_1,\dots,F_r))\), \(F_i\in k[X_1,\dots,X_n]\).
Let \(J=(F_1,\dots,F_r,X_{n+1}G-1)\subset k[X_1,\dots,X_n,X_{n+1}]\). Then \(V(J)\subset\A^{n+1}(k)\) is empty,
since \(G\) vanishes whenever all that \(F_i\)'s are zero. Applying the Weak Nullstellensatz
to \(J\), we see that \(1\in J\), so there is an
equation \(1=\sum A_i(X_1,\dots,X_{n+1})F_i+B(X_1,\dots,X_{n+1})(X_{n+1}G-1)\). Let \(Y=1/X_{n+1}\), and
multiply the equation by a higher power of \(Y\), so that an equation
\(Y^N=\sum C_i(X_1,\dots,X_n,Y)F_i+D(X_1,\dots,X_n,Y)(G-Y)\) in \(k[X_1,\dots,X_n,Y]\) results. Substituting \(G\)
for \(Y\) gives the required equation
\end{proof}

\begin{corollary}[]
If \(I\) is a radical ideal in \(k[X_1,\dots,X_n]\), then \(I(V(I))=I\). So there is a one-to-one
correspondence between radical ideals and algebraic sets
\end{corollary}

\begin{corollary}[]
If \(I\) is a prime ideal, then \(V(I)\) is irreducible. There is a one-to-one correspondence
between prime ideals and irreducible algebraic sets. The maximal ideals correspond to points
\end{corollary}

\begin{proof}
\(I\) is prime \(\Rightarrow\) \(I\) is radical \(\Rightarrow\) \(I(V(I))=I\). \(V(I)\) is irreducible \(\Leftrightarrow\) \(I(V(I))\) is prime
\end{proof}

\begin{corollary}[]
Let \(F\) be a nonconstant polynomial in \(k[X_1,\dots,X_n]\), \(F=F_1^{n_1}\dots F_r^{n_r}\) the
decomposition of \(F\) into irreducible factors. Then \(V(F)=V(F_1)\cup\dots\cup V(F_r)\) is the decomposition
of \(V(F)\) into irreducible components, and \(I(V(F))=(F_1\cdots F_r)\). There is a one-to-one
correspondence between irreducible polynomials \(F\in k[X_1,\dots,X_n]\) (up to multiplication by a
nonzero element of \(k\)) and irreducible hypersurfaces in \(\A^n(k)\)
\end{corollary}

\begin{corollary}[]
Let \(I\) be an ideal in \(k[X_1,\dots,X_n]\). Then \(V(I)\) is a finite set iff \(k[X_1,\dots,X_n]/I\) is
a finite dimensional vector space over \(k\). If this occurs, the number of points in \(V(I)\)
is \textbf{at most} \(\dim_k(k[X_1,\dots,X_n]/I)\)
\end{corollary}

\begin{proof}
Let \(P_1,\dots,P_r\in V(I)\). Choose \(F_1,\dots,F_r\in k[X_1,\dots,X_n]\) s.t. \(F_i(P_j)=0\) iff \(i\neq j\)
and \(F_i(P_i)=1\); let \(\barF_i\) be the \(I\)-residue of \(F_i\). If \(\sum\lambda_i\barF_i=0\), \(\lambda_i\in k\),
then \(\sum\lambda_iF_i\in I\), so \(\lambda_j=(\sum\lambda_iF_i)(P_j)=0\). Thus the \(\barF_i\) are linearly independent
over \(k\), so \(r\le\dim_k(k[X_1,\dots,X_n]/I)\)

Conversely if \(V(I)=\{P_1,\dots,P_r\}\) is finite, let \(P_i=(a_{i1},\dots,a_{in})\), and define \(F_j\)
by \(F_j=\prod_{i=1}^r(X_j-a_{ij})\), \(j=1,\dots,n\). Then \(F_j\in I(V(I))\), so \(F_j^N\in I\) for
some \(N>0\) (Take \(N\) large enough to work for all \(F_j\)).
Taking \(I\)-residues, \(\barF_j^N=0\), so \(\barX_j^{rN}\) is a \(k\)-linear combination
of \(\bar{1},\barX_j,\dots,\barX_j^{rN-1}\). It follows by induction that \(\barX_j^s\) is
a \(k\)-linear combination of \(\bar{1},\barX_j,\dots,\barX_j^{rN-1}\) for all \(s\), and hence that
the set \(\{\barX_1^{m_1},\dots,\barX_n^{m_n}\mid m_i<rN\}\) generates \(k[X_1,\dots,X_n]/I\) as a vector space
over \(k\)
\end{proof}

\begin{exercise}
\label{ex1.32}
Show that both theorems and all of the corollaries are false if \(k\) is not algebraically closed
\end{exercise}

\begin{proof}
Consider \(h:\R[X]\to\C\) by \(h(X)=i\). Then this is a surjection but \(\R\neq\C\). Thus (*) is false.
\end{proof}

\begin{exercise}
\begin{enumerate}
\item Decompose \(V(X^2+Y^2-1,X^2-Z^2-1)\subseteq\A^3(\C)\) into irreducible components
\item Let \(V=\{(t,t^2,t^3)\in\A^3(\C)\mid t\in\C\}\). Find \(I(V)\) and show that \(V\) is irreducible
\end{enumerate}
\end{exercise}

\begin{proof}
\begin{enumerate}
\item \(X^2+Y^2-1=0\wedge X^2-Z^2-1=0\Rightarrow Y=Z=0\wedge X^2=1\)
\item \(V=V(Y-X^2,Z-X^3)\). Consider \(h:\C[X,Y,Z]\to\C[X]\) by \(h(f(x,y,z))=f(x,x^2,x^3)\). Then
\(\C[X,Y,Z]/(Y-X^2,Z-X^3)\cong\C[X]\) and hence \((Y-X^2,Z-X^3)\) is prime and so \(I(V)=(Y-X^2,Z-X^3)\)
\end{enumerate}
\end{proof}

\begin{exercise}
Let \(R\) be a UFD
\begin{enumerate}
\item Show that a monic polynomial of degree two or three in \(R[X]\) is irreducible iff it has no
roots in \(R\)
\item The polynomial \(X^2-a\in R[X]\) is irreducible iff \(a\) is not a square in \(R\)
\end{enumerate}
\end{exercise}

\begin{exercise}
\label{ex1.35}
Show that \(V(Y^2-X(X-1)(X-\lambda))\subset\A^2(k)\) is an irreducible curve for any algebraically closed
field \(k\), and any \(\lambda\in k\)
\end{exercise}

\begin{theorem}[Eisenstein's criterion]
Let \(R\) be an integral domain and let \(f=a_0+a_1T+\dots+a_nT^n\in R[T]\) be a polynomial. Suppose
that there exists a prime ideal \(\fp\) of \(R\) s.t.
\begin{enumerate}
\item \(a_i\in\fp\) for \(i=0,\dots,n-1\)
\item \(a_n\notin\fp\)
\item \(a_0\notin\fp^2\)
\end{enumerate}


Then \(f\) is irreducible in \(R[T]\)
\end{theorem}


\begin{proof}
\href{https://math.stackexchange.com/questions/2668988/exercise-1-35-in-fultons-algebraic-curves}{Solution}
If \(\lambda=0\) and take \(\fp=(x)\). Then \(a_2=1\notin(x)\) and \(a_0=x(x-1)(x-\lambda)\in(x)\) while \(a_0\notin(x^2)\)
as \(\lambda\neq 0\). Thus \(f(y)\) is irreducible

Suppose that \(\lambda=0\). Consider \(\fp=(x-1)\)
\end{proof}

\begin{exercise}
\label{ex1.36}
Let \(I=(Y^2-X^2,Y^2+X^2)\subset\C[X,Y]\). Find \(V(I)\) and \(\dim_{\C}(\C[X,Y]/I)\)
\end{exercise}

\begin{proof}
\(I=(X^2,Y^2)\). Thus element of  \(\C[X,Y]/I\) is of the form \(a+bx+cy+dxy+I\). So \(\{1,x,y,xy\}\)
is a basis for \(\C[X,Y]/I\)
\end{proof}

\begin{exercise}
\label{1.37}
Let \(K\) be any field, \(F\in K[X]\) a polynomial of degree \(n>0\). Show that the
residues \(\bar{1},\barX,\dots,\barX^{n-1}\) form a basis of \(K[X]/(F)\) over \(K\)
\end{exercise}

\begin{proof}
We can view \(F\) as a monic polynomial. Then every residue has degree less than \(n\).
Thus \(\bar{1},\barX,\dots,\barX^{n-1}\) generate \(K[X]/(F)\).
Suppose \(a_0\bar{1}+a_1\barX+\dots+a_{n-1}\barX^{n-1}=\bbar{a_0+a_1X+\dots+a_{n-1}X^{n-1}}=0\). Then
\(a_0+a_1X+\dots+a_{n-1}X^{n-1}\in(F)\). Hence \(a_0=\dots=a_{n-1}=0\)
\end{proof}
\end{document}
