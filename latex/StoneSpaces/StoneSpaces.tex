% Created 2021-03-09 二 09:56
% Intended LaTeX compiler: pdflatex
\documentclass[11pt]{article}
\usepackage[utf8]{inputenc}
\usepackage[T1]{fontenc}
\usepackage{graphicx}
\usepackage{grffile}
\usepackage{longtable}
\usepackage{wrapfig}
\usepackage{rotating}
\usepackage[normalem]{ulem}
\usepackage{amsmath}
\usepackage{textcomp}
\usepackage{amssymb}
\usepackage{capt-of}
\usepackage{hyperref}
% TIPS
% \substack{a\\b} for multiple lines text





% pdfplots will load xolor automatically without option
\usepackage[dvipsnames]{xcolor}

\usepackage{forest}
% two-line text in node by [two \\ lines]
% \begin{forest} qtree, [..] \end{forest}
\forestset{
  qtree/.style={
    baseline,
    for tree={
      parent anchor=south,
      child anchor=north,
      align=center,
      inner sep=1pt,
    }}}
%\usepackage{flexisym}
% load order of mathtools and mathabx, otherwise conflict overbrace

\usepackage{mathtools}
%\usepackage{fourier}
\usepackage{pgfplots}
\usepackage{amsthm, mathabx,  amsmath, commath}
\usepackage{amsfonts}

\usepackage{empheq}
\usepackage{tikz}
\usetikzlibrary{arrows.meta}
\usepackage[most]{tcolorbox}

\newtheorem{theorem}{Theorem}[section]
\newtheorem{definition}{Definition}[section]
\newtheorem{corollary}{Corollary}[section]
\newtheorem{example}{Example}[section]
\newtheorem{lemma}{Lemma}[section]
\newtheorem{proposition}{Proposition}[section]

\newcommand{\bl}[1] {\boldsymbol{#1}}
\newcommand{\Wt}[1] {\stackrel{\sim}{\smash{#1}\rule{0pt}{1.1ex}}}
\newcommand{\wt}[1] {\widetilde{#1}}


%For boxed texts in align, use Aboxed{}
%otherwise use boxed{}

\DeclareMathSymbol{\widehatsym}{\mathord}{largesymbols}{"62}
\newcommand\lowerwidehatsym{%
  \text{\smash{\raisebox{-1.3ex}{%
    $\widehatsym$}}}}
\newcommand\fixwidehat[1]{%
  \mathchoice
    {\accentset{\displaystyle\lowerwidehatsym}{#1}}
    {\accentset{\textstyle\lowerwidehatsym}{#1}}
    {\accentset{\scriptstyle\lowerwidehatsym}{#1}}
    {\accentset{\scriptscriptstyle\lowerwidehatsym}{#1}}
}

\usepackage{graphicx}
    
% text on arrow for xRightarrow
\makeatletter
%\newcommand{\xRightarrow}[2][]{\ext@arrow 0359\Rightarrowfill@{#1}{#2}}
\makeatother


\def \bx {\boldsymbol{x}}
\def \ba {\boldsymbol{a}}
\def \bI {\boldsymbol{I}}
\def \bt {\boldsymbol{t}}
\def \bb {\boldsymbol{b}}
\def \bA {\boldsymbol{A}}
\def \bX {\boldsymbol{X}}
\def \bu {\boldsymbol{u}}
\def \bS {\boldsymbol{S}}
\def \bZ {\boldsymbol{Z}}
\def \bz {\boldsymbol{z}}
\def \by {\boldsymbol{y}}
\def \bw {\boldsymbol{w}}
\def \bT {\boldsymbol{T}}
\def \bS {\boldsymbol{S}}
\def \bm {\boldsymbol{m}}
\def \bW {\boldsymbol{W}}
\def \bY {\boldsymbol{Y}}
\def \bH {\boldsymbol{H}}
\def \blambda {\boldsymbol{\lambda}}
\def \bPhi {\boldsymbol{\Phi}}
\def \btheta {\boldsymbol{\theta}}
\def \bmu {\boldsymbol{\mu}}
\def \bphi {\boldsymbol{\phi}}
\def \bSigma {\boldsymbol{\Sigma}}
\def \lb {\left\{}
\def \rb {\right\}}
\def \caln {\mathcal{N}}
\def \dissum {\displaystyle\Sigma}
\def \dispro {\displaystyle\prod}
\def \E {\mathbb{E}}
\def \Q {\mathbb{Q}}
\def \V {\mathbb{V}}
\def \R {\mathbb{R}}
\def \calq {\mathcal{Q}}
\def \calg {\mathcal{G}}
\def \caln {\mathcal{N}}
\def \calr {\mathcal{R}}
\def \calm {\mathcal{M}}
\def \calc {\mathcal{C}}
\def \bcup {\bigcup}

\author{Peter T. Johnstone}
\date{\today}
\title{Stone Spaces}
\hypersetup{
 pdfauthor={Peter T. Johnstone},
 pdftitle={Stone Spaces},
 pdfkeywords={},
 pdfsubject={},
 pdfcreator={Emacs 27.1 (Org mode 9.3)}, 
 pdflang={English}}
\begin{document}

\maketitle
\tableofcontents


\section{Preliminaries}
\label{sec:orgfd1ad42}

\subsection{Lattices}
\label{sec:orga07e204}
Let \(A\) be a poset in which every finite subset has a join. Then the binary
operation \(\vee\) and the element 0 defiend above satisfy the equations
\begin{enumerate}
\item \(a\vee a=a\)
\item \(a\vee b=b\vee a\)
\item \(a\vee(b\vee c)=(a\vee b)\vee c\)
\item \(a\vee0=a\)
\end{enumerate}


for all \(a,b,c\). We can say that \((A,\vee,0)\) is a commutative monoid
where every element is idempotent. Conversely we have

\begin{theorem}[]
Let \((A,\vee,0)\) be a commutative monoid in which every element is
idempotent. Then there exists a unique partial order on \(A\) s.t.
\(a\vee b\) is the join of \(a\) and \(b\), and 0 is the least element
\end{theorem}

\begin{proof}
If such a partial order exists, we must have \(a\le b\) iff \(a\vee b=b\).
Take this as a definition of \(\le\). Suppose \(a\le b\) and \(b\le c\). Then
\begin{align*}
a\vee c&=a\vee(b\vee c)\\
&=(a\vee b)\vee c\\
&=b\vee c=c
\end{align*}
so \(a\le c\).

Now let \(a,b\) be any two elements of \(A\). Then
\(a\vee(a\vee b)=(a\vee a)\vee b=a\vee b\), so \(a\le a\vee b\) and similarly
\(b\le a\vee b\). But if \(a\le c\) and \(b\le c\), then \((a\vee b)\vee
   c=a\vee(b\vee c)=a\vee c=c\), so \(a\vee b\le c\); i.e. \(a\vee b\) is the
join of \(a\) and \(b\).
\end{proof}

A set with the structure described in the theorem is called a \textbf{semilattice}. A
semilattice homomorphism \(f:A\to B\) (i.e. a map preserving the
distinguished element 0 and the operation \(\vee\)) is necessarily an
order-preserving map.

A \textbf{lattice} is a poset \(A\) in which every finite subset has both a join and a
meet.

\begin{proposition}[]
Suppose \((A,\vee,0)\) and \((A,\wedge,1)\) are semilattices. Then
\((A,\vee,\wedge,0,1)\) is a lattice iff the \textbf{absorptive laws}
\begin{equation*}
a\wedge(a\vee b)=a,\quad a\vee(a\wedge b)=a
\end{equation*}
are satisfied for all \(a,b\in A\)
\end{proposition}

\begin{proof}
If \(a\vee b=b\), then \(a\wedge b=a\wedge(a\vee b)=a\). So the two partial
orders on \(A\) agree.
\end{proof}

\textbf{distributive law}
\begin{equation*}
a\wedge(b\vee c)=(a\wedge b)\vee(a\wedge c)
\end{equation*}

\begin{lemma}[]
If the distributive law holds in a lattice, then so does its dual, i.e. the
identity
\begin{equation*}
a\vee(b\wedge c)=(a\vee b)\wedge(a\vee c)
\end{equation*}
\end{lemma}

\begin{proof}
\begin{align*}
(a\vee b)\wedge(a\vee c)&=
((a\vee b)\wedge a)\vee((a\vee b)\wedge c)\\
&=
\end{align*}
\end{proof}
\end{document}