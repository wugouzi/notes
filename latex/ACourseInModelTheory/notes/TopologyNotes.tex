% Created 2021-08-13 Fri 12:03
% Intended LaTeX compiler: pdflatex
\documentclass[11pt]{article}
\usepackage[utf8]{inputenc}
\usepackage[T1]{fontenc}
\usepackage{graphicx}
\usepackage{grffile}
\usepackage{longtable}
\usepackage{wrapfig}
\usepackage{rotating}
\usepackage[normalem]{ulem}
\usepackage{amsmath}
\usepackage{textcomp}
\usepackage{amssymb}
\usepackage{capt-of}
\usepackage{hyperref}
% TIPS
% \substack{a\\b} for multiple lines text





% pdfplots will load xolor automatically without option
\usepackage[dvipsnames]{xcolor}

\usepackage{forest}
% two-line text in node by [two \\ lines]
% \begin{forest} qtree, [..] \end{forest}
\forestset{
  qtree/.style={
    baseline,
    for tree={
      parent anchor=south,
      child anchor=north,
      align=center,
      inner sep=1pt,
    }}}
%\usepackage{flexisym}
% load order of mathtools and mathabx, otherwise conflict overbrace

\usepackage{mathtools}
%\usepackage{fourier}
\usepackage{pgfplots}
\usepackage{amsthm, mathabx,  amsmath, commath}
\usepackage{amsfonts}

\usepackage{empheq}
\usepackage{tikz}
\usetikzlibrary{arrows.meta}
\usepackage[most]{tcolorbox}

\newtheorem{theorem}{Theorem}[section]
\newtheorem{definition}{Definition}[section]
\newtheorem{corollary}{Corollary}[section]
\newtheorem{example}{Example}[section]
\newtheorem{lemma}{Lemma}[section]
\newtheorem{proposition}{Proposition}[section]

\newcommand{\bl}[1] {\boldsymbol{#1}}
\newcommand{\Wt}[1] {\stackrel{\sim}{\smash{#1}\rule{0pt}{1.1ex}}}
\newcommand{\wt}[1] {\widetilde{#1}}


%For boxed texts in align, use Aboxed{}
%otherwise use boxed{}

\DeclareMathSymbol{\widehatsym}{\mathord}{largesymbols}{"62}
\newcommand\lowerwidehatsym{%
  \text{\smash{\raisebox{-1.3ex}{%
    $\widehatsym$}}}}
\newcommand\fixwidehat[1]{%
  \mathchoice
    {\accentset{\displaystyle\lowerwidehatsym}{#1}}
    {\accentset{\textstyle\lowerwidehatsym}{#1}}
    {\accentset{\scriptstyle\lowerwidehatsym}{#1}}
    {\accentset{\scriptscriptstyle\lowerwidehatsym}{#1}}
}

\usepackage{graphicx}
    
% text on arrow for xRightarrow
\makeatletter
%\newcommand{\xRightarrow}[2][]{\ext@arrow 0359\Rightarrowfill@{#1}{#2}}
\makeatother


\def \bx {\boldsymbol{x}}
\def \ba {\boldsymbol{a}}
\def \bI {\boldsymbol{I}}
\def \bt {\boldsymbol{t}}
\def \bb {\boldsymbol{b}}
\def \bA {\boldsymbol{A}}
\def \bX {\boldsymbol{X}}
\def \bu {\boldsymbol{u}}
\def \bS {\boldsymbol{S}}
\def \bZ {\boldsymbol{Z}}
\def \bz {\boldsymbol{z}}
\def \by {\boldsymbol{y}}
\def \bw {\boldsymbol{w}}
\def \bT {\boldsymbol{T}}
\def \bS {\boldsymbol{S}}
\def \bm {\boldsymbol{m}}
\def \bW {\boldsymbol{W}}
\def \bY {\boldsymbol{Y}}
\def \bH {\boldsymbol{H}}
\def \blambda {\boldsymbol{\lambda}}
\def \bPhi {\boldsymbol{\Phi}}
\def \btheta {\boldsymbol{\theta}}
\def \bmu {\boldsymbol{\mu}}
\def \bphi {\boldsymbol{\phi}}
\def \bSigma {\boldsymbol{\Sigma}}
\def \lb {\left\{}
\def \rb {\right\}}
\def \caln {\mathcal{N}}
\def \dissum {\displaystyle\Sigma}
\def \dispro {\displaystyle\prod}
\def \E {\mathbb{E}}
\def \Q {\mathbb{Q}}
\def \V {\mathbb{V}}
\def \R {\mathbb{R}}
\def \calq {\mathcal{Q}}
\def \calg {\mathcal{G}}
\def \caln {\mathcal{N}}
\def \calr {\mathcal{R}}
\def \calm {\mathcal{M}}
\def \calc {\mathcal{C}}
\def \bcup {\bigcup}

\makeindex
\author{John Doe}
\date{\today}
\title{Topology Notes}
\hypersetup{
 pdfauthor={John Doe},
 pdftitle={Topology Notes},
 pdfkeywords={},
 pdfsubject={},
 pdfcreator={Emacs 27.2 (Org mode 9.5)}, 
 pdflang={English}}
\begin{document}

\maketitle
Copy from Munkres

\begin{definition}
A \textbf{topology} on a set is a collection \(\calt\) of subsets of \(X\) having the following properties
\begin{enumerate}
\item \(\emptyset\) and \(X\) are in \(\calt\)
\item The union of the elements of any subcollection of \(\calt\) is in \(T\)
\item The intersection of the elements of any finite subcollection of \(\calt\) is in \(\calt\)
\end{enumerate}


A set \(X\) for which a topology \(\calt\) has been specified is called a \textbf{topological space}
\end{definition}

\begin{definition}[]
If \(X\) is a set, a \textbf{basis} for a topology on \(X\) is a collection \(\calb\) of subsets of \(X\)
(called \textbf{basis element}) s.t.
\begin{enumerate}
\item for each \(x\in X\), there is at least one basis element \(B\) s.t. \(x\in B\)
\item if \(x\in B_1\cap B_2\), then there is a basis element \(B_3\) s.t. \(x\in B_3\subset B_1\cap B_2\)
\end{enumerate}


If \(\calb\) satisfies these conditions, then we define the \textbf{topology \(\calt\) generated by \(\calb\)} as
follows: A subset \(U\) of \(X\) is said to be open in \(X\) if for each \(x\in U\), there is a
basis \(B\in\calb\) s.t. \(x\in B\subset U\).
\end{definition}

\begin{definition}[]
A collection \(\cala\) of subsets of a space \(X\) is said to \textbf{cover} \(X\), or to be a \textbf{covering}
of \(X\), if \(\bigcup\cala=X\). It is called an \textbf{open covering} of \(X\) if its elements are open subsets of \(X\)
\end{definition}

\begin{definition}[]
A space \(X\) is said to be \textbf{compact} if every open covering \(\cala\) of \(X\) contains a finite
subcollection that also covers \(X\).
\end{definition}

\begin{definition}
A collection \(\cala\) of subsets of the space \(X\) is said to have \textbf{order} \(m+1\) if some point
of \(X\) lies in \(m+1\) elements of \(\cala\), and no point of \(X\) lies in more than \(m+1\)
elements of \(\cala\).
\end{definition}

Given a collection \(\cala\) of subsets of \(X\), a collection \(\calb\) is said to \textbf{refine} \(\cala\), or to
be a \textbf{refinement} of \(\cala\), if for each element \(B\) of \(\calb\) there is an element \(A\) of \(\cala\)
s.t. \(B\subset A\)

\begin{definition}[]
A space \(X\) is said to be \textbf{finite dimensional} if there is some integer \(m\) s.t. for every
open covering \(\cala\) of \(X\), there is an open covering \(\calb\) of \(X\) that refines \(\cala\) and
has order at most \(m+1\). The \textbf{topological dimension} of \(X\) is defined to be the smallest
value of \(m\) for which this statement holds; we denote it by \(\dim X\).
\end{definition}
\end{document}
