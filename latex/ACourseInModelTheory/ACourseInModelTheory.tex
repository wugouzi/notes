% Created 2021-07-24 Sat 21:39
% Intended LaTeX compiler: pdflatex
\documentclass[11pt]{article}
\usepackage[utf8]{inputenc}
\usepackage[T1]{fontenc}
\usepackage{graphicx}
\usepackage{grffile}
\usepackage{longtable}
\usepackage{wrapfig}
\usepackage{rotating}
\usepackage[normalem]{ulem}
\usepackage{amsmath}
\usepackage{textcomp}
\usepackage{amssymb}
\usepackage{capt-of}
\usepackage{hyperref}
% wrong resolution of image
% https://tex.stackexchange.com/questions/21627/image-from-includegraphics-showing-in-wrong-image-size?rq=1

%%%%%%%%%%%%%%%%%%%%%%%%%%%%%%%%%%%%%%
%% TIPS                                 %%
%%%%%%%%%%%%%%%%%%%%%%%%%%%%%%%%%%%%%%
% \substack{a\\b} for multiple lines text
% \usepackage{expl3}
% \expandafter\def\csname ver@l3regex.sty\endcsname{}
% \usepackage{pkgloader}
\usepackage[utf8]{inputenc}

% nfss error
% \usepackage[B1,T1]{fontenc}
\usepackage{fontspec}

% \usepackage[Emoticons]{ucharclasses}
\newfontfamily\DejaSans{DejaVu Sans}
% \setDefaultTransitions{\DejaSans}{}

% pdfplots will load xolor automatically without option
\usepackage[dvipsnames]{xcolor}

%                                                             ┳┳┓   ┓
%                                                             ┃┃┃┏┓╋┣┓
%                                                             ┛ ┗┗┻┗┛┗
% \usepackage{amsmath} mathtools loads the amsmath
\usepackage{amsmath}
\usepackage{mathtools}

\usepackage{amsthm}
\usepackage{amsbsy}

%\usepackage{commath}

\usepackage{amssymb}

\usepackage{mathrsfs}
%\usepackage{mathabx}
\usepackage{stmaryrd}
\usepackage{empheq}

\usepackage{scalerel}
\usepackage{stackengine}
\usepackage{stackrel}



\usepackage{nicematrix}
\usepackage{tensor}
\usepackage{blkarray}
\usepackage{siunitx}
\usepackage[f]{esvect}

% centering \not on a letter
\usepackage{slashed}
\usepackage[makeroom]{cancel}

%\usepackage{merriweather}
\usepackage{unicode-math}
\setmainfont{TeX Gyre Pagella}
% \setmathfont{STIX}
%\setmathfont{texgyrepagella-math.otf}
%\setmathfont{Libertinus Math}
\setmathfont{Latin Modern Math}

 % \setmathfont[range={\smwhtdiamond,\enclosediamond,\varlrtriangle}]{Latin Modern Math}
\setmathfont[range={\rightrightarrows,\twoheadrightarrow,\leftrightsquigarrow,\triangledown,\vartriangle,\precneq,\succneq,\prec,\succ,\preceq,\succeq,\tieconcat}]{XITS Math}
 \setmathfont[range={\int,\setminus}]{Libertinus Math}
 % \setmathfont[range={\mathalpha}]{TeX Gyre Pagella Math}
%\setmathfont[range={\mitA,\mitB,\mitC,\mitD,\mitE,\mitF,\mitG,\mitH,\mitI,\mitJ,\mitK,\mitL,\mitM,\mitN,\mitO,\mitP,\mitQ,\mitR,\mitS,\mitT,\mitU,\mitV,\mitW,\mitX,\mitY,\mitZ,\mita,\mitb,\mitc,\mitd,\mite,\mitf,\mitg,\miti,\mitj,\mitk,\mitl,\mitm,\mitn,\mito,\mitp,\mitq,\mitr,\mits,\mitt,\mitu,\mitv,\mitw,\mitx,\mity,\mitz}]{TeX Gyre Pagella Math}
% unicode is not good at this!
%\let\nmodels\nvDash

 \usepackage{wasysym}

 % for wide hat
 \DeclareSymbolFont{yhlargesymbols}{OMX}{yhex}{m}{n} \DeclareMathAccent{\what}{\mathord}{yhlargesymbols}{"62}

%                                                               ┏┳┓•┓
%                                                                ┃ ┓┃┏┓
%                                                                ┻ ┗┛┗┗

\usepackage{pgfplots}
\pgfplotsset{compat=1.18}
\usepackage{tikz}
\usepackage{tikz-cd}
\tikzcdset{scale cd/.style={every label/.append style={scale=#1},
    cells={nodes={scale=#1}}}}
% TODO: discard qtree and use forest
% \usepackage{tikz-qtree}
\usepackage{forest}

\usetikzlibrary{arrows,positioning,calc,fadings,decorations,matrix,decorations,shapes.misc}
%setting from geogebra
\definecolor{ccqqqq}{rgb}{0.8,0,0}

%                                                          ┳┳┓•    ┓┓
%                                                          ┃┃┃┓┏┏┏┓┃┃┏┓┏┓┏┓┏┓┓┏┏
%                                                          ┛ ┗┗┛┗┗ ┗┗┗┻┛┗┗ ┗┛┗┻┛
%\usepackage{twemojis}
\usepackage[most]{tcolorbox}
\usepackage{threeparttable}
\usepackage{tabularx}

\usepackage{enumitem}
\usepackage[indLines=false]{algpseudocodex}
\usepackage[]{algorithm2e}
% \SetKwComment{Comment}{/* }{ */}
% \algrenewcommand\algorithmicrequire{\textbf{Input:}}
% \algrenewcommand\algorithmicensure{\textbf{Output:}}
% wrong with preview
\usepackage{subcaption}
\usepackage{caption}
% {\aunclfamily\Huge}
\usepackage{auncial}

\usepackage{float}

\usepackage{fancyhdr}

\usepackage{ifthen}
\usepackage{xargs}

\definecolor{mintedbg}{rgb}{0.99,0.99,0.99}
\usepackage[cachedir=\detokenize{~/miscellaneous/trash}]{minted}
\setminted{breaklines,
  mathescape,
  bgcolor=mintedbg,
  fontsize=\footnotesize,
  frame=single,
  linenos}
\usemintedstyle{xcode}
\usepackage{tcolorbox}
\usepackage{etoolbox}



\usepackage{imakeidx}
\usepackage{hyperref}
\usepackage{soul}
\usepackage{framed}

% don't use this for preview
%\usepackage[margin=1.5in]{geometry}
% \usepackage{geometry}
% \geometry{legalpaper, landscape, margin=1in}
\usepackage[font=itshape]{quoting}

%\LoadPackagesNow
%\usepackage[xetex]{preview}
%%%%%%%%%%%%%%%%%%%%%%%%%%%%%%%%%%%%%%%
%% USEPACKAGES end                       %%
%%%%%%%%%%%%%%%%%%%%%%%%%%%%%%%%%%%%%%%

%%%%%%%%%%%%%%%%%%%%%%%%%%%%%%%%%%%%%%%
%% Algorithm environment
%%%%%%%%%%%%%%%%%%%%%%%%%%%%%%%%%%%%%%%
\SetKwIF{Recv}{}{}{upon receiving}{do}{}{}{}
\SetKwBlock{Init}{initially do}{}
\SetKwProg{Function}{Function}{:}{}

% https://github.com/chrmatt/algpseudocodex/issues/3
\algnewcommand\algorithmicswitch{\textbf{switch}}%
\algnewcommand\algorithmiccase{\textbf{case}}
\algnewcommand\algorithmicof{\textbf{of}}
\algnewcommand\algorithmicotherwise{\texttt{otherwise} $\Rightarrow$}

\makeatletter
\algdef{SE}[SWITCH]{Switch}{EndSwitch}[1]{\algpx@startIndent\algpx@startCodeCommand\algorithmicswitch\ #1\ \algorithmicdo}{\algpx@endIndent\algpx@startCodeCommand\algorithmicend\ \algorithmicswitch}%
\algdef{SE}[CASE]{Case}{EndCase}[1]{\algpx@startIndent\algpx@startCodeCommand\algorithmiccase\ #1}{\algpx@endIndent\algpx@startCodeCommand\algorithmicend\ \algorithmiccase}%
\algdef{SE}[CASEOF]{CaseOf}{EndCaseOf}[1]{\algpx@startIndent\algpx@startCodeCommand\algorithmiccase\ #1 \algorithmicof}{\algpx@endIndent\algpx@startCodeCommand\algorithmicend\ \algorithmiccase}
\algdef{SE}[OTHERWISE]{Otherwise}{EndOtherwise}[0]{\algpx@startIndent\algpx@startCodeCommand\algorithmicotherwise}{\algpx@endIndent\algpx@startCodeCommand\algorithmicend\ \algorithmicotherwise}
\ifbool{algpx@noEnd}{%
  \algtext*{EndSwitch}%
  \algtext*{EndCase}%
  \algtext*{EndCaseOf}
  \algtext*{EndOtherwise}
  %
  % end indent line after (not before), to get correct y position for multiline text in last command
  \apptocmd{\EndSwitch}{\algpx@endIndent}{}{}%
  \apptocmd{\EndCase}{\algpx@endIndent}{}{}%
  \apptocmd{\EndCaseOf}{\algpx@endIndent}{}{}
  \apptocmd{\EndOtherwise}{\algpx@endIndent}{}{}
}{}%

\pretocmd{\Switch}{\algpx@endCodeCommand}{}{}
\pretocmd{\Case}{\algpx@endCodeCommand}{}{}
\pretocmd{\CaseOf}{\algpx@endCodeCommand}{}{}
\pretocmd{\Otherwise}{\algpx@endCodeCommand}{}{}

% for end commands that may not be printed, tell endCodeCommand whether we are using noEnd
\ifbool{algpx@noEnd}{%
  \pretocmd{\EndSwitch}{\algpx@endCodeCommand[1]}{}{}%
  \pretocmd{\EndCase}{\algpx@endCodeCommand[1]}{}{}
  \pretocmd{\EndCaseOf}{\algpx@endCodeCommand[1]}{}{}%
  \pretocmd{\EndOtherwise}{\algpx@endCodeCommand[1]}{}{}
}{%
  \pretocmd{\EndSwitch}{\algpx@endCodeCommand[0]}{}{}%
  \pretocmd{\EndCase}{\algpx@endCodeCommand[0]}{}{}%
  \pretocmd{\EndCaseOf}{\algpx@endCodeCommand[0]}{}{}
  \pretocmd{\EndOtherwise}{\algpx@endCodeCommand[0]}{}{}
}%
\makeatother
% % For algpseudocode
% \algnewcommand\algorithmicswitch{\textbf{switch}}
% \algnewcommand\algorithmiccase{\textbf{case}}
% \algnewcommand\algorithmiccaseof{\textbf{case}}
% \algnewcommand\algorithmicof{\textbf{of}}
% % New "environments"
% \algdef{SE}[SWITCH]{Switch}{EndSwitch}[1]{\algorithmicswitch\ #1\ \algorithmicdo}{\algorithmicend\ \algorithmicswitch}%
% \algdef{SE}[CASE]{Case}{EndCase}[1]{\algorithmiccase\ #1}{\algorithmicend\ \algorithmiccase}%
% \algtext*{EndSwitch}%
% \algtext*{EndCase}
% \algdef{SE}[CASEOF]{CaseOf}{EndCaseOf}[1]{\algorithmiccaseof\ #1 \algorithmicof}{\algorithmicend\ \algorithmiccaseof}
% \algtext*{EndCaseOf}



%\pdfcompresslevel0

% quoting from
% https://tex.stackexchange.com/questions/391726/the-quotation-environment
\NewDocumentCommand{\bywhom}{m}{% the Bourbaki trick
  {\nobreak\hfill\penalty50\hskip1em\null\nobreak
   \hfill\mbox{\normalfont(#1)}%
   \parfillskip=0pt \finalhyphendemerits=0 \par}%
}

\NewDocumentEnvironment{pquotation}{m}
  {\begin{quoting}[
     indentfirst=true,
     leftmargin=\parindent,
     rightmargin=\parindent]\itshape}
  {\bywhom{#1}\end{quoting}}

\indexsetup{othercode=\small}
\makeindex[columns=2,options={-s /media/wu/file/stuuudy/notes/index_style.ist},intoc]
\makeatletter
\def\@idxitem{\par\hangindent 0pt}
\makeatother


% \newcounter{dummy} \numberwithin{dummy}{section}
\newtheorem{dummy}{dummy}[section]
\theoremstyle{definition}
\newtheorem{definition}[dummy]{Definition}
\theoremstyle{plain}
\newtheorem{corollary}[dummy]{Corollary}
\newtheorem{lemma}[dummy]{Lemma}
\newtheorem{proposition}[dummy]{Proposition}
\newtheorem{theorem}[dummy]{Theorem}
\newtheorem{notation}[dummy]{Notation}
\newtheorem{conjecture}[dummy]{Conjecture}
\newtheorem{fact}[dummy]{Fact}
\newtheorem{warning}[dummy]{Warning}
\theoremstyle{definition}
\newtheorem{examplle}{Example}[section]
\theoremstyle{remark}
\newtheorem*{remark}{Remark}
\newtheorem{exercise}{Exercise}[subsection]
\newtheorem{problem}{Problem}[subsection]
\newtheorem{observation}{Observation}[section]
\newenvironment{claim}[1]{\par\noindent\textbf{Claim:}\space#1}{}

\makeatletter
\DeclareFontFamily{U}{tipa}{}
\DeclareFontShape{U}{tipa}{m}{n}{<->tipa10}{}
\newcommand{\arc@char}{{\usefont{U}{tipa}{m}{n}\symbol{62}}}%

\newcommand{\arc}[1]{\mathpalette\arc@arc{#1}}

\newcommand{\arc@arc}[2]{%
  \sbox0{$\m@th#1#2$}%
  \vbox{
    \hbox{\resizebox{\wd0}{\height}{\arc@char}}
    \nointerlineskip
    \box0
  }%
}
\makeatother

\setcounter{MaxMatrixCols}{20}
%%%%%%% ABS
\DeclarePairedDelimiter\abss{\lvert}{\rvert}%
\DeclarePairedDelimiter\normm{\lVert}{\rVert}%

% Swap the definition of \abs* and \norm*, so that \abs
% and \norm resizes the size of the brackets, and the
% starred version does not.
\makeatletter
\let\oldabs\abss
%\def\abs{\@ifstar{\oldabs}{\oldabs*}}
\newcommand{\abs}{\@ifstar{\oldabs}{\oldabs*}}
\newcommand{\norm}[1]{\left\lVert#1\right\rVert}
%\let\oldnorm\normm
%\def\norm{\@ifstar{\oldnorm}{\oldnorm*}}
%\renewcommand{norm}{\@ifstar{\oldnorm}{\oldnorm*}}
\makeatother

% \stackMath
% \newcommand\what[1]{%
% \savestack{\tmpbox}{\stretchto{%
%   \scaleto{%
%     \scalerel*[\widthof{\ensuremath{#1}}]{\kern-.6pt\bigwedge\kern-.6pt}%
%     {\rule[-\textheight/2]{1ex}{\textheight}}%WIDTH-LIMITED BIG WEDGE
%   }{\textheight}%
% }{0.5ex}}%
% \stackon[1pt]{#1}{\tmpbox}%
% }

% \newcommand\what[1]{\ThisStyle{%
%     \setbox0=\hbox{$\SavedStyle#1$}%
%     \stackengine{-1.0\ht0+.5pt}{$\SavedStyle#1$}{%
%       \stretchto{\scaleto{\SavedStyle\mkern.15mu\char'136}{2.6\wd0}}{1.4\ht0}%
%     }{O}{c}{F}{T}{S}%
%   }
% }

% \newcommand\wtilde[1]{\ThisStyle{%
%     \setbox0=\hbox{$\SavedStyle#1$}%
%     \stackengine{-.1\LMpt}{$\SavedStyle#1$}{%
%       \stretchto{\scaleto{\SavedStyle\mkern.2mu\AC}{.5150\wd0}}{.6\ht0}%
%     }{O}{c}{F}{T}{S}%
%   }
% }

% \newcommand\wbar[1]{\ThisStyle{%
%     \setbox0=\hbox{$\SavedStyle#1$}%
%     \stackengine{.5pt+\LMpt}{$\SavedStyle#1$}{%
%       \rule{\wd0}{\dimexpr.3\LMpt+.3pt}%
%     }{O}{c}{F}{T}{S}%
%   }
% }

\newcommand{\bl}[1] {\boldsymbol{#1}}
\newcommand{\Wt}[1] {\stackrel{\sim}{\smash{#1}\rule{0pt}{1.1ex}}}
\newcommand{\wt}[1] {\widetilde{#1}}
\newcommand{\tf}[1] {\textbf{#1}}

\newcommand{\wu}[1]{{\color{red} #1}}

%For boxed texts in align, use Aboxed{}
%otherwise use boxed{}

\DeclareMathSymbol{\widehatsym}{\mathord}{largesymbols}{"62}
\newcommand\lowerwidehatsym{%
  \text{\smash{\raisebox{-1.3ex}{%
    $\widehatsym$}}}}
\newcommand\fixwidehat[1]{%
  \mathchoice
    {\accentset{\displaystyle\lowerwidehatsym}{#1}}
    {\accentset{\textstyle\lowerwidehatsym}{#1}}
    {\accentset{\scriptstyle\lowerwidehatsym}{#1}}
    {\accentset{\scriptscriptstyle\lowerwidehatsym}{#1}}
  }


\newcommand{\cupdot}{\mathbin{\dot{\cup}}}
\newcommand{\bigcupdot}{\mathop{\dot{\bigcup}}}

\usepackage{graphicx}

\usepackage[toc,page]{appendix}

% text on arrow for xRightarrow
\makeatletter
%\newcommand{\xRightarrow}[2][]{\ext@arrow 0359\Rightarrowfill@{#1}{#2}}
\makeatother

% Arbitrary long arrow
\newcommand{\Rarrow}[1]{%
\parbox{#1}{\tikz{\draw[->](0,0)--(#1,0);}}
}

\newcommand{\LRarrow}[1]{%
\parbox{#1}{\tikz{\draw[<->](0,0)--(#1,0);}}
}


\makeatletter
\providecommand*{\rmodels}{%
  \mathrel{%
    \mathpalette\@rmodels\models
  }%
}
\newcommand*{\@rmodels}[2]{%
  \reflectbox{$\m@th#1#2$}%
}
\makeatother

% Roman numerals
\makeatletter
\newcommand*{\rom}[1]{\expandafter\@slowromancap\romannumeral #1@}
\makeatother
% \\def \\b\([a-zA-Z]\) {\\boldsymbol{[a-zA-z]}}
% \\DeclareMathOperator{\\b\1}{\\textbf{\1}}

\DeclareMathOperator*{\argmin}{arg\,min}
\DeclareMathOperator*{\argmax}{arg\,max}

\DeclareMathOperator{\bone}{\textbf{1}}
\DeclareMathOperator{\bx}{\textbf{x}}
\DeclareMathOperator{\bz}{\textbf{z}}
\DeclareMathOperator{\bff}{\textbf{f}}
\DeclareMathOperator{\ba}{\textbf{a}}
\DeclareMathOperator{\bk}{\textbf{k}}
\DeclareMathOperator{\bs}{\textbf{s}}
\DeclareMathOperator{\bh}{\textbf{h}}
\DeclareMathOperator{\bc}{\textbf{c}}
\DeclareMathOperator{\br}{\textbf{r}}
\DeclareMathOperator{\bi}{\textbf{i}}
\DeclareMathOperator{\bj}{\textbf{j}}
\DeclareMathOperator{\bn}{\textbf{n}}
\DeclareMathOperator{\be}{\textbf{e}}
\DeclareMathOperator{\bo}{\textbf{o}}
\DeclareMathOperator{\bU}{\textbf{U}}
\DeclareMathOperator{\bL}{\textbf{L}}
\DeclareMathOperator{\bV}{\textbf{V}}
\def \bzero {\mathbf{0}}
\def \bbone {\mathbb{1}}
\def \btwo {\mathbf{2}}
\DeclareMathOperator{\bv}{\textbf{v}}
\DeclareMathOperator{\bp}{\textbf{p}}
\DeclareMathOperator{\bI}{\textbf{I}}
\def \dbI {\dot{\bI}}
\DeclareMathOperator{\bM}{\textbf{M}}
\DeclareMathOperator{\bN}{\textbf{N}}
\DeclareMathOperator{\bK}{\textbf{K}}
\DeclareMathOperator{\bt}{\textbf{t}}
\DeclareMathOperator{\bb}{\textbf{b}}
\DeclareMathOperator{\bA}{\textbf{A}}
\DeclareMathOperator{\bX}{\textbf{X}}
\DeclareMathOperator{\bu}{\textbf{u}}
\DeclareMathOperator{\bS}{\textbf{S}}
\DeclareMathOperator{\bZ}{\textbf{Z}}
\DeclareMathOperator{\bJ}{\textbf{J}}
\DeclareMathOperator{\by}{\textbf{y}}
\DeclareMathOperator{\bw}{\textbf{w}}
\DeclareMathOperator{\bT}{\textbf{T}}
\DeclareMathOperator{\bF}{\textbf{F}}
\DeclareMathOperator{\bmm}{\textbf{m}}
\DeclareMathOperator{\bW}{\textbf{W}}
\DeclareMathOperator{\bR}{\textbf{R}}
\DeclareMathOperator{\bC}{\textbf{C}}
\DeclareMathOperator{\bD}{\textbf{D}}
\DeclareMathOperator{\bE}{\textbf{E}}
\DeclareMathOperator{\bQ}{\textbf{Q}}
\DeclareMathOperator{\bP}{\textbf{P}}
\DeclareMathOperator{\bY}{\textbf{Y}}
\DeclareMathOperator{\bH}{\textbf{H}}
\DeclareMathOperator{\bB}{\textbf{B}}
\DeclareMathOperator{\bG}{\textbf{G}}
\def \blambda {\symbf{\lambda}}
\def \boldeta {\symbf{\eta}}
\def \balpha {\symbf{\alpha}}
\def \btau {\symbf{\tau}}
\def \bbeta {\symbf{\beta}}
\def \bgamma {\symbf{\gamma}}
\def \bxi {\symbf{\xi}}
\def \bLambda {\symbf{\Lambda}}
\def \bGamma {\symbf{\Gamma}}

\newcommand{\bto}{{\boldsymbol{\to}}}
\newcommand{\Ra}{\Rightarrow}
\newcommand{\xrsa}[1]{\overset{#1}{\rightsquigarrow}}
\newcommand{\xlsa}[1]{\overset{#1}{\leftsquigarrow}}
\newcommand\und[1]{\underline{#1}}
\newcommand\ove[1]{\overline{#1}}
%\def \concat {\verb|^|}
\def \bPhi {\mbfPhi}
\def \btheta {\mbftheta}
\def \bTheta {\mbfTheta}
\def \bmu {\mbfmu}
\def \bphi {\mbfphi}
\def \bSigma {\mbfSigma}
\def \la {\langle}
\def \ra {\rangle}

\def \caln {\mathcal{N}}
\def \dissum {\displaystyle\Sigma}
\def \dispro {\displaystyle\prod}

\def \caret {\verb!^!}

\def \A {\mathbb{A}}
\def \B {\mathbb{B}}
\def \C {\mathbb{C}}
\def \D {\mathbb{D}}
\def \E {\mathbb{E}}
\def \F {\mathbb{F}}
\def \G {\mathbb{G}}
\def \H {\mathbb{H}}
\def \I {\mathbb{I}}
\def \J {\mathbb{J}}
\def \K {\mathbb{K}}
\def \L {\mathbb{L}}
\def \M {\mathbb{M}}
\def \N {\mathbb{N}}
\def \O {\mathbb{O}}
\def \P {\mathbb{P}}
\def \Q {\mathbb{Q}}
\def \R {\mathbb{R}}
\def \S {\mathbb{S}}
\def \T {\mathbb{T}}
\def \U {\mathbb{U}}
\def \V {\mathbb{V}}
\def \W {\mathbb{W}}
\def \X {\mathbb{X}}
\def \Y {\mathbb{Y}}
\def \Z {\mathbb{Z}}

\def \cala {\mathcal{A}}
\def \cale {\mathcal{E}}
\def \calb {\mathcal{B}}
\def \calq {\mathcal{Q}}
\def \calp {\mathcal{P}}
\def \cals {\mathcal{S}}
\def \calx {\mathcal{X}}
\def \caly {\mathcal{Y}}
\def \calg {\mathcal{G}}
\def \cald {\mathcal{D}}
\def \caln {\mathcal{N}}
\def \calr {\mathcal{R}}
\def \calt {\mathcal{T}}
\def \calm {\mathcal{M}}
\def \calw {\mathcal{W}}
\def \calc {\mathcal{C}}
\def \calv {\mathcal{V}}
\def \calf {\mathcal{F}}
\def \calk {\mathcal{K}}
\def \call {\mathcal{L}}
\def \calu {\mathcal{U}}
\def \calo {\mathcal{O}}
\def \calh {\mathcal{H}}
\def \cali {\mathcal{I}}
\def \calj {\mathcal{J}}

\def \bcup {\bigcup}

% set theory

\def \zfcc {\textbf{ZFC}^-}
\def \BGC {\textbf{BGC}}
\def \BG {\textbf{BG}}
\def \ac  {\textbf{AC}}
\def \gl  {\textbf{L }}
\def \gll {\textbf{L}}
\newcommand{\zfm}{$\textbf{ZF}^-$}

\def \ZFm {\text{ZF}^-}
\def \ZFCm {\text{ZFC}^-}
\DeclareMathOperator{\WF}{WF}
\DeclareMathOperator{\On}{On}
\def \on {\textbf{On }}
\def \cm {\textbf{M }}
\def \cn {\textbf{N }}
\def \cv {\textbf{V }}
\def \zc {\textbf{ZC }}
\def \zcm {\textbf{ZC}}
\def \zff {\textbf{ZF}}
\def \wfm {\textbf{WF}}
\def \onm {\textbf{On}}
\def \cmm {\textbf{M}}
\def \cnm {\textbf{N}}
\def \cvm {\textbf{V}}

\renewcommand{\restriction}{\mathord{\upharpoonright}}
%% another restriction
\newcommand\restr[2]{{% we make the whole thing an ordinary symbol
  \left.\kern-\nulldelimiterspace % automatically resize the bar with \right
  #1 % the function
  \vphantom{\big|} % pretend it's a little taller at normal size
  \right|_{#2} % this is the delimiter
  }}

\def \pred {\text{pred}}

\def \rank {\text{rank}}
\def \Con {\text{Con}}
\def \deff {\text{Def}}


\def \uin {\underline{\in}}
\def \oin {\overline{\in}}
\def \uR {\underline{R}}
\def \oR {\overline{R}}
\def \uP {\underline{P}}
\def \oP {\overline{P}}

\def \dsum {\displaystyle\sum}

\def \Ra {\Rightarrow}

\def \e {\enspace}

\def \sgn {\operatorname{sgn}}
\def \gen {\operatorname{gen}}
\def \Hom {\operatorname{Hom}}
\def \hom {\operatorname{hom}}
\def \Sub {\operatorname{Sub}}

\def \supp {\operatorname{supp}}

\def \epiarrow {\twoheadarrow}
\def \monoarrow {\rightarrowtail}
\def \rrarrow {\rightrightarrows}

% \def \minus {\text{-}}
% \newcommand{\minus}{\scalebox{0.75}[1.0]{$-$}}
% \DeclareUnicodeCharacter{002D}{\minus}


\def \tril {\triangleleft}

\def \ISigma {\text{I}\Sigma}
\def \IDelta {\text{I}\Delta}
\def \IPi {\text{I}\Pi}
\def \ACF {\textsf{ACF}}
\def \pCF {\textit{p}\text{CF}}
\def \ACVF {\textsf{ACVF}}
\def \HLR {\textsf{HLR}}
\def \OAG {\textsf{OAG}}
\def \RCF {\textsf{RCF}}
\DeclareMathOperator{\GL}{GL}
\DeclareMathOperator{\PGL}{PGL}
\DeclareMathOperator{\SL}{SL}
\DeclareMathOperator{\Inv}{Inv}
\DeclareMathOperator{\res}{res}
\DeclareMathOperator{\Sym}{Sym}
%\DeclareMathOperator{\char}{char}
\def \equal {=}

\def \degree {\text{degree}}
\def \app {\text{App}}
\def \FV {\text{FV}}
\def \conv {\text{conv}}
\def \cont {\text{cont}}
\DeclareMathOperator{\cl}{\text{cl}}
\DeclareMathOperator{\trcl}{\text{trcl}}
\DeclareMathOperator{\sg}{sg}
\DeclareMathOperator{\trdeg}{trdeg}
\def \Ord {\text{Ord}}

\DeclareMathOperator{\cf}{cf}
\DeclareMathOperator{\zfc}{ZFC}

%\DeclareMathOperator{\Th}{Th}
%\def \th {\text{Th}}
% \newcommand{\th}{\text{Th}}
\DeclareMathOperator{\type}{type}
\DeclareMathOperator{\zf}{\textbf{ZF}}
\def \fa {\mathfrak{a}}
\def \fb {\mathfrak{b}}
\def \fc {\mathfrak{c}}
\def \fd {\mathfrak{d}}
\def \fe {\mathfrak{e}}
\def \ff {\mathfrak{f}}
\def \fg {\mathfrak{g}}
\def \fh {\mathfrak{h}}
%\def \fi {\mathfrak{i}}
\def \fj {\mathfrak{j}}
\def \fk {\mathfrak{k}}
\def \fl {\mathfrak{l}}
\def \fm {\mathfrak{m}}
\def \fn {\mathfrak{n}}
\def \fo {\mathfrak{o}}
\def \fp {\mathfrak{p}}
\def \fq {\mathfrak{q}}
\def \fr {\mathfrak{r}}
\def \fs {\mathfrak{s}}
\def \ft {\mathfrak{t}}
\def \fu {\mathfrak{u}}
\def \fv {\mathfrak{v}}
\def \fw {\mathfrak{w}}
\def \fx {\mathfrak{x}}
\def \fy {\mathfrak{y}}
\def \fz {\mathfrak{z}}
\def \fA {\mathfrak{A}}
\def \fB {\mathfrak{B}}
\def \fC {\mathfrak{C}}
\def \fD {\mathfrak{D}}
\def \fE {\mathfrak{E}}
\def \fF {\mathfrak{F}}
\def \fG {\mathfrak{G}}
\def \fH {\mathfrak{H}}
\def \fI {\mathfrak{I}}
\def \fJ {\mathfrak{J}}
\def \fK {\mathfrak{K}}
\def \fL {\mathfrak{L}}
\def \fM {\mathfrak{M}}
\def \fN {\mathfrak{N}}
\def \fO {\mathfrak{O}}
\def \fP {\mathfrak{P}}
\def \fQ {\mathfrak{Q}}
\def \fR {\mathfrak{R}}
\def \fS {\mathfrak{S}}
\def \fT {\mathfrak{T}}
\def \fU {\mathfrak{U}}
\def \fV {\mathfrak{V}}
\def \fW {\mathfrak{W}}
\def \fX {\mathfrak{X}}
\def \fY {\mathfrak{Y}}
\def \fZ {\mathfrak{Z}}

\def \sfA {\textsf{A}}
\def \sfB {\textsf{B}}
\def \sfC {\textsf{C}}
\def \sfD {\textsf{D}}
\def \sfE {\textsf{E}}
\def \sfF {\textsf{F}}
\def \sfG {\textsf{G}}
\def \sfH {\textsf{H}}
\def \sfI {\textsf{I}}
\def \sfJ {\textsf{J}}
\def \sfK {\textsf{K}}
\def \sfL {\textsf{L}}
\def \sfM {\textsf{M}}
\def \sfN {\textsf{N}}
\def \sfO {\textsf{O}}
\def \sfP {\textsf{P}}
\def \sfQ {\textsf{Q}}
\def \sfR {\textsf{R}}
\def \sfS {\textsf{S}}
\def \sfT {\textsf{T}}
\def \sfU {\textsf{U}}
\def \sfV {\textsf{V}}
\def \sfW {\textsf{W}}
\def \sfX {\textsf{X}}
\def \sfY {\textsf{Y}}
\def \sfZ {\textsf{Z}}
\def \sfa {\textsf{a}}
\def \sfb {\textsf{b}}
\def \sfc {\textsf{c}}
\def \sfd {\textsf{d}}
\def \sfe {\textsf{e}}
\def \sff {\textsf{f}}
\def \sfg {\textsf{g}}
\def \sfh {\textsf{h}}
\def \sfi {\textsf{i}}
\def \sfj {\textsf{j}}
\def \sfk {\textsf{k}}
\def \sfl {\textsf{l}}
\def \sfm {\textsf{m}}
\def \sfn {\textsf{n}}
\def \sfo {\textsf{o}}
\def \sfp {\textsf{p}}
\def \sfq {\textsf{q}}
\def \sfr {\textsf{r}}
\def \sfs {\textsf{s}}
\def \sft {\textsf{t}}
\def \sfu {\textsf{u}}
\def \sfv {\textsf{v}}
\def \sfw {\textsf{w}}
\def \sfx {\textsf{x}}
\def \sfy {\textsf{y}}
\def \sfz {\textsf{z}}

\def \ttA {\texttt{A}}
\def \ttB {\texttt{B}}
\def \ttC {\texttt{C}}
\def \ttD {\texttt{D}}
\def \ttE {\texttt{E}}
\def \ttF {\texttt{F}}
\def \ttG {\texttt{G}}
\def \ttH {\texttt{H}}
\def \ttI {\texttt{I}}
\def \ttJ {\texttt{J}}
\def \ttK {\texttt{K}}
\def \ttL {\texttt{L}}
\def \ttM {\texttt{M}}
\def \ttN {\texttt{N}}
\def \ttO {\texttt{O}}
\def \ttP {\texttt{P}}
\def \ttQ {\texttt{Q}}
\def \ttR {\texttt{R}}
\def \ttS {\texttt{S}}
\def \ttT {\texttt{T}}
\def \ttU {\texttt{U}}
\def \ttV {\texttt{V}}
\def \ttW {\texttt{W}}
\def \ttX {\texttt{X}}
\def \ttY {\texttt{Y}}
\def \ttZ {\texttt{Z}}
\def \tta {\texttt{a}}
\def \ttb {\texttt{b}}
\def \ttc {\texttt{c}}
\def \ttd {\texttt{d}}
\def \tte {\texttt{e}}
\def \ttf {\texttt{f}}
\def \ttg {\texttt{g}}
\def \tth {\texttt{h}}
\def \tti {\texttt{i}}
\def \ttj {\texttt{j}}
\def \ttk {\texttt{k}}
\def \ttl {\texttt{l}}
\def \ttm {\texttt{m}}
\def \ttn {\texttt{n}}
\def \tto {\texttt{o}}
\def \ttp {\texttt{p}}
\def \ttq {\texttt{q}}
\def \ttr {\texttt{r}}
\def \tts {\texttt{s}}
\def \ttt {\texttt{t}}
\def \ttu {\texttt{u}}
\def \ttv {\texttt{v}}
\def \ttw {\texttt{w}}
\def \ttx {\texttt{x}}
\def \tty {\texttt{y}}
\def \ttz {\texttt{z}}

\def \bara {\bbar{a}}
\def \barb {\bbar{b}}
\def \barc {\bbar{c}}
\def \bard {\bbar{d}}
\def \bare {\bbar{e}}
\def \barf {\bbar{f}}
\def \barg {\bbar{g}}
\def \barh {\bbar{h}}
\def \bari {\bbar{i}}
\def \barj {\bbar{j}}
\def \bark {\bbar{k}}
\def \barl {\bbar{l}}
\def \barm {\bbar{m}}
\def \barn {\bbar{n}}
\def \baro {\bbar{o}}
\def \barp {\bbar{p}}
\def \barq {\bbar{q}}
\def \barr {\bbar{r}}
\def \bars {\bbar{s}}
\def \bart {\bbar{t}}
\def \baru {\bbar{u}}
\def \barv {\bbar{v}}
\def \barw {\bbar{w}}
\def \barx {\bbar{x}}
\def \bary {\bbar{y}}
\def \barz {\bbar{z}}
\def \barA {\bbar{A}}
\def \barB {\bbar{B}}
\def \barC {\bbar{C}}
\def \barD {\bbar{D}}
\def \barE {\bbar{E}}
\def \barF {\bbar{F}}
\def \barG {\bbar{G}}
\def \barH {\bbar{H}}
\def \barI {\bbar{I}}
\def \barJ {\bbar{J}}
\def \barK {\bbar{K}}
\def \barL {\bbar{L}}
\def \barM {\bbar{M}}
\def \barN {\bbar{N}}
\def \barO {\bbar{O}}
\def \barP {\bbar{P}}
\def \barQ {\bbar{Q}}
\def \barR {\bbar{R}}
\def \barS {\bbar{S}}
\def \barT {\bbar{T}}
\def \barU {\bbar{U}}
\def \barVV {\bbar{V}}
\def \barW {\bbar{W}}
\def \barX {\bbar{X}}
\def \barY {\bbar{Y}}
\def \barZ {\bbar{Z}}

\def \baralpha {\bbar{\alpha}}
\def \bartau {\bbar{\tau}}
\def \barsigma {\bbar{\sigma}}
\def \barzeta {\bbar{\zeta}}

\def \hata {\hat{a}}
\def \hatb {\hat{b}}
\def \hatc {\hat{c}}
\def \hatd {\hat{d}}
\def \hate {\hat{e}}
\def \hatf {\hat{f}}
\def \hatg {\hat{g}}
\def \hath {\hat{h}}
\def \hati {\hat{i}}
\def \hatj {\hat{j}}
\def \hatk {\hat{k}}
\def \hatl {\hat{l}}
\def \hatm {\hat{m}}
\def \hatn {\hat{n}}
\def \hato {\hat{o}}
\def \hatp {\hat{p}}
\def \hatq {\hat{q}}
\def \hatr {\hat{r}}
\def \hats {\hat{s}}
\def \hatt {\hat{t}}
\def \hatu {\hat{u}}
\def \hatv {\hat{v}}
\def \hatw {\hat{w}}
\def \hatx {\hat{x}}
\def \haty {\hat{y}}
\def \hatz {\hat{z}}
\def \hatA {\hat{A}}
\def \hatB {\hat{B}}
\def \hatC {\hat{C}}
\def \hatD {\hat{D}}
\def \hatE {\hat{E}}
\def \hatF {\hat{F}}
\def \hatG {\hat{G}}
\def \hatH {\hat{H}}
\def \hatI {\hat{I}}
\def \hatJ {\hat{J}}
\def \hatK {\hat{K}}
\def \hatL {\hat{L}}
\def \hatM {\hat{M}}
\def \hatN {\hat{N}}
\def \hatO {\hat{O}}
\def \hatP {\hat{P}}
\def \hatQ {\hat{Q}}
\def \hatR {\hat{R}}
\def \hatS {\hat{S}}
\def \hatT {\hat{T}}
\def \hatU {\hat{U}}
\def \hatVV {\hat{V}}
\def \hatW {\hat{W}}
\def \hatX {\hat{X}}
\def \hatY {\hat{Y}}
\def \hatZ {\hat{Z}}

\def \hatphi {\hat{\phi}}

\def \barfM {\bbar{\fM}}
\def \barfN {\bbar{\fN}}

\def \tila {\tilde{a}}
\def \tilb {\tilde{b}}
\def \tilc {\tilde{c}}
\def \tild {\tilde{d}}
\def \tile {\tilde{e}}
\def \tilf {\tilde{f}}
\def \tilg {\tilde{g}}
\def \tilh {\tilde{h}}
\def \tili {\tilde{i}}
\def \tilj {\tilde{j}}
\def \tilk {\tilde{k}}
\def \till {\tilde{l}}
\def \tilm {\tilde{m}}
\def \tiln {\tilde{n}}
\def \tilo {\tilde{o}}
\def \tilp {\tilde{p}}
\def \tilq {\tilde{q}}
\def \tilr {\tilde{r}}
\def \tils {\tilde{s}}
\def \tilt {\tilde{t}}
\def \tilu {\tilde{u}}
\def \tilv {\tilde{v}}
\def \tilw {\tilde{w}}
\def \tilx {\tilde{x}}
\def \tily {\tilde{y}}
\def \tilz {\tilde{z}}
\def \tilA {\tilde{A}}
\def \tilB {\tilde{B}}
\def \tilC {\tilde{C}}
\def \tilD {\tilde{D}}
\def \tilE {\tilde{E}}
\def \tilF {\tilde{F}}
\def \tilG {\tilde{G}}
\def \tilH {\tilde{H}}
\def \tilI {\tilde{I}}
\def \tilJ {\tilde{J}}
\def \tilK {\tilde{K}}
\def \tilL {\tilde{L}}
\def \tilM {\tilde{M}}
\def \tilN {\tilde{N}}
\def \tilO {\tilde{O}}
\def \tilP {\tilde{P}}
\def \tilQ {\tilde{Q}}
\def \tilR {\tilde{R}}
\def \tilS {\tilde{S}}
\def \tilT {\tilde{T}}
\def \tilU {\tilde{U}}
\def \tilVV {\tilde{V}}
\def \tilW {\tilde{W}}
\def \tilX {\tilde{X}}
\def \tilY {\tilde{Y}}
\def \tilZ {\tilde{Z}}

\def \tilalpha {\tilde{\alpha}}
\def \tilPhi {\tilde{\Phi}}

\def \barnu {\bar{\nu}}
\def \barrho {\bar{\rho}}
%\DeclareMathOperator{\ker}{ker}
\DeclareMathOperator{\im}{im}

\DeclareMathOperator{\Inn}{Inn}
\DeclareMathOperator{\rel}{rel}
\def \dote {\stackrel{\cdot}=}
%\DeclareMathOperator{\AC}{\textbf{AC}}
\DeclareMathOperator{\cod}{cod}
\DeclareMathOperator{\dom}{dom}
\DeclareMathOperator{\card}{card}
\DeclareMathOperator{\ran}{ran}
\DeclareMathOperator{\textd}{d}
\DeclareMathOperator{\td}{d}
\DeclareMathOperator{\id}{id}
\DeclareMathOperator{\LT}{LT}
\DeclareMathOperator{\Mat}{Mat}
\DeclareMathOperator{\Eq}{Eq}
\DeclareMathOperator{\irr}{irr}
\DeclareMathOperator{\Fr}{Fr}
\DeclareMathOperator{\Gal}{Gal}
\DeclareMathOperator{\lcm}{lcm}
\DeclareMathOperator{\alg}{\text{alg}}
\DeclareMathOperator{\Th}{Th}
%\DeclareMathOperator{\deg}{deg}


% \varprod
\DeclareSymbolFont{largesymbolsA}{U}{txexa}{m}{n}
\DeclareMathSymbol{\varprod}{\mathop}{largesymbolsA}{16}
% \DeclareMathSymbol{\tonm}{\boldsymbol{\to}\textbf{Nm}}
\def \tonm {\bto\textbf{Nm}}
\def \tohm {\bto\textbf{Hm}}

% Category theory
\DeclareMathOperator{\ob}{ob}
\DeclareMathOperator{\Ab}{\textbf{Ab}}
\DeclareMathOperator{\Alg}{\textbf{Alg}}
\DeclareMathOperator{\Rng}{\textbf{Rng}}
\DeclareMathOperator{\Sets}{\textbf{Sets}}
\DeclareMathOperator{\Set}{\textbf{Set}}
\DeclareMathOperator{\Grp}{\textbf{Grp}}
\DeclareMathOperator{\Met}{\textbf{Met}}
\DeclareMathOperator{\BA}{\textbf{BA}}
\DeclareMathOperator{\Mon}{\textbf{Mon}}
\DeclareMathOperator{\Top}{\textbf{Top}}
\DeclareMathOperator{\hTop}{\textbf{hTop}}
\DeclareMathOperator{\HTop}{\textbf{HTop}}
\DeclareMathOperator{\Aut}{\text{Aut}}
\DeclareMathOperator{\RMod}{R-\textbf{Mod}}
\DeclareMathOperator{\RAlg}{R-\textbf{Alg}}
\DeclareMathOperator{\LF}{LF}
\DeclareMathOperator{\op}{op}
\DeclareMathOperator{\Rings}{\textbf{Rings}}
\DeclareMathOperator{\Ring}{\textbf{Ring}}
\DeclareMathOperator{\Groups}{\textbf{Groups}}
\DeclareMathOperator{\Group}{\textbf{Group}}
\DeclareMathOperator{\ev}{ev}
% Algebraic Topology
\DeclareMathOperator{\obj}{obj}
\DeclareMathOperator{\Spec}{Spec}
\DeclareMathOperator{\spec}{spec}
% Model theory
\DeclareMathOperator*{\ind}{\raise0.2ex\hbox{\ooalign{\hidewidth$\vert$\hidewidth\cr\raise-0.9ex\hbox{$\smile$}}}}
\def\nind{\cancel{\ind}}
\DeclareMathOperator{\acl}{acl}
\DeclareMathOperator{\tspan}{span}
\DeclareMathOperator{\acleq}{acl^{\eq}}
\DeclareMathOperator{\Av}{Av}
\DeclareMathOperator{\ded}{ded}
\DeclareMathOperator{\EM}{EM}
\DeclareMathOperator{\dcl}{dcl}
\DeclareMathOperator{\Ext}{Ext}
\DeclareMathOperator{\eq}{eq}
\DeclareMathOperator{\ER}{ER}
\DeclareMathOperator{\tp}{tp}
\DeclareMathOperator{\stp}{stp}
\DeclareMathOperator{\qftp}{qftp}
\DeclareMathOperator{\Diag}{Diag}
\DeclareMathOperator{\MD}{MD}
\DeclareMathOperator{\MR}{MR}
\DeclareMathOperator{\RM}{RM}
\DeclareMathOperator{\el}{el}
\DeclareMathOperator{\depth}{depth}
\DeclareMathOperator{\ZFC}{ZFC}
\DeclareMathOperator{\GCH}{GCH}
\DeclareMathOperator{\Inf}{Inf}
\DeclareMathOperator{\Pow}{Pow}
\DeclareMathOperator{\ZF}{ZF}
\DeclareMathOperator{\CH}{CH}
\def \FO {\text{FO}}
\DeclareMathOperator{\fin}{fin}
\DeclareMathOperator{\qr}{qr}
\DeclareMathOperator{\Mod}{Mod}
\DeclareMathOperator{\Def}{Def}
\DeclareMathOperator{\TC}{TC}
\DeclareMathOperator{\KH}{KH}
\DeclareMathOperator{\Part}{Part}
\DeclareMathOperator{\Infset}{\textsf{Infset}}
\DeclareMathOperator{\DLO}{\textsf{DLO}}
\DeclareMathOperator{\PA}{\textsf{PA}}
\DeclareMathOperator{\DAG}{\textsf{DAG}}
\DeclareMathOperator{\ODAG}{\textsf{ODAG}}
\DeclareMathOperator{\sfMod}{\textsf{Mod}}
\DeclareMathOperator{\AbG}{\textsf{AbG}}
\DeclareMathOperator{\sfACF}{\textsf{ACF}}
\DeclareMathOperator{\DCF}{\textsf{DCF}}
% Computability Theorem
\DeclareMathOperator{\Tot}{Tot}
\DeclareMathOperator{\graph}{graph}
\DeclareMathOperator{\Fin}{Fin}
\DeclareMathOperator{\Cof}{Cof}
\DeclareMathOperator{\lh}{lh}
% Commutative Algebra
\DeclareMathOperator{\ord}{ord}
\DeclareMathOperator{\Idem}{Idem}
\DeclareMathOperator{\zdiv}{z.div}
\DeclareMathOperator{\Frac}{Frac}
\DeclareMathOperator{\rad}{rad}
\DeclareMathOperator{\nil}{nil}
\DeclareMathOperator{\Ann}{Ann}
\DeclareMathOperator{\End}{End}
\DeclareMathOperator{\coim}{coim}
\DeclareMathOperator{\coker}{coker}
\DeclareMathOperator{\Bil}{Bil}
\DeclareMathOperator{\Tril}{Tril}
\DeclareMathOperator{\tchar}{char}
\DeclareMathOperator{\tbd}{bd}

% Topology
\DeclareMathOperator{\diam}{diam}
\newcommand{\interior}[1]{%
  {\kern0pt#1}^{\mathrm{o}}%
}

\DeclareMathOperator*{\bigdoublewedge}{\bigwedge\mkern-15mu\bigwedge}
\DeclareMathOperator*{\bigdoublevee}{\bigvee\mkern-15mu\bigvee}

% \makeatletter
% \newcommand{\vect}[1]{%
%   \vbox{\m@th \ialign {##\crcr
%   \vectfill\crcr\noalign{\kern-\p@ \nointerlineskip}
%   $\hfil\displaystyle{#1}\hfil$\crcr}}}
% \def\vectfill{%
%   $\m@th\smash-\mkern-7mu%
%   \cleaders\hbox{$\mkern-2mu\smash-\mkern-2mu$}\hfill
%   \mkern-7mu\raisebox{-3.81pt}[\p@][\p@]{$\mathord\mathchar"017E$}$}

% \newcommand{\amsvect}{%
%   \mathpalette {\overarrow@\vectfill@}}
% \def\vectfill@{\arrowfill@\relbar\relbar{\raisebox{-3.81pt}[\p@][\p@]{$\mathord\mathchar"017E$}}}

% \newcommand{\amsvectb}{%
% \newcommand{\vect}{%
%   \mathpalette {\overarrow@\vectfillb@}}
% \newcommand{\vecbar}{%
%   \scalebox{0.8}{$\relbar$}}
% \def\vectfillb@{\arrowfill@\vecbar\vecbar{\raisebox{-4.35pt}[\p@][\p@]{$\mathord\mathchar"017E$}}}
% \makeatother
% \bigtimes

\DeclareFontFamily{U}{mathx}{\hyphenchar\font45}
\DeclareFontShape{U}{mathx}{m}{n}{
      <5> <6> <7> <8> <9> <10>
      <10.95> <12> <14.4> <17.28> <20.74> <24.88>
      mathx10
      }{}
\DeclareSymbolFont{mathx}{U}{mathx}{m}{n}
\DeclareMathSymbol{\bigtimes}{1}{mathx}{"91}
% \odiv
\DeclareFontFamily{U}{matha}{\hyphenchar\font45}
\DeclareFontShape{U}{matha}{m}{n}{
      <5> <6> <7> <8> <9> <10> gen * matha
      <10.95> matha10 <12> <14.4> <17.28> <20.74> <24.88> matha12
      }{}
\DeclareSymbolFont{matha}{U}{matha}{m}{n}
\DeclareMathSymbol{\odiv}         {2}{matha}{"63}


\newcommand\subsetsim{\mathrel{%
  \ooalign{\raise0.2ex\hbox{\scalebox{0.9}{$\subset$}}\cr\hidewidth\raise-0.85ex\hbox{\scalebox{0.9}{$\sim$}}\hidewidth\cr}}}
\newcommand\simsubset{\mathrel{%
  \ooalign{\raise-0.2ex\hbox{\scalebox{0.9}{$\subset$}}\cr\hidewidth\raise0.75ex\hbox{\scalebox{0.9}{$\sim$}}\hidewidth\cr}}}

\newcommand\simsubsetsim{\mathrel{%
  \ooalign{\raise0ex\hbox{\scalebox{0.8}{$\subset$}}\cr\hidewidth\raise1ex\hbox{\scalebox{0.75}{$\sim$}}\hidewidth\cr\raise-0.95ex\hbox{\scalebox{0.8}{$\sim$}}\cr\hidewidth}}}
\newcommand{\stcomp}[1]{{#1}^{\mathsf{c}}}

\setlength{\baselineskip}{0.5in}

\stackMath
\newcommand\yrightarrow[2][]{\mathrel{%
  \setbox2=\hbox{\stackon{\scriptstyle#1}{\scriptstyle#2}}%
  \stackunder[0pt]{%
    \xrightarrow{\makebox[\dimexpr\wd2\relax]{$\scriptstyle#2$}}%
  }{%
   \scriptstyle#1\,%
  }%
}}
\newcommand\yleftarrow[2][]{\mathrel{%
  \setbox2=\hbox{\stackon{\scriptstyle#1}{\scriptstyle#2}}%
  \stackunder[0pt]{%
    \xleftarrow{\makebox[\dimexpr\wd2\relax]{$\scriptstyle#2$}}%
  }{%
   \scriptstyle#1\,%
  }%
}}
\newcommand\yRightarrow[2][]{\mathrel{%
  \setbox2=\hbox{\stackon{\scriptstyle#1}{\scriptstyle#2}}%
  \stackunder[0pt]{%
    \xRightarrow{\makebox[\dimexpr\wd2\relax]{$\scriptstyle#2$}}%
  }{%
   \scriptstyle#1\,%
  }%
}}
\newcommand\yLeftarrow[2][]{\mathrel{%
  \setbox2=\hbox{\stackon{\scriptstyle#1}{\scriptstyle#2}}%
  \stackunder[0pt]{%
    \xLeftarrow{\makebox[\dimexpr\wd2\relax]{$\scriptstyle#2$}}%
  }{%
   \scriptstyle#1\,%
  }%
}}

\newcommand\altxrightarrow[2][0pt]{\mathrel{\ensurestackMath{\stackengine%
  {\dimexpr#1-7.5pt}{\xrightarrow{\phantom{#2}}}{\scriptstyle\!#2\,}%
  {O}{c}{F}{F}{S}}}}
\newcommand\altxleftarrow[2][0pt]{\mathrel{\ensurestackMath{\stackengine%
  {\dimexpr#1-7.5pt}{\xleftarrow{\phantom{#2}}}{\scriptstyle\!#2\,}%
  {O}{c}{F}{F}{S}}}}

\newenvironment{bsm}{% % short for 'bracketed small matrix'
  \left[ \begin{smallmatrix} }{%
  \end{smallmatrix} \right]}

\newenvironment{psm}{% % short for ' small matrix'
  \left( \begin{smallmatrix} }{%
  \end{smallmatrix} \right)}

\newcommand{\bbar}[1]{\mkern 1.5mu\overline{\mkern-1.5mu#1\mkern-1.5mu}\mkern 1.5mu}

\newcommand{\bigzero}{\mbox{\normalfont\Large\bfseries 0}}
\newcommand{\rvline}{\hspace*{-\arraycolsep}\vline\hspace*{-\arraycolsep}}

\font\zallman=Zallman at 40pt
\font\elzevier=Elzevier at 40pt

\newcommand\isoto{\stackrel{\textstyle\sim}{\smash{\longrightarrow}\rule{0pt}{0.4ex}}}
\newcommand\embto{\stackrel{\textstyle\prec}{\smash{\longrightarrow}\rule{0pt}{0.4ex}}}

% from http://www.actual.world/resources/tex/doc/TikZ.pdf

\tikzset{
modal/.style={>=stealth’,shorten >=1pt,shorten <=1pt,auto,node distance=1.5cm,
semithick},
world/.style={circle,draw,minimum size=0.5cm,fill=gray!15},
point/.style={circle,draw,inner sep=0.5mm,fill=black},
reflexive above/.style={->,loop,looseness=7,in=120,out=60},
reflexive below/.style={->,loop,looseness=7,in=240,out=300},
reflexive left/.style={->,loop,looseness=7,in=150,out=210},
reflexive right/.style={->,loop,looseness=7,in=30,out=330}
}


\makeatletter
\newcommand*{\doublerightarrow}[2]{\mathrel{
  \settowidth{\@tempdima}{$\scriptstyle#1$}
  \settowidth{\@tempdimb}{$\scriptstyle#2$}
  \ifdim\@tempdimb>\@tempdima \@tempdima=\@tempdimb\fi
  \mathop{\vcenter{
    \offinterlineskip\ialign{\hbox to\dimexpr\@tempdima+1em{##}\cr
    \rightarrowfill\cr\noalign{\kern.5ex}
    \rightarrowfill\cr}}}\limits^{\!#1}_{\!#2}}}
\newcommand*{\triplerightarrow}[1]{\mathrel{
  \settowidth{\@tempdima}{$\scriptstyle#1$}
  \mathop{\vcenter{
    \offinterlineskip\ialign{\hbox to\dimexpr\@tempdima+1em{##}\cr
    \rightarrowfill\cr\noalign{\kern.5ex}
    \rightarrowfill\cr\noalign{\kern.5ex}
    \rightarrowfill\cr}}}\limits^{\!#1}}}
\makeatother

% $A\doublerightarrow{a}{bcdefgh}B$

% $A\triplerightarrow{d_0,d_1,d_2}B$

\def \uhr {\upharpoonright}
\def \rhu {\rightharpoonup}
\def \uhl {\upharpoonleft}


\newcommand{\floor}[1]{\lfloor #1 \rfloor}
\newcommand{\ceil}[1]{\lceil #1 \rceil}
\newcommand{\lcorner}[1]{\llcorner #1 \lrcorner}
\newcommand{\llb}[1]{\llbracket #1 \rrbracket}
\newcommand{\ucorner}[1]{\ulcorner #1 \urcorner}
\newcommand{\emoji}[1]{{\DejaSans #1}}
\newcommand{\vprec}{\rotatebox[origin=c]{-90}{$\prec$}}

\newcommand{\nat}[6][large]{%
  \begin{tikzcd}[ampersand replacement = \&, column sep=#1]
    #2\ar[bend left=40,""{name=U}]{r}{#4}\ar[bend right=40,',""{name=D}]{r}{#5}\& #3
          \ar[shorten <=10pt,shorten >=10pt,Rightarrow,from=U,to=D]{d}{~#6}
    \end{tikzcd}
}


\providecommand\rightarrowRHD{\relbar\joinrel\mathrel\RHD}
\providecommand\rightarrowrhd{\relbar\joinrel\mathrel\rhd}
\providecommand\longrightarrowRHD{\relbar\joinrel\relbar\joinrel\mathrel\RHD}
\providecommand\longrightarrowrhd{\relbar\joinrel\relbar\joinrel\mathrel\rhd}
\def \lrarhd {\longrightarrowrhd}


\makeatletter
\providecommand*\xrightarrowRHD[2][]{\ext@arrow 0055{\arrowfill@\relbar\relbar\longrightarrowRHD}{#1}{#2}}
\providecommand*\xrightarrowrhd[2][]{\ext@arrow 0055{\arrowfill@\relbar\relbar\longrightarrowrhd}{#1}{#2}}
\makeatother

\newcommand{\metalambda}{%
  \mathop{%
    \rlap{$\lambda$}%
    \mkern3mu
    \raisebox{0ex}{$\lambda$}%
  }%
}

%% https://tex.stackexchange.com/questions/15119/draw-horizontal-line-left-and-right-of-some-text-a-single-line
\newcommand*\ruleline[1]{\par\noindent\raisebox{.8ex}{\makebox[\linewidth]{\hrulefill\hspace{1ex}\raisebox{-.8ex}{#1}\hspace{1ex}\hrulefill}}}

% https://www.dickimaw-books.com/latex/novices/html/newenv.html
\newenvironment{Block}[1]% environment name
{% begin code
  % https://tex.stackexchange.com/questions/19579/horizontal-line-spanning-the-entire-document-in-latex
  \noindent\textcolor[RGB]{128,128,128}{\rule{\linewidth}{1pt}}
  \par\noindent
  {\Large\textbf{#1}}%
  \bigskip\par\noindent\ignorespaces
}%
{% end code
  \par\noindent
  \textcolor[RGB]{128,128,128}{\rule{\linewidth}{1pt}}
  \ignorespacesafterend
}

\mathchardef\mhyphen="2D % Define a "math hyphen"

\def \QQ {\quad}
\def \QW {​\quad}

\setcounter{secnumdepth}{2}
\setcounter{tocdepth}{2}
\def \acl {\text{acl}}
\author{Katin Tent \& Martin Ziegler}
\date{\today}
\title{A Course in Model Theory}
\hypersetup{
 pdfauthor={Katin Tent \& Martin Ziegler},
 pdftitle={A Course in Model Theory},
 pdfkeywords={},
 pdfsubject={},
 pdfcreator={Emacs 27.2 (Org mode 9.5)}, 
 pdflang={English}}
\begin{document}

\maketitle
\tableofcontents

\section{The Basics}
\label{sec:org2d69eaf}

\subsection{Structures}
\label{sec:org948ed44}
\begin{definition}[]
A \textbf{language} \(L\) is a set of constants, function symbols and relation symbols
\end{definition}

\begin{definition}[]
Let \(L\) be a language. An \textbf{\(L\)-structure} is a pair \(\fA=(A,(Z^{\fA})_{Z\in L})\) where
\begin{center}
\begin{tabular}{ll}
\(A\) & if a non-empty set, the \textbf{domain} or \textbf{universe} of \(\fA\)\\
\(z^{\fA}\in A\) & if \(Z\) is a constant\\
\(Z^{\fA}:A^n\to A\) & if \(Z\) is an \(n\)-ary function symbol\\
\(Z^{\fA}\subseteq A^n\) & if \(Z\) is an \(n\)-ary relation symbol\\
\end{tabular}
\end{center}
\end{definition}



\begin{definition}[]
Let \(\fA,\fB\) be \(L\)-structures. A map \(h:A\to B\) is called a
\textbf{homomorphism} if for all \(a_1,\dots,a_n\in A\)
\begin{equation*}
 \begin{array}{rcl}
 h(c^{\fA})&=&c^{\fB}\\
 h(f^{\fA}(a_1,\dots,a_n))&=&f^{\fB}(h(a_1),\dots,h(a_n))\\
 R^{\fA}(a_1,\dots,a_n)&\Rightarrow&R^{\fB}(h(a_1),\dots,h(a_n))
 \end{array}
\end{equation*}

We denote this by
\begin{equation*}
 h:\fA\to\fB
\end{equation*}

If in addition \(h\) is injective and
\begin{equation*}
 R^{\fA}(a_1,\dots,a_n)\Leftrightarrow R^{\fB}(h(a_1),\dots,h(a_n))
\end{equation*}
for all \(a_1,\dots,a_n\in A\), then \(h\) is called an (isomorphic)
\textbf{embedding}. An \textbf{isomorphism} is a surjective embedding
\end{definition}

\begin{definition}[]
We call \(\fA\) a \textbf{substructure} of \(\fB\) if \(A\subseteq B\) and if the inclusion map is an embedding
from \(\fA\) to \(\fB\). We denote this by
\begin{equation*}
\fA\subseteq\fB
\end{equation*}
We say \(\fB\) is an \textbf{extension} of \(\fA\) if \(\fA\) is a substructure of \(\fB\)
\end{definition}




\begin{lemma}[]
\label{lemma1.1.8}
Let \(h:\fA \isoto\fA'\) be an isomorphism and \(\fB\) an
extension of \(\fA\). Then there exists an extension \(\fB'\) of \(\fA'\) and
an isomorphism \(g:\fB \isoto\fB'\) extending \(h\)
\end{lemma}

For any family \(\fA_i\) of substructures of \(\fB\), the intersection of the
\(A_i\) is either empty or a substructure of \(\fB\). Therefore if \(S\) is
any non-empty subset of \(\fB\), then there exists a smallest substructure
\(\fA=\la S\ra^{\fB}\) which contains \(S\). We call the \(\fA\) the
substructure \textbf{generated} by \(S\)

\begin{lemma}[]
If \(\fa=\la S\ra\), then every homomorphism \(h:\fA\to\fB\) is determined by
its values on \(S\)
\end{lemma}

\begin{definition}[]
Let \((I,\le)\) be a \textbf{directed partial order}. This means that for all
\(i,j\in I\) there exists a \(k\in I\) s.t. \(i\le k\) and \(j\le k\). A
family \((\fA_i)_{i\in I}\) of \(L\)-structures is called \textbf{directed} if
\begin{equation*}
i\le j\Rightarrow\fA_i\subseteq\fA_j
\end{equation*}
If \(I\) is linearly ordered, we call \((\fA_i)_{i\in I}\) a \textbf{chain}
\end{definition}

If a structure \(\fA_1\) is isomorphic to a substructure \(\fA_0\) of itself,
\begin{equation*}
 h_0:\fA_0\isoto\fA_1
\end{equation*}
then Lemma \ref{lemma1.1.8} gives an extension
\begin{equation*}
 h_1:\fA_1\isoto\fA_2
\end{equation*}
Continuing in this way we obtain a chain 
\(\fA_0\subseteq \fA_1\subseteq\fA_2\subseteq\dots\)
and an increasing sequence
\(h_i:\fA_i\isoto\fA_{i+1}\) of isomorphism

\begin{lemma}[]
Let \((\fA_i)_{i\in I}\) be a directed family of \(L\)-structures. Then
\(A=\bigcup_{i\in I}A_i\) is the universe of a (uniquely determined)
\(L\)-structure
\begin{equation*}
\fA=\bigcup_{i\in I}\fA_i
\end{equation*}
which is an extension of all \(\fA_i\)
\end{lemma}

A subset \(K\) of \(L\) is called a \textbf{sublanguage}. An \(L\)-structure becomes a
\(K\)-structure, the \textbf{reduct}.
\begin{equation*}
\fA\restriction K=(A,(Z^{\fA})_{Z\in K})
\end{equation*}
Conversely we call \(\fA\) an \textbf{expansion} of \(\fA\restriction K\).
\begin{enumerate}
\item Let \(B\subseteq A\) , we obtain a new language
\begin{equation*}
L(B)=L\cup B
\end{equation*}
and the \(L(B)\)-structure 
\begin{equation*}
\fA_B=(\fA,b)_{b\in B}
\end{equation*}
Note that \(\Aut(\fA_B)\) is the group of automorphisms of \(\fA\) fixing
\(B\) elementwise. We denote this group by \(\Aut(\fA/B)\)
\end{enumerate}


Let \(S\) be a set, which we call the set of sorts. An \(S\)-sorted
language \(L\) is given by a set of constants for each sort in \(S\), and
typed function and relations. For any tuple \((s_1,\dots,s_n)\) and
\((s_1,\dots,s_n,t)\) there is a set of relation symbols and function
symbols respectively. An \(S\)-sorted structure is a pair
\(\fA=(A,(Z^{\fA})_{Z\in L})\), where 
\begin{alignat*}{2}      
&A&&\text{if a family $(A_s)_{s\in S}$ of non-empty sets}\\
&Z^{\fA}\in A_s&&\text{if $Z$ is a constant of sort $s\in S$}\\
&Z^{\fA}:A_{s_1}\times\dots\times A_{s_n}\to A_t&&\text{if $Z$ is a
function symbol of type $(s_1,\dots,s_n,t)$}\\
&Z^{\fA}\subseteq A_{s_1}\times\dots\times A_{s_n}&&\text{if $Z$ is a
relation symbol of type $(s_1,\dots,s_n)$}
\end{alignat*}

\begin{examplle}[]
Consider the two-sorted language \(L_{Perm}\) for permutation groups with a
sort \(x\) for the set and a sort \(g\) for the group. The constants and
function symbols for \(L_{Perm}\) are those of \(L_{Group}\) restricted to
the sort \(g\) and an additional function symbol \(\varphi\) of type \((x,g,x)\). Thus
an \(L_{Perm}\)-structure \((X,G)\) is given by a set \(X\) and an
\(L_{Group}\)-structure \(G\) together with a function \(X\times G\to X\)
\end{examplle}

\subsection{Language}
\label{sec:orge6b4057}
\begin{lemma}[]
\label{lemma1.2.11}
Suppose \(\vv{b}\) and \(\vv{c}\) agree on all variables which are free
in \(\varphi\). Then 
\begin{equation*}
\fA\vDash\varphi[\vv{b}]\Leftrightarrow\fA\vDash\varphi[\vv{c}]
\end{equation*}
\end{lemma}

We define
\begin{equation*}
\fA\vDash\varphi[a_1,\dots,a_n]
\end{equation*}
by \(\fA\vDash\varphi[\vv{b}]\), where \(\vv{b}\) is an assignment
satisfying \(\vv{b}(x_i)=a_i\). Because of Lemma \ref{lemma1.2.11} this is
well defined.




Thus \(\varphi(x_1,\dots,x_n)\) defines an \(n\)-ary relation
\begin{equation*}
\varphi(\fA)=\{\bbar{a}\mid\fA\vDash\varphi[\bbar{a}]\}
\end{equation*}
on \(A\), the \textbf{realisation set} of \(\varphi\). Such realisation sets are called
\textbf{0-definable subsets} of \(A^n\), or 0-definable relations

Let \(B\) be a subset of \(A\). A \textbf{\(B\)-definable} subset of \(\fA\) is a set
of the form \(\varphi(\fA)\) for an \(L(B)\)-formula \(\varphi(x)\). We also say that \(\varphi\)
are defined \textbf{over} \(B\) and that the set \(\varphi(\fA)\) is defined by \(\varphi\). We call
two formulas \textbf{equivalent} if in every structure they define the same set.

Atomic formulas and their negations are called \textbf{basic}. Formulas without
quantifiers are Boolean combinations of basic formulas. It is convenient to
allow the empty conjunction and the empty disjunction. For that we introduce
two new formulas: the formula \(\top\), which is always true, and the formula
\(\bot\), which is always false. We define
\begin{align*}
&\bigwedge_{i<0}\pi_i=\top\\
&\bigvee_{i<0}\pi_i=\bot
\end{align*}

A formula is in \textbf{negation normal form} if it is built from basic formulas using
\(\wedge,\vee,\exists,\forall\)

\begin{lemma}[]
Every formula can be transformed into an equivalent formula which is in negation normal form
\end{lemma}

\begin{proof}
Let \(\sim\) denote equivalence of formulas. We consider formulas which are built
using \(\wedge,\vee,\exists,\forall,\neg\) and move the negation symbols in front of atomic formulas using
\begin{align*}
\neg(\varphi\wedge\psi)&\sim(\neg\varphi\vee\neg\psi)\\
\neg(\varphi\vee\psi)&\sim(\neg\varphi\wedge\neg\psi)\\
\neg\exists x\varphi&\sim\forall x\neg\varphi\\
\neg\forall x\varphi&\sim\exists x\neg\varphi\\
\neg\neg\varphi&\sim\varphi
\end{align*}
\end{proof}

\begin{definition}[]
A formula in negation normal form which does not contain any existential
quantifier is called \textbf{universal}. Formulas in negation normal form without
universal quantifiers are called \textbf{existential}
\end{definition}

\begin{lemma}[]
\label{lemma1.2.16}
Let \(h:\fA\to\fB\) be an embedding. Then for all existential formulas \(\varphi(x_1,\dots,x_n)\) and
all \(a_1,\dots,a_n\in A\) we have
\begin{equation*}
\fA\vDash\varphi[a_1,\dots_,a_n]\Rightarrow\fB\vDash\varphi[h(a_1),\dots,h(a_n)]
\end{equation*}
For universal \(\varphi\), the dual holds
\begin{equation*}
\fB\vDash\varphi[h(a_1),\dots,h(a_n)]\Rightarrow
\fA\vDash\varphi[a_1,\dots,a_n]
\end{equation*}
\end{lemma}


Let \(\fA\) be an \(L\)-structure. The \textbf{atomic diagram} of \(\fA\) is
\begin{equation*}
\Diag(\fA)=\{\varphi\text{ basic $L(A)$-sentence}\mid\fA_A\vDash\varphi\}
\end{equation*}

\begin{lemma}[]
The models of \(\Diag(\fA)\) are precisely those structures
\((\fB,h(a))_{a\in A}\) for embeddings \(h:\fA\to\fB\)
\end{lemma}

\begin{proof}
The structures \((\fB,h(a))_{a\in A}\) are models of the atomic diagram by Lemma \ref{1.2.16}.  For
the converse, note that a map \(h\) is an embedding iff it preserves the validity of all formulas
of the form
\begin{align*}
&(\neg)x_1\dot{=}x_2\\
&c\dot=x_1\\
&f(x_1,\dots,x_n)\dot=x_0\\
&(\neg)R(x_1,\dots,x_n)
\end{align*}
\end{proof}

\begin{exercise}
\label{ex1.2.3}
Every formula is equivalent to a formula in prenex normal form:
\begin{equation*}
Q_1x_1\dots Q_nx_n\varphi
\end{equation*}
The \(Q_i\) are quantifiers and \(\varphi\) is quantifier-free
\end{exercise}

\begin{proof}
\begin{align*}
&(\forall x)\phi\wedge\psi\vDash\rmodels
\forall x(\phi\wedge\psi)\text{ if }\exists x\top(\text{at least one individual exists})\\
&(\forall x\phi)\vee\psi\vDash\rmodels\forall x(\phi\vee\psi)\\
&(\exists x\phi)\wedge\psi\vDash\rmodels\exists x(\phi\wedge\psi)\\
&(\exists x\phi)\vee\psi\vDash\rmodels\exists x(\phi\vee\psi)\text{ if }\exists x\top\\
&\neg\exists x\phi\vDash\rmodels\forall x\neg\phi\\
&\neg\forall x\phi\vDash\rmodels\exists x\neg\phi\\
&(\forall x\phi)\to\psi\vDash\rmodels\exists x(\phi\to\psi)\text{ if }\exists x\top\\
&(\exists x\phi)\to\psi\vDash\rmodels\forall x(\phi\to\psi)\\
&\phi\to(\exists x\psi)\vDash\rmodels\exists x(\phi\to\psi)\text{ if }\exists x\top\\
&\phi\to(\forall x\psi)\vDash\rmodels\forall x(\phi\to\psi)
\end{align*}
\end{proof}

\subsection{Theories}
\label{sec:org02d6865}
\begin{definition}[]
An \textbf{\(L\)-theory} \(T\) is a set of \(L\)-sentences
\end{definition}

A theory which has a model is a \textbf{consistent} theory. We call a set \(\Sigma\) of
\(L\)-formulas \textbf{consistent} if there is an \(L\)-structure and \textbf{an assignment}
\(\vv{b}\) \textbf{s.t.} \(\fA\vDash[\vv{b}]\) for all \(\varphi\in\Sigma\)

\begin{lemma}[]
Let \(T\) be an \(L\)-theory and \(L'\) be an extension of \(L\). Then \(T\)
is consistent as an \(L\)-theory iff \(T\) is consistent as a \(L'\)-theory
\end{lemma}


\begin{lemma}[]
\label{lemma1.3.4}
\begin{enumerate}
\item If \(T\vDash\varphi\) and \(T\vDash(\varphi\to\psi)\), then \(T\vDash\psi\)
\item If \(T\vDash\varphi(c_1,\dots,c_n)\) and the constants \(c_1,\dots,c_n\)
occur neither in \(T\) nor in \(\varphi(x_1,\dots,x_n)\), then
\(T\vDash\forall x_1\dots x_n\varphi(x_1,\dots,x_n)\)
\end{enumerate}
\end{lemma}

\begin{proof}
\begin{enumerate}
\setcounter{enumi}{1}
\item Let \(L'=L\setminus\{c_1,\dots,c_n\}\). If the \(L'\)-structure is a
model of \(T\) and \(a_1,\dots,a_n\) are arbitrary elements, then
\((\fA,a_1,\dots,a_n)\vDash\varphi(c_1,\dots,c_n)\). This means
\(\fA\vDash\forall x_1\dots x_n\varphi(x_1,\dots,x_n)\).
\end{enumerate}
\end{proof}

\(S\) and \(T\) are called \textbf{equivalent}, \(S\equiv T\), if \(S\) and \(T\) have
the same models

\begin{definition}[]
A consistent \(L\)-theory \(T\) is called \textbf{complete} if for all \(L\)-sentences
\(\varphi\)
\begin{equation*}
T\vDash\varphi \quad\text{ or }\quad T\vDash\neg\varphi
\end{equation*}
\end{definition}

\begin{definition}[]
For a complete theory \(T\) we define
\begin{equation*}
\abs{T}=\max(\abs{L},\aleph_0)
\end{equation*}
\end{definition}

The typical example of a complete theory is the theory of a structure \(\fA\)
\begin{equation*}
\Th(\fA)=\{\varphi\mid\fA\vDash\varphi\}
\end{equation*}

\begin{lemma}[]
\label{lemma1.3.7}
A consistent theory is complete iff it is maximal consistent, i.e., if it is
equivalent to every consistent extension
\end{lemma}

\begin{definition}[]
Two \(L\)-structures \(\fA\) and \(\fB\) are called \textbf{elementary equivalent}
\begin{equation*}
\fA\equiv\fB
\end{equation*}
if they have the same theory
\end{definition}

\begin{lemma}[]
Let \(T\) be a consistent theory. Then the following are equivalent
\begin{enumerate}
\item \(T\) is complete
\item All models of \(T\) are elemantarily equivalent
\item There exists a structure \(\fA\) with \(T\equiv\Th(\fA)\)
\end{enumerate}
\end{lemma}

\begin{proof}
\(1\to3\to2\to1\)
\end{proof}





\section{Elementary Extensions and Compactness}
\label{sec:org78ab730}
\subsection{Elementary substructures}
\label{sec:org7d803f5}
Let \(\fA,\fB\) be two \(L\)-structures. A map \(h:A\to B\) is called
\textbf{elementary} if for all \(a_1,\dots,a_n\in A\) we have
\begin{equation*}
\fA\vDash\varphi[a_1,\dots,a_n]\Leftrightarrow
\fB\vDash\varphi[h(a_1),\dots,h(a_n)]
\end{equation*}
which is actually saying \((\fA,a)_{a\in A}\equiv(\fB,a)_{a\in A}\).
We write
\begin{equation*}
h:\fA\embto\fB
\end{equation*}
\begin{lemma}[]
\label{lemma2.1.1}
The models of \(\Th(\fA_A)\) are exactly the structures of the form
\((\fB,h(a))_{a\in A}\) for elementary embeddings \(h:\fA\embto\fB\)
\end{lemma}

We call \(\Th(\fA_A)\) the \textbf{elemantary diagram} of \(\fA\)

A substructure \(\fA\) of \(\fB\) is called \textbf{elementary} if the inclusion map
is elementary. In this case we write
\begin{equation*}
\fA\prec\fB
\end{equation*}

\begin{theorem}[Tarski's Test]
\label{thm2.1.2}
Let \(\fB\) be an \(L\)-structure and \(A\) a subset of \(B\). Then \(A\) is
the universe of an elementary substructure iff every \(L(A)\)-formula
\(\varphi(x)\) which is satisfiable in \(\fB\) can be satisfied by an element of \(A\)
\end{theorem}

\begin{proof}
If \(\fA\prec\fB\) and \(\fB\vDash\exists\varphi(x)\), we also have \(\fA\vDash\exists x\varphi(x)\) and there eexists \(a\in A\)
s.t. \(\fA\vDash\varphi(a)\). Thus \(\fB\vDash\varphi(a)\)

Conversely, suppose that the condition of Tarski'test is satisfied. First we show that \(A\) is
the universe of a substructure \(\fA\). The \(L(A)\)-formula \(x\dot=x\) is satisfiable in \(\fA\),
so \(A\) is not empty. If \(f\in L\) is an \(n\)-ary function symbol \((n\ge 0)\) and \(a_1,\dots,a_n\) is
from \(A\), we consider the formula
\begin{equation*}
\varphi(x)=f(a_1,\dots,a_n)\dot=x
\end{equation*}
Since \(\varphi(x)\) is always satisfied by an element of \(A\), it follows that \(A\) is closed
under \(f^{\calb}\)

Now we show, by induction on \(\psi\), that
\begin{equation*}
\fA\vDash\psi\Leftrightarrow\fB\vDash\psi
\end{equation*}
for all \(L(A)\)-sentences \(\psi\).

For \(\psi=\exists x\varphi(x)\). If \(\psi\) holds in \(\fA\). If \(\psi\) holds in \(\fA\), there exists \(a\in A\)
s.t. \(\fA\vDash\varphi(a)\). The induction hypothesis yields \(\fB\vDash\varphi(x)\), thus \(\fB\vDash\psi\). For the converse
suppose \(\psi\) holds in \(\fB\). Then \(\varphi(x)\) is satisfied in \(\calb\) and by Tarski's test we
find \(a\in A\) s.t. \(\fB\vDash\varphi(a)\). By induction \(\fA\vDash\varphi(a)\) and \(\fA\vDash\psi\)
\end{proof}

We use Tarski's Test to construct small elementary substructures

\begin{corollary}[]
\label{cor2.1.3}
Suppose \(S\) is a subset of the \(L\)-structure \(\fB\). Then \(\fB\) has a
elementary substructure \(\fA\) containing \(S\) and of cardinality at most
\begin{equation*}
\max(\abs{S},\abs{L},\aleph_0)
\end{equation*}
\end{corollary}

\begin{proof}
We construct \(A\) as the union of an ascending sequence \(S_0\subseteq S_1\subseteq\dots\) of subsets of \(B\). We
start with \(S_0=S\). If \(S_i\) is already defined, we choose an element \(a_\varphi\in B\) for
every \(L(S_i)\)-formula \(\varphi(x)\) which is satisfiable in \(\fB\) and define \(S_{i+1}\) to
be \(S_i\) together with these \(a_{\varphi}\).

An \(L\)-formula is a finite sequence of symbols from \(L\), auxiliary symbols and logical
symbols. These are \(\abs{L}+\aleph_0=\max(\abs{L},\aleph_0)\) many symbols and there are
exactly\(\max(\abs{L},\aleph_0)\) many \(L\)-formulas

Let \(\kappa=\max(\abs{S},\abs{L},\aleph_0)\). There are \(\kappa\) many
\(L(S)\)-formulas: therefore \(\abs{S_1}\le\kappa\). Inductively it follows
for every \(i\) that \(\abs{S_i}\le\kappa\). Finally we have \(\abs{A}\le\kappa\cdot\aleph_0=\kappa\)
\end{proof}

A directed family \((\fA_i)_{i\in I}\) of structures is \textbf{elementary} if
\(\fA_i\prec\fA_j\) for all \(i\le j\)

\begin{theorem}[Tarski's Chain Lemma]
\label{thm2.1.4}
The union of an elementary directed family is an elementary extension of all
its members
\end{theorem}

\begin{proof}
Let \(\fA=\bigcup_{i\in I}(\fA_i)_{i\in I}\). We prove by induction on
\(\varphi(\bbar{x})\) that for all \(i\) and \(\bbar{a}\in\fA_i\)
\begin{equation*}
\fA_i\vDash\varphi(\bbar{a})\Leftrightarrow\fA\vDash\varphi(\bbar{a})
\end{equation*}
\end{proof}

\begin{exercise}
\label{ex2.1.1}
Let \(\fA\) be an \(L\)-structure and \((\fA_i)_{i\in I}\) a chain of elementary substructures of \(\fA\).
Show that \(\bigcup_{i\in I}A_i\) is an elementary substructure of \(\fA\).
\end{exercise}

\begin{exercise}
\label{ex2.1.2}
Consider a class \(\calc\) of \(L\)-structures. Prove
\begin{enumerate}
\item Let \(\Th(\calc)=\{\varphi\mid \fA\vDash\varphi\text{ for all }\fA\in\calc\}\) be the \textbf{theory of} \(\calc\). Then \(\fM\) is a model
of \(\Th(C)\) iff \(\fM\) is elementary equivalent to an ultraproduct of elements of \(\calc\)
\item Show that \(\calc\) is an elementary class iff \(\calc\) is closed under ultraproduct and elementary equivalence
\item Assume that \(\calc\) is a class of finite structures containing only finitely many structures of
size \(n\) for each \(n\in\omega\). Then the infinite models of \(\Th(\calc)\) are exactly the models of
\begin{equation*}
\Th_a(\calc)=\{\varphi\mid\fA\vDash\varphi\text{ for all but finitely many }\fA\in\calc\}
\end{equation*}
\end{enumerate}
\end{exercise}

\begin{proof}
Chang\&Keisler p220
\end{proof}


\subsection{The Compactness Theorem}
\label{sec:orgb85e7f3}
We call a theory \(T\) \textbf{finitely satisfiable} if every finite subset of \(T\) is consistent

\begin{theorem}[Compactness Theorem]
Finitely satisfiable theories are consistent
\end{theorem}

Let \(L\) be a language and \(C\) a set of new constants. An \(L(C)\)-theory
\(T'\) is called a \textbf{Henkin theory} if for every \(L(C)\)-formula \(\varphi(x)\) there
is a constant \(c\in C\) s.t.
\begin{equation*}
\exists x\varphi(x)\to\varphi(c)\in T'
\end{equation*}
The elements of \(C\) are called \textbf{Henkin constants} of \(T'\)

An \(L\)-theory \(T\) is \textbf{finitely complete} if it is finitely satisfiable and if
every \(L\)-sentence \(\varphi\) satisfies \(\varphi\in T\) or \(\neg\varphi\in T\)

\begin{lemma}[]
Every finitely satisfiable \(L\)-theory \(T\) can be extended to a finitely
complete Henkin Theory \(T^*\)
\end{lemma}

Note that conversely the lemma follows directly from the Compactness Theorem. Choose a
model \(\fA\) of \(T\). Then \(\Th(\fA_A)\) is a finitely complete Henkin theory with \(A\) as a set
of Henkin constants

\begin{proof}
We define an increasing sequence \(\emptyset=C_0\subseteq C_1\subseteq\cdots\) of new constants by assigning to
every \(L(C_i)\)-formula \(\varphi(x)\) a constant \(c_{\varphi(x)}\) and
\begin{equation*}
C_{i+1}=\{c_{\varphi(x)}\mid \varphi(x)\text{ a }L(C_i)\text{-formula}\}
\end{equation*}
Let \(C\) be the union of the \(C_i\) and \(T^H\) the set of all Henkin axioms
\begin{equation*}
\exists x\varphi(x)\to\varphi(c_{\varphi(x)})
\end{equation*}
for \(L(C)\)-formulas \(\varphi(x)\). It is easy to see that one can expand every \(L\)-structure to a
model of \(T^H\). Hence \(T\cup T^H\) is a finitely satisfiable Henkin theory. Using the fact that the
union of a chain of finitely satisfiable theories is also finite satisfiable, we can apply Zorn's
Lemma and get a maximal finitely satisfiable \(L(C)\)-theory \(T^*\) which contains \(T\cup T^H\). As
in Lemma \ref{lemma1.3.7} we show that \(T^*\) is finitely complete: if neither \(\varphi\) nor \(\neg\varphi\)
belongs to \(T^*\), neither \(T^*\cup\{\varphi\}\) nor \(T^*\cup\{\neg\varphi\}\) would be finitely satisfiable. Hence
there would be a finite subset \(\Delta\) of \(T^*\) which would be consistent neither with \(\varphi\) nor
with \(\neg\varphi\). Then \(\Delta\) itself would be inconsistent and \(T^*\) would not be finite satisfiable.
This proves the lemma.
\end{proof}

\begin{lemma}[]
\label{lemma2.2.3}
Every finitely satisfiable \(L\)-theory \(T\) can be extended to a finitely
complete Henkin theory \(T^*\)
\end{lemma}

\begin{lemma}[]
Every finitely complete Henkin theory \(T^*\) has a model \(\fA\) (unique up
to isomorphism) consisting of constants; i.e.,
\begin{equation*}
(\fA,a_c)_{c\in C}\vDash T^*
\end{equation*}
with \(A=\{a_c\mid c\in C\}\)
\end{lemma}

\begin{proof}
Since \(T^*\) is finite complete, every sentence which follows from a finite subset of \(T^*\)
belongs to \(T^*\)

Define for \(c,d\in C\)
\begin{equation*}
c\simeq d\Leftrightarrow c\dot=d\in T^*
\end{equation*}
\(\simeq\) is an equivalence relation. We denote the equivalence class of \(c\) by \(a_c\), and set
\begin{equation*}
A=\{a_c\mid c\in C\}
\end{equation*}
We expand \(A\) to an \(L\)-structure \(\fA\) by defining
\begin{align*}
R^{\fA}(a_{c_1},\dots,a_{c_n})&\Leftrightarrow R(c_1,\dots,c_n)\in T^*\tag{\star}\\
f^{\fA}(a_{c_1},\dots,a_{c_n})&\Leftrightarrow f(c_1,\dots,c_n)\dot=c_0\in T^*\tag{\star\star}
\end{align*}

We have to show that this is well-defined. For (\(\star\)) we have to show that
\begin{equation*}
a_{c_1}=a_{d_1},\dots,a_{c_n}=a_{d_n}, R(c_1,\dots,c_n)\in T^*
\end{equation*}
implies \(R(d_1,\dots,d_n)\in T^*\), which is obvious.

For (\(\star\star\)), we have to show that for all \(c_1,\dots,c_n\) there exists \(c_0\)
with \(f(c_1,\dots,c_n)\dot=c_0\in T^*\).

Let \(\fA^*\) be the \(L(C)\)-structure \((\fA,a_c)_{c\in C}\). We show by induction on the complexity
of \(\varphi\) that for every \(L(C)\)-sentence \(\varphi\)
\begin{equation*}
\fA^*\vDash\varphi\Leftrightarrow\varphi\in T^*
\end{equation*}
\end{proof}

\begin{corollary}[]
We have \(T\vDash\varphi\) iff \(\Delta\vDash\varphi\) for a finite subset \(\Delta\) of \(T\)
\end{corollary}



\begin{corollary}[]
\label{cor2.2.5}
A set of formulas \(\Sigma(x_1,\dots,x_n)\) is consistent with \(T\) if and only
if every finite subset of \(\Sigma\) is consistent with \(T\)
\end{corollary}

\begin{proof}
Introduce new constants \(c_1,\dots,c_n\). Then \(\Sigma\) is consistent with \(T\) is
and only if \(T\cup\Sigma(c_1,\dots,c_n)\) is consistent. Now apply the
Compactness Theorem
\end{proof}

\begin{definition}[]
Let \(\fA\) be an \(L\)-structure and \(B\subseteq A\). Then \(a\in A\)
\textbf{realises} a set of \(L(B)\)-formulas \(\Sigma(x)\) if \(a\) satisfied all formulas
from \(\Sigma\). We write 
\begin{equation*}
\fA\vDash\Sigma(a)
\end{equation*}

We call \(\Sigma(x)\) \textbf{finitely satisfiable} in \(\fA\) if every finite subset of \(\Sigma\)
is realised in \(\fA\)
\end{definition}

\begin{lemma}[]
\label{lemma2.2.7}
The set \(\Sigma(x)\) is finitely satisfiable in \(\fA\) iff there is an
elementary extension of \(\fA\) in which \(\Sigma(x)\) is realised
\end{lemma}

\begin{proof}
By Lemma \ref{lemma2.1.1} \(\Sigma\) is realised in an elementary extension of \(\fA\)
iff \(\Sigma\) is consistent with \(\Th(\fA_A)\). So the lemma follows from the
observation that a finite set of \(L(A)\)-formulas is consistent with
\(\Th(\fA_A)\) iff it is realised in \(\fA\)
\end{proof}

\index{type}
\begin{definition}[]
Let \(\fA\) be an \(L\)-structure and \(B\) a subset of \(A\). A set \(p(x)\)
of \(L(B)\)-formulas is a \textbf{type} over \(B\) if \(p(x)\) is maximal finitely
satisfiable in \(\fA\). We call \(B\) the \textbf{domain} of \(p\). Let
\begin{equation*}
S(B)=S^{\fA}(B)
\end{equation*}
denote the set of types over \(B\).
\end{definition}

Every element \(a\) of \(\fA\) determines a type
\begin{equation*}
\tp(a/B)=tp^{\fA}(a/B)=\{\varphi(x)\mid\fA\vDash\varphi(a),\varphi\text{ an $L(B)$-formula}\}
\end{equation*}
So an element \(a\) realises the type \(p\in S(B)\) exactly if
\(p=\tp(a/B)\). If \(\fA'\) is an elementary extension of \(\fA\), then
\begin{equation*}
S^{\fA}(B)=S^{\fA'}(B)\quad\text{ and }\quad
\tp^{\fA'}(a/B)=\tp^{\fA}(a/B)
\end{equation*}
If \(\fA'\vDash p(x)\) then \(\fA'\vDash\exists xp(x)\), so
\(\fA\vDash\exists xp(x)\).

We use the notation \(\tp(a)\) for \(\tp(a/\emptyset)\)

Maximal finitely satisfiable sets of formulas in \(x_1,\dots,x_n\) are called
\textbf{\(n\)-types} and
\begin{equation*}
S_n(B)=S_N^{\fA}(B)
\end{equation*}
denotes the set of \(n\)-types over \(B\).
\begin{equation*}
\tp(C/B)=\{\varphi(x_{c_1},\dots,x_{c_n})\mid\fA\vDash\varphi(c_1,\dots,c_n),\varphi
\text{ an $L(B)$-formula}\}
\end{equation*}

\begin{corollary}[]
Every structure \(\fA\) has an elementary extension \(\fB\) in which all
types over \(A\) are realised
\end{corollary}

\begin{proof}
We choose for every \(p\in S(A)\) a new constant \(c_p\). We have to find a
model of
\begin{equation*}
\Th(\fA_A)\cup\bigcup_{p\in S(A)}p(c_p)
\end{equation*}
This theory is finitely satisfiable since every \(p\) is finitely satisfiable
in \(\fA\).

Or use Lemma \ref{lemma2.2.7}. Let \((p_\alpha)_{\alpha<\lambda}\) be an enumeration of
\(S(A)\). Construct an elementary chain
\begin{equation*}
\fA=\fA_0\prec\fA_1\prec\dots\prec\fA_\beta\prec\dots(\beta\le\lambda)
\end{equation*}
s.t. each \(p_\alpha\) is realised in \(\fA_{\alpha+1}\) (by recursion
theorem on ordinal numbers)

Suppose that the elementary chain \((\fA_{\alpha'})_{\alpha'<\beta}\) is already
constructed. If \(\beta\) is a limit ordinal, we let
\(\fA_\beta=\bigcup_{\alpha<\beta}\fA_\alpha\), which is elementary by Lemma \ref{thm2.1.4}. If
\(\beta=\alpha+1\) we  first note that \(p_\alpha\) is also finitely
satisfiable in \(\fA_\alpha\), therefore we can realise \(p_\alpha\) in a
suitable elementary extension \(\fA_\beta\succ\fA_\alpha\) by Lemma
\ref{lemma2.2.7}. Then \(\fB=\fA_\lambda\) is the model we were looking for
\end{proof}

\subsection{The Löwenheim-Skolem Theorem}
\label{sec:orge3f6f25}
\begin{theorem}[Löwenheim-Skolem]
Let \(\fB\) be an \(L\)-structure, \(S\) a subset of \(B\) and \(\kappa\) an infinite
cardinal
\begin{enumerate}
\item If
\begin{equation*}
\max(\abs{S},\abs{L})\le\kappa\le\abs{B}
\end{equation*}
then \(\fB\) has an elementary substructure of cardinality \(\kappa\) containing \(S\)
\item If \(\fB\) is infinite and
\begin{equation*}
\max(\abs{\fB},\abs{L})\le\kappa
\end{equation*}
then \(\fB\) has an elementary extension of cardinality \(\kappa\)
\end{enumerate}
\end{theorem}

\begin{proof}
\begin{enumerate}
\item Choose a set \(S\subseteq S'\subseteq B\) of cardinality \(\kappa\) and apply Corollary \ref{cor2.1.3}
\item We first construct an elementary extension \(\fB'\) of cardinality at least \(\kappa\). Choose a
set \(C\) of new constants of cardinality \(\kappa\). As \(\fB\) is infinite, the theory
\begin{equation*}
\Th(\fB_B)\cup\{\neg c\dot=d\mid c,d\in C,c\neq d\}
\end{equation*}
is finitely satisfiable. By Lemma \ref{lemma2.1.1} any model \((\fB'_B,b_c)_{c\in C}\) is an
elementary extension of \(\calb\) with \(\kappa\) many different elements \((b_c)\)

Finally we apply the first part of the theorem to \(\calb'\) and \(S=B\)
\end{enumerate}
\end{proof}

\begin{corollary}[]
A theory which has an infinite model has a model in every cardinality \(\kappa\ge\max(\abs{L},\aleph_0)\)
\end{corollary}

\begin{definition}[]
Let \(\kappa\) be an infinite cardinal. A theory \(T\) is called
\textbf{\(\kappa\)-categorical} if for all models of \(T\) of cardinality \(\kappa\) are isomorphic
\end{definition}

\begin{theorem}[Vaught's Test]
\label{thm2.3.4}
A \(\kappa\)-categorical theory \(T\) is complete if the following conditions
are satisfied
\begin{enumerate}
\item \(T\) is consistent
\item \(T\) has no finite model
\item \(\abs{L}\le\kappa\)
\end{enumerate}
\end{theorem}

\begin{proof}
We have to show that all models \(\fA\) and \(\fB\) of \(T\) are elemantarily
equivalent. As \(\fA\) and \(\fB\) are infinite, \(\Th(\fA)\) and
\(\Th(\fB)\) have models \(\fA'\) and \(\fB'\) of cardinality \(\kappa\). By
assumption \(\fA'\) and \(\fB'\) are isomorphic, and it follows that
\begin{equation*}
\fA\equiv\fA'\equiv\fB'\equiv\fB
\end{equation*}
\end{proof}

\begin{examplle}[]
\begin{enumerate}
\item The theory \(\DLO\) of dense linear orders without endpoints is
\(\aleph_0\)-categorical and by Vaught's test complete. Let
\(A=\{a_i\mid i\in\omega\}\), \(B=\{b_i\mid i\in\omega\}\).
We inductively define
sequences \((c_i)_{i<\omega}\), \((d_i)_{i<\omega}\) exhausting \(A\) and \(B\).
Assume that \((c_i)_{i<m},(d_i)_{i<m}\) have defined so that \(c_i\mapsto
      d_i,i<m\) is an order isomorphism. If \(m=2k\) let \(c_m=a_j\) where
\(a_j\) is the element with minimal index in \(\{a_i\mid i\in\omega\}\)
not occurring in \((c_i)_{i<m}\). Since \(\fB\) is a dense linear order
without endpoints there is some element \(d_m\in\{b_i\mid i\in\omega\}\)
s.t. \((c_i)_{i\le m}\) and \((d_i)_{i\le m}\) are order isomorphic. If
\(m=2k+1\) we interchange the roles of \(\fA\) and \(\fB\)
\item For any prime \(p\) or \(p=0\), the theory \(\ACF_p\) of algebraically closed fields of
characteristic \(p\) is \(\kappa\)-categorical for any \(\kappa>\aleph_0\)
\end{enumerate}
\end{examplle}


Consider the Theorem \ref{thm2.3.4} we strengthen our definition
\begin{definition}[]
Let \(\kappa\) be an infinite cardinal. A theory \(T\) is called
\textbf{\(\kappa\)-categorical} if it is complete, \(\abs{T}\le\kappa\) and, up to
isomorphism, has exactly one model of cardinality \(\kappa\)
\end{definition}


\section{Quantifier Elimination}
\label{sec:orgbc2669b}
\subsection{Preservation theorems}
\label{sec:org69f8240}
\begin{lemma}[Separation Lemma]
Let \(T_1, T_2\) be two theories. Assume \(\calh\) is a set of sentences
which is closed under \(\wedge,\vee\) and contains \(\bot\) and \(\top\).
Then the following are equivalent
\begin{enumerate}
\item There is a sentence \(\varphi\in\calh\) which separates \(T_1\) from
\(T_2\). This means
\begin{equation*}
  T_1\vDash\varphi \quad\text{ and }\quad
  T_2\vDash\neg\varphi
\end{equation*}
\item All models \(\fA_1\) of \(T_1\) can be separated from all models \(\fA_2\)
of \(T_2\) by a sentence \(\varphi\in\calh\). This means
\begin{equation*}
  \fA_1\vDash\varphi \quad\text{ and }\quad\fA_2\vDash\neg\varphi
\end{equation*}
\end{enumerate}
\end{lemma}

For 1, suppose \(T_1=T\cup\{\psi\}\) and \(T_2=T\cup\{\neg\psi\}\). If \(T_1\vDash\varphi\) and \(T_2\vDash\neg\varphi\), then
\(T\vDash\psi\to\varphi\) and \(T\vDash\neg\psi\to\neg\varphi\) which is equivalent to \(T\vDash\varphi\to\psi\). Thus we have \(T\vDash\varphi\leftrightarrow\psi\).

\begin{proof}
\(2\to1\). For any model \(\fA_1\) of \(T_1\) let \(\calh_{\fA_1}\) be the
set of all sentences from \(\calh\) which are true in \(\fA_1\). (2) implies
that \(\calh_{\fA_1}\) and \(T_2\) cannot have a common model. By the
Compactness Theorem there is a finite conjunction \(\varphi_{\fA_1}\) of
sentences from \(\calh_{\fA_1}\) inconsistent with \(T_2\). Clearly
\begin{equation*}
 T_1\cup\{\neg\varphi_{\fA_1}\mid\fA_1\vDash T_1\}
\end{equation*}
is inconsistent. Again by compactness \(T_1\) implies a disjunction \(\varphi\) of
finitely many of the \(\varphi_{\fA_1}\) (Corollary \ref{cor2.2.5}) and
\begin{equation*}
T_1\vDash\varphi \quad\text{ and }\quad T_2\vDash\neg\varphi
\end{equation*}
\end{proof}



For structures \(\fA,\fB\) and a map \(f:A\to B\) preserving all formulas
from a set of formulas \(\Delta\), we use the notation
\begin{equation*}
f:\fA\to_\Delta\fB
\end{equation*}
We also write
\begin{equation*}
\quad\fA\Rightarrow_\Delta\fB
\end{equation*}
to express that all sentences from \(\Delta\) true in \(\fA\) are also true in \(\fB\)

\begin{lemma}[]
\label{lemma3.1.2}
Let \(T\) be a theory, \(\fA\) a structure and \(\Delta\) a set of formulas, closed
under existential quantification, conjunction and substitution of variables.
Then the following are equivalent
\begin{enumerate}
\item All sentences \(\varphi\in\Delta\) which are true in \(\fA\) are
consistent with \(T\)
\item There is a model \(\fB\vDash T\) and a map \(f:\fA\to_\Delta\fB\)
\end{enumerate}
\end{lemma}

\begin{proof}
\(2\to 1\). Assume \(f:\fA\to_\Delta\fB\vDash T\). If \(\varphi\in\Delta\) is true in \(\fA\), it is also true in \(\fB\) and
therefore consistent with \(T\).

\(1\to2\). Consider \(\Th_\Delta(\fA_A)\), the set of all sentences \(\delta(\bbar{a})\)
(\(\delta(\bbar{x})\in\Delta\)), which are true in \(\fA_A\). The models
\((\fB,f(a)_{a\in A})\) of this theory correspond to maps
\(f:\fA\to_\Delta\fB\). \textbf{This means that we have to find a model of}
\(T\cup\Th_\Delta(\fA_A)\). To show finite satisfiability it is enough to
show that \(T\cup D\) is consistent for every finite subset \(D\) of
\(\Th_\Delta(\fA_A)\). Let \(\delta(\bbar{a})\) be the conjunction of the elements
of \(D\). Then \(\fA\) is a model of \(\varphi=\exists\bar{x}\delta(\barx)\), so by assumption \(T\) has a
model \(\fB\) which is also a model of \(\varphi\). This means that there is a tuple \(\barb\) s.t. \((\fB,\barb)\vDash\delta(\bara)\)
\end{proof}

Lemma \ref{lemma3.1.2} applied to \(T=\Th(\fB)\) shows that
\(\fA\Rightarrow_\Delta\fB\) iff there exists a map \(f\) and a structure
\(\fB'\equiv\fB\) s.t. \(f:\fA\to_\Delta\fB'\)

\begin{theorem}[]
\label{thm3.1.3}
Let \(T_1\) and \(T_2\) be two theories. Then the following are equivalent
\begin{enumerate}
\item There is a universal sentence which separates \(T_1\) from \(T_2\)
\item No model of \(T_2\) is a substructure of a model of \(T_1\)
\end{enumerate}
\end{theorem}

\begin{proof}
\(1\to 2\).  Let \(\varphi\) be a universal sentence which separates \(T_1\) and \(T_2\). Let \(\fA_1\) be a model
of \(T_1\) and \(\fA_2\) a substructure of \(\fA_1\). Since \(\fA_1\) is a model of \(\varphi\), \(\fA_2\) is also a
model of \(\varphi\). Therefore \(\fA_2\) cannot be a model of \(T_2\)

 \(2\to1\).   Here we add some details for the proof \(2\to 1\). If \(T_1\) and \(T_2\) cannot be separated by a
universal sentence, then they have models \(\fA_1\) and \(\fA_2\)  which cannot be separated by a
universal sentence. That is, for all universal sentence \(\varphi\), if \(\fA_1\vDash\varphi\) then \(\fA_2\vDash\varphi\).
Thus \(\fA_1\Rightarrow_\forall\fA_2\), here \(\Rightarrow_\forall\) means for all universal sentence.

Now note that
\begin{equation*}
\fA_1\vDash\varphi\to\fA_2\vDash\varphi\quad\Leftrightarrow\quad\fA_2\vDash\neg\varphi\to\fA_2\vDash\neg\varphi
\end{equation*}
and \(\neg\varphi\) is an existential sentence. Hence we have
\begin{equation*}
\fA_2\Rightarrow_{\exists}\fA_1
\end{equation*}

The reason that we want to use \(\exists\) is that it holds in the substructure case and we could
imagine that \(\fA_2\subseteq\fA_1\) (I guess this is our intuition). Now by Lemma \ref{lemma3.1.2} we
have \(\fA_1'\equiv\fA_1\) and a map \(f:\fA_2\to_\exists\fA_1'\).
Apparently \(\fA_1'\vDash\Diag(\fA_2)\) and \(f\) is an embedding. Hence \(\fA_1'\) is a model of \(T_1\) and \(T_2\)
\end{proof}

\begin{definition}[]
For any \(L\)-theory \(T\), the formulas \(\varphi(\bbar{x}),\psi(\bbar{x})\) are said
to be \textbf{equivalent} modulo \(T\) (or relative to \(T\)) if \(T\vDash\forall\bbar{x}(\varphi(\bbar{x})\leftrightarrow\psi(\bbar{x}))\)
\end{definition}

\begin{corollary}[]
\label{cor3.1.5}
Let \(T\) be a theory
\begin{enumerate}
\item Consider a formula \(\varphi(x_1,\dots,x_n)\). The following are equivalent
\begin{enumerate}
\item \(\varphi(x_1,\dots,x_n)\) is, modulo \(T\), equivalent to a universal formula
\item If \(\fA\subseteq\fB\) are models of \(T\) and \(a_1,\dots,a_n\in A\),
then \(\fB\vDash\varphi(a_1,\dots,a_n)\) implies \(\fA\vDash\varphi(a_1,\dots,a_n)\)
\end{enumerate}
\item We say that a theory which consists of universal sentences is universal.
Then \(T\) is equivalent to a universal theory iff all substructures of
models of \(T\) are again models of \(T\)
\end{enumerate}
\end{corollary}

\begin{proof}
\begin{enumerate}
\item Assume (2). We extend \(L\) by an \(n\)-tuple \(\bbar{c}\) of new
constants \(c_1,\dots,c_n\) and consider theory
\begin{equation*}
T_1=T\cup\{\varphi(\bbar{c})\}\quad\text{ and }\quad
T_2=T\cup\{\neg\varphi(\bbar{c})\}
\end{equation*}
Then (2) says the substructures of models of \(T_1\) cannot be models of
\(T_2\). By Theorem \ref{thm3.1.3} \(T_1\) and \(T_2\) can be separated by a
universal \(L(\bbar{c})\)-sentence \(\psi(\bbar{c})\). By Lemma
\ref{lemma1.3.4}, \(T_1\vDash\psi(\bbar{c})\) implies
\begin{equation*}
T\vDash\forall\bbar{x}(\varphi(\bbar{x})\to\psi(\bbar{x}))
\end{equation*}
and from \(T_2\vDash\neg\psi(\bbar{c})\) we see
\begin{equation*}
T\vDash\forall\bbar{x}(\neg\varphi(\bbar{x})\to\neg\psi(\bbar{x}))
\end{equation*}
\item Suppose a theory \(T\) has this property. Let \(\varphi\) be an axiom of \(T\). If
\(\fA\) is a substructure of \(\fB\), it is not possible for \(\fB\) to be
a model of \(T\) and for \(\fA\) to be a model of \(\neg\varphi\) at the same
time. By Theorem \ref{thm3.1.3} there is a universal sentence \(\psi\) with
\(T\vDash\psi\) and \(\neg\varphi\vDash\neg\psi\). Hence all axioms of
\(T\) follow from
\begin{equation*}
T_\forall=\{\psi\mid T\vDash\psi,\psi\text{ universal}\}
\end{equation*}
\end{enumerate}
\end{proof}

An \(\forall\exists\)-formula is of the form
\begin{equation*}
\forall x_1\dots x_n\psi
\end{equation*}
where \(\psi\) is existential
\begin{lemma}[]
\label{lemma3.1.6}
Suppose \(\varphi\) is an \(\forall\exists\)-sentence, \((\fA_i)_{i\in I}\) is a
directed family of models of \(\varphi\) and \(\fB\) the union of the \(\fA_i\). Then
\(\fB\) is also a model of \(\varphi\).
\end{lemma}

\begin{proof}
Write
\begin{equation*}
\varphi=\forall\bbar{x}\psi(\bbar{x})
\end{equation*}
where \(\psi\) is existential. For any \(\bbar{a}\in B\) there is an \(A_i\)
containing \(\bbar{a}\), clearly \(\psi(\bbar{a})\) holds in \(\fA_i\). As
\(\psi(\bbar{a})\) is existential it must also hold in \(\fB\)
\end{proof}

\begin{definition}[]
We call a theory \(T\) \textbf{inductive} if the union of any directed family of
models of \(T\) is again a model
\end{definition}

\begin{theorem}[]
\label{thm3.1.8}
Let \(T_1\) and \(T_2\) be two theories. Then the following are equivalent
\begin{enumerate}
\item there is an \(\forall\exists\)-sentence which separates \(T_1\) and \(T_2\)
\item No model of \(T_2\) is the union of a chain (or of a directed family) of
models of \(T_1\)
\end{enumerate}
\end{theorem}


\begin{proof}
\(1\to 2\). Assume \(\varphi\) is a \(\forall\exists\)-sentence which separates \(T_1\) from \(T_2\), \((\fA_i)_{i\in I}\) is a
directed family of models of \(\varphi\), by Lemma \ref{lemma3.1.6} \(\fB\) is also a model of \(\varphi\).
Since \(\fB\vDash\varphi\), \(\fB\) cannot be a model of \(T_2\)

\(2\to1\). If (1) is not true, Suppose \(\fA\vDash T_1\) and \(\fB^0\vDash T_2\). Then
\begin{equation*}
\fA\Rightarrow_{\forall\exists}\fB^0
\end{equation*}
Again we have
\begin{equation*}
\fB^0\Rightarrow_{\exists\forall}\fA
\end{equation*}
we have a map
\begin{equation*}
f':\fB^0\to_{\exists\forall}\fA^0
\end{equation*}
where \(\fA^0\equiv\fA\). Since \(\forall\)-sentences are also \(\exists\forall\)-sentences, we thus have a map \(f:\fB^0\to_{\forall}\fA^0\).

Here we need to prove that \(\fB^0\) is isomorphic to a substructure of \(\fA^0\), which is clear
since \(f\) is an embedding.
Then we can assume that \(\fB^0\subseteq\fA^0\) and \(f\) is the inclusion map. Then
\begin{equation*}
\fA_B^0\Rightarrow_\exists\fB_B^0
\end{equation*}
(Here we are talking about existential sentences in the original language.
 If \(\fB^0\vDash\exists\barx\varphi(\barx)\) for some \(\varphi(\barx)\), then \(\fB^0\vDash\varphi(\barb)\). So we can use constants
 \(B\) to talk about existential sentences)
 Applying Lemma \ref{lemma3.1.2} again, we obtain an extension \(\fB_B^1\) of
 \(\fA_B^0\) with \(\fB_B^1\equiv\fB_B^0\), i.e. \(\fB^0\prec\fB^1\). Hence we
 have an infinite chain
\begin{gather*}
\fB^0\subseteq\fA^0\subseteq^1\fB^1\subseteq\fA^1\subseteq\fB^2\subseteq\cdots\\
\fB^0\prec\fB^1\prec\fB^2\prec\cdots\\
\fA^i\equiv\fA
\end{gather*}
Let \(\fB\) be the union of the \(\fA^i\).  Since \(\fB\) is also the union
of the elementary chain of the \(\fB^i\), it is an elementary extension of
\(\fB^0\) and hence a model of \(T_2\). But the \(\fA^i\) are models of
\(T_1\), so (2) does not hold
\end{proof}

\begin{corollary}[]
Let \(T\) be a theory
\begin{enumerate}
\item For each sentence \(\varphi\) the following are equivalent
\begin{enumerate}
\item \(\varphi\) is, modulo \(T\), equivalent to an \(\forall\exists\)-sentence
\item If
\begin{equation*}
\fA^0\subseteq\fA^1\subseteq\cdots
\end{equation*}
and their union \(\fB\) are models of \(T\), then \(\varphi\) holds in \(\fB\) if
it is true in all the \(\fA^i\)
\end{enumerate}
\item \(T\) is inductive iff it can be axiomatised by \(\forall\exists\)-sentences
\end{enumerate}
\end{corollary}

\begin{proof}
\begin{enumerate}
\item Theorem \ref{lemma3.1.6} shows that \(\forall\exists\)-formulas are preserved
by unions of chains. Hence (a)\(\Rightarrow\)(b). For the converse
consider the theories
\begin{equation*}
T_1=T\cup\{\varphi\} \quad\text{ and }\quad T_2=T\cup\{\neg\varphi\}
\end{equation*}
Part (b) says that the union of a chain of models of \(T_1\) cannot be a
model of \(T_2\). By Theorem \ref{thm3.1.8} we can separate \(T_1\) and
\(T_2\) by an \(\forall\exists\)-sentence \(\psi\). Hence
\(T\cup\{\varphi\}\vDash\psi\) and
\(T\cup\{\neg\varphi\}\vDash\neg\psi\)
\item Clearly \(\forall\exists\)-axiomatised theories are inductive. For the
converse assume that \(T\) is inductive and \(\varphi\) is an axiom of \(T\). Ifpp
\(\fB\) is a union of models of \(T\), it cannot be a model of
\(\neg\varphi\). By Theorem \ref{thm3.1.8} there is an
\(\forall\exists\)-sentence \(\psi\) with \(T\vDash\psi\) and
\(\neg\varphi\vDash\neg\psi\). Hence all axioms of \(T\) follows from
\begin{equation*}
T_{\forall\exists}=\{\psi\mid T\vDash\psi,\psi\text{ $\forall\exists$-formula}\}
\end{equation*}
\end{enumerate}
\end{proof}

\begin{exercise}
\label{ex3.1.1}
Let \(X\) be a topological space, \(Y_1\) and \(Y_2\) quasi-compact subsets, and \(\calh\) a set of
clopen subsets. Then the following are equivalent
\begin{enumerate}
\item There is a positive Boolean combination \(B\) of elements from \(\calh\) s.t. \(Y_1\subseteq B\)
and \(Y_2\cap B=\emptyset\)
\item For all \(y_1\in Y_1\) and \(y_2\in Y_2\) there is an \(H\in\calh\) s.t. \(y_1\in H\) and \(y_2\not\in H\)
\end{enumerate}
\end{exercise}

\begin{proof}
\(2\to 1\). Consider an element \(y_1\in Y_1\) and \(\calh_{y_1}\), the set of all elements of \(\calh\)
containing \(y_1\). 2 implies that the intersection of the sets in \(\calh_{y_1}\) is disjoint
from \(Y_2\). So a finite intersection \(h_{y_1}\) of elements of \(\calh_{y_1}\) is disjoint
from \(Y_2\). The \(h_{y_i},y_1\in Y_1\), cover \(Y_1\). So \(Y_1\) is contained in the union \(H\) of
finitely many of the \(h_{y_i}\). Hence \(H\) separates \(Y_1\) from \(Y_2\)
\end{proof}
\subsection{Quantifier elimination}
\label{sec:org20e117a}
\begin{definition}[]
A theory \(T\) has \textbf{quantifier elimination} if every \(L\)-formula
\(\varphi(x_1,\dots,x_n)\) in the theory is equivalent modulo \(T\) to some
quantifier-free formula \(\rho(x_1,\dots,x_n)\)
\end{definition}

For \(n=0\), this means that modulo \(T\) every sentence is equivalent to a quantifier-free
sentence. \uline{If \(L\) has no constants, \(\top\) and \(\bot\) are the only quantifier free sentences.}
\uline{Then \(T\) is either inconsistent or complete.}

It's easy to transform any theory \(T\) into a theory with quantifier
elimination if one is willing to expand the language: just enlarge \(L\) by
adding an \(n\)-place relation symbol \(R_{\varphi}\) for every \(L\)-formula
\(\varphi(x_1,\dots,x_n)\) and \(T\) by adding all axioms
\begin{equation*}
\forall x_1,\dots,x_n(R_\varphi(x_1,\dots,x_n)\leftrightarrow\varphi(x_1,\dots,x_n))
\end{equation*}
The resulting theory, the \textbf{Morleyisation} \(T^m\) of \(T\), has quantifier
elimination

A \textbf{prime structure} of \(T\) is a structure which embeds into all models of
\(T\)

\begin{lemma}[]
\label{lemma3.2.2}
A consistent theory \(T\) with quantifier elimination which possess a prime
structure is complete
\end{lemma}

\begin{proof}
If \(\fM,\fN\vDash T\) and \(\fM\vDash\varphi\) and
\(\fN\vDash\neg\varphi\). Suppose prime structure is \(\fH\), then \(\fH\vDash\varphi\) and \(\fH\vDash\neg\varphi\) since we have quantifier elimination
\end{proof}

\begin{definition}[]
A \textbf{simple existential formula} has the form
\begin{equation*}
\varphi=\exists y\rho
\end{equation*}
for a quantifier-free formula \(\rho\). If \(\rho\) is a conjunction of basic formulas, \(\varphi\)
is called \textbf{primitive existential}
\end{definition}

\begin{lemma}[]
\label{lemma3.2.4}
The theory \(T\) has quantifier elimination iff every primitive existential
formula is, modulo \(T\), equivalent to a quantifier-free formula
\end{lemma}

\begin{proof}
We can write every simple existential formula in the form \(\exists y\bigvee_{i<n}\rho_i\) for \(\rho_i\) which are
conjunctions of basic formulas. This shows that every simple existential formula is equivalent to
a disjunction of primitive existential formulas, namely to \(\bigvee_{i<n}(\exists y\rho_i)\). We can therefore
assume that every simple existential formula is, modulo \(T\), equivalent to a quantifier-free
formula

We are now able to eliminate the quantifiers in arbitrary formulas in prenex normal form
(Exercise \ref{ex1.2.3})
\begin{equation*}
Q_1x_1\dots Q_nx_n\rho
\end{equation*}
if \(Q_n=\exists\), we choose a quantifier-free formula \(\rho_0\) which,
modulo \(T\), is equivalent to \(\exists x_n\rho\) and proceed with the
formula \(Q_1x_1\dots Q_{n-1}x_{n-1}\rho_0\). If \(Q_n=\forall\), we
find a quantifier-free \(\rho_1\) which is, modulo \(T\), equivalent to
\(\exists x_n\neg\rho\) and proceed with \(Q_1x_1\dots Q_{n-1}x_{n-1}\neg\rho_1\)
\end{proof}


\begin{theorem}[]
\label{thm3.2.5}
For a theory \(T\) the following are equivalent
\begin{enumerate}
\item \(T\) has quantifier elimination
\item For all models \(\fM^1\) and \(\fM^2\) of \(T\) with a common substructure
\(\fA\) we have
\begin{equation*}
\fM_A^1\equiv\fM_A^2
\end{equation*}
\item For all models \(\fM^1\) and \(\fM^2\) of \(T\) with a common substructure
\(\fA\) and for all primitive existential formulas \(\varphi(x_1,\dots,x_n)\)
and parameter \(a_1,\dots,a_n\) from \(A\) we have
\begin{equation*}
\fM^1\vDash\varphi(a_1,\dots,a_n)\Rightarrow\fM^2\vDash\varphi(a_1,\dots,a_n)
\end{equation*}
(this is exactly the equivalence relation)
\end{enumerate}

If \(L\) has no constants, \(\fA\) is allowed to be the empty ``structure''
\end{theorem}

\begin{proof}
\(1\to 2\). Let \(\varphi(\bara)\) be an \(L(A)\)-sentence which holds in \(\fM^1\). Choose a
quantifier-free \(\rho(\barx)\) which is, modulo \(T\), equivalent to \(\varphi(\barx)\). Then
\begin{center}
\begin{tabular}{llllllll}
\(\fM^1\) & \(\vDash\) & \(\varphi(\bara)\) & \(\Rightarrow\) & \(\fM^1\) & \(\vDash\) & \(\rho(\bara)\) & \\
 &  &  & \(\Rightarrow\) & \(\fA\) & \(\vDash\) & \(\rho(\bara))\) & \(\Rightarrow\)\\
\(\fM^2\) & \(\vDash\) & \(\rho(\bara))\) & \(\Rightarrow\) & \(\fM^2\) & \(\vDash\) & \(\varphi(\bara)\) & \\
\end{tabular}
\end{center}

\(3\to1\). Let \(\varphi(\bbar{x})\) be a primitive existential formula. In order
to show that \(\varphi(\bbar{x})\) is equivalent, modulo \(T\), to a
quantifier-free formula \(\rho(\bbar{x})\) we extend \(L\) by an \(n\)-tuple
\(\bbar{c}\) of new constants \(c_1,\dots,c_n\). \textbf{We have to show that we can}
\textbf{separate \(T\cup\{\varphi(\bbar{c})\}\) and \(T\cup\{\neg\varphi(\bbar{c})\}\) by a}
\textbf{quantifier free sentence \(\rho(\bbar{c})\)}. Then \(T\vDash\varphi(\barc)\to\rho(\barc)\)
and \(T\vDash\neg\varphi(\barc)\to\neg\rho(\barc)\). Hence \(T\vDash\varphi(\barc)\leftrightarrow\rho(\barc)\).

We apply the Separation Lemma
(\(\calh\) hear is the set of quantifier-free sentence). Let
\(\fM^1\) and \(\fM^2\) be two models of \(T\) with two distinguished
\(n\)-tuples \(\bbar{a}^1\) and \(\bbar{a}^2\). Suppose that
\((\fM^1,\bbar{a}^1)\) and \((\fM^2,\bbar{a}^2)\) satisfy the same
quantifier-free \(L(\bbar{c})\)-sentences. We have to show that
\begin{equation*}
\fM^1\vDash\varphi(\bbar{a}^1)\Rightarrow
\fM^2\vDash\varphi(\bbar{a}^2)\tag{\star}
\end{equation*}
which says that if \(T\)'s model \(\fA_1,\fA_2\) satisfies the same quantifier-free sentences, then
\(\fM^1\Rightarrow_{\exists}\fM^2\). If \(\fM^1\vDash T\cup\{\varphi(\barc)\}\) and \(\fM^2\vDash T\cup\{\neg\varphi(\barc)\}\) and satisfy the same
quantifier-free \(L(\barc)\) sentence, then \(\fM^1\subseteq \fM^2\) , a contradiction.
Thus we finish the proof

Consider the substructure \(\fA^i=\la\bbar{a}^i\ra^{\fM^i}\), generated by
\(\bbar{a}^i\). If we can show that there is an isomorphism
\begin{equation*}
f:\fA^1\to\fA^2
\end{equation*}
taking \(\bbar{a}\) to \(\bbar{a}\), we may assume that \(\fA^1=\fA^2=\fA\)
and \(\bbar{a}^1=\bbar{a}^2=\bbar{a}\). Then \(\star\) follows directly from 3.

Every element of \(\fA^1\) has the form \(t^{\fM^1}[\bbar{a}^1]\) for an
\(L\)-term \(t(\bbar{x})\). The isomorphism \(f\)to be constructed must
satisfy
\begin{equation*}
f(t^{\fM^1}[\bbar{a}^1])=t^{\fM^2}[\bbar{a}^2]
\end{equation*}
We define \(f\) by this equation and have to check that \(f\) is well defined
and injective. Assume
\begin{equation*}
s^{\fM^1}[\bbar{a}^1]=t^{\fM^1}[\bbar{af^1}]
\end{equation*}
Then \(\fM^1,\bbar{a}^1\vDash s(\bbar{c})\dot{=}t(\bbar{c})\), and by our
assumption, \(\fM^1\) and \(\fM^2\) satisfy the same quantifier-free \(L(\barc)\)-sentence,  it also
holds in \((\fM^2,\bbar{a}^2)\), which means
\begin{equation*}
s^{\fM^2}[\bbar{a}^2]=t^{\fM^2}[\bbar{a}^2]
\end{equation*}
Swapping the two sides yields injectivity.

Surjectivity is clear. It remains to show that \(f\) commutes with the
interpretation of the relation symbols. Now
\begin{equation*}
\fM^1\vDash R\left[t_1^{\fM^1}[\bbar{a}^1],\dots,t_m^{\fM^1}[\bbar{a}^1]\right]
\end{equation*}
is equivalent to \((\fM^1,\bbar{a}^1)\vDash R(t_1(\bbar{c}),\dots,t_m(\bbar{c}))\), which is equivalent to
\((\fM^2,\bbar{a}^2)\vDash R(t_1(\bbar{c}),\dots,t_m(\bbar{c}))\), which in turn is equivalent to
\begin{equation*}
\fM^2\vDash R\left[t_1^{\fM^2}[\bbar{a}^2],\dots,t_m^{\fM^2}[\bbar{a}^2]\right]
\end{equation*}
\end{proof}

Note that (2) of Theorem \ref{thm3.2.5} is saying that \(T\) is \textbf{substructure
complete}; i.e., for any model \(\fM\vDash T\) and substructure
\(\fA\subseteq\fM\) the theory \(T\cup\Diag(\fA)\) is complete

\begin{definition}[]
We call \(T\) \textbf{model complete} if for all models \(\fM^1\) and \(\fM^2\) of
\(T\)
\begin{equation*}
\fM^1\subseteq\fM^2\Rightarrow\fM^1\prec\fM^2
\end{equation*}
\end{definition}

\(T\) is model complete iff for any \(\fM\vDash T\) the theory
\(T\cup\Diag(\fM)\) is complete

Note that if \(\fM_1\vDash\Diag(\fM)\), then there is an embedding \(h:\fM\to\fM_1\) and \(\fM_1\) is isomorphic to
an extension \(\fM_1'\) of \(\fM\). Then we have \(\fM\subseteq\fM_1'\).

So here we are actually saying that all embeddings are elementary

\begin{lemma}[Robinson's Test]
\label{lemma3.2.7}
Let \(T\) be a theory. Then the following are equivalent
\begin{enumerate}
\item \(T\) is model complete
\item For all models \(\fM^1\subseteq\fM^2\) of \(T\) and all existential
sentences \(\varphi\) from \(L(M^1)\)
\begin{equation*}
\fM^2\vDash\varphi\Rightarrow\fM^1\vDash\varphi
\end{equation*}
\item Each formula is, modulo \(T\), equivalent to a universal formula
\end{enumerate}
\end{lemma}

\begin{proof}
\(1\leftrightarrow3\). Corollary \ref{cor3.1.5}

(2) and Corollary \ref{cor3.1.5} shows that all existential sentences are, modulo \(T\), equivalent
to a universal sentence. Then by induction we can show 3. \href{https://math.stackexchange.com/questions/321737/proof-of-robinsons-test/2050990}{(Details)}
\end{proof}

If \(\fM^1\subseteq\fM^2\) satisfies (2), we call \(\fM^1\) \textbf{existentially
closed} in \(\fM^2\). We denote this by
\begin{equation*}
\fM^1\prec_1\fM^2
\end{equation*}

\begin{definition}[]
Let \(T\) be a theory. A theory \(T^*\) is a \textbf{model companion} of \(T\) if the
following three conditions are satisfied
\begin{enumerate}
\item Each model of \(T\) can be extended to a model of \(T^*\)
\item Each model of \(T^*\) can be extended to a model of \(T\)
\item \(T^*\) is model complete
\end{enumerate}
\end{definition}

\begin{theorem}[]
\label{thm3.2.9}
A theory \(T\) has, up to equivalence, at most one model companion \(T^*\)
\end{theorem}

\begin{proof}
If \(T^+\) is another model companion of \(T\), every model of \(T^+\) is
contained in a model of \(T^*\) and conversely. Let \(\fA_0\vDash T^+\) .
Then \(\fA_0\) can be embedded in a model \(\fB_0\) of \(T^*\). In turn
\(\fB_0\) is contained in a model \(\fA_1\) of \(T^+\). In this way we find
two elementary chains \((\fA_i)\) and \((\fB_i)\), which have a common union
\(\fC\). Then \(\fA_0\prec\fC\) and \(\fB_0\prec\fC\) implies
\(\fA_0\equiv\fB_0\) since \(T\) are all sentences. Thus \(\fA_0\) is a model of \(T^*\)
\end{proof}
\subsubsection{Existentially closed structures and the Kaiser hull}
\label{sec:org51be538}
Let \(T\) be an \(L\)-theory. It follows from \ref{thm3.1.3} that the models
of \(T_\forall=\{\varphi\mid T\vDash\varphi\text{ where $\varphi$ is universal}\}\) are the substructures of models of \(T\). The conditions
(1) and (2) in the definition of ``model companion'' can therefore be
expressed as
\begin{equation*}
T_{\forall}=T_{\forall}^*
\end{equation*}
(1 and 2 says \(\Mod(T_\forall)=\Mod(T_\forall^*)\))
Hence the model companion of a theory \(T\) depends only on \(T_{\forall}\).


\begin{definition}[]
An \(L\)-structure \(\fA\) is called \textbf{\(T\)-existentiallay closed} (or
\textbf{\(T\)-ec}) if
\begin{enumerate}
\item \(\fA\) can be embedded in a model of \(T\)
\item \(\fA\) is existentially closed in every extension which is a model of \(T\)
\end{enumerate}
\end{definition}

A structure \(\fA\) is \(T\)-ec exactly if it is \(T_{\forall}\)-ec. Since
every model of \(\fB\) of \(T_{\forall}\) can be embedded in a model \(\fM\)
of \(T\) and \(\fA\subseteq\fB\subseteq\fM\) and \(\fA\prec_1\fM\) implies \(\fA\prec_1\fB\)

\begin{lemma}[]
\label{lemma3.2.11}
Every model of a theory \(T\) can be embedded in a \(T\)-ec structure
\end{lemma}

\begin{proof}
Let \(\fA\) be a model of \(T_{\forall}\). We choose an enumeration
\((\varphi_\alpha)_{\alpha<\kappa}\) of all existential \(L(A)\)-sentences and
construct an ascending chain \((\fA_\alpha)_{\alpha\le\kappa}\) of models of
\(T_{\forall}\). We begin with \(\fA_0=\fA\). Let \(\fA_\alpha\) be
constructed. If \(\varphi_\alpha\) holds in an extension of \(\fA_\alpha\)
which is a model of \(T\) we let \(\fA_{\alpha+1}\) be such a model.
Otherwise we set \(\fA_{\alpha+1}=\fA_{\alpha}\). For limit ordinals \(\lambda\) we define
\(\fA_\lambda\) to be the union of all \(\fA_\alpha\). \(\fA_\lambda\) is
again a model of \(T_{\forall}\)

The structure \(\fA^1=\fA_\kappa\) has the following property: every existential \(L(A)\)-sentence which
holds in an extension of \(\fA^1\) that is a model of \(T\) holds in \(\fA^1\). Now in the same
manner, we construct \(\fA^2\) from \(\fA^1\), etc. The union \(\fM\) of the chain \(\fA^0\subseteq\fA^1\subseteq\fA^2\subseteq\dots\) is the
desired \(T\)-ec structure
\end{proof}

Every elementary substructure \(\fN\) of a \(T\)-ec structure \(\fM\) is
again \(T\)-ec: Let \(\fN\subseteq\fA\) be a model of \(T\). Since
\(\fM_N\Rightarrow_{\exists}\fA_N\), there is an embedding of \(\fM\) in an
elementary extension \(\fB\) of \(\fA\) which is the identity on \(N\).
Since \(\fM\) is existentially closed in \(\fB\), it follows that \(\fN\) is
existentially closed in \(\fB\) and therefore also in \(\fA\)

\begin{center}\begin{tikzcd}
&\fB&\\
\fA\arrow[ur,"\prec"]&&\fM\arrow[ul,"\prec_1"']\\
&\fN\arrow[ul,"\prec_1"]\arrow[ur,"\prec"']&
\end{tikzcd}\end{center}

\begin{lemma}[]
Let \(T\) be a theory. Then there is a biggest inductive theory \(T^{\KH}\)
with \(T_{\forall}=T_{\forall}^{\KH}\). We call \(T^{\KH}\) the \textbf{Kaiser hull}
of \(T\)
\end{lemma}

\begin{proof}
Let \(T^1\) and \(T^2\) be two inductive theories with
\(T_{\forall}^1=T_{\forall}^2=T_{\forall}\). We have to show
that \((T^1\cup T^2)_\forall=T_\forall\). Note that for every model \(\fA\vDash T^1\) and \(\fB\vDash T^2\) we have
\(\fA\Rightarrow_\forall\fB\) and vice versa. Then we have the embeddings just like model companions.
Let \(\fM\) be a model
of \(T\), as in the proof of \ref{thm3.2.9} we extend \(\fM\) by a chain
\(\fA_0\subseteq\fB_0\subseteq\fA_1\subseteq\fB_1\subseteq\cdots\) of models
of \(T^1\) and \(T^2\). The union of this chain is a model of \(T^1\cup T^2\)
\end{proof}

\begin{lemma}[]
\label{lemma3.2.13}
The Kaiser hull \(T^{KH}\) is the \(\forall\exists\)-part of the theory of
all \(T\)-ec structures
\end{lemma}

\begin{proof}
Let \(T^*\) be the \(\forall\exists\)-part of the theory of all \(T\)-ec
structures. Since \(T\)-ec structures are models of \(T_{\forall}\), we have
\(T_\forall\subseteq T^*_\forall\). It follows from \ref{lemma3.2.11} that
\(T_\forall^*\subseteq T_\forall\). Hence \(T^*\) is contained in the Kaiser Hull.

It remains to show that every \(T\)-ec structure \(\fM\) is a model of the Kaiser hull. Choose a
model \(\fN\) of \(T^{KH}\) which contains \(\fM\). Then \(\fM\prec_1\fN\). This implies \(\fN\Rightarrow_{\forall\exists}\fM\) and
therefore \(\fM\vDash T^{KH}\)
\end{proof}

This implies that \(T\)-ec strctures are models of \(T_{\forall\exists}\)

\begin{theorem}[]
For any theory \(T\) the following are equivalent
\begin{enumerate}
\item \(T\) has a model companion \(T^*\)
\item All models of \(K^{\KH}\) are \(T\)-ec
\item The \(T\)-ec structures form an elementary class.
\end{enumerate}


If \(T^*\) exists, we have
\begin{equation*}
T^*=T^{\KH}=\text{ theory of all $T$-ec structures}
\end{equation*}
\end{theorem}

\begin{proof}
\(1\to 2\): let \(T^*\) be the model companion of \(T\). As a model complete theory

\(3\to 1\): Assume that the \(T\)-ec structures are exactly the models of the theory \(T^+\). By
\ref{lemma3.2.11} we have \(T_\forall=T_\forall^+\). Criterion \ref{lemma3.2.7} implies that \(T^+\) is model
complete. So \(T^+\) is the model companion of \(T\).
\end{proof}


\begin{exercise}
\label{ex3.2.1}
Let \(L\) be the language containing a unary function \(f\) and a binary
relation symbol \(R\) and consider the \(L\)-theory \(T=\{\forall x\forall
    y(R(x,y)\to (R(x,f(y))))\}\). Showing the follow
\begin{enumerate}
\item For any \(T\)-structure \(\fM\) and \(a,b\in M\) with
\(b\not\in\{a,f^{\fM}(a),(f^{\fM})^2(a),\dots\}\) we have
\(\fM\vDash\exists z(R(z,a)\wedge\neg R(z,b))\)
\item Let \(\fM\) be a model of \(T\) and \(a\) an element of \(M\) s.t.
\(\{a,f^{\fM}(a),(f^{\fM})^2(a),\dots\}\) is infinite. Then in an
elementary extension \(\fM'\) there is an element \(b\) with
\(\fM'\vDash\forall z(R(z,a)\to R(z,b))\)
\item The class of \(T\)-ec structures is not elementary, so \(T\) does not
have a model companion
\end{enumerate}
\end{exercise}

\begin{exercise}
\label{ex3.2.3}
A theory \(T\) with quantifier elimination is axiomatisable by sentences of
the form
\begin{equation*}
\forall x_1\dots x_n\psi
\end{equation*}
where \(\psi\) is primitive existential formula
\end{exercise}
\subsection{Examples}
\label{sec:org4303015}
\textbf{Infinite sets}. The models of the theory  \(\Infset\) of \textbf{infinite sets} are all
infinite sets without additional structure. The language \(L_{\emptyset}\) is
empty, the axioms are (for \(n=1,2,\dots\))
\begin{itemize}
\item \(\exists x_0\dots x_{n-1}\bigwedge_{i<j<n}\neg x_i\dot{=}x_j\)
\end{itemize}
\begin{theorem}
The theory \(\Infset\) of infinite sets has quantifier elimination and is complete
\end{theorem}

\begin{proof}
Since the language is empty, the only basic formula is \(x_i=x_j\) and
\(\neg(x_i=x_j)\). By Lemma \ref{lemma3.2.4} we only need to consider primitive
existential formulas. Then every sentence is actually saying there is \(n\) different elements.
Then for any \(\fM^1,\fM^2\vDash\Infset\), they have a common substructure \(\fA\) with \(\omega\) different elements.
Visibly, \(\fM^1_A\equiv\fM^2_A\)
\end{proof}

\textbf{Dense linear orderings}.
\begin{align*}
&\forall a,b(a\le b\wedge b\le a\to a\dot{=}b)\\
&\forall a,b,c(a\le b\wedge b\le c\to a\le c)\\
&\forall a,b(a\le b\vee b\le a)\\
&\forall a,b\exists c(a< b\to a< c< b)
\end{align*}
\begin{theorem}[]
\(\DLO\) has quantifier elimination
\end{theorem}

\begin{proof}
Let \(A\) be a finite common substructure of the two models \(O_1\) and
\(O_2\). We choose an ascending enumeration \(A=\{a_1,\dots,a_n\}\). Let
\(\exists y\rho (y)\) be a simple existential \(L(A)\)-sentence, which is
true in \(O_1\) and assume \(O_1\vDash\rho(b_1)\). We want to extend the
order preserving map \(a_i\mapsto a_i\) to an order preserving map
\(A\cup\{b_1\}\to O_2\). For this we have an image \(b_2\) of \(b_1\). There
are four cases
\begin{enumerate}
\item \(b_1\in A\), we set \(b_2=b_1\)
\item \(b_1\in(a_i,a_{i+1})\). We choose \(b_2\) in \(O_2\) with the same property
\item \(b_1\) is smaller than all elements of \(A\). We choose a \(b_2\in O_2\)
of the same kind
\item \(b_1\) is bigger than all \(a_i\). Choose \(b_2\) in the same manner
\end{enumerate}


This defines an isomorphism \(A\cup\{b_1\}\to A\cup\{b_2\}\), which show that \(O_2\vDash\rho(b_2)\)
\end{proof}


\textbf{Modules}. Let \(R\) be a (possibly non-commutative) ring with 1. An
\(R\)-module
\begin{equation*}
\fM=(,0,+,-,r)_{r\in R}
\end{equation*}
is an abelian group \((M,0,+,-)\) together with operations \(r:M\to M\) for
every ring element \(r\in R\). We formulate the axioms in the language
\(L_{Mod}(R)=L_{AbG}\cup\{r\mid r\in R\}\). The theory \(\sfMod(R)\) of
\(R\)-modules consists of
\begin{align*}
&\AbG\\
&\forall x,y\; r(x+y)\dot{=}rx+ry\\
&\forall x\;(r+s)x\dot{=}rx+sx\\
&\forall x\;(rs)x\dot{=}r(sx)\\
&\forall x\;1x\dot{=}x
\end{align*}
for all \(r,s\in R\). Then \(\Infset\cup\sfMod(R)\) is the theory of all
infinite \(R\)-modules

A module over fields is a vector space

\begin{theorem}[]
Let \(K\) be a field. Then the theory of all infinite \(K\)-vector spaces has
quantifier elimination and is complete
\end{theorem}

\begin{proof}
Let \(A\) be a common finitely generated substructure (i.e., a subspace) of
the two infinite \(K\)-vector spaces \(V_1\) and \(V_2\). Let \(\exists y\rho(y)\) be a simple
existential \(L(A)\)-sentence which holds in \(V_1\).
Choose a \(b_1\) from \(V_1\) which satisfies \(\rho(y)\). If \(b_1\) belongs to
\(A\), we finished. If not, we choose a \(b_2\in V_2\setminus A\). Possibly
we have to replace \(V_2\) by an elementary extension. The vector spaces
\(A+Kb_1\) and \(A+Kb_2\) are isomorphic by an isomophism which maps \(b_1\)
to \(b_2\) and fixes \(A\) elementwise. Hence \(V_2\vDash\rho(b_2)\)

\textbf{The theory is complete since a quantifier-free sentence is true in a vector
space iff it is true in the zero-vector space.} ?
\end{proof}

\begin{definition}[]
An \textbf{equation} is an \(L_{Mod}(R)\)-formula \(\gamma(\bbar{x})\) of the form
\begin{equation*}
r_1x_1+\dots+r_mx_m=0
\end{equation*}
A \textbf{positive primitive} formula (\textbf{pp}-formula) is of the form
\begin{equation*}
\exists\bbar{y}(\gamma_1\wedge\dots\wedge\gamma_n)
\end{equation*}
where the \(\gamma_i(\bbar{xy})\) are equations
\end{definition}

\begin{theorem}[]
\label{thm3.3.5}
For every ring \(R\) and any \(R\)-module \(M\), every \(L_{Mod}(R)\)-formula
is equivalent (modulo the theory of \(M\)) to a Boolean combination of
positive primitive formulas
\end{theorem}

\begin{remark}
\begin{enumerate}
\item We assume the class of positive primitive formulas to be closed under \(\wedge\)
\item A pp-formula \(\varphi(x_1,\dots,x_n)\) defines a subgroup \(\varphi(M^n)\) of \(M^n\):
\begin{equation*}
M\vDash\varphi(0) \quad\text{ and }\quad M\vDash\varphi(x)\wedge\varphi(y)\to\varphi(x-y)
\end{equation*}
\end{enumerate}
\end{remark}

\begin{lemma}[]
\label{lemma3.3.7}
Let \(\varphi(x,y)\) be a pp-formula and \(a\in M\). Then \(\varphi(M,a)\) is empty or a coset of \(\varphi(M,0)\)
\end{lemma}

\begin{proof}
\(M\vDash\varphi(x,a)\to(\varphi(y,0)\leftrightarrow\varphi(x+y,a))\)

Or, if \(x,y\in \varphi(M,a)\), then \(\varphi(x-y,0)\).
\end{proof}

\begin{corollary}[]
Let \(a,b\in M\), \(\varphi(x,y)\) a pp-formula. Then (in \(M\)) \(\varphi(x,a)\) and \(\varphi(x,b)\) are equivalent
or contradictory
\end{corollary}

\begin{lemma}[B. H. Neumann]
\label{lemma3.3.9}
Let \(H_i\) denote subgroups of some abelian group. If \(H_0+a_0\subseteq\bigcup_{i=1}^nH_i+a_i\)
and \(H_0/(H_0\cap H_i)\) is infinite for \(i>k\), then \(H_0+a_0\subseteq\bigcup_{i=1}^kH_i+a_i\)
\end{lemma}

\begin{lemma}[]
Let \(A_i,i\le k\), be any sets. If \(A_0\) is finite, then \(A_0\subseteq\bigcup_{i=1}^k A_i\) iff
\begin{equation*}
\sum_{\Delta\subseteq\{1,\dots,k\}}(-1)^{\abs{\Delta}}\abs{A_0\cap\bigcap_{i\in\Delta}A_i}=0
\end{equation*}
\end{lemma}

\textbf{Algebraically closed fields}.
\begin{theorem}[Tarski]
The theory \(\sfACF\) of algebraically closed fields has quantifier elimination
\end{theorem}

\begin{proof}
Let \(K_1\) and \(K_2\) be two algebraically closed fields and \(R\) a common
subring. Let \(\exists y\rho(y)\) be a simple existential sentence with
parameters in \(R\) which hold in \(K_1\). We have to show that \(\exists
   y\rho(y)\) is also true in \(K_2\).

Let \(F_1\) and \(F_2\) be the quotient fields of \(R\) in \(K_1\) and
\(K_2\), and let \(f:F_1\to F_2\) be an isomorphism which is the identity on
\(R\). Then \(f\) extends to an isomorphism \(g:G_1\to G_2\) between the
relative algebraic closures \(G_i\) of \(F_i\) in \(K_i\). Choose an element \(b_1\in K_1\) which
satisfies \(\rho(y)\)

\begin{center}\begin{tikzcd}[column sep=40pt]
K_1&K_2\\
G_1\ar[u,dash,"G_1(b_1)" description]\ar[r,"g"]&G_2\ar[u,dash]\\
F_1\ar[u,dash]\ar[r,"f"]&F_2\ar[u,dash]\\
R\ar[u,dash]\ar[r,"id"]&R\ar[u,dash]
\end{tikzcd}\end{center}
\end{proof}

\begin{corollary}[]
\(\ACF\) is model complete
\end{corollary}

\(\ACF\) is not complete: for prime numbers \(p\) let
\begin{equation*}
\ACF_p=\ACF\cup\{p\cdot 1\dot=0\}
\end{equation*}
be the theory of algebraically closed fields of characteristic \(p\) and
\begin{equation*}
\ACF_0=\ACF\cup\{\neg n\cdot 1\dot=0\mid n=1,2,\dots\}
\end{equation*}
be the theory of algebraically closed fields of characteristic 0.

\begin{corollary}[]
The theories \(\ACF_p\)  and \(\ACF_0\) are complete
\end{corollary}

\begin{proof}
This follows from Lemma \ref{lemma3.2.2} since the prime fields are prime structures for these theories
\end{proof}

\begin{corollary}[Hilbert's Nullstellensatz]
Let \(K\) be a field. Then any proper ideal \(I\) in \(K[X_1,\dots,X_n]\) has a zero in the algebraic
closure \(\acl(K)\)
\end{corollary}

\begin{proof}
As a proper ideal, \(I\) is contained in a maximal ideal \(P\). Then \(L=K[X_1,\dots,X_n]/P\) is an
extension field of \(K\) in which the cosets of the \(X_i\) are a zero of \(I\).
\end{proof}


\section{Countable Models}
\label{sec:org3eb3ee8}
\subsection{The omitting types theorem}
\label{sec:org5ffbf93}
\begin{definition}[]
Let \(T\) be an \(L\)-theory and \(\Sigma(x)\) a set of \(L\)-formulas. A model
\(\fA\) of \(T\) not realizing \(\Sigma(x)\) is said to \textbf{omit} \(\Sigma(x)\). A
formula \(\varphi(x)\) \textbf{isolates} \(\Sigma(x)\) if
\begin{enumerate}
\item \(\varphi(x)\) is consistent with \(T\)
\item \(T\vDash\forall x(\varphi(x)\to\sigma(x))\) for all \(\sigma(x)\in\Sigma(x)\)
\end{enumerate}
\end{definition}

A set of formulas is often called a \textbf{partial type}.

\begin{theorem}[Omitting Types]
If \(T\) is countable and consistent and if \(\Sigma(x)\) is not isolated in
\(T\), then \(T\) has a model which omits \(\Sigma(x)\)
\end{theorem}

If \(\Sigma(x)\) is isolated by \(\varphi(x)\) and \(\fA\) is a model of \(T\), then
\(\Sigma(x)\) is realised in \(\fA\) by all realisations \(\varphi(x)\). Therefore the
converse of the theorem is true for \textbf{complete} theories \(T\): if \(\Sigma(x)\) is
isolated in \(T\), then it is realised in every model of \(T\) 

\begin{proof}
We choose a countable set \(C\) of new constants and extend \(T\) to a theory
\(T^*\) with the following properties
\begin{enumerate}
\item \(T^*\) is a Henkin theory: for all \(L(C)\)-formulas \(\psi(x)\) there
exists a constant \(c\in C\) with \(\exists x\psi(x)\to\psi(c)\in T^*\)
\item for all \(c\in C\) there is a \(\sigma(x)\in\Sigma(x)\) with \(\neg\sigma(c)\in
      T^*\)
\end{enumerate}


We construct \(T^*\) inductively as the union of an ascending chain
\begin{equation*}
T=T_0\subseteq T_1\subseteq T_1\subseteq\dots
\end{equation*}
of consistent extensions of \(T\) by finitely many axioms from \(L(C)\), in
each step making an instance of (1) or (2) true.

Enumerate \(C=\{c_i\mid i<\omega\}\) and let \(\{\psi_i(x)\mid i<\omega\}\)
be an enumeration of the \(L(C)\)-formulas

Assume that \(T_{2i}\) is the already constructed. Choose some \(c\in C\)
which doesn't occur in \(T_{2i}\cup\{\psi_i(x)\}\) and set
\(T_{2i+1}=T_{2i}\cup\{\exists x\psi_i(x)\to\psi_i(c)\}\).

Up to equivalence \(T_{2i+1}\) has the form \(T\cup\{\delta(c_i,\bbar{c})\}\) for
an \(L\)-formula \(\delta(x,\bbar{y})\) and a tuple \(\bbar{c}\in C\) which
doesn't contain \(c_i\). Since \(\exists\bbar{y}\delta(x,\bbar{y})\) doesn't
isolate \(\Sigma(x)\), for some \(\sigma\in\Sigma\) the formula
\(\exists\bbar{y}\delta(x,\bbar{y})\wedge\neg\sigma(x)\) is consistent with \(T\).
Thus \(T_{2i+2}=T_{2i+1}\cup\{\neg\sigma(c_i)\}\) is consistent

Take a model \((\fA',a_c)_{c\in C}\) of \(T^*\). Since \(T^*\) is a Henkin
theory, Tarski's Test \ref{thm2.1.2} shows that \(A=\{a_c\mid c\in C\}\) is the
universe of an elementary substructure \(\fA\) (Lemma \ref{lemma2.2.3}). By
property (2), \(\Sigma(x)\) is omitted in \(\fA\)
\end{proof}

\begin{corollary}[]
\label{cor4.1.3}
\label{ex4.1.1}
Let \(T\) be countable and consistent and let
\begin{equation*}
\Sigma_0(x_0,\dots,x_{n_0}),\Sigma_1(x_1,\dots,x_{n_1}),\dots
\end{equation*}
be a sequence of partial types. If all \(\Sigma_i\) are not isolated, then
\(T\) has a model which omits all \(\Sigma_i\)
\end{corollary}

\begin{proof}

If \(\Sigma_0(x),\Sigma_1(x),\dots\). Then
\(T_{2i+2}=T_{2i+1}\cup\{\neg\sigma_m(c_{mn})\}\)

If \(\Sigma(x_1,\dots,x_n)\), then
\(T_{2i+1}=T_{2i}\cup\{\exists\bbar{x}\psi_i(\bbar{x})\to\psi_i(\bbar{c})\}\).

Combine the two case
\end{proof}
\subsection{The space of types}
\label{sec:org05645eb}
Fix a theory \(T\). An \textbf{\(n\)-type} is a maximal set of formulas
\(p(x_1,\dots,x_n)\) consistent with \(T\). We denote by \(S_n(T)\) the set
of all \(n\)-types of \(T\). We also write \(S(T)\) for \(S_1(T)\).
\(S_0(T)\) is all complete extensions of \(T\)

If \(B\) is a subset of an \(L\)-structure \(\fA\), we recover
\(S_n^{\fA}(B)\) as \(S_n(\Th(\fA_B))\). In particular, if \(T\) is complete
and \(\fA\) is any model of \(T\), we have \(S^{\fA}(\emptyset)=S(T)\)

For any \(L\)-formula \(\varphi(x_1,\dots,x_n)\), let \([\varphi]\) denote the set of all
types containing \(\varphi\).

\begin{lemma}[]
\begin{enumerate}
\item \([\varphi]=[\psi]\) iff \(\varphi\) and \(\psi\) are equivalent modulo \(T\)
\item The sets \([\varphi]\) are closed under Boolean operations. In fact
\([\varphi]\cap[\psi]=[\varphi\wedge\psi]\), \([\varphi]\cup[\psi]=[\varphi\vee\psi]\),
\(S_n(T)\setminus[\varphi]=[\neg\varphi]\), \(S_n(T)=[\top]\) and \(\emptyset=[\bot]\)
\end{enumerate}
\end{lemma}

It follows that the collection of sets of the form \([\varphi]\) is closed under
finite intersection and includes \(S_n(T)\). So these sets form a basis of a
topology on \(S_n(T)\)

\begin{lemma}[]
The space \(S_n(T)\) is 0-dimensional and compact
\end{lemma}

\begin{proof}
Being 0-dimensional means having a basis of clopen sets. Our basic open sets
are clopen since their complements are also basic open

If \(p\) and \(q\) are two different types, there is a formula \(\varphi\) contained in
\(p\) but not in \(q\). It follows that \([\varphi]\) and \([\neg\varphi]\) are
open sets which separate \(p\) and \(q\). This shows that \(S_n(T)\) is
Hausdorff

To prove compactness, we need to show that any collection of closed subsets
of \(X\) with the finite intersection property has nonempty intersection.
Could check  \href{http://www.msc.uky.edu/droyster/courses/fall99/math4181/classnotes/notes5.pdf}{this}

Consider a family \([\varphi_i]\) (\(i\in I\)), with the finite intersection property.This
means that \(\varphi_{i_i}\wedge\dots\wedge\varphi_{i_k}\) are consistent
with \(T\). So Corollary \ref{cor2.2.5} \(\{\varphi_i\mid i\in I\}\) is
consistent with \(T\) and can be extended to a type \(p\), which then belongs
to all \([\varphi_i]\).  
\end{proof}

\begin{lemma}[]
All clopen subsets of \(S_n(T)\) has the form \([\varphi]\)
\end{lemma}

\begin{proof}
It follows from Exercise \ref{ex3.1.1} that we can separate any two disjoint
closed subsets of \(S_n(T)\) by a basic open set.
\end{proof}

The Stone duality theorem asserts that the map
\begin{equation*}
X\mapsto\{C\mid C\text{ clopen subset of }X\}
\end{equation*}
yields an equivalence between the category of 0-dimensional compact spaces
and the category of Boolean algebras. The inverse map assigns to every
Boolean algebra to its \textbf{Stone space}

\begin{definition}[]
A map \(f\) from a subset of a structure \(\fA\) to a structure \(\fB\) is
\textbf{elementary} if it preserves the truth of formulas; i.e., \(f:A_0\to B\) is
elementary if for every formula \(\varphi(x_1,\dots,x_n)\) and \(\bbar{a}\in A_0\)
we have
\begin{equation*}
\fA\vDash\varphi(\bbar{a})\Rightarrow\fB\vDash\varphi(f(\bbar{a}))
\end{equation*}
\end{definition}

\begin{lemma}[]
Let \(\fA\) and \(\fB\) be \(L\)-structures, \(A_0\) and \(B_0\) subsets of
\(A\) and \(B\), respectively. Any elementary map \(A_0\to B_0\) induces a
continuous surjective map \(S_n(B_0)\to S_n(A_0)\)
\end{lemma}

\begin{proof}
If \(q(\bbar{x})\in S_n(B_0)\), we define
\begin{equation*}
S(f)(q)=\{\varphi(x_1,\dots,x_n,\bbar{a})\mid\bbar{a}\in A_0,\varphi(x_1,\dots,x_n,f(\bbar{a}))\in q(\bbar{x})\}
\end{equation*}
If \(\varphi(\bbar{x},f(\bbar{a}))\not\in q(\bbar{x})\), then
\(\fB\not\vDash\varphi(\bbar{x},\bbar{a})\). Therefore \(\fA\not\vDash\varphi(\bbar{x},\bbar{a})\).
\(S(f)\) defines a map from \(S_n(B_0)\) to \(S_n(A_0)\). Moreover, it is
surjective since
\(\{\varphi(x_1,\dots,x_n,f(\bbar{a}))\mid\varphi(x_1,\dots,x_n,a)\in p\}\) is
finitely satisfiable for all \(p\in S_n(A_0)\). And \(S(f)\) is continuous
since \([\varphi[x_1,\dots,x_n,f(\bbar{a})]]\) is the preimage of
\([\varphi(x_1,\dots,x_n,\bbar{a})]\) under \(S(f)\)
\end{proof}

There are two main cases
\begin{enumerate}
\item An elementary bijection \(f:A_0\to B_0\) defines a homeomorphism
\(S_n(A_0)\to S_n(B_0)\). We write \(f(p)\) for the image of \(p\)
\item If \(\fA=\fB\) and \(A_0\subseteq B_0\), the inclusion map induces the
\textbf{restriction} \(S_n(B_0)\to S_n(A_0)\). We write \(q\restriction A_0\) for
the restriction of \(q\) to \(A_0\). We call \(q\) an extension of
\(q\restriction A_0)\)

\begin{lemma}[]
A type \(p\) is isolated in \(T\) iff \(p\) is an isolated point in
\(S_n(T)\). In fact, \(\varphi\) isolates \(p\) iff \([\varphi]=\{p\}\). That is, \([\varphi]\)
is an \textbf{atom} in the Boolean algebra of clopen subsets of \(S_n(T)\)
\end{lemma}

\begin{proof}
If \(\varphi\) isolates \(p\). Then \(\varphi\in p\) and hence \([\varphi]=\{\varphi\}\).

If \([\varphi]=\{p\}\), then \(\varphi\in p\). What's more,
\(\fM\vDash\varphi\Leftrightarrow \fM\vDash p\) in \(T\)


The set \([\varphi]\) is a singleton iff \([\varphi]\) is non-empty and cannot be
divided into two non-empty clopen subsets \([\varphi\wedge\psi]\) and
\(\varphi\wedge\neg\psi\). This means that for all \(\psi\) either \(\psi\) or
\(\neg\psi\) follows from \(\varphi\) modulo \(T\). So \([\varphi]\) is a singleton iff \(\varphi\)
generates the type
\begin{equation*}
\la\varphi\ra=\{\psi(\bbar{x})\mid T\vDash\forall\bbar{x}(\varphi(\bbar{x})\to\psi(\bbar{x}))\}
\end{equation*}
\end{proof}

We call a formula \(\varphi(x)\) \textbf{complete} if
\begin{equation*}
\{\psi(\bbar{x})\mid T\vDash\forall\bbar{x}(\varphi(\bbar{x})\to\psi(\bbar{x}))\}
\end{equation*}
is a type.
\end{enumerate}
\begin{corollary}
A formula isolates a type iff it is complete
\end{corollary}

\begin{exercise}
\label{ex4.2.2}
\begin{enumerate}
\item Closed subsets of \(S_n(T)\) have the form
\(\{p\in S_n(T)\mid\Sigma\subseteq p\}\), where \(\Sigma\) is any set of formulas
\item Let \(T\) be countable and consistent. Then any meagre\footnote{A subset of a topological space is \textbf{nowhere dense} if its closure has no interior. A countable union of nowhere dense sets is meagre} subset \(X\)
of \(S_n(T)\) can be omitted, i.e., there is a model which omits all
\(p\in X\)
\end{enumerate}
\end{exercise}

\begin{proof}
\begin{enumerate}
\item The sets \([\varphi]\) are a basis for the closed subsets of \(S_n(T)\). So the
closed sets of \(S_n(T)\) are exactly the intersections
\(\bigcap_{\varphi\in\Sigma}[\varphi]=\{p\in S_n(T)\mid\Sigma\subseteq p\}\)
\item The set \(X\) is the union of a sequence of countable nowhere dense sets
\(X_i\). We may assume that \(X_i\) are closed, i.e., of the form
\(\{p\in S_n(T)\mid \Sigma_i\subseteq p\}\). That \(X_i\) has no interior means
that \(\Sigma_i\) is not isolated. The claim follows now from Corollary \ref{cor4.1.3}
\end{enumerate}
\end{proof}




\begin{exercise}
\label{ex4.2.3}
Consider the space \(S_\omega(T)\) of all complete types in variables
\(v_0,v_1,\dots\). Note that \(S_\omega(T)\) is again a compact space and
therefore not meagre by Baire's theorem
\begin{enumerate}
\item Show that
\(\{\tp(a_0,a_1,\dots)\mid\text{ the $a_i$ enumerate a model of }T\}\)
 is comeagre in \(S_\omega(T)\)
\end{enumerate}
\end{exercise}
\subsection{\(ℵ_0\)-categorical theories}
\label{sec:org03569b3}
\begin{theorem}[]
\label{thm4.3.1}
Let \(T\) be a countable complete theory. Then \(T\) is
\(\aleph_0\)-categorical iff for every \(n\) there are only finitely many
formulas \(\varphi(x_1,\dots,x_n)\) up to equivalence relative to \(T\)
\end{theorem}

\begin{definition}[]
An \(L\)-structure \(\fA\) is \textbf{\(\omega\)-saturated} if all types over finite
subsets of \(A\) are realised in \(\fA\)
\end{definition}

The types in the definition are meant to be 1-types. On the other hand, it is
not hard to see that an \(\omega\)-saturated structure realises all
\(n\)-types over finite sets (Exercise \ref{ex4.3.9}) for all \(n\ge1\). The
following lemma is a generalisation of the \(\aleph_0\)-categoricity of \(\DLO\).

\begin{lemma}[]
\label{lemma4.3.3}
Two elementarily equivalent, countable and \(\omega\)-saturated structures
are isomorphic
\end{lemma}

\begin{proof}
Suppose \(\fA\) and \(\fB\) are as in the lemma. We choose enumerations
\(A=\{a_0,a_1,\dots\}\) and \(B=\{b_0,b_1,\dots\}\). Then we construct an
ascending sequence \(f_0\subseteq f_1\subseteq \cdots\) of finite elementary
maps
\begin{equation*}
f_i:A_i\to B_i
\end{equation*}
between finite subsets of \(\fA\) and \(\fB\). We will choose the \(f_i\) in
such a way that \(A\) is the union of the \(A_i\) and \(B\) the union of the
\(B_i\). The union of the \(f_i\) is then the desired isomorphism between
\(\fA\) and \(\fB\)

The empty map \(f_0=\emptyset\) is elementary since \(\fA\) and \(\fB\) are
elementarily equivalent. Assume that \(f_i\) is already constructed. There
are two cases:

\(i=2n\); We will extend \(f_i\) to \(A_{i+1}=A_i\cup\{a_n\}\). Consider the
type
\begin{equation*}
p(x)=\tp(a_n/A_i)=\{\varphi(x)\mid\fA\vDash\varphi(a_n), \varphi(x)\text{ a $L(A_i)$-formula}\}
\end{equation*}
Since \(f_i\) is elemantarily, \(f_i(p)(x)\) is in \(\fB\) a type over
\(B_i\). Since \(\fB\) is \(\omega\)-saturated, there is a realisation \(b'\)
of this type. So for \(\bbar{a}\in A_i\)
\begin{equation*}
\fA\vDash\varphi(a_n,\bbar{a}) \Rightarrow\fB\vDash\varphi(b',f_i(\bbar{a}))
\end{equation*}
This shows that \(f_{i+1}(a_n)=b'\) defines an elementary extension of
\(f_i\)

\(i=2n+1\); we exchange \(\fA\) and \(\fB\)
\end{proof}

\begin{proof}[Proof of Theorem \ref{thm4.3.1}]
Assume that there are only finitely many \(\varphi(x_1,\dots,x_n)\) relative to
\(T\) for every \(n\). By Lemma \ref{lemma4.3.3} it suffices to show that all
models of \(T\) are \(\omega\)-saturated. Let \(\fM\) be a model of \(T\) and
\(A\) an \(n\)-element subset. If there are only \(N\) many formulas, up to
equivalence, in the variable \(x_1,\dots,x_{n+1}\), there are, up to
equivalence in \(\fM\), at most \(N\) many \(L(A)\)-formulas \(\varphi(x)\). Thus,
each type \(\varphi(x)\in S(A)\) is isolated (w.r.t. \(\Th(\fM_A)\)) by a smallest formula \(\varphi_p(x)\)
(obviously conjunction). Each element of \(M\) which realises
\(\varphi_p(x)\) also realises \(p(x)\), so \(\fM\) is \(\omega\)-saturated.

Conversely, if there are infinitely many \(\varphi(x_1,\dots,x_n)\) modulo \(T\)
for some \(n\), then - as the type space \(S_n(T)\) is compact - there must
be some non-isolated type \(p\). By the Omitting Types Theorem there is a
countable model of \(T\) in which this type is not realised. On the other
hand, there also exists a countable model of \(T\) realizing this type. So
\(T\) is not \(\aleph_0\)-categorical
\end{proof}

The proof shows that a countable complete theory with infinite models is
\(\aleph_0\)-categorical iff all countable models are \(\omega\)-saturated

\begin{definition}[]
An \(L\)-structure \(\fM\) is \textbf{\(\omega\)-homogeneous} if for every elementary
map \(f_0\) defined on a finite subset \(A\) of \(M\)  and for any \(a\in M\)
there is some element \(b\in M\) s.t.
\begin{equation*}
f=f_0\cup\{\la a,b\ra\}
\end{equation*}
is elementary
\end{definition}

\(f=f_0\cup\{\la a,b\ra\}\) is elementary iff \(b\) realises
\(f_0(\tp(a/A))\)

\begin{corollary}[]
Let \(\fA\) be a structure and \(a_1,\dots,a_n\) elements of \(\fA\). Then
\(\Th(\fA)\) is \(\aleph_0\)-categorical iff \(\Th(\fA,a_1,\dots,a_n)\) is \(\aleph_0\)-categorical
\end{corollary}

\begin{examplle}[]
The following theories and \(\aleph_0\)-categorical
\begin{enumerate}
\item \(\Infset\) (saturated)
\item For every finite field \(\F_q\), the theory of infinite \(\F_q\)-vector
spaces. (Vector spaces over the same field and of the same dimension are
isomorphic)
\item The theory \(\DLO\) of dense linear orders without endpoints. This follows
from Theorem \ref{thm4.3.1} since \(\DLO\) has quantifier elimination: for
every \(n\) there are only finitely many (say \(N_n\)) ways to order \(n\)
elements. Each of these possibility corresponds to a complete formula
\(\psi(x_1,\dots,x_n)\). Hence there are up to equivalence, exactly
\(2^{N_n}\) many formulas \(\varphi(x_1,\dots,x_n)\)
\end{enumerate}
\end{examplle}

\begin{definition}[]
A theory \(T\) is \textbf{small} if \(S_n(T)\) are at most countable for all \(n<\omega\)
\end{definition}

\begin{lemma}[]
A countable complete theory is small iff it has a countable
\(\omega\)-saturated model
\end{lemma}

\begin{proof}
If \(T\) has a finite model \(\fA\), \(T\) is small and \(\fA\) is
\(\omega\)-saturated (countable assignment). So we may assume that \(T\) has infinite models
\end{proof}


\appendix
\section{Fields}
\label{sec:org95fd940}
\section{{\bfseries\sffamily TODO} Don't understand}
\label{sec:org1ad78d0}
Lemma \ref{lemma3.2.13}

Exercise \ref{ex3.2.3}

theorem \ref{thm4.3.1} need to enhance my TOPOLOGY and ALGEBRA!!!

\label{Problem1}
\end{document}
