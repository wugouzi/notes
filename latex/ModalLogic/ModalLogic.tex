% Created 2020-08-30 日 10:29
% Intended LaTeX compiler: pdflatex
\documentclass[11pt]{article}
\usepackage[utf8]{inputenc}
\usepackage[T1]{fontenc}
\usepackage{graphicx}
\usepackage{grffile}
\usepackage{longtable}
\usepackage{wrapfig}
\usepackage{rotating}
\usepackage[normalem]{ulem}
\usepackage{amsmath}
\usepackage{textcomp}
\usepackage{amssymb}
\usepackage{capt-of}
\usepackage{hyperref}
\usepackage{minted}
% TIPS
% \substack{a\\b} for multiple lines text





% pdfplots will load xolor automatically without option
\usepackage[dvipsnames]{xcolor}

\usepackage{forest}
% two-line text in node by [two \\ lines]
% \begin{forest} qtree, [..] \end{forest}
\forestset{
  qtree/.style={
    baseline,
    for tree={
      parent anchor=south,
      child anchor=north,
      align=center,
      inner sep=1pt,
    }}}
%\usepackage{flexisym}
% load order of mathtools and mathabx, otherwise conflict overbrace

\usepackage{mathtools}
%\usepackage{fourier}
\usepackage{pgfplots}
\usepackage{amsthm, mathabx,  amsmath, commath}
\usepackage{amsfonts}

\usepackage{empheq}
\usepackage{tikz}
\usetikzlibrary{arrows.meta}
\usepackage[most]{tcolorbox}

\newtheorem{theorem}{Theorem}[section]
\newtheorem{definition}{Definition}[section]
\newtheorem{corollary}{Corollary}[section]
\newtheorem{example}{Example}[section]
\newtheorem{lemma}{Lemma}[section]
\newtheorem{proposition}{Proposition}[section]

\newcommand{\bl}[1] {\boldsymbol{#1}}
\newcommand{\Wt}[1] {\stackrel{\sim}{\smash{#1}\rule{0pt}{1.1ex}}}
\newcommand{\wt}[1] {\widetilde{#1}}


%For boxed texts in align, use Aboxed{}
%otherwise use boxed{}

\DeclareMathSymbol{\widehatsym}{\mathord}{largesymbols}{"62}
\newcommand\lowerwidehatsym{%
  \text{\smash{\raisebox{-1.3ex}{%
    $\widehatsym$}}}}
\newcommand\fixwidehat[1]{%
  \mathchoice
    {\accentset{\displaystyle\lowerwidehatsym}{#1}}
    {\accentset{\textstyle\lowerwidehatsym}{#1}}
    {\accentset{\scriptstyle\lowerwidehatsym}{#1}}
    {\accentset{\scriptscriptstyle\lowerwidehatsym}{#1}}
}

\usepackage{graphicx}
    
% text on arrow for xRightarrow
\makeatletter
%\newcommand{\xRightarrow}[2][]{\ext@arrow 0359\Rightarrowfill@{#1}{#2}}
\makeatother


\def \bx {\boldsymbol{x}}
\def \ba {\boldsymbol{a}}
\def \bI {\boldsymbol{I}}
\def \bt {\boldsymbol{t}}
\def \bb {\boldsymbol{b}}
\def \bA {\boldsymbol{A}}
\def \bX {\boldsymbol{X}}
\def \bu {\boldsymbol{u}}
\def \bS {\boldsymbol{S}}
\def \bZ {\boldsymbol{Z}}
\def \bz {\boldsymbol{z}}
\def \by {\boldsymbol{y}}
\def \bw {\boldsymbol{w}}
\def \bT {\boldsymbol{T}}
\def \bS {\boldsymbol{S}}
\def \bm {\boldsymbol{m}}
\def \bW {\boldsymbol{W}}
\def \bY {\boldsymbol{Y}}
\def \bH {\boldsymbol{H}}
\def \blambda {\boldsymbol{\lambda}}
\def \bPhi {\boldsymbol{\Phi}}
\def \btheta {\boldsymbol{\theta}}
\def \bmu {\boldsymbol{\mu}}
\def \bphi {\boldsymbol{\phi}}
\def \bSigma {\boldsymbol{\Sigma}}
\def \lb {\left\{}
\def \rb {\right\}}
\def \caln {\mathcal{N}}
\def \dissum {\displaystyle\Sigma}
\def \dispro {\displaystyle\prod}
\def \E {\mathbb{E}}
\def \Q {\mathbb{Q}}
\def \V {\mathbb{V}}
\def \R {\mathbb{R}}
\def \calq {\mathcal{Q}}
\def \calg {\mathcal{G}}
\def \caln {\mathcal{N}}
\def \calr {\mathcal{R}}
\def \calm {\mathcal{M}}
\def \calc {\mathcal{C}}
\def \bcup {\bigcup}

\graphicspath{{../images/ModalLogic}}
\author{wugouzi}
\date{\today}
\title{Modal Logic}
\hypersetup{
 pdfauthor={wugouzi},
 pdftitle={Modal Logic},
 pdfkeywords={},
 pdfsubject={},
 pdfcreator={Emacs 26.3 (Org mode 9.4)}, 
 pdflang={English}}
\begin{document}

\maketitle
\tableofcontents \clearpage
\section{Basic Concepts}
\label{sec:org98e621f}
\subsection{Modal Languages}
\label{sec:orgb6998a8}
\begin{definition}[]
The \textbf{basic modal language} is defined using  a set of \textbf{proposition letters} \(\Phi\)
whose elements are usually denoted \(p,q,r\) and so on, and a unary modal
operator \(\lozenge\). The well-formed \textbf{formulas} \(\phi\) of the basic modal
language are given by the rule
\begin{equation*}
\phi::=p\mid\bot\mid\neg\phi\mid\psi\vee\phi\mid\lozenge\phi
\end{equation*}
\end{definition}

\begin{definition}[]
A \textbf{modal similarity type} is a pair \(\tau=(O,\rho)\) where \(O\) is a non-empty
set, and \(\rho\) is a function \(O\to\N\). The elements of \(O\) are called \textbf{modal
operators}; we use \(\triangle\), \(\triangle_0,\triangle_1,\dots\) to denote
elements of \(O\). The function \(\rho\) assigns to each operator \(\delta\in O\) a
finite \textbf{arity}
\end{definition}

\begin{definition}[]
A \textbf{modal language} \(ML(\tau,\Phi)\) is built up using a modal similarity type
\(\tau=(O,\rho)\) and a set of proposition letters \(\Phi\). The set \(Form(\tau,\Phi)\) of
\textbf{modal formulas} over \(\tau\) and \(\Phi\) is given by the rule
\begin{equation*}
\phi:=p\mid\bot\mid\neg\phi\mid\phi_1\vee\phi_2\mid\triangle(\phi_1,\dots,\phi_{\rho(\triangle)})
\end{equation*}
where \(p\) ranges over elements of \(\Phi\)
\end{definition}

\begin{definition}[]
For each \(\triangle\in O\) the \textbf{dual} \(\triangledown\) of \(\triangle\) is defined
as \(\triangledown(\phi_1,\dots,\phi_n):=\neg\triangle(\neg\phi_1,\dots,\neg\phi_n)\)
\end{definition}

\begin{examplle}[The Basic Temporal Language]
The basic temporal language is built using a set of unary operators \(O=\{\la
   F\ra,\la P\ra\}\). The intended interpretation of a formula \(\la F\ra\phi\)
is ' \(\phi\) will be true at some Future time' and the intended interpretation of
\(\la P\ra\phi\) is ' \(\phi\) was true at some Past time.' This language is called
the \textbf{basic temporal language}. Their duals are written as \(G\) and \(H\) ('it
is Going to be the case' and 'it always Has been the case')
\end{examplle}

\subsection{Models and Frames}
\label{sec:orgdce4592}
\begin{definition}[]
A \textbf{frame} for the basic modal language is a pair \(\fF=(W,R)\) s.t.
\begin{enumerate}
\item \(W\) is a non-empty set
\item \(R\) is a binary relation on \(W\)
\end{enumerate}


A \textbf{model} for the basic modal language is a pair \(\fM=(\fF,V)\), where \(\fF\)
is a frame for the basic modal language and \(V\) is a function assigning to
each proposition letter \(p\) in \(\Phi\) a subset \(V(p)\) of \(W\). The function
\(V\) is called a \textbf{valuation}. \(\fM\) is \textbf{based on} the frame \(\fF\)
\end{definition}

\begin{definition}[]
Suppose \(w\) is a state in a model \(\fM=(W,R,V)\). Then \(\phi\) is \textbf{satisfied} in
\(\fM\) at state \(w\) if
\begin{align*}
\fM,w\Vdash p&\quad\text{iff}\quad
w\in V(p),\text{ where } p\in\Phi\\
\fM,w\Vdash\bot&\quad\text{iff}\quad\text{never}\\
\fM,w\Vdash\neg\phi&\quad\text{iff}\quad
\text{not }\fM,w\Vdash\phi\\
\fM,w\Vdash\phi\vee\psi&\quad\text{iff}\quad
\fM,w\Vdash\phi\text{ or }\fM,w\Vdash\psi\\
\fM,w\Vdash\lozenge\phi&\quad\text{iff}\quad
\text{ for some }v\in W\text{ with }Rwv\text{ we have }\fM,v\Vdash\phi
\end{align*}
It follows that \(\fM,w\Vdash\Box\phi\) iff for all \(v\in W\) s.t.
\(Rwv\), we have \(\fM,v\Vdash\phi\)
\end{definition}

\begin{definition}[]
Let \(\tau\) be a modal similarity type. A \textbf{\(\tau\)-frame} is a tuple \(\fF\)
consisting of the following ingredients
\begin{enumerate}
\item a non-empty set \(W\)
\item for each \(n\ge0\), and each \(n\)-ary modal operator \(\triangle\) in the
similarity type \(\tau\), an \((n+1)\)-ary relation \(R_{\triangle}\)
\end{enumerate}
\end{definition}

\(\phi\) is \textbf{satisfied at a state \(w\)} in a model
\(\fM=(W,\{R_{\triangle}\mid\triangle\in\tau\},V)\) when
\(\rho(\triangle)\iffalse<\fi>0\) if
\begin{align*}
\fM,w\Vdash\triangle(\phi_1,\dots,\phi_n)\quad\text{iff}\quad&
\text{for some }v_1,\dots,v_n\in W\text{ with } R_{\triangle} wv_1\dots v_n\\
&\text{we have, for each }i,\fM,v_i\Vdash\phi_i
\end{align*}

When \(\rho(\triangle)=0\) we define
\begin{equation*}
\fM,w\Vdash\triangle \quad\text{ iff }\quad
w\in R_{\triangle}
\end{equation*}

\begin{definition}[]
The set of all formulas that are valid in a class of frames \(\sfF\)is called
the \textbf{logic} of \(\sfF\) (notation: \(\Lambda_{\sfF}\))
\end{definition}

\subsection{General Frames}
\label{sec:org00ed62b}
\begin{definition}[]
Given an \((n+1)\)-ary relation \(R\) on a set \(W\), we define the following
\(n\)-ary operation \(m_R\) on the power set \(\calp(W)\) of \(W\):
\begin{equation*}
m_R(X_1,\dots,X_n)=\{w\in W\mid Rww_1\dots w_n\text{ for some }
w_1\in X_1,\dots,w_n\in X_n\}
\end{equation*}
\end{definition}

\section{Models}
\label{sec:orga66b40d}
\subsection{Invariance Results}
\label{sec:orgc99a402}
\begin{definition}[]
Let \(\fM\) and \(\fM'\) be models of the same modal similarity type \(\tau\), and
let \(w\) and \(w'\) be states in \(\fM\) and \(\fM'\) respectively. The
\textbf{\(\tau\)-theory} (or \textbf{\(\tau\)-type}) \textbf{of} \(w\) is the set of all
\(\tau\)-formulas satisfied at \(w\): that is,
\(\{\phi\mid\fM,w\Vdash\phi\}\). We say that \(w\) and \(w'\) are \textbf{(modally)
equivalent} (\(w\leftrightsquigarrow w'\)) if they have the same \(\tau\)-theories

The \textbf{\(\tau\)-theory} of the model \(\fM\) is the set of all \(\tau\)-formulas
satisfied by all states in \(fM\); that is, \(\{\phi\mid\fM\Vdash\phi\}\)
Models \(\fM\) and \(\fM'\) are called
\textbf{(modally) equivalent} (\(\fM\leftrightsquigarrow\fM'\)) if their theories are identical
\end{definition}

\subsubsection{Disjoint Unions}
\label{sec:org3c788a9}
\subsubsection{Generated submodels}
\label{sec:orge650ae6}
\begin{definition}[]
Let \(\fM=(W,R,V)\) and \(\fM'=(W',R',V')\) be two models; we say that
\(\fM'\) is a \textbf{submodel} of \(\fM\) if \(W'\susbeteq W\), \(R'\) is the
restriction of \(R\) to \(W'\), and \(V'\) is the restriction of \(V\) to
\(\fM'\). We say that \(\fM'\) is a \textbf{generated submodel} of \(\fM\)
(\(\fM'\rightarrowtail\fM\)) if \(\fM'\) is a submodel of \(\fM\) and for
all points \(w\) the following closure condition holds
\begin{equation*}
\text{if }w\text{ is in }\fM'\text{ and }Rwv,\text{ then }v\text{ is in }\fM'
\end{equation*}

Let \(fM\) be a model, and \(X\) a subset of the domain of \(\fM\); the
\textbf{submodel generated by} \(X\) is the smallest generated submodel of \(\fM\)
whose domain contains \(X\). A \textbf{rooted} or \textbf{point generated} model is a model
that is generated by a singleton set, the element of which is called the
\textbf{root} of the frame
\end{definition}

\subsubsection{Morphism for modalities}
\label{sec:org57ae52d}
\begin{definition}[Homomorphisms]
Let \(\tau\) be a modal similarity type and let \(\fM\) and \(\fM'\) be
\(\tau\)-models. By a \textbf{homomorphism} \(f:\fM\to\fM'\), we mean a function \(f:W\to
    W'\) satisfying
\begin{enumerate}
\item For each proposition letter \(p\) and each element \(w\) from \(\fM\), if
\(w\in V(p)\), then \(f(w)\in V'(p)\)
\item For each \(n\ge0\) and each \(n\)-ary \(\triangle\in\tau\) and
\((n+1)\)-tuple \(\bbar{w}\) from \(\fM\), if \((w_0,\dots,w_n)\in
       R_{\triangle}\), then \((f(w_0),\dots,f(w_n))\in R_{\triangle}'\) (the
\textbf{homomorphic condition})
\end{enumerate}
\end{definition}

\begin{definition}[Strong Homomorphisms, Embeddings and Isomorphisms]
Let \(\tau\) be a modal similarity type and let \(\fM\) and \(\fM'\) be
\(\tau\)-models. By a \textbf{strong homomorphism} \(f:\fM\to\fM'\), we mean a function \(f:W\to
    W'\) satisfying
\begin{enumerate}
\item For each proposition letter \(p\) and each element \(w\) from \(\fM\) iff
\(w\in V(p)\), then \(f(w)\in V'(p)\)
\item For each \(n\ge0\) and each \(n\)-ary \(\triangle\in\tau\) and
\((n+1)\)-tuple \(\bbar{w}\) from \(\fM\) iff \((w_0,\dots,w_n)\in
       R_{\triangle}\), then \((f(w_0),\dots,f(w_n))\in R_{\triangle}'\) (the
\textbf{strong homomorphic condition})
\end{enumerate}


An \textbf{embedding} of \(\fM\) into \(\fM'\) is a strong homomorphism
\(f:\fM\to\fM'\) which is injective. An \textbf{isomorphism} is a bijective strong homomorphism
\end{definition}

\begin{proposition}[]
Let \(\tau\) be a modal similarity type and let \(\fM\) and \(\fM'\) be
\(\tau\)-models. Then the following holds
\begin{enumerate}
\item for all elements \(w\) and \(w'\) of \(\fM\) and \(\fM'\), respectively,
if there exists a surjective strong homomorphism \(f:\fM\to\fM'\) with
\(f(w)=w'\), then \(w\) and \(w\) are modally equivalent
\item If \(\fM\cong\fM'\), then \(\fM\leftrightsquigarrow\fM'\)
\end{enumerate}
\end{proposition}

\begin{definition}[Bounded Morphisms - the Basic Case]
Let \(\fM\) and \(\fM'\) be models for the basic modal language. A mapping
\(f:\fM=(W,R,V)\to\fM'=(W',R',V')\) is a \textbf{bounded morphsim} if it satisfies
\begin{enumerate}
\item \(w\) and \(f(w)\) satisfy the same proposition letters
\item \(f\) is a homomorphism w.r.t. the relation \(R\) (if \(Rwv\) then \(R'f(w)f(v)\))
\item If \(R'f(w)v'\) then there exists \(v\) s.t. \(Rwv\) and \(f(v)=v'\) (the
\textbf{back condition})
\end{enumerate}


If there is a \textbf{surjective} bounded morphism from \(\fM\) to \(\fM'\), then we
say that \(\fM'\) is a \textbf{bounded morphic image} of \(\fM\), and write \(\fM\twoheadrightarrow\fM'\)
\end{definition}

\begin{proposition}[]
Let \(\tau\) be a modal similarity type and let \(\fM\) and \(\fM'\) be
\(\tau\)-models s.t. \(f:\fM\to\fM'\) is a bounded morphism. Then for each
modal formula \(\phi\), and each element \(w\) of \(\fM\) we have
\(\fM,w\Vdash\phi\) iff \(\fM',f(w)\Vdash\phi\).
\end{proposition}

\begin{proposition}[]
Assume that \(\tau\) is a modal similarity type containing only diamonds. Then for
any rooted \(\tau\)-models \(\fM\) there exists a tree-like \(\tau\)-models
\(\fM'\) s.t. \(\fM'\twoheadrightarrow\fM\). Hence any satisfiable
\(\tau\)-formula is satisfiable in a tree-like model
\end{proposition}




\subsection{Bisimulations}
\label{sec:org0e6a77f}
\begin{definition}[Bisimulation - the Basic Case]
Let \(\fM=(W,R,V)\) and \(\fM=(W',R',V')\) be two models

A non-empty binary relation \(Z\subseteq W\times W'\) is called a \textbf{bisimulation
between} \(\fM\) and \(\fM'\) (notation: \(Z:\fM\leftrightarroweq\fM')\) if
\begin{enumerate}
\item If \(wZw'\) then \(w\) and \(w'\) satisfy the same proposition letters
\item If \(wZw'\) and \(Rwv\), then there exists \(v'\) (in \(\fM'\)) s.t.
\(vZv'\) and \(R'w'v'\) (the \textbf{forth condition})
\item The converse of (2): if \(wZw'\) and \(R'w'v'\), then there exists \(v\)
(in \(\fM\)) s.t. \(vZv'\) and \(Rwv\) (the \textbf{back condition})
\end{enumerate}


When \(Z\) is a bisimulation linking two states \(w\) in \(\fM\) and \(w'\)
in \(\fM'\) we say that \(w\) and \(w'\) are \textbf{bisimilar}, and we write
\(Z:\fM,w\leftrightarroweq \fM',w'\). If there is a bisimulation, we sometimes
write \(\fM,w\leftrightarroweq \fM',w'\) or \(w\leftrightarroweq w'\)
\end{definition}

\begin{definition}[Bisimulation - the General Case]
Let \(\tau\) be a modal similarity type, and let
\(\fM=(W,R_{\triangle},V)_{\triangle\in\tau}\) and
\(\fM'=(W',R_{\triangle}',V')_{\triangle\in\tau}\) be \(\tau\)-models. A
non-empty binary relation \(Z\subseteq W\times W'\) is called a \textbf{bisimulation}
between \(\fM\) and \(\fM'\) (\(Z:\fM\leftrightarroweq\fM'\)) if the above
condition 1 is satisfied and
\begin{enumerate}
\setcounter{enumi}{1}
\item If \(wZw'\) and \(R_{\triangle}wv_1\dots v_n\) then there are
\(v_1',\dots,v_n'\in W'\) s.t. \(R'_{\triangle}w'v_1'\dots v_n'\) and for
all \(i\) (\(1\le i\le n\)) \(v_iZv_i'\) (the \textbf{forth} condition)
\item If \(wZw'\) and \(R'_{\triangle}w'v_1'\dots v_n'\) then there are
\(v_1,\dots,v_n\in W\) s.t. \(R_{\triangle}wv_1\dots v_n\) and for
all \(i\) (\(1\le i\le n\)) \(v_iZv_i'\) (the \textbf{back} condition)
\end{enumerate}
\end{definition}

\begin{proposition}[]
Let \(\tau\) be a modal similarity type, and let \(\fM,\fM'\) and \(\fM_i\) (\(i\in
   I\)) be \(\tau\)-models
\begin{enumerate}
\item If \(\fM\cong\fM'\), then \(\fM\leftrightarroweq\fM'\)
\item For every \(i\in I\), and every \(w\) in \(\fM_i\),
\(\fM_i,w\leftrightarroweq\biguplus_i\fM_i,w\)
\item If \(\fM'\rightarrowtail\fM\), then \(\fM',w\leftrightarroweq\fM,w\) for
all \(w\) in \(\fM'\)
\item If \(f:\fM\twoheadrightarrow\fM'\), then
\(\fM,w\leftrightarroweq\fM',f(w)\) for all \(w\) in \(\fM\)
\end{enumerate}
\end{proposition}

\begin{proof}
Suppose \(\fM=(W,R_{\triangle},V)_{\triangle\in\tau}\) and
\(\fM'=(W',R_{\triangle}',V')_{\triangle\in\tau}\) 

\begin{enumerate}
\item Suppose \(f:\fM\cong\fM'\), then we define \(wZw'\) iff \(w'=f(w)\) where
\(w\in W,w'\in W'\). Bisimulation comes from the definition of the isomorphism
\item Define the relation \(Z=\{(w,w)\mid
      w\in\fM_i\}\subseteq\fM_i\times\biguplus\fM_i\). The first condition comes
from the invariance. The forth condition is obvious. For the back
condition, if \(R_{\triangle}'w'v_1'\dots v_n'\) and \(w'\in W\), then
\(v_1',\dots,v_n'\in W\) since each \(R_{\triangle,i}\) is disjoint and we
have \(R_{\triangle,i}w'v_1'\dots v_n'\)
\item Define the relation \(Z=\{(w,w)\mid w\in\fM'\}\subseteq\fM'\times\fM\).
The first condition comes from the invariance. Forth condition is obvious.
For the back condition, suppose \(wZw\) and \(R'_{\triangle}wv_1'\dots
      v_n'\), by the definition, \(v_1',\dots,v_n'\in W\) and
\(R_{\triangle}wv_1'\dots v_n'\)
\item Define \(Z=\{(w,f(w)\mid w\in W)\}\). The first condition comes from the
definition. If \(wZw'\) and \(R_{\triangle}wv_1\dots v_n\), then
\(R'_{\triangle}f(w)f(v_1)\dots f(v_n)\). If \(wZw'\) and
\(R_{\triangle}'w'v_1'\dots v_n\), then there is \(v_1,\dots,v_n\) s.t.
\(R_{\triangle}wv_1,\dots,v_n\) and \(f(v_i)=v_i'\) for \(1\le i\le n\)
\end{enumerate}
\end{proof}

\begin{theorem}[]
\label{thm2.20}
Let \(\tau\) be a modal similarity type, and let \(\fM, \fM'\) be \(\tau\)-models.
Then, for every \(w\in W\) and \(w'\in W'\), \(w\leftrightarroweq w'\)
implies that \(w\leftrightsquigarrow w'\). In other words, modal formulas are
invariant under bisimulation
\end{theorem}

\begin{proof}
Induction on the complexity of \(\phi\).

Suppose \(\phi\) is \(\diamond\psi\), we have \(\fM,w\Vdash\diamond\psi\) iff there
exists a \(v\) in \(\fM\) s.t. \(Rwv\) and \(\fM,v\Vdash\psi\). As
\(w\leftrightarroweq w'\), there exists a \(v'\) in \(\fM'\) s.t. \(R'w'v'\)
and \(v\leftrightarroweq v'\). By the I.H., \(\fM',v'\Vdash\psi\), hence \(\fM',w'\Vdash\diamond\psi\)
\end{proof}

\begin{examplle}[Bisimulation and First-Order Logic]
\label{example2.22}

\begin{center}
\includegraphics[width=.9\linewidth]{/media/wu/file/stuuudy/notes/images/ModalLogic/BisimilarModels.png}
\end{center}
\end{examplle}

\begin{examplle}[]
\label{example2.23}

\begin{center}
\includegraphics[width=.9\linewidth]{/media/wu/file/stuuudy/notes/images/ModalLogic/NotBisimilar.png}
\end{center}
\end{examplle}

\(\fM\) is \textbf{image-finite} if for each state \(u\) in \(\fM\) and each relation
\(R\) in \(\fM\), the set \(\{(v_1,\dots,v_n)\mid Ruv_1\dots v_n\}\) is
finite

\begin{theorem}[Hennessy-Milner Theorem]
\label{thm2.24}
Let \(\tau\) be a modal similarity type and let \(\fM\) and \(\fM'\) be two
image-finite \(\tau\)-models. Then for every \(w\in W\) and \(w'\in W'\),
\(w\leftrightarroweq w'\) iff \(w\leftrightsquigarrow w'\)
\end{theorem}

\begin{proof}
Assume that our similarity type \(\tau\) only contains a single diamond. The
direction from left to right follows from Theorem \ref{thm2.20}

Suppose \(w\leftrightsquigarrow w'\). The first condition is immediate. If
\(Rwv\), assume there is no \(v'\) in \(\fM'\) with \(R'w'v'\) and
\(v\leftrightsquigarrow v'\). Let \(S'=\{u'\mid R'w'u'\}\). Note that \(S'\)
must be non-empty, for otherwise \(\fM',w'\Vdash\Box\bot\), which would
contradict \(w\leftrightsquigarrow w'\) since \(\fM,w\Vdash\diamond\top\).
Furthermore, as \(\fM'\) is image-finite, \(S'\) must be finite, say
\(S'=\{w_1',\dots,w_n'\}\). By assumption, for every \(w_i'\in S'\) there
exists a formula \(\psi_i\) s.t. \(\fM,v\Vdash\psi_i\), but
\(\fM',w_i'\not\Vdash\psi_i\). It follows that
\begin{equation*}
\fM,w\Vdash\diamond(\psi_1\wedge\dots\wedge\psi_n)\quad\text{ and }\quad
\fM',w'\not\Vdash\diamond(\psi_1\wedge\dots\wedge \psi_n)
\end{equation*}
\end{proof}

\begin{exercise}
\label{ex2.2.8}
Suppose that \(\{Z_i\mid i\in I\}\) is a non-empty collection of
bisimulations between \(\fM\) and \(\fM'\). Prove that the relation
\(\bigcup_{i\in I}Z_i\) is also a bisimulation between \(\fM\) and \(\fM'\).
Conclude that if \(\fM\) and \(\fM'\) are bisimilar, then there is a maximal
bisimulation between \(\fM\) and \(\fM'\).
\end{exercise}

\begin{proof}
\begin{enumerate}
\item If \((w,w')\in\bigcup_{i\in I}Z_i\), then \((w,w')\in Z_j\) for some
\(j\in I\) and hence they satisfy the same propositional letters
\item If \((w,w')\in\bigcup_{i\in I}Z_i\) and \(R_{\triangle}wv_1\dots v_n\),
since \((w,w')\in Z_j\) for some \(j\in I\), we have
\(R'_{\triangle}w'v_1'\dots v_n'\) and \(v_iZ_jv_i'\) for all \(1\le i\le
      n\), which means \((v_i,v_i')\in\bigcup_{i\in I}Z_i\) for all \(1\le i\le n\)
\item similarly
\end{enumerate}
\end{proof}
\end{document}
