% Created 2020-09-09 三 19:24
% Intended LaTeX compiler: pdflatex
\documentclass[11pt]{article}
\usepackage[utf8]{inputenc}
\usepackage[T1]{fontenc}
\usepackage{graphicx}
\usepackage{grffile}
\usepackage{longtable}
\usepackage{wrapfig}
\usepackage{rotating}
\usepackage[normalem]{ulem}
\usepackage{amsmath}
\usepackage{textcomp}
\usepackage{amssymb}
\usepackage{capt-of}
\usepackage{hyperref}
\usepackage{minted}
% TIPS
% \substack{a\\b} for multiple lines text





% pdfplots will load xolor automatically without option
\usepackage[dvipsnames]{xcolor}

\usepackage{forest}
% two-line text in node by [two \\ lines]
% \begin{forest} qtree, [..] \end{forest}
\forestset{
  qtree/.style={
    baseline,
    for tree={
      parent anchor=south,
      child anchor=north,
      align=center,
      inner sep=1pt,
    }}}
%\usepackage{flexisym}
% load order of mathtools and mathabx, otherwise conflict overbrace

\usepackage{mathtools}
%\usepackage{fourier}
\usepackage{pgfplots}
\usepackage{amsthm, mathabx,  amsmath, commath}
\usepackage{amsfonts}

\usepackage{empheq}
\usepackage{tikz}
\usetikzlibrary{arrows.meta}
\usepackage[most]{tcolorbox}

\newtheorem{theorem}{Theorem}[section]
\newtheorem{definition}{Definition}[section]
\newtheorem{corollary}{Corollary}[section]
\newtheorem{example}{Example}[section]
\newtheorem{lemma}{Lemma}[section]
\newtheorem{proposition}{Proposition}[section]

\newcommand{\bl}[1] {\boldsymbol{#1}}
\newcommand{\Wt}[1] {\stackrel{\sim}{\smash{#1}\rule{0pt}{1.1ex}}}
\newcommand{\wt}[1] {\widetilde{#1}}


%For boxed texts in align, use Aboxed{}
%otherwise use boxed{}

\DeclareMathSymbol{\widehatsym}{\mathord}{largesymbols}{"62}
\newcommand\lowerwidehatsym{%
  \text{\smash{\raisebox{-1.3ex}{%
    $\widehatsym$}}}}
\newcommand\fixwidehat[1]{%
  \mathchoice
    {\accentset{\displaystyle\lowerwidehatsym}{#1}}
    {\accentset{\textstyle\lowerwidehatsym}{#1}}
    {\accentset{\scriptstyle\lowerwidehatsym}{#1}}
    {\accentset{\scriptscriptstyle\lowerwidehatsym}{#1}}
}

\usepackage{graphicx}
    
% text on arrow for xRightarrow
\makeatletter
%\newcommand{\xRightarrow}[2][]{\ext@arrow 0359\Rightarrowfill@{#1}{#2}}
\makeatother


\def \bx {\boldsymbol{x}}
\def \ba {\boldsymbol{a}}
\def \bI {\boldsymbol{I}}
\def \bt {\boldsymbol{t}}
\def \bb {\boldsymbol{b}}
\def \bA {\boldsymbol{A}}
\def \bX {\boldsymbol{X}}
\def \bu {\boldsymbol{u}}
\def \bS {\boldsymbol{S}}
\def \bZ {\boldsymbol{Z}}
\def \bz {\boldsymbol{z}}
\def \by {\boldsymbol{y}}
\def \bw {\boldsymbol{w}}
\def \bT {\boldsymbol{T}}
\def \bS {\boldsymbol{S}}
\def \bm {\boldsymbol{m}}
\def \bW {\boldsymbol{W}}
\def \bY {\boldsymbol{Y}}
\def \bH {\boldsymbol{H}}
\def \blambda {\boldsymbol{\lambda}}
\def \bPhi {\boldsymbol{\Phi}}
\def \btheta {\boldsymbol{\theta}}
\def \bmu {\boldsymbol{\mu}}
\def \bphi {\boldsymbol{\phi}}
\def \bSigma {\boldsymbol{\Sigma}}
\def \lb {\left\{}
\def \rb {\right\}}
\def \caln {\mathcal{N}}
\def \dissum {\displaystyle\Sigma}
\def \dispro {\displaystyle\prod}
\def \E {\mathbb{E}}
\def \Q {\mathbb{Q}}
\def \V {\mathbb{V}}
\def \R {\mathbb{R}}
\def \calq {\mathcal{Q}}
\def \calg {\mathcal{G}}
\def \caln {\mathcal{N}}
\def \calr {\mathcal{R}}
\def \calm {\mathcal{M}}
\def \calc {\mathcal{C}}
\def \bcup {\bigcup}

\graphicspath{{../images/ModalLogic}}
\author{wugouzi}
\date{\today}
\title{Modal Logic}
\hypersetup{
 pdfauthor={wugouzi},
 pdftitle={Modal Logic},
 pdfkeywords={},
 pdfsubject={},
 pdfcreator={Emacs 26.3 (Org mode 9.4)}, 
 pdflang={English}}
\begin{document}

\maketitle
\tableofcontents \clearpage
\section{Basic Concepts}
\label{sec:org3c5a016}
\subsection{Modal Languages}
\label{sec:orga3b8101}
\begin{definition}[]
The \textbf{basic modal language} is defined using  a set of \textbf{proposition letters} \(\Phi\)
whose elements are usually denoted \(p,q,r\) and so on, and a unary modal
operator \(\lozenge\). The well-formed \textbf{formulas} \(\phi\) of the basic modal
language are given by the rule
\begin{equation*}
\phi::=p\mid\bot\mid\neg\phi\mid\psi\vee\phi\mid\lozenge\phi
\end{equation*}
\end{definition}

\begin{definition}[]
A \textbf{modal similarity type} is a pair \(\tau=(O,\rho)\) where \(O\) is a non-empty
set, and \(\rho\) is a function \(O\to\N\). The elements of \(O\) are called \textbf{modal
operators}; we use \(\triangle\), \(\triangle_0,\triangle_1,\dots\) to denote
elements of \(O\). The function \(\rho\) assigns to each operator \(\delta\in O\) a
finite \textbf{arity}
\end{definition}

\begin{definition}[]
A \textbf{modal language} \(ML(\tau,\Phi)\) is built up using a modal similarity type
\(\tau=(O,\rho)\) and a set of proposition letters \(\Phi\). The set \(Form(\tau,\Phi)\) of
\textbf{modal formulas} over \(\tau\) and \(\Phi\) is given by the rule
\begin{equation*}
\phi:=p\mid\bot\mid\neg\phi\mid\phi_1\vee\phi_2\mid\triangle(\phi_1,\dots,\phi_{\rho(\triangle)})
\end{equation*}
where \(p\) ranges over elements of \(\Phi\)
\end{definition}

\begin{definition}[]
For each \(\triangle\in O\) the \textbf{dual} \(\triangledown\) of \(\triangle\) is defined
as \(\triangledown(\phi_1,\dots,\phi_n):=\neg\triangle(\neg\phi_1,\dots,\neg\phi_n)\)
\end{definition}

\begin{examplle}[The Basic Temporal Language]
The basic temporal language is built using a set of unary operators \(O=\{\la
   F\ra,\la P\ra\}\). The intended interpretation of a formula \(\la F\ra\phi\)
is ' \(\phi\) will be true at some Future time' and the intended interpretation of
\(\la P\ra\phi\) is ' \(\phi\) was true at some Past time.' This language is called
the \textbf{basic temporal language}. Their duals are written as \(G\) and \(H\) ('it
is Going to be the case' and 'it always Has been the case')

Let's denote the converse of a relation \(R\) by \(R^\smallsmile\). We will
call a frame of the form \((T,R,R^\smallsmile)\) a \textbf{bidirectional frame}, and a
model built over such a frame a \textbf{bidirectional model}. From now on, we will
only interpret basic temporal language in bidirectional models. That is, if
\(\fM=(T,R,R^\smallsmile,V)\) is a bidirectional model then
\begin{align*}
\fM,t\Vdash F\phi \quad&\text{ iff }\quad
\exists s(Rts\wedge \fM,s\Vdash\phi)\\
\fM,t\Vdash P\phi \quad&\text{ iff }\quad
\exists s(R^\smallsmile ts\wedge \fM,s\Vdash\phi)
\end{align*}
\end{examplle}

\begin{examplle}[An Arrow Language]
The type \(\tau_\to\) of \textbf{arrow logic} is a similarity type with modal
operators other than diamonds. The language of arrow logic is designed to
talk about the objects in arrow structures. The well-formed formulas \(\phi\) are
given by
\begin{equation*}
\phi:=p\mid\bot\mid\neg\phi\mid\phi\vee\psi\mid\phi\circ\psi\mid
\otimes\phi\mid 1'
\end{equation*}
1' ('identity') is a nullary modality, the 'converse' operator \(\otimes\) is
a diamond, and the 'composition' operator \(\circ\) is a dyadic operator.
Possible readings of these operators are:
\begin{alignat*}{3}
&1'&&\text{identity}&&\text{'skip'}\\
&\otimes\phi&&\text{converse}&&\text{'\(\phi\) conversely'}\\
&\phi\circ\psi\quad&&\text{composition}\quad&&\text{'first \(\phi\), then \(\psi\)'}
\end{alignat*}
\end{examplle}
\subsection{Models and Frames}
\label{sec:org06d22b5}
\begin{definition}[]
A \textbf{frame} for the basic modal language is a pair \(\fF=(W,R)\) s.t.
\begin{enumerate}
\item \(W\) is a non-empty set
\item \(R\) is a binary relation on \(W\)
\end{enumerate}


A \textbf{model} for the basic modal language is a pair \(\fM=(\fF,V)\), where \(\fF\)
is a frame for the basic modal language and \(V\) is a function assigning to
each proposition letter \(p\) in \(\Phi\) a subset \(V(p)\) of \(W\). The function
\(V\) is called a \textbf{valuation}. \(\fM\) is \textbf{based on} the frame \(\fF\)
\end{definition}

\begin{definition}[]
Suppose \(w\) is a state in a model \(\fM=(W,R,V)\). Then \(\phi\) is \textbf{satisfied} in
\(\fM\) at state \(w\) if
\begin{align*}
\fM,w\Vdash p&\quad\text{iff}\quad
w\in V(p),\text{ where } p\in\Phi\\
\fM,w\Vdash\bot&\quad\text{iff}\quad\text{never}\\
\fM,w\Vdash\neg\phi&\quad\text{iff}\quad
\text{not }\fM,w\Vdash\phi\\
\fM,w\Vdash\phi\vee\psi&\quad\text{iff}\quad
\fM,w\Vdash\phi\text{ or }\fM,w\Vdash\psi\\
\fM,w\Vdash\lozenge\phi&\quad\text{iff}\quad
\text{ for some }v\in W\text{ with }Rwv\text{ we have }\fM,v\Vdash\phi
\end{align*}
It follows that \(\fM,w\Vdash\Box\phi\) iff for all \(v\in W\) s.t.
\(Rwv\), we have \(\fM,v\Vdash\phi\)
\end{definition}

\begin{definition}[]
Let \(\tau\) be a modal similarity type. A \textbf{\(\tau\)-frame} is a tuple \(\fF\)
consisting of the following ingredients
\begin{enumerate}
\item a non-empty set \(W\)
\item for each \(n\ge0\), and each \(n\)-ary modal operator \(\triangle\) in the
similarity type \(\tau\), an \((n+1)\)-ary relation \(R_{\triangle}\)
\end{enumerate}
\end{definition}

\(\phi\) is \textbf{satisfied at a state \(w\)} in a model
\(\fM=(W,\{R_{\triangle}\mid\triangle\in\tau\},V)\) when
\(\rho(\triangle)\iffalse<\fi>0\) if
\begin{align*}
\fM,w\Vdash\triangle(\phi_1,\dots,\phi_n)\quad\text{iff}\quad&
\text{for some }v_1,\dots,v_n\in W\text{ with } R_{\triangle} wv_1\dots v_n\\
&\text{we have, for each }i,\fM,v_i\Vdash\phi_i
\end{align*}

When \(\rho(\triangle)=0\) we define
\begin{equation*}
\fM,w\Vdash\triangle \quad\text{ iff }\quad
w\in R_{\triangle}
\end{equation*}

\begin{definition}[]
The set of all formulas that are valid in a class of frames \(\sfF\)is called
the \textbf{logic} of \(\sfF\) (notation: \(\Lambda_{\sfF}\))
\end{definition}

\subsection{General Frames}
\label{sec:orge32e2e7}
\begin{definition}[]
Given an \((n+1)\)-ary relation \(R\) on a set \(W\), we define the following
\(n\)-ary operation \(m_R\) on the power set \(\calp(W)\) of \(W\):
\begin{equation*}
m_R(X_1,\dots,X_n)=\{w\in W\mid Rww_1\dots w_n\text{ for some }
w_1\in X_1,\dots,w_n\in X_n\}
\end{equation*}
\end{definition}

\section{Models}
\label{sec:org1134d12}
\subsection{Invariance Results}
\label{sec:org7a779b4}
\begin{definition}[]
Let \(\fM\) and \(\fM'\) be models of the same modal similarity type \(\tau\), and
let \(w\) and \(w'\) be states in \(\fM\) and \(\fM'\) respectively. The
\textbf{\(\tau\)-theory} (or \textbf{\(\tau\)-type}) \textbf{of} \(w\) is the set of all
\(\tau\)-formulas satisfied at \(w\): that is,
\(\{\phi\mid\fM,w\Vdash\phi\}\). We say that \(w\) and \(w'\) are \textbf{(modally)
equivalent} (\(w\leftrightsquigarrow w'\)) if they have the same \(\tau\)-theories

The \textbf{\(\tau\)-theory} of the model \(\fM\) is the set of all \(\tau\)-formulas
satisfied by all states in \(fM\); that is, \(\{\phi\mid\fM\Vdash\phi\}\)
Models \(\fM\) and \(\fM'\) are called
\textbf{(modally) equivalent} (\(\fM\leftrightsquigarrow\fM'\)) if their theories are identical
\end{definition}

\subsubsection{Disjoint Unions}
\label{sec:orgb55b0a3}
\subsubsection{Generated submodels}
\label{sec:orga1a00db}
\begin{definition}[]
Let \(\fM=(W,R,V)\) and \(\fM'=(W',R',V')\) be two models; we say that
\(\fM'\) is a \textbf{submodel} of \(\fM\) if \(W'\subseteq W\), \(R'\) is the
restriction of \(R\) to \(W'\), and \(V'\) is the restriction of \(V\) to
\(\fM'\). We say that \(\fM'\) is a \textbf{generated submodel} of \(\fM\)
(\(\fM'\rightarrowtail\fM\)) if \(\fM'\) is a submodel of \(\fM\) and for
all points \(w\) the following closure condition holds
\begin{equation*}
\text{if }w\text{ is in }\fM'\text{ and }Rwv,\text{ then }v\text{ is in }\fM'
\end{equation*}

Let \(fM\) be a model, and \(X\) a subset of the domain of \(\fM\); the
\textbf{submodel generated by} \(X\) is the smallest generated submodel of \(\fM\)
whose domain contains \(X\). A \textbf{rooted} or \textbf{point generated} model is a model
that is generated by a singleton set, the element of which is called the
\textbf{root} of the frame
\end{definition}

\subsubsection{Morphism for modalities}
\label{sec:org9d0bec7}
\begin{definition}[Homomorphisms]
Let \(\tau\) be a modal similarity type and let \(\fM\) and \(\fM'\) be
\(\tau\)-models. By a \textbf{homomorphism} \(f:\fM\to\fM'\), we mean a function \(f:W\to
    W'\) satisfying
\begin{enumerate}
\item For each proposition letter \(p\) and each element \(w\) from \(\fM\), if
\(w\in V(p)\), then \(f(w)\in V'(p)\)
\item For each \(n\ge0\) and each \(n\)-ary \(\triangle\in\tau\) and
\((n+1)\)-tuple \(\bbar{w}\) from \(\fM\), if \((w_0,\dots,w_n)\in
       R_{\triangle}\), then \((f(w_0),\dots,f(w_n))\in R_{\triangle}'\) (the
\textbf{homomorphic condition})
\end{enumerate}
\end{definition}

\begin{definition}[Strong Homomorphisms, Embeddings and Isomorphisms]
Let \(\tau\) be a modal similarity type and let \(\fM\) and \(\fM'\) be
\(\tau\)-models. By a \textbf{strong homomorphism} \(f:\fM\to\fM'\), we mean a function \(f:W\to
    W'\) satisfying
\begin{enumerate}
\item For each proposition letter \(p\) and each element \(w\) from \(\fM\) iff
\(w\in V(p)\), then \(f(w)\in V'(p)\)
\item For each \(n\ge0\) and each \(n\)-ary \(\triangle\in\tau\) and
\((n+1)\)-tuple \(\bbar{w}\) from \(\fM\) iff \((w_0,\dots,w_n)\in
       R_{\triangle}\), iff \((f(w_0),\dots,f(w_n))\in R_{\triangle}'\) (the
\textbf{strong homomorphic condition})
\end{enumerate}


An \textbf{embedding} of \(\fM\) into \(\fM'\) is a strong homomorphism
\(f:\fM\to\fM'\) which is injective. An \textbf{isomorphism} is a bijective strong homomorphism
\end{definition}

\begin{proposition}[]
Let \(\tau\) be a modal similarity type and let \(\fM\) and \(\fM'\) be
\(\tau\)-models. Then the following holds
\begin{enumerate}
\item for all elements \(w\) and \(w'\) of \(\fM\) and \(\fM'\), respectively,
if there exists a surjective strong homomorphism \(f:\fM\to\fM'\) with
\(f(w)=w'\), then \(w\) and \(w\) are modally equivalent
\item If \(\fM\cong\fM'\), then \(\fM\leftrightsquigarrow\fM'\)
\end{enumerate}
\end{proposition}

\begin{definition}[Bounded Morphisms - the Basic Case]
Let \(\fM\) and \(\fM'\) be models for the basic modal language. A mapping
\(f:\fM=(W,R,V)\to\fM'=(W',R',V')\) is a \textbf{bounded morphsim} if it satisfies
\begin{enumerate}
\item \(w\) and \(f(w)\) satisfy the same proposition letters
\item \(f\) is a homomorphism w.r.t. the relation \(R\) (if \(Rwv\) then \(R'f(w)f(v)\))
\item If \(R'f(w)v'\) then there exists \(v\) s.t. \(Rwv\) and \(f(v)=v'\) (the
\textbf{back condition})
\end{enumerate}


If there is a \textbf{surjective} bounded morphism from \(\fM\) to \(\fM'\), then we
say that \(\fM'\) is a \textbf{bounded morphic image} of \(\fM\), and write
\(\fM\twoheadrightarrow\fM'\)
\end{definition}

\begin{proposition}[]
Let \(\tau\) be a modal similarity type and let \(\fM\) and \(\fM'\) be
\(\tau\)-models s.t. \(f:\fM\to\fM'\) is a bounded morphism. Then for each
modal formula \(\phi\), and each element \(w\) of \(\fM\) we have
\(\fM,w\Vdash\phi\) iff \(\fM',f(w)\Vdash\phi\).
\end{proposition}


Let \(\tau\) be a modal similarity type containing only diamonds (thus if \(\fM\)
is a \(\tau\)-model, it has the form \((W,R_1,\dots,V)\) where each \(R_i\)
is a binary relation on \(W\)) . In this context we will call a
\(\tau\)-model \(\fM\) \textbf{tree-like} if the structure \((W,\bigcup_i R_i,V)\) is
a tree

\begin{proposition}[]
\label{prop2.15}
Assume that \(\tau\) is a modal similarity type containing only diamonds. Then for
any rooted \(\tau\)-models \(\fM\) there exists a tree-like \(\tau\)-models
\(\fM'\) s.t. \(\fM'\twoheadrightarrow\fM\). Hence any satisfiable
\(\tau\)-formula is satisfiable in a tree-like model
\end{proposition}

\begin{proof}
Let \(w\) be the root of \(\fM\). Define the model \(\fM'\) as follows. Its
domain \(W'\) consist of all finite sequences \((w,u_1,\dots,u_n)\) s.t.
\(n\ge0\) and for some modal operators \(\la a_1\ra,\dots,\la
    a_n\ra\in\tau\) there is a path \(wR_{a_1}u_1\cdots R_{a_n}u_n\) in \(\fM\).
Define \((w,u_1,\dots,u_n)R'_a(w,v_1,\dots,w_m)\) to hold if
\(m=n+1,u_i=v_i\) for \(i=1,\dots,n\) and \(R_au_nv_m\) holds in \(\fM\).
That is, \(R'_a\) relates two sequences iff the second is an extension of
the first with a state from \(\fM\) that is a sucessor of the last element
of the first sequence. Finally, \(V'\) is defined by putting
\((w,u_1,\dots,u_n)\in V'(p)\) iff \(u_n\in V(p)\). The mapping
\(f:(w,u_1,\dots,u_n)\mapsto u_n\) defines a surjective bounded morphism
from \(\fM'\) to \(\fM\)
\end{proof}




\subsection{Bisimulations}
\label{sec:org16ef059}
\begin{definition}[Bisimulation - the Basic Case]
\label{def2.16}
Let \(\fM=(W,R,V)\) and \(\fM=(W',R',V')\) be two models

A non-empty binary relation \(Z\subseteq W\times W'\) is called a \textbf{bisimulation
between} \(\fM\) and \(\fM'\) (notation: \(Z:\fM\leftrightarroweq\fM')\) if
\begin{enumerate}
\item If \(wZw'\) then \(w\) and \(w'\) satisfy the same proposition letters
\item If \(wZw'\) and \(Rwv\), then there exists \(v'\) (in \(\fM'\)) s.t.
\(vZv'\) and \(R'w'v'\) (the \textbf{forth condition})
\item The converse of (2): if \(wZw'\) and \(R'w'v'\), then there exists \(v\)
(in \(\fM\)) s.t. \(vZv'\) and \(Rwv\) (the \textbf{back condition})
\end{enumerate}


When \(Z\) is a bisimulation linking two states \(w\) in \(\fM\) and \(w'\)
in \(\fM'\) we say that \(w\) and \(w'\) are \textbf{bisimilar}, and we write
\(Z:\fM,w\leftrightarroweq \fM',w'\). If there is a bisimulation, we sometimes
write \(\fM,w\leftrightarroweq \fM',w'\) or \(w\leftrightarroweq w'\)
\end{definition}

\begin{definition}[Bisimulation - the General Case]
Let \(\tau\) be a modal similarity type, and let
\(\fM=(W,R_{\triangle},V)_{\triangle\in\tau}\) and
\(\fM'=(W',R_{\triangle}',V')_{\triangle\in\tau}\) be \(\tau\)-models. A
non-empty binary relation \(Z\subseteq W\times W'\) is called a \textbf{bisimulation}
between \(\fM\) and \(\fM'\) (\(Z:\fM\leftrightarroweq\fM'\)) if the above
condition 1 is satisfied and
\begin{enumerate}
\setcounter{enumi}{1}
\item If \(wZw'\) and \(R_{\triangle}wv_1\dots v_n\) then there are
\(v_1',\dots,v_n'\in W'\) s.t. \(R'_{\triangle}w'v_1'\dots v_n'\) and for
all \(i\) (\(1\le i\le n\)) \(v_iZv_i'\) (the \textbf{forth} condition)
\item If \(wZw'\) and \(R'_{\triangle}w'v_1'\dots v_n'\) then there are
\(v_1,\dots,v_n\in W\) s.t. \(R_{\triangle}wv_1\dots v_n\) and for
all \(i\) (\(1\le i\le n\)) \(v_iZv_i'\) (the \textbf{back} condition)
\end{enumerate}
\end{definition}

\begin{proposition}[]
Let \(\tau\) be a modal similarity type, and let \(\fM,\fM'\) and \(\fM_i\) (\(i\in
   I\)) be \(\tau\)-models
\begin{enumerate}
\item If \(\fM\cong\fM'\), then \(\fM\leftrightarroweq\fM'\)
\item For every \(i\in I\), and every \(w\) in \(\fM_i\),
\(\fM_i,w\leftrightarroweq\biguplus_i\fM_i,w\)
\item If \(\fM'\rightarrowtail\fM\), then \(\fM',w\leftrightarroweq\fM,w\) for
all \(w\) in \(\fM'\)
\item If \(f:\fM\twoheadrightarrow\fM'\), then
\(\fM,w\leftrightarroweq\fM',f(w)\) for all \(w\) in \(\fM\)
\end{enumerate}
\end{proposition}

\begin{proof}
Suppose \(\fM=(W,R_{\triangle},V)_{\triangle\in\tau}\) and
\(\fM'=(W',R_{\triangle}',V')_{\triangle\in\tau}\) 
\(\fM_i\subseteq \biguplus_i\fM_i\)
\begin{enumerate}
\item Suppose \(f:\fM\cong\fM'\), then we define \(wZw'\) iff \(w'=f(w)\) where
\(w\in W,w'\in W'\). Bisimulation comes from the definition of the isomorphism
\item Define the relation \(Z=\{(w,w)\mid
      w\in\fM_i\}\subseteq\fM_i\times\biguplus\fM_i\). The first condition comes
from the invariance. The forth condition is obvious. For the back
condition, if \(R_{\triangle}'w'v_1'\dots v_n'\) and \(w'\in W\), then
\(v_1',\dots,v_n'\in W\) since each \(R_{\triangle,i}\) is disjoint and we
have \(R_{\triangle,i}w'v_1'\dots v_n'\)
\item Define the relation \(Z=\{(w,w)\mid w\in\fM'\}\subseteq\fM'\times\fM\).
The first condition comes from the invariance. Forth condition is obvious.
For the back condition, suppose \(wZw\) and \(R'_{\triangle}wv_1'\dots
      v_n'\), by the definition, \(v_1',\dots,v_n'\in W\) and
\(R_{\triangle}wv_1'\dots v_n'\)
\item Define \(Z=\{(w,f(w)\mid w\in W)\}\). The first condition comes from the
definition. If \(wZw'\) and \(R_{\triangle}wv_1\dots v_n\), then
\(R'_{\triangle}f(w)f(v_1)\dots f(v_n)\). If \(wZw'\) and
\(R_{\triangle}'w'v_1'\dots v_n\), then there is \(v_1,\dots,v_n\) s.t.
\(R_{\triangle}wv_1,\dots,v_n\) and \(f(v_i)=v_i'\) for \(1\le i\le n\)
\end{enumerate}
\end{proof}

\begin{theorem}[]
\label{thm2.20}
Let \(\tau\) be a modal similarity type, and let \(\fM, \fM'\) be \(\tau\)-models.
Then, for every \(w\in W\) and \(w'\in W'\), \(w\leftrightarroweq w'\)
implies that \(w\leftrightsquigarrow w'\). In other words, modal formulas are
invariant under bisimulation
\end{theorem}

\begin{proof}
Induction on the complexity of \(\phi\).

Suppose \(\phi\) is \(\diamond\psi\), we have \(\fM,w\Vdash\diamond\psi\) iff there
exists a \(v\) in \(\fM\) s.t. \(Rwv\) and \(\fM,v\Vdash\psi\). As
\(w\leftrightarroweq w'\), there exists a \(v'\) in \(\fM'\) s.t. \(R'w'v'\)
and \(v\leftrightarroweq v'\). By the I.H., \(\fM',v'\Vdash\psi\), hence \(\fM',w'\Vdash\diamond\psi\)
\end{proof}

\begin{examplle}[Bisimulation and First-Order Logic]
\label{example2.22}

\begin{center}
\includegraphics[width=.9\linewidth]{/media/wu/file/stuuudy/notes/images/ModalLogic/BisimilarModels.png}
\end{center}
\end{examplle}

\begin{examplle}[]
\label{example2.23}

\begin{center}
\includegraphics[width=.9\linewidth]{/media/wu/file/stuuudy/notes/images/ModalLogic/NotBisimilar.png}
\end{center}
\end{examplle}

\(\fM\) is \textbf{image-finite} if for each state \(u\) in \(\fM\) and each relation
\(R\) in \(\fM\), the set \(\{(v_1,\dots,v_n)\mid Ruv_1\dots v_n\}\) is
finite

\begin{theorem}[Hennessy-Milner Theorem]
\label{thm2.24}
Let \(\tau\) be a modal similarity type and let \(\fM\) and \(\fM'\) be two
image-finite \(\tau\)-models. Then for every \(w\in W\) and \(w'\in W'\),
\(w\leftrightarroweq w'\) iff \(w\leftrightsquigarrow w'\)
\end{theorem}

\begin{proof}
Assume that our similarity type \(\tau\) only contains a single diamond. The
direction from left to right follows from Theorem \ref{thm2.20}

Suppose \(w\leftrightsquigarrow w'\). The first condition is immediate. If
\(Rwv\), assume there is no \(v'\) in \(\fM'\) with \(R'w'v'\) and
\(v\leftrightsquigarrow v'\). Let \(S'=\{u'\mid R'w'u'\}\). Note that \(S'\)
must be non-empty, for otherwise \(\fM',w'\Vdash\Box\bot\), which would
contradict \(w\leftrightsquigarrow w'\) since \(\fM,w\Vdash\diamond\top\).
Furthermore, as \(\fM'\) is image-finite, \(S'\) must be finite, say
\(S'=\{w_1',\dots,w_n'\}\). By assumption, for every \(w_i'\in S'\) there
exists a formula \(\psi_i\) s.t. \(\fM,v\Vdash\psi_i\), but
\(\fM',w_i'\not\Vdash\psi_i\). It follows that
\begin{equation*}
\fM,w\Vdash\diamond(\psi_1\wedge\dots\wedge\psi_n)\quad\text{ and }\quad
\fM',w'\not\Vdash\diamond(\psi_1\wedge\dots\wedge \psi_n)
\end{equation*}
\end{proof}

\begin{exercise}
\label{ex2.2.8}
Suppose that \(\{Z_i\mid i\in I\}\) is a non-empty collection of
bisimulations between \(\fM\) and \(\fM'\). Prove that the relation
\(\bigcup_{i\in I}Z_i\) is also a bisimulation between \(\fM\) and \(\fM'\).
Conclude that if \(\fM\) and \(\fM'\) are bisimilar, then there is a maximal
bisimulation between \(\fM\) and \(\fM'\).
\end{exercise}

\begin{proof}
\begin{enumerate}
\item If \((w,w')\in\bigcup_{i\in I}Z_i\), then \((w,w')\in Z_j\) for some
\(j\in I\) and hence they satisfy the same propositional letters
\item If \((w,w')\in\bigcup_{i\in I}Z_i\) and \(R_{\triangle}wv_1\dots v_n\),
since \((w,w')\in Z_j\) for some \(j\in I\), we have
\(R'_{\triangle}w'v_1'\dots v_n'\) and \(v_iZ_jv_i'\) for all \(1\le i\le
      n\), which means \((v_i,v_i')\in\bigcup_{i\in I}Z_i\) for all \(1\le i\le n\)
\item similarly
\end{enumerate}
\end{proof}

\begin{remark}[Bisimulations for the Basic Temporal Language and Arrow Logic]
When working with the basic temporal language, we usually work with models
\((W,R,V)\) and implicitly take \(R_p\) to be \(R^\smallsmile\). Thus we need
a notion of bisimulation between models \((W,R,V)\) and \((W',R',V')\) to be
a relation \(Z\) between the states of the two models that satisfies the
clauses of Definition \ref{def2.16}, and in addition the following
\begin{enumerate}
\setcounter{enumi}{3}
\item If \(wZw'\) and \(Rvw\), then there exists \(v'\) in \(\fM'\) s.t.
\(vZv'\) and \(R'v'w'\)
\item Converse of 4: if \(wZw'\) and \(R'v'w'\), then there exists \(v\) in
\(\fM\) s.t. \(vZv'\)
\end{enumerate}
\end{remark}

\subsection{Finite Models}
\label{sec:orgd994d75}
\begin{definition}[Finite Model Property]
Let \(\tau\) be a modal similarity type, and let \(\sfM\) be a class of
\(\tau\)-models. We say that \(\tau\) has the \textbf{finite model property w.r.t.} \(\sfM\)
if the following holds: if \(\phi\) is a formula of similarity type \(\tau\), and \(\phi\) is
satisfiable in some model in \(\sfM\), then \(\phi\) is satisfiable in a \textbf{finite}
model in \(\sfM\)
\end{definition}
\subsubsection{Selecting a finite submodel}
\label{sec:org36052c5}
\begin{definition}[Degree]
We define the \textbf{degree} of modal formulas as follows:
\begin{align*}
\deg(p)\quad&=\quad 0\\
\deg(\bot)\quad&=\quad0\\
\deg(\neg\phi)\quad&=\quad\deg(\phi)\\
\deg(\phi\vee\psi)\quad&=\quad\max\{\deg(\phi),\deg(\psi)\}\\
\deg(\triangle(\phi_1,\dots,\phi_n))\quad&=\quad
1+\max\{\deg(\phi_1),\dots,\deg(\phi_2)\}
\end{align*}
\end{definition}

\begin{proposition}[]
\label{prop2.29}
Let \(\tau\) be a finite modal similarity type, and assume our collection of
proposition letters is finite as well
\begin{enumerate}
\item for all \(n\), up to logical equivalence there are only finitely many
formulas of degree at most \(n\)
\item for all \(n\), and every \(\tau\)-model \(\fM\) and state \(w\) of
\(\fM\), the set of all \(\tau\)-formulas of degree at most \(n\) that
are satisfied by \(w\), is equivalent to a single formula
\end{enumerate}
\end{proposition}

\begin{definition}[$n$-Bisimulation]
Let \(\fM\) and \(\fM'\) be models, and let \(w\) and \(w'\) be states of
\(\fM\) and \(\fM'\), respectively. We say that \(w\) and \(w'\) are
\textbf{\(n\)-bisimilar} (\(w\leftrightarroweq_nw'\)) if there exists a sequence of
binary relations \(Z_n\subseteq\cdots\subseteq Z_0\) with the following
properties (for \(i+1\le n\))
\begin{enumerate}
\item \(wZ_nw'\)
\item if \(vZ_0v'\) then \(v\) and \(v'\) agree on all proposition letters
\item if \(vZ_{i+1}v'\) and \(Rvu\) then there exists \(u'\) with \(R'v'u'\)
and \(uZ_iu'\)
\item if \(vZ_{i+1}v'\) and \(R'v'u'\), then there exists \(u\) with \(Rvu\)
and \(uZ_iu'\)
\end{enumerate}
\end{definition}

\begin{proposition}[]
\label{prop2.31}
Let \(\tau\) be a finite modal similarity type, \(\Phi\) a finite set of proposition
letters, and let \(\fM\) and \(\fM'\) be models for this language. Then for
every \(w\) in \(\fM\) and \(w'\) in \(\fM'\), the following are equivalent
\begin{enumerate}
\item \(w\leftrightarroweq_nw'\)
\item \(w\) and \(w'\) agree on all modal formulas of degree at most \(n\).
\end{enumerate}
\end{proposition}

\begin{proof}
\(2\to 1\). if \(n=0\), obvious.

If \(n=k\) and the proposition holds. Now suppose \(n=k+1\). Now \(w\) and
\(w'\) agree on all modal formulas of degree at most \(n+1\). If
there is not \(v,v'\) s.t. \(v\) and \(v'\) agree on all modal formulas of
degree at most \(n\) and \(Rwv\) and \(Rwv'\). Let \(S'=\{u'\mid R'w'u'\}\)
and \(S'\) is finite, say \(S'==\{w_1',\dots,w_n'\}\). By assumption, for
every \(w_i'\in S'\) there exists a formula \(\psi_i\)  of degree at most
\(n\) s.t. \(\fM,v\Vdash\psi_i\) but \(\fM',w_i'\not\Vdash\psi_i\). It
follows that
\begin{equation*}
\fM,w\Vdash\diamond(\psi_1\wedge\dots\wedge\psi_n)
\text{ and }
\fM',w'\not\Vdash\diamond(\psi_1\wedge\dots\wedge\psi_n)
\end{equation*}
\end{proof}

\begin{definition}[]
Let \(\tau\) be a modal similarity type containing only diamonds. Let
\(\fM=(W,R_1,\dots,R_n,\dots,V)\) be a rooted \(\tau\)-model with root
\(w\). The notion of the \textbf{height} of states in \(\fM\) is defined by
induction.

The only element of height 0 is the rot of the model; the states of height
\(n+1\) are those immediate successors of elements of height \(n\) that have
not yet assigned a height smaller than \(n+1\). The \textbf{height of a model} \(\fM\)
is the maximum \(n\) s.t. there is a state of height \(n\) in \(\fM\), if
such a maximum exists; otherwise the height of \(\fM\) is infinite

For a natural number \(k\), the \textbf{restriction} of \(\fM\) to \(k\)
(\(\fM\restriction k\)) is defined as the submodel containing only states
whose height is at most \(k\). \((\fM\restriction
    k)=(W_k,R_{1k},\dots,R_{nk},\dots,V_k)\), where
\(W_k=\{v\mid\text{height}(v)\le k\}\), \(R_{nk}=R_n\cap(W_k\times W_k)\),
and for each \(p\), \(V_k(p)=V(p)\cap W_k\)
\end{definition}


\begin{lemma}[]
\label{lemma2.33}
Let \(\tau\) be a modal similarity type that contains only diamonds. Let \(\fM\) be
a rooted \(\tau\)-models, and let \(k\) be a natural number. Then for every
state \(w\) of \((\fM\restriction k)\), we have \((\fM\restriction
    k),w\leftrightarroweq_l\fM,w\), where \(l=k-\text{height}(w)\)
\end{lemma}

\begin{theorem}[Finite Model Property - via Selection]
Let \(\tau\) be a modal similarity type containing only diamonds, and let \(\phi\) be a
\(\tau\)-formula. If \(\phi\) is satisfiable, then it is satisfiable on a finite model
\end{theorem}

\begin{proof}
Fix a modal formula \(\phi\) with \(\deg(\phi)=k\). We restrict our modal similarity
type \(\tau\) and our collection of proposition letters to the modal operators and
proposition letters actually occurring in \(\phi\). Let \(\fM_1,w_1\) be s.t.
\(\fM_1,w_1\Vdash\phi\). By Proposition \ref{prop2.15}, there exists a
tree-like model \(\fM_2\) with root \(w_2\) s.t. \(\fM_2,w_2\Vdash\phi\).
Let \(\fM_3:=(\fM_2\restriction k)\). By Lemma \ref{lemma2.33} we have
\(\fM_2,w_2\leftrightarroweq_k\fM_3,w_2\) and by Proposition \ref{prop2.31} it
follows that \(\fM_3,w_2\Vdash\phi\)

By induction on \(n\le k\) we define finite sets of states \(S_0,\dots,S_k\)
and a (final) model \(\fM_4\) with domain \(S_0\cup\cdots\cup S_k\); the
points in each \(S_n\) will have height \(n\)

Define \(S_0\) to be the singleton \(\{w_2\}\). Next, assume that
\(S_0,\dots,S_n\) have already been defined. Fix an element \(v\) of
\(S_n\). By Proposition \ref{prop2.29} there are only finitely many
non-equivalent modal formulas whose degree is at most \(k-n\), say
\(\psi_1,\dots,\psi_m\). For each formula that is of the form \(\la
    a\ra\chi\) and holds in \(\fM_3\) at \(v\), select a state \(u\) from
\(\fM_3\) s.t. \(R_avu\) and \(\fM_3,u\Vdash\chi\). Add all these \(u\)s to
\(S_{n+1}\), and repeat this selection process for every state in \(S_n\).
\(S_{n+1}\) is defined as the set of all points that have been selected in
this way

Finally, define \(\fM_4\) as follows. Its domain is \(S_0\cup\dots\cup
    S_k\); as each \(S_i\) is finite, \(\fM_4\) is finite. The relations and
valuation are obtained by restricting the relations and valuations of
\(\fM_3\) to the domain of \(\fM_4\)
\end{proof}
\subsubsection{Finite models via filtrations}
\label{sec:orgba74b76}

\begin{definition}[]
A set of formulas \(\Sigma\) is \textbf{closed under subformulas} (or \textbf{subformula closed}) if
for all formulas \(\phi\), \(\phi'\): if \(\phi\vee\phi'\in\Sigma\) then so are
\(\phi\) and \(\phi'\); if \(\neg\phi\in\Sigma\) then so is \(\phi\); and if
\(\triangle(\phi_1,\dots,\phi_n)\in\Sigma\) then so are \(\phi_1,\dots,\phi_n\)
\end{definition}

\begin{definition}[Filtrations]
We work in the basic modal language. Let \(\fM=(W,R,V)\) be a model and \(\Sigma\) a
subformula closed set of formulas. Let \(\leftrightsquigarrow_\Sigma\) be
the relation on the states of \(\fM\) defined by
\begin{equation*}
w\leftrightsquigarrow_\Sigma v \text{ iff for all }\phi\in\Sigma:
(\fM,w\Vdash\phi\text{ iff }\fM,v\Vdash\phi)
\end{equation*}
Note that \(\leftrightsquigarrow_\Sigma\) is an equivalence relation. We
denote the equivalence class of a state \(w\) of \(\fM\) w.r.t.
\(\leftrightsquigarrow_\Sigma\) by \(\abs{w}_\Sigma\), or simply
\(\abs{w}\). The mapping \(w\mapsto\abs{w}\) is called the \textbf{natural map}

Let \(W_\Sigma=\{\abs{w}_\Sigma\mid w\in W\}\). Suppose \(\fM_\Sigma^f\) is
any model \((W^f,R^f,V^f)\) s.t.
\begin{enumerate}
\item \(W^f=W_\Sigma\)
\item if \(Rwv\) then \(R^f\abs{w}\abs{v}\)
\item if \(R^f\abs{w}\abs{v}\) then for all \(\diamond\phi\in\Sigma\), if
\(\fM,v\Vdash\phi\) then \(\fM,w\Vdash\diamond\phi\)
\item \(V^f(p)=\{\abs{w}\mid\fM,w\Vdash p\}\), for all proposition letters
\(p\) in \(\Sigma\)
\end{enumerate}


\(\fM_\Sigma^f\) is called a \textbf{filtration of \(fM\) through} \(\Sigma\); we will
often suppress subscripts and write \(\fM^f\) instead of \(\fM_\Sigma^f\)
\end{definition}
\label{def2.36}
\begin{center}
\includegraphics[width=6cm]{/media/wu/file/stuuudy/notes/images/ModalLogic/Filtration.png}
\end{center}

Let \(\fM=(\N,R,V)\) , where \(R=\{(0,1),(0,2),(1,3)\}\cup\{(n,n+1)\mid
    n\ge2\}\), and \(V\) has \(V(p)=\N\setminus\{0\}\) and \(V(q)=\{2\}\)

Further assume \(\Sigma=\{\diamond p,p\}\). \(\Sigma\) is subformula closed. Then,
the model
\(\fN=(\{\abs{0},\abs{1}\},\{(\abs{0},\abs{1}),(\abs{1},\abs{1})\},V')\),
where \(V'(p)=\{\abs{1}\}\) is a filtration of \(\fM\) through \(\Sigma\). \(\fN\) is
not a bounded morphic image of \(\fM\): any bounded morphism would have to
preserve the formula \(q\)

\begin{proposition}[]
Let \(\Sigma\) be a finite subformula closed set of basic modal formulas. For any
model \(\fM\), if \(\fM^f\) is a filtration of \(\fM\) through a subformula
closed set \(\Sigma\), then \(\fM^f\) contains at most \(2^n\) nodes (where \(n\)
denotes the size of \(\Sigma\))
\end{proposition}

\begin{proof}
The states of \(\fM^f\) are the equivalence classes in \(W_\Sigma\). Let
\(g\) be the function with domain \(W_\Sigma\) and range \(\calp(\Sigma)\)
defined by \(g(\abs{w})=\{\phi\in\Sigma\mid\fM,w\Vdash\phi\}\). It follows
from the definition of \(\leftrightsquigarrow_\Sigma\) that \(g\) is well
defined and injective. Thus \(\abs{W_\Sigma}\le 2^n,n=\abs{\Sigma}\)
\end{proof}

\begin{theorem}[Filtration Theorem]
Consider the basic modal language. Let \(\fM^f=(W_\Sigma,R^f,V^f)\) be a
filtration of \(\fM\) through a subformula closed set \(\Sigma\). Then for all
formulas \(\phi\in\Sigma\), and all nodes \(w\) in \(\fM\), we have
\(\fM,w\Vdash\phi\) iff \(\fM^f,\abs{w}\Vdash\phi\)
\end{theorem}

\begin{proof}
Suppose \(\diamond\phi\in\Sigma\) and \(\fM,w\Vdash\diamond\phi\). Then there
is a \(v\) s.t. \(Rwv\) and \(\fM,v\Vdash\phi\). As \(\fM^f\) is a
filtration, \(R^f\abs{w}\abs{v}\). As \(\Sigma\) is a subformula closed,
\(\phi\in\Sigma\), thus by the inductive hypothesis
\(\fM^f,\abs{v}\Vdash\phi\). Hence \(\fM^f,\abs\Vdash\diamond\phi\)

Suppose \(\diamond\phi\in\Sigma\) and \(\fM^f,\abs{w}\Vdash\diamond\phi\).
Thus there is a state \(\abs{v}\) in \(\fM^f\) s.t. \(R^f\abs{w}\abs{v}\)
and \(\fM^f,\abs{v}\Vdash\phi\). As \(\phi\in\Sigma\), we have
\(\fM,v\Vdash\phi\). By the definition, we have \(\fM,w\Vdash\diamond\phi\)
\end{proof}

Note that clauses 2 and 3 of Definition \ref{def2.36} are designed to make the
modal case of the inductive step go through.

Define
\begin{enumerate}
\item \(R^s\abs{w}\abs{v}\) iff \(\exists w'\in\abs{w}\exists v'\in\abs{v}Rw'v'\)
\item \(R^l\abs{w}\abs{v}\) iff for all formulas \(\diamond\phi\in\Sigma\):
\(\fM,v\Vdash\phi\) implies \(\fM,w\Vdash\diamond\phi\)
\end{enumerate}


These relations give rise to the \textbf{smallest} and \textbf{largest} filtrations respectively

\begin{lemma}[]
Consider the basic modal language. Let \(\fM\) be any model, \(\Sigma\) any
subformula closed set of formulas, \(W_\Sigma\) the set of equivalence
classes induced by \(\leftrightsquigarrow_\Sigma\), and \(V^f\) the standard
valuation on \(W_\Sigma\). Then both \((W_\Sigma,R^s,V^f)\) and
\((W_\Sigma,R^l,V^f)\) are filtrations of \(\fM\) through \(\Sigma\). Furthermore, if
\((W_\Sigma, R^f,V^f)\) is any filtration of \(\fM\) through \(\Sigma\), then
\(R^s\subseteq R^f\subseteq R^l\)
\end{lemma}

\begin{proof}
If \(Rwv\), if \(\fM,v\Vdash\phi\), then \(\fM,w\Vdash\diamond\phi\), hence
\(R^l\abs{w}\abs{v}\)

For any \((W_\Sigma,R^f,V^f)\). \(R^s\subseteq R^f\) by clause 2.
\(R^f\subseteq R^l\) by clause 2
\end{proof}

\begin{theorem}[Finite Model Property - via Filtrations]
Let \(\phi\) be a basic modal formula. if \(\phi\) is satisfiable, then it is satisfiable
on a finite model. Indeed, it is satisfiable on a finite model containing at
most \(2^m\) nodes, where \(m\) is the number of subformulas of \(\phi\)
\end{theorem}

\begin{proof}
Assume that \(\phi\) is satisfiable on a model \(\fM\); take any filtration of
\(\fM\) through the set of subformulas .
\end{proof}

\begin{lemma}[]
Let \(\fM\) be a model, \(\Sigma\) a subformula closed set of formulas, and
\(W_\Sigma\) the set of equivalence classes induced on \(\fM\) by
\(\leftrightsquigarrow_\Sigma\). Let \(R^t\) be the binary relation on
\(W_\Sigma\) defined by
\begin{equation*}
R^t\abs{w}\abs{v} \text{ iff for all }\phi, \text{ if }\diamond\phi\in\Sigma
\text{ and }\fM,v\Vdash\phi\vee\diamond\phi\text{ then }\fM,w\Vdash\diamond\phi
\end{equation*}
If \(R\) is transitive then \((W_\Sigma,R^t,V^f)\) is a filtration and
\(R^t\) is transitive
\end{lemma}

\begin{definition}[]
Let \((W,R,V)\) be a transitive frame. A \textbf{cluster} on \((W,R,V)\) is a
maximal, nonempty equivalence class under \(R\). That is, \(C\subseteq W\)
is a cluster if the restriction of \(R\) to \(C\) is an equivalence relation

A cluster is \textbf{simple} if it consists of a single reflexive point, and \textbf{proper}
if it consists more than one point
\end{definition}
\subsection{The Standard Translation}
\label{sec:orge342f77}
\begin{definition}[]
For \(\tau\) a modal similarity type and \(\Phi\) a collection of proposition letters, let
\(\call_\tau^1(\Phi)\) be the first-order language (with equality) which has
unary predicates \(P_0,P_1,\dots\) corresponding to the proposition letters
\(p_0,p_1,\dots\) in \(\Phi\), and an \((n+1)\)-ary relation symbol \(R_\triangle\)
for each (\(n\)-ary) modal operator \(\triangle\) in our similarity type. We
write \(\alpha(x)\) to denote a first-order formula \(\alpha\) with one free variable, \(x\)
\end{definition}

\begin{definition}[Standard Translation]
Let \(x\) be a first-order variable. The \textbf{standard translation} \(ST_x\) taking
modal formulas to first-order formulas in \(\call_\tau^1(\Phi)\) is defined as
\begin{align*}
ST_x(p)&\quad\text{=}\quad Px\\
ST_x(\bot)&\quad\text{=}\quad x\neq x\\
ST_x(\neg\phi)&\quad\text{=}\quad \neg ST_x(\phi)\\
ST_x(\phi\vee\psi)&\quad\text{=}\quad ST_x(\phi)\vee ST_x(\psi)\\
ST_x(\triangle(\phi_1,\dots,\phi_n))&\quad\text{=}\quad \exists y_1\dots
\exists y_n(R_{\triangle} xy_1\dots y_n\wedge\\
&\hspace{2cm}ST_{y_1}(\phi_1)\wedge\dots\wedge ST_{y_n}(\phi_n) )
\end{align*}
where \(y_1,\dots,y_n\) are fresh variables.
\end{definition}

\begin{gather*}
ST_x(\diamond\phi)=\exists y(Rxy\wedge ST_y(\phi))\\
ST_x(\box\phi)=\forall y(Rxy\to ST_y(\phi))
\end{gather*}

\begin{proposition}[Local and Global Correspondence on Models]
Fix a modal similarity type \(\tau\), and let \(\phi\) be a \(\tau\)-formula. Then
\begin{enumerate}
\item For all \(\fM\) and all states \(w\) of \(\fM\): \(\fM,w\Vdash \phi\) iff
\(\fM\models ST_x(\phi)[w]\)
\item For all \(\fM\): \(\fM\Vdash\phi\) iff \(\fM\models\forall x ST_x(\phi)\)
\end{enumerate}
\end{proposition}

\begin{proposition}[]
\begin{enumerate}
\item Let \(\tau\) be a modal similarity type that only contains diamonds. Then, every
\(\tau\)-formula \(\phi\) is equivalent to a first-order formula containing at
most two variables
\item If \(\tau\) does not contain modal operators \(\triangle\) whose arity exceeds
\(n\), all \(\tau\)-formulas are equivalent to first-order formulas
containing at most \((n+1)\) vairables
\end{enumerate}
\end{proposition}

\begin{proof}
Assume \(\tau\) contains only diamonds \(\la a\ra,\la b\ra\). Fix two distinct
variables \(x\) and \(y\). Define two variants \(ST_x\) and \(ST_y\) of the
standard translation as follows
\begin{alignat*}{2}
&ST_x(p)=Px&&ST_y(p)=Py\\
&ST_x(\bot)=x\neq x\quad&&ST_y(\bot)=y\neq y\\
&ST_x(\neg\phi)=\neg ST_x(\phi)&&ST_y(\neg\phi)=\neg ST_y(\phi)\\
&ST_x(\phi\vee\psi)=ST_x(\phi)\vee ST_x(\psi)&&ST_y(\phi\vee\psi)=ST_y(\phi)\vee ST_y(\psi)\\
&ST_x(\la a\ra\phi)=\exists y(R_axy\wedge ST_y(\phi))\quad&&
ST_y(\la a\ra\phi)=\exists x(R_ayx\wedge ST_x(\phi))
\end{alignat*}
Then for any \(\tau\)-formula \(\phi\), its \(ST_x\)-translation contains at most
the two variables \(x\) and \(y\), and \(ST_x(\phi)\) is equivalent to the
original standard translation of \(\phi\)
\end{proof}

\begin{examplle}[]
\begin{align*}
ST_x(\diamond(\box p\to q))&=
\exists y(Rxy\wedge ST_y(\box p\to q))\\
&=\exists y(Rxy\wedge(\forall x(Ryx\to ST_x(p))\to Qy))\\
&=\exists y(Rxy\wedge(\forall x(Ryx\to Px)\to Qy))
\end{align*}
\end{examplle}

\(Rxx\) is not equivalent to any modal formula. Suppose \(\phi\) is a modal formula
s.t. \(ST_x(\phi)\) is equivalent to \(Rxx\). Let \(\fM\) be a singleton
reflexive model and let \(w\) be the unique state in \(\fM\); obviously
\(\fM\models Rxx[w]\). Let \(\fN\) be a model based on the strict ordering of
the integers; for every integer \(v\), \(\fN\models\neg Rxx[v]\). Let \(Z\)
be the relation which links every integer with the unique state in \(fM\),
and assume that the valuations in \(\fN\) and \(\fM\) are s.t. \(Z\) is a
bisimulation.
\begin{equation*}
\fM\models Rxx[w]\Rightarrow\fM,w\Vdash\phi\Rightarrow\fN,v\Vdash\phi
\Rightarrow\fN\models Rxx[v]
\end{equation*}

\begin{definition}[]
Let \(\tau\) be a modal similarity type, \(\sfC\) a class of \(\tau\)-models, and \(\Gamma\)
a set of formulas over \(\tau\). We say that \(\Gamma\) \textbf{defines} of \textbf{characterizes} a class
\(\sfK\) of models \textbf{within} \(\sfC\) if for all models \(\fM\) in \(\sfC\) we
have that \(\fM\) is in \(\sfK\) iff \(\fM\Vdash\Gamma\). If \(\sfC\) is the
class of all \(\tau\)-models, we simply say that \(\Gamma\) defines or characterizes
\(\sfK\); we omit brackets whenever \(\Gamma\) is a singleton. We say that a formula \(\phi\)
defines a \textbf{property} whenever \(\phi\) defines the class of models satisfying the property
\end{definition}
\subsection{Modal Saturation via Ultrafilter Extensions}
\label{sec:org5a298c8}
\subsubsection{M-saturation}
\label{sec:orgcae0532}
\begin{definition}[Hennessy-Milner Classes]
Let \(\tau\) be a modal similarity type, and \(\sfK\) a class of \(\tau\)-models.
\(\sfK\) is a \textbf{Hennessy-Milner} class, or \textbf{has the Hennessy-Milner property}, if
for every two models \(\fM\) and \(\fM'\) in \(\sfK\) and any two states
\(\w,w'\) of \(\fM\) and \(\fM'\), respectively, \(w\leftrightsquigarrow w'\)
implies \(\fM,w\leftrightarroweq \fM',w'\)
\end{definition}

For example, by Theorem \ref{thm2.24} the class of image-finite models has the
Hennessy-Milner property.

Suppose we are working in the basic modal language. Let \(\fM=(W,R,V)\) be a
model, let \(w\) be a state in \(W\) and let
\(\Sigma=\{\phi_0,\phi_1,\dots\}\) be an infinite set of formulas. Suppose
that \(w\) has successors \(v_0,v_1,\dots,\) where respectively
\(\phi_0,\phi_0\wedge\phi_1,\phi_0\wedge\phi_1\wedge\phi_2,\dots\) hold. If
there is no successor \(v\) of \(w\) where \textbf{all} formulas from \(\Sigma\) hold \textbf{at the
same time}, then the model is in some sense incomplete. A model is called
m-saturated if incompleteness of this kind does not occur

Suppose that we are looking for a successor of \(w\) at which every formula
\(\phi_i\) of the infinite set of formulas
\(\Sigma=\{\phi_0,\phi_1,\dots\}\) holds. M-saturation is a kind of
compactness property, according to which it suffices to find satisfying
successors of \(w\) for arbitrary finite approximations of \(\Sigma\)

\begin{definition}[M-saturation]
Let \(\fM=(W,R,V)\) be a model of the basic modal similarity type, \(X\) a
subset of \(W\) and \(\Sigma\) a set of modal formulas. \(\Sigma\) is \textbf{satisfiable} in the set
\(X\) if there is a state \(x\in X\) s.t. \(\fM,x\Vdash\phi\) for all
\(\phi\in\Sigma\). \(\Sigma\) is \textbf{finitely satisfiable} in \(X\) if every finite subset
of \(\Sigma\) is satisfiable in \(X\)

The model \(\fM\) is called \textbf{\(m\)-saturated} if it satisfies the following
condition for every state \(w\in W\) and every set \(\Sigma\) of modal formulas:

\begin{center}
If \(\Sigma\) is finitely satisfiable in the set of successors of \(w\), \par
then
\(\Sigma\) is satisfiable in the set of successors of \(w\)
\end{center}
\end{definition}
\end{document}
